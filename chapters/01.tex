\documentclass[output=paper]{langsci/langscibook} 
\title{Mediterranean Lingua Franca}
\author{%
 Jane Meier  \affiliation{University of Eden}  
}
% \chapterDOI{} %will be filled in at production

\epigram{Change epigram in chapters/01.tex or remove it there }

\abstract{This chapter explores the effect of Arabic contact on Lingua Franca, an almost exclusively oral pidgin spoken across the Mediterranean and along the North African coastline from the 17\textsuperscript{th}–19\textsuperscript{th} centuries. The chapter highlights the phonological and lexical impact Arabic appears to have had on Lingua Franca.}

\maketitle
\begin{document}

\section{Overview and historical development}

Today, a lingua franca is a term describing a language used by two or more linguistic groups as a means of communication, often for economic motives. Typically, none of the groups speak the chosen language as their native tongue. The original and eponymous Lingua Franca, however, was a trading language, used among and between Europeans and Arabs across the Mediterranean (Kahane and Kahane, 1976) Its exact historical and geographical roots – as well as its precise lexifier languages – prove elusive. \citet{Hall1966} dates Lingua Franca’s birth to the era of the crusades, while other linguists (Cifoletti, 2004; Minervini, 1996) suggest that it took root on the North African Barbary Coast (in the Regencies of Algiers, Tunis and Tripoli) at the close of the 16\textsuperscript{th} Century. 

Contention extends to its very name. There are several discrete etymological suggestions for the term \textit{Franca}.  Some linguists interpret \textit{Franca} as meaning ‘French’ (Hall, 1966, p. 3)). Hall claims that France’s regional significance in the medieval era meant that its languages, specifically Provençal, were adopted across the Mediterranean, and were a key constituent of the original Lingua Franca. In their etymological study of Lingua Franca, Kahane and Kahane (1976, p. 25) assert that the name \textit{Lingua} \textit{Franca} is rooted in the East and the Byzantine tradition, stemming from the Greek word \textit{Phrangika}, which denoted Venetian as much as Italian, or indeed, as any Western language (Kahane and Kahane, 1976, p. 31). An alternative etymology for \textit{Lingua} \textit{Franca}, espoused by Schuchardt (1909, p. 74) among others, is from the Arabic, \textit{lisān} \textit{al-faranǧ} ‘language of the Franks’. This initially referred to Latin and then to describe a trading language employed largely by Jews across the Mediterranean. It later came to encompass the languages of all Europeans, but particularly Italians (Kahane and Kahane, 1976, p. 26).  

Evolving from its maritime origins, by the late 16\textsuperscript{th} century, Lingua Franca was the language of pirates of the North African Barbary coast and their captured slaves, and, as such, the subject of legend and myth. Indeed, the variation found in the accounts of Lingua Franca, and descriptions of its linguistic makeup lead some linguists (Minervini, 1996; Mori, 2016) to suggest that there may have been multiple Lingua(s) Franca(s) or that it was simply L2 Italian. As Schuchardt (1909, p. 88), identified, Lingua Franca – perhaps above all in its resistance to theoretical classification – adheres to Heraclitus’ philosophy of \textit{panta} \textit{rei}  ‘everything is in flux’. 

Contemporaneous descriptions of Lingua Franca detailing its lexifiers and, in some cases, its salient features, come mostly from the North African Barbary Regencies and from the Levant. While the writers of these descriptions often identify Italian and Spanish as lexifiers, there are also, if fewer, mentions of Portuguese, French, Provençal, Arabic, Turkish and Greek (see below for further detail). This speaks to the hypothesis that there were multiple Lingua Francas, or perhaps more appropriately lingua francas. It also raises the frequent subjectivity of the source’s writer and their consequent interpretation of Lingua Franca. Their native language appears to have a bearing on the makeup of the Lingua Franca recorded. It may influence the lexicon they hear, as well as the orthography they employ in their account. Equally, there is the subjectivity of the researcher to bear in mind. The assumption that a French source, for example, has represented Lingua Franca in a particular manner overlooks the fact that the European residents, particularly of port cities across the Mediterranean, most likely would have been multilingual, with an ability to adapt their lexicon to maximize understanding and communication with their interlocutor. 

The most widespread documentation of Lingua Franca comes from the Levant and into North West Africa. Algiers, and to a lesser extent, Tunis and Tripoli, had long been the crucible of Mediterranean piracy, and as the slave trade of Barbary pirates increased – with over a million European slaves held there between the 16th and 19th century (Davis, 2004, p. 23) so too did the domains and usage of Lingua Franca. The 16th–17th century Spanish Abbott Diego del Haedo described it as follows: 

\textit{La} \textit{que} \textit{los} \textit{Moros} \textit{e} \textit{Turcos} \textit{llaman} \textit{Franca…} \textit{siendo} \textit{todo} \textit{una} \textit{mexcla} \textit{de} \textit{lenguas} \textit{cristianas} \textit{y} \textit{de} \textit{vocablos,} \textit{que} \textit{son} \textit{por} \textit{la} \textit{mayor} \textit{parte} \textit{Italiano} \textit{e} \textit{espanoles} \textit{y} \textit{algunos} \textit{portugueses...Este} \textit{hablar} \textit{Franco} \textit{es} \textit{tan} \textit{general} \textit{que} \textit{non} \textit{hay} \textit{casa} \textit{do} \textit{no} \textit{se} \textit{use}

‘that which the Arabs and Turks call Franco…being a mix of Christian languages and words, which are in the majority Italian and Spanish and some Portuguese, this Franca is so widespread that there isn’t a house [in Algiers] where it isn’t spoken’  (Haedo, 1612, p. 24 JN’s translation).

Despite its alleged profusion in the Barbary coast, and numerous references in various contemporary sources, the corpus of Lingua Franca is remarkably limited. The exclusively European documentary sources (from contemporary diplomats, travellers, priests and slaves) provide mostly phrases and individual words and a handful of short dialogues. The most fulsome examples come from literature, and, as such, only provide indirect, and potentially less authentic, evidence of the contact vernacular. 

For example, the alleged earliest record (Grion, 1890) of Lingua Franca comes from an anonymous poem, \textit{Contrasto} \textit{della} \textit{Zerbitana,} found by Grion in a 14\textsuperscript{th} century Florentine codex, and apparently written in the late thirteenth or early fourteenth century on the island of Djerba, off the coast of Tunisia. The sixteenth century \textit{La} \textit{Zingana} (Giancarli, 1545) has an eponymous Arabic-speaking heroine whose language features hallmarks of Lingua Franca, while the speeches of the Turkish characters in Molière’s \textit{Le} \textit{Sicilien} (1667) and \textit{Le} \textit{Bourgeois} \textit{gentilhomme} \REF{ex:key:1798} also appear to share a number of its defining linguistic traits.

The first detailed description and documentation of Lingua Franca comes in Haedo’s 1612 \textit{Topographia}, a comprehensive study of Algiers, with a chapter devoted to the languages spoken there. Haedo was a Spanish abbot who spent several years in Algiers at the close of the 16\textsuperscript{th} century and was even imprisoned for a number of months. His \textit{Topographia}  details the urban features, social makeup and linguistic mix of Algiers, creating an impression of Lingua Franca’s ubiquity across multiple domains, and indispensability to daily commercial, and even domestic, life: 

\begin{quote}
	Este hablar Franco es tan general que non hay casa do no se use 
\end{quote}

\begin{quote}
	‘This Franca is so widespread that there isn’t a house [in Algiers] where it isn’t spoken’. 
\end{quote}

\begin{quote}
	Other early sources are predominantly French. A Trinitarian priest, Pierre Dan, was almost contemporary with Haedo in Algiers, while in the mid-17\textsuperscript{th} century the diplomat Savary de Brèves travelled to Tripoli and Chevalier D’Arvieux, King Louis XIV’s envoy to the region, and advisor to Molière on Turkish and Arabic matters, visited both Algiers and Tunis. All these men offered excerpts of Lingua Franca in their writings, as well as descriptions of its character and lexifiers (Dan, 1637; D’Arvieux, 1735; Savary de Brèves, 1628)
\end{quote}

Certainly 17\textsuperscript{th}{}-century Algiers and the other two Barbary Regencies of Tunis and Tripoli provided the conditions for what had previously been a pre-pidgin – with limited lexicon and a lack of stability – to evolve into the language of daily life, permeating all echelons of society and facilitating contact among and between the plurilingual populations of the urban centres. The Barbary states were, from the late 16\textsuperscript{th} century under the \textit{de} \textit{jure} but not \textit{de} \textit{facto} control of the Ottoman empire, whose support was needed to shore up the rule of the Greek Barbarossa brothers who had ousted Spanish forces from North Africa. The two brothers (named for their red beards), Aruj and Hizir, gradually brought much of North Africa under Turkish sovereignty through a series of naval challenges and, later, city-sieges securing power over coastal areas. The indigenous population rallied to the brothers’ cry and although the elder, Aruj, was slain, Hizir assumed control of Algiers in the early 16\textsuperscript{th} century (Tinniswood, 2010, p. 8; Weiss, 2011, p. 10) He immediately offered the Ottoman Empire control over the brothers’ conquests in order to bolster his own position and ward off threats from Spain. Ottoman rule was compounded over the following decades (Clissold, 1977, p. 27). 

In reality, however, the Regencies had unstable political systems with local elites vying for power. Their economy was driven by corsairing, and the real power lay in the hands of the mostly European renegades who carried out raids on land and at sea, seizing cargo and most importantly human booty, sold as slaves on their return (Plantet, 1889, introduction).

The huge influx of captured Europeans swelled the urban population and created multinational, multidenominational and notably multilingual societies. The Flemish diplomat D’Aranda, imprisoned in Algiers in the 1660s, wrote of hearing 22 languages in the slave quarters of the city (D’Aranda, 1662). Lingua Franca emerged as a contact language accessible to the majority of slaves (though not all), given its Romance-influenced lexicon. Although the elites were predominantly Arabic speaking, Europeans permeated the upper levels of Barbary society through their economic sway as corsairs and high levels of inter-marriage of Arabs and Europeans. Lingua Franca quickly became the default language within the slave quarters, known locally as \textit{bagnios}, seemingly a Lingua Franca term, and in master-slave relationships. Authors who detail the use of Lingua Franca across more than 250 years and throughout the regencies, including (Pananti, 1841) and (Broughton, 1839), report the regular use of Lingua Franca by Arabic-speaking slave owners, including the Pashas, Beys and various dignitaries of the ruling households. 

As noted above, Lingua Franca also elicits various opinions regarding its key lexifiers, though one common point of agreement among its contemporary witnesses and speakers is that Italian\footnote{Italian is a catch-all term, used by contemporary authors in Barbar , and linguists today, as identified by \citet{Trivellato2009}: \textrm{‘I write “Italian”, “Portuguese” and “Spanish”, but recall that European written languages in the epoch were not fully standardized’ \citep[178]{Trivellato2009}. Venetian and Tuscan were both described as Italian, for example.}}and Spanish are mentioned repeatedly as principal lexifiers. Descriptions mostly include at least three lexifying languages though not always the same three: while Italian and Spanish are consistently named, Provençal also features (D’Arvieux, 1735, p. 235, vol.5) as does Portuguese (D’Aranda, 1662, p. 22; Haedo, 1612, p. 24).  A much later account from the Italian merchant Pananti, briefly imprisoned in Algiers, mentions Arabic as a lexifier: \textit{la} \textit{Lingua} \textit{franca} \textit{è} \textit{una} \textit{mistura} \textit{d’italiano,} \textit{di} \textit{arabo} \textit{e} \textit{di} \textit{spagnolo} ‘Lingua Franca is a mix of Italian, Arabic and Spanish’ (Pananti, 1841, p. 201) Within the same memoir, however, Pananti refers to African rather than Arabic as one of three lexifiers of Lingua Franca (the other two being Italian and Spanish). Such inconsistency is the hallmark of many of the sources, compounding an already confused and often contradictory picture of the language. As \citet{Selbach2008} observes, ‘lexical variants were as much a part of the language as variant lexifiers’ (Selbach, 2008, p. 18).

The contribution of multiple languages to Lingua Franca is borne out by its lexicon: there are often several alternatives for a single meaning, as listed in the \textit{Dictionnaire} (Anonymous, 1830) the sole comprehensive lexical record of the language. For example, ‘to do’ is translated by \textit{far,} \textit{fazir} (from the Portuguese \textit{fazer}),and \textit{counchiar} (likely from the Sicilian \textit{cunzari}~(Cifoletti, 2004, p. 316)(Anonymous, 1830, p. 34). Lingua Franca – as substantiated by its corpus – is always labelled as such by a non-speaker. The European authors of descriptions of Lingua Franca, its diffusion and its usage, present it as, if not foreign, certainly removed and remote from their own languages, and attribute the speaking of Lingua Franca to the Arabs and Turks (even if it is clearly a language that lexically is much closer to European, and specifically Romance, languages) (Bonaparte, 1877; Clarke, 1877). In her comprehensive if subjective account and an  alysis of Lingua Franca, Dakhlia highlights how citations, often expressing insults and aggressive warnings, and usually introduced into the texts by witnesses to Lingua Franca in direct speech punctuated with exclamation marks, underline what she terms \textit{le} \textit{choc} \textit{linguistique} \textit{de} \textit{l’altérité} \textit{et} \textit{de} \textit{la} \textit{barbarie} ‘the linguistic shock (or jolt) of otherness and barbarism’ (Dakhlia, 2008, p. 351). This choice of words is important. Dakhlia’s association of otherness and barbarism suggests that Lingua Franca, according to the European documentary sources, was the language of the Arab oppressor. While this may apply to the corsairs and slave-masters, speaking Lingua Franca to their European captives, there are other instances within the corpus where Arabic elites use the language in diplomatic, even philosophical, exchanges (Frank, 1850, p. 70; Pananti, 1841, p. 69) .

Lingua Franca’s demise dates from 1830 as a consequence of the outlawing of slavery and the start of the French colonization of North Africa. As identified in correspondence from well-placed sources in Schuchardt’s archive, Lingua Franca became known alternatively as Sabir (“HSA 1882: Letter 6-7473,” n.d.), with a later incarnation which Schuchardt dubbed Judeo-Sabir (\citealt{Schuchardt1909}, trans. 1980: 87). Residual elements seem to persist, however, in other contemporary jargons and languages. The pidgin spoken in Algeria, Pataouète, while largely lexified by French and Arabic, also features a significant number of words that are identified by Lanly as Lingua Franca in origin. For example, Duclos’ 1992 Pataouète dictionary enumerates at least 30 words whose etymology she specifies as Lingua Franca. These include \textit{baroufa} ‘quarrel, \textit{fantasia} ‘pride, delusion’ \textit{mercanti} ‘merchant’, and \textit{rabia} ‘rage’ Christian’ (Duclos, 1992).

\begin{itemize}
	\item \section{Contact with Arabic}
\end{itemize}

As mentioned above, a substantial proportion of Lingua Franca speakers appear to have had Arabic as their first language. Inevitably, there would have been transfer when they spoke Lingua Franca, although, given the shared history of Arabic and European cultures in Sicily, Spain and other parts of the Mediterranean, it is perhaps hard to state unequivocally whether lexical influences stem from the contact of Arabic with Lingua Franca directly, or its earlier contact with Romance languages. \citet{Pellegrini1972} identified the many Arabic loanwords integrated into Italian, particularly in the realms of trade, conflict and exploration. A number of these are included in the \textit{Dictionnaire} \REF{ex:key:1830} \textit{magazino} ‘shop’ from the Arabic, \textit{maḫāzīn} ‘storage facility’, and \textit{fondaco} ‘trading post’ from the Arabic \textit{funduq} ‘hotel, inn’. Both would already have been in use in Italian, and, as such, this only complicates further an etymological study of Lingua Franca’s lexicon.\z

\begin{itemize}
	\item \section{Contact-induced changes in Lingua Franca}
\end{itemize}

Contact-induced changes in Lingua Franca with regard to Arabic are relatively limited, evident predominantly in its lexicon but also, to some extent, in its phonology. It is perhaps even an overstatement to consider Arabic’s influence as contact-induced change; rather it might be viewed simply as an additional lexifier.

\begin{itemize}
	\item \begin{itemize}
		\item \subsection{Phonology}
	\end{itemize}
\end{itemize}

The lack of written record and potential unreliability of the sources’ excerpts of Lingua Franca makes the identification of a definitive phonemic inventory both difficult, and at times, inconclusive. Overall, Lingua Franca follows Romance, and predominantly Italian, though with elements of Venetian and Spanish, phonology. Venetian influence is also evident in the Lingua Franca tendency to omit final vowels following -\textit{l}, -\textit{n}, -\textit{r} e.g. \textit{colazion} instead of \textit{colazione} ‘breakfast’. Both Venetian and Lingua Franca exhibit examples of degemination (e.g. \textit{tuto} vs. Tuscan \textit{tutto} ‘all’) and voicing of intervocalic stops – \textit{segredo} rather than \textit{segreto} ‘secret’ (Ursini, 2011). The voicing of \textit{t} to \textit{d} is consistent with the Spanish that also influenced phonologically elements of Lingua Franca. An example from the \textit{Dictionnaire} (1830, p. 63) that illustrates both the plosive voicing and the final vowel omission is \textit{padron} ‘master’, an epithet that recurs throughout the corpus. However, in terms of the language’s vocalic system, Arabic appears to exert influence. \citet{Cifoletti2004} suggests that Arabic influence on the realization of vocalic elements can be seen in the \textit{Dictionnaire}: \textit{bonou} from the Italian \textit{buono} ‘good’ evidences a simplification of the diphthong \textit{uo}, while \textit{gratzia} alters the final vowel of Italian \textit{grazie} ‘thank you’ to -\textit{a}. \citet{Bergareche1993} confirms the diphthong simplification; for example Italian \textit{uovo} ‘egg’, \textit{duole} ‘hurt’, \textit{buono} ‘good’ are reduced in Lingua Franca words \textit{obo,} \textit{dole,} \textit{bono} (Bergareche, 1993, p. 444). 

There is some evidence in the \textit{Dictionnaire} \REF{ex:key:1830} of reduction in the number of qualities for short vowels in Lingua Franca from a typical five-vowel Romance system to the more impoverished systems found in North African Arabic varieties, seen, for example in the frequent realization of final \-\textit{{}-a} as <e>, as in \textit{scoura} from \textit{scure} ‘axe’ or \textit{gratzia} from \textit{grazie} ‘thanks’, or <i>, as in \textit{sempri} from \textit{sempre} ‘always’ or \textit{grandi} from \textit{grande} ‘big’ . \citet{Bergareche1993} reinforces this, citing the Lingua Franca words \textit{mouchou} ‘much, many’, \textit{poudir}, ‘to be able’ \textit{inglis} ‘English’ with their roots in Spanish (\textit{mucho}, \textit{poder}, \textit{ingles}) as evidence of a reduction in vowel qualities as a result of contact with Arabic (Bergareche, 1993, p. 444). 

Unlike most words in Lingua Franca that have a typical Romance vowel ending, Arabic-derived words generally retain their consonant ending: \textit{rouss} from \textit{ruzz} ‘rice’, \textit{maboul} from \textit{mahbūl} ‘stupid’ (Cifoletti, 2004, p. 38). Non-Romance influence on Lingua Franca is also evidenced in the regular substitution of /b/ for /p/, which is lacking in the phonemic inventory of most Arabic varieties. The \textit{Dictionnaire} features a number of replacements of this type (Cifoletti, 2004, p. 38): \textit{esbinac} ‘spinach’, \textit{nabolitan} ‘Neapolitan’ (Anonymous, 1830, p. 96). In the dramatic corpus, Minervini (1996, pp. 257–60) analyses the speech of the eponymous heroine of Giancarli’s \textit{La} \textit{Zingana} \REF{ex:key:1545}, and comments on the frequent substitution of /p/ and /v/ by /b/, offering examples such as \textit{cattiba} (\textit{cattiva} ‘nasty’ in Italian), \textit{bericola} \textit{(pericolo} ‘danger’ in Italian)  (1996, p. 260), the native Arabic of the \textit{Zingana} allegedly influencing her pronunciation. Given, however, that these examples come from a work of fiction, they do not provide conclusive evidence of influence. 

Perhaps given that Lingua Franca as attested is replete with abbreviation, ellipsis, and omissions, it predictably features examples of aphaeresis. For example, many Romance-derived items beginning with a syllable that resembles the Arabic definite article see this omitted in Lingua Franca. Examples include \textit{bassiador} for \textit{ambasciatore} ‘ambassador’, \textit{bastantza} for \textit{abbastanza} ‘enough’, and \textit{rigar} for \textit{irrigare} ‘to water’. As with many other linguistic features, however, the similarity between Lingua Franca and Venetian dialect must be considered, as some of these words exist in their abbreviated form in Venetian (Schuchardt, 1909).

\begin{itemize}
	\item \begin{itemize}
		\item \subsection{Lexicon}
	\end{itemize}
\end{itemize}

From my quantitative analysis of the material in the \textit{Dictionnaire} and other available sources, it is apparent that there were very few Arabic lexemes in Lingua Franca’s lexicon. Of the more than 2,100 entries in the 1830 \textit{Dictionnaire}, 32 are of Arabic origin, and of the 176 additional Lingua Franca lexemes identified in the corpus of attestations, only nine have an Arabic etymology. However, a number of the individual items are regularly repeated in the corpus, and, as such, Arabic appears a more influential lexifier than in pure number of items.

Romance/non-Romance (often Arabic) doublets feature particularly in terms of place names, and officialdom within the Regencies. The port of Tunis was known by its French, Italian and Arabic names, seemingly interchangeably: \textit{La} \textit{Goulette,} \textit{La} \textit{Goletta,} \textit{and} \textit{Wādi} \textit{l-Ḥalq} ‘the gullet’. In his manual for future consuls, the outgoing English consul of Tripoli, Knecht, (Pennell, 1982, p. 94) enumerates the hierarchies within the Pasha’s household and city administration. Many of these involve a combination of Arabic or Turkish and Italian, or perhaps Lingua Franca. Key positions include a \textit{hasnadawr} \textit{Grande} \textit{ed} \textit{un} \textit{hasndawr} \textit{Piccolo} – ‘a senior treasurer and a junior treasurer (Pennell, 1982: 97; my translation).  \textit{Hasnadawr} comes from the Ottoman Turkish \textit{hazinedar} or \textit{haznadar} (from Arabic \textit{ḫāzin} \textit{ad-dār}) ‘Lord treasurer of the household’ (Gilson, 1987, p. 167).

Another example is \textit{Kecchia} \textit{Grande} and \textit{Kecchia} \textit{Piccolo} ‘chief administrator and assistant administrator’ (Pennell, 1982, p. 104). In the letters written by members of the household of Richard Tully, British Consul to Tripoli at the close of the eighteenth century, there is a reference to the ‘Great Chiah and the Little Chiah (Tully, 1819, p. 70 vol.1), surely an Anglicization of the title. \textit{Kecchia} may derive from the Tunisian Arabic \textit{kāhiya} ‘chief officer of an administrative district’ – \textit{kecchia} is an Italianised (or, again, possibly Lingua Franca-influenced) orthography and pronunciation. Similarly, \textit{sotto} \textit{rais} (from the Italian and Arabic literally meaning ‘under captain’) denoted the second in command of the harbour (Pennell, 1982: 97, 100). The commander is referred to separately as the \textit{rays} \textit{de} \textit{la} \textit{marina} ‘chief of the port’ (Pennell, 1982, p. 92). Again, one finds the combination of Arabic and Italian. (\textit{Rays} is spelt in two different ways,\footnote{raʾīs is the standard Arabic form.} which highlights how, pre-standardisation of European languages, orthography was erratic, even within a single document.) Another example is a proverb regarding the eradication of plague by the Day of St John. Two variations on the proverb are cited – by Poiret in 1802 and by (Rehbinder, 1800) 

(Poiret, 1802, p. 227): \textit{Saint} \textit{Jean} \textit{venir,} \textit{Gandouf} \textit{andar} ‘[The day of] St. John comes, the plague leaves’

(Rehbinder, 1800): \textit{Saint} \textit{Jean} \textit{venir,} \textit{buba} \textit{andar} ‘[The day of] St. John comes, the plague leaves’. \textit{Buba} ‘plague’ appears to come originally from the Greek, \textit{$\beta} \textit{o\upsilon} \textit{\beta} \textit{\text{\textgreek{'w}}\nu $, boubōn} ‘groin’, suggesting yet another potential lexifying influence on Lingua Franca. 

Selbach remarks on how such nomenclature in Lingua Franca ‘allowed for much room to manoeuver, and for speakers to mark their religious, political and cultural identity’ (Selbach, 2008). \textit{Gandouf} could also derive plausibly from the Arabic (Selbach, 2008, p. 45). According to Schuchardt, \textit{gandouf/gandufa} comes from the Arabic \textit{ɣunduba/ɣundūb} ‘swollen tonsils’ (\citealt{Schuchardt1909}, 72). 

This example raises the issue of words already common to Arabic and Romance languages, since contact between them, as discussed above, had been prolonged and extensive. Similarly, the French (and possibly Italian or even Lingua Franca word \textit{avanie} ‘fine, insult, affront’) occurs in the corpus (Grandchamp, 1920; Pananti, 1841). Grandchamp defines it thus: 

\begin{quote}
	les avanies étaient des sommes d'argent que les pachas réclamaient aux marchands des échelles sous les prétextes les plus divers, prétextes la plupart du temps injustes, parfois extrêmement bizarres
\end{quote}

\begin{quote}
	‘the fines were sums of money the Pashas demanded of the Levant merchants on various pretexts, pretexts that were for the most part unfair, and at times extremely strange’ (Grandchamp, 1920, p. xiii).
\end{quote}

Although this word would appear to be derived from French, or at least a Romance language, given it was the creation of Ottoman elites, it seems more likely that its origins are Turkish. This is confirmed by \citet{Pihan1847} who suggests that it actually originally derives from the Arabic \textit{hawān} ‘contempt’\footnote{This derivation is confirmed in Le Trésor de la langue francaise informatisé (Dendien, 1994)}, but that,

\begin{quote}
	se dit également des impôts énormes que les Turcs font peser sur les Chrétiens dans le but de les humilier
\end{quote}

\begin{quote}
	‘it applies equally to the enormous taxes the Turks impose on Christians with the goal of humiliating them’ (Pihan, 1847, p. 46)
\end{quote}

Additionally, there are words that appear to have etymologies in multiple languages that are rarely translated, at least by English sources like \citet{Tully1819}.Two terms with similar meanings, \textit{firman} and \textit{teschera} ‘pass, decree’ and ‘pass, edict’, both issued by Ottoman or Arabic rulers, also bear remarkable similarity to Italian words with comparable meanings. \textit{Firman} is from Arabic: \textit{firmān} \textit{/} \textit{faramān}, though originally Persian, and would have come into Arabic through Ottoman Turkish, once again reinforcing how the languages spoken in the region were far from discrete entities. However, \textit{firmare} in Italian means to sign, and the Lingua Franca translationof the French \textit{seing} ‘signature’ and \textit{signature} ‘signature’ in the \textit{Dictionnaire} (1830) is \textit{firmar}. A decree or pass (allowing free passage or safe conduct) would necessarily require an official signature. \textit{Teschera} ‘pass, edict’ might appear to come from the Italian \textit{tessera} ‘pass, ticket’ but there is also the Arabic word \textit{taðkira/taðkara/tazkira/tazakara} (all variant realizations of the same item), which also means ‘permit’ or ‘ticket’. Both words seem integral to Barbary life, and are not translated. \citet{Tully1819} writes: 

\begin{quote}
	‘It is still affirmed that he has a teskerra, or firman, with him for this unfortunate Bashaw. A teskerra is a written order from the Grand Signior, and is held so sacred that every Musulman who receives it must obey its mandate, even to death’ (Tully, 1819, p. 258 vol.1).
\end{quote}

Perhaps the most iconic word in Lingua Franca is \textit{fantasia}. It is mentioned by multiple sources (Broughton, 1839; Haedo, 1612), spanning more than two centuries. Although it appears to be a Romance word, (Schuchardt, 1909, p. 71) confirms that it is used in the Arabic sense of pride, arrogance, as in, for example, Egyptian Arabic \textit{itfanṭaz} ‘to give oneself airs’.

The frequent assimilation of apparent Lingua Franca terms,particularly evident in English sources (Broughton, 1839; Tully, 1819)underlines the extensive uptake of the language and its spread far beyond the master–slave domain into daily life. Several sources testify to the co-opting of Lingua Franca as a more neutral means of communication between Arabic-speaking elites and Europeans. Louis Frank, the Bey of Tunis’ doctor, comments on the deemed impropriety of the Bey speaking formal Italian, and his consequent use of Lingua Franca which permeated all levels of society (Frank, 1850, p. 70):

\begin{quote}
	la langue franque, c’est à dire cet italien ou provençal corrompu qu’on parle dans le Levant, lui est également familière; il avait meme voulu essayer d’apprendre à lire et à écrire l’italien pur-toscan: mais les chefs de la religion l’ont détourné de cette etude, qu’ils prétendaient être indigne d’un prince musulman.
\end{quote}

\begin{quote}
	‘lingua franca, or rather this bad Italian or Provençal spoken in the Levant, is equally familiar to him; he had actually wanted to learn to read and write  pure Tuscan Italian; but his religious chiefs had warned him off such study, which they claimed was unworthy of a Muslim prince’. (Frank, 1850, p. 70; my translation)
\end{quote}

Frank wrote of the intriguing linguistic, socio-political, cultural and even religious conflation evidenced in Lingua Franca, writing of an encounter with a Muslim beggar, who implored: “\textit{Donar} \textit{mi} \textit{meschino} \textit{la} \textit{carità} \textit{d’una} \textit{carrouba}\footnote{\citealt{Until1891} a \textit{carrouba} was worth 1/16 of a Tunisian \textit{piaster}, according to \citet{Rossetti1999}.(1841)} \textit{per} \textit{l’amor} \textit{della} \textit{Santissima} \textit{Trinità} \textit{e} \textit{dello} \textit{gran} \textit{Bonaparte”} ‘Please to give miserable me the charity of a \textit{penny} for the love of the most holy Trinity and the great Bonaparte’ (Frank, 1850, p. 101; my translation). In just this one Lingua Franca sentence, multiple lexifiers are represented: \textit{meschino} is from the Arabic, \textit{miskīn}, and there is Spanish in \textit{carrouba} ‘penny’, \textit{donar} ‘give’, \textit{amor} ‘love’, with the Italian Catholic reference of \textit{Santissima} \textit{Trinità} ‘most holy Trinity’, and French \textit{Bonaparte}. The latter would have still been Emperor and possibly at the height of his power. Bonaparte is qualified by \textit{gran} ‘great’, from the Italian or even Venetian. It suggests how cosmopolitan, multicultural and multilingual Tunis and its population had become that a beggar should speak this way. Even Frank was struck by the incongruity of the beggar’s words: \textit{sa} \textit{supplique} \textit{en} \textit{ces} \textit{termes,} \textit{bien} \textit{étranges} \textit{dans} \textit{la} \textit{bouche} \textit{d’un} \textit{Musulman} ‘his petition in these terms, very odd in the mouth of a Muslim’ (Frank, 1850, p. 101; my translation).

Collections in the National Archives provide some limited evidence of borrowings from Lingua Franca (as opposed to other Romance languages) into Arabic. \citeapo{Hopkins1982} research in the National Archives focuses on two sets of English (and later British) state papers relating to the Barbary regencies and Morocco, including \REF{ex:key:2008}correspondence in Arabic, from the late 16\textsuperscript{th} to late 18\textsuperscript{th} centuries. 

Hopkins adds a glossary to his translations that demonstrates the extent to which the letters from the Barbary States disproportionately feature loans from Romance languages. Hopkins comments that “[f]oreign words are very common and seem to be used quite unselfconsciously (Hopkins, 1982, p. x)”, and many of those words Hopkins isolates are also listed in the \textit{Dictionnaire} \REF{ex:key:1830} as Lingua Franca words. Three such words - \textit{justisiya} ‘justice’ and \textit{markānti} ‘merchant’ and \textit{zabantut} (\textit{sbendout} in Lingua Franca, presumably from the Italian \textit{bandito}) ‘pirate’ occur in a single letter of unspecified provenance but written to the King of England in 1730 by a man claiming to be an Algerine trader in Tripoli (“TNA: SP 71/23/51,” 1730). The incidence of three non-Arabic, Romance words, is noteworthy and it seems plausible that these were Lingua Franca terms in such common usage, that they would be borrowed as a native language alternative, an example of \textit{mot} \textit{juste} switching (Gardner-Chloros, 2009, p. 32).

\begin{itemize}
	\item \section{Conclusion}
\end{itemize}

As demonstrated, the corpus of Lingua Franca – both documentary texts written by European visitors to the Barbary states and the the dramatic works produced by contemporary authors – is limited in lexicon and grammar. This makes description of Lingua Franca challenging, and, likewise, any concrete and substantiated analysis of its relationship with other, particularly non-European, languages.

Nevertheless, this chapter has suggested how Arabic and Romance languages influenced the emergence of Lingua Franca, specifically in terms of its lexicon and phonology. Authors throughout the era of Lingua Franca’s existence, from \citet{Haedo1612} to \citet{Broughton1839} and \citet{Frank1850} reiterate that, despite its overwhelming Romance base, Lingua Franca was spoken predominantly by the often-Arabic speaking slavemasters and rulers of the Barbary states. The plurilingual character of the population of this region, both collectively and individually, compounds an already unclear picture, however, as the fluidity of Barbary society led to European (and often Romance-langauge speaking) corsairs and diplomats alike permeating its upper echelons (\citealt{Haedo1612}: 9; \citealt{Garcès2011}: 129).

The lexical influence of Arabic is most evident in the Romance / Arabic doublets used in official terms and place-names. These are often compound terms, such as \textit{ra’ïs} \textit{de} \textit{la} \textit{marina} ‘captain of the port’. Warrington, English Consul to Tripoli in the late 18\textsuperscript{th} century, uses an Anglicised version of the phrase, \textit{rays} \textit{marina}, suggesting the ubiquity of such doublets (“TNA: FO 161/9,” n.d.) Further evidence from the National Archives (Hopkins, 1982) demonstrates that Lingua Franca words were borrowed in correspondence from Arabic-speaking dignitaries in the Barbary Regencies to the English Secretary of State, reinforcing Lingua Franca borrowings in the written as well as the oral domain. 

In terms of phonology, there seems to be evidence offered by the \textit{Dictionnaire} and the observation of \citet{Haedo1612} of the influence of Arabic on Lingua Franca. Haedo also stated, with regard to the ‘Moors and Turks’ that \textit{no} \textit{saben} \textit{ellos} \textit{variar} \textit{los} \textit{modos,} \textit{tiempos} \textit{y} \textit{casos} ‘they don’t know about gender, tenses and cases’ (Haedo, 1612, p. 24). Given that Lingua Franca lacks for the most part any verbal inflection and an absence of cases, this might be, as Haedo says, a result of contact, but it is typical of most pidgins and cannot be attributed solely to contact with Arabic. Such lack of certainty applies more generally. Arabic evidently exerted some influence on the evolution of Lingua Franca in North Africa, but not to the extent that it can straightforwardly be classified as contact-induced change.

\begin{stylelsUnNumberedSection}
	Further reading
\end{stylelsUnNumberedSection}

Comprehensive English-language introductions to Lingua Franca include my PhD thesis (Nolan, 2018) and Alan D. Corré’s website, \url{https://minds.wisconsin.edu/bitstream/item/3920/go.html}. \citet{Corré2005} compiles a substantial number of texts featuring Lingua Franca, offering translations from the original European languages and a selection of articles analysing and contexualising Lingua Franca. He also provides a glossary, largely derived from the \textit{Dictionnaire}  and Schuchardt’s seminal article \REF{ex:key:1909}. 

Schuchardt’s article \REF{ex:key:1909} offers the earliest detailed historical and linguistic analysis of Lingua Franca. More recent in-depth texts include \citegen{Dakhlia2008} rather subjective book, \citegen{Minervini1996} comprehensive article that focuses predominantly on Italian and French sources, and \citegen{Cifoletti2004} forensic biography and analysis of Lingua Franca, again based almost exlusively on Romance language sources. 

The key source authors are inevitably \citet{Haedo1612}, \citet{Broughton1839} \citet{Pananti1841} and \citet{Frank1850}.

\begin{stylelsUnNumberedSection}
	Acknowledgements
\end{stylelsUnNumberedSection}

\begin{stylelsUnNumberedSection}
	Abbreviations
\end{stylelsUnNumberedSection}

HSA  Hugo Schuchardt Archiv

TNA  The National Archive

\begin{stylelsUnNumberedSection}
	\begin{verbatim}%%move bib entries to  localbibliography.bib
	\end{stylelsUnNumberedSection}
	
	\begin{styleBibliography}
	@misc{Anonymous1830,
	author = {Anonymous},
	note = {Feissat et Demonchy, Marseille.},
	title = {Dictionnaire de la Langue Franque ou petit Mauresque, suivi de quelques dialogues familiers, et d’un vocabulaire de mots arabes les plus usuels; a l’usage des français en Afrique},
	year = {1830}
	}
	
	\end{styleBibliography}
	
	\begin{styleBibliography}
	@book{Bergareche1993,
	address = {El estudio de la lingua franca},
	author = {Bergareche, B. C.},
	note = {Rev. Linguist. Romane 57.},
	publisher = {cuestiones pendientes},
	year = {1993}
	}
	
	\end{styleBibliography}
	
	\begin{styleBibliography}
	@misc{Bonaparte1877,
	author = {Bonaparte, L. L.},
	title = {\citealt{Athenaeum2586}},
	year = {1877}
	}
	
	\end{styleBibliography}
	
	\begin{styleBibliography}
	@misc{Broughton1839,
	author = {Broughton, E.},
	note = {Saunders and Otley, London.},
	title = {Six years residence in Algiers},
	year = {1839}
	}
	
	\end{styleBibliography}
	
	\begin{styleBibliography}
	@misc{Cifoletti2004,
	author = {Cifoletti, G.},
	note = {Il Calamo, Roma.},
	title = {{La} Lingua Franca Barbaresca},
	year = {2004}
	}
	
	\end{styleBibliography}
	
	\begin{styleBibliography}
	@misc{Clarke1877,
	author = {Clarke, H.},
	title = {\citealt{Athenaeum2583}},
	year = {1877}
	}
	
	\end{styleBibliography}
	
	\begin{styleBibliography}
	@misc{Clissold1977,
	author = {Clissold, S.},
	note = {Edek, London.},
	title = {The Barbary Slaves},
	year = {1977}
	}
	
	\end{styleBibliography}
	
	\begin{styleBibliography}
	@misc{Corré2005,
	author = {Corré, A. D.},
	title = {A glossary of Lingua Franca - fifth edition},
	year = {2005}
	}
	
	\end{styleBibliography}
	
	\begin{styleBibliography}
	@misc{Dakhlia2008,
	author = {Dakhlia, J.},
	note = {Actes Sud, Arles.},
	title = {{LI}ngua Franca},
	year = {2008}
	}
	
	\end{styleBibliography}
	
	\begin{styleBibliography}
	@misc{Dan1637,
	author = {Dan, P.},
	note = {P. Rocolet, Paris.},
	title = {{Histoire} de la Barbarie , et de ses corsaires divisée en six livres},
	year = {1637}
	}
	
	\end{styleBibliography}
	
	\begin{styleBibliography}
	@misc{D’Aranda1662,
	author = {D’Aranda, E.},
	note = {Jean Mommart, Brussels.},
	title = {Relation de la captivité et liberation de Sieur Emanuel d’Aranda, jadis esclave à Alger},
	year = {1662}
	}
	
	\end{styleBibliography}
	
	\begin{styleBibliography}
	@misc{D’Arvieux1735,
	author = {D’Arvieux, L.},
	note = {Jean-Charles-Baptiste Delespine, Paris.},
	title = {Mémoires du Chevalier D’Arvieux},
	year = {1735}
	}
	
	\end{styleBibliography}
	
	\begin{styleBibliography}
	@book{Davis2004,
	address = {Christian Slaves, Muslim Masters},
	author = {Davis, R. C.},
	note = {Palgrave Macmillan, London.},
	publisher = {White Slavery in the Mediterranean, the Barbary Coast and Italy, 1500-1800},
	year = {2004}
	}
	
	\end{styleBibliography}
	
	\begin{styleBibliography}
	@misc{Dendien1994,
	author = {Dendien, J.},
	note = {Trésor Lang. Fr. Informatisée.},
	title = {Avanie},
	year = {1994}
	}
	
	\end{styleBibliography}
	
	\begin{styleBibliography}
	@book{Duclos1992,
	address = {Dictionnaire du francais d’Algérie},
	author = {Duclos, J.},
	note = {Editions Bonneton, Paris.},
	publisher = {français colonial, pataouète, français des Pieds-Noirs},
	year = {1992}
	}
	
	\end{styleBibliography}
	
	\begin{styleBibliography}
	@misc{Frank1850,
	author = {Frank, L.},
	note = {Firmin Dido, Paris.},
	title = {Algérie, Etats Tripolitains, {Tunis}},
	year = {1850}
	}
	
	\end{styleBibliography}
	
	\begin{styleBibliography}
	@misc{Gardner-Chloros2009,
	author = {Gardner-Chloros, P.},
	note = {Cambridge University Press, Cambridge.},
	title = {Code-Switching},
	year = {2009}
	}
	
	\end{styleBibliography}
	
	\begin{styleBibliography}
	Giancarli, G., 1545. La Zingana. mantova.
	\end{styleBibliography}
	
	\begin{styleBibliography}
	@book{Gilson1987,
	address = {The Turkish grammar of Thomas Vaughan},
	author = {Gilson, E. H.},
	note = {O. Harrassowitz, Wiesbaden.},
	publisher = {Ottoman Turkish at the end of the XVIIth century,},
	year = {1987}
	}
	
	\end{styleBibliography}
	
	\begin{styleBibliography}
	@misc{Grandchamp1920,
	author = {Grandchamp, P.},
	note = {Société Anonyme de L’imprimerie Rapide, Tunis.},
	title = {{La} {France} en Tunisie à la fin du {XVI}e siècle (1582-1600)},
	year = {1920}
	}
	
	\end{styleBibliography}
	
	\begin{styleBibliography}
	@article{Grion1890,
	author = {Grion, G.},
	journal = {Arch. Glottol. Ital},
	pages = {181–186},
	title = {Farmacopea e lingua franca del Dugento},
	volume = {12},
	year = {1890}
	}
	
	\end{styleBibliography}
	
	\begin{styleBibliography}
	@misc{Haedo1612,
	author = {Haedo, D. de},
	note = {Vallodid.},
	title = {Topographia e historia general de Argel},
	year = {1612}
	}
	
	\end{styleBibliography}
	
	\begin{styleBibliography}
	@misc{Hall1966,
	author = {Hall, R. A.J},
	note = {Cornell University Press.},
	title = {{Pidgin} and {Creole} Languages},
	year = {1966}
	}
	
	\end{styleBibliography}
	
	\begin{styleBibliography}
	@book{Hopkins1982,
	address = {Letters from \citealt{Barbary1576}-1744},
	author = {Hopkins, J. F.},
	note = {Oxford University Press, Oxford.},
	publisher = {Arabic documents in the Public Record Office},
	year = {1982}
	}
	
	\end{styleBibliography}
	
	\begin{styleBibliography}
	HSA 1882: Letter 6-7473, n.d.
	\end{styleBibliography}
	
	\begin{styleBibliography}
	@book{Kahane1976,
	address = {Lingua Franca},
	author = {Kahane, H., Kahane, R.},
	note = {Roman. Philol. XXX, 25–41.},
	publisher = {the story of a term},
	year = {1976}
	}
	
	\end{styleBibliography}
	
	\begin{styleBibliography}
	@misc{Minervini1996,
	author = {Minervini, L.},
	note = {Plurilinguismo, mistilinguismo, pidginizzazione sulle coste del Mediterraneo tra tardo Medioevo e prima età moderna. Medioevo Romanzo XX, 231–301.},
	title = {{La} lingua franca mediterranea},
	year = {1996}
	}
	
	\end{styleBibliography}
	
	\begin{styleBibliography}
	@misc{Molière1798,
	author = {Molière, J.-B. P.},
	note = {Jean Mossy, Marseille.},
	title = {{Le} Bourgeois Gentilhomme},
	year = {1798}
	}
	
	\end{styleBibliography}
	
	\begin{styleBibliography}
	@misc{Molière1667,
	author = {Molière, J.-B. P.},
	note = {Jean Ribov, Paris.},
	title = {{Le} Sicilien},
	year = {1667}
	}
	
	\end{styleBibliography}
	
	\begin{styleBibliography}
	Mori, L., 2016. Italiano di contatto nel Mediterraneo moderno, in: Muru, C., Di Salvo, M. (Eds.), Dragomanni, Sovrani e Mercanti: Pratiche Linguistiche Nelle Relazioni Politiche e Commerciali Del Mediterraneo Moderni. Edizione ETS, Pisa.
	\end{styleBibliography}
	
	\begin{styleBibliography}
	@book{Nolan2018,
	address = {Fact and Fiction},
	author = {Nolan, J.},
	note = {SOAS, London.},
	publisher = {A re-evaluation of Lingua Franca},
	year = {2018}
	}
	
	\end{styleBibliography}
	
	\begin{styleBibliography}
	@misc{Pananti1841,
	author = {Pananti, F.},
	note = {Milan.},
	title = {Avventure e osservazioni sopra le coste di Barberia},
	year = {1841}
	}
	
	\end{styleBibliography}
	
	\begin{styleBibliography}
	@misc{Pellegrini1972,
	author = {Pellegrini, G. B.},
	note = {Padeia, Brescia.},
	title = {Gli Arabismi nelle lingue neolatine con speciale riguardo all’{Italia}},
	year = {1972}
	}
	
	\end{styleBibliography}
	
	\begin{styleBibliography}
	@article{Pennell1982,
	author = {Pennell},
	journal = {Rev. D’histoire Maghrébine CR 25–},
	pages = {91–121},
	title = {Tripoli in the mid eighteenth century: {{A}} guidebook to the city in 1767},
	volume = {26},
	year = {1982}
	}
	
	\end{styleBibliography}
	
	\begin{styleBibliography}
	@misc{Pihan1847,
	author = {Pihan, A. P.},
	note = {Benjamin Duprat, Paris.},
	title = {Glossaire des mots français de l’arabe, du persan et du turc},
	year = {1847}
	}
	
	\end{styleBibliography}
	
	\begin{styleBibliography}
	@misc{Plantet1889,
	author = {Plantet, E.},
	note = {F. Alcan, Paris.},
	title = {Correspondance des deys d’Alger avec la cour de {France}, 1579-1833},
	year = {1889}
	}
	
	\end{styleBibliography}
	
	\begin{styleBibliography}
	@misc{Poiret1802,
	author = {Poiret, J. L.M},
	note = {Callixte Voland, Paris.},
	title = {Voyage en Barbarie ou lettres écrites de l’ancienne Numidie},
	year = {1802}
	}
	
	\end{styleBibliography}
	
	\begin{styleBibliography}
	@misc{Rehbinder1800,
	author = {Rehbinder, J. von},
	note = {Hammerich, Altona.},
	title = {Nachricten und {Bemerkungen} über den algerischen \citealt{Staat1798}-1800},
	year = {1800}
	}
	
	\end{styleBibliography}
	
	\begin{styleBibliography}
	@article{Rossetti1999,
	author = {Rossetti, R.},
	journal = {Englishes Lett. Inglesi Contemp},
	pages = {42–62},
	title = {An introduction to Lingua Franca},
	volume = {8},
	year = {1999}
	}
	
	\end{styleBibliography}
	
	\begin{styleBibliography}
	@misc{SavarydeBrèves1628,
	author = {Savary de Brèves, F.},
	note = {Nicolas Gasse, Paris.},
	title = {Relation des voyages de monsieur de Brèves, tant en Grèce, Terre Saincte et Aegypte, qu’aux Royaumes de {Tunis} et Arger},
	year = {1628}
	}
	
	\end{styleBibliography}
	
	\begin{styleBibliography}
	@misc{Schuchardt1909,
	author = {Schuchardt, H.},
	note = {Cambridge University Press, Cambridge.},
	title = {{Pidgin} and {Creole} Languages},
	year = {1909}
	}
	
	\end{styleBibliography}
	
	\begin{styleBibliography}
	Selbach, R., 2008. The superstrate is not always the lexifier: Lingua Franca in the Barbary \citealt{Coast1530}-1830, in: MIchaelis, Suzanne (Ed.), Roots of Creole Structures: Weighing the Contribution of Substrates and Superstrates. John Benjamins, Amsterdam, pp. 29–58.
	\end{styleBibliography}
	
	\begin{styleBibliography}
	@book{Tinniswood2010,
	address = {Pirates of Barbary},
	author = {Tinniswood, A.},
	note = {Vintage Books, London.},
	publisher = {Corsairs, conquests and captivity in the 17th century,},
	year = {2010}
	}
	
	\end{styleBibliography}
	
	\begin{styleBibliography}
	TNA: FO 161/9, n.d.
	\end{styleBibliography}
	
	\begin{styleBibliography}
	TNA: SP 71/23/51, 1730.
	\end{styleBibliography}
	
	\begin{styleBibliography}
	@book{Tully1819,
	address = {Ten Years’ Residence at the Court of Tripoli},
	author = {Tully, M.},
	note = {H. Colburn, London.},
	publisher = {Letters Written During a Ten Years’ Residence at the Court of \citealt{Tripoli1783}-1995},
	year = {1819}
	}
	
	\end{styleBibliography}
	
	\begin{styleBibliography}
	@book{Ursini2011,
	address = {Dialetti veneti, in},
	author = {Ursini, F.},
	note = {Treccani, Rome.},
	publisher = {Enciclopedia Italiana},
	year = {2011}
	}
	
	\end{styleBibliography}
	
	\begin{styleBibliography}
	@book{Weiss2011,
	address = {Captives and Corsairs},
	author = {Weiss, G.},
	note = {University of California Press, Stanford.},
	publisher = {France and slavery in the early  modern Mediterranean},
	year = {2011}
	}
	
	\end{styleBibliography}
	
	\end{verbatim} 