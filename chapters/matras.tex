\documentclass[output=paper]{langsci/langscibook} 
\author{Yaron Matras\affiliation{University of Manchester}}
\title{Jerusalem Domari}
% \keywords{ Domari, Jerusalem, fusion, compartmentalisation, repertoire} 
\abstract{Jerusalem Domari is the only variety of Domari for which there is comprehensive documentation. The language shows massive influence of Arabic in different areas of structure –~quite possibly the most extensive structural impact of Arabic on any other language documented to date. Arabic influence on Jerusalem Domari raises theoretical questions around key concepts of contact-induced change as well as the relations between systems of grammar and the components of multilingual repertoires; these are dealt with briefly in the chapter, along with the notions of fusion, compartmentalisation of paradigms, and bilingual suppletion.}
\maketitle

\begin{document}
 
\section{ Historical development and current state
}

Domari is a dispersed, non-territorial minority language of \ili{Indo-Aryan} origin that is spoken by traditionally itinerant (peripatetic) populations throughout the Middle East. Fragmented attestations of the language place it as far north as Azerbaijan and as far south as Sudan. The self-appellation \textit{dōm} is \isi{cognate} with those of the \textit{řom} (Roma or Romanies) of Europe and the \textit{lom} of the Caucasus and eastern \isi{Anatolia}. All three populations show linguistic resources of \ili{Indo-Aryan} origin (which in the case of the Lom are limited to vocabulary), as well as traditions of a mobile service economy, and are therefore all believed to have descended from itinerant service castes in India known as \textit{ḍom}. Some Domari-speaking populations are reported to use additional names, including \textit{qurbāṭi} (Syria and Lebanon), \textit{mıtrıp} or \textit{karači} (Turkey and northern Iraq) and \textit{bahlawān} (Sudan), while the surrounding \ili{Arabic}-speaking populations usually refer to them as \textit{nawar}, \textit{ɣaǧar} or \textit{miṭribiyya}. The language retains basic vocabulary of \ili{Indo-Aryan} origin, and shows elements of lexical phonology that place its early development within the {Central} \ili{Indo-Aryan} group of languages. It retains conservative \isi{derivational} as well as present-\isi{tense} \isi{inflectional} verb morphology that goes back to late Middle \ili{Indo-Aryan}, alongside innovations in nominal and past-\isi{tense} verb \isi{inflection} that suggest that the language was contiguous with the Northwestern frontier languages (\ili{Dardic}) during the transition to early modern \ili{Indo-Aryan} (cf. \citealt{Matras2012}).

The first attestation of Palestinian Domari is a list of words and phrases collected by Ulrich Jasper Seetzen\ia{Seetzen, Ulrich Jasper@Seetzen, Ulrich Jasper} in 1806 in the West Bank and published by \citet{Kruse1854}. It was followed by Macalister's (\citeyear{Macalister1914}) grammatical sketch, texts and lexicon, collected in Jerusalem in a community which at the time was still nomadic, moving between the principal West Bank cities of Nablus, Jerusalem and Hebron. This community settled in Jerusalem in the early 1920s, the men taking up wage employment with the British-run municipal services. In the 1940s they abandoned their makeshift tent encampment and moved into rented \isi{accommodation} within the Old City walls, where the community still resides today. Between 1996 and 2000 I carried out fieldwork among speakers in Jerusalem and published a series of works on the language, including two descriptive outlines (\citealt{Matras1999,Matras2011Domari}), annotated stories \citep{Matras2000}, an overview of contact influences \citep{Matras2007Domari}, and a descriptive monograph \citep{Matras2012}. 

A number of sources going back to \citet{Pott1846}, \citet{Newbold1856}, \citet{Paspati1870}, \citet{Patkanoff1907}, and \citet{Black1913} provide language samples collected among the Dom of Lebanon, Syria, Iraq and the Caucasus. These are supplemented by a few more samples collected by ethnographers (cf. \citealt{Matras2012}: 15ff.) and subsequently by data collected in Syria and Lebanon by \citet{Herin2012}. That documentation allowed me to identify a number of differences that appeared to separate a \ili{Northern} group of Domari dialects from a \ili{Southern} group, which latter includes the data recorded in Palestine as well as a sample from Jordan (see \citealt{Matras2012}: 15ff.). That tentative classification has since been embraced by \citet{Herin2014Domari}, who goes a step further and speculates about an early split between two branches of the language. To date, however, published attestation of {Northern} varieties remains extremely fragmented, notwithstanding recent work by Herin (\citeyear{Herin2016}; this volume), while the only comprehensive overview of a {Southern} variety remains that from Jerusalem.

Outside of Jerusalem and its outskirts there are known communities of {Palestinian} Doms in some of the refugee camps on the West Bank and Gaza, as well as in Amman, where a few families sought refuge in 1967. Numbers of speakers were very low in all these communities already in the mid 1990s and the language was only in use among the elderly. During my most recent visit to the Jerusalem community, in  January 2017, it appeared that there was only one single fluent speaker left, who, for obvious reasons, no longer had any practical use for the language, apart from flagging the odd phrase to younger-generation semi-speakers. Jerusalem Domari, and most likely Palestinian Domari in general, must therefore now be considered to be nearly extinct.

\section{Contact languages
}\label{sec2}

Given the migration route that the Dom will have taken to reach the Middle East from South Asia, it is plausible that the language was subjected to repeated and extensive contact influences. \ili{Kurdish} influences on Jerusalem Domari, some of them attributable specifically to \ili{Sorani} \ili{Kurdish}, and some \ili{Persian} items, are apparent in vocabulary, while some of the morpho-syntactic structures (such as extensive use of person affixes, and the use of a uniform synthetic marker of remote \isi{tense} that is external to the person marker) align themselves with various \ili{Iranian} languages. There is also a layer of \ili{Turkic} loans, some of which may be attributable to \ili{Azeri} varieties, while others are traceable to Ottoman rule in Palestine; such items are numerous in the wordlists compiled by Seetzen and Macalister during the Ottoman period,\ia{Seetzen, Ulrich Jasper@Seetzen, Ulrich Jasper}\ia{Macalister, Robert Alexander Stewart@Macalister, Robert Alexander Stewart} but are much less frequent in the materials collected a century later (for a discussion of etymological sources see \citealt{Matras2012}: 426–429).

The circumstances under which speakers of Domari first came into contact with \ili{Arabic} are unknown. There are some indications of a layered influence: Domari tends to retain historical /q/ in \ili{Arabic}-derived words, as in \textit{qahwa} ‘coffee’, \textit{qabil} ‘before’, \textit{qaddēš} ‘how much’, as found in the rural dialects of the West Bank (and elsewhere), whereas contemporary \ili{Jerusalem} \ili{Arabic} (also used by Doms when speaking \ili{Arabic}) shows a glottal stop, as in \textit{ʔahwe}, \textit{ʔabl,} \textit{ʔaddēš}; the word for ‘now’ is \textit{hessaʕ}, while \ili{Jerusalem} \ili{Arabic} has \textit{hallaʔ}. It appears that the community has been fully bilingual in \ili{Arabic} and Domari at least since the early 1800s, with knowledge of \ili{Turkish} having been widespread among adults during the Ottoman rule. Due to the nature of the Doms’ service economy, \ili{Arabic} was an essential vehicle of all professional life, whether metalwork, hawking, begging, or performance, but Domari remained the language of the household until the introduction of compulsory school education under {Jordanian} rule in the 1950s–60s, at which point parents ceased to pass on the language to children. By the 1990s, use of Domari was limited to a small circle of perhaps around forty–fifty elderly people. Due to the multi-generational structure of households it was rare even then for conversations to be held exclusively among Domari speakers. Domari–\ili{Arabic} \isi{bilingualism} has always been unidirectional, with \ili{Arabic} being the language of commerce and public interactions for all Doms, and more recently also of education and media, eventually replacing Domari as a home and community language.

\section{Contact-induced changes in Jerusalem Domari
}

As a result of ubiquitous \isi{bilingualism} among all Domari speakers, Domari talk is chequered not only with expressions that derive from \ili{Arabic}, but also with switches into \ili{Arabic} for stylistic and discourse-strategic purposes such as \isi{emphasis}, direct quotes, side remarks, and so on. The structural intertwining of Domari and \ili{Arabic}, and the degree to which active bilingual speakers maintain a license to incorporate \ili{Arabic} elements into Domari conversation, pose a potential challenge to the descriptive agenda. In the following I discuss those structures that derive from \ili{Arabic}, and are shared with \ili{Arabic} (in the sense that they are employed by speakers both in the context of Domari conversation and in interactions in \ili{Arabic}) but constitute a stable and integral part of the structural inventory of Domari without which Domari talk cannot be formed, and for which there is no non-\ili{Arabic} Domari alternative. All examples are taken from the Jerusalem Domari corpus described in \citet{Matras2012}. Examples from \ili{Arabic} are based on colloquial \ili{Palestinian} \ili{Arabic} as spoken in Jerusalem.

\subsection{Phonology
}

The entire inventory of \ili{Palestinian} \ili{Arabic} phonemes is available in Domari; \ili{Arabic}-derived words that are used in Domari conversation (whether or not they have non-\ili{Arabic} substitutes) do not undergo phonological or phonetic integration, except for the application of Domari grammatical word \isi{stress} on case-inflected nouns (e.g. \textit{lambá} ‘lamp.\textsc{acc}’, from \ili{Arabic} \textit{lámba}). The \isi{pharyngeals} [ḥ] and [{ʕ}] are limited to \ili{Arabic}-derived vocabulary. The sounds [q], [{ɣ}] and [ḷ] as well as [z] and [f] appear primarily in \ili{Arabic}-derived vocabulary, but there is evidence that they entered the language already through contact with \ili{Turkic} and \ili{Iranian} languages. Less clear is the status of the pharyngealised dental consonants /ḍ, ṭ, ṣ/. These are largely confined to \ili{Arabic}-derived vocabulary, but they can also be found in inherited words of \ili{Indo-Aryan} stock, where they often represent original (\ili{Indo-Aryan}) retroflex sounds (cf. \textit{ḍōm} ‘Dom’, \textit{pēṭ} ‘belly’). An ongoing phonological innovation that is shared with \ili{Jerusalem} \ili{Arabic} is the \isi{simplification} of the affricate [{ʤ}] to the fricative [{ʒ}] in inherited lexemes, e.g. \textit{džami} ‘I go’ > \textit{žami}. This triggers a corresponding \isi{simplification} of [{ʧ}] to [{ʃ}], as in \textit{lači} ‘girl’ > \textit{laši}.



\subsection{Morphology}

Domari has not adopted productive word-\isi{derivational} templates from \ili{Arabic}. \ili{Arabic} \isi{inflectional} morphology, however, is productive with some \ili{Arabic}-derived word forms, resulting, in effect, in a compartmentalised morphological structure. \ili{Arabic}-derived plural nouns tend to retain \ili{Arabic} plural \isi{inflection}, but indigenous (inherited, \ili{Indo-Aryan}) plural inflections are added to the word: thus \textit{muslim} ‘Muslim’, plural \textit{musilmīn-e} Muslims-\textsc{pl} \textsc{‘}muslims’; \textit{madrase} ‘school’, dative plural \textit{madāris-an-ka}  (schools-\textsc{pl.obl-dat}) ‘to the schools’. While Jerusalem Domari retains inherited plural marking with nouns derived from both \ili{Indo-Aryan} and \ili{Arabic}, in the closely related variety of the nomadic Doms of Jordan the \ili{Arabic} plural ending \textit{-āt} is often used with inherited nouns: thus \textit{putur} ‘son’, Jerusalem Domari plural \textit{putr-e}, \ili{Jordanian Domari} plural \textit{putr-āt}.

\ili{Arabic} person \isi{agreement} \isi{inflection} is retained with \ili{Arabic}-derived \isi{modal} and aspectual auxiliaries. The auxiliaries \textit{kān} ‘be’, \textit{ṣār} ‘begin’, and \textit{baqa} ‘continue’ take \ili{Arabic} verbal \isi{inflection}, while \textit{bidd-} ‘want’, \textit{ḍall-} ‘continue’, and \textit{ḫallī-} ‘allow’ take \ili{Arabic} nominal-possessive marking:

\ea
\ea
\gll kān-at           par-ar-m-a        wāšī-s  \\
     be.\textsc{prf-3sg.f}   take-\textsc{3sg-1sg-pst}  with-\textsc{3sg}\\
\glt ‘She used to take me with her.’

\ex
\gll dōm-e kān-u kam-k-ad-a  {ḥ}addādīn-e\\
     dom-\textsc{pl} be-\textsc{.prf-3pl} work-\textsc{tr-3pl-pst} blacksmiths-\textsc{pl}\\
\glt ‘The Dom used to work as blacksmiths.’
\z
\z

\ea
\ea
\gll ṣār  qaft-ar-i  min  {ɔ}y-os\\
     begin.\textsc{prf.3sg.m} steal-\textsc{3sg-prog} from father-\textsc{3sg}\\
\glt ‘He started to steal from his father.’

\ex
\gll ṣār-u  kar-and-i  ḥafl-e\\
     begin.\textsc{prf-3pl} do-\textsc{3pl}-\textsc{prog}  party-\textsc{pl}\\
\glt ‘They started to have parties.’
\z
\z

\ea
\ea
\gll š-ird-i  ama-ke bidd-ha qumn-ar\\
     say-\textsc{pfv-f} \textsc{1sg-ben} want-\textsc{3sg.f} eat-\textsc{sbjv.3sg}\\
\glt ‘She said to me that she wants to eat.’

\ex
\gll bidd-i  par-am itžawwiz-om-is\\
     want-\textsc{1sg} take-\textsc{1sg.sbjv} marry-\textsc{1sg.sbjv-3sg.obl}\\
\glt ‘I want to take her and marry her.’
\z
\z

\ea
\ea
\gll ḫallī-hum naḍḍif-k-ad-i ehe marn-an  \\
     let.\textsc{imp.2sg-3pl} clean-\textsc{tr-3pl-prog} these.\textsc{pl} dead-\textsc{obl.pl}\\
\glt {{‘Let them clean up these corpses.’}} 

\ex
\gll ḫallī-h rʕi-k-ar hundar\\
     let.\textsc{imp.2sg}-\textsc{3sg} graze-\textsc{tr-3sg.sbjv} there\\
\glt ‘Let it graze there.’
\z
\z

Inflected \ili{Arabic}-derived auxiliaries include the \isi{existential} verb \textit{kān-} ‘to be’, which is used in Domari, as in \ili{Arabic}, as a past- and future-\isi{tense} \isi{copula}, supplementing the Domari remoteness or ‘external’ past-\isi{tense} marker \textit{-(y)a}, which follows the lexical predication or predicate object:

\ea \gll ihi illi par-d-om-is kān-at yatīm-ēy-a\\
     this.\textsc{f} \textsc{rel} take-\textsc{pst-1sg-3sg.obl} be.\textsc{prf-3sg.f} orphan-\textsc{pred.sg-pst}\\
\glt ‘The one [woman] whom I married [her] was an orphan.’ \label{ihi}
\z

\noindent \ili{Arabic}-derived auxiliaries are also inflected for \isi{tense} following \ili{Arabic} paradigms:

\ea \gll   lāzem tkūn itme mišaṭṭaṭ-hr-es-i\\
       must be.\textsc{impf.sbjv.3sg.f} \textsc{2pl} dispersed-\textsc{itr-2pl-prog}\\
\glt   ‘You must remain dispersed.’
\z

This amounts, in effect, to a functional \isi{compartmentalisation} in verbal morphology: both inherited and \ili{Arabic}-derived lexical verbs take inherited \ili{Indo-Aryan} \isi{inflection}, while \ili{Arabic}-derived \isi{modal} and aspectual auxiliaries take \ili{Arabic} \isi{inflection} (for further discussion see \citealt{Matras2015}).

\ili{Arabic} person \isi{inflection} is also found with the \ili{Arabic}-derived secondary pronominal object marker \textit{iyyā-}, complementiser \textit{inn-}, and conjunction \textit{liʔann-} ‘because’:

\ea \gll   ple illi t-or-im iyyā-hum\\
       money.\textsc{pl} \textsc{rel} give.\textsc{pst-2sg-1sg.obl} \textsc{obj-3pl}\\
\glt   ‘the money that you gave [it] to me’ \label{ple}
\z

\ea\label{most}
\gll aɣlabiyy-osan š-ad-i inn-hom min šamāl-os-ki hnūd-an-ki\\
     majority-\textsc{3pl} say-\textsc{3pl-prog} \textsc{comp-3pl} from        north-\textsc{3sg-abl} india-\textsc{obl.pl-abl}\\
\glt   ‘Most of them say that they are from northern India.’
\z

\ea \gll na kil-d-om barra liʔann-ha wars-ar-i\\
       \textsc{neg} exit-\textsc{pfv-1sg} out because-\textsc{3sg.f} rain-\textsc{3sg-prs}\\
\glt   ‘I did not go out because it was raining.’\label{barra}
\z

\ea \gll     payy-os liʔinn-o ṭāṭ-i kān\\
       husband-\textsc{3sg} because-\textsc{3sg.m} Arab-\textsc{pred.sg} be.\textsc{prf.3sg.m}\\
\glt   ‘Because her husband was an Arab.’
\z

Note that in example \REF{barra} the \isi{agreement} is in the feminine singular, corresponding to the grammatical mapping of the \ili{Jerusalem} \ili{Arabic} construction ‘it rains’ where the (underlying) subject is the feminine noun \textit{dunya} ‘the world’, while in \REF{ple}, resumptive pronoun \isi{agreement} with ‘money’, a plural noun, is in the plural.

Domari is seemingly an exception to the frequently cited generalisation that \isi{derivational} morphology is more likely to be borrowed than \isi{inflectional} morphology (cf. \citealt{Moravcsik1978}; \citealt{Field2002}; \citealt{Matras2009}: §6.2.2). In fact, the constraint on the borrowing of word-\isi{derivational} morphology results from the clash with the principle of the transparency of morphemes (cf. \citealt{Matras2009}: §6.2.2): \ili{Arabic} has few if any word-\isi{derivational} morphemes that can be isolated, relying instead on complex morphological templates into which lexical \isi{roots} are inserted. Nominal plural morphemes have both \isi{inflectional} function (relevant to other elements in the clause) and \isi{derivational} function (having independent meaning in standalone expressions). As shown above, they are replicated in Jerusalem Domari as an integral part of \ili{Arabic} plural word forms. On the other hand, the \isi{replication} of \isi{inflectional} material on auxiliaries is not productive, in that it is not incorporated into the general lexicon, not even with lexical words of \ili{Arabic} origin, but remains confined to the near-wholesale adoption of \isi{modal} and aspectual auxiliaries from \ili{Arabic}. In this respect, \ili{Arabic}-derived \isi{inflectional} paradigms in Domari constitute a case of both \textsc{fusion} as defined in \citet{Matras2009} – the wholesale non-separation of language systems around a particular functional category – and at the same time a case of functional compartmentalisaton as defined in \citet{Matras2015} – the distinct treatment of functional sub-components of a category, here the verbal category, in regard to grammatical \isi{inflection}. 

\subsection{Syntax
}

Generally, Jerusalem Domari shows full congruence with \ili{Palestinian} \ili{Arabic} in most syntactic functions. This includes \isi{word order} rules and the \isi{formation} of both simple and complex clauses. It also includes configurations such as mapping of tenses and \isi{modality} to complement and \isi{conditional} clauses, and the mapping of semantic relations onto case markers. The latter can be adpositional or \isi{inflectional}. For nominal possessive constructions, Domari has two options. The first of those options, illustrated in \REF{kuri}, is what we might call canonical Domari. It corresponds to the inherited \ili{Indo-Aryan} pattern. The second option, illustrated in \REF{boy}, corresponds to the common \ili{Palestinian} \ili{Arabic} construction, which is presented in \REF{bett}. Here Domari replicates the role of the \ili{Arabic} dative \isi{preposition} \textit{la} by means of the inherited Domari ablative/possessive \isi{inflectional} ending \textit{-ki}:

\ea
\ea
{{Canonical} Domari}\\
\gll bɔy-im kuri    \\
     father-\textsc{1sg} house\\ \label{kuri}
\ex
{Convergent Domari}\\
\gll kury-os bɔy-im-ki  \\
       house-\textsc{3sg} father-\textsc{1sg.obl-abl}\\ \label{boy}
\ex
{Arabic}\\
\gll bēt-o la-ʔabū-y  \\
       house-\textsc{3sg.m} to-father-\textsc{obl.1sg}\\
\glt   ‘my father’s house’ \label{bett}
\z
\z

The canonical position of adjectives in Domari is, as in other \ili{Indo-Aryan} languages, before the noun \REF{girla}, while in \ili{Arabic} adjectives follow the noun. However, speakers show an overwhelming preference for avoiding pre-posed adjectives and instead make use of the non-verbal predication marker in order to allow the adjective to follow the noun \REF{girlb}, thereby replicating \ili{Arabic} \isi{word order} patterns \REF{girlc}:

\ea\label{girls}
\ea
{\ili{Canonical} Domari}\\
\gll er-i qišṭoṭ-i šōni  \\
       come.\textsc{pfv-f} little-\textsc{f} girl\\
\glt   ‘A little girl arrived.’ \label{girla}



\ex
{Convergent Domari}\\
\gll   er-i šōni qišṭoṭ-ik  \\
       come.\textsc{pfv-f} girl little-\textsc{pred.sg.f}\\
\glt   ‘A little girl arrived.’ [= ‘A girl arrived, being little.’] \label{girlb}

\ex
{Arabic}\\
\gll ʔižat bint zɣīre  \\
       come.\textsc{prf.3sg.f} girl little.\textsc{f}\\
\glt   ‘A little girl arrived.’ \label{girlc}
\z
\z

The emergence of nominal clauses, facilitated by the availability of non-verbal predication markers, might be regarded as an innovation for an \ili{Indo-Iranian} language, which reinforces sentence-level \isi{convergence} between \ili{Arabic} and Domari:

\ea
\ea
{Domari}\\
\gll wuda bizzot-ēk  \\
     old.\textsc{m} poor-\textsc{pred.sg}\\
     
\ex {Arabic}\\
\gll l-ḫityār miskīn  \\
     \textsc{def}{}-old.man poor\\
\glt ‘The old man is poor.’
\z
\z

Domari, like \ili{Arabic}, shows a strong tendency toward SVO \isi{word order} in categorical sentences in which a thematic perspective is established by linking to a known topical entity:

\ea \gll mām-om putur yāsir gar-a swēq-ē-ta\\
         uncle-\textsc{1sg} son Yassir go.\textsc{pfv-m} market-\textsc{obl.f-dat}\\
\glt     ‘My (paternal) cousin Yassir went to the market.’
\z

By contrast, as seen in example \REF{girls}, Domari shows consistent \isi{convergence} with \ili{Arabic} in regard to the position of the subject after the verb when new topical entities are introduced, especially with verbs that convey movement and change of state and in presentative constructions. Drawing on inherited morphology, this \isi{convergence} in \isi{word order} patterns also allows for the encoding of the pronominal experiencer--recipient through a person affix that is attached to an intransitive verb in presentative constructions, matching the \ili{Arabic} construction:



\ea
\ea
{Domari}\\
\gll er-os-im ḫabar    \\
     come.\textsc{pfv-3sg-1sg.obl}  notice\\
\ex
{Arabic}\\
\gll  ʔažā-ni  ḫabar  \\
     come.\textsc{prf.3sg.m-1sg} notice\\
\glt ‘I received notification’
\z
\z

Complex clauses are also congruent with \ili{Arabic}. Like \ili{Arabic}, Domari shows three distinct co-temporal adverbial constructions. In the first, the subordinate clause is introduced by the conjunction ‘and’ and the verb is finite and indicative:

\ea
\ea
{Domari}\\
\gll kahind-ad-i ū pandži našy-ar-i  \\
     look-\textsc{3pl-prog} and \textsc{3sg} dance-\textsc{3sg-prog}\\
     
\ex {Arabic}\\
\gll b-yitfarražu w hiyye b-turʔuṣ  \\
     \textsc{ind}-look.\textsc{impf.3pl} and \textsc{3sg.f} \textsc{ind}-dance.\textsc{impf.3sg.f}\\
\glt ‘They watch her dance.’
\z
\z

In the second, the subordinated predicate appears in the present \isi{participle}:

\ea
\ea
{Domari}\\
\gll lah-erd-om-is mindir-d-ēk    \\
       see-\textsc{pfv-1sg-3sg.obl} stand-\textsc{pfv-pred.sg.m}\\
 
\ex
{Arabic}\\
\gll šuft-o   wāʔef    \\
     see.\textsc{prf.1sg-3sg.m} standing\\
\glt ‘I saw him standing.’
\z
\z

The final option shows a nominalised verb, whose possessive \isi{inflection} indicates the subject/agent, introduced by the \isi{preposition} ‘with’ in the subordinate position alongside a finite main clause:

\ea
\ea
{Domari}\\
\gll maʕ šuš-im-ki tiknaw-ar-m-i gurg-om  \\
       with sleep-\textsc{1sg.obl-abl} hurt-\textsc{3sg-1sg-prog} neck-1\textsc{sg}\\



\ex
{Arabic}\\
\gll maʕ nōmt-i b-tūžaʕ-ni raʔbt-i  \\
       with sleep-\textsc{obl.1sg} \textsc{ind-3sg.f}-hurt.\textsc{impf.3sg.f-1sg} neck-\textsc{obl.1sg}\\
\glt       ‘As I sleep, my neck hurts.’
\z
\z

R\isi{elative} clauses follow the format of \ili{Arabic} \isi{relative} clauses: they employ the \ili{Arabic}-derived post-nominal relativiser \textit{illi} and show the same distribution rules for pronominal resumption as in \ili{Arabic}:

\ea \gll   ihi illi par-d-om-is kān-at yatīm-ēy-a\\
       this.\textsc{f} \textsc{rel} take-\textsc{pst-1sg-3sg.obl} be.\textsc{prf-3sg.f} orphan-\textsc{pred.sg-pst}\\
\glt   ‘The one [woman] whom I married [her] was an orphan.’
\z

Factual (indicative) complements are introduced by the \ili{Arabic}-derived complementiser \textit{inn-}, which carries \ili{Arabic}-derived \isi{inflection} (as in example \ref{most} above), and show comparable clause structure as in \ili{Arabic}:

\ea
\ea
{Domari}\\
\gll džan-ad-i in-na dōm    \\
     know-\textsc{3pl-prog} \textsc{comp}-\textsc{1pl} Dom\\
\ex
{Arabic}\\
\gll b-yiʕrafu in-na dōm  \\
     \textsc{ind}-know.\textsc{impf.3pl} \textsc{comp}-\textsc{1pl} Dom\\
\glt ‘They know that we are Dom.’
\z
\z

Modal complements and same-subject purpose clauses show, as in \ili{Arabic}, a subjunctive complement, without a complementiser:

\ea
\ea
{Domari}\\
\gll bidd-i  dža-m  ḥaram-ka ṣalli-k-am  \\
       want-\textsc{1sg} go-\textsc{1sg.sbjv} mosque-\textsc{dat} pray-\textsc{tr-1sg.sbjv}\\
\ex
{Arabic}\\
\gll bidd-i  arūḥ ʕa-l-ḥaram aṣalli  \\
       want-\textsc{1sg} go.\textsc{impf.sbjv.1sg} to-\textsc{def-}mosque pray.\textsc{impf.sbjv.1sg} \\
\glt    ‘I want to go to the mosque to pray.’
\z
\z

Adverbial clauses employ \ili{Arabic}-derived adverbial subordinators, including \textit{lamma} ‘when’, as in \REF{lamma}, or composite conjunctions consisting of a \isi{preposition} and complementiser, such as \textit{baʕd} \textit{mā} ‘after’ and \textit{qabil} \textit{mā} ‘before’, as in \REF{after} and \REF{before}, and generally follow \ili{Arabic} sentence organisation and \isi{tense} and \isi{modality} distribution patterns.

\ea\label{lamma} \gll lamma lak-ed-a ḫāl-os inǧann-ahr-a bɔy-om\\
       when see-\textsc{pfv-m} uncle-\textsc{3sg} crazy-\textsc{tr.pfv-m} father-\textsc{1sg}\\
\glt     ‘When he saw his uncle, my father went crazy.’
\z

\ea\label{after} \gll baʕd mā ḫaḷḷaṣ-k-ed-a kam-os gar-a kury-is-ta\\
     after \textsc{comp} finish-\textsc{tr-pfv-m} work-\textsc{3sg} go.\textsc{pfv-m} house-\textsc{3sg.obl-dat}\\
\glt     ‘After he finished his work he went home.’
\z

\ea\label{before} \gll qabil mā dža-m ḫaḷḷaṣ-k-ed-om kam-as\\
       before  \textsc{comp} go-\textsc{1sg.sbjv} finish-\textsc{tr-pfv-1sg} work-\textsc{obl.m}\\
\glt     ‘Before I left I finished my work.’
\z

Conditional clauses similarly draw on the \ili{Arabic} conjunctions \textit{iza} and \textit{law}, both ‘if’, and show similar distribution of \isi{tense} and aspect categories, including the \ili{Arabic}-derived impersonal marker of counter-factuality \textit{kān,} literally ‘was’:

\ea
\ea
{Domari}\\
\gll law er-om ḫužoti kān lah-erd-om-s-a \\
       if come.\textsc{pfv-1sg} yesterday was see-\textsc{pfv-1sg-3sg-pst}\\
\ex
{Arabic}\\
\gll law žīt mbāreḥ kān šuft-o  \\
       if come.\textsc{prf.1sg} yesterday be.\textsc{3sg.m} see.\textsc{prf.1sg-3sg.m}\\
\glt   ‘If I had come yesterday, I would have seen him.’
\z
\z

\subsection{Lexicon
}

Jerusalem Domari shows extensive impact of \ili{Arabic} on the grammatical lexicon, including almost wholesale reliance on \ili{Arabic}-derived material for entire categories. In the pronominal domain, Domari employs, in additional to the secondary pronominal object marker \textit{iyyā-} discussed above, also the \ili{Arabic} reflexive pronoun \textit{ḥāl-}, derived from the word ‘state’, combined with person/possessive \isi{inflection}, and the \ili{Arabic} \isi{reciprocal} pronoun \textit{baʕḍ-}:

\ea \gll naḍḍif-k-ad-a ḥāl-os\\
     clean-\textsc{tr-pfv-m} \textsc{refl-3sg}\\
\glt ‘He cleaned himself.’
\z

\ea \gll tʕarraf-h-r-ēn baʕḍ-ē-man-ta\\
     meet-\textsc{tr.pfv-1pl} \textsc{recp-pl-1pl-dat}\\
\glt ‘We met one another.’
\z

In\isi{definite} expressions draw on \ili{Arabic}-derived forms of category determination including negative \textit{wala}, free choice \textit{ayy} and universal \textit{kull}, which may be combined with inherited ontological markers, as well as on the ontological specifiers \textit{ḥāǧ- \textup{for thing and} maḥall \textup{for location. Indefinite expressions that derive entirely from \ili{Arabic} include temporal} wala marra} {{‘never’,} \textit{dāyman} \textup{‘always’, and universal-thing} \textit{kullši} \textup{‘everything’. \ili{Arabic}-derived} }focus particles are \textit{barḍo} ‘also, too’ and \textit{ḥatta} ‘even’ and \isi{quantifiers} are \textit{kull} ‘every, each’ and \textit{akamm} ‘a few’. Interrogatives are generally inherited (\ili{Indo-Aryan}), with the exception of \textit{qaddēš} ‘how much’. Numerals are all derived from \ili{Arabic} with the exception of the lowest numeral forms (`one' to `five' in citation function and `one' to `three' in attributive role) (see Tables \ref{numerals1}–\ref{numerals2}); all ordinal \isi{numerals} (\textit{awwal} ‘first’, \textit{tāni} `second' etc.) are from \ili{Arabic}.
\\

\begin{table}[]
\begin{tabularx}{.8\textwidth}{QQl}
\lsptoprule Numeral & Citation & Attribute\\
\midrule
1 & \textit{ikak} & \textit{-ak}\\
2 & \textit{diyyes} & \textit{di}\\
3 & \textit{taranes}  & \textit{taran}\\
4 & \textit{štares}  & \textit{ʔarbaʕ}\\
5 & \textit{pʌndžes}  & \textit{ḫamis}\\
6 & \textit{sitt-ēk-i} & \textit{sitt}\\
7 & \textit{sabʕ-ak-i} & \textit{sabaʕ}\\
8 & \textit{tamāni-ak-i} & \textit{tamānye}\\
9 & \textit{tisʕ-ak-i} & \textit{tisʕa}\\
10 & \textit{das} ‘ten’, \textit{ʕašr-ak-i} & \textit{ʕašr}\\
20 & \textit{ʕišrīn-i}, \textit{wīs-i} & \textit{ʕišrīn}\\
21 & \textit{ʕišrīn ū ekak-i} & \textit{wāḥed w ʕišrīn}\\
22 & \textit{ʕišrīn-i ū diyyes-i} & \textit{tnēn w ʕišrīn}\\
23 & \textit{ʕišrīn-i ū taranes-i} & \textit{talāte w ʕišrīn}\\
24 & \textit{ʔarbaʕ ū ʕišrīn} & \textit{ʔarbaʕ w ʕišrīn}\\
100 & \textit{miyyēk hi, siyy-ak}-\textit{i} & \textit{miyye}\\
1000 & \textit{alf-ak-i} & \textit{alf}\\
\lspbottomrule
\end{tabularx}
\caption{Jerusalem Domari numerals}
\label{numerals1}
\end{table}

\begin{table}[]
\begin{tabularx}{.5\textwidth}{Ql}
\lsptoprule Numeral & Form\\
\midrule
30 & \textit{talātīn}\\
40 & \textit{ʔarbaʕīn}\\
50 & \textit{ḫamsīn}\\
60 & \textit{sittīn}\\
70 & \textit{sabʕīn}\\
80 & \textit{tamanīn}\\
90 & \textit{tisʕīn}\\
\lspbottomrule
\end{tabularx}
\caption{Jerusalem Domari higher numerals}
\label{numerals2}
\end{table}

Alongside a very small number of inherited \isi{prepositions} that are used exclusively with pronominal (person-inflected) forms, most \isi{prepositions} are derived from \ili{Arabic} (Table \ref{preps}).

\begin{table}[]
\begin{tabularx}{\textwidth}{QQQQQl}
\lsptoprule

\textit{ʕan} & ‘on, about’ & \textit{ʕašān} & ‘because’ & \textit{nawāḥi} & ‘toward’\\
\textit{maʕ} & ‘with’ & \textit{minšān} & ‘for’ & \textit{qabil} & ‘before’\\
\textit{min} & ‘from’ & \textit{min ɣēr} & ‘without’ & \textit{baʕd} & ‘after’\\
\textit{la, ʕala} & ‘to’ & \textit{min/bi dūn}  & ‘without’ & \textit{laɣāyet} & ‘until’\\
\textit{fi} & ‘in’ & \textit{bēn} & ‘between’ & \textit{bi} & ‘in, for’\\
\textit{zayy} & ‘like’ & \textit{ḥawāli} & ‘around’ & \textit{ḍiḍḍ} & ‘against’\\
\textit{ʕind (ʕand)} & ‘at’ & \textit{min ḍamn} & ‘among’ & \textit{žamb} & ‘next to’\\ \textit{badāl} & ‘instead of’ &  \textit{ʔilla ɣēr} & ‘except for’ & & \\
\lspbottomrule
\end{tabularx}
  \caption{Arabic-derived prepositions in Jerusalem Domari}
  \label{preps}
  \end{table}

\ili{Arabic}-derived grammatical operators at verbal clause level include a series of \isi{modality} adverbs such as \textit{masalan} ‘for example’, \textit{yimken} ‘perhaps’, \textit{atāri} ‘well’, time adverbs such as \textit{hessaʕ} ‘now’ and \textit{baʕdēn} ‘then, afterwards’, and the phasal adverbs \textit{lissa} and \textit{lāyzāl}, both ‘still’. As discussed above, Domari adopts \ili{Arabic} \isi{modal} and aspectual auxiliaries wholesale, i.e. along with their \ili{Arabic}-derived \isi{inflection}. This covers almost the full category of \isi{modal} and aspectual auxiliaries including habitual/iterative \textit{kān} ‘be’, \textit{ṣār} ‘begin’, and \textit{baqa} ‘continue’, \textit{bidd-} ‘want’, \textit{ḍall-} ‘continue’, and \textit{ḫallī-} ‘let’, as well as the impersonal form \textit{lāzem} ‘must’. The only \isi{modal} for which an \ili{Indo-Aryan} form is retained is \textit{sak-} ‘to be able to’. Past-\isi{tense} finite predications take the \ili{Arabic} negator \textit{mā} (alongside inherited \textit{na}) while in non-finite predications the \ili{Arabic} \isi{negation} particle \textit{miš} is used:

\ea \gll mā lak-ed-om-is\\
         \textsc{neg} see-\textsc{pfv-1sg-3sg.obl}\\
\glt     ‘I didn’t see him/her.’
\z

\ea \gll bay-os miš kury-a-m-ēk\\
         wife-\textsc{3sg} \textsc{neg} house-\textsc{obl.f-loc-pred.sg}\\
\glt     ‘His wife is not at home.’
\z

Clause combining relies exclusively on \ili{Arabic}-derived material (connectors and conjunctions) (see Table \ref{conjs}).

\begin{table}[]
\begin{tabularx}{\textwidth}{QQQl}
\lsptoprule

{{\textit{w}}} & ‘and’ & {{\textit{qabil mā}}} & ‘before’\\
{{\textit{wala}}} & ‘and not’, ‘(n)either’ & {{\textit{baʕd mā}}} & ‘after’\\
{{\textit{yā}}} & ‘or’ & {{\textit{min-yōm-mā}}} & ‘since’\\
{{\textit{willa}}} & ‘or (else)’ & {{\textit{iza}}} & ‘if’\\
{{\textit{bass}}} & ‘but’, ‘only’ & {{\textit{law}}} & ‘if’\\
{{\textit{illi}}} & \isi{relative} pronoun & {{\textit{bi-r-raɣim}}} & ‘despite’\\
{{\textit{inn-}}} & ‘that’ & {{\textit{ʕašān}}} & ‘for’, ‘in order to’\\
{{\textit{liʔann}}} & ‘because’ & {{\textit{minšān}}} & ‘for’, ‘in order to’\\
{{\textit{lamma}}} & ‘when’ & {{\textit{ta}}} & ‘in order to’\\
{{\textit{kull mā}}} & ‘whenever’ &  & \\
\lspbottomrule
\end{tabularx}
  \caption{Arabic-derived conjunctions in Jerusalem Domari}
  \label{conjs}
  \end{table}

Likewise, the inventory of discourse particles and interjections is adopted in its entirety from \ili{Arabic}: We find the interjection, tags and filers \textit{yabayyi,} \textit{yaḷḷa}, \textit{xaḷaṣ, waḷḷa,} {{and}} \textit{yaʕni}, as well as segmental markers with a lexical meaning such as \textit{l\nobreakdash-muhimm} ‘anyway’, \textit{l-ḥāṣil} ‘finally’, \textit{ṭayyib} ‘well’, \textit{w ʔiši} ‘and the like’, \textit{w hāda} ‘and so on’, \textit{abṣar} ‘whatever’, and the filler \textit{hāy} ‘that’. The quotation particle \textit{qal/ḫal}, from \ili{Arabic} ‘say’, is not found in \ili{Jerusalem} \ili{Arabic} and appears to represent an older layer of \ili{Arabic} influence (as indicated also by its phonological structure; see §\ref{sec2}). 

The content lexicon equally shows massive impact of \ili{Arabic}. In the Jerusalem Domari corpus of narrational and conversational talk as well as sentence elicitation recorded in the 1990s \citep{Matras2012}, almost two thirds of lexical items are \ili{Arabic}-derived; the count includes single-word insertions from \ili{Arabic}, including attributive nominal \isi{compounds} (noun–possessor and noun–adjective), but excludes phrases containing a finite lexical verb that is \ili{Arabic}-derived (which latter are regarded as optional code-switches). Both \ili{Arabic}-derived nouns and adverbs outnumber inherited (\ili{Indo-Aryan}) counterparts by around 65\% to 35\%, while for verbs and adjectives the numbers are roughly equal. Around 26\% of items of both the Swadesh 100-item list and the Leipzig–Jakarta 100-item list (\citealt{HaspelmathTadmor2009}) are \ili{Arabic}-derived. This puts Domari in the range of languages considered to be ``high borrowers'' by the Leipzig Loanword Typology Project (\citealt{HaspelmathTadmor2009}). Meanings on the list that are replaced by \ili{Arabic} loans in Domari include a number of animals (‘ant’, ‘bird’, ‘fish’), activities (‘to run’, ‘to fly’, ‘to crush’), elements of nature (‘star’, ‘soil’, ‘shade’, ‘ash’, ‘leaf’, ‘\isi{root}’), and some body parts (‘knee’, ‘navel’, ‘liver’, ‘thigh’; also ‘wing’, ‘tail’). On the whole, the meaning and usage of \ili{Arabic}-derived lexemes matches that of \ili{Jerusalem} \ili{Arabic}. Creative processes are marginal and include such processes as the phonological volatility of /q/ (as [q], [x], [qx] and [ɡ]), the alternation between \textit{farǧik-} ‘to show’ (\ili{Arabic} \textit{√frǧ}) and \textit{warǧik-}, and the occasional creative \isi{derivation} such as \textit{bisawahr-} ‘to get married’, from \ili{Arabic} \textit{bi-sawa} ‘together’. 

\ili{Arabic} verbs are integrated into Domari through a \isi{light verb} construction that draws on the inherited verb stems \textit{-k-} ‘to do’ and \textit{-h-} ‘to become’, which are grammaticalised into loan-verb adaptation markers (see \citealt{Matras2012}: 240–244) that are sensitive to valency. This follows a strategy for the adaptation of loan verbs that is widespread across a geographical area stretching from the Balkans and the Caucasus through \isi{Anatolia} and Western Asia and on to the Indian Subcontinent. For some verbs, alternating adaptation markers can indicate change in valency: \textit{ǧawwiz-h-r-i} (marry-\textsc{itr-pfv-f}) ‘she got married’, \textit{ǧawwiz-k-am-is} (marry-\textsc{tr-1sg.sbjv-3sg.obl}) ‘I shall marry her off’. The core of integrated \ili{Arabic} verbs generally derives from the \ili{Arabic} subjunctive–imperative form, which in \ili{Arabic} never occurs in isolation from its person \isi{inflection} in the prefix conjugation, as in \textit{ǧawwiz-} ‘marry’, from *ǧawwiz ‘marry (off)!’ or *tǧawwiz ‘get married!’. Note, however, that the vowel structure of the core does not always correspond to the subjunctive–imperative form of contemporary \ili{Palestinian} \ili{Arabic}, which is quite possibly a further indication of the layered historical influence of \ili{Arabic}. Thus we find \textit{s’il-k-ed-om} (ask-\textsc{tr-pfv-1sg}) ‘I asked’, from *s’il- ‘ask’, while \ili{Palestinian} \ili{Arabic} has \textit{isʔal} ‘ask!’, and \textit{rawwaḥ-ah-r-a} (go-\textsc{itr-pfv-m}) ‘he travelled’, while \ili{Palestinian} \ili{Arabic} has \textit{rawwiḥ} ‘go away!’.



\subsection{Cross-category interplay
}

A typologically curious case of contact-induced change is offered by the use in Jerusalem Domari of three construction types that cut across structural categories. The first pertains to the \isi{comparative} form of adjectives. In the absence of a structurally transparent, isolated and replicable marker of adjective comparison (\isi{comparative} and \isi{superlative}), Domari draws on \ili{Arabic} word forms for all \isi{comparative} adjective forms, even when an inherited (non-\ili{Arabic}) word form is used for the positive form of the adjective, as illustrated in \REF{comp} (cf. Herin, this volume: §3.2)\ia{Herin, Bruno@Herin, Bruno}.

\ea\label{comp}
\ea
{Domari}\\
\gll atu qaštot-ik    \\
           you.\textsc{sg} small-\textsc{pred.sg.f}\\
\glt       ‘You are small.’

\ex
{Domari}\\
\gll atu azɣar mēšī-m-i    \\
           you.\textsc{sg} smaller from-\textsc{1sg-pred.sg}\\
\glt       ‘You are smaller than I.’

\ex
{Arabic}\\
\gll inti zɣīre    \\
           \textsc{2sg.f} small.\textsc{f}\\
\glt       ‘You are small.’

\ex
{Arabic}\\
\gll inti azɣar minn-i    \\
           \textsc{2sg.f} smaller from-\textsc{obl.1sg}\\
\glt       ‘You are smaller than I.’
\z
\z


This \isi{formation} involves essentially the recruitment of an alternative, \ili{Arabic}-derived item from the category of lexical items in order to carry out a grammatical procedure that is \isi{derivational}–\isi{inflectional} by nature (\isi{derivational} in that it modifies meaning, \isi{inflectional} in that it is inherently embedded into a syntactic relationship at the phrase level); thus we have a case of cross-category interplay.

A further case is that of lexical suppletion around \ili{Arabic}-derived \isi{numerals}. Domari and \ili{Arabic} differ typologically in respect of numeral \isi{agreement}: with \ili{Indo-Aryan} \isi{numerals}, the Domari noun appears in the default singular form, while in \ili{Arabic}, \isi{numerals} up to `ten' take plural \isi{agreement}. The clash is resolved in Domari in such a way that \ili{Arabic}-derived \isi{numerals} under `ten' invariably trigger an \ili{Arabic}-derived lexical item even when an inherited form of the corresponding lexeme is available:

\ea
\ea
\gll ḥkum-ke-d-os taran  wars maḥkame\\
           sentence-\textsc{tr-pfv-3sg} three year court\\
\glt       ‘The court sentenced him to three years.’

\ex
\gll eh-r-a  ʕumr-om sitte snīn\\
           become-\textsc{pfv-m} age-\textsc{1sg} six year\textsc{.pl}\\
\glt       ‘I turned six years old.’
\z
\z

Such alternation is systematic (see further examples in Table \ref{inh}) and might be regarded as a case of \isi{bilingual suppletion}, where every countable noun in the language for which an inherited (\ili{Indo-Aryan}) word form exists also has an \ili{Arabic}-derived counterpart that is used with \isi{numerals} between `three' and `ten'.
\\

\begin{table}[]
\begin{tabularx}{\textwidth}{Ql}
\lsptoprule

Inherited numeral and singular noun & \ili{Arabic} numeral and plural noun\\
\midrule
\textit{di dīs} \textit{taran dīs} ‘two days three days’ & \textit{sabaʕ-t-iyyām} ‘seven days’\\
\textit{taran mas} ‘three months’ & \textit{ḫamas-t-ušhur} ‘five months’\\
\textit{taran wars} ‘three years’ & \textit{sitte snīn} ‘six years’\\
\textit{taran zard} ‘three pounds’ & \textit{ḫamas līrāt} ‘five pounds’\\
 \lspbottomrule
\end{tabularx}
  \caption{Some phrases from the corpus containing numerals   and nouns}
  \label{inh}
  \end{table}

Finally, while Domari lacks a \isi{definite} \isi{article}, the \ili{Arabic} \isi{definite} \isi{article} \textit{l-} is employed with \isi{definite} noun phrases where both the noun and the numeral-attribute are derived from \ili{Arabic}:

\ea
\gll mar-d-e l-ʔarbaʕ ḫurfān\\
         kill-\textsc{pfv-3pl} \textsc{def}-four lamb\textsc{.pl}\\
\glt     ‘They slaughtered the four lambs.’
\z

\ea\gll dīr-os it-tānye eh-r-i muhandis-ēk\\
         daughter-\textsc{3sg} \textsc{def}-second.\textsc{f} become-\textsc{pfv-f}        engineer-\textsc{pred.sg.f}\\
\glt     ‘Her other daughter became an engineer.’
\z

\section{Conclusion
}

The comparison with Macalister's (\citeyear{Macalister1914}) materials offers some scope for observations in respect of the historical development of contact-induced change over the past century in at least two areas of structure, namely the loss of \ili{Turkish}-derived vocabulary as well as of some of the inherited \ili{Indo-Aryan} vocabulary (around 55 words are attested in Macalister’s materials that were not familiar to the speakers I interviewed), and the adoption of fully-inflected \isi{modal} and aspectual auxiliaries, compared to their use as impersonal forms in Macalister’s material. One has to bear in mind, however, that Macalister’s corpus is based on work with just a single speaker. Nevertheless, these changes provide some indication that the impact of \ili{Arabic} continued to expand during the last century in which the language was spoken, a period during which the Doms lost much of their distinct culture and lifestyle as a result of the shift from a semi-nomadic service economy to a settled, wage-based but still socially isolated and stigmatised community. 

The impact of \ili{Arabic} on Domari prompts a theoretical challenge around identifying a form of the language that is structurally inseparable from \ili{Arabic}. This can be illustrated by the following two examples:

\ea \label{aktar1}
\ea
{Domari}\\
\gll aktar min talātīn ḫamsa w talātīn sana mā lak-ed-om-is\\
     more from thirty five and thirty year \textsc{neg} see-\textsc{pfv-1sg-3sg.obl}\\ \label{aktar}
\ex
{Arabic}\\
\gll aktar min talātīn ḫamsa w talatīn sana mā šuft-ha\\
     more from thirty five and thirty year \textsc{neg} see.\textsc{prf.1sg-3sg.f}\\ \label{aktar2}
\glt ‘I haven’t seen her for more than thirty, thirty five years.’
\z
\z

\ea \label{umr1}
\ea
{Domari}\\
\gll kān ʕumr-om yimken sitte snīn sabʕa snīn\\
     be\textsc{.prf.3sg.m} age-\textsc{1sg} maybe six years seven years\\ \label{umr}
\ex
{Arabic}\\
\gll kān ʕumr-i yimken sitte snīn sabʕa snīn\\
     be\textsc{.prf.3sg.m} age-\textsc{obl.1sg} maybe six years seven years\\ \label{umr2}
\glt ‘I was maybe six or seven years old.’
\z
\z

Both \REF{aktar} and \REF{umr} are unambiguously identifiable to speakers as Domari utterances; moreover, their meaning cannot be conveyed in Domari in any other way. Yet they each differ in just one single element from their respective counterpart \ili{Arabic} utterances in \REF{aktar2} and \REF{umr2}: the use of the lexical verb with subject and object \isi{agreement} (Domari \textit{lak-ed-om-is} ‘I saw her’, \ili{Arabic} \textit{šuf-t-ha}) in the first, and the use of the \textsc{1sg} possessive marker (Domari \-\textit{{}-om}, \ili{Arabic} \textit{-i\textup{) with the word} ʕumr \textup{‘age’ in the second. Despite being isolated examples, \REF{aktar1}–\REF{umr1} illustrate the considerable extent of structural overlap between the two languages. Furthermore, the examples discussed above of \isi{bilingual suppletion} in number \isi{agreement} and adjective comparison, and the productive use of \ili{Arabic} person \isi{agreement} \isi{inflection} with auxiliaries and with some complementisers and secondary object markers, mean in effect that active command of \ili{Arabic} is a pre-requisite for speaking Domari.}}

It follows that Domari provides us with an opportunity to reconsider the taxonomy of contact-induced \isi{language change} phenomena. It is not a Mixed Language by conventional definitions (cf. \citealt{BakkerMatras2013}; \citealt{Matras2009}: chapter 10) since the \ili{Indo-Aryan} source of grammatical \isi{inflection} in all word classes is overwhelmingly consistent with the source of basic lexical vocabulary and of deictic and anaphoric elements (demonstrative and personal pronouns, interrogatives, and spatial adverbs). Impressionistically speaking, it is a language with ``heavy borrowing'' in that it shows the adoption of \ili{Arabic}-derived material in a wide range of different structural categories. But the distribution of some of this material, taking into account the ubiquitous active \isi{bilingualism} among Domari speakers,  lends itself to the postulation of several particular types of contact-induced structural change, which I have labeled above \isi{fusion} (wholesale non-separation of languages around a particular structural category, e.g. clause connectors and \isi{modal} auxiliaries), \isi{inflectional} compartmentalisaton (the use of \ili{Arabic} \isi{inflectional} paradigms with particular functional categories, notably \isi{modal} and aspectual auxiliaries), and \isi{bilingual suppletion} (activation of speakers’ full command of \ili{Arabic} vocabulary and \isi{inflection} for creative formations around number \isi{agreement} and adjective comparison).

\section*{Further Reading}

\citet{Matras2007Domari} outlines contact influences on Jerusalem Domari in the context of a collection of chapters on contact-induced change in a sample of different languages.\\ \citet{Matras2012} provides a descriptive and historical overview of Jerusalem Domari and includes extensive discussion of contact-induced change in the individual chapters as well as a chapter devoted to the impact of \ili{Arabic}.\\ \citet{Matras2009} is a general theoretical discussion of contact-induced change in functional-typological perspective and includes many examples from Jerusalem Domari.\\ Finally, \citet{Matras2015} discusses patterns of morphological borrowing and their theoretical implications and gives as one of the examples the compartmentalisaton of \isi{modal} and aspectual auxiliaries in Jerusalem Domari. 

\section*{Abbreviations}
\begin{multicols}{2}
\begin{tabbing}
\textsc{ipfv} \hspace{1em} \= before common era\kill
\textsc{1, 2, 3} \> 1st, 2nd, 3rd person \\
\textsc{abl} \> ablative \\
\textsc{ben} \> benefactive \\
\textsc{comp} \> complementiser \\
\textsc{dat} \> dative \\
\textsc{f} \> feminine \\
\textsc{imp} \> imperative \\
\textsc{impf} \> imperfect (prefix conjugation) \\
\textsc{ind} \> indicative \\
\textsc{itr} \> intransitive \\
\textsc{loc} \> {locative} \\
\textsc{m} \> masculine \\
\textsc{obl} \> oblique \\
\textsc{pfv} \> perfective \\
\textsc{pred} \> predication (non-verbal) \\
\textsc{prs} \> present \\
\textsc{prf} \> perfect (suffix conjugation) \\
\textsc{prog} \> progressive \\
\textsc{pst} \> past \\
\textsc{recp} \> {reciprocal} \\
\textsc{refl} \> reflexive \\
\textsc{rel} \> relativiser \\
\textsc{sbjv} \> subjunctive \\
\textsc{sg} \> singular \\
\textsc{tr} \> transitive \\
\end{tabbing}
\end{multicols}


\sloppy
\printbibliography[heading=subbibliography,notkeyword=this] 
\end{document}