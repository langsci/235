\documentclass[output=paper]{langsci/langscibook} 
\author{Ahmad Al-Jallad\affiliation{The Ohio State University}}
\title{Pre-Islamic Arabic}
% \keywords{} 
\abstract{This chapter provides an overview of Arabic in contact in the pre-Islamic period, from the early first millennium BCE to the rise of Islam. Contact languages include Akkadian, Aramaic, Ancient South Arabian, Canaanite, Dadanitic, and Greek. The chapter concludes with two case studies on contact-induced development: the emergence of the definite article and the realization of the feminine ending. 
}
\maketitle

\begin{document}
  
\section{Language contact in the pre-Islamic Period}

\begin{quote}
[I]the Djāhiliyya, ‘the Age of Ignorance’ […], the Arabs lived to a great extent in almost complete isolation from the outer world… [t]his accounts for the prima facie astonishing fact that Arabic, though appearing on the stage of history hundreds of years after the Canaanites and Aramaeans, nevertheless in many respects has a more archaic character than these old Semitic languages. The Arabs, being almost completely isolated from outer influences and living under the same primitive conditions of their ancestors preserved the archaic structure of their language. (\citealt{Blau1981}: 18).
\end{quote}
This is the image of Arabic’s pre-Islamic past that emerges from Classical Arabic sources. For writers such as Ibn Khaldūn,\ia{Ibn Khaldūn@Ibn Khaldūn} contact-induced change in Arabic was a by-product of the Arab conquests, and served to explain the differences between the colloquial(s) of his time and the literary language. More than a century and a half of epigraphic and archaeological research in Arabia and adjacent areas has rendered this view of Arabic’s past untenable. Arabic first appears in the epigraphic record in the early first millennium BCE, and for most of its pre-Islamic history, the language interacted in diverse ways with a number of related Semitic languages and Greek. This chapter will outline the various foci of contact between Arabic and other languages in the pre-Islamic period based on documentary evidence. Following this, I offer two short case studies showing how contact-induced change in the pre-Islamic period may explain some of the key features of Arabic today.   

\subsection{Old Arabic}
Old Arabic is an umbrella term for the diverse forms of the language attested in documentary and literary sources from the pre-Islamic period, including inscriptions, papyri, and transcriptions in Greek, Latin, and cuneiform texts. The present usage does not refer to Classical Arabic or the linguistic material attributed to the pre-Islamic period collected in the eighth and ninth centuries CE, such as poetry and proverbs, as we cannot be sure about their authenticity, especially with regard to their linguistic features. \citet{Al-Jallad2017early} defines the corpus of Old Arabic as follows: Safaitic, an Ancient North Arabian script concentrated in the Syro-Jordanian Ḥarrah (end of the 1st millennium BCE to 4th c. CE), Hismaic, an Ancient North Arabian script spanning from central Jordan to northwest Arabia (chronology unclear, but overlapping with Nabataean), the substratum of Nabataean Aramaic, along with a few Arabic-language texts carved in this script (2nd c. BCE to 4th c. CE), the Nabataeo-Arabic inscriptions (3rd c. CE to 5th c. CE), pre-Islamic Arabic script inscriptions (5th c. CE to early 7th c. CE) and isolated inscriptions in the Greek, Dadanitic (the oasis of Dadān, modern day al-ʕUlā, northwest Ḥiǧāz), and Ancient South Arabian alphabets (varied chronology). 

In geographic terms, Old Arabic is attested mainly in the southern Levant, the Sinai, and northwestern Arabia, as far south as Ḥegrā (Madāʔin Ṣāleḥ). Within this area a variety of non-Arabic languages were spoken and written, with which Old Arabic interacted. The main contact language was Imperial Aramaic, which served as a literary language across North Arabia in the latter half of the first millennium BCE until, perhaps, the rise of Islam. Since contact must be viewed through the lens of writing, it is in most cases difficult to determine how extensive multilingualism was outside of literate circles.

\section{Contact Languages}
\subsection{Arabic and Akkadian}
The first attestations of Arabic are preserved in cuneiform documents. While no Arabic texts written in cuneiform have yet been discovered, isolated lexical items survive in this medium. Livingstone (\citeyear{Livingstone1997}) identified an example of the Old Arabic word for `camel' with the definite article in the inscriptions of Tiglathpileser III (744--727 BCE): \textit{a-na-qa-a-te} = (\textit{h/ʔ})\textit{an-nāq-āte} ‘the she-camels’. Aside from this, almost all other Arabic material consists of personal and divine names. There are reports of “Arabs” in Mesopotamia – inhabiting walled towns in western Babylonia – as early as the eight century. BCE (\citealt{Ephal1974}: 112). While we cannot be sure that the people whom the Babylonians called Arabs were in fact Arabic speakers, a few texts in dispersed Ancient North Arabian scripts hail from this region. So far, all seem to contain only personal names with Arabic or Arabian etymologies.\footnote{``Dispersed Ancient North Arabian'' is a temporary term given to the Ancient North Arabian inscriptions on seals, pottery, bricks, etc. which have been found in various parts of Mesopotamia and elsewhere \citep[33]{Macdonald2000}.}  These facts can only suggest the possibility of contact between speakers of Arabic and Akkadian in the early first millennium BCE.

\subsection{Arabic and Canaanite}\label{Canaanite}
Contact between Arabic speakers and speakers of Canaanite languages is documented in the Hebrew Bible (\citealt{Ephal1982}: ch.2 ;\citealt{Retsö2003}: ch.8), and there is one inscription directly attesting to contact between both groups. An Ancient North Arabian inscription from Bāyir, Jordan contains a prayer in Old Arabic to three gods of the Iron Age Canaanite kingdoms of Moab, Ammon, and Edom (\citealt{Hayajnehetal2015}). The text is accompanied by a Canaanite inscription, which remains undeciphered. The reading of the Arabic according to the edition is as follows, with my translation:

\ea Bāyir inscription (\citealt{Hayajnehetal2015})\\
\gll h 	mlkm 	w-kms 	w-qws 	b-km 		   ʕwðn h-ʔsḥy 		m-mdwst	\\
     \textsc{voc} \textsc{pn} \textsc{conj-pn} \textsc{conj-pn} \textsc{prep-2pl.m} protect.\textsc{prf.1pl} \textsc{dem}-well.\textsc{pl} \textsc{prep}-ruin \\
\glt ‘O Malkom, Kemosh, and Qaws, we place under your protection these wells against ruin.’
\z

\subsection{Arabic and Aramaic}
Evidence for contact between Arabic and Aramaic spans from the middle of the first millennium BCE to the late sixth century CE, and is concentrated in the southern Levant and northwest Arabia.\footnote{See \citet{Stein2018} on the role of Aramaic in the Arabian Peninsula in the pre-Islamic period.} Perhaps one of the earliest examples of Arabic speakers using Aramaic as a written language comes from the fifth-century-BCE Nile Delta. A king of Qedar, Qayno \ia{Qayno@Qayno} son of Gośam,\ia{Gośam@Gośam}\footnote{The symbol \textit{ś} denotes the Old Arabic reflex of Classical Arabic <{\arabscript{ش}}>, which is usually transcribed \textit{š}. /ś/ was likely realized as a voiceless lateral fricative [ɬ].} commissioned an Aramaic votive inscription dedicated to \textit{hn-ʔlt} ‘the goddess’ (\citealt{Rabinowitz1956}). Arabic names can be found in transcription across the Levant in Aramaic inscriptions (\citealt{Israel1995}), and in most cases names with an Arabic etymology terminating in the characteristic final \textit{-w}, reflecting an original nominative case (\citealt{Al-Jalladforthcoming}). 

Arabic and Aramaic language contact reaches a climax in the written record at the end of the first millennium BCE with the arrival of inscriptions in the Nabataean script. The Nabataeans established a kingdom in the region of Edom in the fourth century BCE, which at its greatest extent spanned from the Ḥawrān to the northern Ḥiǧāz. While they, like their contemporaries across the Near East, wrote in a form of Imperial Aramaic, the spoken language of the royal house and large segments of the population was Arabic. Unlike other examples of Aramaic written by Arabic speakers so far, Nabataean incorporated Arabic elements into its writing school, such as the optative use of the perfect, the negator \textit{ɣayr}, and a significant number of lexical items relating to daily life (\citealt{Gzella2015} : 242--243).

Perhaps one of the most interesting examples of contact between the two languages is found in Nabataean legal papyri from the Judaean desert (1st--2nd c. CE). These Aramaic-language legal documents contain a number of glosses in Arabic, for example: \textit{ʕqd} /ʕaqd/ ‘contract’; \textit{mʕnm} /maɣnam/ ‘profit’; \textit{prʕ} /faraʕ/ ‘to branch out’; \textit{ṣnʕh} /ṣanʕah/ ‘handiwork’, etc. (\citealt{Yardeni2014}). Macdonald (\citeyear{Macdonald2010}: 20) has suggested, based on this evidence, that Nabataean legal proceedings would have taken place in Arabic, while all written records were made in Aramaic.

In addition to the use of Arabic within Aramaic, a unique votive inscription from ʕEn ʕAvdat (Negev, Israel) contains three verses of an Arabic hymn to the deified Nabataean king ʕObodas embedded within an Aramaic text. While undated (but earlier than \~150 CE), the text is certainly the earliest example of continuous Arabic language written in the Nabataean script, as before this almost all examples are isolated words and personal names.

\ea ʕEn ʕAvdat inscription\footnote{This is my translation; the \textit{editio princeps} is Negev, Naveh \& Shaked (\citeyear{Negevetal1986}); it is discussed most recently in Fiema et al. (\citeyear{Fiemaetal2015}: 399–402) and Kropp (\citeyear{Kropp2017}).} \textcolor{red}{Please supply glosses for the Aramaic portions of this example!}\\
\ea{Aramaic} \\ 
\gll    dkyr b-ṭb q[r]ʔ qdm ʕbdt ʔlhʔ w-dkyr mn ktb grmʔlhy br tymʔlhy šlm l-qbl ʕbdt ʔlhʔ \\
Gloss here here here here here here here here here here here here here here here \\ 
\glt `May he who reads this aloud be remembered for good before ʕObodas the god, and may he who wrote be remembered. May Garmallāhi son of Taymallāhi be secure in the presence of ʕObodat (the god).'\\

\ex{Arabic} \\
\gll   p-ypʕl lʔ pdʔ w-lʔ ʔtrʔ p-kn hnʔ ybʕ-nʔ ʔl-mwtw lʔ ʔbʕ-h p-kn hnʔ ʔrd grḥw lʔ yrd-nʔ \\
       \textsc{conj}-act.\textsc{impf.3sg.m} \textsc{neg} ransom.\textsc{acc} \textsc{conj}-\textsc{neg} scar.\textsc{acc} \textsc{conj}-be.\textsc{inf} here seek.\textsc{impf.3sg.m-1pl} \textsc{def}-death.\textsc{nom} \textsc{neg} make.obtain.\textsc{inf-3sg.m} \textsc{conj}-be.\textsc{inf} here want.\textsc{prf.3sg.m} wound.\textsc{nom} \textsc{neg} want.\textsc{impf.3sg.m-1pl} \\
\glt `May he act that there be neither ransom nor scar; so be it that death would seek us, may he not aid its seeking, and so be it that a wound would desire (a victim), let it not desire us!'\\

\ex{Aramaic} \\
\gll   grmʔlhy ktb yd-h \\
Gloss needed here \\
\glt `Garmallāhi, the writing of his hand.’\\
\z
\z

The presence of Aramaic is much more lightly felt in the desert hinterland to the east and north. A small handful of Safaitic--Aramaic bilingual inscriptions are known (\citealt{Hayajneh2009}: 214--215). In one Safaitic text, produced by a Nabataean, the author gives his name and affiliation to social groups in a type of Aramaic, but then writes the remainder of the inscription in Old Arabic, suggesting that this individual may have been bilingual. 

\ea Nabataean Safaitic (\citealt{Al-Jallad2015Safaitic}: 19; C 2820) \textcolor{red}{Gloss needed here too please!} \\ 
\gll l ʔʔsd bn rbʔl bn ʔʔsd bn rbʔl nbṭwy slmy w brḥ ḫlqt śty h-dr w tð̣r h-smy \\
Gloss here here here here here here here here here here here here here here here here here \\
\glt `By ʔAʔsad son of Rabbʔel son of  ʔAʔsad son of Rabbʔel, the Nabataean Salamite, and he set off from this place for the period of winter and kept watch for the rains.' \\
\z

A handful of Aramaic loans are found in the Safaitic inscriptions: \textit{sfr} ‘writing’; \textit{ʔsyt} ‘hide, trap’, \textit{lṣṭ} ‘thief’, ultimately from Greek \textit{ληστής}. Other words, such as \textit{mdbr} /madbar/ ‘the Hamad, wilderness’ and \textit{nḫl} /naḫl/ ‘valley’, are absent in Classical Arabic yet appear in the Northwest Semitic languages. These do not appear to be loans, however, as their meanings and phonologies are local and Arabic, respectively. They should instead be regarded as genuine cognates that did not make it into the Islamic-period lexica. 

\subsection{\textit{Provincia Arabia} and the Nabataeo-Arabic script}\label{Provincia}
In 106 CE, under circumstances that remain poorly understood, the Romans annexed the Nabataean Kingdom and established their Province of Arabia. While Nabataean political independence ended, their script, writing tradition and language continued to thrive and evolve. This is exemplified by the famous tomb inscription of Raqōś bint ʕAbd-Manōto from Madāʔin Ṣāliḥ.\ia{Raqōś bint ʕAbd-Manōto@Raqōś bint ʕAbd-Manōto} Dated to 267 CE, the text is a legal inscription associated with the grave of a woman who died in al-Ḥegr. Unlike other grave inscriptions at this site, the Raqōś\ia{Raqōś bint ʕAbd-Manōto@Raqōś bint ʕAbd-Manōto} inscription is composed almost entirely in Arabic, with the Aramaic components restricted to the introductory demonstrative \textit{dnh} ‘this’, the words for ‘son’ and ‘daughter’, the dating formula, and the name of the deity. The Aramaic components are bolded below:

\ea Madāʔin Ṣāliḥ inscription (JSNab 17)\footnote{For the latest discussion of this text, see Macdonald’s contribution to Fiema et al. (\citeyear{Fiemaetal2015}: 402--405).}\label{Raqosh}\\ 
\gll   \textbf{dnh} qbrw ṣnʕ-h kʕbw \textbf{br} ḥrtt l-rqwš \textbf{brt} ʕbdmnwtw ʔm-h w-h hlkt fy ʔl-ḥgrwy \textbf{šnt} \textbf{mʔh} \textbf{w-štyn} \textbf{w-tryn} \textbf{b-yrḥ} \textbf{tmwz} w-lʕn \textbf{mryʕlmʔ} mn yšnʔ ʔl-qbrw d[ʔ] 	w-mn 		yftḥ-h ḥšy w wld-h w-lʕn mn yqbr w-{y}ʕly mn-h \\
       \textsc{dem} grave build.\textsc{prf.3sg.m-3sg.m} \textsc{pn} son \textsc{pn} \textsc{prep-pn} daughter \textsc{pn} mother-\textsc{3sg.m} \textsc{conj-3sg.f} die.\textsc{prf.3sg.f} \textsc{prep} \textsc{def-pn} year hundred \textsc{conj}-sixty \textsc{conj-two} \textsc{prep}-month Tammūz \textsc{conj}-curse.\textsc{prf.3sg.m} \textsc{pn} \textsc{rel} desecrate.\textsc{impf.3sg.m} \textsc{def}-grave \textsc{dem} \textsc{conj-rel} open.\textsc{impf.3sg.m-3sg.m} except w children-\textsc{3sg.m} \textsc{conj}-curse.\textsc{prf.3sg.m} \textsc{rel} bury.\textsc{impf.3sg.m} \textsc{conj}-remove.\textsc{impf.3sg.m} \textsc{prep-3sg.m} \\
\glt   `\textbf{This} is a grave that Kaʕbo \textbf{son} of Ḥāreθat constructed for Raqōś \textbf{daughter} of ʕAbd-Manōto,\ia{Raqōś bint ʕAbd-Manōto@Raqōś bint ʕAbd-Manōto} his mother, and she perished in al-Ḥegro \textbf{year one hundred and sixty two in the month of Tammūz}. May \textbf{the Lord of the World}  curse anyone who desecrates this grave and anyone who would open it, with the exception of his children, and may he curse anyone who would bury or remove from it (a body).'\\
\z

During the same period, the classical Nabataean script begins to evolve towards what we consider the Arabic script (\citealt{Nehmé2010}). Its letter forms take on a more cursive character, and the connecting element of each letter goes across the bottom of the text. Nehmé considers the letter forms typical of the Arabic script to have evolved from Nabataean between the third and fifth centuries CE. In inscriptions from this period, the Arabic component begins to increase at the expense of Aramaic (\citealt{Nehmé2017}). This trend may suggest that knowledge of Aramaic was waning in these centuries, or that the writing tradition itself was transforming – Aramaic was slowly being replaced by Arabic. If we think in terms of writing schools, there may not have been much Arabic/--Aramaic bilingualism in Arabia outside of the scribal class – indeed, scholars have continued to debate whether Nabataean Aramaic was ever a colloquial, and there are good arguments to doubt that it was (\citealt{Gzella2015}: 240). The remnants of Aramaic in the latest phases of the Nabataeo-Arabic inscriptions, however, most certainly functioned as a code, grams for Arabic words, a situation comparable to the Aramaeograms of Pahlavi. \textcolor{red}{(Reference for this would be good here).}

\subsection{The Arabic inscriptions of the sixth century CE}
In Arabic inscriptions of the sixth century, written Arabic and Aramaic continue the stable situation of contact witnessed in the Nabataeo-Arabic period. Aramaic fossils are employed in dating formulae and the word for ‘son’, and possibly the first person pronoun. But otherwise, the language of these texts is entirely Arabic. Perhaps the most famous among these is the inscription of Jebel Usays, given in \REF{Usays}, in which the Aramaic components are bolded.

\ea\label{Usays} Jebel Usays inscription\footnote{For the latest discussion of this text, see Macdonald’s contribution to Fiema et al. (\citeyear{Fiemaetal2015}: 405).}\\
\gll   \textbf{ʔnh}\footnotemark{} rqym \textbf{br} mʕrf ʔl-ʔwsy ʔrsl-ny ʔlḥrt ʔl-mlk ʕly ʔsys mslḥh snt 423 \\
       \textsc{1sg} \textsc{pn} son \textsc{pn} \textsc{def}-Awsite send.\textsc{prf.3sg.m-1sg} \textsc{pn} \textsc{def}-king \textsc{prep} Usays border\_guard year 423  \\
\glt   `\textbf{I}, Ruqaym \textbf{son} of Muʕarrif the Awsite al-Ḥāriθ the king sent me to Usays as a border guard, year 423 (= 528/9 CE).'\\
\z\footnotetext{While it has been suggested that the spelling \textit{ʔnh} reflects a pausal form \citep{Larcher2010}, it seems more likely in light of the Thaʕlabah Nabataeo-Arabic inscription \citep{AvnerNehmeRobin2013}, which spells `I' as \textit{ʔnh}, that this form reflects the Aramaic spelling of the pronoun rather than an Arabic variant.}

\subsection{Arabic, Greek and Aramaic in sixth-century Petra}
In 1993, a corpus of carbonized Greek papyri – some 140 rolls – was discovered at the Byzantine church of Petra.\footnote{These papyri are edited in a five-volume series, the \textit{Petra Papyri I--V} (2002--2018), various editors, Amman: American Center of Oriental
Research. See \citet{Arjavaetal2018} for the last volume.} These documents attest to a trilingual situation at the city: Greek served as the official administrative language, while Arabic and Aramaic appear to have been spoken languages. The microtoponyms (names of small plots of lands and vineyards) are in both Arabic and Aramaic, and often times the same word is expressed in both languages, as in Table \REF{tab:Petra}. 

\begin{table}[H]
\caption{Arabic--Aramaic equivalents in the Petra Papyri (\citealt{Al-Jallad2018Petra}) \textcolor{red}{Please supply the page numbers of your chapter for this reference.}}
\label{tab:Petra}
\begin{tabular}{lllll}
\lsptoprule
Translation & Arabic & Aramaic \\
\midrule
`land markers' & \multicolumn{1}{l}{Αραμ /ārām/} & \multicolumn{1}{l}{Εραμαεια /eramayyā/}  \\
`farm' & \multicolumn{1}{l}{αλ-Ναϲβα /al-naṣbah/} & \multicolumn{1}{l}{Ναϲβαθα /naṣbatā/}  \\
`canal' & \multicolumn{1}{l}{αλ-Κεϲεβ /al-qeṣeb/} & \multicolumn{1}{l}{Κιϲβα/Κειϲβα /qiṣbā/}  \\
\lspbottomrule
\end{tabular}
\end{table}

This naturally suggests that, alongside literacy in Greek, there was spoken bilingualism in Arabic and Aramaic, perhaps a stable situation extending back to Nabataean times. 

\subsection{Arabic and Ancient South Arabian}
Classical Arabic sources note a situation of close contact between Arabic and “Ḥimyaritic”, a term used for a language they associated with the pre-Islamic kingdom of Ḥimyar in what is today Yemen. The pre-Islamic inscriptions from the northern Yemeni Jawf, the so-called Ḥaram region, attest to a similar situation. These texts are composed in Sabaic, but contain a significant admixture of non-Sabaic linguistic material. Some scholars (e.g. \citealt{Robin1991}) have considered Arabic to be the contributing source, but in most cases the non-Sabaic linguistic features are not specific to Arabic, such as the use of the causative verb ʔaCCaC, which is attested in Aramaic and Ge’ez for example, rather than haCCaC as in Sabaic. As Macdonald (\citeyear{Macdonald2000}: 55) rightly puts it, these inscriptions are basically Sabaic, with a small admixture from North Arabian languages, but not necessarily Arabic. Four texts from this region, however, exhibit the Arabic isogloss of \textcolor{red}{\textit{lam} for past-tense negation}, suggesting that some form of Arabic may have contributed to their mixed character.\footnote{For a list of the Ḥaram inscriptions, see Macdonald (\citeyear{Macdonald2000}: 61), who labels these texts Sabaeo-North-Arabian.}   

Mixed North/South Arabian texts can be found further to the north, in Naǧrān and Qaryat al-Fāw. The most famous is perhaps the grave inscription of \textit{Rbbl bn Hfʕm}. This unique text attests features that can be attributed to both non-Sabaic and Sabaic sources. On the non-Sabaic side, it uses the definite article \textit{ʔl}, the causative morpheme \textit{ʔ-} rather than \textit{h-}, and occasionally the 3rd person pronoun \textit{h} rather than \textit{hw}. At the same time, the text employs mimation, clitic pronouns with long vowels, e.g. \textit{-hw}, and prepositions not known in Arabic (\citealt{Al-Jallad2018ANA}: 30 \textcolor{red}{-- Is this the correct 2018 article?}). At Naǧrān, one occasionally encounters Arabic lexical items, such as \textit{ldy} ‘at’ and \textit{ʕnd} ‘with’ in otherwise perfectly good South Arabian texts. So then, how are we to interpret the mixed character of these texts? For Qaryat al-Fāw, Durand (\citeyear{Durand2017}: 95, fn.32) has suggested, based on the significant amount of Petraean pottery, that a sizable Nabataean colony existed at the oasis. It could be the case that Nabataean colonists introduced Arabic to the oasis, where it naturally gained prestige as a trade language given its links with the north. The mixed nature of some of the inscriptions of this site could therefore be interpreted in two ways. If they reflect a spoken variety, then perhaps they are the result of convergence between the Arabic introduced by the Nabataeans and Sabaic, similar to the modern dialects of Yemeni Arabic today, which are essentially Arabic with a significant South Arabian admixture.\footnote{On these varieties, see \citet{Watson2018}.} If we are dealing with an artificial scribal register, then the language may be the result of a scribe attempting to produce a text in Arabic, for an Arabic-speaking customer, but inadvertently introducing Sabaicisms from the language he is more used to writing. A similar phenomenon might be at play in the Aramaic–Hasaitic tomb inscription from Mleiha.\footnote{Hasaitic is the name given to the pre-Islamic script and language of East Arabia.} There, the scribe – seemingly unintentionally – uses the Aramaic word for son, \textit{br}, in the Hasaitic portion of the text, suggesting perhaps that he was bilingual and more used to writing in Aramaic (\citealt{Overlaetetal2016}).

\subsection{Arabic in the Ḥiǧāz}
Before the arrival of the Nabataeans, the written language of the oasis of al-ʕUlā and associated environs in the northern Ḥiǧāz was Dadanitic, a non-Arabic Central Semitic language. Dadanitic and Arabic are closely related, but still distinct languages. A few texts, however, display features that are unambiguously Arabic. The best known of these is JSLih 384. This short text is written in the Dadanitic script but seems to be, in other respects, produced in a dialect of Old Arabic, notably making use of the relative pronoun \textit{ʔlt} /ʔallatī/. Two other Dadanitic texts make use of the Arabic construction \textit{ʔn yfʕl}, that is, the use of the subordinator \textit{ʔan} with a modal verb. In addition to this, one occasionally finds the \textit{ʔ(l)} definite article employed in these inscriptions. The interpretation of this contact situation, like that in South Arabia, is unclear. Do these few texts represent the writings of travelers or immigrants from the north, whose spoken language influenced the dictation of text to the scribe? Or do they reflect unique points on a dialect continuum? The complex linguistic situation at ancient Dadan is the subject of a fascinating study by \citet{Kootstra2019}.

\subsection{Arabic and the languages of the Thamudic inscriptions}
Even more difficult to distill is the possible contact situation between Arabic and the more shadowy pre-Arabic Semitic languages of north and central Arabia. We are afforded a small glimpse of these languages by the laconic Thamudic inscriptions, mainly those classified in the C, D, and F scripts.\footnote{Thamudic B, C, and D are discussed in Macdonald (\citeyear{Macdonald2000}) and Al-Jallad (\citeyear{Al-Jallad2017early}, \citeyear{Al-Jallad2018ANA}); Thamudic F is outlined in Prioletta \& Robin (\citeyear{PriolettaRobin2018}).}  While it is difficult to say much about the languages these scripts express, they are clearly distinct from Arabic (\citealt{Al-Jallad2017early}: 321--322). The only evidence for contact between Arabic and any of these languages is found in the tomb inscription of Raqōś\ia{Raqōś bint ʕAbd-Manōto@Raqōś bint ʕAbd-Manōto} at Madāʔin Ṣāliḥ, illustrated in \REF{Raqosh}. This text, as discussed in §\ref{Provincia}, is written mainly in Arabic, with a few fossilized Aramaic components. Alongside the main inscription, there is a short text inscribed in the Thamudic D script stating: \textit{zn rqś bnt ʕbdmnt} ‘This is Raqōś, daughter of ʕAbdo-Manōto’.\ia{Raqōś bint ʕAbd-Manōto@Raqōś bint ʕAbd-Manōto} The use of the demonstrative \textit{zn} /zin/(?) rather than the Arabic demonstrative \textit{dʔ} /ðā/ or perhaps its feminine equivalent \textit{dy} /ðī/, employed in the Nabataean text, indicates that we are dealing with a third language. Did Raqōś\ia{Raqōś bint ʕAbd-Manōto@Raqōś bint ʕAbd-Manōto} originally hail from a nomadic community who spoke a non-Arabic Semitic language expressed in the Thamudic D script? And did she later come to live in Arabic-speaking Ḥegrā? Was the use of this script on her grave a tribute to her heritage? These questions are impossible to answer with the data available to us now, but they widen the scope of investigation when examining Arabic’s history. The available fragments of evidence support the suggestion put forth recently by Souag (\citeyear{Souag2018blog}): we must consider the possibility of unknown Semitic substrate(s) in the development of early Arabic.

\subsection{Arabic and Greek}
The nexus of Arabic--Greek contact, based on the inscriptions known so far, is the Syro-Jordanian Harrah, the basalt desert that spans from the Hawrān to northern Arabia. Greek inscriptions are occasionally found throughout this region, interacting with the local Arabic dialects in diverse ways. The commonest type of bilingual text consists of simple signatures in Safaitic and Greek. These texts only prove that the author knew how to write his name in Greek, and do not constitute evidence for genuine bilingualism. 

An example of limited bilingualism is attested in the Graeco-Arabic inscription A1, illustrated in \REF{Taym}, where the author begins writing his name in a Hellenized form, but then, perhaps having exhausted his Greek, shifts to Arabic written in Greek letters to finish his inscription.

\ea Graeco-Arabic inscription A1 (\citealt{Al-Jalladal-Manaser2016}: 56)\label{Taym}\\
\ea{Greek} \\ 
\gll    Θαιμος Γαφαλου \\
        Taym Gaḥfal\\
\glt `Taym, son of Gaḥfal'\\

\ex{Arabic} \\
\gll   l-tm bn gḥfl \\
       \textsc{prep}-Taym son Gaḥfal \\
\glt `for/by Taym, son of Gaḥfal'\\
\z
\z

 Another particularly informative example is the second inscription discussed by \citet{Al-Jalladal-Manaser2016}, illustrated in \REF{Ghawth}. The author carves a short text in both Greek and Old Arabic, indicating that he knew both languages but that his command of Arabic was obviously better.

\ea Graeco-Arabic inscription 2 (\citealt{Al-Jalladal-Manaser2016}: 58) \textcolor{red}{Please check/correct glossing!}\label{Ghawth}\\
\ea{Arabic} \\
\gll   l-ɣθ w tḥll ʔfwh ʕql sr \\
       \textsc{prep}-Ghawth \textsc{conj} go.\textsc{prf.3sg.m} \textcolor{red}{?} protected.area Sayr \\
\glt `By Ghawth and he went into the protected area of Sayr.'\\
\ex{Greek} \\ 
\gll    Γαυτος ἀπῆλθεν [ε]ἰς τόν Ακελον Σαιρου \\
Ghawth.\textsc{nom} depart.\textsc{aor.3sg} \textsc{prep} \textsc{def.m.acc.sg} \textit{ʕaql}.\textsc{acc.sg} Sayr.\textsc{gen}\\ 
\glt `Ghawth, he went away to the \textit{ʕaql} of Sayr.'\\
\z
\z

The author translates the Arabic into Greek effectively, but seems not to have known the Greek word for the culturally specific term \textit{ʕaql}, ‘a protected area of pasturage’. In this case, he simply wrote the word out in Greek: \textit{Ακελον}.

There is evidence that some nomadic Arabic speakers did master the Greek language, as one sometimes comes across very well-composed texts in Greek, attesting to full-scale bilingualism, at least in writing (for example A2 in \citet{Al-Jalladal-Manaser2015}. This level of bilingualism, however, must have been rare. There is no appreciable influence from Greek on the Arabic of the Safaitic inscriptions. A few loanwords are known, e.g. \textit{qṣr} ‘Caesar’, \textit{lṣṭ} ‘thief', but these more likely come through Aramaic.

\subsection{Arabic in eastern Arabia}
The inscriptional record of eastern Arabia is relatively poor when compared to the western two-thirds of the Peninsula. Nevertheless, the extant texts point towards contact between Aramaic and the local Arabian language, called Hasaitic by scholars. This language, however, cannot be regarded as a form of Arabic, and there are no pre-Islamic attestations of Arabic from eastern Arabia yet (\citealt{Al-Jallad2018ANA}: 260--261). 

\section{Grammatical features arising from contact}
This section offers a contact-based explanation for two linguistic features found in Old Arabic: the definite article, and the realization of the feminine ending. I hope to show how contact-based explanations, coupled with documentary evidence, can help inform the development of grammatical features that come to characterize later forms of Arabic.

\subsection{Definite article}
It has long been established that the overt marking of definiteness in the Semitic languages is a relatively late innovation (\citealt{HuehnergardRubin2011}: 260--261). All varieties of Arabic today attest some form of the definite article -- most commonly variants of \textit{ʔal} but other forms exist as well, mainly in southwest Arabia, including \textit{am}, \textit{an}, and \textit{a-}, with gemination of the following consonant. In light of the comparative evidence, did Arabic innovate this feature independently or was contact with other Semitic languages involved?

The evidence suggests that the prefixed article *han- emerged in the central Levant sometime in the late second millennium BCE, after the diversification of Northwest Semitic (\citealt{Tropper2001,Gzella2006,Pat-El2006}). It seems clear that by the early first millennium BCE, the article had spread across the southern Levant and to North Arabia, as it is found in Taymanitic, Thamudic B, and Dadanitic, as well as in the Old Arabic of the Safaitic inscriptions. In the latter case, contact with Canaanite is substantiated in the inscriptional record in the form of the Bāyir inscription (see §\ref{Canaanite} above). 

All of these languages, including the earliest Old Arabic, took over the form of the article unchanged; that is \textit{h-} with the assimilation of the /n/ before a consonant, the exception being Dadanitic, which preserves the /n/ before laryngeal consonants, e.g. \textit{h-mlk} /ham-malk/ ‘the king’ vs. \textit{hn-ʔʕly} ‘the upper’ /han-ʔaʕlay/. We cannot, however, argue for the spread of the definite article to Proto-Arabic. The original, article-less situation is attested in the inscriptions of Central Jordan stretching down to the Hismā, known as Hismaic (\citealt{GrafZwettler2004}). These texts are in unambiguously Arabic language, but they lack the definite article. The \textit{h-}morpheme exists, but it has a strong demonstrative force. Indeed, in a few Nabataean--Hismaic bilingual inscriptions, the definite article \textit{ʔl} of the Nabataean component is rendered as zero in the Hismaic text (\citealt{Hayajneh2009}). A minority of Safaitic inscriptions also lack the definite article (\citealt{Al-Jallad2018ANA}), showing that it had not spread to all varieties of Arabic even as late as the turn of the Era. Thus, like Hebrew and Aramaic, the earliest linguistic stages of Arabic – and indeed Proto-Arabic – lacked a definite article. Contact with Canaanite then seems to be the likeliest explanation for the appearance of the \textit{h-}article in Old Arabic.

While the \textit{h-} article is the commonest form in Old Arabic, whence the \textit{ʔal} form? The \textit{ʔal} article appears to be a later development from the original \textit{han} article, through two irregular sound changes: \textit{h} > \textit{ʔ} and \textit{n} > \textit{l}.\footnote{The origins of the \textit{al}-article are discussed in detail in \citep{Al-Jalladforthcoming}.} The former is well attested in Arabic (e.g. the causative ʔaCCaCa from haCCaCa), while the latter is not uncommon in loans (e.g. \textit{finǧān} vs. \textit{finǧāl} ‘cup’). The \textit{ʔal} article appears to have developed in the western dialects of Old Arabic, attested first in the Nile Delta (cf. the famous \textit{αλιλατ} \textit{al-ʔilat} ‘the goddess’ mentioned in Herodotus, \textit{Histories} I: 131), and is the regular form of the article in the dialect of the Nabataeans, who were situated in ancient Edom, stretching south to the Ḥiǧāz. The \textit{ʔal}-article is attested sporadically at Dadān in the western Ḥiǧāz as well. Based on the inscriptional record, the \textit{ʔal}-article was a typical linguistic feature of settled, rather than nomadic groups, being attested most frequently in the Nabataean dialect, and in cities and oases like Petra and Ḥegrā. The nomads used a variety of definite article forms. It was perhaps not until the rise of Islam, and the resulting prestige given to official Arabic of the Umayyad state, that the \textit{ʔal} article began to dominate at the expense of other forms.

\subsection{The feminine ending}
In most modern Arabic dialects, the feminine ending *-at is realized as \textit{-a(h)} in all contexts except the construct state, where it retains its original form \textit{-at}. In Classical Arabic, it is \textit{-at} in all situations, except for in utterance-final position, where it is realized as \textit{-ah}. The Quranic Consonantal Text resembles the situation in the modern dialects, as do the transitional Nabataeo-Arabic and sixth-century Arabic script inscriptions (\citealt{Nehmé2017}). Yet, if we go back further to the first century CE, it seems that varieties of Arabic written in the Hismaic and Safaitic script never experienced the sound change \textit{-at} > \textit{-ah} in any position – the feminine ending is always written as <t>. In the Arabic of the Nabataeans, however, the sound change of \textit{-at} to \textit{-ah} seems to have operated as early as the third century BCE (\citealt{Al-Jallad2017early}: §5.2.1). 

The sound change \textit{-at} > \textit{-ah} is common in the Central Semitic languages, but the distribution can vary. In Phoenician, it applies to verbs and not nouns, while in Hebrew it applies equally to nouns and verbs (\citealt{HuehnergardRubin2011}: 265--266). The most common Arabic distribution matches Aramaic: it applies to nouns but not verbs. I would suggest that, since this sound change is first attested in a dialect of Arabic for which we have abundant evidence of heavy contact with Aramaic, this is a contact-induced change (see also van Putten, this volume).\ia{van Putten, Marijn@van Putten, Marijn} For this reason, the change is not attested in the ancient nomadic dialects, where, as we have seen above, there is little evidence for contact with Aramaic. Thus, like the \textit{ʔal} article, the \textit{-at} to \textit{-ah} change would have been a typical feature of Arabic dialects of settled groups in the pre-Islamic period. In later forms of Arabic, the change spreads even to nomadic dialects, as we find it operational today across the Arabian Peninsula. Yet, the chronology of this diffusion is not quite clear. In an important study by \citet{vanPutten2017}, the Dosiri dialect of Kuwait appears to preserve the archaic situation where the feminine ending is realized as \textit{-at} in all positions.

\section{Conclusion}
Contact must be factored into our understanding of language change for Arabic at every attested stage. A summary of the facts above show that Arabic was in most intense contact with Aramaic, a situation that persisted for over a millennium prior to the rise of Islam, which may explain the high number of Aramaic loanwords into Arabic, and indeed some striking structural parallels, such as the distribution of the sound change \textit{-at} > \textit{-ah}. At the same time, there is very little evidence for contact with Sabaic (Old South Arabian), a contact situation only represented by a small number of mixed texts.  This nicely matches the absence of South Arabian influence on Old Arabic and later forms of the language, with the exception of those dialects spoken in southwest Arabia.

\section*{Further reading}
\citet{Al-Jallad2018ANA} provides a comprehensive outline of the languages and scripts of pre-Islamic North Arabia.\\
\citet{Macdonald2003} gives a description of the multilingual environment of ancient Nabataea. \\
\citet{Nehmé2010} outlines the development of the Arabic script based on the newest Nabataeo-Arabic inscriptions from Northwest Arabia.\\
\citet{Stein2018} gives an outline of the use of Aramaic in pre-Islamic Arabia.\\

\section*{Acknowledgments}
\textcolor{red}{Do you have any you want to add?}

\section*{Abbreviations}

\begin{tabularx}{.45\textwidth}{lQ}
\textsc{1, 2, 3} & 1st, 2nd, 3rd person \\
\textsc{acc} & accusative \\
\textsc{aor} & aorist \\
BCE & before Common Era \\
\textsc{c} & common gender \\
c. & century \\
CE & Common Era \\
\textsc{def} & definite \\
\textsc{dem} & demonstrative \\
\textsc{f} & feminine \\
\end{tabularx}
\begin{tabularx}{.45\textwidth}{lQ}
\textsc{impf} & imperfect \\
\textsc{inf} & infinitive \\
\textsc{m} & masculine \\
\textsc{nom} & nominative \\
\textsc{pl} & plural \\
\textsc{pn} & proper noun \\
\textsc{prep} & preposition \\
\textsc{prf} & perfect \\
\textsc{sg} & singular \\
\textsc{voc} & vocative \\
\end{tabularx}


\sloppy
\printbibliography[heading=subbibliography,notkeyword=this] 
\end{document}
