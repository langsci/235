\documentclass[output=paper]{langsci/langscibook} 
\author{William M. Cotter\affiliation{University of Arizona}}
\title{Dialect contact and phonological change}
\abstract{This chapter examines phonological and phonetic changes that have been documented and analyzed in spoken Arabic varieties, occurring as a result of dialect contact. The factors contributing to dialect contact in Arabic-speaking communities vary, from economic migration which has encouraged individuals to move into new dialect areas seeking work, to migration that stems from political violence and upheaval. These diverse factors have contributed to the large-scale migration of Arabic speakers to other parts of the Arabic speaking world. As a result, dialect contact is rampant, and decades of Arabic sociolinguistic research have shown that the phonological and phonetic effects of these contact situations have been quite profound.}
\IfFileExists{../localcommands.tex}{
  % add all extra packages you need to load to this file 
\usepackage{graphicx}
\usepackage{tabularx}
\usepackage{amsmath} 
\usepackage{multicol}
\usepackage{lipsum}
\usepackage[stable]{footmisc}
\usepackage{adforn}
%%%%%%%%%%%%%%%%%%%%%%%%%%%%%%%%%%%%%%%%%%%%%%%%%%%%
%%%                                              %%%
%%%           Examples                           %%%
%%%                                              %%%
%%%%%%%%%%%%%%%%%%%%%%%%%%%%%%%%%%%%%%%%%%%%%%%%%%%%
% remove the percentage signs in the following lines
% if your book makes use of linguistic examples
\usepackage{./langsci/styles/langsci-optional} 
\usepackage{./langsci/styles/langsci-lgr}
\usepackage{morewrites} 
%% if you want the source line of examples to be in italics, uncomment the following line
% \def\exfont{\it}

\usepackage{enumitem}
\newlist{furtherreading}{description}{1}
\setlist[furtherreading]{font=\normalfont,labelsep=\widthof{~},noitemsep,align=left,leftmargin=\parindent,labelindent=0pt,labelwidth=-\parindent}
\usepackage{phonetic}
\usepackage{chronosys,tabularx}
\usepackage{csquotes}
\usepackage[stable]{footmisc} 

\usepackage{langsci-bidi}
\usepackage{./langsci/styles/langsci-gb4e} 

  \makeatletter
\let\thetitle\@title
\let\theauthor\@author 
\makeatother

\newcommand{\togglepaper}[1][0]{ 
  \bibliography{../localbibliography}
  \papernote{\scriptsize\normalfont
    \theauthor.
    \thetitle. 
    To appear in: 
    Christopher Lucas and Stefano Manfredi (eds.),  
    Arabic and contact-induced language change
    Berlin: Language Science Press. [preliminary page numbering]
  }
  \pagenumbering{roman}
  \setcounter{chapter}{#1}
  \addtocounter{chapter}{-1}
}

\newfontfamily\Parsifont[Script=Arabic]{ScheherazadeRegOT_Jazm.ttf} 
\newcommand{\arabscript}[1]{\RL{\Parsifont #1}}
\newcommand{\textarabic}[1]{{\arabicfont #1}}

\newcommand{\textstylest}[1]{{\color{red}#1}}

\patchcmd{\mkbibindexname}{\ifdefvoid{#3}{}{\MakeCapital{#3}
}}{\ifdefvoid{#3}{}{#3 }}{}{\AtEndDocument{\typeout{mkbibindexname could
not be patched.}}}

%command for italic r with dot below with horizontal correction to put the dot in the prolongation of the vertical stroke
%for some reason, the dot is larger than expected, so we explicitly reduce the font size (to \small)
%for the time being, the font is set to an absolute value. To be more robust, a relative reduction would be better, but this might not be required right now
\newcommand{\R}{r\kern-.05ex{\small{̣}}\kern.05ex}


\DeclareLabeldate{%
    \field{date}
    \field{year}
    \field{eventdate}
    \field{origdate}
    \field{urldate}
    \field{pubstate}
    \literal{nodate}
}

\renewbibmacro*{addendum+pubstate}{% Thanks to https://tex.stackexchange.com/a/154367 for the idea
  \printfield{addendum}%
  \iffieldequalstr{labeldatesource}{pubstate}{}
  {\newunit\newblock\printfield{pubstate}}
}
 
  %% hyphenation points for line breaks
%% Normally, automatic hyphenation in LaTeX is very good
%% If a word is mis-hyphenated, add it to this file
%%
%% add information to TeX file before \begin{document} with:
%% %% hyphenation points for line breaks
%% Normally, automatic hyphenation in LaTeX is very good
%% If a word is mis-hyphenated, add it to this file
%%
%% add information to TeX file before \begin{document} with:
%% %% hyphenation points for line breaks
%% Normally, automatic hyphenation in LaTeX is very good
%% If a word is mis-hyphenated, add it to this file
%%
%% add information to TeX file before \begin{document} with:
%% \include{localhyphenation}
\hyphenation{
affri-ca-te
affri-ca-tes
com-ple-ments
homo-phon-ous
start-ed
Meso-potam-ian
morpho-phono-logic-al-ly
morpho-phon-em-ic-s
Palestin-ian
re-present-ed
Ki-nubi
ḥawār-iyy-ūn
archa-ic-ity
fuel-ed
de-velop-ment
pros-od-ic
Arab-ic
in-duced
phono-logy
possess-um
possess-ive-s
templ-ate
spec-ial
espec-ial-ly
nat-ive
pass-ive
clause-s
potent-ial-ly
Lusignan
commun-ity
tobacco
posi-tion
Cushit-ic
Middle
with-in
re-finit-iz-ation
langu-age-s
langu-age
diction-ary
glossary
govern-ment
eight
counter-part
nomin-al
equi-valent
deont-ic
ana-ly-sis
Malt-ese
un-fortun-ate-ly
scient-if-ic
Catalan
Occitan
ḥammāl
cross-linguist-ic-al-ly
predic-ate
major-ity
ignor-ance
chrono-logy
south-western
mention-ed
borrow-ed
neg-ative
de-termin-er
European
under-mine
detail
Oxford
Socotra
numer-ous
spoken
villages
nomad-ic
Khuze-stan
Arama-ic
Persian
Ottoman
Ottomans
Azeri
rur-al
bi-lingual-ism
borrow-ing
prestig-ious
dia-lects
dia-lect
allo-phone
allo-phones
poss-ible
parallel
parallels
pattern
article
common-ly
respect-ive-ly
sem-antic
Moroccan
Martine
Harrassowitz
Grammatic-al-ization
grammatic-al-ization
Afro-asiatica
Afro-asiatic
continu-ation
Semit-istik
varieties
mono-phthong
mono-phthong-ized
col-loquial
pro-duct
document-ary
ex-ample-s
ex-ample
termin-ate
element-s
Aramaeo-grams
Centr-al
idioms
Arab-ic
Dadan-it-ic
sub-ordin-ator
Thamud-ic
difficult
common-ly
Revue
Bovingdon
under
century
attach
attached
bundle
graph-em-ic
graph-emes
cicada
contrast-ive
Corriente
Andalusi
Kossmann
morpho-logic-al
inter-action
dia-chroniques
islámica
occid-ent-al-ismo
dialecto-logie
Reichert
coloni-al
Milton
diphthong-al
linguist-ic
linguist-ics
affairs
differ-ent
phonetic-ally
kilo-metres
stabil-ization
develop-ments
in-vestig-ation
Jordan-ian
notice-able
level-ed
migrants
con-dition-al
certain-ly
general-ly
especial-ly
af-fric-ation
Jordan
counter-parts
com-plication
consider-ably
inter-dent-al
com-mun-ity
inter-locutors
com-pon-ent
region-al
socio-historical
society
simul-taneous
phon-em-ic
roman-ization
Classic-al
funeral
Kurmanji
pharyn-geal-ization
vocab-ulary
phon-et-ic
con-sonant
con-sonants
special-ized
latter
latters
in-itial
ident-ic-al
cor-relate
geo-graphic-al-ly
Öpengin
Kurd-ish
in-digen-ous
sunbul
Christ-ian
Christ-ians
sekin-în
fatala
in-tegration
dia-lect-al
Matras
morpho-logy
in-tens-ive
con-figur-ation
im-port-ant
com-plement
ḥaddād
e-merg-ence
Benjmamins
struct-ure
em-pir-ic-al
Orient-studien
Anatolia
American
vari-ation
Jastrow
Geoffrey
Yarshater
Ashtiany
Edmund
Mahnaz
En-cyclo-pædia
En-cyclo-paedia
En-cyclo-pedia
Leiden
dia-spora
soph-is-ic-ated
Sasan-ian
every-day
domin-ance
Con-stitu-tion-al
religi-ous
sever-al
Manfredi
re-lev-ance
re-cipi-ent
pro-duct-iv-ity
turtle
Morocco
ferman
Maghreb-ian
algérien
stand-ard
systems
Nicolaï
Mouton
mauritani-en
Gotho-burg-ensis
socio-linguist-ique
plur-al
archiv-al
Arab-ian
drop-ped
dihāt
de-velop-ed
ṣuḥbat
kitāba
kitābat
com-mercial
eight-eenth
region
Senegal
mechan-ics
Maur-itan-ia
Ḥassān-iyya
circum-cision
cor-relation
labio-velar-ization
vowel
vowels
cert-ain
īggīw
series
in-tegrates
dur-ative
inter-dent-als
gen-itive
Tuareg
tălămut
talawmāyət
part-icular
part-icular-ly
con-diment
vill-age
bord-er
polit-ical
Wiesbaden
Uni-vers-idad
Geuthner
typo-logie
Maur-itanie
nomades
Maur-itan-ian
dia-lecto-logy
Sahar-iennes
Uni-vers-ity
de-scend-ants
NENA-speak-ing
speak-ing
origin-al
re-captured
in-habit-ants
ethnic
minor-it-ies
drama-tic
local
long-stand-ing
regions
Nineveh
settle-ments
Ṣəndor
Mandate
sub-stitut-ing
ortho-graphy
re-fer-enced
origin-ate
twenti-eth
typ-ic-al-ly
Hobrack
never-the-less
character-ist-ics
character-ist-ic
masc-uline
coffee
ex-clus-ive-ly
verb-al
re-ana-ly-se-d
simil-ar-ities
de-riv-ation
im-pera-tive
part-iciple
dis-ambi-gu-ation
dis-ambi-gu-a-ing
phen-omen-on
phen-omen-a
traktar
com-mun-ity
com-mun-ities
dis-prefer-red
ex-plan-ation
con-struction
wide-spread
us-ual-ly
region-al
Bulut
con-sider-ation
afro-asia-tici
Franco-Angeli
Phono-logie
Volks-kundliche
dia-lectes
dia-lecte
select-ed
dis-appear-ance
media
under-stand-able
public-ation
second-ary
e-ject-ive
re-volu-tion
re-strict-ive
Gasparini
mount-ain
mount-ains
yellow
label-ing
trad-ition-al-ly
currently
dia-chronic
}
\hyphenation{
affri-ca-te
affri-ca-tes
com-ple-ments
homo-phon-ous
start-ed
Meso-potam-ian
morpho-phono-logic-al-ly
morpho-phon-em-ic-s
Palestin-ian
re-present-ed
Ki-nubi
ḥawār-iyy-ūn
archa-ic-ity
fuel-ed
de-velop-ment
pros-od-ic
Arab-ic
in-duced
phono-logy
possess-um
possess-ive-s
templ-ate
spec-ial
espec-ial-ly
nat-ive
pass-ive
clause-s
potent-ial-ly
Lusignan
commun-ity
tobacco
posi-tion
Cushit-ic
Middle
with-in
re-finit-iz-ation
langu-age-s
langu-age
diction-ary
glossary
govern-ment
eight
counter-part
nomin-al
equi-valent
deont-ic
ana-ly-sis
Malt-ese
un-fortun-ate-ly
scient-if-ic
Catalan
Occitan
ḥammāl
cross-linguist-ic-al-ly
predic-ate
major-ity
ignor-ance
chrono-logy
south-western
mention-ed
borrow-ed
neg-ative
de-termin-er
European
under-mine
detail
Oxford
Socotra
numer-ous
spoken
villages
nomad-ic
Khuze-stan
Arama-ic
Persian
Ottoman
Ottomans
Azeri
rur-al
bi-lingual-ism
borrow-ing
prestig-ious
dia-lects
dia-lect
allo-phone
allo-phones
poss-ible
parallel
parallels
pattern
article
common-ly
respect-ive-ly
sem-antic
Moroccan
Martine
Harrassowitz
Grammatic-al-ization
grammatic-al-ization
Afro-asiatica
Afro-asiatic
continu-ation
Semit-istik
varieties
mono-phthong
mono-phthong-ized
col-loquial
pro-duct
document-ary
ex-ample-s
ex-ample
termin-ate
element-s
Aramaeo-grams
Centr-al
idioms
Arab-ic
Dadan-it-ic
sub-ordin-ator
Thamud-ic
difficult
common-ly
Revue
Bovingdon
under
century
attach
attached
bundle
graph-em-ic
graph-emes
cicada
contrast-ive
Corriente
Andalusi
Kossmann
morpho-logic-al
inter-action
dia-chroniques
islámica
occid-ent-al-ismo
dialecto-logie
Reichert
coloni-al
Milton
diphthong-al
linguist-ic
linguist-ics
affairs
differ-ent
phonetic-ally
kilo-metres
stabil-ization
develop-ments
in-vestig-ation
Jordan-ian
notice-able
level-ed
migrants
con-dition-al
certain-ly
general-ly
especial-ly
af-fric-ation
Jordan
counter-parts
com-plication
consider-ably
inter-dent-al
com-mun-ity
inter-locutors
com-pon-ent
region-al
socio-historical
society
simul-taneous
phon-em-ic
roman-ization
Classic-al
funeral
Kurmanji
pharyn-geal-ization
vocab-ulary
phon-et-ic
con-sonant
con-sonants
special-ized
latter
latters
in-itial
ident-ic-al
cor-relate
geo-graphic-al-ly
Öpengin
Kurd-ish
in-digen-ous
sunbul
Christ-ian
Christ-ians
sekin-în
fatala
in-tegration
dia-lect-al
Matras
morpho-logy
in-tens-ive
con-figur-ation
im-port-ant
com-plement
ḥaddād
e-merg-ence
Benjmamins
struct-ure
em-pir-ic-al
Orient-studien
Anatolia
American
vari-ation
Jastrow
Geoffrey
Yarshater
Ashtiany
Edmund
Mahnaz
En-cyclo-pædia
En-cyclo-paedia
En-cyclo-pedia
Leiden
dia-spora
soph-is-ic-ated
Sasan-ian
every-day
domin-ance
Con-stitu-tion-al
religi-ous
sever-al
Manfredi
re-lev-ance
re-cipi-ent
pro-duct-iv-ity
turtle
Morocco
ferman
Maghreb-ian
algérien
stand-ard
systems
Nicolaï
Mouton
mauritani-en
Gotho-burg-ensis
socio-linguist-ique
plur-al
archiv-al
Arab-ian
drop-ped
dihāt
de-velop-ed
ṣuḥbat
kitāba
kitābat
com-mercial
eight-eenth
region
Senegal
mechan-ics
Maur-itan-ia
Ḥassān-iyya
circum-cision
cor-relation
labio-velar-ization
vowel
vowels
cert-ain
īggīw
series
in-tegrates
dur-ative
inter-dent-als
gen-itive
Tuareg
tălămut
talawmāyət
part-icular
part-icular-ly
con-diment
vill-age
bord-er
polit-ical
Wiesbaden
Uni-vers-idad
Geuthner
typo-logie
Maur-itanie
nomades
Maur-itan-ian
dia-lecto-logy
Sahar-iennes
Uni-vers-ity
de-scend-ants
NENA-speak-ing
speak-ing
origin-al
re-captured
in-habit-ants
ethnic
minor-it-ies
drama-tic
local
long-stand-ing
regions
Nineveh
settle-ments
Ṣəndor
Mandate
sub-stitut-ing
ortho-graphy
re-fer-enced
origin-ate
twenti-eth
typ-ic-al-ly
Hobrack
never-the-less
character-ist-ics
character-ist-ic
masc-uline
coffee
ex-clus-ive-ly
verb-al
re-ana-ly-se-d
simil-ar-ities
de-riv-ation
im-pera-tive
part-iciple
dis-ambi-gu-ation
dis-ambi-gu-a-ing
phen-omen-on
phen-omen-a
traktar
com-mun-ity
com-mun-ities
dis-prefer-red
ex-plan-ation
con-struction
wide-spread
us-ual-ly
region-al
Bulut
con-sider-ation
afro-asia-tici
Franco-Angeli
Phono-logie
Volks-kundliche
dia-lectes
dia-lecte
select-ed
dis-appear-ance
media
under-stand-able
public-ation
second-ary
e-ject-ive
re-volu-tion
re-strict-ive
Gasparini
mount-ain
mount-ains
yellow
label-ing
trad-ition-al-ly
currently
dia-chronic
}
\hyphenation{
affri-ca-te
affri-ca-tes
com-ple-ments
homo-phon-ous
start-ed
Meso-potam-ian
morpho-phono-logic-al-ly
morpho-phon-em-ic-s
Palestin-ian
re-present-ed
Ki-nubi
ḥawār-iyy-ūn
archa-ic-ity
fuel-ed
de-velop-ment
pros-od-ic
Arab-ic
in-duced
phono-logy
possess-um
possess-ive-s
templ-ate
spec-ial
espec-ial-ly
nat-ive
pass-ive
clause-s
potent-ial-ly
Lusignan
commun-ity
tobacco
posi-tion
Cushit-ic
Middle
with-in
re-finit-iz-ation
langu-age-s
langu-age
diction-ary
glossary
govern-ment
eight
counter-part
nomin-al
equi-valent
deont-ic
ana-ly-sis
Malt-ese
un-fortun-ate-ly
scient-if-ic
Catalan
Occitan
ḥammāl
cross-linguist-ic-al-ly
predic-ate
major-ity
ignor-ance
chrono-logy
south-western
mention-ed
borrow-ed
neg-ative
de-termin-er
European
under-mine
detail
Oxford
Socotra
numer-ous
spoken
villages
nomad-ic
Khuze-stan
Arama-ic
Persian
Ottoman
Ottomans
Azeri
rur-al
bi-lingual-ism
borrow-ing
prestig-ious
dia-lects
dia-lect
allo-phone
allo-phones
poss-ible
parallel
parallels
pattern
article
common-ly
respect-ive-ly
sem-antic
Moroccan
Martine
Harrassowitz
Grammatic-al-ization
grammatic-al-ization
Afro-asiatica
Afro-asiatic
continu-ation
Semit-istik
varieties
mono-phthong
mono-phthong-ized
col-loquial
pro-duct
document-ary
ex-ample-s
ex-ample
termin-ate
element-s
Aramaeo-grams
Centr-al
idioms
Arab-ic
Dadan-it-ic
sub-ordin-ator
Thamud-ic
difficult
common-ly
Revue
Bovingdon
under
century
attach
attached
bundle
graph-em-ic
graph-emes
cicada
contrast-ive
Corriente
Andalusi
Kossmann
morpho-logic-al
inter-action
dia-chroniques
islámica
occid-ent-al-ismo
dialecto-logie
Reichert
coloni-al
Milton
diphthong-al
linguist-ic
linguist-ics
affairs
differ-ent
phonetic-ally
kilo-metres
stabil-ization
develop-ments
in-vestig-ation
Jordan-ian
notice-able
level-ed
migrants
con-dition-al
certain-ly
general-ly
especial-ly
af-fric-ation
Jordan
counter-parts
com-plication
consider-ably
inter-dent-al
com-mun-ity
inter-locutors
com-pon-ent
region-al
socio-historical
society
simul-taneous
phon-em-ic
roman-ization
Classic-al
funeral
Kurmanji
pharyn-geal-ization
vocab-ulary
phon-et-ic
con-sonant
con-sonants
special-ized
latter
latters
in-itial
ident-ic-al
cor-relate
geo-graphic-al-ly
Öpengin
Kurd-ish
in-digen-ous
sunbul
Christ-ian
Christ-ians
sekin-în
fatala
in-tegration
dia-lect-al
Matras
morpho-logy
in-tens-ive
con-figur-ation
im-port-ant
com-plement
ḥaddād
e-merg-ence
Benjmamins
struct-ure
em-pir-ic-al
Orient-studien
Anatolia
American
vari-ation
Jastrow
Geoffrey
Yarshater
Ashtiany
Edmund
Mahnaz
En-cyclo-pædia
En-cyclo-paedia
En-cyclo-pedia
Leiden
dia-spora
soph-is-ic-ated
Sasan-ian
every-day
domin-ance
Con-stitu-tion-al
religi-ous
sever-al
Manfredi
re-lev-ance
re-cipi-ent
pro-duct-iv-ity
turtle
Morocco
ferman
Maghreb-ian
algérien
stand-ard
systems
Nicolaï
Mouton
mauritani-en
Gotho-burg-ensis
socio-linguist-ique
plur-al
archiv-al
Arab-ian
drop-ped
dihāt
de-velop-ed
ṣuḥbat
kitāba
kitābat
com-mercial
eight-eenth
region
Senegal
mechan-ics
Maur-itan-ia
Ḥassān-iyya
circum-cision
cor-relation
labio-velar-ization
vowel
vowels
cert-ain
īggīw
series
in-tegrates
dur-ative
inter-dent-als
gen-itive
Tuareg
tălămut
talawmāyət
part-icular
part-icular-ly
con-diment
vill-age
bord-er
polit-ical
Wiesbaden
Uni-vers-idad
Geuthner
typo-logie
Maur-itanie
nomades
Maur-itan-ian
dia-lecto-logy
Sahar-iennes
Uni-vers-ity
de-scend-ants
NENA-speak-ing
speak-ing
origin-al
re-captured
in-habit-ants
ethnic
minor-it-ies
drama-tic
local
long-stand-ing
regions
Nineveh
settle-ments
Ṣəndor
Mandate
sub-stitut-ing
ortho-graphy
re-fer-enced
origin-ate
twenti-eth
typ-ic-al-ly
Hobrack
never-the-less
character-ist-ics
character-ist-ic
masc-uline
coffee
ex-clus-ive-ly
verb-al
re-ana-ly-se-d
simil-ar-ities
de-riv-ation
im-pera-tive
part-iciple
dis-ambi-gu-ation
dis-ambi-gu-a-ing
phen-omen-on
phen-omen-a
traktar
com-mun-ity
com-mun-ities
dis-prefer-red
ex-plan-ation
con-struction
wide-spread
us-ual-ly
region-al
Bulut
con-sider-ation
afro-asia-tici
Franco-Angeli
Phono-logie
Volks-kundliche
dia-lectes
dia-lecte
select-ed
dis-appear-ance
media
under-stand-able
public-ation
second-ary
e-ject-ive
re-volu-tion
re-strict-ive
Gasparini
mount-ain
mount-ains
yellow
label-ing
trad-ition-al-ly
currently
dia-chronic
} 
  \togglepaper[1]%%chapternumber
}{}

\begin{document}
\maketitle 
 
 
\section{Introduction}

In this chapter, I discuss research that has examined the outcomes of \ili{Arabic} \isi{dialect contact} and the influence of contact on phonological change in spoken \ili{Arabic} varieties. This chapter also discusses the interface between phonology and phonetics, and the effect of contact on these areas of the linguistic system. Given space constraints, I discuss only a portion of the published work in these areas, giving some priority to recent doctoral dissertations that have contributed to this body of research. Further, I exclude work that has investigated the effects of contact on the morphology and syntax of \ili{Arabic} (e.g. \citealt{Al-WerEtAl2015}; \citealt{GafterHoresh2015}; Leddy-Cecere, this volume; Lucas, this volume; Manfredi, this volume). \ia{Leddy-Cecere, Thomas@Leddy-Cecere, Thomas}\ia{Lucas, Christopher@Lucas, Christopher}\ia{Manfredi, Stefano@Manfredi, Stefano}

  Although \ili{Arabic} sociolinguistics is an increasingly robust area of linguistic research, limiting my discussion to cases of contact-induced phonological and phonetic change is perhaps unsurprising, given the scholarly history of dialect-contact research and its place within sociolinguistics. Sociolinguistics has made great progress towards the goal of analyzing the full scope of variation in languages around the world. However, historically, and to some extent still today, examinations of variation and change in the realms of phonology and phonetics have been the meat and potatoes of sociolinguistic work. I would argue that this is true of \ili{Arabic} sociolinguistic work as well. 

  From Labov's (\citeyear{Labov1963}) early work on Martha’s Vineyard, phonetics and phonology have been at the heart of analyses of \isi{dialect contact}. As a result, much of what we know about \ili{Arabic} \isi{dialect contact} has stemmed from earlier foundational research on \isi{dialect contact} in the \ili{English}-speaking world. Within this work on \ili{English}, research by Milroy (\citeyear{Milroy1987}), Trudgill (\citeyear{Trudgill1986,Trudgill2004}), Britain (\citeyear{Britain2002}), and Britain \& Trudgill (\citeyear{BritainTrudgill2009}), among many others, has shown how \isi{dialect contact} often plays out, and how that contact influences language variation and change. 

  However, research on \ili{Arabic} has moved beyond simply testing the hypotheses put forward by scholars of \ili{English} \isi{dialect contact}, playing its own role in refining sociolinguistic theory. Notably, \ili{Arabic} sociolinguistics has refined our understanding of \isi{diglossia} \citep{Ferguson1959}. \citet{Ibrahim1986} and \citet{Haeri2000} have reoriented our understanding of \ili{Arabic} \isi{diglossia} from Ferguson’s High–Low dichotomy to one that draws on locally meaningful understandings of linguistic \isi{prestige}. In doing so, this work has moved our discussion away from analyzing \ili{Arabic} through the lens of “standard” or “nonstandard” varieties or variants, setting the stage for decades of research that has examined contact-induced change in \ili{Arabic} varieties.


 
 \subsection{The potential limitations of borrowing and imposition for Arabic dialect contact research}


Before moving on to a discussion of a number of specific cases of \ili{Arabic} \isi{dialect contact}, I briefly address the potential limitations of Van Coetsem’s (\citeyear{VanCoetsem1988,VanCoetsem2000}) framework for discussions of \isi{dialect contact}, as opposed to language contact. After discussing Van Coetsem’s approach, I shift my focus to discuss \ili{Arabic} \isi{dialect contact} through a theoretical lens that has proven productive in earlier sociolinguistic work (\citealt{Trudgill1986,Trudgill2004}). 

  In analyzing phonological change as a result of \isi{dialect contact}, Van Coetsem’s framework presents a number of possible challenges. One specific issue is that in many cases, a clear distinction between the borrowing or \isi{imposition} of linguistic forms is challenging to establish in cases of \ili{Arabic} \isi{dialect contact}. Scholars may encounter challenges in attempting to assert the agentivity of the \isi{recipient language} in making a case for the borrowing of, for example, aspects of a dialect’s phonology into the phonology of another dialect. Asserting the agentivity of the \isi{source language} in making the case for \isi{imposition} is similarly challenging. These challenges \isi{stem} from the cognitive orientation of Van Coetsem’s framework, which, as Lucas (\citeyear[521]{Lucas2015}) notes, is not based on social realities or variation in the power and \isi{prestige} that a given dialect or language may hold. 

  The approach that many scholars within sociolinguistics and allied fields like linguistic anthropology have taken is, in contrast, inherently social. We concern ourselves with the social life of language, and although we do not discredit cognitive approaches to language acquisition and use, in much of the work on \isi{dialect contact}, we have foregrounded \isi{social factors} in our analyses of \isi{language change}. However, it is worth noting that within sociolinguistic research on \isi{second dialect acquisition}, researchers have highlighted the role of \isi{social factors}, as well as the constraints placed on acquisition by the linguistic system (e.g. \citealt{Nycz2013,Nycz2016}).  

  With the above discussion in mind, I argue that Van Coetsem’s framework is less readily applicable to the cases that I describe in this chapter. Instead, I suggest that outcomes of \ili{Arabic} \isi{dialect contact} are better analyzed through the framework advocated for within sociolinguistics. It is to that framework that I now turn.  

  As \citet{Trudgill2004} notes in discussing \isi{new dialect} \isi{formation}, \isi{dialect contact} often progresses in stages. One of the earliest stages in this process is \textsc{leveling} \citep[83]{Trudgill2004}, which results in the reduction of forms from a given dialect. These forms may be, but do not have to be, socially marked, e.g. \isi{affrication} of /k/ to [č] in certain \ili{Arabic} dialects. Most importantly for Trudgill, during \isi{leveling} certain variants of a given feature will supplant others \citep[85]{Trudgill2004}. As a result, forms that are socially marked may be leveled out, while unmarked forms may survive even if they were not a majority variant. In those cases where socially marked forms are present, they are often reduced across generations. Trudgill also describes processes of \isi{interdialect} development, where forms arise out of the interaction between dialects, such as \isi{reallocation}, where surviving forms are reallocated in some way, and \isi{focusing}, whereby a new variety born out of contact begins to stabilize \citep{Trudgill2004}. 

  What I feel that this framework offers in discussions of \ili{Arabic} \isi{dialect contact} is an acknowledgement of the social issues that may influence linguistic change, especially in situations of contact. In the remainder of this chapter I discuss cases of contact-induced change in \ili{Arabic} varieties. In doing so, I draw on sociolinguistic understandings of how contact-induced changes take hold and progress. 

\section{Contact-induced changes in the phonology of Arabic dialects}

When discussing \ili{Arabic} \isi{dialect contact}, a brief discussion of the typology of \ili{Arabic} is useful, as it provides a shared lexicon for discussing the outcomes of contact. \citet{Cadora1992} offers an ecolinguistic taxonomy of \ili{Arabic}, describing a continuum of \ili{Arabic} varieties containing linguistic features ranging from what he describes as \ili{Bedouin} in provenance, to those that can be considered sedentary. In presenting a related contrast, Cadora describes features that situate dialects as being urban versus those that are rural. 

  However, what Cadora offers is not a hard and fast classification of \ili{Arabic} varieties. Instead, his typology highlights linguistic features that typically group together within dialect types, providing a way to conceptualize the similarities and differences across these varieties. Importantly for this chapter, sites of contact between \ili{Arabic} dialects are also often sites of contact between \textit{types} of dialects as well. 

  In my own work on \ili{Palestinian} \ili{Arabic} this has been the case, with Gaza City offering one example of contact between different \ili{Palestinian} \ili{Arabic} dialects. Today, the dialect of \ili{Gaza} City has both \ili{Bedouin} and urban sedentary features, dialect types that likely came into contact in Gaza as a result of Palestinian refugee migration (see de Jong's \citeyear{DeJong2000} discussion of \ili{Gaza} City). This contact is undeniable given Gaza’s current demographic reality, which suggests that its population is roughly 70\% refugee.\footnote{This figure has been reported by the United Nations Relief Works Agency for Palestinian Refugees but only reflects refugees that have actually registered with the U.N. (\url{https://www.unrwa.org/where-we-work/gaza-strip}, accessed 07/01/2020). Other estimates place the percentage of Gaza’s population that are refugees as closer to 80\%.}  It is also unsurprising given that Gaza has long been a site of contact. This history of contact has resulted in a city dialect that looks different than other urban \ili{Palestinian} varieties spoken in major cities like {Jerusalem} or Nablus. 

  The above example serves as a way of framing the linguistic discussion of contact-induced phonological change provided below. I begin by covering documented consonantal changes that have grown out of contact, before moving on to vocalic changes and the need for additional research in this area as studies of \ili{Arabic} contact move forward. 


 
 \subsection{Consonantal changes}


One of the most widely discussed linguistic features within work on \ili{Arabic} \isi{dialect contact} has been the variable realization of the voiceless uvular stop /q/. Motivation for the scholarly interest in /q/ likely stems from a number of factors. First, the \isi{phoneme} has a wide range of dialectal variation, with dialectal realizations including a true voiceless uvular [q], as well as [k, ɡ, ʔ] and an additional [k] variant articulated between a velar and uvular \citep{Shahin2011}. Second, interest in /q/ is also likely due to the high social salience of its variation in many \ili{Arabic}-speaking communities (see \citealt{Hachimi2012}; \citealt{CotterHoresh2015}). 

  The result is that /q/ has been one of the most heavily studied features in \ili{Arabic} sociolinguistics. Variation and change in /q/ has been discussed in a number of different communities throughout the \ili{Arabic} speaking world, including: Palestine (\citealt{AbdEl-Jawad1987}; \citealt{Al-Shareef2002}; \citealt{CotterHoresh2015}; \citealt{Cotter2016}); Egypt \citep{Haeri1997}; Iraq (\citealt{Blanc1964,Abu-Haidar1991});\footnote{Both Abu-Haidar and Blanc’s analyses are dialectological and descriptive in scope, however both ultimately discuss what appear to be processes of change taking place for /q/ within what Blanc termed “communal” (i.e. religio-sectarian) varieties of \ili{Arabic} in \ili{Baghdad}.}  Jordan (\citealt{AbdEl-Jawad1981}; \citealt{Al-Wer2007}; \citealt{Al-WerHerin2011}); Morocco (\citealt{Hachimi2007,Hachimi2012}); and Bahrain \citep{Holes1987}, among others. 

  What these cases suggest are robust processes of linguistic change in the realization of /q/ coming as a result of factors such as migration and \isi{dialect contact}. While the social patterning of these changes (e.g. stratified along age, \isi{gender}, or sectarian lines) have been as diverse as the communities in which /q/ has been analyzed, across these contexts we see regular patterns of change in /q/ over time. 

  Taking the case investigated by \citet{Cotter2016} as an example, we can see how patterns of change in /q/ may progress over time. In the speech of Jaffa Palestinian refugees in the Gaza Strip, \citet{Cotter2016} showed that across three generations Jaffa refugees in \ili{Gaza} showed progressively lower use of their traditional [ʔ] realization of /q/, instead beginning to favor the voiced velar [ɡ] variant that is common in \ili{Gaza} City \ili{Arabic}. Within the oldest generation of this community, Jaffa refugees showed near categorical retention of the glottal variant, and little rudimentary \isi{leveling}. However, the second generation of Jaffa refugees showed substantial variability between [ʔ] and [ɡ], while in the third generation in the study, speakers showed higher rates of usage of the [ɡ] variant that is native to the \ili{Gaza} City dialect. 

However, as \citet{CotterHoresh2015} discuss, variability in /q/ is often situated within broader \isi{identity} projects that speakers and communities have undertaken. It is important then that analyses of \ili{Arabic} \isi{dialect contact} also consider the broader ethnographic context in which this contact takes place. 

Another area of interest for researchers examining \isi{dialect contact} has been the interdental fricatives /θ, ð, ð̣/. Across \ili{Arabic} varieties, these phonemes quite often vary between realizations as true interdental fricatives [θ, ð, ð̣] and their stop counterparts [t, d, ḍ] (\citealt{Al-Wer1997,Al-Wer2003daad,Al-Wer2011}). In addition to descriptive work that has documented the realization of the interdentals across \ili{Arabic} varieties, they have also been examined as sociolinguistic variables in cases of \isi{dialect contact}. 

  For example, Holes (\citeyear{Holes1987,Holes1995}) investigated sociolinguistic variation in the realization of /θ, ð, ð̣/ roughly split along sectarian lines in the speech of Arab and Baḥārna speakers in \ili{Bahrain}. In Bahrain, in the dialect of Sunni Arabs these phonemes are traditionally pronounced as [θ, ð, ð̣], whereas in the dialect of Shi’i speakers they are pronounced as [f, d, ḍ]. Holes (\citeyear{Holes1995}: 275) details that in the speech of young literate speakers in Manama, intercommunal dialect realizations of the interdentals have emerged that are generally centered on the Sunni Arab realizations of these phonemes. More recently, \citet{Al-Essa2008} examined the interdentals in the speech of \ili{Najdi} \ili{Arabic} speakers living in Jeddah, \isi{Saudi Arabia}, an Urban Hijazi \ili{Arabic} dialect area. Although \ili{Najdi} \ili{Arabic} typically retains the interdental realization of these phonemes, Al-Essa concluded that degree of contact with Urban Hijazi speakers was a significant factor influencing whether \ili{Najdi} speakers adopted the stop realizations common in Urban Hijazi \ili{Arabic}. 

Additionally, \citet{Alghamdi2014} investigated the interdentals through the lens of migration and contact in the Saudi Arabian city of \isi{Mecca}. Alghamdi describes what may be the beginning of a change from the traditional interdental realization of these phonemes in the direction of their stop counterparts. As Alghamdi (\citeyear{Alghamdi2014}: 112) notes, if it is the case that an incipient change in the interdentals exists in \ili{Mecca}, the results of her study suggest that female speakers may be leading this change. This finding supports earlier sociolinguistic work, which has highlighted that female speakers are often at the vanguard of linguistic change. 

  Change in the interdentals has also been examined as part of \isi{new dialect} \isi{formation} in the Jordanian capital of \ili{Amman}. As \citet{Al-Wer2007} describes (see also Al-Wer, this volume),\ia{Al-Wer, Enam@Al-Wer, Enam} \ili{Amman} \ili{Arabic} has grown out of contact between speakers of two different dialect types: urban \ili{Palestinian} and traditional \ili{Jordanian} varieties, which differ in their realizations of the interdentals. Urban \ili{Palestinian} dialects typically favor non-interdental realizations [t, d, ḍ], while, in contrast, traditional \ili{Jordanian} dialects retain the interdentals [θ, ð, ð̣]. Al-Wer describes the case of the interdentals in \ili{Amman} as a process of \isi{focusing} \citep{Trudgill2004} that has arisen out of contact. In Trudgill’s terms (drawing on \citealt{LePageTabouret-Keller1985}), \isi{focusing} is one part of the process of new-dialect \isi{formation}, whereby features of input dialects are leveled and stability emerges, resulting in new shared linguistic norms. Al-Wer describes that, in \ili{Amman}, \isi{focusing} of the interdentals in the direction of their stop counterparts [t, d, ḍ] has taken place (\citealt{Al-Wer2007}: 66). In addition, Al-Wer notes that, as a result of contact, \ili{Amman} \ili{Arabic} has also focused towards the common \ili{Palestinian} [ž] realization of /ǧ/ at the expense of the traditional \ili{Jordanian} [ǧ] (\citealt{Al-Wer2007}: 66).  

  In addition to the work by \citet{Al-Essa2008} and \citet{Alghamdi2014} discussed above, a more recent case of contact-induced change in \isi{Saudi Arabia} has been identified: the voiced lateral fricative [ɮˤ] realization of 〈\kern 0.75pt{\arabscript{ض}}〉. \citet{Al-WerAl-Qahtani2016} investigate /ɮˤ/ as a variable in the dialect of Tihāmat Qaḥtān. What this work shows is that in the Tihāmat Qaḥtān variety, the lateral [ɮˤ] represents a conservative, traditional variant of the \isi{phoneme}, whereas the voiced interdental fricative [ð̣] represents the innovative variant. As a result of \isi{dialect contact}, \citet{Al-WerAl-Qahtani2016} describe an intergenerational process of change towards the voiced \isi{emphatic} interdental [ð̣], with use of the historic [ɮˤ] variant receding over time.

  Another area of interest in \isi{dialect contact} research has been \isi{affrication}. As descriptive work has shown, \isi{affrication} of certain phonemes, notably /k/ in the direction of [č], is common in \ili{Arabic} dialects. As an example of this process, \citet{Shahin2011} notes that in rural varieties of \ili{Palestinian} \ili{Arabic}, /k/ palatalizes to become an affricate [č] (e.g. \textit{čīfak} ‘how are you (\textsc{sg.m})?’ < \textit{kīfak}). While typologically this \isi{affrication} is common, processes of \isi{affrication} or de-\isi{affrication} have also been noted as the outcome of contact. 

  \citet{Al-Essa2008} investigated \isi{affrication} of /k/ and /g/ in the speech of \ili{Najdi} \ili{Arabic} migrants in Jeddah, and found that the \isi{affrication} that is a common feature of this variety had been almost completely undone in this migrant community. Examining this change in light of \isi{dialect contact}, Al-Essa concludes that this deaffrication represents the \isi{leveling} out of marked regional dialect forms as a result of contact (\citealt{Trudgill1986}; \citealt{KerswillWilliams2000}). More recently, \citet{Al-WerEtAl2015} note that the \isi{conditional}, root-based distribution of the affricate [č] for /k/ in the Sult variety of \ili{Arabic} in Jordan, which, although it has receded (\citealt{Al-Wer1991}), now interacts with other innovative features in Sult that show potential stratification along religious lines. 

  Elsewhere in Jordan, notably in \ili{Amman}, \citet{Al-Wer2007} describes the \isi{leveling} of the affricate [č] across generations. The city dialect that has emerged in \ili{Amman}, which has Sulti \ili{Arabic} as one of its input varieties, underwent rudimentary \isi{leveling} \citep{Trudgill2004} within the first generation. This \isi{leveling} resulted in the loss of this affricate variant of /k/ in the speech of Sulti migrants. In this case, Al-Wer describes the deaffrication of [č] as stemming from its status as a marked feature of Horani \ili{Arabic} varieties like that of Sult. This marked status makes it a primary candidate for the kinds of \isi{leveling} that sociolinguists have identified in other cases of contact. 


 
 \subsection{Vocalic changes}


In general, the \ili{Arabic} vocalic system remains understudied within research on \ili{Arabic} varieties. However, multiple cases of change linked to \isi{dialect contact} have been identified. One of the most well studied cases of contact-induced vocalic change in \ili{Arabic} is perhaps better thought of as a morphophonological change: the \ili{Arabic} feminine \isi{gender} marker. The feminine \isi{gender} marker is a word final vocalic morpheme that is realized variably across \ili{Arabic} varieties. The realization of this vowel varies from an unraised [a] to [æ, ɛ, e], or even as high as [i] (e.g. \citealt{Al-Wer2007}; \citealt{Naïm2011}; \citealt{Shahin2011}; \citealt{Woidich2011}). 

Even within one region, the full range of variation in this morpheme can be seen. Taking the Levant as an example, the \ili{Lebanese} capital, \ili{Beirut}, is known for raising this vowel to [e] or even [i] \citep{Naïm2011}. The Syrian capital, \ili{Damascus}, is known to raise to [e] \citep{Lentin2011Damascus}. Urban \ili{Palestinian} (\citealt{Rosenhouse2011}; \citealt{Shahin2011}) is also often described as raising to [e], while \ili{Amman} (\citealt{Al-Wer2007}) raises this vowel to [ɛ]. These city varieties can be contrasted with, for instance, the variety of \ili{Cairo} \citep{Woidich2011}, which does not raise this vowel, leaving it as [a]. 

  This morpheme is particularly interesting within a discussion of \isi{dialect contact} because raising of this vowel is phonologically conditioned. The phonological factors that constrain raising vary across dialects, with urban \ili{Levantine} \ili{Arabic} (e.g. Syria, Palestine, Lebanon) providing one example of these factors. In urban \ili{Levantine}, the following rules constrain the raising of this vowel (\citealt{Grotzfeld1980}: 181; \citealt{Levin1994}: 44-45; \citealt{Al-Wer2007}: 68):
\begin{enumerate}
    \item The default realization of the vowel is raised: [e]
    \item The vowel is unraised, realized as [a], when:
    \begin{enumerate}
        \item   it occurs after back consonants (i.e. \isi{pharyngeal}, glottal, post-velar, \isi{emphatic}/\isi{pharyngealized}): /ḥ, ʕ, ʔ, h, ṣ, ḍ/ð̣, ḫ, ɣ, q/;
        \item   it occurs after /r/, but only when preceding /r/ there is no   high front vowel. In cases where a high front vowel does   precede /r/, raising is allowed, e.g. [kbi:re] ‘big (\textsc{f})’.
    \end{enumerate}
\end{enumerate}

Below I provide two specific documented examples of contact and change in the feminine \isi{gender} marker. First, \citet{CotterHoresh2015} investigated change in the feminine \isi{gender} marker in the speech of refugees originally from the Palestinian city of Jaffa who now live as refugees in the Gaza Strip. This sample included both speakers who were expelled from Jaffa after the creation of the state of Israel in 1948 and their descendants. Their traditional urban \ili{Palestinian} dialect (\citealt{Horesh2000}; \citealt{Shahin2011}) is one that raises the feminine \isi{gender} marker to [e], subject to the phonological conditioning mentioned above. In contrast, based on the available dialectological information, the dialect of \ili{Gaza} City does not raise this vowel \citep{Bergsträßer1915}. 

\citet{CotterHoresh2015} highlight a process of contact-induced change that has taken place in this community. Across generations, the realization of this vowel appears to be lowering and backing, moving from [e] in the direction of [a]. The result is that younger Jaffa refugee speakers realize the vowel closer to the [a] common in \ili{Gaza} City. This type of change is perhaps unsurprising in a city like Gaza, given that the population of Gaza is overwhelmingly comprised of refugees, including large communities who are of [a] dialect types for this feature. This diversity and the high numbers of refugees in Gaza means that the city, and the territory generally, is a site where many dialects of \ili{Palestinian} \ili{Arabic} are in intimate contact. What remains to be determined is whether or not new linguistic norms are emerging in the dialect of \ili{Gaza} City more generally as a result of this contact.

One other case, which is discussed in more detail by Al-Wer (this volume),\ia{Al-Wer, Enam@Al-Wer, Enam} provides a succinct example of the intersection between phonetics, phonology, and \ili{Arabic} \isi{dialect contact}. In discussing the \isi{formation} of \ili{Amman} \ili{Arabic}, \citet{Al-Wer2007} notes the centrality of vocalic change to the \isi{formation} of the dialect. The feminine \isi{gender} marker represents one feature that has helped to define the variety of \ili{Amman}. 

As \citet{Al-Wer2007} describes, through contact between \ili{Palestinian} and \ili{Jordanian} \ili{Arabic} dialects in \ili{Amman}, the realization of the feminine \isi{gender} marker has focused on [ɛ], the indigenous \ili{Jordanian} realization (as in e.g. the Horani dialect of Sult, see \citealt{Herin2014salt}). However, although \ili{Amman} \ili{Arabic} has focused on the \ili{Jordanian} phonetic realization of this vowel, it has retained urban \ili{Palestinian} phonology, which blocks raising in the environment of back consonants as defined above (\citealt{Al-Wer2007}: 69). This is less restrictive than in Horani \ili{Arabic}, where raising is also blocked in the environment of velar consonants such as /k/ and the labiovelar /w/ (\citealt{Al-WerEtAl2015}: 77). 

  Finally, I mention one other case of vocalic change that has been documented as an outcome of \isi{dialect contact}: the \isi{diphthongs} [ay] and [aw]. \citet{Alghamdi2014} investigated \isi{monopthongization} of the traditional \ili{Arabic} \isi{diphthongs} /ay/ and /aw/ in the speech of {Ghamdi} migrants in \isi{Mecca}. Alghamdi found that the \isi{diphthongs} common in the dialect spoken by this migrant community were monophthongizing, reflecting a change towards the norms of \ili{Mecca} \ili{Arabic}, which lacks \isi{diphthongs}. Alghamdi’s analysis of the \isi{diphthongs} provides an example of dialect \isi{leveling} borne out of contact, noting two additional aspects of this variable in the speech of this migrant community: i) Alghamdi describes the high degree of social salience that the \isi{diphthongs} have in this community and their possible stigmatization in \isi{Mecca}, and ii) that retention of the \isi{diphthongs} is uncommon in Saudi \ili{Arabic}, making the \ili{Ghamdi} realization a minority realization in \isi{Saudi Arabia} generally. These two facts create an environment conducive to change.

\section{Conclusion}

In examining \ili{Arabic} \isi{dialect contact}, a growing body of research highlights that the phonology and phonetics of \ili{Arabic} represent rich sites for linguistic change. As the examples that I have provided throughout this chapter, and those discussed elsewhere throughout this volume suggest, we can identify a number of cases where \isi{dialect contact} has influenced the directionality and extent of change in \ili{Arabic} dialects. With the findings of this selection of work in mind, a number of areas remain open for \isi{future} investigation. 

  Perhaps the most pressing of these is the reality that, although I have highlighted work here that investigates vocalic change, the vocalic system of \ili{Arabic} varieties remains drastically understudied. Although phonetic research on the vocalic system of \ili{Arabic} varieties continues to grow (see e.g. \citealt{HassanHeselwood2011}; \citealt{KhattabAl-Tamimi2014}; \citealt{Al-TamimiKhattab2015}), we know little about sociophonetic changes that may take place in cases of contact like those discussed in this chapter. Given the scope of \isi{dialect contact} in the \ili{Arabic}-speaking world, much of which has come as a result of mass migration throughout the region, investigating the potential for processes such as vocalic chain-shifting (\citealt{Al-Wer2007}) represents an important next step for research on language variation and change in \ili{Arabic}. I would argue that more robust investigation of vocalic change in \ili{Arabic} dialects represents a pressing area of concern for \ili{Arabic} sociolinguistics. 

  In addition, examples like the feminine \isi{gender} marker in \ili{Amman} (\citealt{Al-Wer2007}) open the door for \isi{future} work that investigates the potential for blending of the phonetics and phonology of different \ili{Arabic} varieties as a result of contact. Although \ili{Amman} is a somewhat different case, given that it represents an example of \isi{new dialect} \isi{formation}, a close examination of phonetics and phonology together in contact situations will provide us with an opportunity to examine how dialect \isi{focusing} and \isi{leveling} takes place, and how the linguistic systems of multiple different \ili{Arabic} varieties interact and regularize through contact. 

  Additional research that looks more closely at change in the vocalic system of \ili{Arabic} dialects will go a long way towards enriching the depth of \ili{Arabic} sociolinguistic research. This is especially true of work that examines cases of \isi{dialect contact}. However, beyond sociolinguistics, a closer examination of the vocalic system will contribute to the description and documentation of \ili{Arabic} dialects, which will further enrich linguistic research that investigates the varieties of \ili{Arabic} spoken around the world. 

\section*{Further reading}

As I have discussed throughout this chapter, Al-Wer's (\citeyear{Al-Wer2007}) work details a number of aspects of how the dialect of Jordan’s capital, \ili{Amman}, emerged as a result of contact between \ili{Jordanian} and \ili{Palestinian} dialects. It offers a clear picture of how contact has played out in \ili{Amman} with respect to a number of linguistic features, and how the city’s dialect emerged over successive generations. 

Cotter \& Horesh’s (\citeyear{CotterHoresh2015}) \isi{article} examines contact in the Gaza Strip, one of the more understudied areas within \ili{Arabic} linguistic research. The \isi{article} draws on sociolinguistic fieldwork conducted in Gaza City, as well Jaffa and the West Bank, to analyze change in three specific features of \ili{Palestinian} \ili{Arabic}.

Holes' (\citeyear{Holes1987}) work on \ili{Bahrain} provided one of the early accounts of sociolinguistic variation and change in \ili{Arabic}. In addition, this work provides a clear example of variation in \ili{Arabic} that has been stratified on sectarian, or religious, lines. 

\section*{Acknowledgements}

I would like to thank Christopher Lucas and the anonymous reviewer for their comments on this chapter. In addition, I would like to thank Uri Horesh and Enam Al-Wer for their feedback, as well as the attendees of the \ili{Arabic} and Contact-Induced Change workshop at the 23rd International Conference on Historical Linguistics for their feedback on my own research as it appears in this chapter. 



\sloppy
\printbibliography[heading=subbibliography,notkeyword=this]
\end{document}
