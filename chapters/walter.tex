\documentclass[output=paper]{langsci/langscibook}
\author{Mary Ann Walter\affiliation{Middle East Technical University, Northern Cyprus Campus}}
\title{Cypriot Maronite Arabic}\abstract{Cypriot Maronite Arabic is a severely endangered variety that has been in intensive language contact with Greek for approximately a millennium. It presents an interesting case of a language with extensive contact effects which are largely limited to the phonological domain.}


\begin{document}
\maketitle

\section{Current state and historical development of Cypriot Maronite Arabic}

Cypriot Maronite \ili{Arabic} (CyA) is a minority language spoken by a small community on the island of {Cyprus}. Although essentially moribund, it is currently the focus of preservation and revitalization efforts.


\subsection{\label{bkm:Ref525121157}Historical development of Cypriot Maronite Arabic}

The time of arrival of this community of \ili{Arabic} speakers to {Cyprus} is unknown. The island was occupied by an Arab garrison subsequent to Muʕāwiya’s invasion of 649 CE,\ia{Muʕāwiya@Muʕāwiya} but the garrison was then removed and, presumably, the \ili{Arabic} speakers left as well. More likely, a permanent presence dates back to the population movements of the ninth and tenth centuries during disruptions to Byzantine rule.\footnote{See §\ref{bkm:Ref525122525} for a discussion of where the CyA-speaking community originated from and the dialectological affiliation of this variety of \ili{Arabic}.} Subsequent waves of Arab emigration to {Cyprus} are documented during the early crusading period. Such movements also quite likely took place during {Lusignan} (French crusader) rule in {Cyprus} (1192–1489), for some portion of which the Anatolian city of Adana, where \ili{Arabic} is still widely spoken (see Procházka, this volume),\ia{Procházka, Stephan@Procházka, Stephan} was also held by {Lusignan} rulers. Speakers of not only \ili{Arabic} but a locally distinct version of \ili{Arabic} in {Cyprus} are mentioned by Arab historians beginning in the thirteenth century, thereby providing a \textit{terminus} \textit{ad} \textit{quem} to its dialectal development \citep{Borg2004}.


As fellow communicants in the Catholic church, the Maronite community was granted certain privileges of independent worship during the {Lusignan} period, which were later lost during Venetian (1489–1571) and Ottoman rule (1571–1878), at which time some retaliation occurred on the part of the Orthodox community \citep{Gulle2016}. After the Ottoman conquest of {Cyprus} in 1571, the Maronite community was at first placed under the administration of the Orthodox bishop, but regained religious autonomy shortly thereafter.

The number of Maronite villages underwent a steady decline during the Ottoman period, from over thirty to only five at the time of British occupation of the island in 1878 (\citealt{BaiderKariolemou2015}; though it is unclear if this is associated with any actual population decline). The remaining five villages are all located in the northwestern area of the island. However, as of the twentieth century at least, only one of them was home to speakers of CyA, the others having linguistically assimilated to \ili{Cypriot Greek} entirely. The CyA-speaking village is {Kormakiti}(s) (also known as Kormacit and Koruçam in CyA and in \ili{Turkish}, respectively).

Both the Cypriot liberation struggle of the 1950s against the British, and the years after independence was attained in 1960, saw increased communal conflict between the Turkish and Greek communities on the island. This period witnessed increasing separation of communities, as Turkish Cypriots withdrew into ethnic enclaves, and  culminated in the 1974 conflict between Greece-sponsored coup plotters, military forces of Turkey, and local Cypriots on various sides, the result of which was a \textit{de} \textit{facto} division of the island between the Republic of Cyprus-controlled territory in the south, which was majority Greek Orthodox and \ili{Greek}-speaking, and the Muslim and \ili{Turkish}-speaking northern part of the island. This northern area subsequently declared independence, but remains unrecognized by any other country except the Republic of Turkey to this day.

It is important to note that the {relative} geographical separation between Greek Cypriots and Turkish Cypriots dates only from this recent period, as refugees sought safety within their own communities. This entailed a radical change in the social circumstances of CyA speakers, who moved to the capital city of Nicosia essentially \textit{en} \textit{masse}. Thus, they went from living in a Maronite village in which community life could be conducted in CyA, to being a tiny percentage of a large urban population. Not only that, but the pre-1974 population surrounding the CyA-speaking Maronite village of {Kormakiti} was composed of \ili{Greek} speakers, whereas the current local population around the village is comprised of \ili{Turkish} speakers (many of whom also know \ili{Greek}, but no longer use it as a language of public life).


Since 1974 the permanent population of the village of {Kormakiti} has amounted to at most a couple of hundred residents, with the rest of the Maronite community residing primarily in the capital city Nicosia. The Maronite community has occupied a special place in Cypriot society, as for three decades they alone had the ability to freely cross the UN-monitored “Green Line” (buffer zone) dividing the island. Thus connections with the village have been maintained throughout this period, and weekend visits are common. Since 2003 the line has been crossable for all Cypriots.

\subsection{Current situation of Cypriot Maronite Arabic}
The Cypriot Maronite community currently numbers roughly 5,000 individuals. However, only approximately one thousand are CyA speakers (estimates range from 900 to perhaps 1300; \citealt{CouncilofEurope2017}).


All CyA speakers are bilingual in \ili{Cypriot Greek}, with \ili{Greek} as their dominant language, and currently living in a heavily \ili{Greek}-dominant urban area. There are currently no fluent native speakers under the age of thirty. Due to these factors, the CyA language was designated as severely {endangered} by UNESCO in 2002.

However, the accession of the Republic of {Cyprus} to the {European Union} in 2004 has led to an influx of both institutional and financial support for CyA. In its 2004 initial report on its implementation of the European Charter for Regional or Minority Languages (ECRML), which it ratified in 2002, the Republic of {Cyprus} declared \ili{Armenian} as such a language in {Cyprus}. Although CyA was explicitly excluded as being “only” a dialect and therefore in no need of protection, this formulation was not accepted by ECRML, and CyA was thenceforth officially recognized as a minority language of {Cyprus} as well. Since 2008 Maronites have been officially recognized as a separate community within {Cyprus}, and are no longer required to identify themselves as Greek Cypriots (or Turkish Cypriots) on government documents.

The change in designation of the Cypriot Maronites as a linguistic as well as religious minority community led to associated changes in the linguistic rights legally accorded to them. After decades of waiting, one state school in Nicosia is now designated as Maronite and offers optional after-school classes in CyA for its approximately 100 Maronite students, the majority of whom have now joined the classes. Adults may also study CyA now at the new community center. Funding was also made available for a one-to-two week summer language immersion camp for Maronite youth in {Kormakiti} village, attendance at which has risen to approximately 100. For the first time, training seminars for teachers have also been organized, concomitantly with codification efforts towards a written version of CyA. Sporadic writing in CyA has been carried out using the Greek alphabet. (See the community websites in the \nameref{FR} section at the end of this chapter).


Outside the government, there is also an NGO \textit{Hki fi Sanna} (‘speak in our language’) with the goal of promoting CyA use. Usage remains community- and home-based, as \ili{Standard Greek} (and \ili{English}) is the language of written and broadcast media. The {Cyprus} Center of the Peace Research Institute Oslo (PRIO) has undertaken a project entitled \textit{The} \textit{protection} \textit{and} \textit{revival} \textit{of} \textit{Cypriot} \textit{Maronite} \textit{Arabic}. The scope of the project included a variety of community activities, as well as meetings with \ili{Sami} (Norway) community members for sharing revitalization strategies, described in the resulting publication (\citealt{PRIO2009}). Finally, a project at the University of {Cyprus} titled \textit{The} \textit{creation} \textit{of} \textit{an} \textit{archive} \textit{of} \textit{oral} \textit{tradition} \textit{for} \textit{Cypriot} \textit{Maronite} \textit{Arabic} is currently underway under the supervision of Dr. Marilena Karyolemou, though with no web presence or published deliverables to date. There is thus some reason for optimism regarding the {future} of CyA.

\section{\label{bkm:Ref525122525}Contact languages}


CyA has undergone {intensive} language contact with \ili{Cypriot Greek} for the entirety of its presence in {Cyprus}, which may extend to a millennium (see §\ref{bkm:Ref525121157}). This contact has intensified since the removal of the population from the traditionally Maronite and CyA-speaking village of {Kormakiti} to the capital city, Nicosia.


This move has also resulted in a concomitantly larger social role for \ili{Standard Greek}. {Cyprus} is a diglossic society in which \ili{Cypriot Greek} coexists with \ili{Standard Greek}, the language of education and formal domains.\footnote{Some in fact refer to triglossia, encompassing \ili{Standard Greek}, koinéized \ili{Cypriot Greek}, and various other local varieties, with the island-wide koine taking a mesolectal position \citep{Arvaniti2010}.} In moving to Nicosia, the children of the community also began attending schools with Greek Cypriot children, rather than their own village schools. Only in the last few years has a primary school been designated specifically for Maronite children. Most of them still attend other schools, and the Maronite school is in any case also ({Standard}-)\ili{Greek}-medium and follows the same national curriculum (with the addition of optional after-school weekly CyA language classes).

Therefore, the influence of \ili{Greek} has increased radically through contact with Greek classmates and neighbors, as well as intermarriages with Greek Cypriots and Maronites from other, non-CyA speaking villages. Such situations are common due to the small size of the Maronite community, and typically CyA is not used in these households.


In comparison, contact with \ili{Turkish} has been limited. Although remaining residents of {Kormakiti} are now surrounded by \ili{Turkish} speakers, the village remains quite set apart socially, to the extent that all water supplies are trucked in rather than plumbing systems being shared. In Borg's (\citeyear{Borg1985}) texts, speakers do mention using \ili{Turkish} with some speakers employed as farm workers, however. Contact with \ili{Turkish} speakers in Nicosia is, of course, rare.

{Cyprus} is “double-diglossic”: the same situation as with \ili{Cypriot} and \ili{Standard Greek} holds also with respect to Cypriot and {Standard} \ili{Turkish}. To the extent that contact with \ili{Turkish} does occur, it is with Cypriot \ili{Turkish} rather than {Standard} \ili{Turkish}, unlike \ili{Greek}, where both varieties are prominent in the lives of CyA speakers.

There is next to no contact with other varieties of \ili{Arabic}. The Maronite clergy in {Cyprus} often come from Lebanon, and some intermarriage occurred in the more distant past between the Cypriot and Lebanese Maronite communities, but this no longer occurs. \citet{Roth2004} refers to the “double {minoritization}” of CyA speakers with respect to both the Cypriot context and the wider Arabophone context – in both, their speech variety is considered deviant and unintelligible.


While early research on CyA identifies it as a \ili{Levantine} variety of \ili{Arabic} \citep{Tsiapera1969}, Borg (\citeyear{Borg1985,Borg2004}) argues strongly for an \ili{Anatolian} origin with significant \ili{Aramaic} {substrate} influence. Because the \ili{Aramaic} influence, if any, must have occurred in the pre-{Cyprus} period, contact with \ili{Aramaic} will not be considered further here, despite its putative influence. A substantial discussion can be found in \citet{Borg2004}.

Another \ili{Semitic} language, \ili{Syriac}, is the liturgical language of the Maronite community. However, no instruction is available in \ili{Syriac} in {Cyprus}, so its use is limited to rote recitation during (very sparsely attended) church services, at which transliterations and \ili{Greek} translations are also provided.


\ili{English} is the third official language of the Republic of {Cyprus} (along with \ili{Greek} and \ili{Turkish}) and is widely spoken. Instruction in \ili{English} begins in primary school in the national curriculum, and private \ili{English}-medium schools are also widespread. However, contact with \ili{English} postdates contact with \ili{Greek} and \ili{Turkish} (beginning only after 1878 and intensifying in the twentieth century) and appears to have had no effects on CyA language structures.

The French school in Nicosia is traditionally a popular choice for Maronite families, so that competence in \ili{French} has also been common in the community – a shared characteristic with Lebanese Maronite society. However, like English, this appears not to have influenced CyA grammar in any significant way.

Remaining minority languages of {Cyprus} include \ili{Armenian} and a variety of \ili{Romani} locally called \ili{Kurbetça}/ Gurbetça. The reports of ECRML specify that there has been no contact requested or arranged between the Armenian and Maronite community institutions, however. The small size of the communities (each less than 1\% of the population) no doubt also reduces the chances of contact. As for \ili{Kurbetça}, it is unclear whether or not it is still actually spoken on the island. Members of this community are \ili{Turkish}-speaking and interact little if at all with the Maronite community.

Finally, the most common immigrant language after English is \ili{Russian}, which occupies an increasingly prominent place in the linguistic landscape of {Cyprus}. There are now several \ili{Russian}-medium schools on the island. However, these are primarily located outside the capital, and its recent appearance means that it also has not influenced CyA.


Therefore, the next section will focus on contact effects from \ili{Greek} on CyA.

\section{Contact-induced changes}
According to Borg, the doyen of CyA studies, “linguistic acculturation to \ili{Greek} in [CyA] is fairly extensive…and involves {transfer} of allophonic rules, function words, and virtually unrestricted borrowing of content words in the context of {codeswitching}” as well as “a significant degree of {calquing} on \ili{Greek} {idioms}” (\citeyear[64]{Borg2004}). This occurs to such an extent that he describes CyA as “\ili{Greek} in transparent \ili{Arabic} garb”, although “the degree of hellenization…tends to be concealed…the {inflectional} pattern of [CyA] having largely resisted significant intrusion of \ili{Greek} morphological elements” (\citeyear[65]{Borg2004}).

In the remainder of this section, we will examine examples of such \ili{Greek} influence, particularly in the phonological domain. At the same time, the remarkable persistence of CyA language patterns in the face of {intensive} contact, especially in the morphological domain, will be discussed.

\subsection{Phonology}


CyA phonology has been heavily restructured in comparison with other varieties of \ili{Arabic}, resulting in what Roth (\citeyear[55]{Roth2004}) calls “total {convergence}” of the phonological system with \ili{Cypriot Greek}. Similarly, Gulle (\citeyear[47]{Gulle2016}) refers to the “complete adoption of \ili{Greek} phonology.”

Like other varieties of \ili{Arabic} in intense contact with non-\ili{Semitic} languages, CyA has lost the series of so-called {emphatic}, guttural or {pharyngealized} consonants. The obstruents have {merged} with their non-{emphatic} counterparts, and the {pharyngeal} fricative \textit{ḥ} has {merged} with the original glottal fricative \textit{h}, which in turn is now pronounced as a velar fricative [x] under the influence of \ili{Greek}, as in the examples in \tabref{tab:walter:1}.\footnote{Examples are taken from Borg's (\citeyear{Borg2004}) glossary except where noted otherwise. CyA forms are given in his {orthography}. “\ili{Arabic}” forms are the presumed etymological source forms, typically shared by \ili{Standard} \ili{Arabic} as well as other varieties.}


\begin{table}
\begin{tabular}{>{\itshape}l>{\itshape}ll}
\lsptoprule
{\normalfont CyA} & {\normalfont Arabic} & {Gloss}\\\midrule
taraf & ṭaraf & ‘end’\\
txin & ṭaḥīn & ‘flour’\\
pakar & baqar & ‘cattle’\\
axsen & aḥsan & ‘better’\\
\lspbottomrule
\end{tabular}
\caption{\label{bkm:Ref13774757}Reflexes of emphatic and guttural consonants in CyA\label{tab:walter:1}}
\end{table}

The sole survivals among the \ili{Arabic} consonants that have no counterparts in \ili{Greek} are the interdental consonants and the {pharyngeal} glide /ʕ/ (see example \ref{deep} below). It is interesting that the {pharyngeal} glide, perhaps the most typologically unusual, remains as a sort of iconic survivor of the \ili{Arabic} phonemic inventory. The retention of this {phoneme}, alongside the loss of so many others, implies that the radical changes to the consonant inventory of CyA, though clearly linked to \ili{Greek} influence, cannot be wholly attributed to {imposition} in the sense of Van Coetsem (\citeyear{VanCoetsem1988,VanCoetsem2000}) – or at least, is evidence of significant resistance to such {imposition}. In any case, {imposition} would presumably be due to late learners of CyA, and it is doubtful that CyA was ever acquired in this way by speakers from outside the community.

As for the vowels, the \ili{Arabic} {vowel length} contrast has also been lost, unstressed (formerly) short vowels deleted, and mid vowels have joined the inventory, resulting in a five-vowel inventory matching that of \ili{Greek}, as illustrated in \tabref{tab:walter:2}.

\begin{table}
\begin{tabular}{>{\itshape}l>{\itshape}ll}
\lsptoprule
{\normalfont CyA} & {\normalfont Arabic} & {CyA Gloss}\\\midrule
ipn & ibn & ‘son’\\
umm & umm & ‘mother’\\
tarp & darb & ‘road’\\
klep & kilāb & ‘dogs’\\
yaxtop & yaktub & ‘he writes’\\
ten & yadayn & ‘hands\textsc{.du}’\\
\lspbottomrule
\end{tabular}
\caption{\label{bkm:Ref13775635}Illustration of the innovative vowel system of CyA\label{tab:walter:2}}
\end{table}


This unsurprising result also occurred in other contact varieties such as \ili{Maltese} and \ili{Andalusi} \ili{Arabic}, although may have evolved without the influence of contact, as in some \ili{Levantine} varieties.


Phonotactically speaking, CyA remains more permissive than \ili{Cypriot Greek}, in that it “allows a wider range of final consonants and is alone [{relative} to \ili{Cypriot Greek}] in allowing final clusters” \citep[51]{Newton1964}.

The effect of (\ili{Cypriot}) \ili{Greek} has not been limited to the phonemic inventory. CyA also conforms in the realm of alternations. Like \ili{Cypriot Greek}, CyA has absolute neutralization of voicing in stop consonants, as illustrated in \tabref{tab:walter:3}.

\begin{table}
\begin{tabular}{>{\itshape}l>{\itshape}ll}
\lsptoprule
{\normalfont CyA} & {\normalfont Arabic} & {Gloss}\\\midrule
sipel & sabal & ‘stubble’\\
{}ʕates & ʕadas & ‘lentils’\\
pakar & baqar & ‘cattle’\\
\lspbottomrule
\end{tabular}
\caption{\label{bkm:Ref13775945}Voicing neutralization in CyA stop consonants\label{tab:walter:3}}
\end{table}

It also has the same palatalization and spirantization rules (with the latter applying to the first member of {consonant clusters}), as well as {epenthesis} of transitional occlusives in clusters (\citealt{Tsiapera1969}; \citealt{Borg1985}; \citealt{Roth2004}), as illustrated in \tabref{tab:walter:4}.

\begin{table}
\begin{tabular}{>{\itshape}l>{\itshape}lll}
\lsptoprule
{\normalfont CyA} & {\normalfont Arabic} & {Gloss} & {Phonological process}\\\midrule
kʲilp & kalb & ‘dog’ & Palatalization\\
xtuft & katabt & ‘I wrote’ & Spirantization\\
pkyut & buyūt & ‘houses’ & Consonant {epenthesis}\\
\lspbottomrule
\end{tabular}
\caption{\label{bkm:Ref13776308}Greek-derived phonological processes in CyA\label{tab:walter:4}}
\end{table}

As with changes in the {phoneme} inventory, these additions to the phonological rules of CyA imply considerable L2 pronunciation effects of \ili{Cypriot Greek}, even though it was presumably typically acquired later in life than CyA, a puzzling apparent contradiction.

\subsection{Morphology}

According to Newton (\citeyear[43]{Newton1964}), “words of \ili{Arabic} […] origin retain the full morphological apparatus of \ili{Arabic} while those of \ili{Cypriot}-\ili{Greek} […] origin appear exactly as they do in the mouths of {monolingual} speakers of the \ili{Greek} dialect.” He goes on to state that “the exceptions to the rule that the morphemes of any one word are either exclusively [\ili{Cypriot Greek}] or exclusively [\ili{Arabic}] in origin would seem to be few,” and that \ili{Greek} verbs “are conjugated exactly as they are when they occur in [\ili{Greek}].” Example sentences that he provides contain multiple code-switches between \ili{Arabic} and \ili{Greek}-origin words, as in (\ref{bkm:Ref14086274}), where \ili{Greek} words are highlighted in bold.

\ea\label{bkm:Ref14086274}
{CyA}{} {\citealt[49]{Newton1964}}\\
\gll paxsop \textbf{na} \textbf{enicaso} xamse \textbf{kamares}\\
     intend.\textsc{impf.1sg} \textsc{sbjv} rent.\textsc{prs.1sg} five room.\textsc{pl}\\
\glt  ‘I intend to rent five rooms.’\z

Newton (\citeyear[50]{Newton1964}) concludes that neither source “would be in a position to claim an undisputed majority [of words/morphemes].” \citet{Gulle2016} also discusses examples of “loss of systemic integration” morphologically, with respect to noun plurals, meaning that \ili{Greek}-origin nouns are used with \ili{Greek} affixal morphology rather than being integrated into the CyA morphological system. The example in (\ref{bkm:Ref14102235}) illustrates the use of \ili{Greek}-origin nouns with \ili{Greek} plural morphology intact (in bold) in a CyA matrix sentence.

\ea\label{bkm:Ref14102235}
{CyA}{} {\citealt{Borg1985}: 183, 193}\\
\gll allik p-\textbf{petrokop-i} n-tammet l-\textbf{ispiriðk-ya} ta kan-yišelu \textbf{fayy-es}\\
     \textsc{dem}.\textsc{pl} \textsc{def}{}-stonecutter-\textsc{pl} \textsc{pass-}end.\textsc{prf.3sg.f} \textsc{def-}match-\textsc{pl} \textsc{comp} \textsc{prog.pst-}light.\textsc{impf.3pl} dynamite.hole-\textsc{pl}\\
\glt ‘While those stonecutters were igniting sticks of dynamite, the matches got used up.’
\z

On the whole, the picture is of a language somewhat similar to \ili{Maltese} (see Lucas \& Čéplö, this volume),\ia{Lucas, Christopher}\ia{Čéplö, Slavomír@Čéplö, Slavomír} in that we have two morphological systems operating in parallel, depending on the etymological origin of the {root} (\ili{Romance} or \ili{Arabic}, in the case of \ili{Maltese}; \ili{Greek} or \ili{Arabic}, in the case of CyA). Alternatively, we could say that speech in CyA is replete with {code-switching}, and the use of such \ili{Greek} forms says nothing about the system of CyA itself.


The main exception to morphological non-interaction between CyA and \ili{Greek} is the use of the \ili{Greek} {diminutive} suffix \textit{{}-ui} (feminine \textit{{}-ua}) with native CyA words, noted by all three of the major authors on CyA (Borg, Tsiapera, and Newton). For example, this suffix is used with \ili{Arabic} nouns such as \textit{xmara} ‘female donkey’ and \textit{pint} `girl', yielding \textit{xmarua} ‘small donkey’ and \textit{pindua} ‘girl’ (\citealt{Newton1964}: 43–44). \citet{Tsiapera1964} additionally notes the borrowing of two adjectival suffixes, \textit{{}-edin} (which makes nouns into adjectives) and nominal masculine singular \textit{{}-o}.

Relatedly, \citet{Gulle2016} observes that CyA lacks marking for {directive} and {locative}, unlike other \ili{Arabic} varieties but like spoken \ili{Greek}. Accusative case marking is used in spoken \ili{Greek} for this purpose, but due to the lack of overt case marking in CyA, such constructions are unmarked entirely.

\ea
{CyA}{} {\citealt[44]{Gulle2016}}\\
{\setlength{\multicolsep}{0pt}\begin{multicols}{2}
\ea \gll k-kafene\\
     \textsc{def}-cafe  \\
\glt ‘(in) the cafe’
\ex \gll fi-l-lixkali\\
     in-\textsc{def}-field\\
\glt ‘in the field’\label{bkm:Ref14168726} \label{field}
\z
\end{multicols}}
\z

Occasional use of \ili{Arabic} \textit{fi} ‘in,’ as in other varieties and example (\ref{bkm:Ref14168726}), was attributed by some CyA speakers of Gulle’s acquaintance to the influence of \ili{Levantine} \ili{Arabic}. For at least one speaker, the usage of {locative}/directional \textit{fi} appeared to be influenced by {calquing} from \ili{Standard Greek}.

However, Borg (\citeyear[3]{Borg2004}) notes similar usage in \ili{Old} \ili{Arabic} and \ili{Hebrew}, such that \ili{Greek} is not necessarily the source of this pattern. Gulle (\citeyear[47]{Gulle2016}) concludes that “the tense--aspect--{modality} (TAM) system [of CyA] is surprisingly almost completely intact”, adding only the exception of the use of the \ili{Greek} {modal} verb \textit{prepi} in necessitative constructions.

Finally, the occasional borrowing of the \ili{Greek} plural morpheme is observed. However, this is sporadic, and a quantitative investigation of pluralization based on Borg's (\citeyear{Borg2004}) glossary \citep{Walter2017} reveals that native non-suffixal plurals are still used for over half of all pluralizable nouns, at percentages even higher than those posited for other \ili{Arabic} varieties. \ili{Greek} plurals were given for only 8 of the 251 nouns.

Therefore, although the typically-\ili{Arabic} use of {non-concatenative} plural morphology is indeed subject to some degree of suffixal regularization (17\% of cases) and somewhat more restricted in terms of the variety of plural forms in CyA, the effect of \ili{Greek} plural forms has been negligible.

Plural {formation}, perhaps the most distinctive and cross-linguistically idiosyncratic morphological characteristic of CyA, thus appears remarkably robust in the face of contact. This echoes the retention of the {pharyngeal} glide in the phonological domain.

On the whole, as Borg (\citeyear[57]{Borg1994}) states, “the external impact on the native morphological patterns of [CyA] is slight.”

\subsection{Syntax}

According to Roth (\citeyear[70]{Roth2004}), “syntax is a linguistic domain particularly permeable to interference from \ili{Greek}” (author’s translation). By this she means that function words are doubled with loans from \ili{Greek}, in particular with {relative} clause markers and more complex constructions, as well as the use of \ili{Greek} and \ili{Arabic}-origin {negation} markers in combination. The example in (\ref{beer}) demonstrates the CyA use of the native \textit{ma} {negation} morpheme concurrently with \ili{Greek} \textit{me}…\textit{me}. In this case, phonetic similarity may have aided the adoption of \textit{me}.


\ea\label{beer}
\ea {CyA}{} {\citealt[149]{Borg1985}}\\
\gll ma-pišrap me pira me mpit\\
    \textsc{neg}-drink.\textsc{impf.1sg} \textsc{neg} beer \textsc{neg} wine  \\
\glt ‘I don’t drink either beer or wine.’

\ex
\ili{Cypriot Greek}\\
\gll em-pinno me piran me krasin\\
    \textsc{prog}-drink.\textsc{prs.1sg} \textsc{neg}  beer.\textsc{acc} \textsc{neg} wine.\textsc{acc} \\
\glt ‘I don’t drink either beer or wine.’
\z
\z

It is unclear, however, whether all or most of this is simply {code-switching} and whether it should be termed syntactic rather than lexical influence.

A syntactic change which does not involve {code-switching} or lexical borrowing is the development of a predicative {copula} (lacking in the present {tense} in most varieties of \ili{Arabic}) from \ili{Arabic} pronouns, discussed by both \citet{Roth2004} and \citet{Borg1985}, and illustrated in (\ref{bkm:Ref14184988}).



\protectedex{\ea {CyA}{} {\citealt[134]{Borg1985}}\label{bkm:Ref14184988}\\
{\setlength{\multicolsep}{0pt}\begin{multicols}{2}
\ea \gll l-iknise e maftux-a\\
     \textsc{def}-church \textsc{3sg.f} open-\textsc{f}  \\
\glt ‘The church is open.’ \label{church}
\ex \gll p-pkyara enne maʕak\\
     \textsc{def}-well.\textsc{pl}  \textsc{3pl} deep.\textsc{pl}\\
\glt ‘The wells are deep.’ \label{deep}
\z
\end{multicols}}
\z}

In example (\ref{church}), the {copula} corresponds to the third-person feminine pronoun ‘she’ (also \textit{e}, < \textit{hiya}). Likewise, the {copula} \textit{enne} in (\ref{deep}) corresponds to the third-person plural pronoun ‘they’ (also \textit{enne}, < \textit{hunna}). The development of this {copula} presumably replicates the obligatory present-{tense} {copula} found in \ili{Greek}. See Lucas \& Čéplö (this volume) for a similar phenomenon in \ili{Maltese}.\ia{Lucas, Christopher}\ia{Čéplö, Slavomír@Čéplö, Slavomír}

Finally, both \citet{Roth2004} and \citet{Newton1964} document variable placement of adjectives, according to both \ili{Arabic} and \ili{Greek} norms, as illustrated in (\ref{children}).


\ea\label{children}
\ea {CyA}{} {\citealt[72]{Roth2004}}\\
\gll m-mor-a li-zʕar\\
     \textsc{def}-child-\textsc{pl} \textsc{def}-small.\textsc{pl}  \\
\glt ‘small children’

\ex
\ili{Lebanese} Arabic{} {\citealt[48]{Newton1964}}\\
\gll l-bēt l-ikbīr\\
     \textsc{def}-house \textsc{def}-big  \\
\glt ‘the big house’

\ex
{CyA}{} {\citealt[47]{Newton1964}}\\
\gll li-kbir payt\\
     \textsc{def}-big house  \\
\glt ‘the big house’

\ex
\ili{Cypriot Greek}{}{\citealt[48]{Newton1964}}\\
\gll to meálo spítin\\
     \textsc{def} big house  \\
\glt ‘the big house’
\z
\z

However, \citet{Borg2004} notes that so-called “peripheral” varieties of colloquial \ili{Arabic} have been said to employ freer {word order} than others, so the variation in noun–adjective ordering may be an independent internal development (or alternatively, perhaps peripheral varieties are by nature more subject to contact, which leads to this pattern of variation).


In summary, syntax, like morphology, shows relatively little influence of language contact, especially in contrast to the phonological system. As {word order} is already relatively flexible in both CyA and \ili{Cypriot Greek} (e.g. with respect to subject–verb ordering; \citealt{Newton1964}: 48–49), this is perhaps to be expected.

\subsection{Lexicon}

According to \citet{Newton1964}, of the 630 common lexical items which he elicited, 38\% were \ili{Greek} in origin. However, he goes on to say that the percentage is lower in running speech, in which typically the most common (and therefore native \ili{Arabic} origin) vocabulary was used. Newton raises the possibility (\citeyear[51]{Newton1964}) that CyA consists of “\ili{Arabic} plus a large number of \ili{Cypriot} [\ili{Greek}] phrases thrown in whenever [a speaker’s] \ili{Arabic} fails him or the fancy takes him.” Tsiapera (\citeyear[124]{Tsiapera1964}) concurs, stating that “any speaker of [CyA] has a minimum of about thirty per cent of \ili{Greek} lexical items in his speech which are not assimilated into the phonological and morphological system of his native language.” She identifies the semantic fields of government and politics, numerical systems including weights and measures, and adverbial particles as particularly dominated by words of \ili{Greek} origin.

This percentage contrasts with the relatively small number of \ili{Greek}-origin items appearing in Borg's (\citeyear{Borg2004}) glossary. However, the difference in elicitation contexts must be kept in mind – Newton’s work occurring in the Cypriot context and himself being competent in \ili{Greek}, versus Borg’s work occurring partly overseas and himself an Arabist rather than a scholar of \ili{Greek}.

\citet{Roth2004} refers to the drastic reduction of the lexicon, and estimates that it includes at most 1300 items. Borg's (\citeyear{Borg2004}) glossary contains roughly 2000 entries (corresponding to 720 lexical consonantal {roots}), which he considers to be a “substantial portion” (though not all) of the “depleted” \ili{Arabic}-origin CyA lexicon.

Gulle (\citeyear[45]{Gulle2016}) notes suppletion in the paradigm of the verb ‘to come’, with imperative forms borrowed from \ili{Greek}. The consonantal {root} of the verb ‘to come’, in CyA as elsewhere in \ili{Arabic}, is \textit{√žy}, as seen in the form \textit{ža} ‘he came’. However, CyA imperative forms of this verb (\textit{ela,} \textit{eli,} \textit{elu}, in masculine singular, feminine singular, and plural forms, respectively) are clearly based on \ili{Greek} \textit{ela,} \textit{elate} (singular and plural, respectively). This particular case seems to reflect a pan-Balkan spread of this item, as \textit{ela/elate} are also used in Bulgarian (personal knowledge).

In summary, universal {bilingualism} and \ili{Greek} dominance among CyA speakers results in widespread use of code-switched \ili{Greek} vocabulary and associated morphology, with marginal lexical suppletion. However, there is very little loan material integrated into the CyA grammatical system.

As a final note, Hadjidemetriou's (\citeyear{Hadjidemetriou2009}) doctoral dissertation examines language contact between CyA and \ili{Cypriot Greek} (as well as \ili{Armenian} and \ili{Cypriot Greek}), in the opposite direction, to identify any effects of CyA on \ili{Cypriot Greek}. Unsurprisingly, however, given the current dominance of \ili{Cypriot Greek} for these speakers, no such effects were found, in any of the above domains.

\section{Conclusion}

CyA appears to present a counterexample to Van Coetsem’s notion of the {stability gradient}, which claims that phonology (and syntax) are more stable than other domains (the lexicon). It is clear that for CyA, phonology has been the least stable domain. The observed phonological {convergence} to \ili{Greek} is of the type that suggests pervasive effects of L2 pronunciation (except for the retention of the {pharyngeal} glide). Yet it is difficult to imagine any sociolinguistic scenario in which CyA was taken up in any significant numbers by \ili{Greek} speakers from outside the community, and the typical acquisition scenario (when CyA was still acquired by children) has been use of CyA as a home language, and \ili{Greek} as a school language, thereby generating sequential (though eventually probably \ili{Greek}-dominant) bilinguals. The historical record is unfortunately lacking any relevant information that could shed light on the situation.

The most urgent issue for {future} research on CyA is undoubtedly the need for additional documentation efforts. In particular, naturalistic texts and audio recordings are a desideratum. It is to be hoped that the documentation and revitalization efforts currently underway will remedy this situation.

\section*{Further reading}\label{FR}

Tsiapera's (\citeyear{Tsiapera1969}) work is the only one so far to consider CyA in its totality as a spoken language, although not at great length, and it drew subsequent criticism of the author’s lack of background knowledge of the \ili{Arabic} language. This monograph does, however, have the additional advantage of a publication date very close in time to the radical change in the sociolinguistic circumstances of CyA speakers due to ethnic tensions in the island, culminating in their near-unanimous relocation from traditionally Maronite villages to the capital city Nicosia.

Borg's (\citeyear{Borg1985}) foundational work on morphophonology is still the most extensive resource on CyA grammar. He takes a historical perspective on changes from earlier \ili{Arabic} to contemporary CyA, both contact-driven and otherwise, and also includes substantial textual material in CyA at the end. These texts are currently the only published ones available.

The follow-up volume by \citet{Borg2004} includes a substantial introductory essay situating CyA within the range of \ili{Arabic} dialects and elucidating the influences of the main contact language, \ili{Cypriot Greek}. The lexical entries are enriched by comparisons with dialectal forms from other varieties of \ili{Arabic}, as well as \ili{Greek}, \ili{Aramaic}, and other contact languages where relevant.

The most up-to-date and reliable information regarding CyA and its speakers, including documentation, preservation and revival efforts, may be found in the Council of Europe (\citeyear{CouncilofEurope2017}) report.


The following two community websites contain information on CyA institutions and activities in both \ili{Greek} and English, including contact information, historical background, archived copies of the monthly (\ili{Greek}-language) community newsletter, and so on.

\url{http://www.maronitesofcyprus.com} (in both \ili{Greek} and English)


\url{http://kormakitis.net/portal/} (in \ili{Greek})



\section*{Acknowledgements}
My sincere thanks to Christopher Lucas and Stefano Manfredi for organizing the workshop on \ili{Arabic} and language contact at the 23rd International Conference on Historical Linguistics, and their editing work on this volume. I would also like to thank the audience at ICHL for their insightful comments, as well as those of the members of the Processing and Acquisition of Language Lab at Cambridge University. Remaining errors and infelicities are of course my own.



\section*{Abbreviations}
\begin{tabularx}{.5\textwidth}{@{}lQ@{}}
\textsc{acc} & accusative\\
CE & Common Era\\
CyA & Cypriot Maronite Arabic\\
\textsc{comp} & {complementizer}\\
\textsc{def} & {definite}\\
\textsc{dem} & demonstrative\\
\textsc{du} & dual\\
ECRML & European Charter for Regional or Minority Languages\\
\textsc{f} & feminine\\
\textsc{impf} & imperfect (prefix conjugation)\\
\textsc{pass} & {passive}\\
\end{tabularx}%
\begin{tabularx}{.5\textwidth}{@{}lQ@{}}
\textsc{prf} & perfect (suffix conjugation)\\
\textsc{pl} & plural\\
PRIO & Peace Research Institute Oslo\\
\textsc{prog} & progressive \\
\textsc{prs} & present\\
\textsc{pst} & past {tense}\\
\textsc{sbjv} & subjunctive\\
\textsc{sg} & singular\\
TAM & tense--aspect--{modality}\\
UN & United Nations\\
UNESCO & United Nations Educational, Scientific and Cultural Organization
\end{tabularx}%


{\sloppy\printbibliography[heading=subbibliography,notkeyword=this]}
\end{document}
