\documentclass[output=paper]{langsci/langscibook} 
\author{Eleanor Coghill\affiliation{Uppsala University}}
\title{Neo-Aramaic}
% \keywords{} 
\abstract{This paper examines the impact of Arabic on the North-Eastern Neo-Aramaic dialects, a diverse group of Semitic language varieties native to a region spanning Iraq, Turkey, Syria and Iran. While the greatest contact influence comes from varieties of Kurdish, Arabic has also had considerable influence, both directly and indirectly via other regional languages. Influence is most apparent in lexicon and phonology, but also surfaces in morphology and syntax.}
\maketitle
\begin{document}

\section{Current state and historical development}

The Aramaic language (\ili{Semitic}, \ili{Afro-Asiatic}) has nearly three thousand years of documented history up to the present day. Once widely used, both as a first language and as a language of trade and officialdom, since the Arab conquests of the seventh century it has steadily shrunk in its geographical coverage. Today its descendants, the Neo-Aramaic dialects, only remain in pockets, especially in remoter regions, and are spoken almost exclusively by religious–ethnic minorities. Four branches of the language family exist today: due to diversification these cannot be considered a single language. Indeed, the largest branch, North-{Eastern} Neo-Aramaic (NENA), which is treated in this chapter, itself consists of many mutually incomprehensible dialects. Its closest relation is \ili{Ṭuroyo}/\ili{Ṣurayt}, which is spoken by {Christians}, known as \textit{Suryoye}, indigenous to the area immediately west of NENA’s western edge in Turkey. Another member of this branch ({Central} Neo-Aramaic) was \ili{Mlaḥso}, but this was nearly wiped out during the First World War, and its last speaker apparently died in the 1990s. 

The NENA dialects are, or were, spoken in a contiguous region stretching across northeastern Iraq, southeastern Turkey, northeastern Syria and northwestern Iran. The majority ethnicity in this region is the {Kurds}. NENA’s native speakers are exclusively from Christian and Jewish communities. The {Christians} belong to a variety of churches: the Church of the East, the Chaldean Catholic Church (which split off from the Church of the East when it came into communion with Rome), and (in fewer numbers) the \ili{Syriac} Orthodox Church and its uniate counterpart, the \ili{Syriac} Catholic Church. The {Christians}’ traditional religious–ethnic endonym is \textit{Surāye} and they call their language \textit{Sūraθ} or \textit{Sūrət} (depending on dialectal pronunciation). In other languages, and sometimes in their own, they identify mainly as Assyrians or Chaldeans.

The {Jews} are called \textit{hudāye} or \textit{hulāʔe} (depending on dialectal pronunciation), and they call their language \textit{lišāna} \textit{deni/nošan} ‘our language’ or \textit{hulaula} ‘Jewishness’. In Israel, where most now live, they are known as \textit{kurdím}, reflecting their geographical origin in the \ili{Kurdish} region, rather than their ethnic {identity}.

Historically, the NENA-speaking {Christians} usually lived in rural mono-ethnic villages and predominantly practiced agriculture, animal husbandry and crafts. {Jews} lived in both villages and towns, alongside other ethnic groups such as {Kurds}. They had diverse professions: tradesmen (pedlars, merchants and shopkeepers), craftsmen, peasants and landowners (\citealt{BrauerPatai1993}: 205, 212).

The region to which NENA is indigenous was, until, the twentieth century, highly diverse in terms of ethnicity, religion and language. Some of this diversity remains, but a great deal has been lost, due to the persecutions and ethnic cleansing that went on during that century and which were not unknown prior to it. During the First World War, Christian communities in {Anatolia}, being viewed as a fifth column in league with Russia, suffered murderous attacks and deportations. This affected not only Armenians and Greeks, but also the Sūraθ\textit{{}-}speaking Surāye and \ili{Ṭuroyo}\textit{{}-}speaking Suryoye, as well as the many Arabic-speaking Christian communities in the region (the extirpation of some of these is documented in \citealt{Jastrow1978}: 3–17).\footnote{The relationship between language and ethno-religious {identity} was and remains complex. Many {Christians} belonging to the \ili{Syriac} churches spoke and continue to speak yet other regional languages, including varieties of \ili{Turkish}, \ili{Armenian} and \ili{Kurdish}.} By the 1920s, the Hakkari province of Turkey had been emptied of its many communities of Surāye: survivors ended up in Iraq and Iran. Some Sūraθ-speaking villages remained in the neighbouring Şırnak and \ili{Siirt} provinces, but in the late twentieth century these too were mostly emptied of their inhabitants, during the conflict between the \ili{Turkish} state and the {Kurds}.

In Iraq too the twentieth century was far from peaceful for the NENA-speaking communities. After a massacre in the 1930s, a proportion of the survivors of the genocide moved from Iraq to Syria, where they settled along the Khabur river, still in their tribal groups. Others remained in Iraq, in some places in their original communities, in other places in mixed communities, where a koiné form of Sūraθ arose. After the founding of Israel, there was a backlash against {Jews} in Iraq, and almost all {Jews} left the country for Israel during the 1950s. In Israel their heritage and language were for the most part not appreciated and the language was not passed on to younger generations. Most remaining speakers are now elderly and some dialects have already died out.

From the 1960s onwards, conflicts between \ili{Kurdish} groups and the Iraqi state resulted in the destruction of numerous northern Iraqi villages, including many Christian ones. Other villages were appropriated by \ili{Kurdish} tribes. The war in 1990–1991, the international sanctions and the invasion of 2003 and subsequent instability further affected these communities, as they did all Iraqis, and resulted in a dramatic shrinking of the Christian community in Iraq. In 2014, when ISIS captured large swathes of northern Iraq, many {Christians} and other non-Sunni minorities had to leave their villages overnight. These villages were later recaptured, but, in the absence of extensive rebuilding and due to fears of a recurrence, many inhabitants have not returned and seek to leave the country. The outlook is therefore bleak for these communities and for their language.

\section{\label{bkm:Ref534214034}Contact languages}

The main contact language for NENA is – and has been for long time – \ili{Kurdish} (\ili{Iranian}, Indo-European), in its many varieties, as {Kurds} are by far the largest ethnic group in the region as a whole, excepting Iranian Azerbaijan, where Azeris predominate.\footnote{Small communities of \ili{Turkic}-speaking Turkmens are also found within northern Iraq. Their dialects share features with both Anatolian \ili{Turkish} varieties and Iranian \ili{Azeri} \citep{Bulut2007}.} {Kurds} have also been politically dominant: during the Ottoman period, {Christians} and {Jews} were in the power and under the protection of local \ili{Kurdish} rulers, the aghas (see \citealt{Sinha2000}: 11–12; \citealt[223]{BrauerPatai1993}). Most NENA speakers in the \ili{Kurdish}-speaking areas at this time seem to have spoken the local \ili{Kurdish} dialect.\footnote{For such information we rely mainly on statements in grammatical descriptions, where the researcher asked their informants about this. For instance, Hoberman (\citeyear[9]{Hoberman1989}) states, ``All my informants who grew to adulthood in Kurdistan report that they spoke fluent \ili{Kurdish} (\ili{Kurmanji})''. Other references for {Jews}’ competence in \ili{Kurdish} are: Sabar (\citeyear[216]{Sabar1978}), Mutzafi (\citeyear[5]{Mutzafi2004}), Khan (\citeyear[198]{Khan2007}) and Khan (\citeyear[11]{Khan2009}); for the {Christians} see Sinha (\citeyear{Sinha2000}: 12–13) and Khan (\citeyear[18]{Khan2008}).} It is not surprising, therefore, that there is more influence from \ili{Kurdish} than from any other language across most if not all of the NENA dialects, even if its extent varies from dialect to dialect.

What role, then, has Arabic played? To summarize: there has been longstanding direct contact with small Arabic-speaking communities in what are otherwise \ili{Kurdish}-speaking regions; there has been indirect contact through loans transmitted via \ili{Kurdish} and \ili{Azeri} varieties; finally, there has been intense contact more recently due to the establishment of states with Arabic as the national language, as well as various other modern developments. In the remainder of this section, we will go through these three types of contact in turn.

Although the region is not majority Arabic-speaking, there have been long-standing Arabic-speaking communities in certain parts of it: moreover many of these were Jewish and Christian, like the NENA-speakers, so one might well expect more social contacts with them. The Arabic dialects across the region are overwhelmingly of the \textit{qəltu} \ili{Mesopotamian}–\ili{Anatolian} type (contrasted with the southern \ili{Iraqi}/\ili{Bedouin} \textit{gələt} type).\footnote{The two types of \ili{Mesopotamian}–\ili{Anatolian} Arabic dialects are labelled by scholars according to the shibboleth of the form ‘I said’: \textit{qəltu} vs. \textit{gələt} (\citealt{Blanc1964}: 5–8). \textit{qəltu} dialects realize *q as /q/, while \textit{gələt} dialects (such as Muslim \ili{Baghdadi}), which are \ili{Bedouin} or \ili{Bedouin}-influenced, realize it as /g/. \textit{Qəltu} dialects also preserve the 1\textsc{sg} {inflection} \textit{-u} on the suffix-conjugation verb. See \citet{Talay2011} for an overview of \ili{Mesopotamian}--\ili{Anatolian} Arabic varieties. Note that some \ili{Bedouin} influence may be seen in the Muslim \textit{qəltu} dialects spoken on the plain south of \ili{Mardin} \citep[30]{Jastrow1978}.}

Christian \textit{qəltu} Arabic speakers could be found in the city of \ili{Mosul} (alongside \textit{qəltu} Arabic speakers of other religions) on the edge of the NENA-speaking Nineveh Plain (also known as the \ili{Mosul} Plain). They are also present in two villages on the Nineveh Plain, namely Bəḥzāni and Baḥšiqa. Arabic-speaking Yazidis\footnote{Elsewhere, Yazidis are \ili{Northern Kurdish}-speaking.} also live in these villages, as well as (in Baḥšiqa) some Muslim Arabs \citep[24]{Jastrow1978}. The Christian NENA speakers of the Nineveh Plain, therefore, had ample opportunity to come into contact with Arabic. To find more Christian Arabic-speaking communities in or near the NENA region, we have to travel quite far, to what are now the \ili{Turkish} provinces of Şırnak, \ili{Siirt} and \ili{Mardin}. In this region there were many Christian \textit{qəltu} Arabic-speaking communities living in villages and towns until the First World War; fewer afterwards. The settlements with such communities included \ili{Āzəḫ} (\ili{Turkish} \textit{İdil}) and Ǧazīra (\textit{Cizre}) in Şırnak province, as well as provincial centres \ili{Siirt} and \ili{Mardin} (\citealt{Jastrow1978}: 1–23). Thus, Christian Arabic speakers were in close proximity to speakers of NENA dialects in the Bohtan and Cudi regions of Şırnak province, as well as to speakers of \ili{Ṭuroyo}/\ili{Ṣurayt} in \ili{Mardin} Province.

Jewish \textit{qəltu} Arabic-speaking communities were also found in both northern Iraq and southeastern Turkey. In Iraq, Arabic was spoken by the {Jews} of \ili{Mosul}, ʕAqra (\ili{Kurdish} \textit{Akre}) and Arbil (Erbil; \ili{Kurdish} \textit{Hawler}), as well as of the village of Ṣəndor, near Duhok \citep[9]{Hoberman1989}. These all left in the 1950s. Further afield, there were also some Jewish Arabic speakers in Urfa, Diyarbakır, Siverek and Çermik \citep[4]{Jastrow1978}, who also migrated to Israel. There are known to have been contacts between NENA-speaking and Arabic-speaking {Jews}, through family connections and commerce. Mutzafi (\citeyear[6]{Mutzafi2004}) reports such contacts involving the Jewish men of Koy Sanjaq and the Arabic-speaking {Jews} of Kurdistan. Sabar (\citeyear{Sabar1978}: 216–217) relates that the {Jews} of Zakho would visit relatives who had moved to Mosul and Baghdad. On the other hand, Hoberman (\citeyear[9]{Hoberman1989}) stated that the {Jews} of ʕAmədya knew no more than a few words of \ili{Iraqi} Arabic.

To sum up, historically, Christian NENA speakers only had direct local contact with Arabic speakers (of their own faith) in \ili{Mosul} and the Nineveh Plain in Iraq and Şırnak province in Turkey. The NENA-speaking {Jews}, on the other hand, had Arabic-speaking co-religionists not only in \ili{Mosul}, but also within Iraqi Kurdistan itself.

While most NENA dialects show greatest influence from the majority languages of the region – \ili{Kurdish} and (in Iranian Azerbaijan) Iranian \ili{Azeri} – these also played a role in transferring Arabic influence to NENA. Arabic, as the language of Islam, has had a great influence on \ili{Kurdish} varieties and \ili{Azeri}, especially in the lexicon, and many originally Arabic words have been transmitted to NENA via these languages. Sometimes it is difficult to identify the immediate donor of such words, but phonetics and morphology can help (see §\ref{bkm:Ref13232790}).

During the twentieth century, with the founding of the states of Iraq and Syria, Arabic became the language of the states that most NENA-speakers found themselves in. They came into contact with it through education, officialdom, military service, radio and trade. Many {Christians} from the north of Iraq moved south to the major (Arabic-speaking) cities, Mosul, Baghdad and Basra, where, in some cases, they shifted to speaking Arabic, while keeping in close contact with relatives back in the north. By the end of the twentieth century most NENA speakers in Iraq and Syria would have been at ease in Arabic. Naturally these later developments did not affect speakers in Turkey and Iran, who, instead, developed greater competence in \ili{Turkish} and \ili{Persian}, respectively. Jewish speakers from Iraq, who had left the region by the end of the 1950s, would have had less exposure to Arabic through these means.

It should be mentioned that there has also been influence from European languages, namely from \ili{French} (via the influence of the Catholic Church among the Chaldean Catholic communities) and from \ili{English} (dating to the British Mandate period, as well as the period of globalization from the late twentieth century), though some lexical borrowings from these languages may have been mediated by Arabic.


\section{Contact-induced changes in North-Eastern Neo-Aramaic}

Contact influence on NENA\footnote{Sources for the main contact languages, if not indicated, are as follows: \ili{Iraqi} Arabic (specifically Muslim \ili{Baghdadi}): \citet{WoodheadEtAl1967}; \ili{Northern Kurdish} (i.e. \ili{Kurmanji}/Bahdini): \citet{Chyet2003}. Although Muslim \ili{Baghdadi} Arabic is not the dialect in closest contact with NENA, as a \ili{Mesopotamian} dialect it shares much lexicon with more northerly varieties (which do not have a dictionary). The transcription of \ili{Northern Kurdish} words is based on the conventional {orthography}, as given in Chyet (\citeyear{Chyet2003}: xxxix–xl): an IPA transcription is also given. The source for the Christian Alqosh and Christian Telkepe data is the author’s own fieldwork. Other sources are referenced in the text. The author’s own NENA data is transcribed in IPA except as follows: \textit{č} [ʧ], \textit{j} [ʤ] (equivalent to Arabic \textit{ǧ}), \textit{y} [j], \textit{ḥ} [ħ], \textit{x} between [x] and [χ], and \textit{\.g} between [ɣ] and [ʁ]. Apart from \textit{ḥ}, consonants with a dot under are the {emphatic} (velarized/{pharyngealized}) versions of the undotted consonant; for instance, the symbol \textit{ð̣} represents [ðˤ]. Some dialects have {emphasis} extending across whole words: such words are conventionally indicated with a superscript cross, e.g. \textrm{\textsuperscript{+}}\textrm{\textit{sadra} }\textrm{(equivalent to} \textrm{\textit{ṣạḍ{\R}ạ}}). The schwa symbol \textit{ə} is used to transcribe a NENA vowel that is, in non-{emphatic} contexts, typically pronounced as [ɪ]. Phonemically contrastive length in vowels is indicated with a macron, e.g. \textit{ā} [aː]. The vowels /i/, /e/ and /o/ are usually realized long: [iː], [eː] and [oː]. NENA words from other sources have had their transcription adjusted in some cases to bring them closer to this system: the original transcription may be checked in the referenced sources.} seems to have arisen mainly through long-term bi- and multi-lingualism, rather than {language shift}. Indeed, if any shift has taken place, it is more likely to have involved NENA speakers who converted to Islam and shifted to \ili{Kurdish}.\footnote{It often happened that Christian girls were (occasionally by arrangement, but often unwillingly) kidnapped by {Kurds} for the purpose of marriage. Any children would have been considered {Kurds}.} Furthermore, much of Iraq was in earlier times Aramaic-speaking, so it can be assumed that over the centuries a shift took place from Aramaic to Arabic. Some Aramaic {substrate} features can indeed be seen in \ili{Iraqi} Arabic dialects, such as a kind of {differential object marking} (\citealt{Coghill2014}: 360–361).

Using Van Coetsem’s (\citeyear{VanCoetsem1988,VanCoetsem2000}) distinctions between changes due to borrowing (by agents dominant in the {recipient language}) and {imposition} (by agents dominant in the {source language}), the contact influences from Arabic attested in NENA are clearly of the first kind, namely borrowing.

Borrowing from Arabic into NENA is of interest particularly as a case of {transfer} between related and typologically similar languages, as both are \ili{Semitic}. Like Arabic and other \ili{Semitic} languages, NENA has in its verbal morphology, and to a lesser extent in its nominal morphology, a {non-concatenative} {root-and-pattern} system, complemented by affixes. Thus, with the triradical {root} \textit{√šql}, we get such forms as \textit{k-šāqəl} ‘he takes’, \textit{šqəl-lə} ‘he took’, \textit{šqāla} ‘taking’, \textit{šaqāla} ‘taker’, \textit{šqila} ‘taken’, and so on.

Arabic influence in NENA is considerable in the realm of the lexicon, but this has very often occurred via other contact languages, rather than directly. (All the contact languages show great influence from Arabic, at least in the lexicon). Direct lexical borrowing or morphological and structural borrowing from Arabic are less common: they are however well attested in the Christian dialects of the Nineveh Plain, as well as some Jewish dialects of the Lišāna Deni branch in northern Iraq, including the dialects of Zakho, Nerwa and ʕAmədya (\ili{Kurdish} \textit{Amêdî}, Arabic \textit{al\nobreakdash-ʕAmādiyya}).

It is difficult to establish with any certainty which contact influences entered the dialects at which time. The earliest Christian and Jewish NENA texts (from the 16th and 17th centuries)\footnote{The Jewish manuscripts date to the 17th century, but the texts may have been composed earlier (\citealt{Sabar1976}: xxix, xliii–xlvi). The Christian manuscripts date to the 18th century but the composition of the texts can be dated to the 16th and 17th centuries \citep[16]{Mengozzi2002}.} already show considerable contact influence from \ili{Kurdish} and Arabic. The extent of Arabic influence in the early Jewish Lišāna Deni texts \citep{Sabar1984} is quite surprising. The towns in which these texts originate lie deep in Kurdistan, relatively far from the Arabic speaking part of Iraq. As we have seen in §\ref{bkm:Ref534214034}, however, {Jews} in Kurdistan had contacts with Arabic-speaking co{}-religionists. Some contact influence in the NENA dialects is clearly of recent date, such as {loanwords} from \ili{English}, which probably date to the twentieth century. The {prospective} construction of the Christian Nineveh Plain dialects, which appears to be a structural borrowing from vernacular Arabic (see §\ref{bkm:Ref534214095}), seems to have developed only in the last hundred years or so \citep[375]{Coghill2010}.

By the end of the twentieth century, Arabic was having an immense influence on the speech of Christian Aramaic-speaking communities living in northern Iraq, expecially those close to \ili{Mosul}, such as the town of Qaraqosh. Khan (\citeyear[9]{Khan2002}) found that most people from Qaraqosh introduced Arabic words and phrases into their Neo-Aramaic without adaptation.  Khan attributes this to the policy of Arabicization in Iraq, which meant that schoolchildren were only educated in Arabic. He found significantly greater influence from Arabic in the younger generation’s speech. In Christian Qaraqosh, as in the neighbouring dialects of Christian Alqosh and Christian Telkepe (author’s fieldwork), a large number of Arabic {loanwords} have recently been absorbed into the lexicon. Nevertheless, as Khan remarks, “the proportion of Arabic loans that have penetrated the {core vocabulary} of the dialect and replaced existing Aramaic words are relatively few.” This may, however, not be the case with speakers who have grown up in Arab-majority cities such as Baghdad. In my admittedly limited experience with such speakers, they use a noticeably greater proportion of Arabic {loanwords}, even sometimes for basic vocabulary, e.g. \ili{Iraqi} Arabic \textit{ð̣ēʕa} for \textit{māθɒ} ‘village’ (heard from a Christian Telkepe speaker who grew up in Baghdad before settling in the US).

\subsection{Lexicon}
\subsubsection{\label{bkm:Ref13232790}Introduction}

All NENA dialects have adopted a large number of {loanwords}. While \ili{Kurdish} predominates among these, Arabic {loanwords} are also common, especially among the Christian dialects of the Nineveh Plain and the Jewish Lišāna Deni dialects.

 Khan (\citeyear[516]{Khan2002}) makes a useful distinction for Christian Qaraqosh between “(i) loan-words that do not have any existing Aramaic equivalent and (ii) those for which a native Aramaic substitute is still available in the dialect.”\footnote{Note, however, that apparent synonyms are not always identical in meaning. Christian Alqosh \textit{šəbbakiyə} (< Ar. \textit{šubbāk}) is used for a modern glass window, while the inherited lexeme \textit{kāwə} is used for the traditional type of window.} These two types seem to reflect two layers of borrowing, an earlier one and a recent one, which, in many cases, is akin to {code-switching}. Most \ili{Kurdish} loans belong to the first type, while Arabic loans are most common in the second, though earlier loans do exist. Borrowed Arabic nouns of the second type show little or no adaptation to native morphology, Khan finds. Verbs, however, are always adapted to NENA verbal morphology. Most are slotted into the existing NENA verbal derivations (see §\ref{bkm:Ref13233110}).

Khan (\citeyear[516]{Khan2002}) remarks that speakers of Christian Qaraqosh are generally aware of the Aramaic alternatives to these Arabic loans and can give them if asked. It could be, however, that subsequent generations will have had little exposure to the older synonyms.\footnote{The fieldwork for the monograph on this dialect was carried out around the year 2000.} Khan notes that some of these older synonyms are themselves {loanwords}, in some cases from Arabic, but so integrated and long-standing that many speakers may not be aware of this. Examples include the recent Arabic loan \textit{fəkr} (< Arabic \textit{fikr}) and the older loan \textit{taxmanta} (f. {infinitive} of NENA \textit{√txmn} Q ‘to think’, denominal < Arabic \textit{taḫmīn} ‘estimation’; see §\ref{bkm:Ref13233110}), both meaning ‘thought’.

Many {loanwords} are common to several languages of the region, especially words specific to local culture or to technologies. While the ultimate source can usually be identified, it can sometimes be hard to determine the immediate donor of the loan. 

Nevertheless, there is sometimes evidence that can establish the immediate donor. This is the case, for example, for Arabic words ending in the feminine suffix \textit{tāʔ} \textit{marbūṭa} (\ili{Standard} Arabic \textit{{}-a(t)}). The Arabic morpheme is realized with the final /t/ in suffixed forms and in the construct (i.e. followed by a possessor). When borrowed into NENA, the /t/ is not realized in the absolute (isolated) form of the word, as in Arabic, e.g. Alqosh \textit{sāʕa} ‘hour’ (Ar. \textit{sāʕa}). This contrasts with \ili{Kurdish}, which has the /t/ in all forms, e.g. \ili{N. Kurd.} \textit{sa‘et} [sɑːˈʕæt] ‘hour’. In some NENA dialects, in certain words, the /t/ appears as \textit{{}-ət-} in suffixed forms, replicating a pattern in (\textit{qəltu}) Arabic. Sometimes this leads to {back-formations} (see §\ref{bkm:Ref534226861}). In other items the \textit{tāʔ} \textit{marbūṭa} is realized as \textit{{}-at} in all contexts, as it typically is in \ili{Kurdish}, and this suggests it was borrowed via \ili{Kurdish}. An example of the latter is Jewish Betanure/Jewish Challa \textit{ʕaširat} ‘tribe’, pl. \textit{ʕaširatte} (\citealt{Mutzafi2008}: 103; \citealt{Fassberg2010}: 270). This is borrowed from \ili{Northern Kurdish} \textit{‘eşîret} [ʕæʃiːˈræt], which borrowed it from Ar. \textit{ʕašīra(t)} ‘tribe’, almost certainly via \ili{Persian} and/or Ottoman \ili{Turkish}. Another example, \textit{ʕādat} ‘custom’, is given by Maclean in his grammar of ``Vernacular \ili{Syriac}'' \citep[35]{Maclean1895}, where he states that nouns ending in \textit{\nobreakdash-at} are feminine.\footnote{\textrm{In Maclean’s dictionary} \textrm{\citep[235]{Maclean1901}, he gives} \textrm{\textit{ʕādat} }\textrm{({orthography} adjusted)}\textrm{ }\textrm{as the form in the Christian Urmia dialect and as one of the variants in ``Alqosh'', by which he means the Nineveh Plain dialects (the other variant being} \textrm{\textit{ʕāde}}\textrm{, which, lacking the final /t/, appears to be directly borrowed from Arabic). He gives} \textrm{\textit{ʕādəta,}} \textrm{on the other hand, for his ``Ashirat'' dialect group, which was spoken in ``central Kurdistan'' (today’s Hakkari province of Turkey). This looks like the {back-formations} from direct Arabic loans discussed in §\ref{bkm:Ref534226861}, which is a little surprising, as one would not expect much direct contact with Arabic in that region. It is, however, a large and diverse group of dialects, and he does not specify in which precise dialect it was attested.}} Fox (\citeyear[91]{Fox2009}), writing of Christian Bohtan, also views Arabic loans ending in \textit{{}-at} as having been borrowed via \ili{Kurdish}. Examples in this dialect are: \textit{sahat} ‘hour’, \textit{hakowat} ‘tale’, \textit{qəṣṣat} ‘story’, \textit{kəflat} ‘family’ (< \ili{N. Kurd.} \textit{kuflet} [kʊf\kern 1ptˈlæt] \textit{{\textasciitilde} k’ulfet} [kʰʊlˈfæt] ‘wife, family’ < Ar. \textit{kulfa} ‘trouble’) and \textit{məllat} ‘nation’ (< \ili{N. Kurd.} \textit{milet} [mɪˈlæt] < Ar. \textit{milla}). Some of the same examples (\textit{məḷḷat} and \textit{qəṣṣat}) may also be found in Christian ʕUmra: Hobrack (\citeyear[108]{Hobrack2000}) takes these to have been borrowed via \ili{Turkish}, but, given the overwhelming influence of \ili{Kurdish} in the region, it seems more plausible that they were borrowed via \ili{Kurdish}.\footnote{The \ili{Kurdish} forms attested in dictionaries are not always what we would expect as the sources of these forms, however. Thus we find \textit{ḧekyat} [ħækjɑːt] \textit{{\textasciitilde} ḧikyet} [ħɪk\textrm{ˈ}jæt] ‘story’  and \textit{qise} [qɪ\textrm{ˈ}sæ] ‘story’ (not \textit{qiset}). A variant of the latter ending in /t/, however, is found in a nineteenth-century dictionary cited in Chyet (\citeyear{Chyet2003}: 490–491).}

Sometimes there are other indications in the word’s form that it was borrowed via \ili{Kurdish}: the common NENA word \textit{šūla} ‘work’ derives ultimately from Arabic \textit{šuɣl}. \ili{Northern Kurdish} has also borrowed this word, as \textit{şuxul} [ʃuˈxul] with a variant \textit{şûl} [ʃuːl]. It is perhaps the latter which is the immediate origin of the NENA word.

The {gender} in NENA can also suggest the immediate source of a {loanword}. For instance, \textit{qalam} ‘pen’ in Arabic has masculine {gender}, but, loaned into \ili{Northern Kurdish} as \textit{qelem}, it may have feminine or masculine {gender} (\citealt{Chyet2003}: 478; \citealt{Rizgar1993}: 322). That \textit{qalāma} ‘pen’ has feminine {gender} in certain NENA dialects (e.g. Alqosh; \citealt[199]{Coghill2004}) suggests that it was borrowed via \ili{Kurdish}, not directly from Arabic.

It is difficult to date {loanwords} in a predominantly unwritten language. Nevertheless, we do have written texts in both the Christian Nineveh Plain and the Jewish Lišāna Deni dialects going back at least four hundred years, and even in early texts the proportion of lexemes that were borrowed was high. Arabic loans are conspicuous in both sets of texts. Sabar (\citeyear[208]{Sabar1984}) found that in a typical Jewish text from Nerwa, 30\% of lexemes are ultimately of Arabic origin (whether directly or via another language).

Loanwords may be adapted to varying degrees and in varying ways to the {recipient language}. §§\ref{bkm:Ref13229660}–\ref{closedclass} deal with the ways in which loans in different word classes may be integrated, as well as the ways in which they retain characteristics of the donor language, {focusing} on Arabic loans.

\subsubsection{\label{bkm:Ref13229660}Integration of nouns}


Most NENA nouns end in the nominal suffix \textit{{}-a} (usually, but not exclusively, masculine nouns) or \textit{{}-ta{\textasciitilde}-θa} (feminine nouns). Older borrowed nouns usually have one of these endings, e.g. Christian Alqosh \textit{ʕamma} ‘paternal uncle’ (< Ar. \textit{ʕamm}), \textit{ʕašāya} ‘dinner’ (< \ili{Iraqi} Ar. \textit{ʕaša}) \textit{ḥadāda} ‘blacksmith’ (< Ar. \textit{ḥaddād}), \textit{ʕāṣərta} ‘early evening’ (< \ili{Iraqi} Ar. \textit{ʕaṣir}) and \textit{maʕwəlta} ‘axe (or similar tool)’ (< \ili{Iraqi} Ar. \textit{maʕwal} ‘pickaxe’). Even if they do not, they are adapted to NENA {stress} patterns. Thus Ar. \textit{ḥayaw\'{ā}n} ‘animal’ is borrowed (possibly via \ili{N. Kurd.} \textit{ḧeywan} [ħɛjˈwɑːn]) as \textit{ḥɛwan} in Christian Alqosh, which has penultimate {stress} \citep[81]{Coghill2004}.

More recent loans, on the other hand, may be used without any such modifications, e.g. Christian Alqosh \textit{ʕamal} ‘thing’ (< Ar. \textit{ʕamal} ‘work’), \textit{xām} ‘linen’ (\ili{Iraqi} Ar. \textit{ḫām} ‘raw; cotton cloth’), and \textit{sāʕa} ‘hour’ (f., < Ar. \textit{sāʕa} f.). They often occur also in their original Arabic plural forms, e.g. Christian Alqosh \textit{fallāḥín} ‘farmers’ and \textit{ʔaʕdād} ‘(large) numbers’. 

Many Arabic {loanwords} come with the Arabic feminine marker \textit{tāʔ} \textit{marbūṭa} (\ili{Standard} Arabic \textit{\nobreakdash-a}). In \textit{qəltu} Arabic dialects this usually has two realizations: \textit{\nobreakdash-a} after {emphatic} or back consonants, otherwise a high vowel such as \textit{\nobreakdash-e} or \textit{\nobreakdash-i}.\footnote{See Jastrow (\citeyear[40]{Jastrow1979}) for the conditioned \textit{imāla} (raising of a-vowels) in the \textit{tāʔ} \textit{marbūṭa} in the Arabic dialect of \ili{Mosul}, and Jastrow (\citeyear[70]{Jastrow1990book}) for the same in the Jewish Arabic dialect of ʕAqra and Arbīl.} Such loans in NENA usually also have the same distribution, that is \textit{{}-e} (or the dialectal variant \textit{-ə}), except after an {emphatic} or back consonant, when it is \textit{\nobreakdash-a} (Telkepe \textit{\nobreakdash-ɒ}), e.g. Christian Alqosh \textit{baṭālə} ‘idleness’ and \textit{rawð̣a} ‘kindergarten’ and Christian Telkepe \textit{ʕādə} ‘custom’ and \textit{qəṣṣɒ} ‘story’ (see also §\ref{bkm:Ref534226861}).

Some loans appear to have come from \ili{Standard} Arabic and have the \textit{\nobreakdash-a} regardless, e.g. Christian Telkepe \textit{lahjɒ} ‘dialect’ and \textit{madrasɒ} ‘school’. Christian Qaraqosh seems to always represent the \textit{tāʔ} \textit{marbūṭa} as \textit{{}-a} \citep[204]{Khan2002}.

Borrowed nouns are quite commonly given Aramaic {derivational} suffixes. For instance, Jewish Azerbaijani \textit{amona} ‘paternal uncle’ has a borrowed {stem}, \textit{am-}, from Ar. \textit{ʕamm} ‘paternal uncle’ via \ili{Kurdish} or \ili{Azeri}, but an Aramaic {derivation}, \nobreakdash-\textit{ona}, originally with {diminutive} function \citep[165]{Garbell1965}. An example from the early Lišāna Deni texts is \textit{\.garibūθa} ‘foreignness’, from Arabic \textit{ɣarīb} ‘foreign, strange’ and the NENA abstract ending \textit{{}-ūθa} \citep[205]{Sabar1984}.

NENA often adopts the {gender} of the donor language, where that language has nominal genders (as in the case of Arabic and \ili{Northern Kurdish}, which both have masculine--feminine {gender} systems). Thus, the following Christian Alqosh words share the same {gender} as their Arabic source: \textit{ʕašāya} ‘dinner’ (m., like \ili{Iraqi} Ar. \textit{ʕaša}) and \textit{daʔwa} ‘wedding party’ (f., like Arabic \textit{daʕwa} ‘invitation, party’). The {loanword} \textit{ʕāṣərta} ‘early evening’ is, however, feminine (as indicated by the NENA feminine ending \textit{{}-ta}), while the Arabic source (\ili{Iraqi} Ar. \textit{ʕaṣir}) is masculine. In \ili{Northern Kurdish}, however, it is feminine (\textit{{}'esir} [ʕæˈsɪɾ]), and this may have influenced the {gender}, which, in turn, motivated the adding of the feminine suffix.

In Christian Telkepe, some Arabic {loanwords} of the structure *CaCC have, when not suffixed, an epenthetic vowel between the final two consonants. This is absent when a suffix beginning with a vowel is added, i.e. the construct suffix \textit{{}-əd} or a possessive pronominal suffix. This follows the rules in the donor language: those Arabic dialects which have the epenthetic vowel (including \ili{Baghdadi} and some \textit{qəltu} dialects, such as \ili{Mosul}) also lose it under similar conditions.\footnote{For \ili{Baghdadi} Arabic, see Erwin (\citeyear{Erwin1963}: 56–58).} Examples include \textit{ʕaqəl} ‘mind’: \textit{ʕaql-əd=baxtɒ} [mind-\textsc{cstr}=woman] ‘a woman's mind'; and \textit{ḥarub} ‘war’: \textit{p\nobreakdash-ḥarb\nobreakdash-əd=sawāstipūl} [in-war-\textsc{cstr}=Sebastopol] ‘in the Crimean war’. It is interesting to note that the same rule is also found for Arabic {loanwords} in \ili{Kurdish} \citep[5]{Thackston2006}.

Occasionally, {loanwords} are adapted to the native {root-and-pattern} templates, following the selection of a {root}. This frequently occurs when the {root} is also borrowed as a verb. Thus we find Christian Qaraqosh \textit{ʔəjbona} ‘a will, wish’ \citep[517]{Khan2002}, alongside the verb \textit{√ʔjb} I ‘to please’ (< Ar. \textit{√ʕǧb} IV), by {analogy} with native words on the pattern CəCCona, e.g. \textit{yəqðona} ‘a burn’ (< \textit{√yqð} I ‘to burn’). Sabar (\citeyear[205]{Sabar1984}) gives further examples from the early Lišāna Deni texts. More often, however, borrowed nouns are not adapted to native templates, e.g. Alqosh \textit{ḥanafiya} ‘tap’ (< Ar. \textit{ḥanafiyya}), or only coincidentally follow a native noun pattern (Arabic and NENA share many similar patterns), e.g. \textit{qahwa} ‘coffee’ (< Ar. \textit{qahwa}), which fits into the common Aramaic pattern CaCCa.

NENA dialects all have a variety of plural suffixes, the most common being perhaps \textit{\nobreakdash-e} (or its dialectal variant \textit{\nobreakdash-ə}). Loanwords, like inherited words, take a wide variety of native plural suffixes, but certain suffixes may be preferred or dispreferred for {loanwords}, in combination with other factors. For instance in Christian Alqosh feminine {loanwords} are not attested with the Aramaic plural suffixes \nobreakdash-\textit{wāθa} and \nobreakdash-\textit{awāθa}, while the loan-plural \nobreakdash-\textit{at} (< Ar. \nobreakdash-\textit{āt}) is almost exclusively found with {loanwords} \citep[347]{Coghill2005}. Recent Arabic loans in Christian Nineveh Plain dialects often occur, unadapted, in their Arabic plural form (see §\ref{bkm:Ref534226861}).

\subsubsection{Integration of adjectives}

Like nouns, loan adjectives may occasionally be adapted to the native {root-and-pattern} templates, after the selection of a {root}. For instance, Arabic \textit{ʔazraq} ‘blue’ (\textit{√zrq}) is borrowed by Christian Alqosh as \textit{zroqa} ‘blue’, by {analogy} with certain inherited colour adjectives of the form CCoCa, such as \textit{smoqa} ‘red’. Another example is Christian Alqosh \textit{ʕadola} ‘straight’ (cf. \ili{Iraqi} Ar. \textit{ʕadil} ‘straight’ and Christian Qaraqosh which has borrowed it simply as \textit{ʕadəl}).\footnote{Attested inherited words of the pattern CaCoCa are all in fact nouns in Christian Alqosh, e.g. \textit{ʔalola} ‘street’. The pattern CaCūCa might be more expected, being found with several common adjectives, e.g. \textit{xamūṣa} ‘sour’.} More often the {stem} of the loan adjective is borrowed more or less unchanged, as in Christian Alqosh \textit{faqira} ‘poor’ (Ar. \textit{faqīr}), coincidentally fitting the inherited adjectival pattern CaCiCa. Adapted loan adjectives tend to take NENA {inflection} (e.g. f. \textit{{}-ta{\textasciitilde}-θa}, pl. \textit{{}-ə}). Unadapted loan adjectives usually take no {inflection} at all, e.g. Christian Telkepe \textit{qə́rməzi} ‘purple’ (Ar. \textit{qirmizī} m. ‘crimson’) and \textit{ð̣aʕíf} ‘thin’ (\ili{Iraqi} Ar. \textit{ð̣aʕīf} m. ‘thin, weak’).

Loan-adjectives of a certain group including colours and bodily traits behave in a special manner in some NENA dialects: they take Aramaic {inflection} for masculine and plural, but a special {inflection} \textit{\nobreakdash-ə} (identical to the plural ending) for the feminine. This occurs in Christian Qaraqosh particularly with Arabic loan adjectives, e.g. \textit{ṭarša} ‘deaf’ (f./pl. \textit{ṭaršə}, < Ar. m. \textit{ʔaṭraš}, f. \textit{ṭaršāʔ}) and \textit{zarqa} ‘blue’ (f./pl. \textit{zarqə}, < Ar. m. \textit{ʔazraq}, f. \textit{zarqāʔ}), see Khan (\citeyear[219]{Khan2002}). It appears to come from a dialectal reflex (\textit{\nobreakdash-ē}) of the Arabic \textit{\nobreakdash-āʔ} feminine ending, found especially with adjectives of these semantic groups.\footnote{Oddly enough, however, the realization as \textit{-ē} seems to be restricted to \ili{Anatolian} \textit{qəltu} Arabic dialects (where it is stressed, e.g. \ili{Āzəḫ} \textit{lāl\'{ē}} ‘dumb’), and not found in the dialects in Iraq \citep[76]{Jastrow1978}. Other words ending in *\textit{{}-āʔ} in have \textit{{}-ē} (unstressed) in \textit{qəltu} Arabic dialects, but only as cases of \textit{imāla} (raising of a-vowels) conditioned by a neighbouring high vowel.}  In Christian Alqosh it is also found with {loanwords} of \ili{Northern Kurdish} origin, e.g. \textit{kačal-a} ‘bald’ (f./pl. \textit{kačal\nobreakdash-ə}, from \ili{N. Kurd.} \textit{k’eçel} [kʰæˈʧæl]).

In Arabic and \ili{Kurdish}, adjectives normally follow the head noun, as in NENA. There are, however, a few pseudo-adjectival modifiers borrowed from Arabic which precede the noun in Arabic and are uninflected. These show the same behaviour when borrowed into NENA. One is \textit{ʔawwal} ‘first’ in Christian Alqosh (a synonym to the inherited word \textit{qamāya} ‘first’), as in \textit{ʔawwal꞊ga} ‘the first time’ – compare Arabic \textit{ʔawwal} \textit{marra} ‘the first time’. Another is \textit{\.ger} ‘other’ (< \ili{Iraqi} Ar. \textit{ɣēr}), which is attested in Jewish Betanure, e.g. \textit{\.ger꞊məndi} ‘something else’ \citep[105]{Mutzafi2008} – compare \ili{Iraqi} Arabic \textit{ɣēr} \textit{yōm} ‘another day’. Another {loanword}, \textit{xoš} ‘good’, invariably precedes the noun, e.g. Christian Telkepe \textit{xoš꞊ʔixālɒ} ‘good food’. This seems to originate in \ili{Iranian} (\ili{Persian} or \ili{Kurdish}), but is also common in \ili{Iraqi} and \ili{Anatolian} Arabic dialects (as \textit{ḫōš}), as well as in \ili{Turkic} varieties (as \textit{hoş} [hoʃ] or \textit{xoş} [xoʃ]). In all these languages it precedes the noun, regardless of the usual {word order}.

\subsubsection{\label{bkm:Ref13233110}Integration of verbs}

The borrowing of verbs has been identified as potentially more complicated than the borrowing of other lexemes, due to their tendency to be morphologically complex \citep[175]{Matras2009}. The borrowing of verbs in a \ili{Semitic} language presents particular issues, due to the unusual {root-and-pattern} system. In \ili{Semitic} languages verb lexemes are composed of a {root} (typically consisting of three – occasionally four – consonants or semi-vowels) and a {derivation} (also known as ``{stem}'', ``form'', ``measure'', ``binyan'' or ``theme''). NENA dialects mostly have three triradical derivations (I, II and III) and at least one quadriradical {derivation} (Q). A borrowed verb will usually be integrated into this system. Three main strategies have been identified for the borrowing of verbs in NENA. One, common also in other \ili{Semitic} languages (\citealt{Wohlgemuth2009}: 173–180), is {root} extraction, whereby from the phonological matter of the source verb a tri- or quadriradical {root} is selected. This is usually then allocated to a verbal {derivation}. A second is the borrowing of not only the {root} but also some of the morphology of the Arabic {derivation}: see below and §\ref{bkm:Ref13233345}. A third is the {light verb} strategy, whereby the loanverb consists of a {light verb} (with meanings such as `become' or `make') and a (verbal) noun, the latter containing the main semantic content. 

The {light verb} strategy is found in some NENA dialects, but usually with \ili{Kurdish} or \ili{Turkish} verbs, which already consist of a {light verb} plus noun. It is not used to integrate Arabic loanverbs, although sometimes the noun in the predicate ultimately comes from Arabic.

The root-extraction strategy is well attested across NENA dialects and is particularly common with Arabic loanverbs. This is unsurprising, as these already have a {root}, which in many cases can simply be adopted as it is. For instance, Arabic \textit{√ɣlb} I ‘to win’ (\textit{ɣalaba} ‘he won’) is borrowed as Christian Telkepe \textit{√\.glb} I ‘to win’. Sometimes the {root} is adapted, to conform to the rules of {root} {formation} in NENA. For instance, ‘geminate’ {roots}, where the final two radicals are identical (√C\textsubscript{1}C\textsubscript{2}C\textsubscript{3}, where C\textsubscript{2}=C\textsubscript{3}), are rare in NENA, and apparently absent altogether in {derivation} I. Just as inherited geminate {roots} were converted into middle-\textit{y} {roots} (√C\textsubscript{1}yC\textsubscript{3}), so too are Arabic geminate {roots}. Thus, Arabic \textit{√sdd} I ‘to close, stop up’ is borrowed as Christian Alqosh \textit{√syd} I ‘to close, seal’ (compare inherited \textit{√qy{\R}} I ‘to be cold’ < \textit{√qrr}).

Sometimes {derivational} affixes are adopted as radicals, often replacing a weak radical. For instance, Arabic {derivation} VIII verb \textit{ittafaqa} (\textit{√wfq}) is borrowed by Christian Alqosh as \textit{√tfq} I ‘to meet’, with the VIII {derivational} infix \textit{\nobreakdash-t\nobreakdash-} reanalysed as a radical. Frequently the {root} is borrowed not from a true verb but from a (verbal) noun or adjective. Thus, the NENA verb \textit{√txmn} Q (found, e.g., in Jewish Betanure and Christian Qaraqosh, and as \textit{√txml} Q in Alqosh) is borrowed from the Arabic noun \textit{taḫmīn} (possibly via \ili{Northern Kurdish} \textit{t’exmîn} [tʰæxˈmiːn] ‘supposition, guess’), itself a {derivation} of Arabic \textit{√ḫmn} II ‘to guess’ (\textit{ḫammana} ‘he guessed’). The /t/ of the NENA {root} is not found in the Arabic {root}, but can only come from the verbal noun. This is an {extension} of an inherited \ili{Semitic} strategy of deriving verbs from nouns. See Sabar (\citeyear{Sabar1984}; \citeyear{Sabar2002}: 52) and Garbell (\citeyear[166]{Garbell1965}) for more on the creation of verbal {roots} from non-Aramaic verbs.

The process of integration does not end with the establishment of a {root}, however. Every verb lexeme must also have a {derivation}. Tendencies can also be identified for this \citep{Coghill2015}. Arabic loanverbs already have a {derivation}, but the majority of Arabic derivations have no {cognate} or functional equivalent in NENA. Where there is a {cognate}, there are also some formal and functional similarities, and thus such cases are usually loaned into the {cognate} {derivation}. Thus, for instance, Arabic \textit{√ʕdl} II (\textit{ʕaddala}) ‘to put in order’ is borrowed as Christian Telkepe \textit{√ʕdl} II ‘to fix, tidy’ (e.g. \textit{mʕudəlli} ‘I tidied’), Telkepe {derivation} II being the {cognate} of the Arabic {derivation} of the same number.

Verbs in Arabic derivations that have no {cognate} are sometimes allocated to derivations that bear some similarity in form or function to the original {derivation}. For instance, the NENA {derivation} most closely resembling Arabic {derivation} III in form is {derivation} II (the two share the template -CvCvC-, as opposed to -vCCvC-). Thus Arabic \textit{√hğr} III (\textit{hāğara}) ‘to emigrate’ is borrowed as Christian Telkepe \textit{√hjr} II ‘to emigrate’ (e.g. \textit{mhujera} ‘they emigrated’).

Arabic derivations VIII and X may be treated differently: in Christian Iraqi dialects, in particular those of the Nineveh Plain, the {derivational} morphology may itself be borrowed along with the lexeme (see §\ref{bkm:Ref13233345}).

\subsubsection{Grammatical words and closed classes}\label{closedclass}

NENA has freely borrowed grammatical words such as {prepositions}, conjunctions and particles of various functions, and some of these are Arabic, though most are \ili{Kurdish}. In some cases, the original Arabic items may have been borrowed via \ili{Kurdish}. In Christian Alqosh we find the {preposition} \textit{ṣob} ‘towards, near’ (< Ar. \textit{ṣawba} ‘towards’, cf. \ili{Iraqi} Ar. \textit{ṣōb} ‘direction’) and \textit{baḥás} ‘about, concerning’ (< \ili{N. Kurd.} \textit{be\"{h}s} [bæħs] ‘discussion (about)’ < Ar. \textit{baḥθ}). Another example is \textit{m-badal} ‘instead of’ (< \textit{m-} ‘from’ + \ili{Iraqi} Ar. \textit{badāl}; \citealt{Coghill2004}: 300). In Jewish Challa we also find \textit{m-badal} and, in addition, \textit{mābayn} ‘between, among’ (< Ar. \textit{mā} \textit{bayn}; \citealt{Fassberg2010}: 149, 151). Even in Jewish Arbel, which generally shows less Arabic influence, we find \textit{ḍidd} ‘against’ (< Ar. \textit{ḍidd}; \citealt{Khan1999}: 188).

Loan {prepositions} are not a new phenomenon in NENA, but are already attested in the early Jewish Lišāna Deni texts (\citealt{Sabar1984}: 208), e.g. \textit{ʕann\nobreakdash-ɩd} ‘about’ (< Ar. \textit{ʕan} ‘about’), \textit{ṣōb} ‘beside’ (< Ar. \textit{ṣawba}). By {analogy} with certain native {prepositions}, some have been extended with the construct suffix \textit{\nobreakdash-əd}, e.g. \textit{ʕann\nobreakdash-ɩd}.

A p{article} that has been commonly borrowed is \textit{bas} ‘only; but’ (cf. \ili{Iraqi} Ar. \textit{bass} ‘enough; only; but’). This may have been borrowed via \ili{Northern Kurdish} \textit{bes} [bæs] ‘enough; but’.

Many dialects, including Christian Alqosh and Christian Telkepe, use \textit{kabira} to express ‘much’ or ‘very’. This derives from Arabic \textit{kabīr} ‘big’. In Christian Qaraqosh (\citealt{Khan2002}: 284–5) they use another Arabic loan for the same meaning: \textit{ḥel {\textasciitilde} ḥelə} (cf. \ili{Iraqi} Ar. \textit{ḥēl} ‘with force’).

Other particles commonly borrowed are \textit{fa} (roughly ‘and so’ in both Arabic and NENA) and \textit{lo} ‘or; either’ (\ili{Iraqi} Ar. \textit{lō}). The adverb \textit{baʕdén} ‘then; later’ (< Ar. \textit{baʕdēn}) is attested frequently in the Christian dialects of Alqosh, Telkepe and Qaraqosh, despite the presence of an inherited synonym, \textit{baθər꞊dəx} [after꞊how] ‘then; later’.

In Christian Alqosh and Christian Qaraqosh, a particle \textit{də\nobreakdash-} is used with imperatives to give the command a sense of urgency or encouragement. This is already attested in the early Jewish Lišāna Deni texts (\citealt{Sabar1976}: xl). This appears to come from \ili{Northern Kurdish} \textit{de} [dæ] with the same function. A similar {participle} (\textit{dē\nobreakdash-,} \textit{də\nobreakdash-}) is found in both \textit{qəltu} and \ili{Baghdadi} Arabic (\citealt{Jastrow1978}: 310–311).

\subsection{Phonology}

Two types of phonological contact influences in NENA will be considered here: new phonemes adopted through contact, and allophonic alternations influenced by contact.

\subsubsection{New phonemes}

NENA dialects have gained several new phonemes through language contact. These phonemes have entered the dialects via {loanwords} that were not fully adapted to Aramaic phonology. Some new phonemes are restricted to {loanwords}, while others have developed also in native words, through processes such as combination (creating affricate phonemes) and assimilation. As might be expected, \ili{Kurdish} {loanwords} are responsible for the majority of the borrowed phonemes, but Arabic has also played a role, especially in those dialects closest to the Arabic-speaking region, i.e. the Christian dialects of the Nineveh Plain. The examples given below are from the Christian Alqosh dialect of this group (\citealt{Coghill2004}: 11–25, with adapted transcription).

Some of the borrowed phonemes in NENA dialects have been introduced by both \ili{Kurdish} and Arabic {loanwords}. These include /j/ [ʤ] and /č/ [ʧ]. The latter is not found in \ili{Standard} Arabic, but is found in \ili{Mesopotamian} dialects of Arabic. The {phoneme} /f/ seems to be borrowed predominantly from Arabic, although this {phoneme} also exists in \ili{Kurdish}. Examples of {loanwords} with these three phonemes are: \textit{ješ} ‘army’ (< \ili{Iraqi} Ar. \textit{ǧēš}), \textit{jullə} ‘clothes’ (< \ili{N. Kurd.} \textit{cil} [ʤɪl]), \textit{čārək} ‘quarter’ (< \ili{N. Kurd.} \textit{čarêk} [ʧɑːˈreːk]) \textit{√čyk} I ‘to pierce’ (< \ili{Iraqi} Ar. \textit{√čkk} I), and \textit{faqira} ‘poor’ (< Ar. \textit{faqīr}).

The {phoneme} /č/ is also found in certain native Aramaic words, as a result of the combination of /t/ and /š/, e.g. \textit{čeri} in \textit{čeri} \textit{qamāya} ‘October’ (< *tšeri, {cognate} with Christian Qaraqosh \textit{təšri} and \ili{CSyr} \textit{tešri {\textasciitilde} tešrin} ‘Tishrin’).

The Arabic {phoneme} /ð̣/ [ðˤ] is found in many {loanwords} in Iraqi NENA dialects, e.g. \textit{√ḥð̣r} III ‘to prepare’ (< \ili{Iraqi} Ar. \textit{√ḥð̣r} II). In most Mespotamian dialects of Arabic in contact with NENA, /ḍ/ is rarely found, as it has {merged} with /ð̣/. Nevertheless, one {loanword} in Alqosh and Qaraqosh has the /ḍ/ {phoneme}, namely \textit{ʔoḍa} ‘room’, which originally comes from \ili{Turkish} \textit{oda}. While \ili{Turkish} is not considered to have {emphatic consonants}, it does have vowel harmony, and words with back vowels have been interpreted as having {emphatic consonants}, when borrowed into \textit{qəltu} (and other) Arabic dialects (\citealt{Jastrow1978}: 51–52). Thus the \textit{qəltu} dialect of Qarṭmin, in which *ḍ and *ð̣ have {merged} as /ð̣/, also has \textit{ʔōḍa} ‘room’ \citep[70]{Jastrow1978}. NENA \textit{ʔoḍa} was borrowed from \ili{Turkish} either via a local Arabic variety or directly, in which case its speakers must have also interpreted back-voweled \ili{Turkish} words as {emphatic}.\footnote{\ili{Northern Kurdish} also has this word, but Chyet's (\citeyear{Chyet2003}) dictionary only gives variants without {emphasis} (e.g. \textit{ode}), although Iraqi \ili{Kurdish} dialects do often preserve {emphasis} in Arabic {loanwords} (\citealt{Chyet2003}: viii; see also Öpengin, this volume).}\ia{Öpengin, Ergin@Öpengin, Ergin}

The {pharyngeals} /ʕ/ and /ḥ/, which in most inherited Aramaic lexemes have shifted to /ʔ/ and /x/ respectively, have been reintroduced through {loanwords} from both Arabic and the Classical \ili{Syriac} used in the church. Examples for /ʕ/ are: \textit{ʕamma} ‘uncle’ (< Ar. \textit{ʕamm}), \textit{√ʕyš} I ‘to live’ (< Ar. \textit{√ʕyš} I), \textit{ʕəddāna} ‘time’ (\ili{CSyr} \textit{ʕeddānā}). Examples for /ḥ/ are: \textit{√jrḥ} I ‘to get injured’ (< Ar. \textit{√ǧrḥ} I ‘to injure’), \textit{√ḥð̣r} III ‘to prepare’ (< \ili{Iraqi} Ar. \textit{√ḥð̣r} II), \textit{mšiḥa} ‘Christ’ (< \ili{CSyr} \textit{mšiḥā}), and \textit{ḥaṭṭāya} ‘sinner’ (< \ili{CSyr} \textit{ḥaṭṭāyā}). In some Arabic loans, however, /ʕ/ has shifted to /ʔ/, perhaps indicating that they belong to an earlier stratum, e.g. Christian Alqosh \textit{daʔwa} ‘wedding party’ (Ar. \textit{daʕwa}). Some cases of /ʕ/ and /ḥ/ in Alqosh, as in other NENA dialects, are original: the shift to /ʔ/ and /x/ respectively has been blocked in certain phonetic environments, particularly in the neighbourhood of {emphatic consonants} or /q/, e.g. \textit{raḥūqa} ‘far’ (< *raḥḥūqa), see Khan (\citeyear{Khan2002}: 40–41). Furthermore, /ḥ/ has arisen in the third person singular possessive suffixes, as a shift from original *h. This appears to be a strategy of disambiguating these suffixes from the phonetically similar nominal endings (see \citealt{Coghill2008}: 96–97).

The voiced uvular fricative was an allophone of the voiced velar stop /g/ in earlier Aramaic. In NENA it {merged} with *ʕ and shifted to a glottal stop /ʔ/. Like the {pharyngeals}, it has been reintroduced into NENA through {loanwords} from both Arabic and Classical \ili{Syriac}, e.g. \textit{√\.glb} I ‘to win, defeat’ (< Ar. \textit{√ɣlb} I) and \textit{pa\.gra} ‘body’ (< \ili{CSyr} \textit{pa\={g}rā}). It has also arisen in native words through regular assimilation of /x/ to a following voiced consonant. In the case of the verb \textit{√\.gẓd} I ‘to reap’ (< *√xẓd < *√xṣd  < *√ḥṣd), the voiced allophone, originally only found in certain forms, has spread by {analogy} throughout the paradigm \citep[20]{Coghill2004}.

The cases of /č/, the {pharyngeals}, and /\.g/ show how new phonemes may arise through borrowing, while being assisted by internal developments.

\subsubsection{Allophonic sound alterations}

Some NENA dialects, such as Christian Alqosh \citep[27]{Coghill2004}, exhibit final {devoicing} of voiced consonants, e.g. \textit{mjāwəb} [mˈdʒæup] ‘answer!’ (cf. \textit{mjawobə} ‘to answer’ with [b]) and \textit{qapa\.g} [ˈqɑpɐχ] ‘lid’ (cf. \textit{qapa\.gəd-dəstiθa} ‘saucepan lid’, with [ʁ]). There is also a strong tendency towards {devoicing} in both \textit{qəltu} Arabic \citep[98]{Jastrow1978} and the \ili{Kurdish} dialects of Iraq \citep[49]{MacKenzie1961}, so it seems to be an areal feature (see also Akkuş, this volume on contact-induced {devoicing} in \ili{Anatolian} Arabic, and Lucas \& Čéplö, this volume on the same phenomenon in \ili{Maltese}).\ia{Lucas, Christopher@Lucas, Christopher}\ia{Čéplö, Slavomír@Čéplö, Slavomír}\ia{Akkuş, Faruk@Akkuş, Faruk}

\subsection{Morphology}

NENA dialects have borrowed a variety of morphemes from regional languages via lexical loans. As these become more integrated into the language, they may be found not only in the original {loanwords} but also with new words, including inherited lexemes. NENA being a \ili{Semitic} language, it is possible for morphological borrowings to be a templatic pattern rather than a single phonetic chunk: indeed, some verbal {derivational} patterns have been borrowed from Arabic, as will be shown in §\ref{bkm:Ref13233345}.

\subsubsection{\label{bkm:Ref534226861}Nominal inflection}

A grammatical suffix that has been borrowed by some Iraqi dialects is the Arabic feminine sound plural suffix \textit{{}-āt}. In Christian Alqosh and Christian Qaraqosh, as well as the Jewish Lišāna Deni dialects of northern Iraq, it has been integrated into the native morphology: as these dialects have penultimate {stress} in nouns, the suffix itself is not stressed in these dialects as it is in Arabic (\citealt{Coghill2004}: 272–273; \citeyear{Coghill2005}; \citealt{Khan2002}: 193–194). Accordingly it has also been shortened to \textit{\nobreakdash-at}, e.g. Christian Alqosh \textit{makina} ‘machine’, pl. \textit{makinat}, \textit{maḥallə} ‘town quarter’, pl. \textit{maḥallat}. In Alqosh and Qaraqosh it is only attested with feminine nouns. It is not, however, restricted to Arabic loans, but has been extended to other foreign words, e.g. Alqosh \textit{pošiya} ‘turban’ (\ili{N. Kurd.} \textit{pʼoşî} [pʰoːˈʃiː]) pl. \textit{pošiyat}. In Alqosh and Qaraqosh it is even found with some native Aramaic words, e.g. Christian Qaraqosh \textit{ʔarnuwa} ‘rabbit’, pl. \textit{ʔarnuwat} ‘rabbits’; \textit{ʔilāna} ‘tree’, pl. \textit{ʔilānat} ‘trees’.

In some words, probably borrowed during the more recent and more intense period of contact with Arabic, the original {stress} and length of the ending is preserved, e.g. Christian Alqosh \textit{hol\'{ā}t} ‘halls’ and Christian Qaraqosh \textit{badl\'{ā}t} ‘suits’ and \textit{gadl\'{ā}t} ‘tresses’ \citep[194]{Khan2002}. (Note, however, that the latter is an Aramaic word). This is always the case in Telkepe, e.g. \textit{jəddɒ} ‘midwife’, pl. \textit{jədd\'{ā}t} and \textit{traktar} ‘tractor’, pl. \textit{traktar\'{ā}t}. Note that in Telkepe, as in Arabic, this plural is sometimes found with masculine nouns, e.g. \textit{mez} (m.) ‘table’, pl. \textit{mez\'{ā}t} or \textit{primuz} (m.) ‘primus stove’, pl. \textit{primuz\'{ā}t}.

Apart from the Christian Nineveh Plain dialects, \textit{\nobreakdash-at} is attested regularly as a plural in some of the Jewish Lišāna Deni dialects, spoken further to the north. As mentioned in §\ref{bkm:Ref534214034}, these Jewish communities would have had contact with spoken Arabic through connections with their co-religionists.

In the modern Jewish dialect of Zakho, \textit{\nobreakdash-at} is used with the following types of nouns (\citealt{Sabar2002}: 44–45): feminine Arabic loans ending in \textit{\nobreakdash-a} or \textit{\nobreakdash-e} (i.e. the dialectal version of the Arabic feminine suffix \textit{tāʔ} \textit{marbūṭa}; see §\ref{bkm:Ref13229660}), some nouns of \ili{Kurdish} origin ending in \textit{\nobreakdash-e} (perhaps by {analogy} with Arabic loans ending in \textit{\nobreakdash-e}), and nouns ending in certain borrowed suffixes, namely the {diminutive} suffix \textit{\nobreakdash-ka} (f. \textit{\nobreakdash-ke}) borrowed from \ili{Kurdish}, the professional suffix \textit{{}-či} borrowed from \ili{Turkish}, and the ending \textit{\nobreakdash-o}. It is also one of the two most common plurals for European {loanwords}, e.g. \textsuperscript{+}\textit{pākētat} ‘packets (of cigarettes)’ \citep[57]{Sabar1990}. This suggests it is particularly associated with {loanwords}, regardless of origin. In Jewish Duhok (also Lišāna Deni), however, it is attested with a native Aramaic word, \textit{raʔolat} ‘brooks’ \citep[45]{Sabar2002}. It seems therefore that the morpheme has been extended far beyond its original distribution.

The plural \textit{{}-at} does not seem to have spread to all Lišāna Deni dialects, however: it is not mentioned in the grammars of Jewish Challa \citep{Fassberg2010} and Jewish Betanure \citep{Mutzafi2008}. It has, nevertheless, an early origin: it is found in the late seventeenth-century manuscripts originating in the towns of ʕAmədya and Nerwa. I found one example of it in the grammar of the modern ʕAmədya dialect \citep[70]{Greenblatt2011}, namely \textit{maymonke} (f.) ‘monkey’, pl. \textit{maymonkat}, probably because it has the \ili{Kurdish} {diminutive} suffix (see above).

Across the border in Turkey, another Christian dialect has this plural ending, that is the dialect of ʕUmra (\ili{Turkish} name \textit{Dereköyü}), close to the town of Cizre. In this region of Turkey there are or were several Arabic-speaking communities, including Christian Arabic speakers in Cizre (until the First World War; see \citealt{Jastrow1978}: 17), so it is not surprising that there should be influence from Arabic. In this dialect, \nobreakdash-\textit{at} is mostly attested with borrowed feminine nouns ending in \textit{\nobreakdash-e}, though there are also a couple ending in \nobreakdash-\textit{a}, both masculine and feminine \citep[114]{Hobrack2000}. The majority have the \ili{Kurdish} {diminutive} suffix \textit{\nobreakdash-ka} (f. \textit{\nobreakdash-ke}) mentioned above in relation to Jewish Zakho.

In the Christian dialects of Iraq, as spoken currently, it is common to use Arabic words with their original plural morphology, probably because almost all speakers speak Arabic with native or near-native competence and many concepts are more familiar or only available to them in this language.\footnote{Younger NENA speakers who have grown up in the \ili{Kurdish}-controlled region since 1991 may have less competence in Arabic, however.} Thus, apart from the \textit{\nobreakdash-āt} plural, we also find the masculine sound plural suffix \textit{{}-in} and the {non-concatenative} broken plurals, e.g. Christian Alqosh \textit{fallāḥ\nobreakdash-ín} ‘farmers’, and \textit{barāmíl} ‘barrels’ (sg. \textit{barmíl}) \citep[273]{Coghill2004}. We even find such examples in the late seventeenth-century manuscripts written in Jewish Lišāna Deni dialects, e.g. \textit{\.gāfılīn} ‘fools’ and \textit{ʔarwāḥ} ‘spirits’ (\citealt{Sabar1984}: 205–206).

Many Arabic {loanwords} come with the Arabic feminine marker \textit{tāʔ} \textit{marbūṭa}, either the \textit{qəltu} Arabic variants or the \ili{Standard} Arabic \textit{\nobreakdash-a} (§\ref{bkm:Ref13229660}). In some dialects of the Nineveh Plain, the \textit{tāʔ} \textit{marbūṭa} is borrowed along with its connecting allomorph \textit{{}-ət}. In Arabic the /t/ is only realized in construct state (as the head of a genitive phrase) or before possessive suffixes.

In Christian Qaraqosh the isolated form of such loans ends in \textit{{}-a}, like inherited masculine nouns, although the {gender} is feminine (as in the source words). When possessive suffixes are added, however, the /t/ is realized, as in Arabic (\citealt{Khan2002}: 204–206). Thus Qaraqosh \textit{badla} ‘suit of clothes’ (cf. \ili{Iraqi} Arabic \textit{badla}) becomes \textit{badl\nobreakdash-ətt\nobreakdash-əḥ} [suit-\textsc{f-3sg.m]} ‘his suit of clothes’. The gemination of the /t/ is not found in the Arabic forms, but can be explained as follows. In \ili{Mosul} Arabic, unlike in many Arabic dialects, the \textit{tāʔ} \textit{marbūṭa} takes the {stress}, when any possessive suffix is added: \textit{báṣali} ‘onion’, \textit{baṣal\nobreakdash-ә́t\nobreakdash-ak} [onion-\textsc{f-2sg.m}] ‘your onion’ \citep[105]{Jastrow1983}. It is likely that the /ə/ vowel in the NENA morpheme \textit{{}-ətt-} imitates the vowel of the Arabic morpheme. The {stress} pattern fits well into NENA, which has penultimate {stress}. However, in NENA /ə/ is dispreferred in an open syllable, especially when stressed. The /t/ is probably geminated in order to close the syllable so as to conform to this preference.\footnote{Khan (\citeyear[206]{Khan2002}) gives two other possible derivations: a combination of Arabic f. \textit{\nobreakdash-ət} and Aramaic f. \textit{\nobreakdash-ta} (though the latter is not found on the isolated form) or the NENA independent genitive particle \textit{did-}. The explanation above seems to me to be simpler, however.} This mechanism has parallels elsewhere in NENA.

These same {loanwords} take the Arabic plural \textit{{}-at} discussed above. Even some Aramaic feminine words in Christian Qaraqosh have acquired both \textit{{}-ətt-} and \textit{\nobreakdash-at}, e.g. \textit{ʔarnuwa} (f.) ‘rabbit’, \textit{ʔarnuwəttəḥ} ‘his rabbit’, \textit{ʔarnuwat} ‘rabbits’. But \textit{{}-ətt-} is also found with some Aramaic feminine words that have native plurals, e.g. \textit{bira} (f.) ‘well’, \textit{birāθa} ‘wells’, \textit{birəttəḥ} ‘his well’. In exceptional cases \textit{{}-ətt-} may also be used with feminine words with the Aramaic f. ending \textit{{}-ta{\textasciitilde}-θa}, e.g. \textit{šwiθa} ‘bed’, \textit{šwiyāθa} ‘beds’, \textit{šwiθəttəḥ} ‘his bed’. It seems, therefore, that in Qaraqosh this is now a morphological borrowing independent of the {loanwords} it was originally borrowed with.

In Christian Telkepe, vernacular Arabic nouns with \textit{tāʔ} \textit{marbūṭa} are borrowed ending in either \textit{\nobreakdash-ə} or \textit{\nobreakdash-ɒ}, matching the two realizations of the \textit{tāʔ} \textit{marbūṭa} in \textit{qəltu} Arabic (§\ref{bkm:Ref13229660}). As in Qaraqosh, these nouns retain their feminine {gender} in Telkepe. They also have the \textit{\nobreakdash-ətt-} allomorph before possessive suffixes, e.g. \textit{ṣəḥḥɒ} (f.) ‘health’,  \textit{ṣəḥəttux} [\textit{ṣəḥ-ətt-ux} health-\textsc{f-2sg.m}] ‘your (m.) health’; \textit{qubbə} (f.) ‘room’, \textit{qubbətte} [\textit{qubb-ətt-e} room-\textsc{f-3sg.m}] ‘his room’. The infix seems to be used productively with Arabic words, as and when they are used. One example in Telkepe is not borrowed from a feminine with \textit{tāʔ} \textit{marbūṭa}, namely \textit{čāyi} (f.) ‘tea’ (cf. \ili{Iraqi} Ar. \textit{čāy} (m.)). This word is, however, feminine in \ili{Northern Kurdish} (\textit{çay} [ʧɑːj]), whence it may have been borrowed.

Christian Alqosh seems to have gone a step further, creating {back-formations} from the suffixed forms. Thus the unsuffixed forms also have \textit{{}-ətt-}, e.g. \textit{ṣaḥətta} ‘health’, \textit{qaṣətta} ‘story’ and \textit{məllətta} ‘religious community’. When the plural suffix (always the feminine plural \textit{\nobreakdash-yāθa}) is added, one /t/ alone is preserved, suggesting that the second is now analysed as part of the feminine singular ending \textit{\nobreakdash-ta}, while \textit{\nobreakdash-ət\nobreakdash-} is analysed as part of the {stem}\textit{:} \textit{qaṣət\nobreakdash-ta} ‘story’, \textit{qaṣət\nobreakdash-yāθa} ‘stories’; \textit{məllət\nobreakdash-ta} ‘community’, \textit{məllət\nobreakdash-yāθa} ‘communities’.

Similar forms are also attested in Jewish Challa (Lišāna Deni), but without the gemination of the /t/, e.g. \textit{məlləta} ‘ethnic group’, \textit{ʕādəta} ‘custom’ \citep[52]{Fassberg2010}. Rather than explaining the /t/ as originating in the Arabic suffixed {stem}, as I have done above, Fassberg suggests that the /t/ is present because the words were borrowed via (\ili{Northern}) \ili{Kurdish}, which realizes the \textit{tāʔ} \textit{marbūṭa} as a final /t/ even when the noun is unsuffixed: \textit{milet} [mɪˈlæt] and \textit{ʕadet} [ʕɑːˈdæt] \citep[387]{Chyet2003}. Khan (\citeyear[206]{Khan2002}) also suggests this route for Qaraqosh. This explanation would not explain why the unaffixed forms in Qaraqosh do not end in /t/, nor why the preceding vowel in all these dialects is /ə/ rather than /a/ (the nearest phonetic equivalent to \ili{Kurdish} 〈e〉). In fact, there are some clear loans of Arabic words via \ili{Kurdish} which end in \textit{{}-at} in the singular unsuffixed form (see §\ref{bkm:Ref13232790}). The \ili{Kurdish} route would furthermore not explain the close association in Qaraqosh of this morpheme with words taking an \textit{\nobreakdash-at} plural, which seems to have been borrowed directly from Arabic. It seems more likely, therefore, that the Qaraqosh, Alqosh and Challa feminine nouns with infixed \textit{\nobreakdash-ət\nobreakdash-} have been borrowed directly from Arabic and are influenced by the Arabic suffixed forms, which have this infix.



\subsubsection{\label{bkm:Ref13233345}Verbal derivation}

The NENA verbal system consists of both synthetic and analytic verb forms. The synthetic verb forms are formed from two stems, the Present Base and the Past Base, e.g. Christian Alqosh \textit{k\nobreakdash-šaql\nobreakdash-i} [\textsc{ind\nobreakdash-}take.\textsc{pres\nobreakdash-3pl]} ‘they take’ and \textit{šqəl\nobreakdash-lɛ} [take.\textsc{past\nobreakdash-3pl}] ‘they took’. Analytic forms involve {auxiliary verb(s)} or verboids combined with non-finite verb forms, such as the {infinitive} or participles, or, less often, with finite verb forms. Like Arabic, NENA has a verbal system based on the {root-and-pattern} system. As also in Arabic, a verb lexeme typically has a {triconsonantal} {root} and a verbal {derivational} class (see §\ref{bkm:Ref13233110}). While \ili{Standard} Arabic has ten fairly common triradical verbal derivations, NENA dialects typically have only three or four inherited verbal derivations.

Morphological loans may be found in the verbal system. Christian NENA dialects of the Nineveh Plain and elsewhere have partially borrowed Arabic verbal derivations along with borrowed verb lexemes. NENA and Arabic have some {cognate} verbal derivations and the relationships are relatively transparent. Most Arabic loanverbs are allocated to a NENA {derivation} that is formally or functionally similar to the donor {derivation} (and often {cognate}). See §\ref{bkm:Ref13233110} for discussion of this. In the case of Arabic verbal derivations VIII and X, however, this is not possible, as no NENA derivations have the characteristic affixes \textit{{}-t-} and \textit{(i)st-}. In some cases, the affix may instead be analysed as a radical (§\ref{bkm:Ref13233110}). In others, loanverbs in these derivations are borrowed with this {derivational} morphology, i.e. with the affixes. This has, in effect, created new derivations, the Ct- and St- derivations.

\tabref{tab:coghill:1} gives all hitherto attested examples of verbs in the new derivations from Christian Telkepe, but additional verbs are attested in Christian Qaraqosh \citep[130]{Khan2002}.

\begin{table}
\caption{Arabic loanverbs borrowed into the new NENA derivations\label{tab:coghill:1}}
\begin{tabularx}{.75\textwidth}{QQ}
\lsptoprule
{NENA verb} & {Source verb}\\\midrule
{\textit{√ḥrm} Ct- ‘to respect’} & {Ar. \textit{√ḥrm} VIII (\textit{iḥtarama})}\\
\textit{√xlf} Ct- ‘to differ’ & {Ar. \textit{√ḫlf} VIII (\textit{iḫtalafa})}\\
{\textit{√ḥfl} Ct- ‘to celebrate’} & {Ar. \textit{√ḥfl} VIII (\textit{iḥtafala})}\\
{\textit{√ʕml} St- ‘to use’} & {Ar. \textit{√ʕml} X (\textit{istaʕmala})}\\
\textit{√\.gll} St- ‘to exploit’ & {Ar. \textit{√ɣll} X (\textit{istaɣalla})}\\
\lspbottomrule
\end{tabularx}
\end{table}

When Arabic verbs in derivations VIII and X are borrowed as they are, their characteristic consonantal clusters \textit{{}-Ct-} and \textit{{}-st-} are preserved and not broken up by an epenthetic vowel, even if this results in a {syllabic structure} that is dispreferred in the NENA dialect (such as a stressed short vowel in an open syllable), e.g. \textit{k-ma}\kern -0.5pt\textbf{\textit{ḥt}}\kern -1pt\textit{arəm} [\textsc{ind}\nobreakdash-respect.\textsc{pres.}3\textsc{sg.m}] ‘he respects’. This may be in order to preserve a salient characteristic of the original Arabic forms.

The vowel pattern in these derivations is, on the other hand, variable, even within the speech of one speaker. For instance, in the Present Base of the St-{derivation}, we find məstaCaCC-, məstaCCəC- and məstaCəCC- (e.g. \textit{məstaʕaml-,} \textit{məstaʕməl-,} \textit{məstaʕəml-} ‘use’) as variants of one and the same form. What are the reasons for this variability? Firstly, Arabic derivations VIII and X are morpho-phonemically more complex than the native Aramaic derivations. The {consonant clusters} bring the necessity of epenthetic vowels: this leads to at least one short vowel in an open syllable, which is disfavoured in Telkepe. Where the epenthetic vowel is placed is still optional and in flux. Secondly, there is a conflict between the characteristic vowels of the \ili{Iraqi} Arabic source and the vowels typical of Aramaic derivations. Sometimes the former may be more influential and sometimes the latter.

The new Ct- and St- derivations in NENA have not been extended to inherited {roots} nor used productively, unlike some Arabic derivations in \ili{Western} Neo-Aramaic. See \citet{Coghill2015} for full details of the new derivations found in NENA, \ili{Western} Neo-Aramaic and other Neo-Aramaic varieties.

\subsection{\label{bkm:Ref534214095}Syntax and pattern borrowings}

A syntactic borrowing attested only in the Christian Nineveh Plain dialects is the {grammaticalization} of a {prospective} auxiliary (and, as a further step, uninflected particle) on the model of the vernacular Arabic {prospective} {future} particle \textit{raḥ-}, which is attested in nearby \ili{Mosul} Arabic (author’s fieldwork), as well as more widely across the \ili{Syrian} and \ili{Mesopotamian} Arabic dialects \citep[304]{Jastrow1978}. Example (\ref{bkm:Ref534224242}) shows the Neo-Aramaic construction (with the particle) and example (\ref{bkm:Ref534224256}) shows the Arabic construction.

\protectedex{
\ea\label{bkm:Ref534224242}Christian Telkepe NENA (author’s fieldwork)\\
\gll zi-napl-ɒ\\
     \textsc{prsp-}fall.\textsc{pres}{}-\textsc{3sg.f}\\
\glt ‘She’s going to fall.’\z
}

\protectedex{
\ea\label{bkm:Ref534224256}Christian \ili{Mosul} Arabic (author’s fieldwork)\\
\gll ɣāḥ-təqaʕ š-šaǧaɣa!\\
     \textsc{prsp-}fall.\textsc{impf.3sg.f} \textsc{def}{}-tree\\
\glt ‘The tree’s going to fall!’\z
}

In both cases the gram has developed from a verb ‘to go’ in a form with imperfective or imperfective-like functions.\footnote{In the case of the Nineveh Plain dialects, it originates in a verb that originally had perfect aspect, e.g. \textit{zil-ən} ‘I have gone’, possibly with the implication of ‘I am on my way’. It had also acquired a meaning of imminent {future} ‘I am about to go’, in effect ‘I am in the process of just leaving’, hence ``imperfective-like functions''.} Such a development is of course extremely common in the world’s languages and does not need a contact explanation. Nevertheless, there is evidence that contact played a role. The construction is only found in NENA dialects close to the Arabic-speaking zone of Iraq, i.e. near to \ili{Mosul}. Furthermore, the most mature versions of the gram (formally and functionally) are found in the villages closest to \ili{Mosul}. The gram seems to have developed only in the last 100 years or so, as it is not attested in texts or mentioned in grammars of those dialects before then. See Coghill (\citeyear{Coghill2010,Coghill2012}) for more details.

NENA shares a number of {idiomatic} expressions with neighbouring languages. Among these are formulae used regularly in specific contexts, such as telling a story or expressing thanks, congratulations or condolences. One that is widespread in NENA dialects, as well as several neighbouring languages, is the opening formula to a fictional story, which begins ‘there was (and) there wasn’t’: see also Chyet (\citeyear{Chyet1995}: 236–237). It is attested in various dialects of NENA, \ili{Ṭuroyo}, \ili{Kurdish}, \ili{Azeri}, \ili{Persian} and Arabic, e.g.:

\begin{itemize}[noitemsep]

\item[] Christian Alqosh NENA   \textit{ʔəθwa꞊w} \textit{laθwa} \citep[268]{Coghill2009}

\item[] Christian Bohtan NENA   \textit{ətwa} \textit{lətwa} \citep{Fox2009}

\item[] Akre \ili{Kurdish}   \textit{hebo} \textit{nebo} [hæˈboː næˈboː] (\citealt{MacKenzie1962}: 288) 

\item[] Iranian \ili{Azeri}   \textit{(bir)} \textit{vármɨš} \textit{(bir)} \textit{jóxmuš} \citep[175]{Garbell1965}

\item[] \ili{Christian Bəḥzāni Arabic}   \textit{kān} \textit{w} \textit{ma} \textit{kān} \citep[404]{Jastrow1981}\footnote{This is a variant (along with \textit{kān} \textit{ma} \textit{kān}, attested in \ili{Palestinian} Arabic) of the well-known formula \textit{kān} \textit{yā} \textit{ma} \textit{kān} ‘once upon a time’. While \textit{kān} \textbf{\textit{w}} \textit{ma} \textit{kān} clearly means ‘there was and there was not’, \textit{kān} \textbf{\textit{yā}} \textit{ma} \textit{kān} has been interpreted in different ways both by scholars and native speakers. Taking \textit{yā} \textit{ma} in its meaning of ‘how much’, it can be understood as ‘there was, how much there was!’ Alternatively, the \textit{ma} is understood as a negator, as is found in the formula in the other languages. See \citet{Lentin1995} for a discussion of \textit{kān} \textit{yā} \textit{ma} \textit{kān} and similar expressions.}
\end{itemize}

When such formulae are shared by multiple regional languages, it is difficult to say for certain which language NENA borrowed them from. \ili{Kurdish} is usually the assumed donor, simply because it is the language most in contact with NENA and which has had the greatest influence at all levels. Given, however, that many speakers knew other regional languages as well, they may have heard such expressions in several languages.

Proverbs are another area in which there are shared expressions across the regional languages (\citealt{Chyet1995}: 234–236; \citealt{Garbell1965}: 175; \citealt{Segal1955}). An example is ‘He who knows, knows. He who doesn’t know, says “a handful of lentils”.’ This stems from a folktale and means something like ‘looks can be deceiving’ (\citealt{Chyet1995}: 235–236). It is attested in \ili{Kurdish}, \ili{Iraqi} Arabic, and NENA, as illustrated in (\ref{bkm:Ref534225045})–(\ref{bkm:Ref534225057}).

\ea\label{bkm:Ref534225045}\ili{Iraqi} Arabic \citep[235]{Chyet1995} \\
\gll il-yidrī   yidrī     w-il ma yidrī     gað̣bit ʕadas\\
    \textsc{rel}\textup{{}-know.}\textsc{impf.3sg.m} \textup{know.}\textsc{impf.3sg.m} \textup{and-}\textsc{rel} \textsc{neg} \textup{know.}\textsc{impf.3sg.m} handful\textsc{.cs} lentils\\
\glt ‘He who knows knows, he who doesn’t know (says) “a handful of lentils”.’\z

\ea\label{bkm:Ref534225057}Jewish Zakho NENA (Segal \citeyear[262]{Segal1955}, adapted transcription)\\
\gll aw d-k-īʔe   k-īʔe     aw d-lá k-īʔe   g-mēnüx bi-ṭloxe\\
    \textsc{3sg.m} \textsc{rel}\textup{{}-}\textsc{ind-}\textup{know}\textsc{.pres.3sg.m} \textsc{ind}\textup{{}-know.}\textsc{pres.3sg.m} \textsc{3sg.m} \textsc{rel}\textup{{}-not} \textsc{ind}\textup{{}-know.}\textsc{pres.3sg.m} \textsc{ind-}\textup{look.}\textsc{pres}.\textsc{3sg.m} \textup{at-lentils}\\
\glt ‘He who knows knows, he who doesn’t know looks at a handful of lentils.’\z

\citet{Sabar1978}, who lists proverbs used by the {Jews} of Zakho, states also that many proverbs were not translated into NENA, but used in the original language, whether \ili{Kurdish} or Arabic.

There are also some areas of structural {convergence} in the region’s languages, where the donor language cannot be definitely identified. For instance, all the languages (NENA, \ili{Sorani}, \ili{Northern Kurdish}, \ili{Persian}, \ili{Turkish}, \ili{Azeri}, Iraqi \ili{Turkmen} and \textit{qəltu} Arabic) have enclitic copulas, as illustrated in (\ref{bkm:Ref534225894})–(\ref{bkm:Ref534225903}).

\ea\label{bkm:Ref534225894}Akre \ili{Kurdish} \citep[175]{MacKenzie1961}\\
\gll ew kî꞊e \textup{[æw ˈkiːæ]}\\
     \textsc{dem} who꞊\textsc{prs.cop.3sg}\\
\glt ‘Who is that?’ 
\z

\ea Christian Telkepe NENA (author’s fieldwork)\\
\gll man꞊ilə\\
     who꞊\textsc{prs.cop.3sg.m}\\
\glt ‘Who is he?’\z

\ea\label{bkm:Ref534225903}Jewish Arbel Arabic (\citealt{Jastrow1990book}: 37, 46) \\
\gll mani꞊we\\
     who꞊\textsc{3sg.m}\\
\glt ‘Who is he?’\z

Another shared structure is the use of finite subordinate clauses in subjunctive mood, rather than infinitives, as complements. In earlier Aramaic varieties, such as Classical \ili{Syriac}, both were used (\citealt{Nöldeke1904Syriac}: 224–226), but in NENA only finite verbs are used, as in example (\ref{bkm:Ref534226048}).

\ea\label{bkm:Ref534226048}Christian Telkepe NENA\\
\gll k-əbə       d-āxəl\\
    \textsc{ind-}\textup{want.}\textsc{pres}.\textsc{3sg.m} \textsc{comp}\textup{{}-eat.}\textsc{pres}.\textsc{3sg.m}\\
\glt ‘He wants to eat.’\z

Finite verbs in an irrealis mood are also used in such subordinate clauses in \textit{qəltu} (and other vernacular) Arabic (e.g. \citealt{Jastrow1990book}: 65), \ili{Northern Kurdish} (\citealt{MacKenzie1961}: 208–209), \ili{Sorani} (\citealt{MacKenzie1961}: 134–135), Iraqi \ili{Turkmen} (\citealt{Bulut2007}: 175–176), and Iranian \ili{Azeri} (Fariba Zamani, personal communication). The development in \ili{Turkic} is attributed to \ili{Iranian} influence (\citealt{Bulut2007}: 175–176). This parallels the loss of the {infinitive} and its replacement by finite verb forms in the Balkan Sprachbund (see, e.g., \citealt{Joseph2009}).

The existence of markers in the {noun phrase} to specify for indefiniteness (and in many cases specificity, e.g. ‘a certain man’) is widespread in the area, being found in NENA (\textit{xa-} ‘one, a (certain)’), \ili{Northern Kurdish} (\textit{\nobreakdash-ek} [ɛk] < \textit{yek} `one'), \ili{Sorani} (\textit{\nobreakdash-ēk} [eːk] < \textit{yek}), \textit{qəltu} Arabic (\textit{faɣəd} < \textit{fard} `individual'), \ili{Baghdadi} Arabic (\textit{fadd} < \textit{fard}) and \ili{Turkish}/\ili{Azeri} (\textit{bir} `one').


\section{Conclusion}

Though not the dominant contact language, Arabic has influenced NENA dialects considerably, especially those in close contact with Arabic-speaking population centres, namely in the Christian Nineveh Plain dialects, the Jewish Lišāna Deni dialects and in the Christian dialects in Şırnak province in Turkey.

The influence from Arabic is manifested mostly in lexicon, phonology and morphology, and less in syntax.

Arabic influence has occurred in different phases. Earlier Arabic influence was mostly indirect, via \ili{Kurdish} loans, but direct borrowing seems to have occurred too.

In the twentieth and twenty-first centuries, Arabic influence has increased dramatically in the dialects spoken in Iraq, due to mass education exclusively in Arabic, as well as national media, military service, improved transport, and migration to the \ili{Iraqi} cities. Most NENA speakers are bilingual and speak Arabic with native competence, and this has affected how they use Arabic words within their own language. Typically, recent loans are unadapted and close to {code-switching}.

As much of the fieldwork on which this description depends was undertaken in the late twentieth century or first few years of the twenty-first century, in {future} research it would be interesting to look at the speech of young people today and see whether much has changed. It would also be worth comparing the speech of communities in their ancestral villages with {diaspora} communities living in (or who have recently left) Baghdad or Basra.

\section*{Further reading}

Most work on NENA and language contact has focused on contact with \ili{Kurdish}. To my knowledge, only three works are dedicated to contact with Arabic, none of which is an overview: Sabar's (\citeyear{Sabar1984}) study of Arabic influence in the early texts in Jewish Lišāna Deni; Coghill’s (\citeyear{Coghill2010,Coghill2012}) research into a {prospective} construction found in the Christian Nineveh Plain dialects, which has apparently grammaticalized under influence from Arabic; and Coghill's (\citeyear{Coghill2015}) study of new verbal derivations borrowed from Arabic into various Neo-Aramaic languages, including NENA.

Khan's (\citeyear{Khan2002}) grammar of Christian Qaraqosh contains a great deal of information, scattered through the volume, about contact influences from Arabic, Qaraqosh being one of the dialects most affected by such influence.

\section*{Acknowledgements}

I would like to thank all the Neo-Aramaic speakers who have generously given of their time in my fieldwork and the fieldwork of other scholars which is cited in this chapter. Much of my research on NENA and language contact took place in the \ili{German} Research Foundation-funded project, \textit{Neo-Aramaic} \textit{morphosyntax} \textit{in} \textit{its} \textit{areal-linguistic} \textit{context.} I would also like to thank the editors of this volume for their helpful comments.

\section*{Abbreviations}

\begin{tabularx}{.55\textwidth}{@{}lQ@{}}
\textsc{1, 2, 3} & 1st, 2nd, 3rd person \\
Ar.            & Arabic\\
\textsc{comp}    & {complementizer}\\
\textsc{cop}     & {copula}\\
\textsc{cs}     & construct state\\
CSyr          & Classical Syriac\\
\textsc{dem}     & demonstrative \\
\textsc{f}/f.    & feminine\\
\textsc{impf}   & Imperfect (prefix conjugation)\\
\textsc{ind}     & indicative\\
\end{tabularx}%
\begin{tabularx}{.5\textwidth}{@{}lQ@{}}
\textsc{m}/m.    & masculine\\
N. Kurd.        & Northern Kurdish \\ 
\textsc{neg}    & negator\\
\textsc{past}   & NENA Past Base\\
\textsc{pl}/pl.  & plural\\
\textsc{pres}    & NENA Present Base \\
\textsc{prs}    & present \\
\textsc{prsp}    & {prospective} \\
\textsc{rel}     & {relativizer}\\
\textsc{sg}  & singular
\end{tabularx}%

\section*{Symbols}

\begin{tabularx}{1\textwidth}{@{}lQ@{}}
I, II, III etc. & Arabic verbal derivations\\
I, II, III, Q & NENA verbal derivations\\
= & links two words or morphemes in a phrase with a single {stress} on the second component (including but not limited to proclitics)\\
꞊ & links two words or morphemes in a phrase with a single {stress} on the first component (including but not limited to enclitics)
\end{tabularx}%

{\sloppy\printbibliography[heading=subbibliography,notkeyword=this]}
\end{document}
