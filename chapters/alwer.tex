\documentclass[output=paper]{langsci/langscibook} 
\ChapterDOI{10.5281/zenodo.3744549}
\author{Enam Al-Wer\affiliation{University of Essex}}
\title{New-dialect formation: The Amman dialect}
\abstract{One fascinating outcome of dialect contact is the formation of totally new dialects from scratch, using linguistic stock present in the input dialects, as well as creating new combinations of features, and new features not present in the original input varieties. This chapter traces the formation of one such case from Arabic, namely the dialect of Amman, within the framework of the variationist paradigm and the principles of new-dialect formation.}

\begin{document}
\maketitle 
\section{Contact and new-dialect formation} \label{new}
\subsection{Background and principles}\largerpage

The emergence of new dialects is one of the possible outcomes of prolonged and frequent contact between speakers of mutually intelligible but distinct varieties. The best-known cases of varieties that emerged as a result of contact and mixture of linguistic elements from different dialectal stock are the so-called \isi{colonial} varieties, namely those varieties of \ili{English}, \ili{French}, \ili{Spanish} and \ili{Portuguese} which emerged in the former colonies in the {Southern} Hemisphere and the Americas.\footnote{Among the studies that investigated such varieties are: \citet{Trudgill2004}, \citet{GordonEtAl2004}, \citet{Sudbury2000} and \citet{Schreier2003} for \ili{English}; Poirier (\citeyear{Poirier1994}; cited in \citealt{Trudgill2004}) for \ili{French}; \citet{Lipski1994} and \citet{Penny2000} for \ili{Spanish}; and \citet{Mattoso1972} for \ili{Portuguese}.} In addition to \isi{colonial} situations, the establishment of new towns can also lead to the development of new dialects; a case in point is Milton Keynes (UK), which was investigated by Paul Kerswill.\footnote{See \citet{KerswillWilliams2005}.} For \ili{Arabic}, similar situations of contact are abundant, largely due to voluntary or forced displacement of populations, growth of existing cities and the establishment of new ones. To date, however, the only study of a brand \isi{new dialect} is the on-going investigation of the dialect of Amman, the capital city of Jordan, which is anticipated to provide a model for the study of \isi{dialect contact} and koinéization in other burgeoning conurbations elsewhere in the Arab World. The bulk of this chapter will be dedicated to the details of this case.

Several other studies in Arab cities have focused on contact as a primary agency through which innovations permeate the speech of migrant groups. Although no new dialects emerge in such situations, new patterns and interdialectal forms are common. For instance, \citet{Al-Essa2009} reports that among the residents of the city of Jeddah, those who originally emigrated from various locations in Najd generally converge to the dialect of Jeddah, but also use innovations that do not occur in the target dialect, such as the second person singular feminine suffix -\textit{ki} in words ending in a consonant, as in \textit{ʔumm-ki} for \ili{Najdi} \textit{ʔumm-its} and Jeddah \textit{ʔumm-ik} ‘your (\textsc{sg.f}) mother’. Similarly, \citet{Alghamdi2014} found interdialectal forms of the \isi{diphthongs} /ay/ and /aw/ (viz. narrow diphthongal variants [ɛi], [ɔʊ]), as well as the \isi{monophthongs} [ɛː] and [ɔː], in the speech of {Ghamdi} migrants who originally came to \isi{Mecca} from Al-Bāḥa in the southwest of \isi{Saudi Arabia}. In \ili{Casablanca}, rapid urbanization led to immigration of large numbers of groups from all over Morocco, and subsequent contact between different dialects. Hachimi (\citeyear[97]{Hachimi2007}) suggests that this situation resulted in “the disruption of the rural/urban dichotomy that once dominated \ili{Moroccan} dialects and identities”, and the emergence of new categories of identification, which are symbolized through the usage of a mixture of features from different dialects.

In this context, it is worth pointing out some methodological challenges concerning the measurement of contact as an independent variable in quantitative sociolinguistics, and some improvements that have been made in research on \ili{Arabic}. Contact is often invoked as an explanatory factor in contact linguistics in general, and has indeed been incorporated in theoretical formulations (e.g. \citealt{ThomasonKaufman1988}). In quantitative sociolinguistics, however, analysis of contact as a constraint on linguistic variation requires treating it as a variable from the outset of research, and finding ways to quantify it, in esssentially the same way that social categories such as age, \isi{gender} and class are factored into the analysis. But how can contact be quantified? Recognizing the crucial role that (dialect) contact plays in the structure of variation and mechanisms of change, a number of quantitative studies have tested various methods of quantification. Al-Essa's (\citeyear{Al-Essa2009}) study, mentioned above, was the first known quantification of contact in studies of this sort. In order to do this, she measured the speakers’ level of exposure to the target features through an index, consisting of a four-point scale, which gave a numerical value to each speaker’s level of contact. Four criteria were used to determine the numerical value assigned to each speaker: friendships at school and work; involvement in neighbourhood affairs; friendship with speakers of the target dialect; kinship and intermarriage in the family \citep[208]{Al-Essa2009}. Alghamdi's (\citeyear{Alghamdi2014}) study in \isi{Mecca} utilized and adapted Chambers' (\citeyear{Chambers2000}) concept of \isi{regionality}, by devising a \isi{regionality} index based on the speakers’ date of arrival in the city and place of residence. In \citet{Al-Wer2002}, I suggested that in some cases level of education may be treated as an indication of level of contact with outside communities; and \citet{Horesh2014} elicited information that was indicative of levels of contact between the speakers’ L1 \ili{Arabic} and L2 \ili{Hebrew}, which were later converted into factor groups, one of which was language of education, thus demonstrating that type of education can also be used to measure contact.\footnote{Several additional doctoral theses completed at the University of Essex address this issue.}

\subsection{Theoretical framework}

The study of the \isi{formation} of new dialects is credited particularly to the work of Peter Trudgill. In his \textit{Dialects} \textit{in} \textit{contact} (\citeyear{Trudgill1986}) he laid the theoretical foundations of research in the field, arguing that ``face-to-face interaction'' is a prerequisite for linguistic adaptation and \isi{diffusion} of linguistic innovations.\footnote{\citet{Trudgill1986} integrated insights from Accommodation Theory \citep{Giles1973} in the study of \isi{dialect contact}.} Focusing on the \isi{formation} of \ili{New Zealand} {English}, \citet{Trudgill2004} suggests a three-stage approach to dialect \isi{formation}, which roughly corresponds to three successive generations of speakers.\footnote{In the same year, and based on the same data, the team working on the Origins of \ili{New Zealand} {English} (ONZE) project, in which Peter Trudgill participated, also published a co-authored book on the topic (see \citealt{GordonEtAl2004}).} These stages are very briefly summarized below, and illustrated using examples from Amman in §\ref{three}.\footnote{Trudgill (\citeyear{Trudgill2004}: 83–128) discusses and illustrates each stage with data from the ONZE corpus.}

\begin{altdescription}
\item[Stage I (first generation):] rudimentary \isi{leveling}. This stage stipulates that at the initial point of contact and interaction between adult speakers of different regional and social varieties, minority and very localized linguistic features are leveled out.
\item[Stage II (second generation):] variability and mixing. At this stage, the first lo\-cal\-ly-born generation of children are presented with a plethora of features to choose from. Their speech contains considerable inter-individual and intra-individual variability, and new combinations of features.
\item[Stage III (third generation):] emergence of stable and relatively uniform dialect. At this stage, \isi{focusing} (\citealt{LePageTabouret-Keller1985}; see §\ref{focussec}) gives rise to a crystalized dialect. 
\end{altdescription}

Trudgill (\citeyear[149]{Trudgill2004})\footnote{See also \citet{TrudgillEtAl2000} and \citet{Trudgill2008}.} concludes that the processes of dialect mixture and new-dialect \isi{formation} are not haphazard but “deterministic in nature”, “mechanical and inevitable”, and that, in \textit{tabula rasa} situations, social and attitudinal factors do not play a role in the \isi{formation} of new dialects.\footnote{Cf. Labov's (\citeyear{Labov2001}) principle of density.} “Determinism” in new-dialect \isi{formation} and “the minor role that \isi{social factors}, such as \isi{identity}, play in \textit{tabula rasa} situations” have instigated a wide and interesting debate among scholars. For instance, Tuten (\citeyear[261]{Tuten2008}) proposes that “community \isi{identity} \isi{formation} and koiné \isi{formation} are simultaneous and mutually dependent processes”. Mufwene (\citeyear[258]{Mufwene2008}) agrees that common \isi{identity} “is not part of the processes that produce new dialects”; but rather a by-product of it. \citet{Schneider2008} elaborates on two issues: the relationship between \isi{accommodation} and \isi{identity}, and “the changing role of \isi{identity}” in different \isi{colonial} and postcolonial phases (\citeyear{Schneider2008}: 262), pointing out cases of features from \isi{colonial} varieties where the origins and spread of these features coincided with “a heightened national or social awareness” (\citeyear{Schneider2008}: 266). \citet{Bauer2008} contests Trudgill’s implicit suggestion that \isi{accommodation} leads directly to dialect mixing, on the basis that individuals vary in the extent to which they accommodate to others, and vary depending on the context; and in some cases no \isi{accommodation} takes place, that is, \isi{accommodation} is sporadic. He maintains that “it is not the \isi{accommodation} as such that leads to dialect mixing; rather, it is the use that \isi{accommodation} is put to by the next generation that leads to dialect mixing” (\citeyear{Bauer2008}: 272). On the role of \isi{identity}, Bauer contends that the very choice of a particular variant over another is indirectly an expression of “complex kinds of \isi{identity}” (\citeyear{Bauer2008}: 273).\footnote{For more details, see \citet{Bauer2008}; and for Trudgill’s responses to these points, see the discussion and rejoinder in \textit{Language} \textit{in} \textit{Society,} 2008, vol. 37.}

\subsection{Mechanisms}

The mechanisms involved in new-dialect \isi{formation} fall under two broad headings: \textit{koinéization} and \textit{focusing}. Below are brief explanations of these mechanisms, to be followed by illustrations from data from Amman in the relevant sections.


\subsubsection{Koinéization}

Trudgill (\citeyear{Trudgill2004}: 84–88) uses \textit{koinéization} as an umbrella term to refer to five processes, which operate at the same or different stages in the \isi{formation} of new dialects: (i) mixing, which, as the name suggests, involves the use of features which originally came from different dialects; (ii) \isi{leveling}, which involves gradual reduction and ultimate loss of minority features, that is, features that have least representation in the dialect mix; (iii) unmarking, a sub-type of \isi{leveling}, which refers to the survival of unmarked and more regular forms even if they are not the majority forms; (iv) \isi{interdialect} development, which are forms that arise out of interaction between different forms in the original mix, and can include phonetically, morphologically and syntactically  intermediate forms; (v) \isi{reallocation}, which refers to the survival of more than one variant of the same feature, which then undergoes \isi{reallocation} in the new system; \isi{reallocation} can be linguistic, social or stylistic.

\subsubsection{Focusing} \label{focussec}

This term was introduced into sociolinguistics by \citet{LePageTabouret-Keller1985} to refer to the process whereby the new system “acquires norms and stability”. A focused dialect contrasts with a diffuse (or non-focused) linguistic situation, where there is no consensus over norms, and no stability of usage.\footnote{See Le Page \& Tabouret-Keller (\citeyear{LePageTabouret-Keller1985}: 181–182).}

\section{Dialect formation in Amman}

\subsection{History and demographics}

Amman has no traditional dialect simply because until relatively recently it had no indigenous inhabitants. Though an important centre in ancient times, it remained largely deserted until the early years of the twentieth century.\footnote{Amman’s ancient history is traced to the Ammonites (eighth century BCE), who called it \textit{Rabbath} \textit{Ammon} ‘the great (or royal) city of the Ammonites’; the Romans changed its name to \textit{Philadelphia;} the Arab Ummayads took over in the seventh century CE and restored its \ili{Semitic} name, \textit{Amman}.}

In 1921, it was designated as the capital of Transjordan (the land east of the River Jordan), which became the Kingdom of Jordan in 1946. It thus attracted migrants from other parts of the country, as well as from Palestine, Syria and Lebanon. By the 1930s, the population had grown to 10,000 inhabitants, and by 1946 it stood at 65,000. The early migrants consisted of two groups: (i) the majority were economic migrants (traders and shop keepers as well as labourers) or civil servants, who were appointed in the state administration; (ii) the rest were political activists (mostly individuals from Syria and Lebanon, which were then still under French colonialist rule). The first group included families from both sides of the River Jordan, namely indigenous Jordanians from the east side, and Palestinians from all parts of historical Palestine. Statistics regarding numbers from each group are unavailable, but I was able to collect fairly reliable information, through ethnographic interviews, about the provenance of a large sector of the first generation of migrants. According to my research, the vast majority came from two particular locations: the Jordanian city of Sult (20 kilometres northwest of Amman), and the Palestinian city of Nablus (110 kilometres from Amman).

The city continued to receive waves of migrants from other locations in Jordan and from Palestine, especially following the two wars in 1948 and 1967, which resulted in the occupation of historical Palestine, and the displacement of well over three million Palestinians over the years, most of whom sought refuge in Jordan. Between 1950 and 1990, the population of Amman doubled more than fifteen times, to reach approximately two million by 2004. According to the 2018 census estimate, the city is home to 2,554,923 Jordanian nationals, and 1,452,603 non- Jordanians, that is, a total of over four million people live in the city currently.\footnote{Department of Statistics, Jordan: \url{http://dosweb.dos.gov.jo/DataBank/Population_Estimares/PopulationEstimates.pdf} (accessed 06/01/2020).} Given the political situation in the region, the population of Amman is forecast to reach six million by 2025. 

Against this demographic background, there are three important points to note:

\begin{enumerate}
\item 
There is no geographically neutral variety of spoken \ili{Jordanian} \ili{Arabic}. All speakers therefore use some form of local dialect, regardless of social class.

\item 
Whereas in neighbouring countries (Syria, Egypt, Lebanon, Palestine), the dialect of the capital acts as a standard \isi{prestigious} norm, Jordan never had a linguistic centre of its own.

\item 
Jordanians, and Palestinians, generally identify themselves with the area in which their forebears lived, rather than the locality in which they were born and bred. However, recently a growing number of inhabitants of Amman (particularly third and fourth generations) have begun to identify themselves as ``Ammanis''.

\end{enumerate}

The emergence of a distinctive and focused dialect of Amman, in tandem with the emerging Ammani \isi{identity}, represents a radical shift in the sociolinguistic patterns from a plethora of local varieties to a situation similar to that described for neighbouring states in §\ref{new}.

\subsection{The Amman Project}

This research traces the \isi{formation} of this \isi{new dialect} from inception to stabilization over three generations, spanning a period of approximately the last eighty years. It initially focused on generational differences, by investigating the developments in the speech of three generations of families who originally came from the Jordanian city of Sult and the Palestinian city of Nablus; this initial investigation confirmed the following hypotheses:

\begin{itemize}
\item A \isi{new dialect} has emerged and its usage has stabilized.

\item This dialect is unique -- it grew as an outcome of the contact between \ili{Jordanian} and urban \ili{Palestinian} dialects, but is distinct from these input varieties.

\item The \isi{formation} of a distinctive dialect in Amman is closely associated with \isi{relative} stabilization in the population, possibly during 1970–1990, and the development of an Ammani community with its own \isi{identity}.\footnote{Full details can be found in Al-Wer (\citeyear{Al-Wer2002},\citeyear{Al-Wer2003kum}).}

\end{itemize}

The second phase of the research focused on the younger generation from affluent West Amman; and the final phase, ongoing, expands the sample to include speakers from less affluent East Amman. Altogether, the research aims to collect data from approximately 120 speakers, from both sides of the city. The project on Amman itself is complemented by past and ongoing research (by myself and others) on sociolinguistic trends in areas outside Amman, which provide two valuable types of relevant information: (i) further evidence of the input varieties; (ii) spreading of innovative features of the Amman dialect to other parts of the country.

The framework of analysis adopted here is the Variationist Sociolinguistic Paradigm, as described in Labov’s trilogy (\citeyear{Labov1994,Labov2001,Labov2010}). More specifically, the project is guided by the principles of \isi{dialect contact} and new-dialect \isi{formation}, as outlined in §\ref{new}. As discussed earlier, one of the dominant issues in the study of the \isi{formation} of new dialects is the debate over the types of factors which determine it. The Amman project offers an opportunity to investigate these issues in detail, particularly because it is still possible to trace the different stages of \isi{formation} over the three generations of native inhabitants.

\subsection{Formation over three generations} \label{three}

Based on the analysis of speech samples from three generations of Ammanis, the \isi{formation} of the dialect is a textbook case. Many of the processes of koinéization explained above are operative, as will be demonstrated presently.

\subsubsection{Stage I: first generation}

The first generation arrived in Amman during the 1930s as adults. The most noticeable aspect of their speech is that it can easily be identified with the original dialects of the places from which they migrated, while localized features are leveled out (cf. rudimentary \isi{leveling}; \citealt{Trudgill2004}). The features which are lost at this stage are summarized below.

\begin{description}[wide,nosep]
\item[\textit{\ili{Jordanian} input}.] Traditional \ili{Jordanian} dialects, including the dialect of Sult, are known to affricate /k/ to [ʧ] in front-vowel environments generally, as in /keːf/ > [ʧeːf] > ‘how’. This feature is still widely used, especially in northern varieties, as well as in Sult\footnote{On this feature in Traditional \ili{Jordanian} dialects, see \citet{Al-Hawamdeh2015}, \citet{Herin2010} and  \citet{HerinAl-wer2013}.} – where most of the early migrants in Amman came from. Already in the first generation, this feature is completely lost; all instances of this variable were rendered with [k]. In other words, first-generation speakers deaffricate /k/.\footnote{See \citet{Al-Wer2007} for more details.} Although \isi{conditional} \isi{affrication} of /k/ is fairly widespread in the region’s rural dialects, and is certainly not a minority feature in \ili{Jordanian} dialects, its use is heavily stigmatized, and none of the urban dialects have it. Stigmatization is the likely reason that motivates the loss of this feature.

Also characteristic of the traditional dialects is the maintenance of a \isi{gender} distinction in the second and third person plural pronouns, and pronominal, verbal and nominal suffixes. For example: \textit{ʔintu} ‘you (\textsc{pl.m})’, \textit{ʔintin} ‘you (\textsc{pl.f})’; \textit{ʔumm-hum} ‘their (\textsc{m}) mother’, \textit{ʔumm-hin} ‘their (\textsc{f}) mother’; \textit{rāḥu} ‘they (\textsc{m}) have left’, \textit{rāḥin} ‘they (\textsc{f}) have left’; \textit{ḥilwīn} ‘pretty (\textsc{pl.m})’, \textit{ḥilwāt} ‘pretty (\textsc{pl.f})'. What we find in Amman is \isi{gender} neutralization in these forms, such that the masculine form is used to refer to both genders. The traditional system is currently variable in all major Jordanian cities, and seems to be giving way to a neutralized form, as in Amman, which is an indication that Amman has become a focal point from which linguistic innovations radiate. No particular social value is attached to the traditional feature (maintenance of \isi{gender} distinction), although there is awareness that it is characteristically found in provincial towns and villages. The observation that it has become variable in many cities and towns means that it is also becoming a minority feature in urban areas in particular. The change affecting this feature in \ili{Jordanian} dialects in general may be described as a form of \isi{simplification}, where the number of distinct forms in the paradigm as a whole is reduced. Additionally, none of the urban \ili{Palestinian} dialects maintain a \isi{gender} distinction in these categories. In contact situations especially, the direction of change is normally towards the simpler system (the “Simplification Preference”; \citealt{Lass1997}: 253).\footnote{It should be pointed out that the urban \ili{Palestinian} dialect, similarly to all city dialects in the region, has the \textit{{}-on and -kon} endings, which are used with both genders, rather than masculine \textit{{}-um} and \textit{{}-ku}, which are the koiné forms in modern \ili{Jordanian} dialects; see \citet{Al-Wer2003kum} for more details about this feature.} 

\item[\textit{\ili{Palestinian} input}.] In urban \ili{Palestinian}, the high-\isi{frequency} terms \textit{mbāriḥ} `yesterday’ and \textit{sāʕa} ‘hour/time’ are pronounced with raised vowels: \textit{mbēriḥ} and \textit{sēʕa.} This is an extremely marked pronunciation in the context of Jordan, as no \ili{Jordanian} dialect has it. It is also a feature that is overtly commented upon, and often used to mimic dialects that have it. Extreme raising of /ā/ generally is a hallmark of many urban \ili{Palestinian} dialects, most notably in the dialect of \ili{Jerusalem}; as will be explained later, third-generation speakers with urban Palestinian heritage use considerably lower variants than first- and second-generation speakers from the same group. It is possible that lowering in these high-\isi{frequency} items in the first generation is the onset of the change that escalated in successive generations. In the third generation of this group, the speakers change their pronunciation in these items only, but continue to use noticeably higher variants of /ā/ in other items, for example, ‘Amman’ is pronounced as [ʕəmmɛːn]; \textit{fālit} ‘loose’ is pronounced as [fɛːlɪt].
\end{description}

\subsubsection{Stage II: second generation}
This is the first locally-born generation; the majority of the speakers in the sample fall in this category, or arrived as very young children (under ten). The speech of members of this generation shows extreme inter-speaker and intra-speaker variability, and a mixture of features from both norms (\ili{Jordanian} and urban \ili{Palestinian}). For example, the same speaker is found to use the second person plural pronominal suffixes -\textit{ku} (\ili{Jordanian}) and \textit{{}-kon} (\ili{Palestinian}), e.g. \textit{kēf} \textit{ḥāl-ku} \textit{{\textasciitilde} kīf ḥāl-kon} ‘how are you (\textsc{pl})'. In this example, we also find alternation in the vowel of the item \textit{kēf} {\textasciitilde}\textit{kīf} ‘how’; the former is \ili{Jordanian} while the latter is typical of urban \ili{Palestinian} (and urban \ili{Levantine} in general). The data also contained a mixture of \ili{Jordanian} and \ili{Palestinian} third person plural suffixes \textit{{}-hum} and \textit{\--hon,} e.g. \textit{šift-hum} {\textasciitilde} \textit{šift-hon} ‘I have seen them’. At the level of phonology, speakers in this generation use a mixture of \ili{Jordanian} [ɡ] and urban \ili{Palestinian} [ʔ], which are variants of historical /q/; and a mixture of interdental and stop counterparts of /θ/, /ð/ and /ð̣/. Importantly, in this generation there is a complication in sociolinguistic correlations: whereas in the first generation there is a one-to-one relationship between origin and the dialect used, in the second generation certain groups from both backgrounds use features characteristic of the other group’s dialect. The particular sub-groups that do this are the Jordanian women, who in this generation use \ili{Palestinian} [ʔ] almost consistently, as well as a high rate of the stop variants of interdentals (see above), and use both \ili{Jordanian} \textit{ʔiḥna} and \ili{Palestinian} \textit{niḥna} ‘we’. The second most divergent group (from their heritage dialect) is Palestinian men; they use \ili{Jordanian} [ɡ] at a rate of 50\%, or more in some cases. The remaining groups, \ili{Jordanian} men and Palestinian women, are considerably more conservative with respect to their heritage variants, although they too are variable. What this pattern shows is that \isi{gender} emerges as an important social factor in this generation, in addition to dialectal heritage, which continues to influence individuals’ behaviour, but interacts with \isi{gender} at the same time. 

\subsubsection{Stage III: third generation}

Third-generation Ammanis were all born in the city (in the 1970s). They diverge from their parents’ and grandparents’ dialects, and speak a clearly distinct dialect, regardless of their own dialectal heritage. The mixture and variability we saw in the second generation is much reduced in the third generation; there is, instead, stability in the usage of many features, including intermediate fudged forms, new patterns, and new features that were not present in the input varieties. The third generation agree on the characteristics of Ammani, and have intuitions as to what you can and cannot say in this dialect. Importantly, they express affiliation with the city; for instance, they identify themselves as ``Ammanis'', by which they mean that they are native to the city. In other words, the \isi{formation} of the dialect is simultaneously a \isi{formation} of a community.

In this generation, \isi{gender} emerges as a major organizing category; for instance, all the women in this generation, regardless of dialectal heritage, use [ʔ] consistently, while the men continue to use both [ɡ] and [ʔ]. The variability in men’s speech is constrained by context and interlocutor for the most part; whereas the speech of women is not subject to these constraints.\footnote{Details about this feature can be found in \citet{Al-WerHerin2011}.} The development in the use of variants of the variable (q), as explained above, is a clear example of a variable that has undergone social and stylistic \isi{reallocation} (see §\ref{new} above) in the sense that both variants [ɡ] and [ʔ], which originally come from different dialects in the input varieties, have survived the koinéization process but no longer signify ethnicity or dialectal background straightforwardly; the use of one or the other is now subject to layers of constraints. As far as the interdental sounds are concerned, both \isi{gender} groups use the stop variants more often than the interdental variants. But while the men vary between affricate [ʤ] and fricative [ʒ] of the variable (ǧ), the women use [ʒ] almost consistently.

In addition to the features listed under stage III, the following features are at an advanced stage of \isi{focusing} in the \isi{new dialect}:

\begin{itemize}
\item The feminine ending \textit{{}-a.} The input varieties differ in the phonology and phonetics of the realization of the feminine ending in the unbound state. In traditional \ili{Jordanian}, the low vowel [a] is the default choice, except after coronal sounds, where it is raised to [ɛ]. In urban \ili{Palestinian} dialects, the default choice is [e], or raised [e̝], except after \isi{pharyngeal} and \isi{emphatic consonants} in general.\footnote{A preceding /r/ blocks raising in general unless there is an /i/-type vowel in the environment; for a complete account of the phonology of the feminine ending, see \citet{Al-WerEtAl2015}.} In Amman, the third generation consistently use a fudged form made up of urban \ili{Palestinian} phonology and the \ili{Jordanian} phonetic property of the raised vowel, such that they raise /a/ to [ɛ] except after back sounds, e.g. \textit{mukinsɛ} ‘broom', \textit{ḥilwɛ} ‘pretty’, \textit{ṣaʕbɛ} ‘difficult’, but \textit{rāyḥa} ‘has gone/is going’; \textit{žāmʕa} ‘university’; \textit{ṣulṭa} ‘authority’.

\item In morphophonology, a new form of the second person plural suffix has emerg\-ed, and is used consistently. The input forms are: \ili{Jordanian} \textit{{}-ku}, as in \textit{gultilku} ‘I have told you (\textsc{pl})’; and urban \ili{Palestinian} \textit{{}-kon,} as in \textit{ʕmiltilkon} ‘I have made for you (\textsc{pl})’. The form that has been focused in Amman is \textit{{}-kum,} thus \textit{ʕindkum} ‘you (\textsc{pl}) have’; \textit{ḥakētilkum} ‘I have told you (\textsc{pl)}’. The success of this form, rather than one from the input varieties, is explained with reference to markedness and \isi{simplification}.\footnote{For analysis of this development, see the full details in \citet{Al-Wer2003kum}.}

\item In morphology, the input varieties differ in the conjugation of the third person masculine imperfect verb form. In both dialects, \ili{Jordanian} and \ili{Palestinian}, the imperfect takes a \textit{b}{}- prefix, but whereas in \ili{Jordanian} dialects \textit{yod} is dropped from the \isi{stem} in the \textit{b}{}-imperfect in all environments, in \ili{Palestinian} dialects it is dropped in open syllables only. For example: \ili{Jordanian} \textit{biḥki} ‘he talks’, \textit{binuṭṭu} ‘they jump’; urban \ili{Palestinian} \textit{byiḥki,} \textit{binuṭṭu.} Ammanis (third generation) drop \textit{yod} everywhere except where it carries person information, namely in glottal-initial verbs \textit{ʔakal} ‘to eat’, and \textit{ʔaxað} ‘to take’; thus we get \textit{biḥki,} \textit{binuṭṭu,} but \textit{byākul} ‘he eats’ (\isi{stem} \textit{ʔakal} ‘to eat’), \textit{byāḫdu} ‘they take’ (\isi{stem} \textit{ʔaḫað} ‘to take’).\footnote{There are further complications and variations in the conjugations of these verbs; for these details see \citet{Al-Wer2014}.}
\end{itemize}

\section{Conclusion}

The \isi{formation} of the Amman dialect is simultaneously the \isi{formation} of a community; and the \isi{social factors} involved in the \isi{formation} of the dialect evolve and realign accordingly. One of the most interesting aspects of this process is that none of the factors become totally irrelevant. For instance, dialectal heritage – which, in the case in hand, coincides with ethnicity ({Jordanian}/Palestinian) – is the most important predictor in the speech of the first generation. In the second generation, \isi{gender} emerges as an important factor, but the linguistic developments at this stage can only be understood as an interaction between the old and new social constraints; for instance, in stage II, it is not merely women who use [ʔ] rather than [ɡ], but it is \textit{Jordanian} women who diverge from their heritage variant; and it is not the behaviour of men in general that explains the evolution in the re-distribution of these variants, but specifically the behaviour \textit{Palestinian} men. These two sub-groups (Jordanian women and Palestinian men) are responsible for the diversification of, firstly, their respective group’s linguistic repertoire and consequently the repertoire of the linguistic system that is passed on to the next generation. In stage III, the third generation’s behaviour responds to two riders: the system inherited from their parents and the changes in the socio-political environment around them. A further realignment of \isi{social factors} occurs in response, and new constraints are added to the old pile; at this stage, the inherited identifications of the variants involved – that is, [ɡ] is \ili{Jordanian} and appropriate for men, [ʔ] is \ili{Palestinian} and appropriate for women – are reformulated through the addition of further new constraints, namely context and interlocutors. Consequently, the usage of the variants involved is redistributed according to style,\footnote{Style as a correlate of linguistic usage can mean different things; here I use it to refer to context (as in \citealt{Labov1972}), and audience or interlocutor (as in \citealt{Bell1984}). For details of how style evolved as a sociolinguistic correlate, see \citet{EckertRickford2001}.} they acquire additional identifications and social meanings, and the social constraints are realigned, such that the role of ethnicity becomes subsidiary, while \isi{gender} and style are the major organizing factors. The younger generation now define [ʔ] as ``Ammani'', and [ɡ] as ``authentic \ili{Jordanian}''. The meaning of ``\ili{Jordanian}'' itself is often negotiated and expanded beyond the limits of ethnicity to denote a regional \isi{identity}, recognizing citizenship as the primary defining component of membership in this group, although the old meaning (those whose \isi{roots} lie on the east side of the river) is not obliterated altogether.\footnote{The question of “who is Jordanian” is, for many, a sensitive issue, which has often caused heated debates on various media platforms.} A further realignment of \isi{social factors} in Amman involves type of profession, which is emerging as a constraint. This may have been precipitated by the expansion of the private sector over the past two decades or so, especially banking and the service industry in general, and the tourism industry. According to preliminary analysis of recently collected data, different types of employment, within and across the two sectors, fall within the realms of different linguistic markets.

The context in which the Amman dialect was formed was \textit{tabula rasa} in the sense that there was no pre-existing Amman dialect. The obvious difference from, say the \textit{tabula rasa} \isi{colonial} situations, is that the early settlers in Amman were not isolated from their original communities or from \ili{Arabic} speakers in the surrounding areas; \isi{social factors} definitely play a role in the \isi{formation} of the dialect in this case. The question therefore is not whether social and attitudinal factors are involved, but rather which \isi{social factors}, how they evolved, and their \isi{relative} importance.

\section*{Further reading}
\begin{furtherreading}
\item[\citet{Al-Wer2011Amman}] provides a brief description of the Amman dialect.
\item[\citet{Al-Wer2007}] summarizes the processes and dynamics of the \isi{formation} of the dialect of Amman, along with a list of thirteen linguistic features that have been focused in this dialect.
\item[\citet{Al-Wer2002furtherreading}] focuses on the long vowels and the realization of the feminine ending in the newly formed dialect of Amman.
\end{furtherreading}

\section*{Abbreviations}

\begin{tabularx}{1\textwidth}{@{}lQ@{}}
\textsc{1, 2, 3} & 1st, 2nd, 3rd person \\
\textsc{f}  &  feminine\\
L1 & first language \\
L2 & second language \\
\textsc{m}  &  masculine\\
ONZE & Origins of New Zealand {English} project \\
\textsc{pl}  &  plural\\
\textsc{sg}  &  singular\\
\end{tabularx}%


{\sloppy\printbibliography[heading=subbibliography,notkeyword=this]}
\end{document}
