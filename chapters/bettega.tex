\documentclass[output=paper]{langsci/langscibook} 
\author{Simone Bettega\affiliation{Università degli Studi di Torino}\lastand 
 Fabio Gasparini\affiliation{Università degli Studi di Napoli ``L'Orientale"}
}
\title{Modern South Arabian languages}
\abstract{In the course of this chapter we will discuss what is known about the effects that contact with Arabic has had on the Modern South Arabian languages of Oman and Yemen. Documentation concerning these languages is not abundant, and even more limited is our knowledge of the history of their interaction with Arabic. By integrating the existing bibliography with as yet unpublished fieldwork materials, we will try to provide as complete a picture of the situation as possible, also discussing the current linguistic and sociolinguistic landscape of Dhofar and Eastern Yemen. }
\IfFileExists{../localcommands.tex}{
  % add all extra packages you need to load to this file 
\usepackage{graphicx}
\usepackage{tabularx}
\usepackage{amsmath} 
\usepackage{multicol}
\usepackage{lipsum}
\usepackage[stable]{footmisc}
\usepackage{adforn}
%%%%%%%%%%%%%%%%%%%%%%%%%%%%%%%%%%%%%%%%%%%%%%%%%%%%
%%%                                              %%%
%%%           Examples                           %%%
%%%                                              %%%
%%%%%%%%%%%%%%%%%%%%%%%%%%%%%%%%%%%%%%%%%%%%%%%%%%%%
% remove the percentage signs in the following lines
% if your book makes use of linguistic examples
\usepackage{./langsci/styles/langsci-optional} 
\usepackage{./langsci/styles/langsci-lgr}
\usepackage{morewrites} 
%% if you want the source line of examples to be in italics, uncomment the following line
% \def\exfont{\it}

\usepackage{enumitem}
\newlist{furtherreading}{description}{1}
\setlist[furtherreading]{font=\normalfont,labelsep=\widthof{~},noitemsep,align=left,leftmargin=\parindent,labelindent=0pt,labelwidth=-\parindent}
\usepackage{phonetic}
\usepackage{chronosys,tabularx}
\usepackage{csquotes}
\usepackage[stable]{footmisc} 

\usepackage{langsci-bidi}
\usepackage{./langsci/styles/langsci-gb4e} 

  \makeatletter
\let\thetitle\@title
\let\theauthor\@author 
\makeatother

\newcommand{\togglepaper}[1][0]{ 
  \bibliography{../localbibliography}
  \papernote{\scriptsize\normalfont
    \theauthor.
    \thetitle. 
    To appear in: 
    Christopher Lucas and Stefano Manfredi (eds.),  
    Arabic and contact-induced language change
    Berlin: Language Science Press. [preliminary page numbering]
  }
  \pagenumbering{roman}
  \setcounter{chapter}{#1}
  \addtocounter{chapter}{-1}
}

\newfontfamily\Parsifont[Script=Arabic]{ScheherazadeRegOT_Jazm.ttf} 
\newcommand{\arabscript}[1]{\RL{\Parsifont #1}}
\newcommand{\textarabic}[1]{{\arabicfont #1}}

\newcommand{\textstylest}[1]{{\color{red}#1}}

\patchcmd{\mkbibindexname}{\ifdefvoid{#3}{}{\MakeCapital{#3}
}}{\ifdefvoid{#3}{}{#3 }}{}{\AtEndDocument{\typeout{mkbibindexname could
not be patched.}}}

%command for italic r with dot below with horizontal correction to put the dot in the prolongation of the vertical stroke
%for some reason, the dot is larger than expected, so we explicitly reduce the font size (to \small)
%for the time being, the font is set to an absolute value. To be more robust, a relative reduction would be better, but this might not be required right now
\newcommand{\R}{r\kern-.05ex{\small{̣}}\kern.05ex}


\DeclareLabeldate{%
    \field{date}
    \field{year}
    \field{eventdate}
    \field{origdate}
    \field{urldate}
    \field{pubstate}
    \literal{nodate}
}

\renewbibmacro*{addendum+pubstate}{% Thanks to https://tex.stackexchange.com/a/154367 for the idea
  \printfield{addendum}%
  \iffieldequalstr{labeldatesource}{pubstate}{}
  {\newunit\newblock\printfield{pubstate}}
}
 
  %% hyphenation points for line breaks
%% Normally, automatic hyphenation in LaTeX is very good
%% If a word is mis-hyphenated, add it to this file
%%
%% add information to TeX file before \begin{document} with:
%% %% hyphenation points for line breaks
%% Normally, automatic hyphenation in LaTeX is very good
%% If a word is mis-hyphenated, add it to this file
%%
%% add information to TeX file before \begin{document} with:
%% %% hyphenation points for line breaks
%% Normally, automatic hyphenation in LaTeX is very good
%% If a word is mis-hyphenated, add it to this file
%%
%% add information to TeX file before \begin{document} with:
%% \include{localhyphenation}
\hyphenation{
affri-ca-te
affri-ca-tes
com-ple-ments
homo-phon-ous
start-ed
Meso-potam-ian
morpho-phono-logic-al-ly
morpho-phon-em-ic-s
Palestin-ian
re-present-ed
Ki-nubi
ḥawār-iyy-ūn
archa-ic-ity
fuel-ed
de-velop-ment
pros-od-ic
Arab-ic
in-duced
phono-logy
possess-um
possess-ive-s
templ-ate
spec-ial
espec-ial-ly
nat-ive
pass-ive
clause-s
potent-ial-ly
Lusignan
commun-ity
tobacco
posi-tion
Cushit-ic
Middle
with-in
re-finit-iz-ation
langu-age-s
langu-age
diction-ary
glossary
govern-ment
eight
counter-part
nomin-al
equi-valent
deont-ic
ana-ly-sis
Malt-ese
un-fortun-ate-ly
scient-if-ic
Catalan
Occitan
ḥammāl
cross-linguist-ic-al-ly
predic-ate
major-ity
ignor-ance
chrono-logy
south-western
mention-ed
borrow-ed
neg-ative
de-termin-er
European
under-mine
detail
Oxford
Socotra
numer-ous
spoken
villages
nomad-ic
Khuze-stan
Arama-ic
Persian
Ottoman
Ottomans
Azeri
rur-al
bi-lingual-ism
borrow-ing
prestig-ious
dia-lects
dia-lect
allo-phone
allo-phones
poss-ible
parallel
parallels
pattern
article
common-ly
respect-ive-ly
sem-antic
Moroccan
Martine
Harrassowitz
Grammatic-al-ization
grammatic-al-ization
Afro-asiatica
Afro-asiatic
continu-ation
Semit-istik
varieties
mono-phthong
mono-phthong-ized
col-loquial
pro-duct
document-ary
ex-ample-s
ex-ample
termin-ate
element-s
Aramaeo-grams
Centr-al
idioms
Arab-ic
Dadan-it-ic
sub-ordin-ator
Thamud-ic
difficult
common-ly
Revue
Bovingdon
under
century
attach
attached
bundle
graph-em-ic
graph-emes
cicada
contrast-ive
Corriente
Andalusi
Kossmann
morpho-logic-al
inter-action
dia-chroniques
islámica
occid-ent-al-ismo
dialecto-logie
Reichert
coloni-al
Milton
diphthong-al
linguist-ic
linguist-ics
affairs
differ-ent
phonetic-ally
kilo-metres
stabil-ization
develop-ments
in-vestig-ation
Jordan-ian
notice-able
level-ed
migrants
con-dition-al
certain-ly
general-ly
especial-ly
af-fric-ation
Jordan
counter-parts
com-plication
consider-ably
inter-dent-al
com-mun-ity
inter-locutors
com-pon-ent
region-al
socio-historical
society
simul-taneous
phon-em-ic
roman-ization
Classic-al
funeral
Kurmanji
pharyn-geal-ization
vocab-ulary
phon-et-ic
con-sonant
con-sonants
special-ized
latter
latters
in-itial
ident-ic-al
cor-relate
geo-graphic-al-ly
Öpengin
Kurd-ish
in-digen-ous
sunbul
Christ-ian
Christ-ians
sekin-în
fatala
in-tegration
dia-lect-al
Matras
morpho-logy
in-tens-ive
con-figur-ation
im-port-ant
com-plement
ḥaddād
e-merg-ence
Benjmamins
struct-ure
em-pir-ic-al
Orient-studien
Anatolia
American
vari-ation
Jastrow
Geoffrey
Yarshater
Ashtiany
Edmund
Mahnaz
En-cyclo-pædia
En-cyclo-paedia
En-cyclo-pedia
Leiden
dia-spora
soph-is-ic-ated
Sasan-ian
every-day
domin-ance
Con-stitu-tion-al
religi-ous
sever-al
Manfredi
re-lev-ance
re-cipi-ent
pro-duct-iv-ity
turtle
Morocco
ferman
Maghreb-ian
algérien
stand-ard
systems
Nicolaï
Mouton
mauritani-en
Gotho-burg-ensis
socio-linguist-ique
plur-al
archiv-al
Arab-ian
drop-ped
dihāt
de-velop-ed
ṣuḥbat
kitāba
kitābat
com-mercial
eight-eenth
region
Senegal
mechan-ics
Maur-itan-ia
Ḥassān-iyya
circum-cision
cor-relation
labio-velar-ization
vowel
vowels
cert-ain
īggīw
series
in-tegrates
dur-ative
inter-dent-als
gen-itive
Tuareg
tălămut
talawmāyət
part-icular
part-icular-ly
con-diment
vill-age
bord-er
polit-ical
Wiesbaden
Uni-vers-idad
Geuthner
typo-logie
Maur-itanie
nomades
Maur-itan-ian
dia-lecto-logy
Sahar-iennes
Uni-vers-ity
de-scend-ants
NENA-speak-ing
speak-ing
origin-al
re-captured
in-habit-ants
ethnic
minor-it-ies
drama-tic
local
long-stand-ing
regions
Nineveh
settle-ments
Ṣəndor
Mandate
sub-stitut-ing
ortho-graphy
re-fer-enced
origin-ate
twenti-eth
typ-ic-al-ly
Hobrack
never-the-less
character-ist-ics
character-ist-ic
masc-uline
coffee
ex-clus-ive-ly
verb-al
re-ana-ly-se-d
simil-ar-ities
de-riv-ation
im-pera-tive
part-iciple
dis-ambi-gu-ation
dis-ambi-gu-a-ing
phen-omen-on
phen-omen-a
traktar
com-mun-ity
com-mun-ities
dis-prefer-red
ex-plan-ation
con-struction
wide-spread
us-ual-ly
region-al
Bulut
con-sider-ation
afro-asia-tici
Franco-Angeli
Phono-logie
Volks-kundliche
dia-lectes
dia-lecte
select-ed
dis-appear-ance
media
under-stand-able
public-ation
second-ary
e-ject-ive
re-volu-tion
re-strict-ive
Gasparini
mount-ain
mount-ains
yellow
label-ing
trad-ition-al-ly
currently
dia-chronic
}
\hyphenation{
affri-ca-te
affri-ca-tes
com-ple-ments
homo-phon-ous
start-ed
Meso-potam-ian
morpho-phono-logic-al-ly
morpho-phon-em-ic-s
Palestin-ian
re-present-ed
Ki-nubi
ḥawār-iyy-ūn
archa-ic-ity
fuel-ed
de-velop-ment
pros-od-ic
Arab-ic
in-duced
phono-logy
possess-um
possess-ive-s
templ-ate
spec-ial
espec-ial-ly
nat-ive
pass-ive
clause-s
potent-ial-ly
Lusignan
commun-ity
tobacco
posi-tion
Cushit-ic
Middle
with-in
re-finit-iz-ation
langu-age-s
langu-age
diction-ary
glossary
govern-ment
eight
counter-part
nomin-al
equi-valent
deont-ic
ana-ly-sis
Malt-ese
un-fortun-ate-ly
scient-if-ic
Catalan
Occitan
ḥammāl
cross-linguist-ic-al-ly
predic-ate
major-ity
ignor-ance
chrono-logy
south-western
mention-ed
borrow-ed
neg-ative
de-termin-er
European
under-mine
detail
Oxford
Socotra
numer-ous
spoken
villages
nomad-ic
Khuze-stan
Arama-ic
Persian
Ottoman
Ottomans
Azeri
rur-al
bi-lingual-ism
borrow-ing
prestig-ious
dia-lects
dia-lect
allo-phone
allo-phones
poss-ible
parallel
parallels
pattern
article
common-ly
respect-ive-ly
sem-antic
Moroccan
Martine
Harrassowitz
Grammatic-al-ization
grammatic-al-ization
Afro-asiatica
Afro-asiatic
continu-ation
Semit-istik
varieties
mono-phthong
mono-phthong-ized
col-loquial
pro-duct
document-ary
ex-ample-s
ex-ample
termin-ate
element-s
Aramaeo-grams
Centr-al
idioms
Arab-ic
Dadan-it-ic
sub-ordin-ator
Thamud-ic
difficult
common-ly
Revue
Bovingdon
under
century
attach
attached
bundle
graph-em-ic
graph-emes
cicada
contrast-ive
Corriente
Andalusi
Kossmann
morpho-logic-al
inter-action
dia-chroniques
islámica
occid-ent-al-ismo
dialecto-logie
Reichert
coloni-al
Milton
diphthong-al
linguist-ic
linguist-ics
affairs
differ-ent
phonetic-ally
kilo-metres
stabil-ization
develop-ments
in-vestig-ation
Jordan-ian
notice-able
level-ed
migrants
con-dition-al
certain-ly
general-ly
especial-ly
af-fric-ation
Jordan
counter-parts
com-plication
consider-ably
inter-dent-al
com-mun-ity
inter-locutors
com-pon-ent
region-al
socio-historical
society
simul-taneous
phon-em-ic
roman-ization
Classic-al
funeral
Kurmanji
pharyn-geal-ization
vocab-ulary
phon-et-ic
con-sonant
con-sonants
special-ized
latter
latters
in-itial
ident-ic-al
cor-relate
geo-graphic-al-ly
Öpengin
Kurd-ish
in-digen-ous
sunbul
Christ-ian
Christ-ians
sekin-în
fatala
in-tegration
dia-lect-al
Matras
morpho-logy
in-tens-ive
con-figur-ation
im-port-ant
com-plement
ḥaddād
e-merg-ence
Benjmamins
struct-ure
em-pir-ic-al
Orient-studien
Anatolia
American
vari-ation
Jastrow
Geoffrey
Yarshater
Ashtiany
Edmund
Mahnaz
En-cyclo-pædia
En-cyclo-paedia
En-cyclo-pedia
Leiden
dia-spora
soph-is-ic-ated
Sasan-ian
every-day
domin-ance
Con-stitu-tion-al
religi-ous
sever-al
Manfredi
re-lev-ance
re-cipi-ent
pro-duct-iv-ity
turtle
Morocco
ferman
Maghreb-ian
algérien
stand-ard
systems
Nicolaï
Mouton
mauritani-en
Gotho-burg-ensis
socio-linguist-ique
plur-al
archiv-al
Arab-ian
drop-ped
dihāt
de-velop-ed
ṣuḥbat
kitāba
kitābat
com-mercial
eight-eenth
region
Senegal
mechan-ics
Maur-itan-ia
Ḥassān-iyya
circum-cision
cor-relation
labio-velar-ization
vowel
vowels
cert-ain
īggīw
series
in-tegrates
dur-ative
inter-dent-als
gen-itive
Tuareg
tălămut
talawmāyət
part-icular
part-icular-ly
con-diment
vill-age
bord-er
polit-ical
Wiesbaden
Uni-vers-idad
Geuthner
typo-logie
Maur-itanie
nomades
Maur-itan-ian
dia-lecto-logy
Sahar-iennes
Uni-vers-ity
de-scend-ants
NENA-speak-ing
speak-ing
origin-al
re-captured
in-habit-ants
ethnic
minor-it-ies
drama-tic
local
long-stand-ing
regions
Nineveh
settle-ments
Ṣəndor
Mandate
sub-stitut-ing
ortho-graphy
re-fer-enced
origin-ate
twenti-eth
typ-ic-al-ly
Hobrack
never-the-less
character-ist-ics
character-ist-ic
masc-uline
coffee
ex-clus-ive-ly
verb-al
re-ana-ly-se-d
simil-ar-ities
de-riv-ation
im-pera-tive
part-iciple
dis-ambi-gu-ation
dis-ambi-gu-a-ing
phen-omen-on
phen-omen-a
traktar
com-mun-ity
com-mun-ities
dis-prefer-red
ex-plan-ation
con-struction
wide-spread
us-ual-ly
region-al
Bulut
con-sider-ation
afro-asia-tici
Franco-Angeli
Phono-logie
Volks-kundliche
dia-lectes
dia-lecte
select-ed
dis-appear-ance
media
under-stand-able
public-ation
second-ary
e-ject-ive
re-volu-tion
re-strict-ive
Gasparini
mount-ain
mount-ains
yellow
label-ing
trad-ition-al-ly
currently
dia-chronic
}
\hyphenation{
affri-ca-te
affri-ca-tes
com-ple-ments
homo-phon-ous
start-ed
Meso-potam-ian
morpho-phono-logic-al-ly
morpho-phon-em-ic-s
Palestin-ian
re-present-ed
Ki-nubi
ḥawār-iyy-ūn
archa-ic-ity
fuel-ed
de-velop-ment
pros-od-ic
Arab-ic
in-duced
phono-logy
possess-um
possess-ive-s
templ-ate
spec-ial
espec-ial-ly
nat-ive
pass-ive
clause-s
potent-ial-ly
Lusignan
commun-ity
tobacco
posi-tion
Cushit-ic
Middle
with-in
re-finit-iz-ation
langu-age-s
langu-age
diction-ary
glossary
govern-ment
eight
counter-part
nomin-al
equi-valent
deont-ic
ana-ly-sis
Malt-ese
un-fortun-ate-ly
scient-if-ic
Catalan
Occitan
ḥammāl
cross-linguist-ic-al-ly
predic-ate
major-ity
ignor-ance
chrono-logy
south-western
mention-ed
borrow-ed
neg-ative
de-termin-er
European
under-mine
detail
Oxford
Socotra
numer-ous
spoken
villages
nomad-ic
Khuze-stan
Arama-ic
Persian
Ottoman
Ottomans
Azeri
rur-al
bi-lingual-ism
borrow-ing
prestig-ious
dia-lects
dia-lect
allo-phone
allo-phones
poss-ible
parallel
parallels
pattern
article
common-ly
respect-ive-ly
sem-antic
Moroccan
Martine
Harrassowitz
Grammatic-al-ization
grammatic-al-ization
Afro-asiatica
Afro-asiatic
continu-ation
Semit-istik
varieties
mono-phthong
mono-phthong-ized
col-loquial
pro-duct
document-ary
ex-ample-s
ex-ample
termin-ate
element-s
Aramaeo-grams
Centr-al
idioms
Arab-ic
Dadan-it-ic
sub-ordin-ator
Thamud-ic
difficult
common-ly
Revue
Bovingdon
under
century
attach
attached
bundle
graph-em-ic
graph-emes
cicada
contrast-ive
Corriente
Andalusi
Kossmann
morpho-logic-al
inter-action
dia-chroniques
islámica
occid-ent-al-ismo
dialecto-logie
Reichert
coloni-al
Milton
diphthong-al
linguist-ic
linguist-ics
affairs
differ-ent
phonetic-ally
kilo-metres
stabil-ization
develop-ments
in-vestig-ation
Jordan-ian
notice-able
level-ed
migrants
con-dition-al
certain-ly
general-ly
especial-ly
af-fric-ation
Jordan
counter-parts
com-plication
consider-ably
inter-dent-al
com-mun-ity
inter-locutors
com-pon-ent
region-al
socio-historical
society
simul-taneous
phon-em-ic
roman-ization
Classic-al
funeral
Kurmanji
pharyn-geal-ization
vocab-ulary
phon-et-ic
con-sonant
con-sonants
special-ized
latter
latters
in-itial
ident-ic-al
cor-relate
geo-graphic-al-ly
Öpengin
Kurd-ish
in-digen-ous
sunbul
Christ-ian
Christ-ians
sekin-în
fatala
in-tegration
dia-lect-al
Matras
morpho-logy
in-tens-ive
con-figur-ation
im-port-ant
com-plement
ḥaddād
e-merg-ence
Benjmamins
struct-ure
em-pir-ic-al
Orient-studien
Anatolia
American
vari-ation
Jastrow
Geoffrey
Yarshater
Ashtiany
Edmund
Mahnaz
En-cyclo-pædia
En-cyclo-paedia
En-cyclo-pedia
Leiden
dia-spora
soph-is-ic-ated
Sasan-ian
every-day
domin-ance
Con-stitu-tion-al
religi-ous
sever-al
Manfredi
re-lev-ance
re-cipi-ent
pro-duct-iv-ity
turtle
Morocco
ferman
Maghreb-ian
algérien
stand-ard
systems
Nicolaï
Mouton
mauritani-en
Gotho-burg-ensis
socio-linguist-ique
plur-al
archiv-al
Arab-ian
drop-ped
dihāt
de-velop-ed
ṣuḥbat
kitāba
kitābat
com-mercial
eight-eenth
region
Senegal
mechan-ics
Maur-itan-ia
Ḥassān-iyya
circum-cision
cor-relation
labio-velar-ization
vowel
vowels
cert-ain
īggīw
series
in-tegrates
dur-ative
inter-dent-als
gen-itive
Tuareg
tălămut
talawmāyət
part-icular
part-icular-ly
con-diment
vill-age
bord-er
polit-ical
Wiesbaden
Uni-vers-idad
Geuthner
typo-logie
Maur-itanie
nomades
Maur-itan-ian
dia-lecto-logy
Sahar-iennes
Uni-vers-ity
de-scend-ants
NENA-speak-ing
speak-ing
origin-al
re-captured
in-habit-ants
ethnic
minor-it-ies
drama-tic
local
long-stand-ing
regions
Nineveh
settle-ments
Ṣəndor
Mandate
sub-stitut-ing
ortho-graphy
re-fer-enced
origin-ate
twenti-eth
typ-ic-al-ly
Hobrack
never-the-less
character-ist-ics
character-ist-ic
masc-uline
coffee
ex-clus-ive-ly
verb-al
re-ana-ly-se-d
simil-ar-ities
de-riv-ation
im-pera-tive
part-iciple
dis-ambi-gu-ation
dis-ambi-gu-a-ing
phen-omen-on
phen-omen-a
traktar
com-mun-ity
com-mun-ities
dis-prefer-red
ex-plan-ation
con-struction
wide-spread
us-ual-ly
region-al
Bulut
con-sider-ation
afro-asia-tici
Franco-Angeli
Phono-logie
Volks-kundliche
dia-lectes
dia-lecte
select-ed
dis-appear-ance
media
under-stand-able
public-ation
second-ary
e-ject-ive
re-volu-tion
re-strict-ive
Gasparini
mount-ain
mount-ains
yellow
label-ing
trad-ition-al-ly
currently
dia-chronic
} 
  \togglepaper[1]%%chapternumber
}{}

\begin{document}
\maketitle 


\section{History of contact between Arabic and the Modern South Arabian languages}\label{sec:1}

Much to the frustration of modern scholars of Semitic, the history of the Modern South Arabian languages (henceforth MSAL) remains largely unknown.\footnote {§\ref{sec:1} was authored by Simone Bettega, while §\ref{sec:2} was authored by Fabio Gasparini. §\ref{sec:3} and §\ref{sec:4} are the result of the conjoined efforts of both authors. In particular, Gasparini was responsible for analyzing most of the primary sources and raw linguistic data, while Bettega worked more extensively on the existing bibliography.} 
To this day, no written attestation of these varieties has been discovered, and it seems safe to assume that they have remained exclusively spoken languages throughout all of their history. Since European researchers became aware of their existence in the first half of the nineteenth century \citep{Wellsted1837}, and until very recently, the MSAL were thought by many to be the descendants of the old (epigraphic) South Arabian languages \citep[16]{Rubin2014}. This assumption has been conclusively disproven by Porkhomovsky’s (\citeyear{Porkhomovsky1997}) article, which also contributed in a significantly to the re-shaping of the proposed model for the Semitic family tree. This modified version of the family tree (which finds further support in the recent works of \citealt{Kogan2015} and \citealt{Edzard2017}) sets the MSAL apart as an independent branch of the West Semitic subgroup, one whose origins are therefore of considerable antiquity. This brings us to the question of when it was that the MSAL (or their forebears) first came into contact with Arabic. This might have happened at any time since Arabic-speaking people started to penetrate into southern Arabia, a process that – as we know from historical records – began in the second half of the first millennium B.C. (\citealt{Robin1991}; \citealt{Hoyland2001}: 47--48). Roughly one thousand years later, almost the whole population of central and northern Yemen was speaking Arabic, and possibly a considerable portion of the southern population as well (\citealt{Beeston1981}: 184; \citealt{Zammit2011}: 295). It is therefore possible that Arabic and the MSAL have been in contact for quite some time, and it seems likely that the intensity and effects of such contact grew stronger after the advent of Islam \citep[247]{Lonnet2011}. It is also possible, as some scholars have written, that the MSAL “represent isolated forms that were never touched by Arabic influence until the modern period” \citep[127]{Versteegh2014book}. Admittedly, evidence to support either one of these hypotheses is scarce, and at present it is probably safer to say that our knowledge of the history of contact between Arabic and the MSAL before the twentieth century is fragmentary at best. This is why studies on the outcomes of such contact are of particular interest, since they could help to shed light on parts of that history. This is also why, in the course of this chapter, we will refrain from addressing the question of how contact with the MSAL affected the varieties of Arabic spoken in Oman and Yemen, and focus solely on the influence of Arabic on the MSAL. Although there is plenty of evidence that South Arabian exerted a powerful influence on the Arabic of the area (see for instance \citealt{Retsö2000} and \citealt{Watson2018}),\footnote{To the point that so-called mixed varieties are reported to exist, whose exact linguistic nature seems difficult to pinpoint. See \citet{WatsonEtAl2006} and \citet{Watson2011SAYemeni} for discussion.} it is often difficult to assess whether this influence is the result of contact with forms of Ancient South Arabian or more recent interaction with the MSAL. Such a discussion, also because of space constraints, is beyond the scope of the present article.

As far as the interaction between Arabic and the MSAL in the twentieth and 21\textsuperscript{st} centuries is concerned, Morris (\citeyear[25]{Morris2017}) provides a good overview of the multilingual environment in which the MSAL were and are spoken: 

\begin{quote}
Speakers of [a Modern South Arabian] language always had to deal with speakers of other MSAL, as well as with speakers of various dialects of Arabic. The Baṭāḥirah, for instance, did nearly all their trade with boats from Ṣūr and other Arabic-speaking ports; they lived and worked with the Arabic-speaking Janaba, while being in contact with speakers of Ḥarāsīs and Mahra. The Ḥarāsīs interacted with the Arabic speakers surrounding their Jiddat al-Ḥarāsīs homeland, traded in the Arabic-speaking markets of the north, and in the summer months went to work at the northern date harvest. Mehri speakers lived beside and traded with Arabic-speaking Kathīri tribesmen in the Nejd region, Śḥerɛt speakers in the mountains, and Arabic speakers in the coastal market towns of Dhofar. Śḥerɛt speakers interacted with the Mahra, some of whom settled among them, and with Arabic-speaking peoples of the coast as well as the desert interior […] There was marriage between Arabic-speaking men of the coastal towns and MSAL-speaking women of the interior, and over time, families of Mehri and Śḥerɛt speakers settled in or near the towns, with the result that even more Arabic speakers became familiar with these languages.
\end{quote}

\section{Current state of contact between Arabic and the Modern South Arabian languages}\label{sec:2}

Today, six Modern South Arabian languages exist, spoken by around 200,000 people in Eastern Yemen (including the island of Soqotra) and Western Oman. These six languages are: Mehri, Hobyōt, Ḥarsūsi, Baṭḥari, Śḥerɛt/Jibbāli and Soqoṭri. They are all to be regarded as endangered varieties, though the individual degree of endangerment varies remarkably. No exact census concerning the number of speakers is currently available (\citealt{Simeone-Senelle2011}: 1075), but we know that Mehri is the most spoken language, with an estimated 100,000 speakers. It is followed by Soqoṭri (about 50,000 speakers), Śḥerɛt (25,000), Ḥarsūsi (a few hundred), Hobyōt (a few hundred) and Baṭḥari (less than 20 speakers). The main causes of endangerment are reckoned to be shift to Arabic and disappearance of traditional local lifestyles. In addition, the current political situation in Yemen is having effects on the linguistic landscape of the region which are difficult to document or foresee: the area is currently inaccessible to researchers, and there is no way to know how the conflict will affect the local communities.

As far as Oman is concerned, the city of Salalah undoubtedly represents the major locus of contact between Arabic and the MSAL. The rapid growth the city has witnessed in recent years, and the improved possibilities of economic development that came with it, have led many Śḥerɛt-speaking breeders from the nearby mountains to settle in the city or its immediate surroundings, where they now employ Arabic on a daily basis as a consequence of mass education and media neglecting other local languages. This has led to a split, in the speakers’ perception, between ``proper'' Śḥerɛt, spoken in the mountains, and the ``city Śḥerɛt'' of Salalah, often regarded as a sort of ``broken'' variety of the language in which, among other things, code-switching with Arabic is extremely frequent. Unfortunately, data on this subject are virtually non-existent, given the extreme difficulty of documenting such an episodic phenomenon (aggravated by the speakers’ understandable reluctance to having their imperfect language proficiency evaluated and recorded). 

Even outside the urban centers, however, contact with Arabic is on the rise. Even the most isolated variety, Soqotri, is apparently undergoing rapid change under the influence of Arabic: the existence of a koineised variety of Soqotri, heavily influenced by Arabic, has been recently reported in Ḥadibo \citep[27]{Morris2017}. This is not to say, of course, that all MSAL are being affected to the same degree: Watson (\citeyear[3]{Watson2012}), for instance, notes how “Mahriyōt [the EasternYemeni variety of Mehri] […] exhibits structures unattested in Mehreyyet [Mehri Omani variety] […] and shows greater Arabic influence both in terms of the number of Arabic terms used, and the length and frequency of Arabic phrases within texts.” However, no MSAL seems at present to be exempt from the effects of contact.

The case of Baṭḥari, which, as we have seen, is the most severely endangered of all the MSAL, exemplifies well the processes of morphological loss and erosion that a language undergoes in the final stages of endangerment. Morris (\citeyear[12]{Morris2017}) reports how already in the 1970s Baṭḥari seemed to display many of the signs of a moribund language. In recent times

\begin{quote}
[t]he younger generations showed little interest in their former language; they were eager to embrace Arabic and to feel themselves part of the wider Arabic Islamic community; and they were proud to call themselves ‘ʕarab’, with all that word’s overtones of Bedouin ancestry and code of honour \citep[11]{Morris2017}.
\end{quote}

In the following sections we will discuss several types of contact-induced changes in the MSAL: although we will use material taken from all varieties, Baṭḥari will be in particular focus due to its singular status.

\section{Contact-induced changes in the MSAL}\label{sec:3}

As already noted, in the course of this chapter we will focus solely on the effects that contact with Arabic has had on the various MSAL. Therefore, Arabic will always be the source language of all the transfer phenomena considered in the next pages, while the recipient language will be, depending on the different examples, one or the other of the six MSAL. Obviously, this poses the question of who the agents of change are and have been in the case of these particular phenomena, and what type(s) of transfer are we confronted with (cf. \citealt{VanCoetsem1988,VanCoetsem2000,Winford2005}). According to the overview of the MSAL’s sociolinguistic status presented above, it should be clear by now that, while the two cases are extremely common of (a) mono- or multilingual MSAL speakers who acquire Arabic as an L2 and (b) bilingual MSAL-Arabic speakers, the opposite is not true (that is, monolingual Arabic speakers who come to acquire one or more MSAL as L2s later in life). In other words, all the transfer phenomena we will be considering in the next paragraphs are either instances of borrowing (brought about by speakers who are dominant in one or more MSAL) or convergence (brought about by speakers who are native speakers of Arabic and at least one MSAL; see \citealt{Lucas2015} for a definition of convergence). 


 
 \subsection{Phonology}
 \subsubsection{Phonetic adaptation of loanwords}

As illustrated in §\ref{sec:key:lex}, lexical borrowings from Arabic are extremely common in the MSAL. As Morris (\citeyear[13]{Morris2017}) remarks, such loanwords are often altered in order for them to acquire a “South Arabian flavour”, so to speak. The phenomenon is not one of simple adaptation dictated by difficulty of articulation, since the sounds that are replaced are present in the phonological inventory of the MSAL. In fact, the opposite appears to be true, these sounds normally being replaced by others which are typical of South Arabian but absent in Arabic. For Baṭḥari, Morris gives the example of Arabic pharyngealised dental fricative /\d{ð}/ (IPA [ð\textsuperscript{ʕ}]) being replaced by the pharyngealised alveolar lateral fricative /ṣ́/ (mostly realised as IPA [ɮ\textsuperscript{ʕ}], see §\ref{sec:key:emph}), as in \textit{raṣ́ṣ́} ‘bruise’ (from Janaybi Arabic \textit{rað̣ð̣}), or Arabic /š/ (IPA [ʃ]) being replaced by /ś/, as in \textit{men} \textit{śān-k}, ‘for you, for your sake’, in place of \textit{men} \textit{šān-k}, \textit{śarray} ‘buyer’ for \textit{šarray}, or \textit{śəmāl} ‘inland, north’ for \textit{šəmāl} (while Baṭḥari \textit{śēməl(i)} is normally used to refer to the left hand only). 

Lexical borrowing can also be the cause of variation in the realisation of certain sounds, as is the case with the phonemes /g/ and /y/ (IPA [g] and [j] respectively), which represent different reflexes of Proto-Semitic *g in different Omani Arabic dialects. It is possible to find traces of this variation in those MSAL that are in contact with more than one variety of Arabic, as is the case with Ḥarsusi: see for instance \textit{fagr} and \textit{fayr}, both meaning ‘dawn’, or the opposition between \textit{yann} ‘madness’ and \textit{genni} ‘jinni’, both from the same etymological root \citep[299]{Lonnet2011}.


 \subsubsection{Affrication of /k/ > [ʧ]}

It can also be the case that some phonetic processes regularly taking place in the local Arabic varieties but otherwise unknown to MSAL phonology are transferred to original MSAL vocabulary. This is what happens in Baṭḥari, where some speakers may show an affricate realisation of the voiceless occlusive [k] > [ʧ], which resembles the Janaybi Arabic realisation of the phoneme /k/ (whose complementary distribution with the voiceless plosive realisation [k] is still unclear). For example, some speakers regularly produce /yənkaʕ/ ‘come.\textsc{3sg.m.sbjv}’ as [jənˈʧaʕ] instead of [jənˈkaʕ]. 


 \subsubsection{Stress}

The structural similarity of Arabic and the MSAL can sometimes cause stress patterns which are typical of the former to be applied to the latter, as is the case with `she began': Soqotri \textit{bédʔɔh}, (local) Arabic \textit{bədáʔat}, Soqotri with an Arabic stress \textit{bədɔ́ʔɔh} \citep[299]{Lonnet2011}.


 \subsubsection{Realisation of emphatics}\label{sec:key:emph}

This is a topic that has attracted the attention of several scholars since the publication of Johnstone’s (\citeyear{Johnstone1975}) article on the subject, because of the realisation of the so-called Semitic “emphatics” as glottalised consonants. Glottalisation is a secondary articulatory process in which narrowing (creaky voice) or closure (ejective realisation) of the glottis takes place: the action of the larynx compresses the air in the vocal tract which, once released, produces a greater amplitude in the stop burst (\citealt{LadefogedMaddieson1996}: 78). 

Lonnet (\citeyear[299]{Lonnet2011}) notes a tendency for speakers of various MSAL to replace the ejective articulation of certain consonants (especially fricatives, see \citealt{RidouaneGendrot2017}) with a pharyngealised realisation, typical of Arabic emphatics. Pharyngealisation is a kind of secondary articulation involving a constriction of the pharynx usually realised through tongue root retraction, resulting in a backed realisation (\citealt{LadefogedMaddieson1996}: 365). It is a well-documented process across Semitic languages. \citet{NaumkinPorkhomovsky1981} note for Soqoṭri an ongoing process of transition from a glottalised to a pharyngealised realisation of emphatics, with only stops being realised as fully glottalised items. Work by Watson \& Bellem (\citeyear{WatsonBellem2010,WatsonBellem2011}) and \citet{WatsonHeselwood2016} shows the co-occurrence of pharyngealisation and glottalisation in relation to pre-pausal phenomena in \d{S}anʕāni Arabic, Mahryōt and Mehreyyet (respectively the westernmost Yemeni and Omani varieties of Mehri). Dufour (\citeyear[22]{Dufour2016}) states that ``le caractère éjectif des phonèmes emphatiques ne fait aucun doute, en jibbali comme en mehri" (“the nature of the emphatic phonemes is undoubtedly ejective, in Jibbali as much as in Mehri”).\footnote{Authors' translation.} Finally, in Baṭḥari only /ḳ/ is realised as a fully ejective consonant [k’]. /ṭ/ and the fricative emphatics, on the the other hand, are described as mainly pharyngealised (and partially voiced, in the case of fricatives; \citealt{Gasparini2017}).

Unfortunately, since there is no thorough phonetic description of any MSAL that predates the 1970s, it is impossible to ascertain whether these realisations (which, again, range from fully glottalised to fully pharyngealised) are the result of the influence of Arabic, or have arisen as the consequence of internal and typologically predictable developments. It is likely, though, that bilingualism and constant contact with Arabic have at least favoured this phonetic change. Evidence in support of this view may come from the fact that speakers who are poorly proficient in Arabic and live in rural and more isolated areas are more likely to preserve a glottalised realisation of the emphatics (as emerges from direct fieldwork observations).


 
 \subsection{Morphology} 
 \subsubsection{Nominal morphology}\label{sec:key:nomorph}

Morphological patterns which are typical of Arabic can enter a language through borrowing, as is the case of the passive participle of simple verbs, which is \textit{mVCCūC} in Arabic and \textit{mVCCīC} in MSAL. Soqotri \textit{maḫlɔḳ}, for instance, is clearly derived from Arabic \textit{maḫlūq} ‘human being’ (lit. ‘created’), while this is not the case for Ḥarsusi \textit{mḫəlīḳ} \citep[299]{Lonnet2011}. Also, in the realm of verbal derivational morphology, certain phenomena can be introduced into the recipient language through lexical borrowing: this is the case with gemination and prefixation of \textit{t-} in Ḥarsusi, as in the participle \textit{mətḥaffi} ‘barefoot’ (from Omani Arabic \textit{mitḥaffi}; \citealt{Lonnet2011}).

In general, Arabic loanwords are normally well integrated in MSAL morphology, probably because of the high degree of structural similarity that exists between these languages. One example, reported by \citet{Lonnet2011}, is that of \textit{bəḳerēt} ‘cow’, a fully integrated loan from Arabic used in Ḥarsusi and Western Yemeni Mehri, which possesses its own plural and diminutive form (\textit{bəḳār} and \textit{bəḳərēnōt}, respectively).

Arabic loans in several MSAL stand out because of their characteristic feminine ending in \textit{-V(h)} instead of \textit{-(V)t}, as in Śḥehri \textit{saʕah} ‘watch’ and \textit{ṭorəh} ‘revolution’ (but consider the more adapted \textit{ris͂t} ‘trigger’ from Omani Arabic \textit{rīšah}; \citealt{Lonnet2011}). 

It is also worth noting that the Arabic ending \textit{-V(h)} is replaced by its MSAL equivalent when the noun is in the construct state, that is, final \textit{-t} reappears. This would also happen in Arabic, but the alteration in the quality of the vowel is a clear signal that the suffix is to be considered an MSAL morpheme. Consider the following example from Morris’ Baṭḥari recordings: 

\ea
Baṭḥari\footnote{Audio 
  file ‘20130929\_B\_B02andB04\_storyofcatchingturtle’ recorded, transcribed and kindly shared with Fabio Gasparini by Miranda Morris. The recording was produced in the context of Morris’ and Watson’s “Documentation and Ethnolinguistic Analysis of the Modern South Arabian languages” project, funded by the Leverhulme Trust. More recordings are accessible at the ELAR archive of SOAS University of London. The transcription has been adapted.
  }\\
\gll mʕayš-it-həm bəss mʕayš-it-həm ḥawla ʕār ḥāmis w-ṣayd śālā mʕayš-ah ḥawīl\\
     sustenance-\textsc{f.cs-3pl.m} only sustenance-\textsc{f.cs-3pl.m} once only turtle and-fish nothing sustenance-\textsc{f} once\\
\glt `Their sustenance, only that! Their sustenance once was only turtle and fish, there was nothing to eat in the past.'
\z
 

The word \textit{mʕayšah} ‘sustenance, food’ is a loanword from Arabic (as the -\textit{ah} ending suggests). When suffixed with the possessive \textsc{3pl.m} pronoun \textit{-həm}, however, Baṭḥari \textit{-it} replaces \textit{-ah/-at} (note also, in the example, the use of the restrictive adverbial particle \textit{bəss} ‘only’, which is a well-integrated loan from dialectal Arabic and occurs in alternation with Baṭḥari \textit{ʕār}). 

Finally, in Baṭḥari the Arabic definite article \textit{(a)l-} is occasionally used instead of the MSAL definite article \textit{a-}:  \textit{bə-l-ḫarifēt} ‘during the rainy season'.


 \subsubsection{Pronouns}\label{sec:key:pro}

The influence of Arabic can be observed, to an extent, even in the pronominal system, especially in those MSAL that are more exposed to contact due to the limited size of their speech communities. \citet{Lonnet2011}, for instance, reports how, despite the fact that in the MSAL the first person suffix pronoun is normally an invariable \textit{-i}, in Ḥarsusi this can be replaced by \textit{-ni} after verbs and prepositions (as is the case with Arabic; see also §\ref{sec:key:syn} for another interesting example concerning the marking of pronominal direct objects).

In addition, Baṭḥari relative pronouns (\textsc{sg}: \textit{l-,} \textit{lī} \textsc{pl}: \textit{əllī}) are close to their equivalent in Janaybi Arabic (and diverge from the rest of MSAL, where a \textit{ð-} element can be found). Baṭḥari has also borrowed the reflexive pronoun \textit{ʕamr-} ‘oneself’ from the Arabic dialect of the Janaba, despite the existence of an original Baṭḥari term with the same meaning, \textit{ḥanef}{}- (note that both terms must always be followed by a suffix personal pronoun). \textit{ʕamr-} has also been given a plural form in Baṭḥari, based on MSAL derivational patterns, \textit{ḥaʕmār-} \citep[14]{Morris2017}.


 \subsubsection{Baṭḥari verbal plural marker \textit{{}-uw}} 

Baṭḥari differs from the rest of the MSAL in that all 2/\textsc{3pl.m} verbal forms are marked by an -\textit{uw} suffix, while in the other languages of the group these persons are marked by a -\textit{Vm} ending and/or by internal vowel change (e.g. Mehri and Ḥarsusi -\textit{kə(u)m} for the \textsc{2pl.m} and -\textit{ə(u)m/}umlaut for the \textsc{3pl.m} of the perfective conjugation\textit{;} \textit{t-…-ə(u)m} and \textit{y/i-…-ə(u)m} respectively for 2 and \textsc{3pl.m} of the imperfective conjugation; \citealt{Simeone-Senelle2011}: 1093--1094). 

The origin of this suffix is uncertain. Its presence might well be connected to contact with Arabic (neighboring dialects have an \textit{{}-u} or \textit{{}-}\textit{ūn} suffix in the \textsc{3pl.m} person of the verb in both the perfective and imperfective conjugation) or to otherwise unattested stages of development internal to the MSAL verbal system. In this regard, Rubin (\citeyear[5]{Rubin2017}) suggests for {Mehreyyet} the presence of a subjacent \textit{{}-ə{}-} in 2\textsuperscript{nd}/3\textsuperscript{rd} plural masculine verbal suffixes which could therefore be somehow related to the Baṭḥari \textit{{}-uw} marker. However, the optional simultaneous presence of apophony within the stem of \textsc{3pl.m} verbal forms (similarly to what happens elsewhere in the MSAL), together with scarcity of data, prevents any conclusive assessment of the topic.


 
 \subsection{Syntax}\label{sec:key:syn}


At present, the syntax of the various MSAL has not been made the object of detailed investigation. The only scientific work dealing with this topic is Watson's (\citeyear{Watson2012}) in-depth analysis of Mehri syntax. However, Watson’s thorough description provides only sporadic insights into the issue of language contact (as for instance the use in Mahriyōt, the Eastern Yemeni variety of Mehri, of a \textit{swē} \textit{{\textasciitilde}} \textit{amma…} \textit{yā} construction to express polycoordination, probably to be regarded as the result of Arabic influence; \citealt{Watson2012}: 298). In general, though, the topic is left unaddressed in the literature, and more research is needed.

Gasparini’s data on Baṭḥari offer an interesting example of Arabic influence on MSAL syntax. In Baṭḥari, as in the other MSAL, pronominal direct and indirect objects may require a particle \textit{t-} to be inserted between them and the verb, depending on the morphological form of the verb itself. Masculine singular imperatives, for instance, require the presence of the marker, as shown in the following example:

\ea	
{Baṭḥari (Gasparini, unpublished data):}\\
\gll zum t-ī t-ih\\
     give.\textsc{imp} \textsc{acc-1sg} \textsc{acc-3sg.m} \\
\glt `Give it to me.'
\z

Example \REF{ex:key:ani}, in contrast, shows that the pronominal indirect object \textit{-(ə)nī} is suffixed directly to the verb as it would be in Arabic (see §\ref{sec:key:pro}).

\ea\label{ex:key:ani}
{Baṭḥari \citep[66]{Gasparini2018}:}\\
\gll zɛm-ənī θrɛh\\
     give.\textsc{imp}-\textsc{1sg} two\textsc{.m}\\
\glt `Give me both (of them).'
\z

 In other words, the introduction of the Arabic form of the object pronoun has caused the Baṭḥari object marker to disappear. Note that informants judged the alternative construction \textit{zum} \textit{t-ī} \textit{θrɛh}, (with the use of the object marker \textit{t}- and the 1\textsc{sg} object pronoun marker \textit{-ī}) to be acceptable, but this form was not produced spontaneously.

A peculiarity of the MSAL spoken in Oman is the use of circumstantial qualifiers, a type of clausal subordination well attested in Gulf Arabic \citep{Persson2009}. Baṭḥari regularly introduces predictive and factual conditional clauses asyndetically by using the structure [\textsc{sbj.pro} w-\textsc{sbj.pro}]. Consider \REF{ex:key:het}:

\ea\label{ex:key:het}
{Baṭḥari (Gasparini, unpublished data)}\\
\gll hēt w-hēt aṣbaḥ-k aḫayr saḥīr-e t-ōk lā w ham aṣbaḥ-k aḫass hāmā-k? w-marað̣ zēd l-ōk nḥā saḥīr-e t-ōk śkīl-e t-ōk mən a-gab\\
     \textsc{2sg.m} and-\textsc{2sg.m} wake\_up\textsc{.prf}-\textsc{2sg.m} better brand.\textsc{ptcp-pl.m} \textsc{acc-2sg.m} \textsc{neg} and if wake\_up.\textsc{prf-2sg.m} worse hear.\textsc{prf-2sg.m} and-illness huge to-\textsc{2sg.m} \textsc{1pl} brand.\textsc{ptcp-pl.m} \textsc{acc-2sg.m} scar.\textsc{ptcp-pl.m} \textsc{acc-2sg.m} because\_of \textsc{def}-infection.\\
\glt `In case you wake up feeling better / (we) do not brand you but in case you wake up feeling worse / do you understand? And you are seriously ill / we brand you and scar you because of the infection.'
\z

The first clause \textit{hēt} \textit{w-hēt} \textit{aṣbaḥ-k } \textit{aḫayr} is an asyndetical circumstantial qualifier functioning as a predictive conditional clause. It contrasts with \textit{w} \textit{ham} \textit{aṣbaḥ-k} \textit{aḫass}, in which the conjunction \textit{ham} introduces a counterfactual conditional clause. 

In Omani Mehri conditional clauses are commonly introduced through conjunction of pronouns \citep[211]{WatsonAl-Mahriforthcoming}. This structure is unattested in Yemeni Mehri:

\ea\label{ex:key:}
{Mehri \citep[211]{WatsonAl-Mahriforthcoming}}\\
\gll     sēh wa-sēh t-ḥam-ah lā ḥib-sa yi-ḳal-am t-ēs ta-ghōm š-ih lā\\
     \textsc{3sg.f} and-\textsc{3sg.f} \textsc{3sg.f}{}-want.\textsc{impf-3sg.m} \textsc{neg} parents-\textsc{3sg.f} \textsc{3m}-let.\textsc{impf-pl} \textsc{acc-3sg.f} \textsc{3sg.f}-go.\textsc{sbjv} with-\textsc{3sg.m} \textsc{neg}\\
\glt  `If she doesn’t want him, her parents won’t let her go with him.'\\
\z
     
These uses closely resemble those of Gulf Arabic, where circumstantial qualifiers are widely attested to codify predictive and factual conditional and consecutive clauses.


 
 \subsection{Lexicon}\label{sec:key:lex}


In the case of the MSAL, it can often be difficult to clearly set apart the effects of Arabisation from those of modernisation and lifestyle changes (which is not surprising, since the two phenomena are interrelated). According to what the speakers themselves report, 

\begin{quote}
{it was only since the introduction of formal education, and the awareness of} [Modern Standard Arabic] via the media, that Arabic became the second language for many of the MSAL speakers in Dhofar, and, in the case of younger speakers, often to the detriment of their proficiency in their MSAL variety \citep[11]{Davey2016}.
\end{quote}

As a consequence, phenomena of borrowing (such as code-switching and loanwords) are particularly common, especially in those varieties (and in the idiolects of individuals) which are more exposed to Arabic. The following is a good example of code-switching in Baṭḥari (note that the speaker in question tended to employ Janaybi forms more than other informants):

\ea\label{ex:key:allah}
{Baṭḥari (Gasparini, unpublished data)}\\
\gll mɛ̄t məssəlīm nə-šāhəd l-ōk w-y-sabbah-uw w-y-kabbər-uw w-y-hālul-uw\\
     die.\textsc{prf.3sg.m} muslim \textsc{1pl-}say\_šahada.\textsc{impf} for-\textsc{2sg.m} and-\textsc{3m-}pray.\textsc{impf-pl} and-3\textsc{m}-pray-\textsc{pl} and-\textsc{3m-}praise\_allah-\textsc{pl}\\
\glt `(If) a muslim dies / we say the \textit{šahada} for you / and they pray and say ‘\textit{allāhu} \textit{ʔakbar’} and praise Allah.'
\z

In \REF{ex:key:allah} the speaker makes use of several Arabic verbs related to the semantic field of religious practices, which are not lexically encoded in Baṭḥari. This might indirectly show the introduction of new ritual practices at a certain point of the history of the tribe. Note that C\textsubscript{2}-geminate stems such as \textit{ysabbahuw} and \mbox{\textit{ykabbəruw}} represent verbal patterns not attested in MSAL morphology, and are therefore easily identifiable as loans.

Morris (\citeyear[15]{Morris2017}) makes the important remark that lexical erosion is directly connected with the loss of importance of a language in the eyes of its speakers. She gives the example of the Baṭḥari word for ‘home, living quarters’, for which speakers nowadays frequently resort to some version of Arabic \textit{bayt}, while the many possible original synonyms are falling into disuse. Many of these (\textit{kədōt}, \textit{mōhen}, \textit{mašʕar}, \textit{mōḫayf}, and \textit{ḫader}) are connected to traditional ways of living which have all but disappeared in the course of the last 40--50 years, so that speakers probably judge them inadequate to refer to modern built houses. 


 \subsubsection{Numerals}

Watson (\citeyear[3]{Watson2012}) reports that ``[w]hile  Mehri cardinal numbers  are  typically  used  for  both  lower  and  higher  cardinals  in  Mehreyyet, Mahriyōt  speakers,  in  common  with  speakers  of  Western  Yemeni  Mehri, almost invariably use Arabic numbers for cardinals above 10.'' This type of lexical substitution connected to numerals higher than ten is also mentioned by \citet{Lonnet2011} and \citet[1088]{Simeone-Senelle2011}, who states that ``[n]owadays the MSAL number system above 10 is only known and used by elderly Bedouin speakers.'' \citet[90]{WatsonAl-Mahri2017} note that it is mostly younger generations (especially in urban settings) who have lost the ability to count beyond ten. Interestingly, they point out that telephone numbers are given exclusively in Arabic, “possibly due to the lack of a single-word MSAL equivalent to Arabic \textit{ṣufr} ‘nothing’.”


 \subsubsection{Spatial reference terms}

According to Watson \& Al-Mahri (\citeyear[91]{WatsonAl-Mahri2017}) the MSAL employ topographically variable absolute spatial reference terms. In other words, these terms can differ depending on the language employed, the moment of the utterance and the position of the speaker in relation to absolute points of reference. For instance, in and around the city of Salalah in Dhofar, both in the mountains and on the coastal plain, the equivalents of the words for ‘sea’ and ‘desert’ are used to indicate south and north, respectively, in both Mehri (\textit{rawram} and \textit{nagd}) and Śḥehri (\textit{ramnam} and \textit{fagir}). This is because the sea lies to the south and the desert lies to the north (beyond the mountains). In other parts of the coastal plain, however, the word for ‘mountains’ (\textit{śḥɛr}) is used to indicate north instead. Another common way to describe south and north is to refer to the direction in which the water flows, with the result that the same word that means ‘south’ on the sea-side of the mountains can be used to indicate ‘north’ on the desert-side. However, all these rather complex sets of terms are being rapidly replaced, particularly in the speech of the younger generations and among urban populations, with the Arabic words for south and north (\textit{ǧanūb} and \textit{šimāl} respectively).


 \subsubsection{Colour terms}

The MSAL lexically encode four basic colour terms: white, black, red and green \citep[261--262]{Bulakh2004}. For example, in Śḥehri one can find \textit{lūn} for ‘white’, \textit{ḥɔr} for ‘black’, \textit{ʕɔfər} for ‘red’ (and warm colours in general, including brown) and \textit{śəẓ́rɔr} for ‘green’ (and everything from green to blue). A fifth colour term, \textit{ṣɔfrɔr} ‘yellow’ (Mehri \textit{ṣāfər}), is most probably an adapted borrowing from Arabic already present at the common MSAL level \citep[271]{Bulakh2004}. 

A preliminary field inquiry on the subject was conducted by Gasparini in 2017, with 6 young speakers from the city of Salalah and its immediate surroundings, all between 20 and 35 years old and all bilingual in Śḥehri and Arabic. The results of the tests showed a remarkable degree of idiolectal variation in the colour labeling systems employed by the informants, with different levels of interference from Arabic. Remarkably, when asked to label colours in Śḥehri from a printed basic colour wheel, which was shown to them during interviews, all the speakers used the Arabic word for ‘blue’, \textit{azraq}, which seems to have replaced \textit{śəẓ́rɔr} (traditionally used for both blue and green, but now confined to the latter). Two speakers also used \textit{aḫḍar} for ‘green’, claiming that they could not recall the Śḥehri term. In addition, only one speaker used \textit{ʕɔfər} for ‘brown’, Arabic \textit{bunnī} being preferred by the other interviewees. The three basic colours ‘white’, ‘black’ and ‘red’, however, were regularly referred to using the Śḥehri forms by all speakers. Summing this up, it would seem that the Śḥehri colour system (at least in urban environments, but see below) is undergoing a radical process of restructuring. The three typologically fundamental colour terms are retained in most contexts, and a distinction between blue and green is being introduced through reduction of the original semantic spectrum of \textit{śəẓ́rɔr}, adoption of the Arabic word for blue, and subsequent replacement of \textit{śəẓ́rɔr} with \textit{aḫḍar} (which indicates only green in Arabic). Further distinctions are either being replaced with the corresponding Arabic terms, or introduced if not part of the original semantic inventory of the language.

On this matter, Watson \& Al-Mahri (\citeyear[90]{WatsonAl-Mahri2017}) argue that colour terms (together with numbers) are often among the first lexical items to be lost in contexts of linguistic endangerment, and that this is precisely the case with the MSAL. They write that even children in rural communities are now employing Arabic terms to refer to the different breeds of cattle (which traditionally used to be referred to by use of the three basic colour terms ‘white’, ‘black’ and ‘red’). This is probably a result of the fact that even in villages younger generations are no longer involved in cattle herding. Examples include \textit{aḥmar} ‘bay’ in place of Mehri \textit{ōfar} or Śḥehri \textit{ʕofer}, \textit{aswad} ‘black’ in place of Mehri \textit{ḥōwar}, and \textit{abya\d{ð}}̣ ‘white’ in place of Mehri \textit{ūbōn}.


 \subsubsection{Other word classes}

Watson \& Al-Mahri (\citeyear[90]{WatsonAl-Mahri2017}) note that, since the introduction of a public school system in Arabic in the 1970s, a number of common lexical items and expressions in Mehri and Śḥehri have been replaced by the corresponding Arabic ones. \citet{Lonnet2011} also remarks that borrowings from Arabic are particularly common among particles and function words, Examples include \textit{nafs} \textit{aš-šī} ‘the same thing’ for Śḥehri \textit{gens}, Mehri \textit{gans}; \textit{lākin} ‘but’ in place of Śḥehri \textit{duʰn} and \textit{min} \textit{duʰn}, Mehri \textit{lahinnah}; \textit{yaʕnī} ‘that is to say’ and \textit{ʕabārah} in place of Śḥehri \textit{yaḫīn}, Mehri \textit{(y)aḫah}; \textit{tamām} ‘fine’ in place of Śḥehri \textit{ḥay\~{s}ōf} and Mehri \textit{hīs} \textit{taww} \textit{{\textasciitilde} histaww}; Mehri and Ḥarsusi vocative \textit{yā} ‘oh’ in place of MSAL \textit{ʔā}-; Śḥehri \textit{bdan}, Mehri \textit{ʔabdan} ‘never, not at all’, against Mehri and Ḥarsusi \textit{bəhawʔ}, Śḥehri \textit{bhoʔ}. Consider also the case of Arabic \textit{bəss} ‘only’, already mentioned in §\ref{sec:key:nomorph}. In Mehri as in Baṭḥari, this particle appears now to be interchangeable with its equivalent \textit{ār}, as example \REF{ex:key:sima} shows:

\ea\label{ex:key:sima}
{Mehri (\citealt{Sima2009}: 328, cited in \citealt{Watson2012}: 371; transcription adapted)}\\
\gll bass ta-ṭʕam-h ḳād aḫah ār ṭʕām ð-maḥḥ\\
     only 2\textsc{sg.m}-taste.\textsc{impf-3sg.m} \textsc{int} fine only taste of-clarified\_butter\\
\glt `Just taste it, like it is just the taste of clarified butter.' 
\z

As is predictable, also in this field Baṭḥari is the language most affected by Arabic: besides those already cited, we might add the expressions \textit{zēn} ‘well’, \textit{(a)barr} ‘outside’ (also in Mehri, as opposed to Soqoṭri \textit{ter}), \textit{ḫalaṣ} ‘and this is it’ (used to end a narrative). Finally, Watson (\citeyear[3]{Watson2012}) remarks how ``Mahriyōt also exhibits structures unattested in Mehreyyet such as ‘What X!’ phrases reminiscent of Arabic, e.g. \textit{maṭwalk} ‘How tall you m.s. are!’’.

\section{Conclusions}\label{sec:4}

Throughout this chapter we have repeatedly pointed out how research on the MSAL, and in particular on the effects that contact with Arabic has had on their evolution, is still far from reaching its mature stage. Much remains to be done, in particular, in terms of sheer documentation, especially in the case of the most endangered varieties (Hobyōt, Ḥarsūsi, Baṭḥari). In addition to this, and although Watson’s (\citeyear{Watson2012}) work has greatly contributed to expanding our knowledge in this area, MSAL syntax remains a strongly neglected field of inquiry. Finally, our knowledge of the history of the MSAL prior to the twentieth century (and therefore the history of their contact with Arabic) is extremely poor. 

It must also be remarked that, although the most widely spoken among the MSAL are undoubtedly better documented, very little is known about the effects that urbanisation has had on their speech communities in recent years. In particular, anecdotal evidence suggests that the varieties of Śḥehri and Soqoṭri spoken in Salalah and Ḥadibo are undergoing rapid change under the influence of Arabic (both the standard variety of the language, which children learn in school, and the dialects). Fieldwork conducted in the two abovementioned urban centres could provide extremely valuable information concerning the effects of contact between Arabic and Modern South Arabian.

Despite the far-from-complete state of research in this field, what we currently know is sufficient to say that contact has had a strong impact on the MSAL. Though this is more evident in the area of lexicon, where borrowings are legion, phonetics and phonology have also been affected (though to a different extent from one language to another). Morphology and syntax, on the contrary, appear to be more resistant to contact-induced change, though in the most endangered varieties one can notice a partial disruption of the original pronominal system and verbal paradigm, and though the seemingly high degree of resistance to external influence shown by MSAL syntax could actually be due to our limited knowledge of the subject.

One last note is due concerning another heavily neglected topic, namely the effects that contact with the MSAL have had on spoken Arabic. Though we have not addressed the question in the course of this paper, evidence drawn from the existing literature \citet{Simeone-Senelle2002} suggests that this influence, too, is not completely absent, and that further research in this direction could produce interesting results. 

\section*{Further reading}

 \citet{Morris2017} can be thought of as a general introduction to contact between MSAL and Arabic. \\
 \citet{WatsonAl-Mahri2017} offer an intriguing account of how language change, contact with Arabic and changes to the traditional environment are all deeply interrelated. \\ \citet{Lonnet2011} – although limited in scope and extension due to its nature as an encyclopedic entry – offers interesting highlights on the effects of contact on the MSAL.
 
 
\section*{Abbreviations}

\begin{tabularx}{.5\textwidth}{@{}lQ@{}}
\textsc{1, 2, 3} & 1st, 2nd, 3rd person \\
\textsc{acc} & accusative \\
\textsc{cs} & construct state \\
\textsc{f} & feminine \\
\textsc{imp} & imperative \\
\textsc{impf} & imperfect (prefix conjugation) \\
\textsc{int} & intensifier \\
\textsc{m} & masculine \\
\end{tabularx}%
\begin{tabularx}{.5\textwidth}{@{}lQ@{}}
\textsc{neg} & negative \\
MSAL & Modern South Arabian \\
\textsc{ptcp} & participle \\
\textsc{pro} & pronoun \\
\textsc{prf} & perfect (suffix conjugation) \\
\textsc{pl} & plural \\
\textsc{sg} & singular \\
\textsc{voc} & vocative \\
\end{tabularx}%


\sloppy
\printbibliography[heading=subbibliography,notkeyword=this]
\end{document}