\documentclass[output=paper]{langsci/langscibook} 
\title{Moroccan Arabic}
\author{Jeffrey Heath\affiliation{University of Michigan}}
% \chapterDOI{} %will be filled in at production

% % % \epigram{Change epigram in chapters/01.tex or remove it there }

\abstract{Morocco, even if the disputed Western Sahara is excluded, is rivaled only by Yemen in its variety of Arabic dialects. Latin/Romance sub- and ad-strata have played crucial roles in this, especially 1) when Arabized Berbers first encountered Romans; 2) during the Muslim and Jewish expulsions from Iberia beginning in 1492; and 3) during the colonial and post-colonial periods.}

\maketitle
\begin{document}

\section{History and current state}

\subsection{History}

\ili{Moroccan Arabic} (MA) initially took shape when Arab-led troops, probably Arabized Berbers from the central Maghreb who spoke a contact variety of \ili{Arabic}, settled precariously in a triangle of Roman cities/towns consisting of Tangier, Salé, and Volubilis, starting around 698 AD. Mid-seventh-century tombstones from Volubilis, inscribed in \ili{Latin}, confirm that Roman \isi{Christians} were present, though in small numbers, when the Arabs arrived. Shortly thereafter, in 710–711, an Arab-led army from Morocco began the conquest of southern Spain, a richer and more secure prize that drew away most of the Arab elite. In Morocco, turnover of the few Arabs and of their Arabized Berber troops was high; they were massacred or put to flight in the Kharijite revolt of 740. The eighth and ninth centuries had perfect conditions for the development of a home-grown \ili{Arabic} in the Roman triangle in Morocco, and in the emerging Andalus, with a strong Latinate substratum.  

The first true Arab city, Fes, was not founded until approximately 798, a century after the first occupation of Morocco, and its population did not bulk up until immigration from Andalus and the central Maghreb began around 817. With a cosmopolitan population, and located outside of the old Roman triangle, its \ili{Andalusi} and non-\ili{Andalusi} quarters may have maintained their respective dialects for a long time. The remainder of Morocco was occupied by Berber tribes until much later.

During the eleventh century, the Arabian \ili{Bedouin} often called Banu Hilāl entered the central Maghreb in large numbers (cf. Benkato, this volume).\ia{Benkato, Adam} They partially bedouinized the \ili{Arabic} dialects in Tunisia and Algeria, producing hybrid varieties that combined pre- and post-Hilalian features. They also gradually pushed their way south and west across the Sahara, bringing their distinctively \ili{Bedouin} \ili{Arabic}, known as \ili{Ḥassāniyya}, into the southern Maghreb, including some oases of southern Morocco proper and the entire Western Sahara. Meanwhile, hybridized Algerian dialects, also reflecting a \ili{Berber} substratum, were spreading into western Morocco, taking \isi{root} in new farming villages in the central plains around Fes, and in the younger cities such as Meknes and Marrakesh \citep{Heath2002}.

In 1492, the Catholic Kings abruptly expelled Spanish \isi{Jews} from Spain, followed by expulsions through 1614 of \isi{Muslims} from Spain and Portugal (see also Vicente, this volume).\ia{Vicente, Ángeles@Vicente, Ángeles} Jewish deportees, whose predominant home language was Judeo-\ili{Spanish}, flooded into the Jewish quarters (\textit{mellahs}) of Moroccan cities, constituting a new Jewish elite. Muslim deportees, variably speaking \ili{Arabic} or \ili{Romance}, arrived in several waves and were more easily assimilated. The Jewish presence in Morocco was strong until 1951, when most \isi{Jews} left for Israel and other destinations.

Moroccan ports participated in growing Mediterranean and Atlantic maritime activity, associated linguistically with \ili{Lingua Franca} (cf. Nolan, this volume)\ia{Nolan, Joanna} and various \ili{Romance} languages along with \ili{Turkish}, in the seventeenth and eighteenth centuries. European precolonial penetration into coastal Morocco in the late nineteenth century later expanded during the French and much smaller {Spanish} protectorates which lasted from 1912 to 1956. Exposure to \ili{French} increased dramatically in this period. 

Also of linguistic relevance is the fact that the Moroccan–Algerian border has been virtually closed for decades, due mainly to political disputes. This has partially sealed off Morocco from the central Maghreb and allowed a specifically Moroccan koiné to flourish.

\subsection{Current situation}

Of the 33 million Moroccans recorded in a 2014 census, nearly all are fluent L1 or (among the \ili{Berber}-speaking minority) L2 speakers of some form of MA. Moreover, except in the thinly populated Western Sahara, the once-robust dialectal variation within MA has now been greatly compressed. The MA that one is likely to hear in cafés in Rabat, Fes, Meknes, Marrakesh, Oujda, and even Tangier is the Moroccan koiné, a hybridized variety mixing pre- and post-Hilalian features and showing heavy \ili{Berber} influence in \isi{prosody} and vocalism.

Many \ili{Berber} dialects, commonly (but inaccurately) classified into three languages (\ili{Tarifiyt}, \ili{Tamazight}, and \ili{Tashelhiyt}), are still widely spoken in the mountain ranges and in the Souss valley along the Atlantic coast near Agadir. However, these \ili{Berber} languages are full of \ili{Arabic} loans, and they are slowly losing ground to \ili{Arabic} in all of the cities and large towns.

\section{Contact languages}

\subsection{General}

This chapter focuses on contact between MA and European languages. \ili{Punic} (Phoenecian) had probably died out locally before the Arab conquest, and \ili{Greek} was a non-factor in spite of nominal Byzantine suzerainty after the fall of the Roman empire. \ili{Berber}–\ili{Arabic} contact is covered elsewhere (see Souag, this volume and Benkato, this volume).\ia{Souag, Lameen}\ia{Benkato, Adam} Diglossic borrowing from literary \ili{Arabic} would take us far afield; on this, see \citet{Sayahi2014} and \citet{Heath1989}.

The hallmark of abrupt \isi{language shift} is powerful substratal influence in phonology and \isi{prosody}. Some \isi{calquing} of grammatical constructions may occur, but this can be difficult to tease apart from morphosyntactic \isi{simplification}. There may be little or no carryover of \isi{core vocabulary} and of concrete grammatical morphemes. The profile of \isi{language shift} contrasts with that of adstratal borrowing during prolonged \isi{bilingualism}, whose manifestations are mainly lexical, and whose complexities involve the morphological and semantic nativization of foreign-source inflected forms (cf. Manfredi, this volume).\ia{Manfredi, Stefano}

\subsection{Late Latin}

The best-kept secret about MA is that, unlike the case elsewhere in the Maghreb, its oldest forms originated by \isi{language shift} (probably rapid) from \ili{Late} Latin (\ili{LL}) to a contact \ili{Arabic} spoken by Berber troops.

There are no written records of colloquial \ili{LL} of the relevant period, either in North Africa or in Europe, but we can surmise that the \ili{LL} spoken in the Roman triangle was intermediate between \ili{Classical Latin} and early Medieval \ili{Romance}, e.g. Medieval \ili{Spanish}. This implies either five or possibly seven vowel qualities, phonemic \isi{stress}, no \isi{vowel length}, and probably some affricates \textit{č} [ʧ] and \textit{ǧ} [ʤ].

\subsection{Medieval Judeo-Spanish}

The major injection of Medieval \ili{Spanish} into the Moroccan heartland was the arrival of expelled Spanish \isi{Jews} in 1492. They joined existing Jewish communities in the large cities, but a cultural divide between the newcomers (\textit{megorashim}) and incumbents (\textit{toshavim}) quickly emerged. We know from rabbinical responsa that Judeo-\ili{Spanish} was still spoken in the central cities for two centuries after 1492 \citep{Chetrit1985}. In far northern Morocco, a form of \ili{Arabic}- and \ili{Hebrew}-influenced Judeo-\ili{Spanish} called \ili{Hakitia} or Haketia remained in vernacular use until the early twentieth century \citep{Benoliel1977}, after which it \isi{merged} with Modern \ili{Spanish}.

\subsection{Modern French and Spanish}

\ili{Spanish} and to some extent \ili{Portuguese} and \ili{Catalan} remained contact influences chiefly in ports through the late nineteenth century, when direct Spanish involvement in northern Morocco became more significant. \ili{Iberian} \isi{loanwords} figure prominently in the early twentieth-century maritime vocabulary provided by \citet{Brunot1920}. During the Protectorates, \ili{French} became a major language of education and administration in most of Morocco, especially in the west-to-east Casablanca–Rabat–Fes–Meknes–Taza corridor, while \ili{Spanish} consolidated its position in the far north. \ili{French} \isi{loanwords} during the early Protectorate are in \citet{Brunot1949}. MA–\ili{French} and MA–\ili{Spanish} \isi{bilingualism} has increased in the postcolonial period due to media and mass education. \ili{English} influence is increasing, mainly through tourism, science education, and finance.

\section{Contact-induced changes in MA}

\subsection{Phonology}

MA dialects – archaic \ili{Pre-Hilalian}, hybridized \ili{Post-Hilalian}, and in the far south the unhybridized \ili{Ḥassāniyya} – differ sharply in vocalic systems, reflecting their different histories \citep{Heath2018}.

\ili{Classical Arabic} (\ili{CA}) had short \{\textit{ĭ} \textit{ă} \textit{ŭ}\} versus long \{\textit{ī} \textit{ā} \textit{ū}\}, \isi{diphthongs} \{\textit{ăy} \textit{ăw}\}, no syncope, and no phonemic \isi{stress}. 

Of the three main types of MA, \ili{Ḥassāniyya} is closest to \ili{CA}. It has short vowels limited to closed syllables: \{\textit{ə} \textit{ă}\} with \textit{ə} < \{*ĭ *ŭ\}, in some dialects (e.g. \ili{Mali}) also some cases of \textit{ŭ}. It distinguishes long \{\textit{ī} \textit{ā} \textit{ū}\} from \isi{diphthongs} \{\textit{ăy} \textit{ăw}\}, and has no phonemic \isi{stress}, but unlike \ili{CA} it does allow syncope of short vowels (cf. \citealt{Taine-Cheikh1988article}). \ili{Ḥassāniyya} shows limited effects of language contact in the phonology of \ili{Berber} \isi{loanwords} (cf. \citealt{Taine-Cheikh1997Zenaga}).

By contrast, the koiné and some other hybrids reduce all three short vowels to just one short vowel \textit{ə} with various allophones, contrasting with full vowels \{\textit{i} \textit{a} \textit{u}\}. The hybrid dialects monophthongize \{*ăy *ăw\} to \isi{merge} with \{\textit{i} \textit{u}\}. The rounding of original short *ŭ often survives next to a velar/uvular consonant, even after syncope (which is productive), suggesting an ongoing feature \isi{transfer} that, if and when fully implemented, would result in underlying labiovelars \{\textit{kʷ} \textit{gʷ} \textit{qʷ} \textit{ḫʷ} \textit{ɣʷ}\} next to \textit{ə} (which becomes phonetic [ʊ]) or before a consonant. Again there is no phonemic \isi{stress}. This is a \ili{Berber}-like system, reflecting deep long-term substratal/adstratal contact.

A more archaic \ili{Berber}-like system, still preserving at least the opposition of short *ĭ~{\textasciitilde}~*ă versus *ŭ and likely at least some \isi{diphthongs}, was brought to Morocco by the early Arabized Berber troops. There it was overlaid on an \ili{LL} substratum that had five to seven vowel qualities, phonemic \isi{stress}, no syncope, and no \isi{vowel length}. The resulting \ili{Pre-Hilalian} MA has: three regular vowels \{\textit{i~a~u}\}, a subset of which (the original short vowels) syncopate in weak metrical positions; phonemic \isi{stress}; and a schwa vowel \textit{ə} confined to posttonic final closed syllables. The \isi{leveling} of \isi{vowel length} distinctions, and the re-splitting of the previously \isi{merged} *i~{\textasciitilde}~*a into \textit{i} and \textit{a} based on consonantal environment, were disruptive to the morphology (see §\ref{morphol}). Both the \isi{leveling}, and the new phonemic \isi{stress}, were shared with speakers of early \ili{Andalusi} \ili{Arabic}, which had a similar \ili{LL} substratum and whose first invaders came from Morocco. This points to an original dialect area in the eighth and ninth centuries, including coastal Andalus and at least the Tangier–Salé axis in Morocco (after Volubilis was abandoned in favor of Fes), differing significantly from even \ili{Pre-Hilalian} central \ili{Maghrebi} dialects, which likely never had major \ili{LL} substratal effects. 

The differences among MA dialect types can be illustrated by forms of ‘big’ (\tabref{tab:heath:1}). The suggested proto-forms are close to \ili{CA} but show some adjustments to short *ĭ and *ŭ. Acute accent marks \isi{stress} in \ili{Pre-Hilalian}. Observe especially that the two homophonous \ili{Pre-Hilalian} \textit{kbír} forms behave differently when a vowel-initial suffix is added. The morphological consequences of length \isi{merger} in \ili{Pre-Hilalian} are considered below. Emphatic /{\R}/ is phonemically distinct from plain /r/ in all varieties. 

\begin{table}
\caption{\label{tab:heath:1}\label{bkm:Ref13483830}The word-family `big' in MA dialect types}
\begin{tabular}{*{5}{l}}
\lsptoprule
{Gloss} & {\normalfont Proto} & {\normalfont Pre-Hilalian} & {\normalfont Hybrid} & {\normalfont Ḥassāniyya}\\\midrule
‘big’ (\textsc{sg.m}) & *kăbīr & \textit{kbír} & \textit{kbir} & \textit{kbīr}\\
‘big’ (\textsc{sg.f}) & *kăbīr-a & \textit{kbír-a} & \textit{kbir-a} & \textit{kbīr-a}\\
‘he got big’ & *kăbĭr & \textit{kbír} & \textit{kbər} & \textit{kbər}\\
‘she got big’ & *kăbĭr-at & \textit{kíbr-ət} & \textit{kbr-ət} & \textit{kəbr-ət}\\
‘bigger’ & *ăkbăr & \textit{kbá{\R}} & \textit{kbə{\R}} & \textit{(ă)kbă{\R}}\\
‘big (\textsc{pl})’ & *kŭbār & \textit{kba{\R}-ín} & \textit{kʷba{\R}} & \textit{kbā{\R}}\\
\lspbottomrule
\end{tabular}
\end{table}

Later adstratal borrowings from \ili{Spanish} and \ili{French}, as well as from \ili{CA}, predictably required adjustments to MA phonology. The most disruptive changes affected \ili{French} borrowings into MA (our data are best for the hybrid koiné). The rich array of \ili{French} vowel qualities had to be squeezed into three MA qualities. \ili{French} \{\textit{i} \textit{ü} \textit{e} \textit{ɛ}\} \isi{merge} as MA \textit{i}. \ili{French} \{\textit{u} \textit{o} \textit{ɔ} \textit{œ}\} \isi{merge} as MA \textit{u}. \ili{French} \textit{a} becomes MA \textit{a}. This compression has had considerable morphological consequences (see §\ref{morphol} below).

The main contribution of \ili{Romance} to MA consonantism is the affricate \textit{č} [ʧ]. In the current koiné, this is present as a \isi{phoneme} (if at all) in the \isi{loanword} \textit{lččina} {\textasciitilde} \textit{ltšina} ‘orange (fruit)’ < \ili{Spanish} \textit{la} \textit{China}, as brought out in the \isi{diminutive} which breaks up the \textit{čč} cluster, hence \textit{lčičin} {\textasciitilde} \textit{ltišin} and further variants \citep{Heath1999}. Archaic northern dialects have more examples of \textit{č}, and these dialects pronounce geminated \textit{ž} as affricate \textit{ǧ} [ʤ].

\subsection{Morphology} \label{morphol}

Direct borrowing of bound function morphemes is rare in MA as in other languages. A notorious exception is \textit{ta-…-t} in abstract nouns of profession, from the \ili{Berber} feminine singular, likely extrapolated from specific \ili{Berber} borrowings like \textit{ta-šəffa{\R}-t} ‘thief’. 

Another glaring exception is the set of D-possessives: \textit{d} (archaic \textit{di}) before nouns, \textit{dyal-} (\ili{Pre-Hilalian} \textit{dyál-}) primarily before pronominal suffixes (e.g. \textit{dyal-i} ‘mine’, \textit{dyal-u} ‘his’). The obvious etymology (\ili{Latin} \textit{dē} > \ili{LL} *de or unstressed *di) presents no phonological or semantic difficulties, but it was rejected by a century of \ili{Maghrebi} Arabists, who favored various far-fetched \ili{Arabic}-internal etymologies. However, an \ili{LL} source is also indicated by its dialectal distribution: \ili{Pre-Hilalian} MA, regional colloquial \ili{Andalusi} \ili{Arabic}, and certain coastal enclaves in Algeria that were likely settled by \ili{Andalusi} merchants. The mysterious prepronominal variant \textit{dyál-} was generalized from \ili{LL} *di él(l)u ‘his’ and \ili{LL} *di él(l)a ‘hers’, which are near-exact matches to the still extant \ili{Pre-Hilalian} \textit{dyál-u} ‘his’ and \textit{dyál-a} ‘hers’. The motivation for this admittedly unusual morphemic borrowing was the need for a new possessive morpheme as \ili{Arabic} dialects gradually abandoned the compound-like \ili{CA} “construct” possessive \citep{Heath2015}. The fact that possessive morphemes are not immune from borrowing is also shown by possessed forms of certain kin terms, with a \ili{Berber} nasal suffix, before nominal possessors in hybrid dialects, as in (koiné) \textit{ḅḅa-yn} \textit{ḥamid} ‘Hamid’s father’, cf. \textit{ḅḅa} ‘father’. 

Verbs as well as nouns are readily borrowed from \ili{Romance} languages into MA. This raises the question of which \ili{Romance} inflected form is borrowed, and what value it is assigned to within the MA \isi{tense} system, which groups 1st/2nd persons versus 3rd person subject splits in the perfect of some verb types. Most \ili{Spanish} verb borrowings look like \ili{Spanish} infinitives, e.g. \textit{f{\R}ina{\R}} ‘to brake’ (<~\textit{frenar}), but more likely reflect a cluster of forms based on this \isi{stem} shape in \ili{Spanish} itself. In addition to the \isi{infinitive}, this set also includes \isi{future} \textit{frenar-é}, \isi{conditional} \textit{frenar-ía}, and forms with \textit{d} instead of \textit{r}, namely \isi{participle} \textit{frenado} and imperative plural \textit{frenad}. Consonant-final borrowed verbs like \textit{f{\R}ina{\R}} behave like native MA quadriliteral verbs, and have identical perfect and imperfect forms. 

By contrast, \ili{French} verbs are regularly borrowed as weak (i.e. vowel-final) verbs, with imperfect and 1st/2nd perfect \textit{i}, versus 3rd-person perfect \textit{a}. An example is ‘declare’: imperfect \textit{\nobreakdash-ḍikla{\R}i} matching perfect 1st/2nd \textit{ḍikla{\R}i-}, versus 3rd \textit{ḍikla{\R}a(-)}. The likely crosslinguistic bridge is the conspicuous cluster of \ili{French} forms ending in orthographic \textit{{}-er} (\isi{infinitive}), \textit{{}-ez} (2\textsc{pl} subject), \textit{{}-ais/-ait/-aient} (imperfect), and \textit{-é(e)(s)} (\isi{participle}). All of these are phonetic [e] or [ɛ] and therefore \isi{merge} as MA \textit{i}, interpretable in MA as the imperfect and 1st/2nd perfect of weak verbs. The marked 3rd-person perfect with final \textit{a} is then easily formed by \isi{analogy} \ia{Lucas, Christopher} \ia{Čéplö, Slavomír@Čéplö, Slavomír}(cf. Lucas \& Čéplö, this volume: §4.2 for a parallel development in \ili{Maltese}).

The \isi{merger} of \isi{vowel length} in \ili{Pre-Hilalian} MA set off a chain reaction of morphophonological restructurings, most notably in the verbal system. The \ili{CA} three-way vocalic opposition of hollow verbs, e.g. for ‘to be’ imperfect \textit{kūn-}, preconsonantal perfect \textit{kŭn-}, and prevocalic (or word-final) perfect \textit{kān-}, is largely preserved in hybrid and \ili{Post-Hilalian} dialects. By contrast, in \ili{Pre-Hilalian} MA, after the momentous vowel-length \isi{merger}, the hollow paradigm was reorganized into a binary opposition of \textit{kún} (imperfect and 1st/2nd perfect) versus \textit{kán} (3rd perfect). This paradigmatic reorganization, which makes no sense semantically and is apparently unique to \ili{Pre-Hilalian} MA, then spread analogically to other verb types, including strong triliterals that have three consonants and no long vowels, e.g. ‘enter’: imperfect \textit{-tḫul} matching 1st/2nd perfect \textit{tḫul-}, but 3rd perfect \textit{tḫal}.

\subsection{Syntax}

Before reaching Morocco, spoken \ili{Arabic} had \isi{prepositions}, possessum--possessor, and \textsc{def--n--adj} order within NPs, preverbal \isi{negation} (cf. Lucas, this volume)\ia{Lucas, Christopher} and complementizers, a perfect/imperfect split in verbs, and pronominal-subject \isi{agreement} on verbs (expressed, in part, by suffixes). \ili{Romance} languages like \ili{Spanish}, and presumably eighth-century \ili{LL}, were already close to this profile, so opportunities for syntactic influence were limited. Some minor \ili{French} complementizers are common in educated MA, as in \textit{au} \textit{lieu} \textit{d’igulu…} ‘instead of them saying’, from \ili{French} \textit{au} \textit{lieu} \textit{de} ‘instead of’ plus MA \textit{igulu} ‘they say’. 

\subsection{Lexicon}

While the \ili{LL} substratum had a profound effect on early MA phonology and morphophonemics, and also left behind a morphemic souvenir in the form of D-possessives, not a single basic \ili{LL} lexical item can be shown to have been preserved in any archaic MA dialect. The most promising candidate for such a retention is dialectal MA \textit{qbṭal} and variants ‘elbow’. The likely etymon is \ili{LL} *cubitellu (later \ili{LL} *kubtɛllu), \isi{diminutive} of \ili{Latin} \textit{cubitu(s)} ‘elbow’, cf. Modern \ili{Spanish} \textit{codillo}. The other possibility, less straightforward semantically, is a reflex of the related adjective, \ili{Latin} \textit{cubitāle}, cf. Modern \ili{Spanish} \textit{codal}. In Morocco, \textit{qbṭal} ‘elbow’ survives in several Judeo-\ili{Arabic} dialects. For \isi{Muslims}, it was recorded in an unspecified location in the unpublished fichier of colonial-period linguist Georges Colin (\citealt{IraquiSinaceur1993}: 1525; \citealt{Prémare1998}: 224), and by me in the 1980s in archaic varieties of the Fes--Sefrou area. \textit{qbṭal} is completely unknown to the great majority of Moroccan \isi{Muslims}. Preservation of \textit{b} shows that \textit{qbṭal} is not a recent borrowing from any form based on Modern \ili{Spanish} \textit{codo}. The \textit{b} was still present in (very) \ili{Old Spanish} \textit{cobdo}, its \isi{diminutive} \textit{cobdillo}, and \textit{cobdal}. “\textit{Cubtíll}” ‘elbow’ is recorded for late \ili{Andalusi} \ili{Arabic} (\citealt{Corriente1997dictionary}: 412; \citealt{Dozy1967}: 302). The geographic and communal distribution of \textit{qbṭal}, especially among \isi{Muslims}, suggests that it was introduced into Morocco by late Medieval Jewish refugees. 

There are, however, hundreds of well-established \ili{Spanish} \isi{loanwords}, especially in northern Morocco. There, \ili{Spanish} is ubiquitous in schools and broadcast media, Spanish tourists are common, and many Moroccans serve as day-laborers in Spanish enclaves Ceuta and Melilla. While \ili{Spanish} got a precolonial head-start, \ili{French} has long since overtaken it in the rest of Morocco. Of special interest are cases where an original \ili{Spanish} borrowing was later gallicized, sometimes only in part. Examples are MA \textit{antiris} ‘(monetary) interest’, a hybrid of \ili{Spanish} \textit{interés} and \ili{French} \textit{intérêt}, and MA \textit{g{\R}abaṭa} ‘necktie’ from \ili{Spanish} \textit{corbata} and \ili{French} \textit{cravate}. Nonsynonymous mergers also occur, as with \textit{ga{\R}ṣun}, attested both as ‘waiter’ (\ili{French} \textit{garçon}) and ‘underpants’ (\ili{Spanish} \textit{calzón}). ‘To sign’ is now usually \textit{\nobreakdash-siɲi/siɲa} or \textit{\nobreakdash-sini/sina} (<~\ili{French} \textit{signer}), but an obsolescent Judeo-\ili{Arabic} variant \textit{siɲa{\R}} with (pseudo-)\ili{Spanish} infinitival ending is attested. Since the \ili{Spanish} synonym is the unrelated \textit{firmar}, MA \textit{siɲa{\R}} must have been formed by applying a borrowing routine “add \textit{-a{\R}} to the \isi{stem}” to \ili{French} stems, probably early in the \isi{colonial} period when still-abundant \ili{Spanish} borrowings were being replaced or hybridized under the influence of the newly dominant \ili{French}.

The process is now coming full circle, as \ili{English} influence expands. The weak verb alternation of final \textit{a/i} is productive for verbs borrowed from \ili{French}, as noted above (cf. again the close parallels in \ili{Maltese}; Lucas \& Čéplö, this volume: §4.2).\ia{Lucas, Christopher} \ia{Čéplö, Slavomír@Čéplö, Slavomír} A borrowing routine “add final \textit{a/i} to the \isi{stem}” extrapolated from \ili{French}/MA pairs, is now extended to \ili{English}, where it has no basis in \ili{English} \isi{inflectional} paradigms. Examples are the comical \textit{ka-y-spiki} \textit{mzyan} ‘he speaks (\ili{English}) well’, and junkie slang like \textit{tt-ṣṭuna} ‘he got stoned’ (\isi{participle} \textit{m-ṣṭuni} ‘stoned’). 

And then there are the many playful translinguistic inventions, concocted among groups of men sitting in cafés, sipping mint tea or smoking… whatever. Nearly all such inventions are ephemeral, but a few have caught on \citep{Heath1987}. Consider the fairly common koiné noun \textit{ḫwadri} ‘pal, buddy’. Unbeknownst to those who now use it, it must have arisen via two successive transformations. First, \ili{Spanish} \textit{padre} and \textit{madre} were playfully combined with the CCaCCi template for denominal occupational derivatives, as though derived from MA \textit{ḅḅa} {\textasciitilde} \textit{bu} ‘father’ and MA \textit{ṃṃ(ʷ)-} ‘mother’. Templatic CCa… is realized as Cwa… when based on a CV… input, as in \textit{ṣwabni} ‘seller of soap’ (<~\textit{ṣabun}). Combining CCaCCi with \textit{padre} and \textit{madre} produces the slang terms (attested but rare) \textit{ṗwaḍ{\R}i} and \textit{ṃwaḍ{\R}i}. The final and most ingenious step was to combine the sub-template Cwadr-i, emergent from these ‘father/mother’ forms, to \textit{ḫa-} {\textasciitilde} \textit{ḫu-} ‘brother’, outputting \textit{ḫwadri}, which then acquires the same ‘buddy’ sense as \ili{American} {English} \textit{bro}. 

\section{Conclusion and prospects}

The broad outlines of historical language contact in Morocco are becoming reasonably clear. The most urgent need is for more material and analysis of Mor\-occan Judeo-\ili{Arabic} (MJA), in forms accessible to international audiences. Ideally we would want to tease apart the original \ili{LL} influence on \ili{Pre-Hilalian} MJA, as preserved by the \textit{toshavim}, from the medieval Judeo-\ili{Spanish} brought to Morocco in 1492 by the \textit{megorashim}. 

Significant Moroccan Arab and \ili{Berber} expat communities exist in France, the Netherlands, Belgium, Germany, Switzerland, and Spain. These \textit{vacanciers} return to Morocco in large numbers during summer vacations and on Muslim holy days. There are opportunities to study them both in Europe \citep{Nortier1990} and in their interactions with other Moroccans. 

Another promising topic for investigation is a semi-pidginized form of MA used by \isi{monolingual} maids in large cities as a kind of foreigner talk to their expat French employers.

\section*{Further reading}

\citet{Heath1989} is a study of lexical and phrasal borrowing/\isi{code-switching} from European languages and from \ili{Standard} \ili{Arabic} in \ili{Moroccan Arabic}.\\
\citet{Nortier1990} examines language contact phenomena among Moroccans in the Netherlands.\\
\citet{Sayahi2014} is a regional study of \ili{Arabic} sociolinguistics and language contact from Spain through Morocco to Tunisia.

\section*{Acknowledgements}

Fieldwork in Morocco was supported by grants from the National Science Foundation (especially BNS 79-04779 in 1979-81) and by a Fulbright research fellowship in 1986. For support while working on MA material in the 1980s, thanks also to the National Endowment for the Humanities, the Deutscher Akademischer Austauschdienst, the Alexander von Humboldt Stiftung, and the {Hebrew} University of {Jerusalem}.


\section*{Abbreviations}
\begin{multicols}{2}
\begin{tabbing}
\textsc{ipfv} \hspace{1em} \= before common era\kill
CA         \> Classical Arabic\\
\textsc{f}  \> feminine\\
L2          \> second language \\
{LL}          \> Late Latin\\
MA         \> Moroccan Arabic\\
\textsc{m}  \> masculine\\
\textsc{pl} \> plural\\
\textsc{sg} \> singular
\end{tabbing}
\end{multicols}


{\sloppy\printbibliography[heading=subbibliography,notkeyword=this]}

\end{document}
