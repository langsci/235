\documentclass[output=paper]{langsci/langscibook} 
\author{Stephan Procházka\affiliation{University of Vienna}}
\title{Arabic in Iraq, Syria, and southern Turkey}
% \keywords{} 
\abstract{This chapter covers the Arabic dialects spoken in the region stretching from the Turkish province of Mersin in the west to Iraq in the east, including Lebanon and Syria. The area is characterized by a high degree of linguistic diversity, and for about two and a half millennia Arabic has come into contact with various other Semitic languages, as well as with Indo-European languages and Turkish. Bilingualism, particularly with Aramaic, Kurdish, and Turkish, has resulted in numerous contact-induced changes in all realms of grammar, including morphology and syntax.}
\maketitle

\begin{document}

\section{Current state and historical development}  \label{state} 

The region discussed in this chapter is linguistically extremely heterogeneous: in it three different Arabic dialect-groups, plus several other languages, are spoken. The two main Arabic dialect-groups are Syrian and Iraqi, the distribution of which does not exactly correspond to the political boundaries of those two countries. Syrian-type dialects are also spoken in Lebanon, in three provinces of southern Turkey (Mersin, Adana,\footnote{The dialects spoken in Mersin and Adana provinces will henceforth referred to as Cilician Arabic.} Hatay), and in one village on Cyprus. In Iraq, Arabic is mainly spoken in Mesopotamia proper, whereas considerable parts of the mountainous parts of the country are Kurdish-speaking. Arabic dialects which are very akin to the Iraqi ones extend into northeastern Syria and southeastern Anatolia (for the latter see Akkuş in this volume). These two groups are geographically divided by a third dialect-group, which came here with an originally (semi-) nomadic population from northern Arabia. Today, this variety preponderates in all villages and most towns between the eastern outskirts of Aleppo and the left bank of the Tigris, and stretching north into the Turkish province of Şanlıurfa. 

The total number of native Arabic speakers in the whole region is estimated to be 54 million (see \tabref{tab:prochazka:1}). The dialects of large urban centers like Beirut, Damascus, Aleppo, Baghdad, and Mosul have become supra-regional prestige varieties that are also used in the media and therefore understood by most inhabitants of the respective countries. The situation is very different in Turkey, where the local Arabic is in sharp decline and public life is exclusively dominated by Turkish. Only recently has the position of Arabic in Turkey been socially enhanced by the influx of nearly 3.5 million Syrian refugees fleeing the civil war that started in 2011.\footnote{\textcolor{red}{The link you supplied --}  \url{http://multeciler.org.tr/turkiyedeki-suriyeli-sayisi/}  \textcolor{red}{-- no longer works. Can you supply a new one?}}

\begin{table}
\begin{tabularx}{.50\textwidth}{Xr}
\lsptoprule
\bfseries Country & \bfseries Speakers\\
\midrule
{Syria} &  17,000,000\\
{Lebanon} & 6,000,000\\
{Iraq} &  30,000,000\\
Turkey &  1,000,000\\
\lspbottomrule
\end{tabularx} 
\caption{Speaker populations for dialects of Arabic}
\label{tab:prochazka:1}
\end{table}

Arabic was spoken in the region long before the advent of Islam \citep[95]{Donner1981} but became the socially dominant language in the wake of the Muslim conquests in the seventh century CE. From that time until the end of the tenth century, when bedouin tribes seized large parts of central and northern Syria, there was probably a continuum of sedentary-type dialects that stretched from Mesopotamia to the northeastern Mediterranean \citep{Procházka2018}. During the Mongol sacking of Iraq in 1258, much of the population was killed or expelled, resulting in far-reaching demographic and linguistic changes as the original sedentary-type dialects could merely hold ground in Baghdad and the larger settlements to its north. Further south they persisted only among the non-Muslim population. Most of today’s Iraq was re-populated by people who spoke bedouin-type dialects (mostly coming from the Arabian Peninsula), which over the centuries have heavily influenced the speech of even most large cities \citep{Holes2007}. Very similar dialects are spoken further south and in the Iranian province of Khuzestan (see Leitner, this volume).\ia{Leitner, Bettina@Leitner, Bettina} The foundation of nation states after World War One caused a significant decrease in contact between the different dialect groups and an almost complete isolation of the Arabic dialects spoken in Turkey.

\section{Contact languages} 

During its two-and-a-half-millennia presence in the region, Arabic has come into contact with many languages, both Semitic and non-Semitic. Those most relevant for the topic will be treated in more detail below (for Syria, see also \citealt{Barbot1961}: 175--177). Akkadian was spoken in southern Iraq until about the turn of the eras, i.e. the first century CE.\footnote{For Akkadian lexical influence on Arabic, see \citet{Holes2002} and \citet{Krebernik2008}.} Greek was the language of administration in Greater Syria until the Arab conquest \citep[185--187]{Magidow2013} and continued to play a role for Orthodox Christians.\footnote{The enormous influence of Modern Greek on the Arabic spoken in the Kormakiti village of Cyprus is discussed by Walter (this volume). For a detailed study, see also \citealt{Borg1985}.\ia{Walter, Mary Ann@Walter, Mary Ann}} During Crusader times, Arabic speakers in Syria came into contact with various medieval European languages; and along the Mediterranean coast the so-called Lingua Franca (see Nolan, this volume)\ia{Nolan, Joanna@Nolan, Joanna} was an important source for the spread of particularly nautical vocabulary for many centuries (\citealt{KahaneKahaneTietze1958}). Since the nineteenth century, locally restricted contacts between Arabic and Armenian and Circassian have existed in parts of Syria and Lebanon. 

  \subsection{Aramaic} \label{aram}
Aramaic is a Northwest Semitic language and thus structurally very similar to Arabic. Different varieties of Aramaic were the main language in Syria and Iraq from the middle of the first millennium BCE and it can be assumed that some contact with Arabic existed even at that time. From the first century CE onwards, the southern fringes of the Fertile Crescent became largely Arabic-dominant and there was significant bilingualism with Aramaic, particularly in the towns along the edge of the steppe, such as Petra, Palmyra, Hatra, and al-Hira \citep{Procházka2018}. Though after the Muslim conquests Arabic eventually became the majority language, it did not oust Aramaic very quickly: the historical sources suggest that Aramaic dominated in the larger towns and the mountainous regions of Syria and Lebanon for a long time. In Iraq, by contrast, the massive influx of Arabs into the cities fostered their rapid arabization, while Aramaic continued to be spoken in the countryside (\citealt{Magidow2013}: 184; 188). But over the centuries, the diverse Aramaic dialects became marginalized and, with very few exceptions, were finally relegated to non-Muslim religious minorities, particularly Christians and Jews, in peripheral regions like Mount Lebanon and the Anti-Lebanon Mountains, where Aramaic was prevalent until the eighteenth century \citep{Retsö2011}. Western Aramaic is still spoken in three Syrian villages, the best known of which is Maaloula.\footnote{The village heavily suffered from the jihadist occupation of 2013--2014, but after government troops had retaken control over the region, many inhabitants returned and began its reconstruction (cf. the reports collected at \url{http://friendsofmaaloula.de/}).} There also remain speakers of Neo-Aramaic in northern Iraq.\footnote{See \citet{Coghill2015} and \url{http://glottolog.org/resource/languoid/id/nort3241}.} 

It is hard to establish the degree of bilingualism in the past, but it can be assumed that it was mostly Aramaic L1 speakers who had a command of Arabic and not vice versa. In the present time, nearly all remaining Aramaic speakers in Syria are fluent in Arabic. In Iraq this is mainly true of those living in the plain just north of Mosul (\citealt{ArnoldBehnstedt1993}; \citealt{Coghill2015}: 86). The influence of different strata of Aramaic on spoken Arabic is a long debated issue, various scholars rating it from considerable to negligible (\citealt{Hopkins1995}: 39; \citealt{Lentin2018}).

  \subsection{Persian and Kurdish} \label{persian} 
For many centuries, Arabic and the two Western Iranian languages Persian and Kurdish have influenced each other on different levels. Persian-speaking communities existed in medieval Iraq, and economic and cultural contacts between Mesopotamia and Iran have continued to the present (cf. \citealt{Gazsi2011}). An important factor of language contact are the holy shrines of the Imams in Kerbela, Najaf, and other Iraqi cities, which have always attracted tens of thousands of Persian speaking Shiites every year. Intensive contacts between speakers of Kurdish and Arabic have existed since at least the tenth century, particularly in Northern Iraq, northeast Syria, and southeast Anatolia (see Akkuş, this volume).\ia{Akkuş, Faruk@Akkuş, Faruk} Until their exodus in the early 1950s, the Arabic-speaking Jewish communities which existed in Iraqi Kurdistan usually had a native-like command of Kurdish \citep[12]{Jastrow1990}. Due to the multilingual character of the region, bilingualism in Kurdish and Arabic is still relatively widespread, particularly in urban settings, though with Kurds usually much more fluent in Arabic than the other way around.\footnote{With significant exceptions in some parts of southeast Anatolia; see Akkuş (this volume).\ia{Akkuş, Faruk@Akkuş, Faruk}} However, for obvious reasons, little linguistic research has been done in Iraq for decades, which makes it impossible to give up-to-date information about the linguistic situation in ethnically-mixed cities like Kirkuk.

 \subsection{Ottoman and Modern Turkish} 
Contacts between spoken Arabic varieties and various Turkic languages existed from the ninth century onwards. These early contacts, however, left hardly any traces in Arabic except for a handful of loanwords. In the sixteenth century, the Ottomans established their rule over most Arab lands, including Syria, Lebanon, and Iraq. This domination lasted four hundred years, until World War One. Particularly in the provinces of Aleppo and Mosul, there were a relatively high percentage of Turkish speakers and probably a significant degree of bilingualism.\footnote{See Wilkins (\citeyear{Wilkins2010}: xv) for Aleppo. Koury (\citeyear{Khoury1987}: 103) maintains that Aleppo’s hinterland was culturally even more Turkish than Arab. For Mosul, see Shields (\citeyear{Shields2004}: 54--55).} As the language of the ruling elite, Turkish had high prestige and therefore was at least rudimentarily spoken by many inhabitants of those regions, especially urban men. The collapse of the Ottoman Empire put an abrupt end to Turkish--Arabic contacts, which today remain intensive only among the Arabic varieties spoken within the borders of Turkey itself, where most Arabic speakers are fluent in Turkish, the dominant language in all contact settings.

In some areas of Syria and in northern Iraq, the Arabic-speaking population lives side by side with several hundred thousand speakers of Turkish and Azeri Turkish, who call themselves Turkmens. Unfortunately, no reliable data on the sociolinguistic settings and the degree of bilingualism exist for those areas. Again, it can be assumed that most of the Turkmens in both countries are dominant in Turkish, but know Arabic as a second language.

 \subsection{French and English} \label{french}

After World War One, Syria and Lebanon stayed under the French mandate and Iraq under the British mandate until they reached independence.\footnote{Iraq in 1932, Lebanon in 1943, Syria in 1946.} French is still widely spoken as a second language in Lebanon, especially by Christians. In Iraq, English has maintained its position as by far the most important foreign language – a fact which was reinforced by the US military occupation from 2003 to 2010. 

  \subsection{Intra-Arabic contacts} 

Contacts between different Arabic varieties, for instance between speakers of rural and urban dialects, happen on an everyday basis and often trigger short-term accommodation without leading to long-lasting changes. The situation is different with regard to the enduring contacts between the bedouin and the sedentary populations, whose dialects considerably differ from each other.\footnote{Since these two speech communities differ from each other in so many ways, it is a relatively robust approach to rate the following features as results of dialect contact and not mere variation (cf. \citealt{Lucas2015}: 533).} Such contacts are most intense at the periphery of the Syrian steppe and along the middle Euphrates, where scattered towns with sedentary dialects like Palmyra, Deir ez-Zor and Hit are surrounded by an originally nomadic population. Though the nomadic way of life has been abandoned by most of them, they still speak bedouin-type Arabic dialects. As the nomads were, for many centuries, socially and economically dominant, speakers of sedentary dialects often adopted linguistic features from the more prestigious bedouin (though reverse instances are also found; cf. (\citealt{Behnstedt1994Dialektkontakt}: 421). Due to the historical circumstances mentioned in §\ref{state}, bedouins also had a strong linguistic impact on Iraqi dialects. In Baghdad the sedentary dialect of the Muslim population has been gradually bedouinized due to massive migration from the countryside to the city \citep{Palva2009}. The Christian and, in former times, Jewish inhabitants preserved their original sedentary-type dialects because they had much less contact with the Muslim newcomers. 

\section{Contact-induced changes} 

Change induced by contact with Aramaic almost exclusively happened through imposition, that is, by Aramaic speakers who had learned Arabic as a second language and later often completely shifted to Arabic. This explains the relatively numerous phonological changes and pattern replications in syntax. Lexical transfers from Aramaic certainly were also made by Arabic-dominant speakers, particularly in semantic fields like agriculture that included novel concepts for the mostly animal-breeding Arabs.

The same is true for transfers from Greek, for which a very low level of bilingualism can be assumed. Thus we find only matter replication in the form of loanwords, mostly in domains where lexical gaps in older layers of spoken Arabic are likely. 

In the case of Kurdish, bilingualism is much more widespread among speakers of the source language, suggesting imposition. This might explain the phonological changes, as speakers dominant in the source language tend to preserve its phonological features \citep[532]{Lucas2015}. The relatively small number of instances of lexical matter replication is probably the result of the fact that Arabic has long been regarded as the more prestigious by speakers of both the source and the recipient language.

The numerous loanwords from Persian into Iraqi Arabic may well be the result of matter replication by agents who were dominant in the recipient language Arabic. Starting with the rule of the Abbasid caliphs in the eighth century CE and continuing to the present, Iranian material culture and cuisine often had a great impact on neighboring Mesopotamia. There were also many intellectuals, among them praised writers of Arabic prose, who were actually Iranians and hence knew both languages. Frequent contacts on the everyday level caused additional borrowing of ordinary vocabulary and the retention of sounds that are replaced in Persian loans found in Classical Arabic or other dialects.\footnote{The phonological changes \textcolor{red}{are not, however, only the result of Persian influence} (cf. §\ref{aram}).}\textsuperscript{} 

Changes induced by contact with Ottoman Turkish may have happened mostly through Arabic-dominant speakers. The current situation of Arabic speakers in Turkey is, however, very different, because at least the last two generations have acquired Turkish as an L2 or even as a second L1 at very young age. Thus, at least some of the contact phenomena described in the following paragraphs may be examples of linguistic convergence (see \citealt{Lucas2015}: 525). 

French and English have largely remained typical “foreign languages” learned at school or in business with a considerable amount of bilingualism only in some urban settings of Lebanon, particularly Beirut. The agents of change are certainly dominant in the recipient language. 

The distinction between the two transfer types is not always clearly discernible in case of intra-Arabic contact-induced changes. In the towns of the Syrian steppe and the middle Euphrates the agents of change were mostly the sedentary population who adapted their speech towards the norms of the socially more prestigious bedouin. However, there has always been inter-marriage, and bedouins often settled in towns and may well have adopted features from the local sedentary variety. Especially in cases like Muslim Baghdadi (see §\ref{state}) we may assume with good reason that the bedouin character of today’s variety developed through both imposition and borrowing. 

  \subsection{Phonology} 
 \subsubsection{Aramaic-induced changes} 

It has been hypothesized that several phonological features of the Syrian and Lebanese dialects are due to contact-induced influence of Aramaic. But in the case of the shift from interdental fricatives to postdental plosives (\textit{ð} > \textit{d}; \textit{θ} > \textit{t}; \textit{ð̣} > \textit{ḍ}) this is unlikely because: (i) this sound change is common crosslinguistically; (ii) it does not occur in all dialects of the region; and (iii) it is found in many other Arabic dialects without an Aramaic substrate. 

A phonotactic characteristic of most dialects spoken along the Mediterranean, from Cilicia in the north to Beirut in the south, is that all unstressed short vowels (including /a/) in open syllables are elided, whereas in other dialects east of Libya only /i/ and /u/ in this position are consistently dropped.\footnote{Therefore, Cantineau (\citeyear[108]{Cantineau1960}) called them \textit{parlers} \textit{non} \textit{différentiels} – a term still very often applied in Arabic dialectology – as they make no distinction in the treatment of the three short vowels.}

\ea
{Cilician Arabic (\citealt{Procházka2002Cukurova}: 31--32; 130)}\\
  \textup{Old Arabic (OA)} *raṣāṣ > \textit{rṣāṣ}  \textup{‘lead, plumb’\\
OA} *miknasa > \textit{mikinsi} ‘\textup{broom’}\footnote{With insertion of an epenthetic /i/ to avoid a sequence of three consonants.}\\
*fataḥ-t > \textit{ftaḥt} ‘\textup{I opened’}\\
  \z
  
Because this rule corresponds to the phonotactics of Aramaic and is otherwise not found in the same degree except in Maghrebi dialects (cf. Benkato, this volume),\ia{Benkato, Adam@Benkato, Adam} pattern replication is likely, though cannot be proved.\footnote{Cf. Diem (\citeyear[47]{Diem1979}); Arnold \& Behnstedt (\citeyear[69--71]{ArnoldBehnstedt1993}); Weninger (\citeyear[748]{Weninger2011Aramaic}).}

In roughly the same region, except Cilicia and many dialects of Hatay,\footnote{Where this phenomenon occurs only in Alawi villages \citep[84]{Arnold1998}.} the diphthongs /ay/ and /aw/ are only preserved in open syllables, but monophthongized to /ē/ and /ō/ respectively in closed syllables. In some regions, for instance on the island of Arwad, both diphthongs merge to /ā/ in closed syllables (\citealt{Behnstedt1997}: map 31). 

\ea\label{ex:prochazka:} 
{Arwad, western Syria \citep[278]{Procházka2013}} \\
    \textup{OA} *bayt, *baytayn > \textit{bāt, baytān} \textup{‘house, two houses’\\
OA} *yawm, *yawmayn >  \textit{yām, yawmān} \textup{‘day, two days’\\
OA} *bayn al-iθnayn > \textit{bān it-tnān} \textup{‘}\textup{between the two’}\\
\z

Likewise, in older layers of Aramaic, diphthongs were usually monophthongized in closed syllables (for Syriac see \citealt{Nöldeke1904}: 34), which makes imposition by L1 speakers of Aramaic rather likely \citep[227]{Fleisch1974Kfar}.

Another striking phenomenon is the split of historical /ā/ into /ō/ and /ē/ that is found in scattered areas of the Levant, particularly northern Lebanon, around the Syrian port of Tartous, the Qalamūn Mountains, and the exclusively Christian town of Maḥarde on the Orontes River.\footnote{For details cf. Behnstedt (\citeyear{Behnstedt1997}: map 32). The conditioned shift \textit{ā} > \textit{ō} is also found in and around Tarsus in Turkey \citep[37--38]{Procházka2002Cukurova}.} Because in many varieties of Aramaic the old Semitic /ā/ is reflected as /ō/, it could be assumed that Aramaic speakers transferred their peculiar pronunciation to Arabic when learning it. Fleisch (\citeyear[49]{Fleisch1974vowels}) rejected the hypothesis of an Aramaic influence, arguing that the conditioned distribution of the two allophones is merely a further development of the [ɒ] : [æ] split widely attested for Lebanon and parts of western Syria. However, in the Syrian Qalamūn Mountains there are dialects with an unconditioned \textcolor{red}{shift} \citep{Behnstedt1992}, and this is precisely the region where the shift from Aramaic to Arabic occurred relatively late, probably after a long phase of bilingualism. In the town of Nabk, for instance, one can infer that the former Aramaic speaking inhabitants would have simply turned every \textit{ā} into \textit{ō} – except those which long before had become \textit{ē} (or \textit{ɛ̄}) as a result of the so-called conditioned \textit{imāla} (i.e. the tendency of long \textit{ā} to be raised towards \textit{ē} or even \textit{ī} if the word contains an \textit{i/ī}).\footnote{Cf. Arnold \& Behnstedt (\citeyear[68]{ArnoldBehnstedt1993}).} Example (\ref{stud}) clearly shows that the distribution of the allophones is not conditioned by the consonantal environment.\footnote{\textcolor{red}{Can I just check that you are happy with every instance in this section where individual sounds are mostly written between phonemic /slashes/, but occasionally are written italicized outside slashes, like words or morphemes. Would you like to retain this distinction, or should we change all to slashes (or phonetic [square brackets] where appropriate)?}}

\ea\label{ex:prochazka:} \label{stud}
{Nabk, Syria \citep[20]{Gralla2006}} \\
    \textup{OA} *ṭābiḫ > \textit{ṭɛ̄beḫ}  \textup{‘cooking’ vs. OA} *ṭālib > \textit{ṭōleb}  \textup{‘student’}\\
\textup{OA} *ḥāmil > \textit{ḥɛ̄mel} \textup{‘pregnant’ vs. OA} *ḥāmiḍ > \textit{ḥōmeḍ’} \textup{‘sour’}\\
\z

In these cases Aramaic influence seems plausible. For the region of Tripoli it may be assumed that Aramaic bilinguals from the adjacent mountains used \textit{ō} instead of \textit{ā} when speaking Arabic and thus reinforced the already existing [ɒ] : [æ] split.\footnote{For discussion see Fleisch (\citeyear[48--50]{Fleisch1974vowels}; \citeyear[133--136]{Fleisch1974Kfar}), Diem (\citeyear[45--46]{Diem1979}); Behnstedt (\citeyear{Behnstedt1992}); Arnold \& Behnstedt (\citeyear[67--68]{ArnoldBehnstedt1993}); Weninger (\citeyear[748]{Weninger2011Aramaic}).}

  \subsubsection{The “new” phonemes /č/, /g/, and /p/} 

Consonantal phonemes that are originally alien to Arabic are found in all Arabic dialects spoken in Turkey, northern Syria, and Iraq. These are the unvoiced affricate /č/, the voiced /g/,\footnote{The sound \textit{g} is prevalent in whole Syria and Lebanon but seems to have phonemic status only in the north \citep[26]{Sabuni1980}. For further examples and discussion see \citet{Ferguson1969}. This “foreign” \textit{g} must therefore be differentiated from the \textit{g} which is the regular reflex of OA \textit{q}. The latter development is found in many bedouin-type dialects.} and the unvoiced /p/, the latter mainly used in Iraq. These sounds were very likely contact-induced, but it is often impossible to discern which language triggered each development: all three sounds are found in Persian, Kurdish, Turkish, and the Lingua Franca. For the dialects of Cilicia, Hatay and Syria, the main source language doubtless was Turkish. The sound /p/ in the Iraqi dialects was probably first introduced through contact with Persian and Kurdish, and then reinforced by Ottoman Turkish. In the bedouin-type dialects of the region, the phonemes /č/ and /g/ are not products of contact-induced change but occur due to internal sound changes, unvoiced /č/ as a conditioned affricated variant of /k/ and /g/ as the ordinary reflex of OA /q/. 

Thus, it can be assumed that over the centuries speakers of the sedentary dialects of Iraq and Syria borrowed either from other languages or from bedouin Arabic varieties words that possess these two sounds, which subsequently were fully incorporated into the phonemic inventory. This development may have been facilitated by the fact that the three sounds /č/, /p/, and /g/ are not fundamentally unfamiliar to Arabic, but are the voiceless/voiced counterparts of the well-established phonemes /ǧ/, /b/, and /k/. It seems no accident that the new sound /č/ is much more often found in dialects that have preserved the affricate /ǧ/ than in those where it has shifted to /ž/, \textcolor{red}{as illustrated in examples \REF{Aleppo} and \REF{Mosul}}.

\ea\label{Aleppo}
Aleppo\footnotemark

\textit{čanṭāye} ‘handbag’ (Turkish \textit{çanta})

\textit{čwāl} ‘sack’ (Turkish \textit{çuval})

\textit{čāy} ‘tea’ (Turkish \textit{çay})

\textit{gaǧaleg} ‘nightgown’ (Turkish \textit{gecelik})
\z

\footnotetext{\textcolor{red}{Please could we add a footnote here along the lines of ``Where sources are not specified for data, it derives from the author's personal knowledge", or otherwise add sources for all data, or state ``own data'' wherever relevant, as you have with example (13)? Please let us know your preference.}}

The words given in \REF{Aleppo} are usually pronounced with \textit{š} instead of \textit{č} in the central Syrian and Lebanese dialects where contact with Turkish was less intense and /ǧ/ is reflected as /ž/.\footnote{Cf. Behnstedt (\citeyear{Behnstedt1997}: maps 18, 19, 25). For details and more examples see Sabuni (\citeyear[205--210]{Sabuni1980}), who lists all words with \textit{č/g} in Aleppo, and Procházka (\citeyear[185]{Procházka2002Adana}) for Cilician Arabic.}

\ea\label{Mosul}
Mosul

\textit{ṣūč} ‘fault’ (Turkish \textit{suç})

\textit{pāča} ‘stew of sheep and cow legs and innards’ (Kurdish/Persian \textit{pāče})

\textit{zangīn} ‘rich’ (Turkish \textit{zengin})
\z

Once integrated into the phonological system, these sounds not only enabled easier integration of loanwords from other languages like French and English (see §\ref{french}), but sometimes also resulted in the spread of assimilation-induced allophones from single words to the whole paradigm or even root. In Aleppo one finds *yəkdeb > \textit{yəgdeb} ‘he lies’ due to assimilation. The \textit{g} subsequently was transferred to other words derived from the root: \textit{gadab} ‘he lied’, \textit{gədbe} ‘lie’, and \textit{gaddāb} ‘liar’ (\citealt{Sabuni1980}: 26, 209). 

Speakers of sedentary dialects who had everyday contact with bedouins – for example the inhabitants of Deir ez-Zor and Khatuniyya – first integrated /č/ and /g/ into their phonemic inventory through the borrowing of typically bedouin vocabulary such as \textit{dabča} ‘a bedouin dance’ (Khawetna) and \textit{ṭabga} ‘milk-bowl’ (Soukhne). These sounds then entered other fields of the lexicon, which led to unpredictable distribution, including doublets, as in \REF{Khawetna}--\REF{Baghdad}.

\ea\label{Khawetna} Khawetna \citep[28--31]{Talay1999}  

\textit{gəṣṣa} ‘forehead’, but \textit{qəṣṣa} ‘story’ (OA *quṣṣa \textit{/} *qiṣṣa)

\textit{dīč} ‘rooster’  (OA *dīk)
\z

\ea\label{Deir}Deir iz-Zor \citep[42--43]{Jastrow1978}. 

\textit{gāʕ} ‘soil’ (OA *qāʕ)

\textit{čam} ‘how much?’ (OA *kam)
\z

\ea\label{Baghdad}Baghdad \citep[18--19]{Palva2009}

\textit{guffa} ‘large basket’ (OA *quffa), but \textit{quful} ‘lock’ (OA *qufl)

\textit{ʕigab} ‘to pass’, but \textit{ʕiqab} ‘to follow’ (both OA *ʕaqab)
\z

The opposition /k/ : /č/ has even entered morphology, particularly with the 2\textsc{sg} suffixes: \textit{ʔabū-k} `your (\textsc{sg.m}) father' vs. \textit{ʔabū-č} ‘your (\textsc{sg.f}) father’. In the Syrian oasis of Soukhne, long-term contact with speakers of bedouin dialects caused a chain of phonetic changes: first /k/ shifted to /č/, which originally was the reflex of OA /ǧ/; then /č/ (< /ǧ/) shifted further to /ts/, which has become a unique feature of the local dialect. The unconditioned shift from /k/ > /č/, which is not found in the bedouin dialects, in turn caused a shift from /q/ > /k/.\footnote{See Behnstedt (\citeyear{Behnstedt1994Soukhne}: 4--11) for details.}

\ea
Soukhne (\citealt{Behnstedt1994Soukhne}: 226, 344, 357, 360)

\textit{kirbi} ‘water-skin’ (< OA *qirba, bedouin \textit{girba})

\textit{čalb} ‘dog’ (< OA *kalb, bedouin \textit{čalib})

\textit{čurr} ‘donkey foal’ (< OA *kurr, bedouin \textit{ku\R\R})

\textit{tsubn} ‘cheese’ (< OA *ǧubn, bedouin \textit{ǧubun})
\z

\subsection{Morphology} 
The Aramaic diminutive suffix -\textit{ūn} has become restrictedly productive in Iraqi Arabic \citep[72]{Masliyah1997}, as illustrated in \REF{una}. In Syria and Lebanon it is only found in fossilized forms such as \textit{šalfūn} ‘young cockerel’ and \textit{qafṣūne} ‘little cage’.  Such kinds of morphological transfer are usually triggered by lexical borrowing. Thus, it may be assumed that this suffix spread from loanwords like \textit{šalfūne} ‘small knife blade’ < Aramaic \textit{šelpūnā} ‘little knife’ (cf. \citealt{Féghali1918}: 82).\footnote{This must be a very old borrowing because the suffix is also found in the Gulf dialects (e.g. \textit{ḥabbūna} ‘a little’ \citealt[279]{Holes2002}) and even in Tunisian Arabic \citep[496]{Singer1984}, where direct Aramaic influence can be excluded.}

\ea\label{una}
Iraq

\textit{darb} ‘road’ > \textit{darbūna} ‘alley’

\textit{gṣayyir} ‘short’ \textit{<} \textit{gṣayyrūn} ‘very short’

\textit{mḥammdūn} hypocoristic form of the name \textit{Muḥammad}
\z

Syrian and Lebanese dialects exhibit a few word patterns that are attested for OA (and other dialects) but seem to have become widespread through contact with Aramaic due to their frequency in the latter. These are the verbal pattern šaC\textsubscript{1}C\textsubscript{2}aC\textsubscript{3} and the (primarily diminutive) nominal patterns C\textsubscript{1}aC\textsubscript{2}C\textsubscript{2}ūC\textsubscript{3} and C\textsubscript{1}aC\textsubscript{2}C\textsubscript{3}ūC\textsubscript{4}.\footnote{For the latter two see \citet{Corriente1969} and \citet{Procházka2004}.}\textsubscript{} 

An example of the first is \textit{šanfaḫ} ‘to puff up’, related to \textit{nafaḫ} ‘to blow up’ (\citealt[83]{Féghali1918}; cf. \citealt{Lentin2018} for further discussion); the nominal forms are illustrated in \REF{donkey} and \REF{fatima}.

\ea\label{donkey}
Aleppo (\citealt{Barthélemy1935}: 104, 158, 851) 

\textit{ǧaḥḥūš} ‘little donkey’ (related to \textit{ǧaḥš} `young donkey')

\textit{ḥassūn} ‘goldfinch’ (related to the personal name \textit{ḥasan})

\textit{namnūme} ‘small louse’ \textcolor{red}{(cp. \textit{naml} `ants')}
\z

The pattern C\textsubscript{1}aC\textsubscript{2}C\textsubscript{2}ūC\textsubscript{3}(i)\textsubscript{} is still productive in the whole region, including the bedouin dialects, to derive hypocoristic forms from personal names:

\ea\label{fatima}
fāṭma > faṭṭūma

ḥalīme > ḥallūma

aḥmad/mḥammad > ḥammūdi 
\z

In all Syrian and Lebanese dialects the \textsc{2pl} and \textsc{3pl} pronouns exhibit an /n/ in place of the /m/ that is found in other Arabic dialects, which makes them look as if they were reflexes of OA feminine forms (\tabref{tab:prochazka:2}). 

\begin{table}
\begin{tabularx}{\textwidth}{Xllll}
\lsptoprule
&    Damascus &   Jerusalem &    OA \textsc{pl.f} & Syriac \textsc{pl.m}\\
\midrule 
\textsc{2pl} & \textit{ʔəntu} \textit{/} \textit{-kon} & \textit{ʔintu}\textit{/}\textit{-kom} & \textit{ʔantunna} \textit{/} \textit{-kunna} & \textit{ʔatton} \textit {/} \textit{-kon}\\
\textsc{3pl} & \textit{hənne(n)} \textit{/} \textit{-hon} & \textit{humme} \textit{/} \textit{-hom} & \textit{hunna} \textit{/} \textit{-hunna} & \textit{hennon} \textit{/} \textit{-hon}\\
\lspbottomrule
\end{tabularx}

\caption{\textsc{2pl} and \textsc{3pl} pronouns}
\label{tab:prochazka:2}
\end{table}

Because generalization of the feminine is unlikely,\footnote{This is mainly because the feminine forms are only used for addressing groups of females, whereas the masculine forms may also refer to a mixed group. Therefore, the masculine forms are certainly more frequent. In all Arabic dialects except those mentioned above, the gender-neutral plural forms are clearly derived from the historical masculine.} these forms have often been explained as a contact-induced change. In Aramaic the corresponding pronouns also have \textit{n} (for Syriac see \citealt{Muraoka2005}: 18). In particular, the 3rd person forms with final \textit{-n} exactly mirror the Aramaic pattern, but lack a plausible intra-Arabic etymology. Thus imposition seems plausible. Nevertheless, substratum influence has been doubted, particularly because of the infrequent evidence of \textit{n-}pronouns in other regions.\footnote{See Owens (\citeyear[244--245]{Owens2006}) and Procházka (\citeyear{Procházka2018}) for details.}\textsuperscript{} 

In addition, the suffixes \textit{-o} (in the west of the region) and \textit{u} (in the east) can be attached to various kinship terms and given names when used for direct address, usually hypocoristically.\footnote{See also Ferguson (\citeyear[187]{Ferguson1997}).} 

\ea
Urfa (own data)

\textit{šnōnak} \textit{ḫayy-o?} ‘Brother, how are you?’

\textit{ǧidd-o} ‘Grandfather!’

\textit{ʕamm-o} ‘(paternal) Uncle!’

\textit{ḫāl-o} ‘(maternal) Uncle!’
\z

In Syria the suffix is also added to female nouns: \textit{ʕamm-t-o} ‘(paternal) Aunt!’ and \textit{ḫāl-t-o} ‘(maternal) Aunt!’, whereas in Iraq the corresponding forms end in \textit{{}-a}: \textit{ʕamm-a,} \textit{ḫāl-a}.

Since this suffix has no overt Arabic etymology\footnote{I warmly thank Jérôme Lentin for extensive discussion of this issue and invaluable help in finding important sources.} it has been assumed to be a borrowing of the Kurdish vocative \textit{-o} (e.g. \citealt{Grigore2007book}: 203). The Persian suffix \textit{-u} also forms affective diminutives,\footnote{E.g. \textit{pesar-u} ‘kid’; \textit{ʕamm-u} is even the common word for ‘uncle’ \citep[1011]{Perry2007}.} which would make Persian influence possible, at least for Iraq.\footnote{In the Iraqi dialects the vowel is \textit{-u}, e.g. \textit{ʕamm-u,} \textit{ḫāl-u} and \textit{ǧidd-u} (\citealt{Abu-Haidar1999}: 145).} However, the distribution of this feature extends far beyond even indirect contact with Kurdish or Persian,\footnote{The suffix is, for instance, attached to given names for endearment in the Gulf dialects, cf. Holes (\citeyear{Holes2016}: 128). The address forms \textit{ya} \textit{ʕamm-u,} \textit{ya} \textit{ḫāl-u} ‘uncle’, \textit{gidd-u} ‘grandfather’, \textit{sitt-u} ‘grandmother’ are used in Cairo, where hypocoristic variants of given names are likewise attested, e.g. \textit{mīšu} for \textit{hišām} \citep[109]{Woidich2006}. The suffix \textit{-o/-u} in address forms is also attested in eastern Sudan (Stefano Manfredi, personal communication), and in the Maghreb; Prunet \& Idrissi (\citeyear{PrunetIdrissi2014}: 184) provide a list of such nouns for Morocco.} though reinforcement and influence on the phonology may be possible for certain regions. Similar endings in Aramaic \citep[88--89]{Fassberg2010} and Ethiopian \citep[122]{Brockelmann1928} suggest a common Semitic origin (see also \citealt{Pat-El2017}: 463--465). 

All dialects of the region have incorporated the Turkish suffix \textit{-çi} [ʤi] into their nominal morphology, as illustrated in \REF{kahrab} and \REF{pancar}. This suffix has become productive and is therefore a good example of morphological matter borrowing (\citealt{GardaniArkadievAmiridze2015}). It is widely used for expressing professions, occupations, and habitual actions – the latter overwhelmingly pejorative, or at least humorous. In Iraqi dialects the suffix is reflected as \textit{{}-či}. In the other varieties, it follows the usual development of *ǧ, which means that it is pronounced \textit{-ǧi} or \textit{-ži.} 

\ea\label{kahrab}
Syria

\textit{kahrab-ži} ‘electrician’ (\textit{kahraba} ‘electricity’)

\textit{nəswān-ži} ‘womanizer’ (\textit{nəswān} ‘women’)

\textit{maškal-ži} ‘troublemaker’ (\textit{məšəkle} ‘problem’)
\z

\ea\label{pancar}
Iraq

\textit{pančar-či} ‘tire repairman’ (\textit{pančar} ‘puncture’)

\textit{mharrib-či} ‘human trafficker’ (\textit{mharrib} ‘one who helps s.o. to escape’)

\textit{ʕarag-či} ‘drunkard’ (\textit{ʕarag} ‘aniseed brandy’)
\z

The suffix clearly fills a morphological gap, because it enables morphologically transparent derivation even from loanwords, by preserving the basic, immediately recognizable word – in contrast to the Arabic C\textsubscript{1}aC\textsubscript{2}C\textsubscript{2}āC\textsubscript{3}{} pattern or participles, which are derived from the root (for details see \citealt{Procházka-Eisl2018}).

To a lesser extent other Turkish suffixes have enhanced the morphological devices of the dialects treated here,\footnote{See Halasi-Kun (\citeyear[68--71]{Halasi-Kun1969}); Sabuni (\citeyear[168]{Sabuni1980}); Masliyah (\citeyear{Masliyah1996}); Procházka (\citeyear[186]{Procházka2002Cukurova}).} specifically the relative suffix \textit{-li}, the privative suffix \textit{-siz}, and the abstract suffix \textit{-lik}, which is reflected as \textit{-loɣiyya} in Iraq, i.e. with the Arabic abstract morpheme affixed. For the most part these suffixes appear in Turkish loanwords, e.g. Cilicia \textit{ṣiḥḥat-li} (< Turkish \textit{sıhhatlı}) ‘healthy’, \textit{raḥaṭ-ṣīz} (< Turkish \textit{rahatsız}) ‘uncomfortable’. Only in Iraq have they gained a certain degree of productivity, particularly \textit{-sizz} and \textit{-loɣiyya}:

\ea
Iraq \citep[293--294]{Masliyah1996} 

\textit{muḫḫ-sizz} ‘stupid, brainless’

\textit{ḥaya-sizz} ‘shameless’

\textit{ḥaywān-loɣiyya} ‘ignorance’ (lit. ‘animal-ness’)

\textit{zmāl-loɣiyya} ‘stupidity’ (lit. ‘donkey-ness’)
\z

Arabic dialects spoken in Turkey not infrequently use light verb constructions (in Turkish grammar mostly called phrasal verbs) which consist of the verb ‘to do’ plus a following noun. Such compound verbs are very frequent in Turkish (and Kurdish) and enable easy integration of foreign vocabulary into the verbal system. The light verbs found in the Arabic dialects show that this formation is a case of selected pattern replication because, first, not all examples are exact copies of the Turkish model, and second, the word order follows the Arabic VO rather than the Turkish OV pattern:

\ea
Harran--Urfa (own data)

\textit{sāwa} \textit{qaza} (Turkish \textit{kaza} \textit{yapmak}) ‘to have an accident’

\textit{sāwa} \textit{ʕēš} (in Turkish not a phrasal verb, but \textit{pişirmek}) ‘to cook’
\z

\ea
Cilician Arabic \citep[198]{Procházka2002Cukurova} 

\textit{sawwa} \textit{zarar} (Turkish \textit{zarar} \textit{vermek}) ‘to harm’

\textit{sawwa} \textit{ḫayir} (Turkish \textit{hayır} \textit{işlemek}) ‘to do a good deed’
\z

\textcolor{red}{Concerning intra-Arabic contact, here we see that this has}\footnotetext{\textcolor{red}{The transition felt too confusingly abrupt without this.}} led to the adoption of typical bedouin-type pronouns into sedentary dialects (cf. \citealt{Palva2009}: 27--29), e.g.:

\ea
Baghdad, Deir ez-Zor, Soukhne 

\textit{ʔəḥna} for \textit{nəḥna} \textsc{1pl}
\z

\ea
Baghdad 

\textit{ʔāni} for \textit{ʔana} \textsc{1sg}
\z

In addition, as shown in \tabref{tab:prochazka:3}, virtually all the eastern sedentary dialects of Syria have copied the typical bedouin-type active participles of the verbs ‘to eat’ and ‘to take’, which exhibit initial \textit{m-} (\citealt{Behnstedt1997}: map 175).

\begin{table}
\begin{tabularx}{.8\textwidth}{lllX}
\lsptoprule
 Bedouin &   Soukhne &  Palmyra &  Damascus\\
 \midrule 
\textit{māčil / māḫið} & \textit{mīčil / mīḫið} & \textit{mākil / māḫið} & \textit{ʔākel / ʔāḫed}\\
\lspbottomrule
\end{tabularx}
\caption{Active participles of the verbs ‘to eat’ / ‘to take’}
\label{tab:prochazka:3}
\end{table}

Finally, in a few places intensive mutual contact has resulted in an interdialect \citep[62]{Trudgill1986} with completely new forms, such as the inflectional suffix \textit{-a} in the Syrian village of Ṣōrān (\citealt{Behnstedt1994Dialektkontakt}: 423--425), as shown in \tabref{tab:prochazka:4}.

\begin{table}
\begin{tabularx}{.8\textwidth}{XXl}
\lsptoprule
 Bedouin &  Sedentary &  Ṣōrān\\
\midrule 
\textit{gāḷ-am} & \textit{qāl-o} & \textit{qāl-a}\\
\lspbottomrule
\end{tabularx}
\caption{3\textsc{pl.m} inflectional suffixes}  
\label{tab:prochazka:4}
\end{table}

  \subsection{Syntax} 

In all but the bedouin-type dialects of the region, two constructions exist which both use an anticipatory pronoun and the preposition \textit{l-} ‘to’: (i) a construction involving analytical marking of a definite direct object, as in (\ref{damas})--(\ref{cilic}); and (ii) a construction involving analytic attribution of a noun, as in (\ref{christ}). The frequency and constraints of these two cases of clitic doubling show great variety, but in general the usage of construction (i) is restricted to specific objects, particularly elements denoting human beings, and construction (ii) is mostly found with inalienable possession, particularly kinship. A detailed discussion of both features is found in \citet{Souag2017clitic}.

\ea
{Damascus \citep[144]{Berlinches2016}} \\
\gll ḥabbēt-o la-ʕamər\\
     love.\textsc{prf.1sg}-\textsc{3sg.m} to-Amr\\
\glt ‘I loved Amr.’ \label{damas}
\z

\ea\label{ex:prochazka:} 
{Baghdad, Christian (\citealt{Abu-Haidar1991}: 116)}\\
\gll  qaɣētū-nu l-əl-əktēb\\
     read.\textsc{prf.1sg}{}-\textsc{3sg.m} to-\textsc{def}{}-book\\
\glt ‘I read the book.’
\z

\ea\label{ex:prochazka:} 
{Cilician Arabic (\textit{ʕalā} instead of \textit{l-}; \citealt[158]{Procházka2002Cukurova}}\\
\gll   biyḥibb-u ʕala ḫāl-u\\
     love.\textsc{impf.ind.3sg.m}{}-\textsc{3sg.m} on uncle-\textsc{3sg.m}\\
\glt ‘He loves his (maternal) uncle.’ \label{cilic}
\z

\ea\label{ex:prochazka:} 
{Baghdad, Christian (Abu-\citealt{Haidar1991}: 116)} \\
\gll maɣt-u l-aḫū-yi\\
     wife\textsc{-3sg.m} to-brother-\textsc{obl.1sg}\\
\glt ‘my brother’s wife’ \label{christ}
\z

Though the preposition \textit{l-} is sometimes attested in Classical Arabic for introducing direct objects and is common even in Modern Standard Arabic for analytic noun annexation, there are good arguments that the two constructions are pattern replications of an Aramaic model.\footnote{Not discussed here are two variants of construction (i), one without the suffix and the other without the preposition (cf. \citealt{Lentin2018}). Among the many studies that are in favor of Aramaic influence are Contini (\citeyear[105]{Contini1999}); Blanc (\citeyear[130]{Blanc1964}); and Weninger (\citeyear[750]{Weninger2011Aramaic}). Diem (\citeyear[47--49]{Diem1979}) and Lentin (\citeyear{Lentin2018}) are more skeptical. Souag (\citeyear[52]{Souag2017clitic}) suggests that at least “the initial stages of the development of clitic doubling in the Levant derive from Aramaic substratum influence, but the current situation also reflects subsequent Arabic-internal developments”.} For one thing, they do not have direct parallels either in OA or in dialects which lacked contact with Aramaic. Example (\ref{rubin}) shows that both constructions have striking parallels in especially the later eastern varieties of Aramaic \citep[94--104]{Rubin2005}. 

\ea\label{ex:prochazka:} \label{rubin}
\ea
{Syriac \citep[100]{Rubin2005}}\\
\gll bnā-y l-bayt-ā\\
     build.\textsc{prf.3sg.m}{}-3\textsc{sg.m} to-house-\textsc{def}\\
     \glt ‘He built the house.’
\ex
{ Syriac \citep[29]{Hopkins1997}\footnote{The same pattern using the linker \textit{d}{}- is more common.}}\\
\gll šm-ēh l-gabr-ā\\
     name-3\textsc{sg.m} to-man-\textsc{def}\\
\glt ‘the name of the man’
\z
\z

In the entire western part of the region including southern Turkey, the preposition \textit{fī} ‘in’, together with a pronominal suffix, is used to express a capability, as in \REF{cowell}. This has a striking parallel in the modern Aramaic \textit{ʔīθ} \textit{b-} ‘there is in' {\textasciitilde} `be able’ \citep[52]{Borg2004}.

\ea\label{ex:prochazka:} \label{cowell}
{Damascus \citep[415]{Cowell1964}} \\
\gll fī-ni sāʕd-ak əb-kamm lēra\\
     in-\textsc{1sg} help.\textsc{impf.1sg-2sg.m} with-some pound \\
\glt ‘Can I help you with a few pounds?’
\z

A final example of possible Aramaic influence is the Syrian particle \textit{šī} that mainly indicates partial specifity, as in \REF{shi}. It might be a pattern replication of the Western Neo-Aramaic form \textit{mett}, used with the same function \citep[49]{Diem1979}. What reduces the likelihood of imposition by Aramaic speakers is the existence of a cognate in Moroccan Arabic which is used with almost the same function.\footnote{Cf. Brustad (\citeyear{Brustad2000}: 19, 26--27); Wilmsen (\citeyear[51--53]{Wilmsen2014}).}

\ea\label{shi} 
{Damascus (own data)}\\
\gll hnīk fī šī ʕ\textcolor{red}{a}mūd\\
     there \textsc{exs} \textsc{indf} column \\
\glt ‘There is some column.’
\z

A hallmark of both sedentary and bedouin-type Iraqi dialects is that reflexes of the noun \textit{fard} ‘individual (thing or person)’ are used to mark different kinds of indefiniteness \citep[118--119]{Blanc1964}. The same form with the same indefinite article function is found in in the Iranian province of Khuzestan, and in all Arabic speaking language islands of Central Asia, i.e. Khorasan, Uzbekistan, and Afghanistan, as illustrated in \REF{fadd}.

\ea\label{fadd} 
{Kirkuk (own data)}\\
\gll taʕrif-lak fadd ṭabīb bāṭiniyye\\
     know.\textsc{impf.2sg.m-dat.2sg.m} \textsc{indf} doctor internal \\
\glt ‘Do you know a doctor of internal medicine?'
\z

It is very likely that the noun \textit{fard} has developed into a kind of indefinite article under the influence of other areal languages, particularly Turkish, Turkmen, Persian, and Neo-Aramaic. However, in contrast to all contact languages, Iraqi Arabic has not grammaticalized the numeral ‘one’ (\textit{wāḥəd}), but \textit{fard}. This clearly indicates that this feature is a case of pattern replication. There are many parallels in the functions of the indefinite articles (such as marking pragmatic salience, semantic individualization, approximation with numerals). Moreover, in all languages they are not fully systematized as a grammatical category as their usage is often optional. 

In the dialects of the Jews of Kurdistan the definite article is often omitted in subject position – a flagrant imitation of the Kurdish model.

\ea\label{ex:prochazka:} 
{Kurdistan Arabic \citep[71]{Jastrow1990}} \\
\gll baʕdēn mudīra baʕatət ḫalf-na\\
     then director send.\textsc{prf.3sg.f} after-\textsc{1pl}\\
\glt ‘Then the director \textcolor{red}{sent for us}.’
\z

An interesting case of calquing which shows the difficulty of distinguishing borrowing from imposition (see Manfredi, this volume) is the conjunction \textit{m-bōr} ‘because, in order to’. It exhibits both matter and pattern transfer, as it is a copy of Kurdish \textit{ji} \textit{ber} \textit{(ku).} In the actual form the Kurdish \textit{ji} ‘from’ was replaced by the Arabic equivalent \textit{m}{}- \citep[64]{Jastrow1979}. 

Syntactic change because of contact with Turkish is restricted to the Arabic dialects spoken in Turkey. In Cilicia and the Harran--Urfa, active participles express evidentiality, that is, they are used in utterances where a speaker refers to second-hand information. As evidentiality is not a common category in Semitic, it is very likely that the bilingual Arabic speakers of those regions copied this linguistic category from Turkish. In Turkish, any second-hand information is obligatorily marked by the verbal suffix \textit{-mış}, whose second function besides evidentiality is to express stativity and perfectivity. The latter two functions are assumed by the active participle in many Arabic dialects, including those in question here. Thus, we can suppose that the stative/perfective function, which is shared by both Arabic active participles and the Turkish suffix \textit{-mış}, was likely the starting point of the development that led to the additional evidential function of Arabic participles. The fact that evidentials seem to spread readily through language contact \citep[10]{Aikhenvald2004} makes Turkish influence even more probable.\footnote{For more examples and further details see Procházka (\citeyear[200--201]{Procházka2002Cukurova}) for Cilicia, and Procházka \& Batan (\citeyear[464--465]{ProcházkaBatan2016}) for the bedouin-type dialects in the Harran--Urfa region.} The example in (\ref{evid}) illustrates how the speaker uses perfect forms for those parts of the narrative he witnessed himself, and participles for secondhand information (perfect forms italic, participles in bold face). 

\ea\label{ex:prochazka:} \label{evid}
{Harran--Urfa (\citealt{ProcházkaBatan2016}: 465)}\\
  ʔiḥne b-zimānāt \textit{čān} ʕid-na ǧār b-al-maḥalle huwwa \textit{māt} \textit{ərtiḥam} əngūl-lu šēḫ mǝṭar […] nahā{\R} rabīʕ-u wāḥad  \textbf{ʕāzm}{}-u ʕala stanbūl \textbf{rāyiḥ} maʕzūm ʕala stanbul \textbf{māḫið} šēḫ mǝṭar əb-sāgt-u\\

\glt ‘Once we had a neighbor in our quarter. He died; he passed away. We called him Sheikh Mǝṭar. One day somebody invited his friend to Istanbul. As he was invited he went to Istanbul and he took Sheikh Mǝṭar with him.’
\z

In most Arabic dialects that are spoken in Turkey, comparatives and superlatives may be expressed by means of the Turkish particles \textit{daha} and \textit{en}, respectively, followed by the simplex instead of the elative form of the adjective. As for comparatives, the use of such constructions is rather restricted, while, at least in Cilician Arabic, they are relatively frequent for the superlative. 

\ea\label{ex:prochazka:} 
{Harran--Urfa (own data)}\\
\gll daha zēn ṣārat\\
     more good become.\textsc{prf.3sg.f}\\
\glt ‘It has become better.’
\z

\ea\label{ex:prochazka:} 
{Cilician Arabic \citep[155]{Procházka2002Cukurova}} \\
\gll mīn en zangīl bi-d-dini\\
     who \textsc{sup} rich in-\textsc{def}-world\\
\glt ‘Who is the richest (person) in the world?’ 
\z

In Cilicia, comparison is often expressed by the elative pattern of an adjective, which is preceded by the particle \textit{issa}. This clearly reflects a calque: the Turkish equivalent of the adverb \textit{issa} ‘still, yet’ is \textit{daha}, which in Turkish is also used as the particle of the comparative. 

\ea\label{ex:prochazka:} 
{Cilician Arabic \citep[202]{Procházka2002Cukurova}} \\
\gll ṣāyir issa aḥsan \\
     become.\textsc{ptcp} more good.\textsc{elat}\\
\glt ‘It became better’. 
\z

\ea\label{ex:prochazka:} 
{Turkish}\\
\gll Daha iyi ol-du.\\
     more good become-\textsc{prf.3sg}\\
\glt ‘It became better.’
\z

Sometimes a change in verb valency occurs as a consequence of the copying of Turkish models. A case found throughout these dialects is the verb \textit{ʕaǧab} ‘to like’: usually in Arabic the entity that is liked is the grammatical subject and the person who likes something is the direct object of the verb; but in the Arabic dialects in question, the construction of this verb reflects its Turkish (and English) usage with the person doing the liking being the grammatical subject.

\ea\label{ex:prochazka:} 
\ea Cilicia \citep[200]{Procházka2002Cukurova}\\
   \gll ʕǧabt bayt-ak\\
     like.\textsc{prf.1sg} house-\textsc{2sg.m}\\
\ex
 Damascus (own data)\\
   \gll bēt-ak ʕažab-ni\\
     house-\textsc{2sg.m} like.\textsc{prf.3sg.m-1sg}\\
\glt ‘I liked your house.’
\z
\z

  \subsection{Lexicon} 
Apart from the Aramaic loanwords also found in Classical Arabic (see \citealt{Retsö2011}; van Putten, this volume) – often in the realms of religion and cult – the dialects of this region exhibit a large number of Aramaic lexemes. They are particularly common in Lebanon and western Syria, but also found in Iraq and even in the bedouin-type dialects (\citealt{Féghali1918}; \citealt{Borg2004}, \citeyear{Borg2008}). A large percentage of these words belong to flora and fauna, agriculture, architecture, tools, kitchen utensils, and other material objects:\footnote{See also Neishtadt (\citeyear[282]{Neishtadt2015}).}

\ea
\textit{ṣumd} \textit{{\textasciitilde} ṣimd} ‘plough’ < Syriac \textit{ṣāmdē} ‘yoke’

\textit{qālūz} ‘bolt (of a door)’ < Syriac \textit{qālūzā}

\textit{nāṭūr} ‘guard (of a vineyard etc.)’ < Syriac \textit{nāṭūrā}

\textit{šaṭaḥ} ‘to spread’ < Syriac \textit{š\textsuperscript{e}}\textit{ṭaḥ}

\textit{šōb} ‘heat, hot’ < Syriac \textit{šawbā}
\z

Many nautical terms and words denoting agricultural products and tools were borrowed by Arabic from Greek, often via other languages, particularly Aramaic,\footnote{This is especially true for words related to Christian liturgy and ritual, which constitute about twenty per cent of the Greek vocabulary that entered the dialects of Syria.} the Lingua Franca, and Turkish: 

\ea
\textit{brāṣa} < Greek \textit{práson} ‘leek’ 

\textit{laḫana} < Greek \textit{láxana} ‘cabbage’

\textit{dərrāʔen} \textit{<} Greek \textit{dōrákinon} ‘peaches’

\textit{ʔabrīm/brīm} ‘keel’ < Greek \textit{prýmnē} ‘stern, poop’

\textit{sfīn} < Greek \textit{sphēn} ‘wedge’
\z

Kurdish borrowings are mainly restricted to northern Iraq, where bilingualism is widespread: 
\ea
Mosul

\textit{pūš} ‘chaff’ \textit{<} Kurdish \textit{pûş}

\textit{hēdi} \textit{hēdi} ‘slowly’ < Kurdish \textit{hêdî} \citep[68]{Jastrow1979}
\z

The intensive cultural and economic contacts between Iraq and Iran led to many Persian loanwords in various domains of the Iraqi dialects. 

\ea
\textit{mēwa} ‘fruit’ < Persian \textit{mīva} \textit{{\textasciitilde} mayva}

\textit{baḫat} ‘luck’ < Persian \textit{baḫt}

\textit{čariḫ} ‘wheel’ \textit{<} Persian \textit{čarḫ} 

\textit{gulguli} ‘pink’ \textit{<} Persian \textit{gol} ‘rose’

\textit{yawāš} ‘slow’ < Persian \textit{yavāš}

\textit{puḫta} ‘mush’ < Persian \textit{poḫte} ‘(well) cooked’
\z

Ottoman Turkish contributed a great deal to culinary vocabulary and the terminology of clothing and (technical) tools of Syria and Iraq.\footnote{The same loanwords are, of course, often found in other regions that were under Ottoman rule, above all in Egypt, but also in Tunisia, Yemen and other regions.} It was even the source of several adverbs and even verbs in the local Arabic varieties (\citealt{Halasi-Kun1969}, \citeyear{Halasi-Kun1973}, \citeyear{Halasi-Kun1982}).

\ea
Syria (Damascus)

\textit{šāwərma} ‘shawarma’ < Turkish \textit{çevirme} 

\textit{ṣāž} ‘iron plate for making bread’ \textit{<} Turkish \textit{saç}

\textit{yalanži} ‘vine-leaves stuffed with rice’ < Turkish \textit{yalancı} ‘liar’ (as they pretend to be “real” \textit{dolma} stuffed with meat)

\textit{šīš} \textit{ṭāwūʔ} ‘spit-roasted chicken’ < Turkish \textit{şiş} \textit{tavuk}

\textit{kǝzlok} ‘glasses’ \textit{<} Turkish \textit{gözlük}

\textit{ʔ\textcolor{red}{ō}ḍa} ‘room’ < Turkish \textit{oda}

\textit{ballaš} ‘to begin’ < Turkish \textit{başla-mak} by metathesis.
\z

\ea
Iraq (Muslim Baghdadi, cf. \citep{Reinkowski1995} 

\textit{qūzi} ‘a dish with roasted mutton’ < Turkish \textit{kuzu} ‘lamb’

\textit{tēl} ‘wire’ < Turkish \textit{tel}

\textit{yašmāɣ} ‘kerchief (for men)’ < Turkish \textit{yaşmak} ‘veil (for women)’

\textit{bōš} ‘empty; neutral’, which yielded also the verb \textit{bawwaš} ‘to put into neutral (gear)’ < Turkish \textit{boş} ‘empty’

\textit{qačaɣ} ‘smuggled goods’ < Turkish \textit{kaçak}
\z

During the last century, the Arabic dialects in Turkey\footnote{For Cilicia see Procházka (\citeyear{Procházka2002Cukurova}, \citeyear[187--199]{Procházka2002Adana}).} have incorporated numerous Turkish words in addition to loanwords from Ottoman times. Among them are terms in education, medicine, sports, media, and technology. Besides these, kinship terms, the vocabulary of everyday life, and structural words like adverbs and discourse markers have infiltrated the dialects from Turkish. 

\ea
Cilician Arabic

\textit{qāyin} \textit{…} ‘-in-law’ (< Turkish \textit{kayın})

\textit{ṭōrūn} ‘grandchild’ (< Turkish \textit{torun})

\textit{bīle} ‘even’ (< Turkish \textit{bile})

\textit{qāršīt} ‘opposite from’ (< Turkish \textit{karşı})
\z

The cases of semantic extension of an Arabic word result from the wider semantic range of its Turkish equivalent which has been transferred into Arabic. Thus, in both Cilician and Harran--Urfa Arabic \textit{sāq/ysūq} ‘to drive’ also occurs with the meaning of ‘to last’ like the Turkish verb \textit{sürmek}. In Harran--Urfa \textit{b-arð̣} ‘on the place/ground (of)’ has become a preposition/conjunction meaning ‘instead’. This can be seen as an instance of contact-induced grammaticalization (\citealt{GardaniArkadievAmiridze2015}: 4) under the influence of Turkish \textit{yerine} ‘instead, in its place’.

\ea
{Harran--Urfa (own data)}\\
 \gll   al-mille tākl-u b-arð̣ al-laḥam\\
     \textsc{def}{}-people eat.\textsc{impf.3sg.f-3sg.m}  in-place \textsc{def}{}-meat\\
\glt  ‘The people eat it instead of meat.’
\z

\ea\label{ex:prochazka:} 
{Harran--Urfa (own data)}\\
\gll    b-arð̣-in tibči ʔigir āya\\
     in-place-\textsc{link} cry.\textsc{impf.2sg.m} read.\textsc{imp.sg.m} verse\\
\glt ‘Instead of crying recite a (Koranic) verse!’
\z

In Iraq, many English words related to Western culture and technology have been, and still are, borrowed into the dialects. The same is true for French in Syria and (particularly) Lebanon (cf. \citealt{Barbot1961}: 176).

\ea
Iraq (words of English origin)

\textit{kitli} < kettle 

\textit{buṭil} < bottle

\textit{glāṣ} < glass

\textit{pančar} ‘flat tire’ (< \textit{puncture})

\textit{pāysikil} < bicycle

\textit{māṭōrsikil} < motorcycle

\textit{lōri} < lorry 

\textit{igzōz} < exhaust (pipe)

\textit{brēk} < brake
\z

\ea
Syria and Lebanon (words of French origin)

\textit{gātto} \textit{{\textasciitilde} gaṭō < gâteau} ‘cake’

\textit{garsōn} \textit{<} \textit{garçon} ‘waiter’

\textit{sēšwār} \textit{<} \textit{séchoir} ‘hair drier’

\textit{kwaffēr} \textit{<} \textit{coiffeur} ‘hair-dresser’

\textit{ʔaṣanṣēr} \textit{<} \textit{ascenseur} ‘elevator’

\textit{grīb} < \textit{grippe} ‘influenza’
\z

Due to long-term contacts, there are mutual borrowings between the bedouin and sedentary dialects of the region. This affects not only specific vocabulary of respective culture but also basic lexical items. Historically, the sedentary dialects have been much more influenced by the bedouin-type dialects than vice versa. 

\section{Conclusion} 

The sociolinguistic history of the regions treated here suggests that the conditions for imposition were relatively restrictive and mainly found in contact settings with Aramaic which, over the centuries, has been given up by most of its speakers in favor of Arabic. Thus, it is not surprising that so many features beyond the lexicon for which contact-induced change can be assumed are related to Aramaic influence. 

Morphological borrowing is in general relatively rare because it presupposes a high intensity of contact (\citealt{GardaniArkadievAmiridze2015}: 1). Practically all cases presented in §\ref{persian} corroborate the universal tendencies that: (i) derivational morphology is more prone to borrowing than inflectional morphology; and (ii) nominalizers and diminutives are very frequently represented in instances of borrowed derivational morphology (\citealt{GardaniArkadievAmiridze2015}: 7; \citealt{Seifart2013}). On the whole, the bedouin-type dialects exhibit significantly fewer contact-induced changes than the sedentary dialects. This may be the result of both the \textcolor{red}{production systems}\footnote{\textcolor{red}{I don't understand this. Can you rephrase?}} of bedouin groups and their tribally organized society, which impedes intense contact with outsiders.

The relative infrequency of contact-induced changes in morphology and syntax found in the Arabic varieties spoken in Turkey have two main reasons: first, the high degree of complete bilingualism is a very recent phenomenon that only pertains to the last two generations; and second, and probably more importantly, the great structural differences between the two languages, which have impeded both matter and pattern replications.

What is still relatively unclear is the degree of historical bilingualism between Arabic on the one hand and Ottoman Turkish, Kurdish, and Persian on the other. Future research would be particularly desirable with regard to Iraq, providing interesting new data on contact-induced changes in multilingual regions like Mosul and Kirkuk where Arabic, Turkmen, and Kurdish speakers have been in contact for a long time. Also, studies like that of  \citet{Neishtadt2015} for Palestine should be carried out for Syrian and especially Iraqi dialects with regard to lexical borrowings from Aramaic. Another completely under-researched topic is idiomatic constructions, in which the mutual influence of most languages in the region may be assumed.

\section*{Further reading}

There are no studies which treat the subject of contacts between Arabic and the other languages of the whole region covered in this chapter. However:\\
\citet{ArnoldBehnstedt1993} is an in-depth study of the mutual contacts between Western Neo-Aramaic and the local Arabic dialects in the Anti-Lebanon Mountains of Syria.\\
\citet{Diem1979} is a pioneer study of substrate influence in the modern Arabic dialects, though with focus on South Arabia, i.e. outside of the region treated in this chapter.\\
\citet{Palva2009} is a very good case study of the diachronic relations between sedentary and bedouin-type dialects in the Iraqi capital Baghdad.\\
\citet{Weninger2011Aramaic} is a concise overview of contact between different varieties of Aramaic and Arabic.

\section*{Abbreviations}

\begin{tabularx}{.45\textwidth}{lQ}
\textsc{1, 2, 3} & 1st, 2nd, 3rd person \\
BCE & before Common Era \\
CE & Common Era \\
\textsc{comp} & complementizer \\
\textsc{def} & definite \\
\textsc{f} & feminine \\
\textsc{elat} & elative \\
\textsc{exs} & existential \\
\textsc{imp} & imperative \\
\textsc{impf} & imperfect \\
\textsc{indf} & indefinite \\
\end{tabularx}
\begin{tabularx}{.45\textwidth}{lQ}
L1 & first language \\
L2 & second language \\
\textsc{link} & linker \\
\textsc{m} & masculine \\
OA & Old Arabic \\
\textsc{obl} & oblique \\
\textsc{pl} & plural \\
\textsc{prf} & perfect \\
\textsc{sg} & singular \\
\textsc{sup} & superlative \\
\end{tabularx}


\section*{Acknowledgements}
\textcolor{red}{Do you have any you would like to add?}

\sloppy
\printbibliography[heading=subbibliography,notkeyword=this] 
\end{document}