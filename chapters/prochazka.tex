\documentclass[output=paper]{langsci/langscibook} 
\ChapterDOI{10.5281/zenodo.3744507}
\author{Stephan Procházka\affiliation{University of Vienna}}
\title{Arabic in Iraq, Syria, and southern Turkey}
\abstract{This chapter covers the Arabic dialects spoken in the region stretching from the Turkish province of Mersin in the west to Iraq in the east, including Lebanon and Syria. The area is characterized by a high degree of linguistic diversity, and for about two and a half millennia Arabic has come into contact with various other Semitic languages, as well as with Indo-European languages and Turkish. Bilingualism, particularly with Aramaic, Kurdish, and Turkish, has resulted in numerous contact-induced changes in all realms of grammar, including morphology and syntax.}
\maketitle

\begin{document}

\section{Current state and historical development}  \label{state} 

The region discussed in this chapter is linguistically extremely heterogeneous: in it three different \ili{Arabic} dialect groups, plus several other languages, are spoken. The two main \ili{Arabic} dialect groups are \ili{Syrian} and \ili{Iraqi}, the distribution of which does not exactly correspond to the political boundaries of those two countries. \ili{Syrian}-type dialects are also spoken in Lebanon, in three provinces of southern Turkey (Mersin, Adana,\footnote{The dialects spoken in Mersin and Adana provinces will henceforth referred to as \ili{Cilician} \ili{Arabic}.} Hatay), and in one village on {Cyprus} (cf. Walter, this volume).\ia{Walter, Mary Ann@Walter, Mary Ann} In Iraq, \ili{Arabic} is mainly spoken in Mesopotamia proper, whereas considerable parts of the mountainous parts of the country are \ili{Kurdish}-speaking. \ili{Arabic} dialects which are very akin to the \ili{Iraqi} ones extend into northeastern Syria and southeastern {Anatolia} (for the latter see Akkuş, this volume).\ia{Akkuş, Faruk@Akkuş, Faruk} These two groups are geographically divided by a third dialect group, which arrived in the region with an originally (semi-) nomadic population from northern Arabia. Today, this variety preponderates in all villages and most towns between the eastern outskirts of Aleppo and the right bank of the Tigris, and stretching north into the Turkish province of Şanlıurfa. 

The total number of native \ili{Arabic} speakers in the whole region is estimated to be 54 million (see \tabref{tab:prochazka:1}). The dialects of large urban centers like \ili{Beirut}, \ili{Damascus}, \ili{Aleppo}, and \ili{Baghdad} have become supra-regional {prestige} varieties that are also used in the media and therefore understood by most inhabitants of the respective countries. The situation is very different in Turkey, where the local \ili{Arabic} is in sharp decline and public life is exclusively dominated by \ili{Turkish}. Only recently has the position of \ili{Arabic} in Turkey been socially enhanced by the influx of more than 3.5 million Syrian refugees fleeing the civil war that started in 2011.\footnote{See UNHCR figures at \url{https://data2.unhcr.org/en/situations/syria/location/113}.}

\begin{table}
\begin{tabularx}{.50\textwidth}{Xr}
\lsptoprule
Country & Speakers\\
\midrule
{Syria} &  17,000,000\\
{Lebanon} & 6,000,000\\
{Iraq} &  30,000,000\\
Turkey &  1,000,000\\
\lspbottomrule
\end{tabularx} 
\caption{Speaker populations for dialects of Arabic}
\label{tab:prochazka:1}
\end{table}

\ili{Arabic} was spoken in the region long before the advent of Islam \citep[95]{Donner1981}, but became the socially dominant language in the wake of the Muslim conquests in the seventh century CE. From that time until the end of the tenth century, when Bedouin tribes seized large parts of central and northern Syria, there was probably a continuum of sedentary-type dialects that stretched from Mesopotamia to the northeastern Mediterranean \citep[291]{Procházka2018Fertile}. During the Mongol sacking of Iraq in 1258, much of the population was killed or expelled. This resulted in far-reaching demographic and linguistic changes as the original sedentary-type dialects were only able to hold ground in \ili{Baghdad} and the larger settlements to its north. Further south they persisted only among the non-Muslim population. Most of today’s Iraq was re-populated by people who spoke \ili{Bedouin}-type dialects (mostly coming from the Arabian Peninsula), which over the centuries have heavily influenced the speech of even most large cities \citep{Holes2007}. Very similar dialects are spoken further south and in the Iranian province of \ili{Khuzestan} (see Leitner, this volume).\ia{Leitner, Bettina@Leitner, Bettina} The foundation of nation states after World War One caused a significant decrease in contact between the different dialect groups and an almost complete isolation of the \ili{Arabic} dialects spoken in Turkey.

\section{Contact languages} 

During its two-and-a-half-millennia presence in the region, \ili{Arabic} has come into contact with many languages, both \ili{Semitic} and non-\ili{Semitic}. Those most relevant for the topic will be treated in more detail below (for Syria, see also \citealt{Barbot1961}: 175--177). \ili{Akkadian} was spoken in southern Iraq until about the turn of the eras, i.e. the first century CE.\footnote{For \ili{Akkadian} lexical influence on \ili{Arabic}, see \citet{Holes2002} and \citet{Krebernik2008}.} \ili{Greek} was the language of administration in Greater Syria until the Arab conquest \citep[185--187]{Magidow2013} and continued to play a role for Orthodox {Christians}.\footnote{The enormous influence of Modern \ili{Greek} on the \ili{Arabic} spoken in the {Kormakiti} village of {Cyprus} is discussed by Walter (this volume). For a detailed study, see also \citet{Borg1985}.\ia{Walter, Mary Ann@Walter, Mary Ann}} During Crusader times, \ili{Arabic} speakers in Syria came into contact with various medieval European languages; and along the Mediterranean coast the so-called \ili{Lingua Franca} (see Nolan, this volume)\ia{Nolan, Joanna@Nolan, Joanna} was an important source for the spread of particularly nautical vocabulary for many centuries (\citealt{KahaneKahaneTietze1958}). Since the nineteenth century, locally restricted contacts between \ili{Arabic} and \ili{Armenian} and \ili{Circassian} have existed in parts of Syria and Lebanon. 

  \subsection{Aramaic} \label{aram}
\ili{Aramaic} is a \ili{Northwest Semitic} language and thus structurally very similar to \ili{Arabic}. Different varieties of \ili{Aramaic} were the main language in Syria and Iraq from the middle of the first millennium BCE and it can be assumed that some contact with \ili{Arabic} existed even at that time. From the first century CE onwards, the southern fringes of the Fertile Crescent became largely \ili{Arabic}-dominant and there was significant {bilingualism} with \ili{Aramaic}, particularly in the towns along the edge of the steppe, such as Petra, Palmyra, Hatra, and al-Ḥīra \citep[260--262]{Procházka2018Fertile}. Though after the Muslim conquests \ili{Arabic} eventually became the majority language, it did not oust \ili{Aramaic} very quickly: the historical sources suggest that \ili{Aramaic} dominated in the larger towns and the mountainous regions of Syria and Lebanon for a long time. In Iraq, by contrast, the massive influx of Arabs into the cities fostered their rapid {Arabization}, while \ili{Aramaic} continued to be spoken in the countryside (\citealt{Magidow2013}: 184; 188). But over the centuries, the diverse \ili{Aramaic} dialects became marginalized and, with very few exceptions, were finally relegated to non-Muslim religious minorities, particularly {Christians} and {Jews}, in peripheral regions like Mount Lebanon and the Anti-Lebanon Mountains, where \ili{Aramaic} was prevalent until the eighteenth century \citep{Retsö2011}. \ili{Western Aramaic} is still spoken in three Syrian villages, the best known of which is Maaloula.\footnote{The village heavily suffered from the jihadist occupation of 2013--2014, but after government troops had retaken control over the region, many inhabitants returned and began its reconstruction (cf. the reports collected at \url{http://friendsofmaaloula.de/}).} There also remain speakers of \ili{Neo-Aramaic} in northern Iraq.\footnote{See \citet{Coghill2012} and \url{http://glottolog.org/resource/languoid/id/nort3241}.} 

It is hard to establish the degree of {bilingualism} in the past, but it can be assumed that it was mostly \ili{Aramaic} L1 speakers who had a command of \ili{Arabic} and not vice versa. In the present time, nearly all remaining \ili{Aramaic} speakers in Syria are fluent in \ili{Arabic}. In Iraq this is mainly true of those living in the plain just north of \ili{Mosul} (\citealt{ArnoldBehnstedt1993}; \citealt{Coghill2012}: 86). The influence of different strata of \ili{Aramaic} on spoken \ili{Arabic} is a long debated issue, various scholars rating it from considerable to negligible (\citealt{Hopkins1995}: 39; \citealt{Lentin2018}).

  \subsection{Persian and Kurdish} \label{persian} 
For many centuries, \ili{Arabic} and the two Western \ili{Iranian} languages \ili{Persian} and \ili{Kurdish} have influenced each other on different levels. \ili{Persian}-speaking communities existed in medieval Iraq, and economic and cultural contacts between Mesopotamia and Iran have continued to the present (cf. \citealt{Gazsi2011}). An important factor of language contact are the holy shrines of the Imams in Kerbela, Najaf, and other Iraqi cities, which have always attracted tens of thousands of \ili{Persian}-speaking Shiites every year. Intensive contacts between speakers of \ili{Kurdish} and \ili{Arabic} have existed since at least the tenth century, particularly in {Northern} Iraq, northeast Syria, and southeast {Anatolia} (see Akkuş, this volume).\ia{Akkuş, Faruk@Akkuş, Faruk} Until their exodus in the early 1950s, the \ili{Arabic}-speaking Jewish communities which existed in Iraqi Kurdistan usually had a native-like command of \ili{Kurdish} \citep[12]{Jastrow1990chapter}. Due to the multilingual character of the region, {bilingualism} in \ili{Kurdish} and \ili{Arabic} is still relatively widespread, particularly in urban settings, though with {Kurds} usually much more fluent in \ili{Arabic} than the other way around.\footnote{With significant exceptions in some parts of southeast {Anatolia}; see Akkuş (this volume).\ia{Akkuş, Faruk@Akkuş, Faruk}} However, for obvious reasons, little linguistic research has been done in Iraq for decades, which makes it impossible to give up-to-date information about the linguistic situation in ethnically-mixed cities like Kirkuk.

 \subsection{Ottoman and Modern Turkish} 
Contacts between spoken \ili{Arabic} varieties and various \ili{Turkic} languages existed from the ninth century onwards. These early contacts, however, left hardly any traces in \ili{Arabic} except for a handful of {loanwords}. In the sixteenth century, the Ottomans established their rule over most Arab lands, including Syria, Lebanon, and Iraq. This domination lasted four hundred years, until World War One. Particularly in the provinces of Aleppo and {Mosul}, there was a relatively high percentage of \ili{Turkish} speakers and probably a significant degree of {bilingualism}.\footnote{See Wilkins (\citeyear{Wilkins2010}: xv) for Aleppo. Koury (\citeyear{Khoury1987}: 103) maintains that Aleppo’s hinterland was culturally even more Turkish than Arab. For \ili{Mosul}, see Shields (\citeyear{Shields2004}: 54--55).} As the language of the ruling elite, \ili{Turkish} had high {prestige} and therefore was at least rudimentarily spoken by many inhabitants of those regions, especially urban men. The collapse of the Ottoman Empire put an abrupt end to \ili{Turkish}--\ili{Arabic} contacts, which today remain {intensive} only among the \ili{Arabic} varieties spoken within the borders of Turkey itself, where most \ili{Arabic} speakers are fluent in \ili{Turkish}, the dominant language in all contact settings.

In some areas of Syria and in northern Iraq, the \ili{Arabic}-speaking population lives side by side with several hundred thousand speakers of \ili{Turkish} and \ili{Azeri} \ili{Turkish}, who call themselves Turkmens. Unfortunately, no reliable data on the sociolinguistic settings and the degree of {bilingualism} exist for those areas. Again, it can be assumed that most of the Turkmens in both countries are dominant in \ili{Turkish}, but use \ili{Arabic} as a second language.

 \subsection{French and English} \label{french}

After World War One, Syria and Lebanon stayed under the French mandate and Iraq under the British mandate until they reached independence.\footnote{Iraq in 1932, Lebanon in 1943, Syria in 1946.} \ili{French} is still widely spoken as a second language in Lebanon, especially by {Christians}. In Iraq, \ili{English} has maintained its position as by far the most important foreign language – a fact which was reinforced by the US military occupation from 2003 to 2010. 

  \subsection{Intra-Arabic contacts} 

Contacts between different \ili{Arabic} varieties, for instance between speakers of rural and urban dialects, happen on an everyday basis and often trigger short-term {accommodation} without leading to long-lasting changes. The situation is different with regard to the enduring contacts between the \ili{Bedouin} and the sedentary populations, whose dialects differ from each other considerably.\footnote{Since these two speech communities differ from each other in so many ways, it is a relatively robust approach to rate the following features as results of {dialect contact} and not mere variation (cf. \citealt{Lucas2015}: 533).} Such contacts are most intense at the periphery of the Syrian steppe and along the middle Euphrates, where scattered towns with sedentary dialects like Palmyra, Deir ez-Zor and Hit are surrounded by an originally nomadic population. Though the nomadic way of life has been abandoned by most of them, they still speak \ili{Bedouin}-type \ili{Arabic} dialects. As the nomads were, for many centuries, both socially and economically dominant, speakers of sedentary dialects often adopted linguistic features from more {prestigious} \ili{Bedouin} (though reverse instances are also found; cf. \citealt{Behnstedt1994Dialektkontakt}: 421). Due to the historical circumstances mentioned in §\ref{state}, Bedouins also had a strong linguistic impact on \ili{Iraqi} dialects. In \ili{Baghdad} the sedentary dialect of the Muslim population has been gradually Bedouinized due to massive migration from the countryside to the city \citep{Palva2009}. The Christian and, in former times, Jewish inhabitants, on the other hand, preserved their original sedentary-type dialects because they had much less contact with the Muslim newcomers. 

\section{Contact-induced changes}\largerpage

Change induced by contact with \ili{Aramaic} almost exclusively happened through {imposition}, that is, by \ili{Aramaic} speakers who had learned \ili{Arabic} as a second language and later often completely shifted to \ili{Arabic}. This explains the relatively numerous phonological changes and pattern replications in syntax. Lexical transfers from \ili{Aramaic} certainly were also made by \ili{Arabic}-dominant speakers, particularly in semantic fields like agriculture that included novel concepts for the mostly animal-breeding Arabs.

The same is true for transfers from \ili{Greek}, for which a very low level of {bilingualism} can be assumed. Thus we find only matter {replication} (in the sense of \citealt{Sakel2007}) in the form of {loanwords}, mostly in domains where lexical gaps in older layers of spoken \ili{Arabic} are likely. 

In the case of transfer from \ili{Kurdish}, {bilingualism} is much more widespread among speakers of the {source language}, suggesting {imposition}. This might explain some of the phonological changes discussed in §\ref{newphonemes}, as speakers dominant in the {source language} tend to preserve its phonological features \citep[532]{Lucas2015}. The relatively small number of instances of lexical matter {replication} is probably the result of the fact that \ili{Arabic} has long been regarded as the more {prestigious} by speakers of both the source and the {recipient language}.

The numerous {loanwords} from \ili{Persian} into \ili{Iraqi} \ili{Arabic} may well be the result of matter {replication} by agents who were dominant in the {recipient language} \ili{Arabic}. Starting with the rule of the Abbasid caliphs in the eighth century CE and continuing to the present, Iranian material culture and cuisine often had a great impact on neighbouring Mesopotamia. There were also many intellectuals, among them praised writers of \ili{Arabic} prose, who were actually Iranians and hence knew both languages. Frequent contacts on the everyday level caused additional borrowing of ordinary vocabulary and the retention of sounds that are replaced in \ili{Persian} loans found in \ili{Classical Arabic} or other dialects.\footnote{The phonological changes are not, however, only the result of \ili{Persian} influence (cf. §\ref{aram}).} 

Changes induced by contact with \ili{Ottoman} {Turkish} may have happened most\-ly through \ili{Arabic}-dominant speakers. The current situation of \ili{Arabic} speakers in Turkey is, however, very different, because at least the last two generations have acquired \ili{Turkish} as an L2 or even as a second L1 at very young age. Thus, at least some of the contact phenomena described in the following paragraphs may be examples of linguistic {convergence} (see \citealt{Lucas2015}: 525). 

\ili{French} and \ili{English} have largely remained typical “foreign languages” learned at school or in business with a considerable amount of {bilingualism} only in some urban settings of Lebanon, particularly Beirut. The agents of change are certainly dominant in the {recipient language}. 

The distinction between the two {transfer} types is not always clearly discern\-ible in case of intra-\ili{Arabic} contact-induced changes. In the towns of the Syrian steppe and the middle Euphrates the agents of change were mostly the sedentary population who adapted their speech towards the norms of the socially more {prestigious} \ili{Bedouin}. However, there has always been inter-marriage, and Bedouins often settled in towns and may well have adopted features from the local sedentary variety. Especially in cases like Muslim \ili{Baghdadi} (see §\ref{state}), we may assume with good reason that the \ili{Bedouin} character of today’s variety developed through both {imposition} and borrowing. 

  \subsection{Phonology}\largerpage
 \subsubsection{Aramaic-induced changes} 

It has been hypothesized that several phonological features of the \ili{Syrian} and \ili{Lebanese} dialects are due to the contact-induced influence of \ili{Aramaic}. But in the case of the shift from interdental fricatives to postdental plosives (/ð/ > /d/; /θ/ > /t/; /ð̣/ > /ḍ/) this is unlikely because: (i) this sound change is common crosslinguistically; (ii) it does not occur in all dialects of the region; and (iii) it is found in many other \ili{Arabic} dialects without an \ili{Aramaic} {substrate}. 

A phonotactic characteristic of most dialects spoken along the Mediterranean, from \ili{Cilicia} in the north to \ili{Beirut} in the south, is that all unstressed short vowels (including /a/) in open syllables are elided,\footnote{Therefore, Cantineau (\citeyear[108]{Cantineau1960book}) called them \textit{parlers} \textit{non} \textit{différentiels} – a term still very often applied in \ili{Arabic} dialectology – as they make no distinction in the treatment of the three short vowels.} whereas in other dialects east of Libya only /i/ and /u/ in this position are consistently dropped.

\ea
{\ili{Cilician} \ili{Arabic} (\citealt{Procházka2002Cukurova}: 31--32; 130)}\\
  \textup{\ili{Old} \ili{Arabic} (\ili{OA})} *raṣāṣ > \textit{rṣāṣ}  \textup{‘lead, plumb’\\
\ili{OA}} *miknasa > \textit{mikinsi} ‘\textup{broom’}\footnote{With insertion of an epenthetic /i/ to avoid a sequence of three consonants.}\\
*fataḥ-t > \textit{ftaḥt} ‘\textup{I opened’}\\
  \z
  
Because this rule corresponds to the phonotactics of \ili{Aramaic} and is otherwise not found to the same degree except in \ili{Maghrebi} dialects (cf. Benkato, this volume),\ia{Benkato, Adam@Benkato, Adam} pattern {replication} is likely, though cannot be proved.\footnote{Cf. Diem (\citeyear[47]{Diem1979}); Arnold \& Behnstedt (\citeyear[69--71]{ArnoldBehnstedt1993}); Weninger (\citeyear[748]{Weninger2011Aramaic}).}

In roughly the same region, except \ili{Cilicia} and many dialects of Hatay,\footnote{Where this phenomenon occurs only in Alawi villages \citep[84]{Arnold1998}.} the {diphthongs} /ay/ and /aw/ are only preserved in open syllables, but monophthongized to /ē/ and /ō/ respectively in closed syllables. In some regions, for instance on the island of \ili{Arwad}, both {diphthongs} {merge} to /ā/ in closed syllables (\citealt{Behnstedt1997}: map 31). 

\ea 
{\ili{Arwad}, western Syria \citep[278]{Procházka2013}} \\
    \textup{\ili{OA}} *bayt, *baytayn > \textit{bāt, baytān} \textup{‘house, two houses’\\
\ili{OA}} *yawm, *yawmayn >  \textit{yām, yawmān} \textup{‘day, two days’\\
\ili{OA}} *bayn al-iθnayn > \textit{bān it-tnān} \textup{‘}\textup{between the two’}\\
\z

Likewise, in older layers of \ili{Aramaic}, {diphthongs} were usually monophthongized in closed syllables (for \ili{Syriac} see \citealt{Nöldeke1904Semitic}: 34), which makes {imposition} by L1 speakers of \ili{Aramaic} rather likely \citep[227]{Fleisch1974Kfar}.

Another striking phenomenon is the split of historical /ā/ into /ō/ and /ē/ that is found in scattered areas of the Levant, particularly northern Lebanon, around the Syrian port of Tartous, the Qalamūn Mountains, and the exclusively Christian town of Maḥarde on the Orontes River.\footnote{For details cf. Behnstedt (\citeyear{Behnstedt1997}: map 32). The conditioned shift /ā/ > /ō/ is also found in and around Tarsus in Turkey \citep[37--38]{Procházka2002Cukurova}.} Because in many varieties of \ili{Aramaic} the old \ili{Semitic} /ā/ is reflected as /ō/, it could be assumed that \ili{Aramaic} speakers transferred their peculiar pronunciation to \ili{Arabic} when learning it. Fleisch (\citeyear[49]{Fleisch1974vowels}) rejected the hypothesis of an \ili{Aramaic} influence, arguing that the conditioned distribution of the two allophones is merely a further development of the [ɒ] : [æ] split widely attested for Lebanon and parts of western Syria. However, in the Syrian Qalamūn Mountains there are dialects with an unconditioned shift \citep{Behnstedt1992}, and this is precisely the region where the shift from \ili{Aramaic} to \ili{Arabic} occurred relatively late, probably after a long phase of {bilingualism}. In the town of Nabk, for instance, one can infer that the former \ili{Aramaic} speaking inhabitants would have simply turned every /ā/ into /ō/ – except those which long before had become [ē] (or [ɛ̄]) as a result of the so-called conditioned \textit{imāla} (i.e. the tendency of long /ā/ to be raised towards [ē] or even [ī] if the word contains an /i/ or /ī/).\footnote{Cf. Arnold \& Behnstedt (\citeyear[68]{ArnoldBehnstedt1993}).} Example (\ref{stud}) clearly shows that the distribution of the allophones is not conditioned by the consonantal environment.

\ea \label{stud}
{Nabk, Syria \citep[20]{Gralla2006}} \\
    \textup{\ili{OA}} *ṭābiḫ > \textit{ṭɛ̄beḫ}  \textup{‘cooking’ vs. \ili{OA}} *ṭālib > \textit{ṭōleb}  \textup{‘student’}\\
\textup{\ili{OA}} *ḥāmil > \textit{ḥɛ̄mel} \textup{‘pregnant’ vs. \ili{OA}} *ḥāmiḍ > \textit{ḥōmeḍ’} \textup{‘sour’}\\
\z

In these cases \ili{Aramaic} influence seems plausible. For the region of Tripoli it may be assumed that \ili{Aramaic} bilinguals from the adjacent mountains used [ō] instead of [ā] when speaking \ili{Arabic} and thus reinforced the already existing [ɒ] : [æ] split.\footnote{For discussion see Fleisch (\citeyear[48--50]{Fleisch1974vowels}; \citeyear[133--136]{Fleisch1974Kfar}), Diem (\citeyear[45--46]{Diem1979}); Behnstedt (\citeyear{Behnstedt1992}); Arnold \& Behnstedt (\citeyear[67--68]{ArnoldBehnstedt1993}); Weninger (\citeyear[748]{Weninger2011Aramaic}).}

  \subsubsection{The “new” phonemes /č/, /g/, and /p/}\label{newphonemes} 

Consonantal phonemes that are originally alien to \ili{Arabic} are found in all \ili{Arabic} dialects spoken in Turkey (see also Akkuş, this volume),\ia{Akkuş, Faruk@Akkuş, Faruk} northern Syria, and Iraq. These are the unvoiced affricate /č/, the voiced /g/,\footnote{The sound [g] is prevalent in whole Syria and Lebanon but seems to have phonemic status only in the north \citep[26]{Sabuni1980}. For further examples and discussion see \citet{Ferguson1969}. This “foreign” /g/ must therefore be differentiated from the /g/ which is the regular reflex of \ili{OA} *q. The latter development is found in many \ili{Bedouin}-type dialects.} and the unvoiced /p/, the latter mainly used in Iraq. The emergence of these sounds was very likely contact-induced, but it is often impossible to discern which language triggered each development: all three sounds are found in \ili{Persian}, \ili{Kurdish}, \ili{Turkish}, and the \ili{Lingua Franca}. For the dialects of Cilicia, Hatay and Syria, the main {source language} doubtless was \ili{Turkish}. The sound /p/ in the \ili{Iraqi} dialects was probably first introduced through contact with \ili{Persian} and \ili{Kurdish}, and then reinforced by \ili{Ottoman} {Turkish}. In the \ili{Bedouin}-type dialects of the region, the phonemes /č/ and /g/ are not products of contact-induced change but occur due to internal sound changes, unvoiced /č/ as a conditioned affricated variant of /k/ and /g/ as the ordinary reflex of \ili{OA} *q. 

Thus, it can be assumed that over the centuries speakers of the sedentary dialects of Iraq and Syria borrowed either from other languages or from \ili{Bedouin} \ili{Arabic} varieties words that possess these two sounds, which subsequently were fully incorporated into the phonemic inventory. This development may have been facilitated by the fact that the three sounds /č/, /p/, and /g/ are not fundamentally unfamiliar to \ili{Arabic}, but are the voiceless/voiced counterparts of the well-established phonemes /ǧ/, /b/, and /k/. It seems no accident that the new sound /č/ is much more often found in dialects that have preserved the affricate /ǧ/ than in those where it has shifted to /ž/, as illustrated in examples \REF{Aleppo} and \REF{Mosul}.

\ea\label{Aleppo}
\ili{Aleppo} \citep[205--210]{Sabuni1980}

\textit{čanṭāye} ‘handbag’ (\ili{Turkish} \textit{çanta})

\textit{čwāl} ‘sack’ (\ili{Turkish} \textit{çuval})

\textit{čāy} ‘tea’ (\ili{Turkish} \textit{çay})

\textit{gaǧaleg} ‘nightgown’ (\ili{Turkish} \textit{gecelik})
\z

The words given in \REF{Aleppo} are usually pronounced with [š] instead of [č] in the central \ili{Syrian} and \ili{Lebanese} dialects where contact with \ili{Turkish} was less intense and /ǧ/ is reflected as /ž/.\footnote{Cf. Behnstedt (\citeyear{Behnstedt1997}: maps 18, 19, 25). For details and more examples see Sabuni (\citeyear[205--210]{Sabuni1980}), who lists all words with \textit{č/g} in \ili{Aleppo}, and Procházka (\citeyear[185]{Procházka2002Adana}) for \ili{Cilician} \ili{Arabic}.}

\ea\label{Mosul}
\ili{Mosul} (own data)

\textit{ṣūč} ‘fault’ (\ili{Turkish} \textit{suç})

\textit{pāča} ‘stew of sheep and cow legs and innards’ (\ili{Kurdish}/\ili{Persian} \textit{pāče})

\textit{zangīn} ‘rich’ (\ili{Turkish} \textit{zengin})
\z

Once integrated into the phonological system, these sounds not only enabled easier integration of {loanwords} from other languages like \ili{French} and \ili{English} (see §\ref{french}), but sometimes also resulted in the spread of assimilation-induced allophones from single words to the whole paradigm or even {root}. In \ili{Aleppo} one finds *yəkdeb > \textit{yəgdeb} ‘he lies’, due to assimilation. The \textit{g} subsequently was transferred to other words derived from the {root}: \textit{gadab} ‘he lied’, \textit{gədbe} ‘lie’, and \textit{gaddāb} ‘liar’ (\citealt{Sabuni1980}: 26, 209). 

Speakers of sedentary dialects who had everyday contact with Bedouins – for example the inhabitants of Deir ez-Zor and Khatuniyya – first integrated /č/ and /g/ into their phonemic inventory through the borrowing of typically \ili{Bedouin} vocabulary such as \textit{dabča} ‘a Bedouin dance’ (Khawetna; \citealt[29]{Talay1999}) and \textit{ṭabga} ‘milk-bowl’ (\ili{Soukhne}; \citealt[310]{Behnstedt1994Soukhne}). These sounds then entered other fields of the lexicon, which led to unpredictable distribution, including doublets, as in \REF{Khawetna}--\REF{Baghdad}.

\ea\label{Khawetna} Khawetna \citep[28--31]{Talay1999}  

\textit{gəṣṣa} ‘forehead’, but \textit{qəṣṣa} ‘story’ (\ili{OA} *quṣṣa \textit{/} *qiṣṣa)

\textit{dīč} ‘rooster’  (\ili{OA} *dīk)
\ex \label{Deir}Deir iz-Zor \citep[42--43]{Jastrow1978}. 

\textit{gāʕ} ‘soil’ (\ili{OA} *qāʕ)

\textit{čam} ‘how much?’ (\ili{OA} *kam)
\ex \label{Baghdad}\ili{Baghdad} \citep[18--19]{Palva2009}

\textit{guffa} ‘large basket’ (\ili{OA} *quffa), but \textit{quful} ‘lock’ (\ili{OA} *qufl)

\textit{ʕigab} ‘to pass’, but \textit{ʕiqab} ‘to follow’ (both \ili{OA} *ʕaqab)
\z

The opposition /k/ : /č/ has even entered morphology, particularly with the 2\textsc{sg} suffixes: \textit{ʔabū-k} `your (\textsc{sg.m}) father' vs. \textit{ʔabū-č} ‘your (\textsc{sg.f}) father’. In the Syrian oasis of \ili{Soukhne}, long-term contact with speakers of \ili{Bedouin} dialects caused a chain of phonetic changes: first /k/ shifted to /č/, which originally was the reflex of \ili{OA} /ǧ/; then /č/ (< /ǧ/) shifted further to /ts/, which has become a unique feature of the local dialect. The unconditioned shift from /k/ > /č/, which is not found in the \ili{Bedouin} dialects, in turn caused a shift from /q/ > /k/.\footnote{See Behnstedt (\citeyear{Behnstedt1994Soukhne}: 4--11) for details.}

\ea
\ili{Soukhne} (\citealt{Behnstedt1994Soukhne}: 226, 344, 357, 360)

\textit{kirbi} ‘water-skin’ (< \ili{OA} *qirba, \ili{Bedouin} \textit{girba})

\textit{čalb} ‘dog’ (< \ili{OA} *kalb, \ili{Bedouin} \textit{čalib})

\textit{čurr} ‘donkey foal’ (< \ili{OA} *kurr, \ili{Bedouin} \textit{ku\R\R})

\textit{tsubn} ‘cheese’ (< \ili{OA} *ǧubn, \ili{Bedouin} \textit{ǧubun})
\z

\subsection{Morphology}
\subsubsection{Diminutive}
The \ili{Aramaic} {diminutive} suffix -\textit{ūn} has become restrictedly productive in \ili{Iraqi} \ili{Arabic} \citep[72]{Masliyah1997}, as illustrated in \REF{una}. In Syria and Lebanon it is only found in fossilized forms such as \textit{šalfūn} ‘young cockerel’ and \textit{qafṣūne} ‘little cage’.  Such kinds of morphological {transfer} are usually triggered by lexical borrowing. Thus, it may be assumed that this suffix spread from {loanwords} like \textit{šalfūne} ‘small knife blade’ < \ili{Aramaic} \textit{šelpūnā} ‘little knife’ (cf. \citealt{Féghali1918}: 82).\footnote{This must be a very old borrowing because the suffix is also found in the \ili{Gulf} dialects (e.g. \textit{ḥabbūna} ‘a little’ \citealt[279]{Holes2002}) and even in \ili{Tunisian} \ili{Arabic} \citep[496]{Singer1984}, where direct \ili{Aramaic} influence can be excluded.}

\ea\label{una}
Iraq \citep[72]{Masliyah1997}

\textit{darb} ‘road’ > \textit{darbūna} ‘alley’

\textit{gṣayyir} ‘short’ \textit{<} \textit{gṣayyrūn} ‘very short’

\textit{mḥammdūn} hypocoristic form of the name \textit{Muḥammad}
\z

\subsubsection{Morphological templates}
\ili{Syrian} and \ili{Lebanese} dialects exhibit a few word patterns (templates) that are attested for \ili{OA} (and other dialects) but seem to have become widespread through contact with \ili{Aramaic} due to their {frequency} in the latter. These are the verbal pattern šaC\textsubscript{1}C\textsubscript{2}aC\textsubscript{3} and the (primarily {diminutive}) nominal patterns C\textsubscript{1}aC\textsubscript{2}C\textsubscript{2}ūC\textsubscript{3} and C\textsubscript{1}aC\textsubscript{2}C\textsubscript{3}ūC\textsubscript{4}.\footnote{For the latter two see \citet{Corriente1969} and \citet{Procházka2004}.}\textsubscript{} 

An example of the first is \textit{šanfaḫ} ‘to puff up’, related to \textit{nafaḫ} ‘to blow up’ (\citealt[83]{Féghali1918}; cf. \citealt[201]{Lentin2018} for further discussion); the nominal forms are illustrated in \REF{donkey} and \REF{fatima}.

\ea\label{donkey}
\ili{Aleppo} (\citealt{Barthélemy1935}: 104, 158, 851) 

\textit{ǧaḥḥūš} ‘little donkey’ (related to \textit{ǧaḥš} `young donkey')

\textit{ḥassūn} ‘goldfinch’ (related to the personal name \textit{ḥasan})

\textit{namnūme} ‘small louse’ (cf. \textit{naml} `ants')
\z

The pattern C\textsubscript{1}aC\textsubscript{2}C\textsubscript{2}ūC\textsubscript{3}(i)\textsubscript{} is still productive in the whole region, including the \ili{Bedouin} dialects, to derive hypocoristic forms from personal names:

\ea\label{fatima}
\textit{fāṭma} > \textit{faṭṭūma}

\textit{ḥalīme} > \textit{ḥallūma}

\textit{aḥmad}/\textit{mḥammad} > \textit{ḥammūdi} 
\z

\subsubsection{Pronouns}
In all \ili{Syrian} and \ili{Lebanese} dialects, as well as in {Anatolia}, the \textsc{2pl} and \textsc{3pl} pronouns exhibit an /n/ in place of the /m/ that is found in other \ili{Arabic} dialects, which makes them look as if they were reflexes of \ili{OA} feminine forms (\tabref{tab:prochazka:2}). 

\begin{table}
\begin{tabularx}{\textwidth}{Xllll}
\lsptoprule
&    \ili{Damascus} &   \ili{Jerusalem} &    \ili{OA} \textsc{pl.f} & \ili{Syriac} \textsc{pl.m}\\
\midrule 
\textsc{2pl} & \textit{ʔəntu} \textit{/} \textit{-kon} & \textit{ʔintu}\textit{/}\textit{-kom} & \textit{ʔantunna} \textit{/} \textit{-kunna} & \textit{ʔatton} \textit {/} \textit{-kon}\\
\textsc{3pl} & \textit{hənne(n)} \textit{/} \textit{-hon} & \textit{humme} \textit{/} \textit{-hom} & \textit{hunna} \textit{/} \textit{-hunna} & \textit{hennon} \textit{/} \textit{-hon}\\
\lspbottomrule
\end{tabularx}

\caption{\textsc{2pl} and \textsc{3pl} pronouns}
\label{tab:prochazka:2}
\end{table}

Because generalization of the feminine is unlikely,\footnote{This is mainly because the feminine forms are only used for addressing groups of females, whereas the masculine forms may also refer to a mixed group. Therefore, the masculine forms are certainly more frequent. In all \ili{Arabic} dialects except those mentioned above, the gender-neutral plural forms are clearly derived from the historical masculine.} these forms have often been explained as a contact-induced change. In \ili{Aramaic} the corresponding pronouns also have /n/ (for \ili{Syriac} see \citealt{Muraoka2005}: 18). In particular, the 3rd person forms with final \textit{-n} exactly mirror the \ili{Aramaic} pattern, but lack a plausible intra-\ili{Arabic} etymology. Thus {imposition} seems plausible. Nevertheless, substratum influence has been doubted, particularly because of the infrequent evidence of \textit{n-}pronouns in other regions.\footnote{See Owens (\citeyear[244--245]{Owens2006}) and Procházka (\citeyear[283--284]{Procházka2018Fertile}) for details.}\textsuperscript{} 

\subsubsection{Vocative suffixes}\label{hypocoristic}
The suffixes \textit{-o} (in the west of the region) and \textit{-u} (in the east) can be attached to various kinship terms and given names when used for direct address, usually hypocoristically.\footnote{See also Ferguson (\citeyear[187]{Ferguson1997}).} 

\ea
Urfa (own data)

\textit{šnōnak} \textit{ḫayy-o?} ‘Brother, how are you?’

\textit{ǧidd-o} ‘Grandfather!’

\textit{ʕamm-o} ‘(paternal) Uncle!’

\textit{ḫāl-o} ‘(maternal) Uncle!’
\z

In Syria the suffix is also added to female nouns: \textit{ʕamm-t-o} ‘(paternal) Aunt!’ and \textit{ḫāl-t-o} ‘(maternal) Aunt!’, whereas in Iraq the corresponding forms end in \textit{{}-a}: \textit{ʕamm-a,} \textit{ḫāl-a}.

Since this suffix has no overt \ili{Arabic} etymology, it has been assumed to be a borrowing of the \ili{Kurdish} {vocative} \textit{-o} (e.g. \citealt{Grigore2007book}: 203). The \ili{Persian} suffix \textit{-u} also forms affective diminutives,\footnote{E.g. \textit{pesar-u} ‘kid’; \textit{ʕamm-u} is even the common word for ‘uncle’ \citep[1011]{Perry2007}.} which would make \ili{Persian} influence possible, at least for Iraq.\footnote{In the \ili{Iraqi} dialects the vowel is \textit{-u}, e.g. \textit{ʕamm-u,} \textit{ḫāl-u} and \textit{ǧidd-u} (\citealt{Abu-Haidar1999}: 145).} However, the distribution of this feature extends far beyond even indirect contact with \ili{Kurdish} or \ili{Persian},\footnote{The suffix is, for instance, attached to given names for endearment in the \ili{Gulf} dialects, cf. Holes (\citeyear{Holes2016}: 128). The address forms \textit{ya} \textit{ʕamm-u,} \textit{ya} \textit{ḫāl-u} ‘uncle’, \textit{gidd-u} ‘grandfather’, \textit{sitt-u} ‘grandmother’ are used in \ili{Cairo}, where hypocoristic variants of given names are likewise attested, e.g. \textit{mīšu} for \textit{hišām} \citep[109]{Woidich2006}. The suffix \textit{-o/-u} in address forms is also attested in eastern Sudan (Stefano Manfredi, personal communication), and in the Maghreb; Prunet \& Idrissi (\citeyear{PrunetIdrissi2014}: 184) provide a list of such nouns for Morocco.} though reinforcement and influence on the phonology may be possible for certain regions. Similar endings in \ili{Aramaic} \citep[88--89]{Fassberg2010} and Ethiopian \citep[122]{Brockelmann1928} suggest a common \ili{Semitic} origin (see also \citealt{Pat-El2017}: 463--465). 

\subsubsection{Turkish derivational suffixes}
All dialects of the region have incorporated the \ili{Turkish} suffix \textit{-ci} [ʤi] into their nominal morphology, as illustrated in \REF{kahrab} and \REF{pancar}. This suffix has become productive and is therefore a good example of morphological {matter borrowing} (\citealt{GardaniArkadievAmiridze2015}). It is widely used for expressing professions, occupations, and habitual actions – the latter overwhelmingly pejorative, or at least humorous. In \ili{Iraqi} dialects the suffix is reflected as \textit{-či}, which corresponds to its pronunciation in the regional \ili{Turkish} varieties. In the other varieties, it follows the usual development of *ǧ, which means that it is realized as \textit{-ǧi} or \textit{-ži.} 

\ea\label{kahrab}
Syria/\ili{Damascus} (own data)

\textit{kahrab-ži} ‘electrician’ (\textit{kahraba} ‘electricity’)

\textit{nəswān-ži} ‘womanizer’ (\textit{nəswān} ‘women’)

\textit{maškal-ži} ‘troublemaker’ (\textit{məšəkle} ‘problem’)
\ex \label{pancar}
Iraq \citep[40--44]{Procházka-Eisl2018}

\textit{pančar-či} ‘tire repairman’ (\textit{pančar} ‘puncture’)

\textit{mharrib-či} ‘human trafficker’ (\textit{mharrib} ‘one who helps s.o. to escape’)

\textit{ʕarag-či} ‘drunkard’ (\textit{ʕarag} ‘aniseed brandy’)
\z

The suffix clearly fills a morphological gap, because it enables morphologically transparent {derivation} even from {loanwords}, by preserving the basic, immediately recognizable word – in contrast to the \ili{Arabic} C\textsubscript{1}aC\textsubscript{2}C\textsubscript{2}āC\textsubscript{3}{} pattern or participles, which are derived from the {root} (for details see \citealt{Procházka-Eisl2018}).

To a lesser extent other \ili{Turkish} suffixes have enhanced the morphological devices of the dialects treated here,\footnote{See Halasi-Kun (\citeyear[68--71]{Halasi-Kun1969}); Sabuni (\citeyear[168]{Sabuni1980}); Masliyah (\citeyear{Masliyah1996}); Procházka (\citeyear[186]{Procházka2002Cukurova}).} specifically the {relative} suffix \textit{-li}, the {privative} suffix \textit{-siz}, and the abstract suffix \textit{-lik}, which is reflected as \textit{-loɣiyya} in Iraq, i.e. with the \ili{Arabic} abstract morpheme affixed. For the most part these suffixes appear in \ili{Turkish} {loanwords}, e.g. \ili{Cilicia} \textit{ṣiḥḥat-li} (< \ili{Turkish} \textit{sıhhatlı}) ‘healthy’, \textit{raḥaṭ-ṣīz} (< \ili{Turkish} \textit{rahatsız}) ‘uncomfortable’. Only in Iraq have they gained a certain degree of productivity, particularly \textit{-sizz} and \textit{-loɣiyya}:

\ea
Iraq \citep[293--294]{Masliyah1996} 

\textit{muḫḫ-sizz} ‘stupid, brainless’

\textit{ḥaya-sizz} ‘shameless’

\textit{ḥaywān-loɣiyya} ‘ignorance’ (lit. ‘animal-ness’)

\textit{zmāl-loɣiyya} ‘stupidity’ (lit. ‘donkey-ness’)
\z

\subsubsection{Light-verb constructions}
\ili{Arabic} dialects spoken in Turkey not infrequently use light-verb constructions (in \ili{Turkish} grammar mostly called phrasal verbs) which consist of the verb ‘to do’ plus a following noun (see also Akkuş, this volume).\ia{Akkuş, Faruk@Akkuş, Faruk} Such compound verbs are very frequent in \ili{Turkish} (and \ili{Kurdish}) and enable easy integration of foreign vocabulary into the verbal system. The {light verbs} found in the \ili{Arabic} dialects show that this {formation} is a case of selected pattern {replication} because, first, not all examples are exact copies of the \ili{Turkish} model, and second, the {word order} follows the \ili{Arabic} VO rather than the \ili{Turkish} OV pattern:

\ea
Harran--Urfa (own data)

\textit{sāwa} \textit{qaza} (\ili{Turkish} \textit{kaza} \textit{yapmak}) ‘to have an accident’

\textit{sāwa} \textit{ʕēš} (in \ili{Turkish} not a phrasal verb, but \textit{pişirmek}) ‘to cook’
\ex 
\ili{Cilician} \ili{Arabic} \citep[198]{Procházka2002Cukurova} 

\textit{sawwa} \textit{zarar} (\ili{Turkish} \textit{zarar} \textit{vermek}) ‘to harm’

\textit{sawwa} \textit{ḫayir} (\ili{Turkish} \textit{hayır} \textit{işlemek}) ‘to do a good deed’
\z

\subsubsection{Intra-Arabic dialect contact}
Concerning intra-\ili{Arabic} contact, here we see that this has led to the adoption of typical \ili{Bedouin}-type pronouns into sedentary dialects (cf. \citealt{Palva2009}: 27--29), e.g.:

\ea
\ili{Baghdad}, Deir ez-Zor, \ili{Soukhne} 

\textit{ʔəḥna} for \textit{nəḥna} \textsc{1pl}
\ex 
\ili{Baghdad} 

\textit{ʔāni} for \textit{ʔana} \textsc{1sg}
\z

In addition, as shown in \tabref{tab:prochazka:3}, virtually all the eastern sedentary dialects of Syria have copied the typical \ili{Bedouin}-type active participles of the verbs ‘to eat’ and ‘to take’, which exhibit initial \textit{m-} (\citealt{Behnstedt1997}: map 175).

\begin{table}
\begin{tabularx}{.8\textwidth}{lllX}
\lsptoprule
 Bedouin\il{Bedouin} &   Soukhne\il{Soukhne} &  Palmyra &  Damascus\il{Damascus}\\
 \midrule 
\textit{māčil / māḫið} & \textit{mīčil / mīḫið} & \textit{mākil / māḫið} & \textit{ʔākel / ʔāḫed}\\
\lspbottomrule
\end{tabularx}
\caption{Active participles of the verbs ‘to eat’ / ‘to take’}
\label{tab:prochazka:3}
\end{table}

Finally, in a few places {intensive} mutual contact has resulted in an {interdialect} \citep[62]{Trudgill1986} with completely new forms, such as the 3\textsc{pl.m} {inflectional} suffix \textit{-a} in the Syrian village of Ṣōrān (\citealt{Behnstedt1994Dialektkontakt}: 423--425), as shown in \tabref{tab:prochazka:4}.

\begin{table}
\begin{tabularx}{.8\textwidth}{XXl}
\lsptoprule
 Bedouin &  Sedentary &  Ṣōrān\\
\midrule 
\textit{gāḷ-am} & \textit{qāl-o} & \textit{qāl-a}\\
\lspbottomrule
\end{tabularx}
\caption{3\textsc{pl.m} inflectional suffixes -- `they said'}  
\label{tab:prochazka:4}
\end{table}

  \subsection{Syntax}
  \subsubsection{Changes due to contact with Aramaic}
  \subsubsubsection{Clitic doubling}

In all but the \ili{Bedouin}-type dialects of the region, two constructions exist which both use an anticipatory pronoun and the {preposition} \textit{l-} ‘to’: (i) a construction involving analytical marking of a {definite} direct object, as in (\ref{damas}--\ref{cilic}); and (ii) a construction involving analytic attribution of a noun, as in (\ref{christ}). The {frequency} and constraints of these two cases of {clitic doubling} show great variety, but in general the usage of construction (i) is restricted to specific objects, particularly elements denoting human beings, and construction (ii) is mostly found with inalienable possession, particularly kinship. A detailed discussion of both features is found in \citet{Souag2017clitic}.\pagebreak\largerpage

\ea
{\ili{Damascus} \citep[144]{Berlinches2016}} \\
\gll ḥabbēt-o la-ʕamər\\
     love.\textsc{prf.1sg}-\textsc{3sg.m} to-Amr\\
\glt ‘I loved Amr.’ \label{damas}
\ex 
{\ili{Baghdad}, Christian (\citealt{Abu-Haidar1991}: 116)}\\
\gll  qaɣētū-nu l-əl-əktēb\\
     read.\textsc{prf.1sg}{}-\textsc{3sg.m} to-\textsc{def}{}-book\\
\glt ‘I read the book.’
\ex  
{\ili{Cilician} \ili{Arabic} (\textit{ʕalā} instead of \textit{l-}; \citealt[158]{Procházka2002Cukurova}}\\
\gll   biyḥibb-u ʕala ḫāl-u\\
     love.\textsc{impf.ind.3sg.m}{}-\textsc{3sg.m} on uncle-\textsc{3sg.m}\\
\glt ‘He loves his (maternal) uncle.’ \label{cilic}
\ex  
{\ili{Baghdad}, Christian (\citealt{Abu-Haidar1991}: 116)} \\
\gll maɣt-u l-aḫū-yi\\
     wife\textsc{-3sg.m} to-brother-\textsc{obl.1sg}\\
\glt ‘my brother’s wife’ \label{christ}
\z

Though the {preposition} \textit{l-} is sometimes attested in \ili{Classical Arabic} for introducing direct objects and is common even in Modern \ili{Standard} Arabic for analytic noun annexation, there are good arguments that the two constructions are pattern replications of an \ili{Aramaic} model.\footnote{Not discussed here are two variants of construction (i), one without the suffix and the other without the {preposition} (cf. \citealt[203]{Lentin2018}). Among the many studies that are in favor of \ili{Aramaic} influence are Contini (\citeyear[105]{Contini1999}); Blanc (\citeyear[130]{Blanc1964}); and Weninger (\citeyear[750]{Weninger2011Aramaic}). Diem (\citeyear[47--49]{Diem1979}) and Lentin (\citeyear{Lentin2018}) are more skeptical. Souag (\citeyear[52]{Souag2017clitic}) suggests that at least “the initial stages of the development of {clitic doubling} in the Levant derive from \ili{Aramaic} substratum influence, but the current situation also reflects subsequent \ili{Arabic}-internal developments”.} For one thing, they do not have direct parallels either in \ili{OA} or in dialects which lacked contact with \ili{Aramaic}. Example (\ref{rubin}) shows that both constructions have striking parallels in especially the later eastern varieties of \ili{Aramaic} \citep[94--104]{Rubin2005}. 

\ea \label{rubin}
\ea
{\ili{Syriac} \citep[100]{Rubin2005}}\\
\gll bnā-y l-bayt-ā\\
     build.\textsc{prf.3sg.m}{}-3\textsc{sg.m} to-house-\textsc{def}\\
     \glt ‘He built the house.’
\ex
{ \ili{Syriac} \citep[29]{Hopkins1997}\footnote{The same pattern using the linker \textit{d}{}- is more common.}}\\
\gll šm-ēh l-gabr-ā\\
     name-3\textsc{sg.m} to-man-\textsc{def}\\
\glt ‘the name of the man’
\z
\z

  \subsubsubsection{\textit{fī} ‘can’}
In the entire western part of the region including southern Turkey, the {preposition} \textit{fī} ‘in’, together with a pronominal suffix, is used to express a capability, as in \REF{cowell}. This has a striking parallel in the modern \ili{Aramaic} \textit{ʔīθ} \textit{b-} ‘there is in' {\textasciitilde} `be able’ \citep[52]{Borg2004}.

\ea \label{cowell}
{\ili{Damascus} \citep[415]{Cowell1964}} \\
\gll fī-ni sāʕd-ak əb-kamm lēra\\
     in-\textsc{1sg} help.\textsc{impf.1sg-2sg.m} with-some pound \\
\glt ‘Can I help you with a few pounds?’
\z

  \subsubsubsection{Specific indefinite \textit{šī}}
A final example of possible \ili{Aramaic} influence is the \ili{Syrian} particle \textit{šī} that mainly indicates partial specifity, as in \REF{shi}. It might be a pattern {replication} of the \ili{Western Neo-Aramaic} form \textit{mett}, used with the same function \citep[49]{Diem1979}. What reduces the likelihood of {imposition} by \ili{Aramaic} speakers is the existence of a {cognate} in \ili{Moroccan Arabic} which is used with almost the same function.\footnote{Cf. Brustad (\citeyear{Brustad2000}: 19, 26--27) and Wilmsen (\citeyear[51--53]{Wilmsen2014}).}

\ea\label{shi} 
{\ili{Damascus} (own data)}\\
\gll hnīk fī šī ʕamūd\\
     there \textsc{exs} \textsc{indf} column \\
\glt ‘There is some column.’
\z

\subsubsection{Changes due to contact with other languages}
\subsubsubsection{Definiteness}
A hallmark of both sedentary and \ili{Bedouin}-type \ili{Iraqi} dialects is that reflexes of the noun \textit{fard} ‘individual (thing or person)’ are used to mark different kinds of indefiniteness \citep[118--119]{Blanc1964}. The same form with the same {indefinite article} function is found in in the Iranian province of \ili{Khuzestan}, and in all \ili{Arabic} speaking language islands of Central Asia, i.e. Khorasan, \ili{Uzbekistan}, and Afghanistan, as illustrated in \REF{fadd}.

\ea\label{fadd} 
{Kirkuk (own data)}\\
\gll taʕrif-lak fadd ṭabīb bāṭiniyye\\
     know.\textsc{impf.2sg.m-dat.2sg.m} \textsc{indf} doctor internal \\
\glt ‘Do you know a doctor of internal medicine?'
\z

It is very likely that the noun \textit{fard} has developed into a kind of {indefinite article} under the influence of other areal languages, particularly \ili{Turkish}, \ili{Turkmen}, \ili{Persian}, and \ili{Neo-Aramaic}. However, in contrast to all contact languages, \ili{Iraqi} \ili{Arabic} has not grammaticalized the numeral ‘one’ (\textit{wāḥəd}), but \textit{fard}. This clearly indicates that this feature is a case of pattern {replication}. There are many parallels in the functions of the indefinite articles (such as marking pragmatic salience, semantic individualization, approximation with {numerals}). Moreover, in all languages they are not fully systematized as a grammatical category as their usage is often optional. 

In the dialects of the {Jews} of \ili{Kurdistan} the {definite} {article} is often omitted in subject position – a flagrant imitation of the \ili{Kurdish} model (see also Akkuş, this volume, for some \ili{Anatolian} dialects).\ia{Akkuş, Faruk@Akkuş, Faruk}

\ea 
{\ili{Kurdistan} \ili{Arabic} \citep[71]{Jastrow1990chapter}} \\
\gll baʕdēn mudīra baʕatət ḫalf-na\\
     then director send.\textsc{prf.3sg.f} after-\textsc{1pl}\\
\glt ‘Then the director sent for us.’
\z

\subsubsubsection{\textit{m-bōr} ‘because, in order to’}
An interesting case of {calquing} which shows the difficulty of distinguishing between borrowing and {imposition} (see Manfredi, this volume) is the conjunction \textit{m-bōr} ‘because, in order to’. It exhibits both matter and pattern {transfer}, as it is a copy of \ili{Kurdish} \textit{ji} \textit{ber} \textit{(ku).} In the actual form the \ili{Kurdish} \textit{ji} ‘from’ was replaced by the \ili{Arabic} equivalent \textit{m}{}- \citep[64]{Jastrow1979}. 

\subsubsubsection{Evidentiality}
Syntactic change because of contact with \ili{Turkish} is restricted to the \ili{Arabic} dialects spoken in Turkey. In \ili{Cilicia} and the Harran--Urfa region, active participles express evidentiality, that is, they are used in utterances where a speaker refers to second-hand information. As evidentiality is not a common category in \ili{Semitic}, it is very likely that the bilingual \ili{Arabic} speakers of those regions copied this linguistic category from \ili{Turkish}. In \ili{Turkish}, any second-hand information is obligatorily marked by the verbal suffix \textit{-mış}, whose second function besides evidentiality is to express stativity and perfectivity. The latter two functions are assumed by the active {participle} in many \ili{Arabic} dialects, including those in question here. Thus, we can suppose that the {stative}/perfective function, which is shared by both \ili{Arabic} active participles and the \ili{Turkish} suffix \textit{-mış}, was likely the starting point of the development that led to the additional evidential function of \ili{Arabic} participles. The fact that evidentials seem to spread readily through language contact \citep[10]{Aikhenvald2004} makes \ili{Turkish} influence even more probable.\footnote{For more examples and further details see Procházka (\citeyear[200--201]{Procházka2002Cukurova}) for \ili{Cilicia}, and Procházka \& Batan (\citeyear[464--465]{ProcházkaBatan2016}) for the \ili{Bedouin}-type dialects in the Harran--Urfa region.} The example in (\ref{evid}) illustrates how the speaker uses perfect forms for those parts of the narrative he witnessed himself, and participles for secondhand information (perfect forms italic, participles in bold face). 

\ea \label{evid}
{Harran--Urfa (\citealt{ProcházkaBatan2016}: 465)}\\
  ʔiḥne b-zimānāt \textit{čān} ʕid-na ǧār b-al-maḥalle huwwa \textit{māt} \textit{ərtiḥam} əngūl-lu šēḫ mǝṭar […] nahā{\R} rabīʕ-u wāḥad  \textbf{ʕāzm}{}-u ʕala stanbūl \textbf{rāyiḥ} maʕzūm ʕala stanbul \textbf{māḫið} šēḫ mǝṭar əb-sāgt-u\\

\glt ‘Once we had a neighbor in our quarter. He died; he passed away. We called him Sheikh Mǝṭar. One day somebody invited his friend to Istanbul. As he was invited he went to Istanbul and he took Sheikh Mǝṭar with him.’
\z

\subsubsubsection{Comparative and superlative}
In most \ili{Arabic} dialects that are spoken in Turkey, comparatives and superlatives may be expressed by means of the \ili{Turkish} particles \textit{daha} and \textit{en}, respectively, followed by the simplex instead of the {elative} form of the adjective (cf. Akkuş, this volume).\ia{Akkuş, Faruk@Akkuş, Faruk} As for comparatives, the use of such constructions is rather restricted, while, at least in \ili{Cilician} \ili{Arabic}, they are relatively frequent for the {superlative}. 

\ea 
{Harran--Urfa (own data)}\\
\gll daha zēn ṣārat\\
     more good become.\textsc{prf.3sg.f}\\
\glt ‘It has become better.’
\ex 
{\ili{Cilician} \ili{Arabic} \citep[155]{Procházka2002Cukurova}} \\
\gll mīn en zangīl bi-d-dini\\
     who \textsc{sup} rich in-\textsc{def}-world\\
\glt ‘Who is the richest (person) in the world?’ 
\z

In \ili{Cilicia}, comparison is often expressed by the {elative} pattern of an adjective, which is preceded by the particle \textit{issa}. This clearly reflects a {calque}: the \ili{Turkish} equivalent of the adverb \textit{issa} ‘still, yet’ is \textit{daha}, which in \ili{Turkish} is also used as the particle of the {comparative}. 

\ea 
{\ili{Cilician} \ili{Arabic} \citep[202]{Procházka2002Cukurova}} \\
\gll ṣāyir issa aḥsan \\
     become.\textsc{ptcp} more good.\textsc{ela}\\
\glt ‘It became better’. 
\ex 
\ili{Turkish}\\
\gll Daha iyi ol-du.\\
     more good become-\textsc{prf.3sg}\\
\glt ‘It became better.’
\z

\subsubsubsection{Valency}
Sometimes a change in verb valency occurs as a consequence of the copying of \ili{Turkish} models. A case found throughout these dialects is the verb \textit{ʕaǧab} ‘to like’: usually in \ili{Arabic} the entity that is liked is the grammatical subject and the person who likes something is the direct object of the verb; but in the \ili{Arabic} dialects in question, the construction of this verb reflects its \ili{Turkish} (and \ili{English}) usage with the person doing the liking being the grammatical subject.

\ea 
\ea \ili{Cilicia} \citep[200]{Procházka2002Cukurova}\\
   \gll ʕǧabt bayt-ak\\
     like.\textsc{prf.1sg} house-\textsc{2sg.m}\\
\ex
 \ili{Damascus} (own data)\\
   \gll bēt-ak ʕažab-ni\\
     house-\textsc{2sg.m} like.\textsc{prf.3sg.m-1sg}\\
\glt ‘I liked your house.’
\z
\z

  \subsection{Lexicon} 
Apart from the \ili{Aramaic} {loanwords} also found in \ili{Classical Arabic} (see \citealt{Retsö2011}; van Putten, this volume) – often in the realms of religion and cult – the dialects of this region exhibit a large number of \ili{Aramaic} lexemes. They are particularly common in Lebanon and western Syria, but also found in Iraq and even in the \ili{Bedouin}-type dialects (\citealt{Féghali1918}; \citealt{Borg2004}; \citeyear{Borg2008}). A large percentage of these words belong to flora and fauna, agriculture, architecture, tools, kitchen utensils, and other material objects:\footnote{See also Neishtadt (\citeyear[282]{Neishtadt2015}). Note that, unless otherwise indicated, lexemes cited in this section are taken from \citet{Barthélemy1935} for \ili{Syrian} dialects, and \citet{WoodheadEtAl1967} and \citet{alBakri1972} for \ili{Iraqi} dialects.}

\ea
\textit{ṣumd} \textit{{\textasciitilde} ṣimd} ‘plough’ < \ili{Syriac} \textit{ṣāmdē} ‘yoke’

\textit{qālūz} ‘bolt (of a door)’ < \ili{Syriac} \textit{qālūzā}

\textit{nāṭūr} ‘guard (of a vineyard etc.)’ < \ili{Syriac} \textit{nāṭūrā}

\textit{šaṭaḥ} ‘to spread’ < \ili{Syriac} \textit{šeṭaḥ}

\textit{šōb} ‘heat, hot’ < \ili{Syriac} \textit{šawbā}
\z

Many nautical terms and words denoting agricultural products and tools were borrowed by \ili{Arabic} from \ili{Greek}, often via other languages, especially \ili{Aramaic},\footnote{This is especially true for words related to Christian liturgy and ritual, which constitute about twenty per cent of the \ili{Greek} vocabulary that entered the dialects of Syria.} the \ili{Lingua Franca}, and \ili{Turkish}: 

\ea
\textit{brāṣa} < \ili{Greek} \textit{práson} ‘leek’ 

\textit{laḫana} < \ili{Greek} \textit{láḫana} ‘cabbage’

\textit{dərrāʔen} \textit{<} \ili{Greek} \textit{dōrákinon} ‘peaches’

\textit{ʔabrīm/brīm} ‘keel’ < \ili{Greek} \textit{prýmnē} ‘stern, poop’

\textit{sfīn} < \ili{Greek} \textit{sfēn} ‘wedge’
\z

\ili{Kurdish} borrowings are mainly restricted to northern Iraq, where {bilingualism} is widespread: 
\ea
\ili{Mosul}

\textit{pūš} ‘chaff’ \textit{<} \ili{Kurdish} \textit{pûş}

\textit{hēdi} \textit{hēdi} ‘slowly’ < \ili{Kurdish} \textit{hêdî} \citep[68]{Jastrow1979}
\z

The {intensive} cultural and economic contacts between Iraq and Iran led to many \ili{Persian} {loanwords} in various domains of the \ili{Iraqi} dialects. 

\ea
\textit{mēwa} ‘fruit’ < \ili{Persian} \textit{mīva} \textit{{\textasciitilde} mayva}

\textit{baḫat} ‘luck’ < \ili{Persian} \textit{baḫt}

\textit{čariḫ} ‘wheel’ \textit{<} \ili{Persian} \textit{čarḫ} 

\textit{gulguli} ‘pink’ \textit{<} \ili{Persian} \textit{gol} ‘rose’

\textit{yawāš} ‘slow’ < \ili{Persian} \textit{yavāš}

\textit{puḫta} ‘mush’ < \ili{Persian} \textit{poḫte} ‘(well) cooked’
\z

\ili{Ottoman} {Turkish} contributed a great deal to culinary vocabulary and the terminology of clothing and (technical) tools of Syria and Iraq.\footnote{The same {loanwords} are, of course, often found in other regions that were under Ottoman rule, above all in Egypt, but also in Tunisia, {Yemen} and other regions.} It was even the source of several adverbs and even verbs in the local \ili{Arabic} varieties (\citealt{Halasi-Kun1969}; \citeyear{Halasi-Kun1973}; \citeyear{Halasi-Kun1982}).

\ea
Syria (\ili{Damascus})

\textit{šāwərma} ‘shawarma’ < \ili{Turkish} \textit{çevirme} 

\textit{ṣāž} ‘iron plate for making bread’ \textit{<} \ili{Turkish} \textit{saç}

\textit{yalanži} ‘vine-leaves stuffed with rice’ < \ili{Turkish} \textit{yalancı} ‘liar’ (as they pretend to be “real” \textit{dolma} stuffed with meat)

\textit{šīš} \textit{ṭāwūʔ} ‘spit-roasted chicken’ < \ili{Turkish} \textit{şiş} \textit{tavuk}

\textit{kǝzlok} ‘glasses’ \textit{<} \ili{Turkish} \textit{gözlük}

\textit{ʔūḍa} ‘room’ < \ili{Turkish} \textit{oda}

\textit{ballaš} ‘to begin’ < \ili{Turkish} \textit{başla-mak} by metathesis.
\ex 
Iraq (Muslim \ili{Baghdadi}, cf. \citealt{Reinkowski1995}) 

\textit{qūzi} ‘a dish with roasted mutton’ < \ili{Turkish} \textit{kuzu} ‘lamb’

\textit{tēl} ‘wire’ < \ili{Turkish} \textit{tel}

\textit{yašmāɣ} ‘kerchief (for men)’ < \ili{Turkish} \textit{yaşmak} ‘veil (for women)’

\textit{bōš} ‘empty; neutral’, which yielded also the verb \textit{bawwaš} ‘to put into neutral (gear)’ < \ili{Turkish} \textit{boş} ‘empty’

\textit{qačaɣ} ‘smuggled goods’ < \ili{Turkish} \textit{kaçak}
\z

During the last century, the \ili{Arabic} dialects in Turkey\footnote{For \ili{Cilicia} see Procházka (\citeyear{Procházka2002Cukurova}; \citeyear[187--199]{Procházka2002Adana}).} have incorporated numerous \ili{Turkish} words in addition to {loanwords} from Ottoman times. Among them are terms in education, medicine, sports, media, and technology. Besides these, kinship terms, the vocabulary of everyday life, and structural words like adverbs and discourse markers have infiltrated the dialects from \ili{Turkish}. 

\ea
\ili{Cilician} \ili{Arabic}

\textit{qāyin} \textit{…} ‘-in-law’ (< \ili{Turkish} \textit{kayın})

\textit{ṭōrūn} ‘grandchild’ (< \ili{Turkish} \textit{torun})

\textit{bīle} ‘even’ (< \ili{Turkish} \textit{bile})

\textit{qāršīt} ‘opposite from’ (< \ili{Turkish} \textit{karşı})
\z

The cases of semantic {extension} of an \ili{Arabic} word result from the wider semantic range of its \ili{Turkish} equivalent which has been transferred into \ili{Arabic}. Thus, in both \ili{Cilician} and Harran--Urfa \ili{Arabic} \textit{sāq/ysūq} ‘to drive’ also occurs with the meaning of ‘to last’ like the \ili{Turkish} verb \textit{sürmek}. In Harran--Urfa \textit{b-arð̣} ‘on the place/ground (of)’ has become a {preposition}/conjunction meaning ‘instead’. This can be seen as an instance of {contact-induced grammaticalization} (\citealt{GardaniArkadievAmiridze2015}: 4) under the influence of \ili{Turkish} \textit{yerine} ‘instead, in its place’.

\ea
{Harran--Urfa (own data)}\\
 \gll   al-mille tākl-u b-arð̣ al-laḥam\\
     \textsc{def}{}-people eat.\textsc{impf.3sg.f-3sg.m}  in-place \textsc{def}{}-meat\\
\glt  ‘The people eat it instead of meat.’
\ex 
{Harran--Urfa (own data)}\\
\gll    b-arð̣-in tibči ʔigir āya\\
     in-place-\textsc{link} cry.\textsc{impf.2sg.m} read.\textsc{imp.sg.m} verse\\
\glt ‘Instead of crying recite a (Koranic) verse!’
\z

In Iraq, many \ili{English} words related to Western culture and technology have been, and still are, borrowed into the dialects. The same is true for \ili{French} in Syria and (particularly) Lebanon (cf. \citealt{Barbot1961}: 176).

\ea
Iraq (words of \ili{English} origin)

\textit{kitli} < kettle 

\textit{buṭil} < bottle

\textit{glāṣ} < glass

\textit{pančar} ‘flat tire’ (< \textit{puncture})

\textit{pāysikil} < bicycle

\textit{māṭōrsikil} < motorcycle

\textit{lōri} < lorry 

\textit{igzōz} < exhaust (pipe)

\textit{brēk} < brake
\ex 
Syria and Lebanon (words of \ili{French} origin)

\textit{gātto} \textit{{\textasciitilde} gaṭō < gâteau} ‘cake’

\textit{garsōn} \textit{<} \textit{garçon} ‘waiter’

\textit{sēšwār} \textit{<} \textit{séchoir} ‘hair drier’

\textit{kwaffēr} \textit{<} \textit{coiffeur} ‘hair-dresser’

\textit{ʔaṣanṣēr} \textit{<} \textit{ascenseur} ‘elevator’

\textit{grīb} < \textit{grippe} ‘influenza’
\z

Due to long-term contacts, there are mutual borrowings between the \ili{Bedouin} and sedentary dialects of the region. This affects not only specific vocabulary of the respective cultures but also basic lexical items. Historically, the sedentary dialects have been much more influenced by the \ili{Bedouin}-type dialects than vice versa. 

\section{Conclusion} 

The sociolinguistic history of the regions treated here suggests that the conditions for {imposition} were relatively restrictive and mainly found in contact settings with \ili{Aramaic}, which, over the centuries, has been given up by most of its speakers in favor of \ili{Arabic}. Thus, it is not surprising that so many features beyond the lexicon for which contact-induced change can be assumed are related to \ili{Aramaic} influence. 

Morphological borrowing is in general relatively rare because it presupposes a high intensity of contact (\citealt{GardaniArkadievAmiridze2015}: 1). Practically all cases presented in §\ref{persian} corroborate the universal tendencies that: (i) {derivational} morphology is more prone to borrowing than {inflectional} morphology; and (ii) nominalizers and diminutives are very frequently represented in instances of borrowed {derivational} morphology (\citealt{GardaniArkadievAmiridze2015}: 7; \citealt{Seifart2013}). On the whole, the \ili{Bedouin}-type dialects exhibit significantly fewer contact-induced changes than the sedentary dialects. This may be the result of both the Bedouin groups' nomadic way of life at the fringes of the desert and their tribally organized society, which impedes intense contact with outsiders.

The {relative} infrequency of contact-induced changes in morphology and syntax found in the \ili{Arabic} varieties spoken in Turkey have two main explanations: first, the high degree of complete {bilingualism} is a very recent phenomenon that only pertains to the last two generations; and second, and probably more importantly, the great structural differences between the two languages, which have impeded both matter and pattern replications.

What is still relatively unclear is the degree of historical {bilingualism} between \ili{Arabic} on the one hand and \ili{Ottoman} {Turkish}, \ili{Kurdish}, and \ili{Persian} on the other. Future research would be particularly desirable with regard to Iraq, providing interesting new data on contact-induced changes in multilingual regions like \ili{Mosul} and Kirkuk, where \ili{Arabic}, \ili{Turkmen}, and \ili{Kurdish} speakers have been in contact for a long time. Also, studies like that of  \citet{Neishtadt2015} for \ili{Palestine} should be carried out for \ili{Syrian} and especially \ili{Iraqi} dialects with regard to lexical borrowings from \ili{Aramaic}. Another completely under-researched topic is {idiomatic} constructions, in which the mutual influence of most languages in the region may be assumed.\pagebreak

\section*{Further reading}
There are no studies which treat the subject of contacts between \ili{Arabic} and the other languages of the whole region covered in this chapter. However:

\begin{furtherreading}
\item \citet{ArnoldBehnstedt1993} is an in-depth study of the mutual contacts between \ili{Western Neo-Aramaic} and the local \ili{Arabic} dialects in the Anti-Lebanon Mountains of Syria.
\item \citet{Diem1979} is a pioneer study of {substrate} influence in the modern \ili{Arabic} dialects, though with focus on South Arabia, i.e. outside of the region treated in this chapter.
\item \citet{Palva2009} is a very good case study of the diachronic relations between sedentary and \ili{Bedouin}-type dialects in the Iraqi capital \ili{Baghdad}.
\item \citet{Weninger2011Aramaic} is a concise overview of contact between different varieties of \ili{Aramaic} and \ili{Arabic}.
\end{furtherreading}

\section*{Acknowledgements}
I am grateful to my colleagues Bettina Leitner and Veronika Ritt-Benmimoun for their valuable comments on earlier drafts of this paper. I warmly thank Jérôme Lentin for extensive discussion of the possible origin of the hypocoristic \textit{-o} suffix (§\ref{hypocoristic}) and his help in finding important sources.


\section*{Abbreviations}
\setlength{\columnsep}{30pt}
\begin{multicols}{2}
\begin{tabbing}
\textsc{ipfv} \hspace{1em} \= before common era\kill
\textsc{1, 2, 3} \> 1st, 2nd, 3rd person \\
BCE \> before Common Era \\
CE \> Common Era \\
\textsc{comp} \> {complementizer} \\
\textsc{def} \> {definite} \\
\textsc{f} \> feminine \\
\textsc{ela} \> {elative} degree \\
\textsc{exs} \> {existential} \\
\textsc{imp} \> imperative \\
\textsc{impf} \> imperfect (prefix conjugation) \\
\textsc{indf} \> indefinite \\
L1 \> first language \\
L2 \> second language \\
\textsc{link} \> linker \\
\textsc{m} \> masculine \\
{OA} \> Old Arabic \\
\textsc{obl} \> oblique \\
\textsc{pl} \> plural \\
\textsc{prf} \> perfect (suffix conjugation) \\
\textsc{sg} \> singular \\
\textsc{sup} \> {superlative}
\end{tabbing}
\end{multicols}


\sloppy
\printbibliography[heading=subbibliography,notkeyword=this] 
\end{document}
