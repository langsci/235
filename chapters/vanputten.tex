\documentclass[output=paper]{langsci/langscibook} 
\author{Marijn van Putten\affiliation{University of Leiden}}
\title{Classical and Modern Standard Arabic}
\abstract{The highly archaic Classical Arabic language and its modern iteration Modern Standard Arabic must to a large extent be seen as highly artificial archaizing registers that are the High variety of a diglossic situation. The contact phenomena found in Classical Arabic and Modern Standard Arabic are therefore often the result of imposition. Cases of borrowing are significantly rarer, and mainly found in the lexicon sphere of the language.}


\begin{document}
\maketitle 
\let\textstyleannotationreference\relax{}
\section{Current state and historical development}

Classical \ili{Arabic} (\ili{CA}) is the highly archaic variety of \ili{Arabic} that, after its codification by the Arab Grammarians around the beginning of the ninth century, becomes the most dominant written {register} of \ili{Arabic}. While forms of Middle \ili{Arabic}, a style somewhat intermediate between \ili{CA} and spoken dialects, gain some traction in the Middle Ages, \ili{CA} remains the most important written {register} for official, religious and scientific purposes. 

From the moment of \ili{CA}’s rise to dominance as a written language, the whole of the \ili{Arabic}-speaking world can be thought of as having transitioned into a state of {diglossia} \citep{Ferguson1959,Ferguson1996}, where \ili{CA} takes up the High {register} and the spoken dialects the Low {register}.\footnote{Diglossic situations are often seen as consisting of a high {register} (often called H) and a low {register} (L). These two are seen to be in complementary distribution, where each {register} is used in designated environments, where the H {register} takes up such domains like formal speeches and writing, while the L {register} is used in personal conversation, oral literature etc.} Representation in writing of these spoken dialects is (almost) completely absent in the written record for much of the Middle Ages. Eventually, \ili{CA} came to be largely replaced for administrative purposes by \ili{Ottoman} {Turkish}, and at the beginning of the nineteenth century, it was functionally limited to religious domains \citep[836]{Glaß2011}. During the nineteenth-century \ili{Arabic} literary revival known as the \textit{Nahḍa}, \ili{CA} goes through a rather amorphous and decentralized phase of modernization, introducing many neologisms for modern technologies and concepts, and many new syntagms became part of modern writing, often calqued upon European languages. After this period, it is customary in scholarly circles to speak of Classical \ili{Arabic} having transitioned into Modern Standard \ili{Arabic} (MSA), despite the insistence of its authors that \ili{CA} and MSA are one and the same language: \textit{al-ʕarabiyya} \textit{l-fuṣḥā} ‘the most eloquent \ili{Arabic} language’ \citep[845]{Ryding2011}.

\section{Contact languages}


Considering the significant time-depth of \ili{CA} and MSA, contact languages have of course changed over time. Important sources of linguistic contact of the pre-Islamic varieties of \ili{Arabic} that come to form the vocabulary for \ili{CA} are \ili{Aramaic}, \ili{Greek} and \ili{Ethio-Semitic}. While there are already some \ili{Persian} {loanwords} in the very first sources of \ili{CA}, this influence continues well into the Classical period, and ends up having a marked effect on \ili{CA} and MSA alike.

\subsection{Aramaic}
\ili{Aramaic} becomes the dominant lingua franca in much of the Achaemenid empire, and both written and spoken varieties of \ili{Aramaic} continue to play an essential role all throughout Arabia, Syria and Mesopotamia right up until the dawn of Islam. As such, a not insignificant amount of vocabulary has been borrowed from \ili{Aramaic} into \ili{Arabic}, which shows up in Classical \ili{Arabic}. Moreover, \ili{Aramaic} was an important language of Christianity and Judaism, and a noticeable amount of religious vocabulary from \ili{Aramaic} has entered Classical \ili{Arabic} (§\ref{bkm:Ref13224460}). There may even be some structural influence on the phonology of pre-Classical \ili{Arabic} that has made it into \ili{Arabic} (§\ref{bkm:Ref12953419}).

\subsection{Greek}

\ili{Greek} was the language of state of the Byzantine Empire, which, when not directly ruling over \ili{Arabic}-speaking populations, was at least in close contact with them. This can be seen in the significant amount of \ili{Greek} vocabulary that can be detected in \ili{CA}. \ili{Aramaic}, however, has often borrowed the same terms that we find in \ili{CA}, and it is usually difficult, if not impossible, to decide whether a \ili{Greek} word entered \ili{Arabic} directly from \ili{Greek} or through the intermediary of \ili{Aramaic} (§\ref{bkm:Ref13224479}).

\subsection{Persian}

After the rise of Islam, \ili{Greek} and \ili{Aramaic} quickly lose the central role they once played in the region, and they do not continue to influence \ili{CA} significantly in the Islamic period. \ili{Persian}, however, of which a number of words can already be detected in the {Quran}, continues to have a pronounced influence on \ili{Arabic}, and many more \ili{Persian} words enter \ili{CA} throughout its history (§\ref{bkm:Ref13224492}).

\subsection{Ethio-Semitic and Ancient South Arabian}

It is widely recognized that some degree of influence from Ethio-\ili{Semitic} can be identified within \ili{CA} (§\ref{bkm:Ref13224664}; \ref{bkm:Ref13224677}). Many of the Ethio-\ili{Semitic} words that have entered into Quranic \ili{Arabic} presumably arrived there through \ili{South Arabian} contact after the invasion of {Yemen} by Christian Ethiopia in the sixth century. Also previous \ili{South Arabian} contact must probably be assumed, and the divine epithet \textit{ar-Raḥmān} is usually thought to be a borrowing from \ili{South Arabian}, where it in turn is probably a borrowing from \ili{Aramaic} (\citealt{Jeffrey2007} [1938]: 140--141). 

While Ethio-\ili{Semitic} contact has been fairly well-researched, research into contact with Ancient South Arabia is still in its infancy. The exact classification of the \ili{Old South Arabian} languages and their relation to Modern \ili{South Arabian} and Ethio-\ili{Semitic} is still very much under debate. A simple understanding of this highly multilingual region seems impossible. Due to the extensive contact within South Arabia and the \ili{South Arabian} languages, it is not always easy to pin down the exact vector of contact between \ili{CA} and these languages of South Arabia and Ethiopia (§\ref{bkm:Ref13224682}).

\subsection{\label{bkm:Ref13224768}Arabic dialects}

The spoken \ili{Arabic} dialects, of course, have had and continue to have a noticeable influence on \ili{CA} and MSA (§\ref{bkm:Ref13224768}; \ref{bkm:Ref13224791}; \ref{bkm:Ref13224796}; \ref{bkm:Ref13224809}; \ref{bkm:Ref13224870}). It seems that from the very moment \ili{CA} became canonized as an official language, it was already a highly artificial {register} that nobody spoke in the form in which it was canonized. Especially the \ili{Ḥiǧāzi} conquerors had a noticeable effect on the language – no doubt through mediation of the Quranic text. Noticeable irregularities in the treatment of the glottal stop, for example, have entered the language, and have influenced the treatment of certain morphological features (§\ref{bkm:Ref13577493}).

\subsection{Ottoman Turkish}

In the Ottoman period, \ili{Ottoman} {Turkish} becomes the official language in use in the Middle East, and replaces many of the sociolinguistic functions that \ili{CA} had previously had. The {imposition} of this official language had a significant effect on the \ili{Arabic} vernaculars throughout the Middle East (even outside the borders of the Ottoman Empire), but also had a noticeable impact on the vocabulary of \ili{CA}, especially in the eighteenth and nineteenth centuries, which feeds into MSA (§\ref{bkm:Ref13483797}).

\section{Contact-induced changes in Classical and Modern Standard Arabic}
\subsection{\label{bkm:Ref12953419}Phonology}

Due to the highly conservative nature of \ili{CA}, finding any obvious traces of contact in phonological change is very difficult. From the period in which Sibawayh describes the phonology of the \textit{ʕarabiyya} until today, only minor changes have taken place in the phonology of \ili{CA}. The most obvious example of this is the loss of the lateral realization of the \textit{ḍād}, which in Sibawayh’s description is still a lateral, while today it is generally pronounced as [dˤ]. Blau (\citeyear[162--163]{Blau1969}) convincingly attributes this development to influence from the modern dialects. In most modern \ili{Arabic} dialects, the reflexes of \textit{ḍ} [ɮˤ] and \textit{ð̣} [ðˤ] {merged} to \textit{ð̣} [ðˤ].\footnote{Not all dialects, however, see Behnstedt (\citeyear[16ff.]{Behnstedt2016Yemen}).} In sedentary dialects that lose the interdentals, this {merged} sound subsequently shifts to \textit{ḍ} [dˤ]. As such, original \textit{ð̣} \textit{and} \textit{ḍ} are either both pronounced as an {emphatic} interdental fricative or both as an {emphatic} dental stop. As virtually all modern dialects, however, have lost the lateral realization of \textit{ḍ}, the sedentary stop realization was repurposed for the realization of \textit{ḍ}, to introduce the phonemic distinction between \textit{ð̣} and \textit{ḍ} in MSA.

As this is a case where the speakers influencing the phonology of the {RL} are SL-dominant, this change of pronunciation of the \textit{ḍ} from a lateral to a stop realization can be seen as a form of {imposition} on the phonology of MSA. It should be noted, however, that the type of {imposition} we are dealing with in this case is of quite a different character than what is traditionally understood as {imposition} within the framework of Van Coetsem (\citeyear{VanCoetsem1988,VanCoetsem2000}). In this case, we see a conscious effort to introduce a phonemic distinction lost in the {SL} between original \textit{ḍ} and \textit{ð̣} by using two different dialectal outcomes of the {merger} of these two phonemes.

Other cases of phonetic {imposition} on MSA from the modern dialects may especially be found in the realization of the \textit{ǧīm}. While Sibawayh’s description of the \textit{ǧīm} was probably a palatal stop [ɟ], today the realization that seems to carry the most {prestige} and is generally adhered to in Quranic recitation is [ʤ]. However, here too we often find {imposition} of the local pronunciation of this sound in MSA. In spoken MSA of Egyptians the \textit{ǧīm} is regularly pronounced as [ɡ], the realization of the \textit{ǧīm} in \ili{Egyptian Arabic}. Likewise, \ili{Levantine} \ili{Arabic} speakers whose reflex of the \textit{ǧīm} is [ʒ] will often use that realization when speaking MSA.

If we shift our focus to developments that began in the pre-Classical period and continue in \ili{CA}, we find that there are several phonetic developments that bear some similarity to developments of \ili{Aramaic}. It has therefore, not unreasonably, been suggested that such developments are the result of contact with \ili{Aramaic}.

The first of these similar phonetic developments shared between Classical \ili{Arabic} and \ili{Aramaic} is the shift of the semivowels  \textit{w} and \textit{y} to \textit{ʔ} between a preceding \textit{ā} and a following short vowel \textit{i} or \textit{u}. This can be seen, for example, in the similar outcomes of the active participles of hollow {roots}. This similarity was already remarked upon and described by Brockelmann (\citeyear[138--139]{Brockelmann1908}), e.g.:

\ea
\ea \ili{CA} *qāwimun > \textit{qāʔimun} ‘standing’\\
\ex \ili{Aram.} *qāwim > \textit{qāʔem} ‘standing’
\z
\z

However, it is clear that, at least in \ili{Nabataean} \ili{Arabic}, this development had not yet taken place \citep[91--93]{Diem1980}, this being a dialect that was certainly in contact with \ili{Aramaic}, as most of the writing of the Nabataeans was in a form of \ili{Aramaic}. As such, we may plausibly suggest that this development took place \textit{after} the establishment of linguistic contact between \ili{Aramaic} and \ili{Arabic}. It is quite difficult to decide whether this development, if we are correct to interpret it as the result of contact-induced change, is the result of {imposition}, borrowing or {convergence}. We do not have a clear enough picture of the sociolinguistic relations between \ili{Aramaic} and pre-Classical \ili{Arabic} to identify the type of contact situation that would have caused it. One is tempted to see it as the result of {imposition} simply because of the fact that phonological borrowing seems to be uncommon \citep[526]{Lucas2015}.\footnote{We cannot discount the possibility of parallel development, however. \ili{Akkadian} seems to have undergone an almost identical development \citep[196]{Huehnergard1997}, where it is not likely to have been the result of contact.} 

As proposed by Al-Jallad (this volume),\ia{Al-Jallad, Ahmad@Al-Jallad, Ahmad} another possible case of contact induced phonological change between \ili{Aramaic} and pre-Classical \ili{Arabic} is the shift of pausal \textit{-at} to \textit{ah}, found only in nouns and not in verbs. Huehnergard \& Rubin (\citeyear[267--268]{HuehnergardRubin2011}) already suggested that this development, which cannot be due to a development in a shared ancestor, may have been the result of areal {diffusion}.

Whether we can really interpret the development of \ili{Aramaic} as similar to that of \ili{CA}, however, depends somewhat on the interpretation of the \ili{Aramaic} evidence. While we can indeed see a development of the original \ili{Aramaic} feminine ending \textit{*-at} that is written with 〈-h〉 in consonantal writing, which might suggest it has shifted to /ah/, one also finds that all other cases of word-final nominal \textit{t} have been lost, while not leaving a consonantal \textit{-h}, e.g.: 

\ea
\ea *ṣalōt > \textit{ṣlō} 〈ṣlw〉 ‘prayer’\\
\ex *zakūt > \textit{zkū} 〈zkw〉 ‘merit, victory’\\
\ex *ešāt > \textit{ʔešā} 〈ʔšʔ〉 ‘fire’\\
\ex *bayt > \textit{bay}  〈by〉 ‘house’
\z
\z

For this parallel loss of final \textit{t} in all other environments, Beyer (\citeyear{Beyer1984}: 96, fn. 4) prefers to interpret the 〈-h〉 as a \textit{mater} \textit{lectionis} for final /ā/ or /a/. In this interpretation, the development of \ili{Aramaic} compared to \ili{Arabic} is quite different, since in \ili{Arabic} the 〈-h〉 is clearly consonantal, and the loss of final \textit{t} does not happen after long vowels in \ili{Arabic}:


\ea {Aramaic} \\
\ea *kalbat > \textit{kalbā} 〈klb〉 ‘bitch’ (\textit{-at\# > -a/ā})\\
\ex *ʔešāt > \textit{ʔešā}   〈ʔšʔ〉 ‘fire’ (\textit{-āt\# > -ā})\\
\z
\z

\ea {Arabic} \\
\ea *kalbat > \textit{kalbah}\\
\ex *kalbāt > \textit{kalbāt} remains unchanged \\
\z
\z

However, if one takes the 〈-h〉 of the feminine to originally represent *-at > \textit{-ah}, and the loss of \textit{t} in other word-final positions to be a different development, one could reasonably attribute the development in \ili{Arabic} to the result of contact with \ili{Aramaic}, as it is clear that in many varieties of pre-Islamic \ili{Arabic}, the *-at > \textit{-ah} shift had not yet taken place.\footnote{For a discussion on the development of the *-at > \textit{-ah} shift in pre-Islamic \ili{Arabic} see Al-Jallad (\citeyear[157--158]{Al-Jallad2017Greek}).}

\subsection{\label{bkm:Ref13577493}Morphology}
\subsubsection{\label{bkm:Ref13224791}Imposition of the taCCiʔah stem II verbal noun for glottal-stop-final verbs}

A well-known feature of Ḥiǧāzī \ili{Arabic} in the early Islamic period, and a feature that is found in many of the modern dialects is the (almost) complete loss of the glottal stop (\citealt{Rabin1951}: 130--131; \citealt{vanPutten2018}). This loss has usually caused glottal-stop-final {roots} to be reanalyzed as final-weak verbs, e.g. \ili{Cairene} \textit{ʔara,} \textit{ʔarēt} ‘he read, I read’ (< \textit{*}qaraʔa, *qaraʔtu).

A typical feature of final-weak verbal noun formations in \ili{CA} is their {formation} of the verbal noun of {stem} II verbs. Sound verbs form verbal nouns using the pattern taCCīC, e.g. \textit{taslīm} ‘greeting’ from \textit{sallama} ‘to greet’. Final-weak verbs, however, regularly use the pattern taCCiyah instead \citep[44]{Fischer2002}, for example, \textit{tasmiyah} ‘naming’ from \textit{sammā} ‘to name’.\footnote{This is an ancient idiosyncrasy of final-weak verbs. While the taCCīC {formation} is not a regular {formation} in other \ili{Semitic} languages, when it does occur, the final-weak verbs have a feminine ending, e.g. \ili{Hebrew} \textit{tarmi-ṯ} ‘betrayal’, \textit{toḏå} ‘praise’ (< *tawdiy-ah or *tawday-ah), see Brockelmann (\citeyear[385--387]{Brockelmann1908}).} 

In \ili{CA}, the \textit{ʔ} generally functions as a regular consonant. Thus a verb like \textit{qaraʔa/yaqraʔu} ‘to read, recite’ does not differ significantly in its behavior from any other {triconsonantal} verb such as \textit{fataḥa/yaftaḥu} ‘to open’. 

However, verbs with \textit{ʔ} as final {root} consonants unexpectedly frequently side with the final-weak verbs when it comes to the verbal noun of {stem} II verbs \citep[128]{Fischer2002}. For example, \textit{hannaʔa/yuhannaʔu} ‘to congratulate’ does not have the expected verbal noun **tahnīʔ, but instead \textit{tahniʔah} ‘congratulation’. Other examples are:

\ea
\ea\textit{nabbaʔa} v.n. \textit{tanbiʔah} (besides \textit{tanbīʔ}) ‘to inform’\\
\ex\textit{barraʔa} v.n. \textit{tabriʔah} ‘to acquit’\\
\ex\textit{hayyaʔa} v.n. \textit{tahyiʔah} (besides \textit{tahyīʔ}) ‘to make ready’\\
\ex\textit{naššaʔa}, v.n. \textit{tanšiʔah} (besides \textit{tanšīʔ}) ‘to raise (a child)’
\z
\z

Some other verbs with the same pattern do have the expected \ili{CA} form such as \textit{baṭṭaʔa} v.n. \textit{tabṭīʔ} ‘to delay’.

This behaviour can plausibly be attributed to the fact that in many (if not most) spoken varieties of \ili{Arabic}, from early on the final-glottal-stop verbs had already {merged} completely with the final-weak verbs, and as such a verb like \textit{hannaʔa} had come to be pronounced as \textit{hannā}, and thus reanalyzed as a final-weak verb. Like original final-weak verbs, their regular verbal noun {formation} would be like \textit{tahniyah}. When verbs of this type were employed in \ili{CA}, the weak {root} consonant \textit{y} was replaced with the etymological glottal stop \textit{ʔ}, rather than completely converting the verbal noun to the regular pattern. This is a clear example of the {imposition} of a morphological pattern onto \ili{CA} grammar by speakers of \ili{Arabic} dialects.

\subsubsection{\label{bkm:Ref13224796}Imposition of the ʔaCCiyāʔ broken plural pattern}

A similar case of {imposition}, where the morphological categories of glottal-stop-final {roots} behave in the grammar as if they are final-weak, may be found in the broken-plural {formation} of CaCīʔ nouns and adjectives. The broken-plural {formation} most generally used for final-weak adjectives with the pattern CaCiyy (< *CaCīy) is ʔaCCiyāʔ. For example, \textit{ɣaniyy} pl. \textit{ʔaɣniyāʔ} ‘rich’, \textit{waliyy} pl. \textit{ʔawliyāʔ} ‘close associate’, \textit{daʕiyy} pl. \textit{ʔadʕiyāʔ} ‘bastard’, \textit{sawiyy} pl. \textit{ʔaswiyāʔ} ‘correct’, \textit{ḫaliyy} pl. \textit{ʔaḫliyāʔ} ‘free’.

For sound nouns of this type, it is much more typical to use the plural formations CiCāC (\textit{kabīr} pl. \textit{kibār} ‘big’) or CuCaCāʔ (\textit{faqīr} pl. \textit{fuqarāʔ} ‘poor’), although there are a couple of sound nouns that do use this plural, such as \textit{qarīb} pl. \textit{ʔaqribāʔ} ‘{relative}’ and \textit{ṣadīq} pl. \textit{ʔaṣdiqāʔ} ‘friend’ \citep[106--107]{Ratcliffe1998}.\footnote{The pattern (with metathesis) is also regular for geminated CaCīC adjectives, e.g. \textit{šadīd}  pl. \textit{ʔašiddāʔ} ‘severe’.}

CaCīC formations where the last {root} consonant is \textit{ʔ}, however, behave in rather unexpected ways in \ili{CA}, usually following the pattern of final-weak nouns, often even replacing the final \textit{ʔ} with \textit{y}, for example: \textit{barīʔ} pl. \textit{ʔabriyāʔ} ‘free’, \textit{radīʔ} pl. \textit{ʔardiyāʔ} ‘bad’. These nouns have plurals that are proper not to the Classical form they have, but rather to the colloquial form without \textit{ʔ}, i.e. \textit{bariyy}, \textit{radiyy}. Once again this can be seen as a clear case of {imposition} of the colloquial \ili{Arabic} forms onto the classical language.\footnote{These two cases of {imposition} of glottal stop-less morphology onto \ili{CA} are two of the more clear and systematic cases, but a close observation of \ili{CA} morphology reveals many more of these somewhat more isolated cases, e.g. \textit{ḫaṭīʔah} ‘sin’ with a plural \textit{ḫaṭāyā}, for which the expected singular would rather be \textit{ḫaṭiyyah}; \textit{bariyyah} pl. \textit{barāyā} ‘creature’ which is a {derivation} from \textit{baraʔa} ‘to create’; \textit{ðurriyyah}, \textit{ðirriyyah} pl. \textit{ðarāriyy} ‘progeny, offspring’, derived from \textit{ðaraʔa} ‘to sow, seed’. Another example of irregular treatment of \textit{ʔ} that is presumably the result of impositition is found in verbal nouns of {stem} VI verbs, and \textit{mafāʕil} plurals of hollow {roots}, which modern textbooks say should not have a \textit{ʔ} despite having the environment that is expected to undergo the shift \textit{āwu/i,} \textit{āyi} > \textit{āʔu/i,} \textit{āʔi} as discussed in §\ref{bkm:Ref12953419}. The lexicographical tradition and Quranic reading traditions often record disagreements on the application of the \textit{hamzah} in such cases. For example, we find both \textit{tanāwuš} and \textit{tanāʔuš} ‘reaching one another’, and \textit{maʕāyiš} and \textit{maʕāʔiš} ‘ways of living’.}

\subsubsection{\label{bkm:Ref13224664}Borrowing of the broken plural pattern CaCāCiCah}

\ili{CA}, like the modern \ili{Arabic} dialects, is well-known for its broken-plural patterns. This is a feature it shares especially with \ili{ Old South Arabian} \citep[1050--1051]{Stein2011}, \ili{Modern South Arabian} languages \citep[1085]{Simeone-Senelle2011} and \ili{Ethio-Semitic} \citep[1132]{Weninger2011OldEth}. The use of broken plurals has caused somewhat of a controversy in the subgrouping of the \ili{Semitic} language family. Scholars who consider broken plurals a shared retention do not view their presence as important for grouping \ili{Arabic}, \ili{Old South Arabian}, the \ili{Modern South Arabian} languages and \ili{Ethio-Semitic} together \citep[159--160]{Huehnergard2005}; while those who consider their presence an innovation in a subset of \ili{Semitic} languages see this as a strong indication that these languages should be grouped together into a South \ili{Semitic} branch (e.g. \citealt{Ratcliffe1998}).

While most scholars today seem to agree that the broken-plural system is a shared retention \citep[1116]{Weninger2011GenEth}, it seems clear that the retention of a highly productive broken-plural system is to be considered an areal feature that clusters around South Arabia and the Horn of Africa. Classical \ili{Arabic} partakes in this areal feature.

A possible case of influence from \ili{Old South Arabian} (and/or \ili{Ethio-Semitic}) into \ili{Arabic} is the introduction of the CaCāCiCah plural {formation}. In the \ili{South Arabian} languages,\footnote{\ili{South Arabian} is used here as a purely geographical descriptive term, not one of classification.} the equivalent plural {formation} CaCāCiCt is extremely productive, and numerous words with four consonants form their plural in this way. For example in Sabaic, mCCCt is the regular plural {formation} to mCCC nouns  of location, e.g. \textit{mḥfd} pl. \textit{mḥfdt} ‘tower’ \citep[34]{Beeston1962}. It is likewise common in \ili{Gəʕəz}, e.g. \textit{tänbäl} pl. \textit{tänabəlt} ‘ambassador’ (\citealt{Dillmann2005} [1907]: 309), and occurs occasionally in Modern \ili{South Arabian}, e.g. \ili{Mehri} \textit{məlēk} pl. \textit{məlaykət} ‘angel’ \citep[68]{Rubin2010}.

While this pattern exists in \ili{CA}, it is much rarer than the other {broken plural} formations of four consonantal forms, i.e. CaCāCiC and CaCāCīC. In the {Quran}, \textit{malak} pl. \textit{malāʔikah} ‘angel’ is the only plural with this pattern. This noun is widely recognized as being a {loanword} from \ili{Gəʕəz} \textit{malʔak,} \textit{malāʔəkt} (\citealt{Jeffrey2007} [1938]: 269), in part on the basis that it shares this plural {formation}: the word seems to have been borrowed together with its plural {formation}. Considering the rarity of this pattern in \ili{Arabic} and how common it is in \ili{South Arabian}, it seems possible that the pattern was introduced into \ili{Arabic} through \ili{South Arabian} contact. However, the absence of other clearly identifiable \ili{South Arabian} {loanwords} with this plural pattern makes it rather difficult to make a strong case for this identification.

Another possible word of \ili{South Arabian} origin with this plural pattern is \textit{tubbaʕ} pl. \textit{tabābiʕah} ‘a Yemenite king’, but evidence that this word is indeed of \ili{Old South Arabian} origin is missing. The word does not occur as a separate word in \ili{Old South Arabian}, and instead is only the first part of several \ili{Old South Arabian} theophoric names such as \textit{tbʕkrb}, \textit{tbʔʕl}. Such names should probably be understood as being related to the {root} \textit{√tbʕ} which, like in \ili{Arabic}, may have had the meaning ‘following’, so such names likely mean ‘follower of the deity KRB’ and ‘follower of the deity ʔL’. Such names being associated with Yemenite kings may have led to the \ili{Arabic} meaning of \textit{tubbaʕ} as ‘Yemenite king’, but in \ili{Old South Arabian} itself it does not seem to have carried a meaning of this kind. 

All in all, the evidence for this really being a pattern that is the result of \ili{South Arabian} influence is rather slim, although the rarity of the pattern in \ili{CA} does make it look unusual. If the interpretation of this plural pattern as being a borrowing from \ili{South Arabian} is correct, it seems that some \ili{South Arabian} nouns were borrowed along with their respective plural. This would be a case of morphological borrowing rather than the more common type of morphological influence through {imposition}.\footnote{This can be seen as a type of “Parallel System Borrowing” similar to that which we find in \ili{Berber} languages. \ili{Berber} languages, like \ili{Arabic}, have apophonic plurals; but \ili{Arabic} nouns are simply borrowed along with their own \ili{Arabic} broken plurals \citep{Kossmann2010}.} 

Note that this plural pattern has become the productive plural pattern for quadriconsonantal {loanwords} regardless of them being of \ili{South Arabian} origin or elsewhere, e.g. \textit{biṭrīq} pl. \textit{baṭāriqah} ‘patrician’ (< \ili{Latin} \textit{patricius}), \textit{ʔusquf} pl. \textit{ʔasāqifah} ‘bishop’ (< \ili{Greek} \textit{epískopos}), \textit{ʔustāð} pl. \textit{ʔasātiðah} ‘master’ (< Middle \ili{Persian} \textit{ōstād}), \textit{tilmīð} pl. \textit{talāmiðah} ‘student’ (< \ili{Aramaic} \textit{talmīḏ}).

\subsection{\label{bkm:Ref13224809}Syntax}

Due to \ili{CA} being the High {register} in a diglossic situation for centuries, we should presumably consider the majority of the written material produced in this language to be written exclusively by non-native speakers. Moreover, a large proportion of its writers all throughout its written history must have been speakers not only of \ili{Arabic} vernaculars but also of entirely different languages such as \ili{Persian} and \ili{Turkish}. It seems highly unlikely that such a multilingual background of authors of Classical \ili{Arabic} would have been completely without effect on the syntax of the language; however, as it is difficult to decide from what moment onward we can speak of true {diglossia}, and what the syntax was like before that period, it has not yet been possible to trace such influences in detail.

There is, however, promising research being done on influence on MSA syntax from the speakers of modern \ili{Arabic} dialects. \citet{Wilmsen2010} convincingly describes one such point of influence in a paper on the treatment of object pronouns in Egyptian and \ili{Levantine} newspapers.

Wilmsen (\citeyear[104]{Wilmsen2010}) shows that, in the case of ditransitive verbs, Egyptian and \ili{Levantine} have a different natural {word order}. In \ili{Egyptian Arabic}, the direct object must precede the indirect object as in (\ref{bkm:Ref533757608}), while in \ili{Levantine} \ili{Arabic} the indirect object preceding the direct object is preferred, as shown in (\ref{bkm:Ref533759667}):

%%1st subexample: change \ea\label{...} to \ea\label{...}\ea; remove \z  
%%further subexamples: change \ea to \ex; remove \z  
%%last subexample: change \z to \z\z 
\ea
\label{bkm:Ref533757608}Egyptian\\
\gll rabbi-na yḫalli\textbf{{}-hū-l-ak}\\
     Lord-\textsc{1pl} keep\textsc{.impf.3sg.m}{}-3\textsc{sg.m-dat-2sg.m}\\
\glt ‘Our lord keep him for you.’
\z

\ea\label{bkm:Ref533759667}\ili{Levantine}\\
\gll aḷḷa yḫallī-l-ak iyyā\\
     God keep\textsc{.impf.3sg.m-dat-2sg.m} \textsc{acc.}3\textsc{sg.m} \\
\glt ‘God keep him for you.’\z

Wilmsen argues that the following two variant sentences in a Reuters news story written in MSA, the original in (\ref{bkm:Ref533761403}), likely written by an Egyptian, and the slightly altered version in (\ref{bkm:Ref533761438}), which appeared in a Lebanese newspaper, show exactly this difference of {word order} found in the respective spoken dialects:

\ea
\label{bkm:Ref533761403}MSA (Egyptian)\\
\gll al-ʔawrāq-i llatī \textbf{sallamat-hā} \textbf{la-hu} ʔarmalat-u ʕabdi l-wahhāb\\
     \textsc{def}{}-papers-\textsc{obl} \textsc{rel.sg.f} give.\textsc{prf.3sg.f-3sg.f} \textsc{dat-3sg.m} widow-\textsc{nom} \textsc{pn}\\
     \glt ‘the papers, which Abdel Wahhab’s widow had \textbf{given} \textbf{him}’
\z

\protectedex{
\ea\label{bkm:Ref533761438}MSA (Lebanese)\\
\gll al-ʔawrāq-i llatī \textbf{sallamat-hu} \textbf{ʔiyyā-hā} ʔarmalat-u ʕabdi l-wahhāb\\
     \textsc{def}{}-papers-\textsc{obl} \textsc{rel.sg.f} give.\textsc{prf.3sg.f-3sg.m} \textsc{acc-3sg.f} widow-\textsc{nom} \textsc{pn}\\
\glt ‘the papers, which Abdel Wahhab’s widow had \textbf{given} \textbf{him}’\z}

Wilmsen (\citeyear[114--115]{Wilmsen2010}) goes on to examine three newspapers (the London-based, largely Lebanese, \textit{al-Ḥayāt} of the years 1996--1997; the Syrian \textit{al-Θawra} of the year 2005 and the Egyptian \textit{al-ʔAhrām}), and shows that with the two most common verbs in the corpus with such argument structure (\textit{manaḥa} ‘to grant’ and \textit{ʔaʕṭā} ‘to give’), the trend is consistently in favour of the pattern found. The recipient–theme order is overwhelmingly favoured in the \ili{Levantine} newspapers, while the theme–recipient order is clearly favoured by the Egyptian newspaper. The results are reproduced in Tables~\ref{tab:vanputten:1} and~\ref{tab:vanputten:2}.


\begin{table}
\begin{tabular}{lrr}
\lsptoprule
{Database} & {theme–recipient} & {recipient–theme}\\\midrule
al-Ḥayāt 96 & 29 & 56\\
al-Ḥayāt 97 & 27 & 52\\
al-Θawra & 27 & 66\\
al-ʔAhrām & 44 & \hphantom{1}8\\
\lspbottomrule
\end{tabular}
\caption{\label{bkm:Ref533762442}\label{tab:vanputten:1}Occurences of theme–recipient and recipient–theme order with \textit{manaḥa} ‘to grant’}
\end{table}

\begin{table}
\begin{tabular}{lrr}
\lsptoprule
{Database} & {theme–recipient} & {recipient–theme}\\\midrule
al-Ḥayāt 96 & 11 & 23\\
al-Ḥayāt 97 & \hphantom{1}8 & 22\\
al-Θawra    &  \hphantom{1}9 & 38\\
al-ʔAhrām   & 33 &  \hphantom{1}2\\
\lspbottomrule
\end{tabular}
\caption{\label{bkm:Ref533762457}\label{tab:vanputten:2}Occurrences of theme–recipient and recipient–theme order with \textit{ʔaʕṭā} ‘to give’}
\end{table}

From this data it is clear that the dialectal background of the author of an MSA text can indeed play a role in how its syntax is constructed, despite both resulting sentences being grammatically acceptable in \ili{CA}/MSA.\footnote{Other works that discuss clear cases of country-specific language use of MSA include \citet{Ibrahim2009}, \citet{Parkinson2003},  \citet{Parkinson2007} and \citet{ParkinsonIbrahim1999}.}

This (and any contact phenomenon in MSA–dialect {diglossia}) should be seen as a case of {imposition}, where the dialect {SL}, in which the speakers/writers are dominant, has influenced the MSA {RL}. 

It stands to reason that such syntactic research could be undertaken with \ili{CA} works as well. Taking into account the biographies of authors, it might be possible to find similar {imposition} effects that can be connected to different dialects and languages in former times. To my knowledge, however, this work has yet to be undertaken.

\subsection{Lexicon}

In terms of lexicon, Jeffery’s indispensable (\citeyear{Jeffrey2007} [1938]) study of the foreign vocabulary in the {Quran} allows us to examine some of the important sources of lexical influence on pre-Classical \ili{Arabic}. Influence from \ili{Greek}, \ili{Aramaic}, \ili{Gəʕəz} and \ili{Persian} are all readily recognizable. 

\subsubsection{\label{bkm:Ref13224677}Gəʕəz}

\citet{Nöldeke1910} is still one of the most complete and important discussions of \ili{Gəʕəz} {loanwords} in \ili{CA}. Both \ili{Gəʕəz} and \ili{Arabic} display a significant amount of religious vocabulary that is borrowed from \ili{Aramaic}. It is quite often impossible to tell whether \ili{Arabic} borrowed the word from \ili{Gəʕəz} or from \ili{Aramaic}. Such examples are \textit{ṭāɣūt} ‘idol’, \ili{Gz.} \textit{ṭaʕot}, \ili{Aram.} \textit{ṭāʕū} ‘error, idol’ \citep[48]{Nöldeke1910}; \textit{tābūt} ‘ark; chest’, \ili{Gz.} \textit{tabot} ‘ark of Noah, ark of the covenant’, \ili{Aram.} \textit{tēḇō} ‘chest; ark’ \citep[49]{Nöldeke1910}.

There is religious vocabulary that is unambiguously borrowed from \ili{Gəʕəz}, e.g. \textit{ḥawāriyyūn} ‘disciples’ < \ili{Gz.} \textit{ḥäwarəya} ‘apostle’ and \textit{muṣḥaf} ‘book (esp. {Quran})’ < \ili{Gz.} \textit{mäṣḥäf} ‘scripture’, but there is also religious vocabulary borrowed unambiguously from \ili{Aramaic}, e.g. \textit{zakāt} ‘alms’ < \ili{Aram.} \textit{zāḵū} ‘merit, victory’; \textit{sifr} ‘large book’ < \ili{Aram.} \textit{sp̄ar, sep̄rā}. It is therefore just as likely that \ili{Arabic} would have borrowed such \ili{Aramaic} {loanwords} via \ili{Gəʕəz}, as directly from \ili{Aramaic}.

Some religious vocabulary from \ili{Aramaic} and \ili{Hebrew} can be shown to have arrived in \ili{Arabic} through contact with \ili{Gəʕəz}, since these words have undergone specific phonetic developments shared between \ili{CA} and \ili{Gəʕəz} but absent in the {source language}. As these often involve core religious vocabulary, and the Christian Axumite kingdom was established centuries before Islam, it seems reasonable to assume such words to be borrowings from \ili{Gəʕəz} into \ili{CA}, e.g. \ili{CA} \textit{ǧahannam} ‘hell’ < \ili{Gz.} \textit{gähännäm} (but \ili{Hebrew} \textit{gehinnom} and \ili{Syriac} \textit{gehannā}) and \ili{CA} \textit{šayṭān} ‘Satan’ < \ili{Gz.} \textit{śäyṭan} (but \ili{Hebrew} \textit{śåṭån} and \ili{Syriac} \textit{sāṭānā}).\footnote{\citet{Leslau1990} often reverses the directionality of such borrowings, though without an explanation as to why he thinks a borrowing from \ili{CA} into \ili{Gəʕəz} is more likely.}

\subsubsection{\label{bkm:Ref13224460}Aramaic}

As already remarked upon by \citet{Retsö2011}, \ili{Aramaic} {loanwords} in Classical \ili{Arabic} often have an extremely archaic character. The \ili{Aramaic} variety that influenced Quranic and pre-Classical \ili{Arabic} had not undergone the famous \textit{bəḡaḏkəp̄aṯ} lenition of post-vocalic simple stops, nor had it lost short vowels in open syllables. This necessarily means that the form of \ili{Aramaic} that influenced Quranic and Classical \ili{Arabic}, even the religious vocabulary, cannot be \ili{Syriac}, which almost certainly underwent both shifts before becoming a dominant religious language. The \textit{bəḡaḏkəp̄aṯ} spirantization can be dated between the first and third centuries CE, and the syncope of short vowels in open syllables takes place sometime in the middle of the third century \citep[41--42]{Gzella2015}. However, Classical \ili{Syriac} itself, as an important vehicular language of Christianity, only emerges in the fourth century CE, well after these developments had taken place \citep[259]{Gzella2015}.

Had \textit{bəḡaḏkəp̄aṯ} taken place, we would expect \ili{Syr.} \textit{ḡ}, \textit{ḏ}, \textit{ḵ}, and \textit{ṯ} to be borrowed with their phonetic equivalents in \ili{CA}: \textit{ɣ}, \textit{ð}, \textit{ḫ}, and \textit{θ} respectively.\footnote{\cite{Retsö2011} suggests that \textit{ḇ} could also be borrowed as \textit{w}. This might be true, but at least the phonetic match in this case is not perfect.} This, however, is not the case; instead these consonants are consistently borrowed with the stop equivalents \textit{ǧ}, \textit{d}, \textit{k}, and \textit{t}, and without the loss of vowels in open syllables, clearly showing that these \ili{Aramaic} {loanwords} predate the phonetic developments in Classical \ili{Syriac}.

\ea
\ea\textit{malakūt} ‘kingdom’, \ili{Syr.} \textit{malḵūṯ-ā} ‘kingdom’ < *malakūt-ā\\
\ex\textit{malik} ‘king’, \ili{Syr.} \textit{mleḵ} ‘king’ < *malik\footnote{This word is not recognized as an \ili{Aramaic} {loanword} by Jeffery (\citeyear[270]{Jeffrey2007}), but it likely is. All the \ili{Semitic} cognates of this noun are derived from a form *malk, which should have been reflected in Classical \ili{Arabic} as \textit{malk}. However, we find it with an extra vowel between the last two {root} consonants. This can be best understood as the epenthetic vowel insertion as it is attested in \ili{Aramaic} which was then subsequently borrowed with this {epenthesis} into \ili{Arabic}. I thank Ahmad Al-Jallad for pointing this out to me.}\\
\ex\textit{masǧid} ‘place of worship, mosque’, \ili{Syr.} \textit{masgeḏ-ā} ‘place of worship’ < *masgid-ā
\z
\z

\noindent Even the proper names of Biblical figures have a markedly un-\ili{Syriac} form.

\ea
\ea\textit{zakariyā}, \textit{zakariyāʔ}, \ili{Syr.} \textit{Zḵaryā} \textit{<} *zakaryā\\
\ex\textit{mīkāʔīl,} \textit{mīkāʔil},\footnote{Most readers of the {Quran} read either \textit{mīkāʔīl} or \textit{mīkāʔil}, only the most dominant tradition today, that of Ḥafṣ, reads it in the highly unusual form \textit{mīkāl} \citep[166]{IbnMujahid}.} \ili{Syr.} \textit{mīḵāʔel} < *mīkāʔēl
\z
\z

In other words, far from \ili{Syriac} being “undoubtedly the most copious source of Qurʾānic borrowings” (\citealt{Jeffrey2007} [1938]: 19), the \ili{Aramaic} vocabulary in the {Quran} seems to not be \ili{Syriac} at all.\footnote{Note that Jeffery (\citeyear{Jeffrey2007} [1938]: 19) explicitly states that by \ili{Syriac} he means any form of Christian \ili{Aramaic}, so, besides \ili{Syriac}, most notably also \ili{Christian Palestinian Aramaic}. However, this caveat hardly solves the chronological problem, as the latter rises to prominence even later.} Any isogloss that would allow us to identify it as such is conspicuously absent. This has important historical implications, as the presence of supposed \ili{Syriac} religious vocabulary in the {Quran} is viewed as an important indication that \ili{Syriac} Christian thought had a pronounced influence on early Islam (e.g. \citealt[82--90]{Mingana1927}; \citealt{Jeffrey2007} [1938]: 19--22).\footnote{Even if we were to accept the possibility that the dating of the lenition and syncope is somehow off by several centuries, the suggestion that “it is possible that certain of the \ili{Syriac} words we find in the Qurʾān were introduced by Muḥammad himself” (\citealt{Jeffrey2007} [1938]: 22) must certainly be rejected. In the grammatical works of Jacob of Edessa (640–708 CE) we have an unambiguous description of the lenition of the consonants (Holger Gzella p.c.). It seems highly unlikely that a wholesale lenition took place in only a few decades between the composition of the {Quran} and the time of his writings.} While this is of course still a possibility, this has to be reconciled with the fact that the majority of clearly monotheistic religious vocabulary was already borrowed from a form of \ili{Aramaic} before the rise of \ili{Syriac} as a major religious language.

This does not mean that \ili{CA} is completely devoid of \ili{Aramaic} {loanwords} that have undergone the lenition of the consonants, and several post-Quranic {loanwords} have been borrowed from a variety which, like \ili{Syriac}, had lenited its stops, e.g.:

\ea
\ea\textit{tilmīð} ‘student’ < \ili{Syr.} \textit{talmīḏā}  \citep[254]{Fraenkel1886}\\
\ex\textit{tūθ,} \textit{tūt} ‘mulberry’ < \ili{Syr.} \textit{tūṯā}  \citep[140]{Fraenkel1886}\\
\ex\textit{ḥiltīθ,} \textit{ḥiltīt} ‘\textit{asa} \textit{foetida}’ < \ili{Syr.} \textit{ḥeltīṯā} \citep[140]{Fraenkel1886}\\
\ex\textit{kāmaḫ,} \textit{kāmiḫ} ‘vinegar sauce’ < \ili{Syr.} \textit{kāmḵā} \citep[288]{Fraenkel1886}\\
\ex\textit{karrāθ,} \textit{kurrāθ} ‘leek’ < \ili{Syr.} \textit{karrāṯā} \citep[144]{Fraenkel1886}
\z
\z

It is interesting to note that \ili{Aramaic} {loanwords} in \ili{Gəʕəz} reflect a similar archaicity, in those cases where this is detectable. The expected lenited \textit{ḵ} is not represented with \ili{Gəʕəz} \textit{ḫ} but with \textit{k}, and short vowels in open syllables are retained. This might suggest that, when looking for religious influences on Islam, we should rather shift our focus to the south, where during the centuries before Islam both Judaism and Christianity were introduced, presumably through the vector of \ili{Gəʕəz}. Some examples of such similarly archaic \ili{Aramaic} {loanwords} in \ili{Gəʕəz} are cited by Nöldeke (\citeyear[31--46]{Nöldeke1910}), e.g.:

\ea \ili{Gəʕəz}\\
\ea \textit{mälʔäk} ‘angel’, cf. \ili{CA} \textit{malak}, \ili{Syr.} \textit{malʔaḵ-ā} < *malʔak-ā\\
\ex \textit{mäläkot} ‘kingdom’, cf. \ili{CA} \textit{malakūt}, \ili{Syr.} \textit{malḵūṯ-ā} < *malakūt-ā\\
\ex \textit{ḥämelät} ‘mantle, headcloth’, \ili{Syr.} \textit{ḥmīlṯ-ā} < *ḥamīlat-ā\\
\ex \textit{näbīy} ‘prophet’, cf. \ili{CA} \textit{nabiyy}, \ili{Syr.} \textit{nḇīʔ{}-ā} < *nabīʔ-ā\\
\ex \textit{mäsīḥ} ‘Messiah’, cf. \ili{CA} \textit{al-masīḥ}, \ili{Syr.} \textit{mšīḥ-ā} < *masīḥ-ā\\
\ex \textit{siʔol} ‘hell’, cf. \ili{Syr.} \textit{siwūl} < *siʔūl (cf. Hebr. \textit{səʔol})\\
\ex \textit{ʔärämi,} \textit{ʔärämāwi,} \textit{ʔärämay} ‘heathen’, cf. \ili{Syr.} \textit{ʔarmāy-ā} < *ʔaramāy-ā\\
\ex \textit{mänarät,} \textit{mänarat} ‘candlestick’, cf. \ili{CA} \textit{manārah}, \ili{Syr.} \textit{mnārṯ-ā} < *manārat-ā
\z
\z

As of yet, there is not a clear historical scenario that helps us better understand how both \ili{CA} and \ili{Gəʕəz}, and, from the scanty information that we currently have, also \ili{Old South Arabian}, ended up with similarly archaic forms of \ili{Aramaic}. This seems to suggest an as yet unattested, very archaic form of \ili{Aramaic} in South Arabia. Alternatively, the syncope and lenition so well-known in \ili{Syriac} may have had a much less broad distribution across the written \ili{Aramaic} dialects than previously thought.\footnote{I hope to discuss the {questions} raised by these {loanwords} in a {future} publication.}

\subsubsection{\label{bkm:Ref13224479}Greek (and Latin)}

Besides this noticeable cluster of \ili{Aramaic} and \ili{Gəʕəz} words, there are of course also \ili{Greek} {loanwords} in \ili{CA}, generally in the semantic fields of economy and administration. Very often \ili{Aramaic} likewise has these words, and it is usually not possible to decide whether \ili{Arabic} borrowed the word from \ili{Aramaic} or directly from \ili{Greek}. The former direction is presumably more likely considering the broad presence of \ili{Aramaic} as a lingua franca. Some examples are e.g. \textit{dīnār} ‘dinar’, \ili{Aram.} \textit{dēnār}, \ili{Gk.} \textit{dēnárion}, \ili{Lat.} \textit{denarius}; \textit{zawǧ} ‘spouse, pair’, \ili{Aram.} \textit{zōḡ} ‘id.’, \ili{Gk.} \textit{zeûgos} ‘yoke’; \textit{ṣirāṭ} ‘way’, \ili{Aram.} \textit{ʔesṭrāṭ} ‘street’, \ili{Gk.} \textit{stráta}, \ili{Lat.} \textit{(via)} \textit{strata}; \textit{qirṭās} ‘parchment, papyrus’, \ili{Aram.} \textit{qarṭīs}, \ili{Gk.} \textit{kʰártēs}; \textit{qaṣr} ‘castle’, \ili{Aram.} \textit{qaṣrā}, \ili{Gk.} \textit{kástron}, \ili{Lat.} \textit{castrum}; \textit{qalam} ‘reed-pen’, \ili{Gk.} \textit{kálamos} ‘reed-pen’.\footnote{Nöldeke (\citeyear[50]{Nöldeke1910}) argues that the \ili{CA} \textit{qalam} must come from \ili{Greek} through \ili{Gz.} \textit{qäläm}. While this is possible, there is nothing about this word that requires us to assume this directionality, nor is it particularly unlikely that \ili{CA} and \ili{Gəʕəz} independently borrowed this word without its \ili{Greek} ending \textit{-os}.}

A new influx of mostly philosophical and scientific \ili{Greek} vocabulary entered \ili{CA} during the early Abbasid period (mid 8th–10th centuries), at the time of the Graeco-\ili{Arabic} translation movement \citep{Gutas1998}. Once again, these words seem to have entered the language through \ili{Syriac} \citep{Gutas2011}. From this translation movement, we have words such as \textit{ǧins} ‘genus’ < \ili{Syr.} \textit{gensā} < \ili{Gk.} \textit{génos}; \textit{faylasūf} ‘philosopher’ < \ili{Syr.} \textit{pīlōsōp̄ā} < \ili{Gk.} \textit{pʰilósopʰos}; \textit{kīmyāʔ} ‘alchemy’ < \ili{Syr.} \textit{kīmīyā} < \ili{Gk.} \textit{kʰēmeía}; and \textit{ʔistāðiyā} ‘stadium’\footnote{Note here the apparent application of the \ili{Syriac} lenition being borrowed as such in \ili{Arabic}, unlike earlier loans. But it may also be possible that the lenition is part of the \ili{Greek} lenition of the \textit{delta} instead, as we see it today in Modern \ili{Greek}.} < \ili{Syr.} \textit{estaḏyā} < \ili{Gk.} \textit{stádion}.

\subsubsection{\label{bkm:Ref13224682}Ancient South Arabian}

It is often difficult to establish from which of the \ili{South Arabian} languages a certain word originates. As \ili{Old South Arabian} retained all the Proto-\ili{Semitic} consonants, a borrowing from \ili{Old South Arabian} or an inheritance from Proto-\ili{Semitic} is often difficult to distinguish in \ili{CA}. While Jeffery (\citeyear{Jeffrey2007} [1938]: 305) identifies a fair number of possible words of \ili{South Arabian} origin, hardly ever does this seem the only possibility. Another issue with identifying \ili{South Arabian} {loanwords} is that we have very scanty knowledge of its vocabulary or its linguistic developments. As a result, \ili{Old South Arabian} identifications can be quite difficult to substantiate.

In recent years several lexical studies have tried to draw connections between \ili{Old South Arabian} and \ili{Arabic} vocabulary, but this is often based on certain semantic extensions or uses of words as described in \ili{CA} dictionaries. While these observations may eventually be proven correct, it is somewhat difficult to evaluate whether we are truly dealing with borrowings in these cases, and the extremely limited knowledge that we have of the vowel system of the different \ili{Old South Arabian} languages makes it difficult to evaluate this in detail. Several interesting suggestions are given by  \cite{Weninger2009},  \cite{Hayajneh2011},  \cite{Elmaz2014} and  \cite{Elmaz2016}. 

To illustrate the difficulties we run into when trying to identify \ili{Old South Arabian} borrowings in \ili{Arabic}, let us examine the word \textit{tārīḫ} pl. \textit{tawārīḫ} ‘date’. From the perspective of \ili{CA} morphology, \textit{tārīḫ} could only be a hypocorrect form of \textit{taʔrīḫ} – which is indeed an attested biform of \textit{tārīḫ}. The existence of the plural \textit{tawārīḫ} rather than \textit{taʔārīḫ}, however, seems to suggest that \textit{taʔrīḫ} is rather a hypercorrect insertion of \textit{hamzah} from an original form \textit{tārīḫ}, which certainly looks foreign in its {formation}.

Both Hebbo (\citeyear[27]{Hebbo1984}) and Weninger (\citeyear[399]{Weninger2009}) have suggested that this word is to be connected with the the widespread \ili{Semitic} {root} \textit{√wrḫ}, related to ‘month’ or ‘moon’ (cf. \ili{Hebrew} \textit{yɛraḥ} < \textit{*warḫ} ‘month’), which exists in \ili{Old South Arabian} but not in Classical \ili{Arabic}.\footnote{Note, however, that the {root} \textit{√wrḫ} ‘month’ is attested unambiguously in the singular, dual (\textit{wrḫn}) and plural (\textit{ʾrḫ}) in the \ili{Old} \ili{Arabic} corpus of \ili{Safaitic} inscriptions \citep[353]{Al-Jallad2015Safaitic}.} The verb \textit{ʔarraḫa} ‘to date’ would then reasonably be taken as a backformation from \textit{tārīḫ}.

However, this explanation still leaves us with many problems. There is perhaps some reason to suppose that in \ili{Old South Arabian} \textit{*aw} would have collapsed to an unknown {monophthong} (Early Sabaic \textit{ywm} ‘day’; {Late} Sabaic \textit{ym}). This might explain why the word is \textit{tārīḫ} and not \textit{**tawrīḫ}, but \textit{tārīḫ} is not actually attested in \ili{Old South Arabian}. So while the suggestion is certainly possible, it seems that another of the many non-\ili{Arabic} Ancient northern Arabian epigraphic languages could likewise have been an origin. Barring further discoveries, many such proposed etymologies remain highly speculative, and drastically simplify the rather complex multilingual situation of pre-Islamic Arabia, where many other sources besides \ili{Old South Arabian} remain possible \citep{Al-Jallad2018ANA}.

\subsubsection{\label{bkm:Ref13224492}Persian}

Whereas with the advent of Islam the influence of \ili{Aramaic}, \ili{Greek} and \ili{Gəʕəz} on \ili{CA} quickly diminished and disappeared, the influence of \ili{Persian} actually increased. While the {Quran} already contains a sizeable number of \ili{Persian} borrowings, this only increases in the following centuries.

Some clear \ili{Persian} borrowings in the {Quran} include: \textit{ʔistabraq} `silk brocade', cf. New \ili{Persian} \textit{istabra} \citep[204]{Eilers1962}; \textit{numruq} `cushion' < Middle \ili{Persian} \textit{namrag}; \textit{kanz} ‘treasure’ < Middle \ili{Persian} \textit{ganz/ganǧ} ‘treasury’ \citep[206]{Eilers1962}. Outside of the {Quran} many other \ili{Persian} words may be found in \ili{Arabic}, e.g. \textit{dīwān} ‘archive, collected writings’ < Early New \ili{Persian} \textit{dīwān} \citep[223]{Eilers1962}, \textit{banafsaǧ}, \textit{manafsaǧ} ‘violet’ < Middle \ili{Persian} \textit{banafš} \citep[596]{Eilers1971}; \textit{barnāmaǧ} ‘program’ < Middle \ili{Persian} \textit{bārnāmag} \citep[217-218]{Eilers1962}; \textit{wazīr} ‘minister’ < Middle \ili{Persian} \textit{wizīr} \citep[207]{Eilers1962}.\footnote{I thank Chams Bernard for updating the transcription of the Middle \ili{Persian} forms.}

\subsubsection{\label{bkm:Ref13483797}Ottoman Turkish}

The influence of \ili{Ottoman} {Turkish} on MSA is significantly less than on the modern \ili{Arabic} dialects, largely due to linguistic purism \citep{Procházka2011Turkish}. Words that have entered MSA are words related to administration, technology and food, but also several other origins are found. For example: \textit{damɣa} ‘stamp’ < \textit{damga}; \textit{ǧumruk} ‘customs’ < \textit{gümrük} (ultimately from \ili{Latin} \textit{commercium}); \textit{bāšā} ‘pasha’ < \textit{paşa}; \textit{bābūr} < \textit{vapur} ‘steam ship’ (ultimately from \ili{French} [\textit{bateau} \textit{à}] \textit{vapeur}); \textit{quṣāǧ} ‘pliers’ < \textit{kıskaç}; \textit{balṭa} ‘axe’ < \textit{balta}; \textit{šāwurma,} \textit{šāwirma} ‘lamb, etc., roasted on a spit’ < \textit{çevirme}; \textit{qāwurma,} \textit{qāwirma} ‘fried meat’ < \textit{kavurma}; \textit{kufta} ‘meatballs’ < \textit{köfte}.

Of some interest is the \textit{-ci} suffix that denotes professions and characterizations in \ili{Turkish}. This suffix has developed some amount of productivity in modern dialects (especially in Iraq, Syria and Egypt), where it may even be suffixed to nouns of non-\ili{Turkish} origin. In MSA the suffix is attested not infrequently, although it would probably go too far to say that it is productive. Some examples are \textit{nawbatǧī} ‘on duty; command of the guard’ < \textit{nawba} ‘shift, rotatation’ + \textit{-ci}; \textit{qahwaǧī} ‘coffeehouse owner’ < \textit{qahwa} ‘coffee’ + \textit{-ci}; \textit{xurdaǧī} ‘dealer in miscellaneous smallwares’ < \textit{hordaci} ‘id.’; \textit{balṭaǧī} ‘sapper, pioneer’ < \textit{baltaci} ‘sapper’; \textit{būyāǧī} ‘painter, bootblack’ < \textit{boyaci} ‘painter’.

\subsubsection{Influence of Standard Average European}

A rather different, but nevertheless important factor of language contact for MSA, especially in the journalistic style, was described by \citet{Blau1969}. Blau argues that, under the influence of what he dubs “Standard Average European” (SAE; cf. \citealt{Whorf1956}), MSA (as well as Modern \ili{Hebrew}) has taken on a large amount of vocabulary,\footnote{For further discussion of the development of Modern Standard \ili{Arabic} technical vocabulary see \citet{Dichy2011} and \citet{Jacquart94}.} phraseology, and syntax similar to the journalistic language use of European languages, though the actual languages of influence could be quite different in different countries (e.g. \ili{Russian} and \ili{Yiddish} for Modern \ili{Hebrew}; \ili{English} for Egyptian MSA, \ili{French} for Lebanese, \ili{Moroccan}, \ili{Tunisian} and Algerian MSA).\footnote{The influence of \ili{French} in terms of borrowings and adaptations is especially salient in literary \ili{Arabic} as used in the Maghreb. \citet{Kropftisch1977} is an excellent study on this topic.} Examples of such influence take up over a hundred pages in Blau’s pioneering work.

Blau identifies examples of lexical expansion of existing words to include lexical associations present in SAE, e.g. \textit{saṭḥī} ‘flat’ is extended in meaning towards ‘superficial’ due to influence of, e.g.  \ili{French} \textit{superficiel} and \ili{German} \textit{oberflächlich} \citep[65]{Blau1969}; \textit{ǧaww} ‘air, atmosphere’ comes to be used in a metaphorical sense in the same way \ili{English} uses ‘atmosphere’, e.g. \textit{ǧawwu} \textit{s-siyāsati} \textit{mukahrabun} ‘the political atmosphere is electrified’ \citep[69]{Blau1969}.

Even whole phrases may show up as loan translations, such as MSA \textit{ʔanqaða} \textit{l-mawqifa} ‘to save the situation’, cf. \ili{French} \textit{sauver} \textit{la} \textit{situation}, \ili{German} \textit{die} \textit{Situation} \textit{retten}; MSA \textit{qatala} \textit{l-waqta} ‘to kill time’, cf. \ili{French} \textit{tuer} \textit{le} \textit{temps,} \ili{German} \textit{die} \textit{Zeit} \textit{totschlagen} \citep[76]{Blau1969}. Even such highly specific metaphorical expressions as ‘to miss the train’, in the meaning of missing an opportunity, appears in MSA \textit{ʔasriʕ} \textit{wa-ʔillā} \textit{fātaka} \textit{l-qiṭāru} ‘hurry, otherwise you will miss the train’ \citep[101]{Blau1969}.

Such linguistic influence, of course, does not lend itself particularly well to be classified within the framework of Van Coetsem (\citeyear{VanCoetsem1988}; \citeyear{VanCoetsem2000}), as the writers of MSA in these cases are dominant in neither the {source language}(s) nor the {recipient language}, a situation which is a rather unique result of the \ili{Arabic} {diglossia} in combination with the influence of foreign journalistic styles that have transformed the way in which MSA is written.

\subsection{\label{bkm:Ref13224870}Influence of the early Islamic vernaculars}

While, as a general rule, Classical \ili{Arabic} retains its archaic features, such as the retention of glottal stop in all positions and the lack of vowel harmony and syncope, we occasionally find single lexical items which optionally allow innovative forms which presumably {stem} from spoken vernaculars before the standardization of the classical language. This tends to be visible especially for words that have lost the glottal stop, a feature usually attributed to the Ḥiǧāzī variety of the early Islamic period. For example, \ili{CA} has \textit{nabiyy} ‘prophet’, \textit{nubuwwah} ‘prophethood’ from the {root} \textit{√nbʔ};\footnote{In several Quranic reading traditions these are still read \textit{nabīʔ} and \textit{nubūʔah}, as expected \citep[106--107]{IbnMujahid}.} likewise \textit{bariyyah} ‘creature’ from the {root} \textit{√brʔ}.\footnote{Read as \textit{barīʔah} in several Quranic reading traditions \citep[693]{IbnMujahid}.}

The likely loss of postconsonantal \textit{ʔ} in Ḥiǧāzī \ili{Arabic} has influenced the way the verb \textit{raʔā} ‘to see’ (\textit{√rʔy}) is conjugated. Its imperfect irregularly loses the \textit{ʔ}: \textit{yarā} ‘he sees’. Similarly the verb \textit{saʔala} ‘to see’ (\textit{√sʔl}) has two different imperatives, either the regular \textit{isʔal} or the Ḥiǧāzī \textit{sal} (< *sʔal). The imperative \textit{ʔalik} ‘send!’ must be the imperative of an otherwise unattested verb \textit{*ʔalʔaka} ‘to send’, which has likewise irregularly lost its postconsonantal \textit{ʔ}. Besides verbs, we may also see the irregular lack of representation of post-consonantal \textit{ʔ} in other nouns, e.g. \textit{malak} ‘angel’, which, considering its plural \textit{malāʔikah} and etymological origin, was presumably originally \textit{*malʔak}.

The pseudo-verbs \textit{niʕma} ‘what a wonderful …’ and \textit{biʔsa} ‘what an evil …’, are presumably originally from \textit{*naʕima} and \textit{*baʔisa}, with vowel harmony and syncope. These original forms have disappeared from the classical language in their pseudo-verbal use, only retaining their verbal meaning: \textit{naʕima} ‘to be happy, glad’ and \textit{baʔisa} ‘to be miserable, wretched’. However, other pseudo-verbs retain both unharmonized and unsyncopated forms as optional variants even in their pseudo-verbal use: \textit{ḥasuna,} \textit{ḥusna,} \textit{ḥasna} ‘how beautiful, magnificent’, and \textit{ʕað̣uma,} \textit{ʕuð̣ma,} \textit{ʕað̣ma} ‘how powerful, mighty’. Such syncopated and harmonized forms are claimed by the Arab grammarians themselves to be part of the eastern dialects, and absent in the Ḥiǧāzī dialects \citep[97]{Rabin1951}, but surprisingly are retained for such pseudo-verbs.

Syncopated forms, while reported for regular verbs as well by the Arab grammarians (e.g. \textit{šihda} or \textit{šahda} for \textit{šahida}), never occur in the Classical language. For some CaCiC nouns, syncopated forms are reported by lexicographers (e.g. \textit{katf} and \textit{kitf} besides \textit{katif}), but it is not clear whether these syncopated forms are used in \ili{CA} outside of these lexicons.

These kinds of dialectal forms that appear to have been incorporated into \ili{CA} are indicative of the artificial amalgam that makes up the language, and require a much more in-depth discussion than the present chapter allows. It seems clear that the vast amount of dialectal variation that is described by the Arab grammarians, judiciously collected by \citet{Rabin1951}, does not end up in \ili{CA}, but some amount of variants are either allowed, or are the only possible form present in the standard. The exact parameters that determine how and why such dialectal forms were incorporated into the language are currently unclear.

\section{Conclusion}

Due to \ili{CA} and MSA being almost exclusively High literary registers, with no true native speakers, the type of language contact that we see in the Islamic period is rather different from what we may see in more natural language contact situations. We mostly see {imposition} of certain dialectal forms onto the Classical ideal. An interesting exception to this is the {calquing} of MSA words and phraseology upon “Standard Average European”, where the speakers are dominant in neither the recipient nor the {source language}.

Borrowing can be detected in phonology, morphology and vocabulary from \ili{Greek}, \ili{Aramaic} and \ili{Ethio-Semitic} from the pre-Islamic period, which were then inherited by \ili{CA}. In the Islamic period, it is mostly vocabulary that is borrowed, with a significant number of loans coming from \ili{Greek}, \ili{Persian} and \ili{Ottoman} {Turkish} into \ili{CA}.

Examining these pre-Islamic borrowings, it has become clear that the \ili{Aramaic} that has primarily influenced \ili{CA}, contrary to what is popularly believed, was not a form of \ili{Syriac}, but rather a more archaic variety. The historical implications of this have not yet been well-integrated into our understanding of pre-Islamic linguistic diversity in Arabia and neighbouring regions.

While some studies have looked at syntactic {imposition} of the spoken dialects onto MSA with promising results, this has not yet been applied to medieval texts written in \ili{CA}. Nevertheless, considering the clear ethnic and geographic diversity of writers of \ili{CA}, it seems likely that {future} work should be able to detect such influences even in the medieval period.

\section*{Further reading}

\citet{Jeffrey2007} [1938] is still one of the most comprehensive books on {loanwords} in Quranic \ili{Arabic}.\\
\citet{Hebbo1984} is an in-depth study of foreign words as they appear in the Sīrah of Ibn Hišām.\\
\citet{Fraenkel1886} is an in-depth discussion of \ili{Aramaic} {loanwords} in \ili{Arabic}, but in some respects outdated.\\
\citet{Nöldeke1910} contains an important section on {loanwords} both from \ili{Arabic} to the Ethio-\ili{Semitic} languages and the other way around.\\
\citet{Blau1969} is a pioneering work researching the interaction between European literary languages and the effects they have on the literary style of Modern Standard \ili{Arabic} and Modern \ili{Hebrew}.

The chapters on language contact in the \textit{Encyclopaedia} \textit{of} \textit{Arabic} \textit{Language} \textit{and} \textit{Linguistics} are also highly useful and informative, and contain many up to date references for contact with \ili{Greek} \citep{Gutas2011}, \ili{Persian} \citep{Asbaghi2011}, \ili{Aramaic} \cite{Retsö2011}, and \ili{Turkish} \citep{Procházka2011Turkish}.

\section*{Acknowledgements}
I thank Stefan Procházka, Christopher Lucas, Maarten Kossmann and Ahmad Al-Jallad for providing me with important references, comments and suggestions.

\section*{Abbreviations}
\begin{multicols}{2}
\begin{tabbing}
\textsc{ipfv} \hspace{1em} \= before common era\kill
*             \>  reconstructed form\\
**            \>  unattested form\\
\textsc {1, 2, 3} \> 1st, 2nd, 3rd person \\
\textsc{acc}  \>  accusative\\
{Aram.}         \>  Aramaic\\
{CA}            \>  Classical {Arabic}\\
CE            \>  Common Era\\
\textsc{dat}  \>  dative\\
\textsc{f}    \>  feminine\\
Gk.          \>  Greek\\
Gz.           \>  Gəʕəz\\
\textsc{impf} \>  imperfect (prefix conjugation)\\
Lat.         \>  Latin\\
\textsc{m}    \>  masculine\\
{MSA}           \>  Modern Standard {Arabic}\\
\textsc{nom}  \>  nominative\\
\textsc{obl}  \>  oblique\\
\textsc{pl}   \>  plural\\
\textsc{pn}   \> personal name\\
\textsc{prf}  \>  perfect (suffix conjugation)\\
\textsc{rel}  \>  {relative} pronoun\\
{RL}          \>  {recipient language}\\
\textsc{sg}   \>  singular\\
{SL}          \>  {source language}\\
Syr.          \>  Syriac\\
v.n.         \>  verbal noun
\end{tabbing}
\end{multicols}


{\sloppy\printbibliography[heading=subbibliography,notkeyword=this]}
\end{document}
