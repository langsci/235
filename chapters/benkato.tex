\documentclass[output=paper]{langsci/langscibook} 
\author{Adam Benkato\affiliation{University of California, Berkeley}}
\title{Maghrebi Arabic}
% \keywords{} 
\abstract{This chapter gives an overview of contact-induced changes in the Maghrebi dialect group in North Africa. It includes both a general summary of relevant research on the topic and a selection of case studies which exemplify contact-induced changes in the areas of phonology, morphology, syntax, and lexicon.
}
\maketitle

\begin{document}
  
\section{The Maghrebi Arabic varieties}

In Arabic dialectology, \textsc{Maghrebi} is generally considered to be one of the main dialect groups of Arabic, denoting the dialects spoken in a region stretching from the Nile delta to Africa’s Atlantic coast -- in other words, the dialects of Mauritania, Morocco, Algeria, Tunisia, Libya, parts of western Egypt, and Malta. The main isogloss distinguishing Maghrebi dialects from non-Maghrebi dialects is the first person of the imperfect, as shown in Table \ref{tab:1:niktib} (cf. Lucas \& Čéplö this volume).\footnote{More about the exact distribution of this isogloss can be found in \citet{Behnstedt2016niktib}.}\ia{Lucas, Christopher@Lucas, Christopher}\ia{Čéplö, Slavomír@Čéplö, Slavomír}


\begin{table}
\caption{First person imperfect `write' in Maghrebi and non-Maghrebi Arabic}
\label{tab:1:niktib}
 \begin{tabular}{lllll} 
  \lsptoprule
         & \multicolumn{2}{c}{Non-Maghrebi}       & \multicolumn{2}{c}{Maghrebi} \\
         & Classical Arabic & Baghdad Arabic & Casablanca Arabic & Maltese \\
           \midrule
Singular & \textit{aktub}            & \textit{aktib}          & \textit{nəktəb}            & \textit{nikteb}  \\
Plural   & \textit{naktub}           & \textit{niktib}         & \textit{nkətbu}            & \textit{niktbu}  \\
\lspbottomrule
\end{tabular}
\end{table}

This Maghrebi group of dialects is in turn traditionally held to consist of two subtypes: those spoken by sedentary populations in the old urban centers of North Africa, and those spoken by nomadic populations. The former of these, usually referred to as “pre-Hilali” (better: “first-layer”) would have originated with the earliest Arab communities established across North Africa ({\textasciitilde}7th--8th centuries CE) up to the Iberian peninsula. The latter of these, usually referred to as “Hilali” (better: “second-layer”), is held to have originated with the westward migration of a large group of Bedouin tribes ({\textasciitilde}11th century CE) out of the Arabian peninsula and into North Africa via Egypt. Their distribution is roughly as follows.\footnote{More will not be said about the subgroups of Maghrebi dialects that have been proposed. For more details about the features and distribution of Maghrebi dialects see \citet{Pereira2011}; for more detail on the complex distribution of varieties in Morocco see \citet{Heath2002}.} First-layer dialects exist in cities such as Tunis, Kairouan, Mahdia, Sousse, Sfax (Tunisia), Jijel, Algiers, Cherchell, Tlemcen (Algeria), Tangier, Tetuan, southern Rif villages, Rabat, Fez, Taza, so-called “northern” dialects (Morocco), Maltese, and formerly Andalusi and Sicilian dialects; most Judeo-Arabic dialects formerly spoken in parts of North Africa are also part of this group. Second-layer dialects are spoken by populations of nearly all other regions, from western Egypt, through all urban and rural parts of Libya, to the remaining urban and rural parts of Algeria and Morocco. Though some differences between these two subtypes are clear (such as [q, ʔ, k] vs. [g] for *q), there have probably been varying levels of interdialectal mixture and contact since the eleventh century CE. In many cases, first-layer varieties of urban centers have been influenced by neighboring second-layer ones, leading to new dialects formed on the basis of inter-dialectal contact. It is important to note that North Africa is becoming increasingly urbanized and so not only is the traditional sedentary/nomadic distinction anachronistic (if it was ever completely true), but also that intensifying dialect contact accompanying urbanization means that new ways of thinking about Maghrebi dialects are necessary. It is also possible to speak of the recent but ongoing koinéization of multiple local varieties into supralocal or even roughly national varieties—thus one can speak, in a general way, of “Libyan Arabic” or “Moroccan Arabic”. This chapter will not deal with contact between mutually intelligible varieties of a language although this is equally important for the understanding of both the history and present of Maghrebi dialects.\footnote{The emergence of new Maghrebi varieties resulting from migration and mixture is discussed in \citet{Pereira2007} and \citet{Gibson2002}, for example. The oft-cited distinction between urban and nomadic dialects is also problematized by the existence of the so-called rural or village dialects (though this is also a problematic ecolinguistic term), on which see \citet{Mion2015}. Dialect contact outside of the Maghreb is discussed by Cotter (this volume).}\ia{Cotter, William M.@Cotter, William M.}

\section{ Languages in contact}

Contact between Arabic and other languages in North Africa began in the late seventh century CE, when Arab armies began to spread westward through North Africa, reaching the Iberian peninsula by the early eighth century CE and founding or occupying settlements along the way. Their dialects would have come into contact with the languages spoken in coastal regions at that time, including varieties of Berber and Late Latin, and possibly even late forms of Punic and Greek. The numbers of Arabic-speakers moving into North Africa at the time of initial conquests was likely to have been quite small.\footnote{See Heath (this volume) for discussion of Late Latin influence in Moroccan Arabic dialects.}\ia{Heath, Jeffrey@Heath, Jeffrey} By the time of the migration of Bedouin groups beginning in the eleventh century, it is doubtful that languages other than Berber and Arabic survived in the Maghreb. The Arabization of coastal hinterlands and the Sahara increased in pace after the eleventh century. Berber varieties continue to be spoken natively by millions in Morocco and Algeria, and by smaller communities in Libya, Tunisia, Mauritania, and Egypt. Any changes in an Arabic variety due to Berber are almost certainly the result of Berber speakers adopting Arabic rather than Arabic speakers adopting Berber \textendash \, the sociolinguistic situation in North Africa is such that L1 Arabic speakers rarely acquire Berber.

  Beginning in the sixteenth century, most of North Africa came under the control of the Ottoman Empire and thus into contact with varieties of Turkish, although the effect of Turkish is essentially limited to cultural borrowings (see below, 3.4). The sociolinguistic conditions in which Turkish was spoken in North Africa are poorly understood.

  The advent of colonialism imposed different European languages on the region, most prominently French (in Mauritania, Morocco, Algeria, and Tunisia), Italian (in Libya), and Spanish (in Morocco). Romance words in dialects outside of Morocco may also derive from forms of Spanish (via Andalusi refugees to North Africa in the 16th--17th centuries) or from the Mediterranean Lingua Franca.\footnote{On the Lingua Franca see Nolan (this volume).}\ia{Nolan, Joanna@Nolan, Joanna}

  The effects on Maghrebi Arabic of contact with Chadic (Hausa) or Nilo-Saharan (Songhay, Tebu) languages is largely unstudied since in most cases data from the relevant Arabic varieties is lacking. Yet some borrowings from these languages can be found in Arabic and Berber varieties throughout the region \citep{Souag2013lexical}.\footnote{See also \citet{Souag2016sahara} for an overview of contact in the Sahara region not limited to Arabic.} Lastly, Hebrew loans are present in most Jewish Arabic dialects of North Africa \citep{Yoda2013}, though unfortunately these dialects hardly exist anymore.

  To restate these facts in Van Coetsem’s (\citeyear{VanCoetsem1988,VanCoetsem2000}) terms, there are two major contact situations at work in Maghrebi Arabic in general, though the specifics will of course differ from variety to variety. The first is change in Arabic driven by source-language (Berber) dominant speakers; this transfer type is \textsc{imposition}. The second is change in Arabic driven by recipient-language (Arabic) dominant speakers where the source language is a European colonial language; this transfer type is called \textsc{borrowing}.\footnote{Another good illustration of the two transfer types in the Van Coetsemian framework can be found in Winford (\citeyear{Winford2005}: 378–381).} So far, ``dominance'' describes linguistic dominance, that is, the fact that a speaker is more proficient in one of the languages involved in the contact situation. However, social dominance, referring to the social and political status of a language (\citealt{VanCoetsem1988}: 13), is also important, especially in North Africa.

\section{ Contact-induced changes in Maghrebi dialects}

\subsection{ Phonology}

Changes in Maghrebi Arabic phonology due to contact with Berber are difficult to prove. There are several cases, for example, where historical changes in Arabic phonology may be argued to be the result of contact with Berber \textit{or} the result of internal developments. These include the change of *ǧ to /ž/ in many varieties, or the emergence of phonemic /ẓ/ \citep{Souag2016sahara}. Another example, the pronunciation /ṭ/ in some first-layer varieties where most Arabic varieties have /ð̣/ has also been explained as a result of Berber influence, or as unclear directionality \citep[187]{Kossmann2013book}, while \citet{Al-Jallad2015Maghreb} argues that it is actually an archaism within Arabic.

  The merger in Arabic of the vowels *{a} and *{i} (and even *{u}) to a single phoneme /ǝ/ in some, especially first-layer, varieties, is often attributed to Berber influence, as many Berber varieties have only a single short vowel phoneme /ǝ/. However Kossmann (\citeyear{Kossmann2013book}: 171–174) points out that Berber also merged older *ă and *ǝ to a single phoneme /ǝ/ and that it cannot be proven that the reduction happened in Berber before it happened in Arabic. Hence, again the directionality of influence is difficult to show.

  Related to this development is also that many Maghrebi varieties disallow vowels in light syllables (often described as the deletion of short vowels in open syllables), such that *\textit{katab} ‘he wrote’ > Tripoli \textit{ktǝb} or *\textit{kitāb} ‘book’ > Algerian \textit{ktāb}.\footnote{Since the short vowels merge to schwa in many Moroccan and Algerian varieties, vowel length is no longer contrastive and it is common to transcribe e.g. \textit{ktab} rather than \textit{ktāb}.} Meanwhile, second-layer varieties often do allow vowels in light syllables (e.g. Benghazi \textit{kitab} ‘he wrote’, Douz \textit{m\textsuperscript{i}}\textit{šē} ‘he went’). While proto-Berber and some modern varieties allow vowels in light syllables, most Berber varieties of Algeria and Morocco do not. This is another example of a similar development wherein the directionality of influence is unclear (see \citealt{Souag2017syllable}: 62–65 for further discussion).

  In the Arabic variety of Ghomara, northwest Morocco, *d and *t are spirantized to /ð/ and /θ/ initially (*d only), postvocalically and finally \citep{Naciri-Azzouz2016}: e.g. \textit{māθǝθ} `she died' (*mātat), \textit{warθ} `inheritance' (although etymologically *warθ, dialects of the wider Jbala region of Morocco have no interdentals so *wart), \textit{ðāba} `now' (*dāba), \textit{ḫǝðma} `work' (*ḫidma), \textit{wāḥǝð} `one' (*wāḥid). Naciri-Azzouz points out that the distribution of spirantization is the same as in Ghomara Berber, a variety spoken by groups in the same region.\footnote{The Berber variety of Ghomara exhibits an extreme amount of influence from dialectal Arabic, see Mourigh (\citeyear{Mourigh2015}). Kossmann (\citeyear[431]{Kossmann2013book}) writes that given the existence of parallel morphological systems for virtually all grammatical categories (nominal, adjectival, pronominal and verbal morphology) and a high loanword count (more than 30\% of basic lexicon is Arabic) it would be possible to call Ghomara Berber a mixed language.}

  New phonemes have been borrowed into Maghrebi varieties through contact with European languages: for example, /p/ and nasalized vowels in more recent French loans in Tunisian Arabic, or /v, č, ǧ/ in Italian loans in Libyan Arabic (\textit{grīǧū} ‘gray’ < \textit{grigio}).

  \subsection{Morphology}\label{Morphology}

In the realm of morphology, changes in Arabic varieties due to contact vary depending on whether the relationship between Arabic and the contact language is substratal, adstratal, or superstratal.

  Morphological influence from Berber on Arabic varieties of the northern Maghreb is not overly common.\footnote{Documentation of the varieties where such influence would be more expected, such as Arabic-speaking towns in the otherwise Berber-speaking Nafusa Mountains in Libya, is lacking.} In some places where Berber/Arabic bilingualism is or was more common, contact has led to the borrowing of Berber nouns into Arabic together with their morphology, a phenomenon known as “parallel system borrowing”.\footnote{For a closer look at parallel system borrowing in the context of Arabic and Berber contact, see \citet{Kossmann2010}, mostly discussing the borrowing of Arabic paradigms into Berber.} In Ḥassāniyya, for example, many nouns have been transferred together with their gender and number marking.\footnote{See Taine-Cheikh (this volume).}\ia{Taine-Cheikh, Catherine@Taine-Cheikh, Catherine} In the dialect of Jijel, Berber singular nouns are transferred together with their prefixes (\textit{āwtūl} ‘hare’, cf. Kabyle \textit{āwtūl}); plurals are then formed in a way which resembles Berber but is not identical (Jijel \textit{āsrǝf}, \textit{āsǝrfǝn} ‘bushe(s)’, cf. Kabyle Berber \textit{āsrǝf}, \textit{īsǝrfǝn}); moreover, the prefix \textit{ā}{}- is also used with nouns of Arabic origin (\textit{āfḫǝd} ‘thigh’, Arabic *\textit{faḫaḏ}) \citep[302–318]{Marçais1956}.

  In Algeria and Morocco the circumfix \textit{tā-...-t}, which occurs on feminine nouns in Berber, can derive abstract nouns (e.g. Jijel \textit{tākǝbūrt} ‘boasting’, \textit{tāwǝḥḥūnt} ‘having labor pains’) and in Moroccan Arabic \textit{tā-...-t} is the regular way of forming nouns of professions and traits (e.g. \textit{tānǝžžā{\R}t} ‘carpentry’) \citep{Kossmann2013chapter}.

  The verbal morphology of Arabic dialects is much less affected by Berber, though Ḥassāniyya again provides an interesting example. It has a causative prefix \textit{sä}- used with both inherited Arabic verbs and borrowed Berber verbs, and most likely to be borrowed from Berber causative forms in \textit{s-/š-} \citep{Taine-Cheikh2008}. 

  Turkish influence on morphology is restricted to the suffix -\textit{ği}/\textit{{}-ži} (< \textit{{}-ci}) used to indicate professions and borrowed widely into Arabic dialects in general. In Tunisia, its use has been extended to derive adjectives of quality from nouns (\textit{sukkārži} ‘drunkard’) and has also even been added to borrowed French nouns (\textit{bankāži} ‘banker’ < French \textit{banque}). As Manfredi (\citeyear[410]{Manfredi2018}) points out, the productivity of this borrowed derivational morpheme is one example of how recipient-language agentivity can introduce morphological innovations based on borrowing.

  French (and other Romance) verbs are also routinely borrowed into Maghrebi varieties. \citet{Talmoudi1986} discusses their integration into different forms of the verbal system of Tunisian Arabic, e.g. \textit{mannak} `to be absent' < French \textit{manquer} or (\textit{t)rānā} `to train' < French \textit{trainer}.

\subsection{ Syntax}

Syntax is often the least documented aspect of the grammar of Maghrebi Arabic varieties and research on contact-induced changes in syntax is still in its infancy. Much attention has been devoted recently to explaining the rise of bipartite negation in Arabic and Berber; in varieties of both languages the word for ‘thing’ (Arabic \textit{šayʔ}, Berber *ḱăra) has been grammaticalized postverbally in a marker of negation:


\ea\label{ex:key:}


Arabic (Benghazi)\\
\gll \textbf{mā}-šift-hā-\textbf{š}\\
     \textbf{\textsc{neg}}-see.\textsc{prf}.\textsc{1sg}-\textsc{3sg.f}-\textbf{\textsc{neg}}\\
\glt `I didn’t see her'.
\z

\ea\label{ex:key:}



Berber (Tarifit)\\
\gll \textbf{wā}  t-ẓ{\R}iɣ  \textbf{ša}\\
     \textbf{\textsc{neg}}  \textsc{3sg.f}-see.\textsc{prf}.\textsc{1sg}  \textbf{\textsc{neg}}\\
\glt `I didn’t see her'.
\z

Although some accounts give no attention to Berber, while others attribute the Arabic development solely to Berber, the development in both languages in the same contexts is probably not a coincidence; though there is no current consensus on the direction of transfer \textendash \,see Lucas (this volume) for discussion.\footnote{See \citet{Lucas2007,Lucas2010,Lucas2018} and \citet{Souag2018thing} for further discussion of the grammaticalization of `thing’ for indefinite quantification and polar question marking in Arabic and Berber. \citet[324–334]{Kossmann2013book} surveys the situation in the Berber languages. See \citet{Lafkioui2013reinventing} for an overview of negation in especially Moroccan Arabic, as well as discussion of a variety of Moroccan Arabic which features the discontinuous morpheme \textit{mā}{}- ... -\textit{bū}, where the latter part has been borrowed from Tarifit.}\ia{Lucas, Christopher@Lucas, Christopher} However, it must be noted that not all Berber varieties have double negation (e.g. Tashelhiyt \textit{ur} \textit{nniγ} \textit{ak} ‘I didn’t tell you’ where the only negator is \textit{ur}).

  In another area, recent work on the variety of Tunis has yielded interesting conclusions: while possessives with French nouns are overwhemingly analytic (\textit{l-prononciation} \textit{mtēʕ-ha} ‘her pronunciation’) and those with Arabic nouns are almost as overwhelmingly synthetic (\textit{nuṭq-u} ‘his pronunciation’), the frequent occurence of French loan nouns may be triggering an increase in the overall frequency of analytical possessives over syntactic ones, including those with Arabic nouns \citep{Sayahi2015}.

  The remainder of this section will discuss one particularly interesting case: the first-layer dialect of Jijel, a city in eastern Algeria. At the time of its description \citep{Marçais1956}, it showed little influence from second-layer varieties, but displayed wide-ranging influence from Berber in multiple domains. In a recent article, \citet{Kossmann2014} has demonstrated how a Berber marker of non-verbal predication was adopted into the Arabic dialect of Jijel as a focus marker. Here I will briefly summarize Kossmann’s arguments with a few examples. In the Jijel dialect, as described by Marçais and reanalyzed by Kossmann, a morpheme \textit{d} occurs in the following syntactic contexts (examples (\ref{nonverbal})\textendash(\ref{left focalizations}) are all from \citealt[129\textendash131]{Kossmann2014}, who retranscribes from Marçais’ texts): before non-verbal predicates (\ref{nonverbal}), in clefts with a noun/pronoun in the cleft (\ref{clefts}), in secondary predication with a specific noun (\ref{secondary pred}), as a marker of subject (or object) focus (\ref{focus}), and in left-moved focalizations (\ref{left focalizations}).

\ea\label{nonverbal}

\gll l-lila  d-ǝl-ʕid\\
     \textsc{def}-night  \textsc{d}-\textsc{def}-feast\\
\glt ‘Tonight is the feast.’
\z

\ea\label{clefts}

\gll d-hum  ǝddǝ  šraw-ǝh   qbǝl-ma  nǝzdad\\
     \textsc{d}-\textsc{3pl.m} \textsc{rel}  buy.\textsc{prf.3pl.m}-\textsc{3sg.m}  before-\textsc{comp} be.born.\textsc{impf.1sg}\\
\glt ‘It is them who bought it before I was born.’
\z

\ea\label{secondary pred}

\gll {\R}a-na  nqǝṭṭʕu-č  d  ǝṭ-ṭraf\\
     \textsc{prst}-\textsc{1pl}  cut.\textsc{impf.1pl-sg}  \textsc{d}  \textsc{def}-pieces\\
\glt ‘We will cut you (into) pieces.’
\z

\ea\label{focus}

\gll tkǝṣṣ\Rǝt  d  l-idura\\
     break.\textsc{prf.3sg.f}  \textsc{d}  \textsc{def}-bowl\\
\glt ‘The bowl has broken.’
\z

\ea\label{left focalizations}

\gll qalu  d  ǝ\R-{\R}biʕ  dǝḫlǝt\\
     say.\textsc{prf.3pl}  \textsc{d}  \textsc{def}-spring  enter.\textsc{prf.3sg.f}\\
\glt ‘They say spring has come.’
\z

Although previous analyses attempted to explain \textit{d} within Arabic, Kossmann notes that an Arabic-internal derivation of \textit{d} is impossible. However, Kabyle, the Berber language neighboring the Jijel area has an element \textit{d} (realized [ð] due to spirantization in Kabyle) which is used in (pro)nominal predicates (\ref{Berber pred}), cleft constructions (\ref{Berber cleft}), and secondary predication when non-verbal (\ref{Berber secondary}). Examples (\ref{Berber pred})\textendash(\ref{Berber secondary}) are all Kabyle Berber, taken from Kossmann (\citeyear[135\textendash136]{Kossmann2014}). This element \textit{d} is attested in Berber more widely, too, and is likely reconstructible to older stages of the language.

\ea\label{Berber pred}


\gll d-yǝlli-m\\
     \textsc{d}-daughter-\textsc{2sg.f}\\
\glt ‘Is it your daughter?’
\z

\ea\label{Berber cleft}

\gll d-ay-ǝn  i  d-tǝnna  abrid  amǝnzu\\
     \textsc{d}-this-\textsc{deict}  \textsc{rel}  hither-say.\textsc{3sg.f}  road  first\\
\glt ‘This is what she said the first time.’
\z

\ea\label{Berber secondary}

\gll ad  nǝǧʕǝl  iman  nn-ǝɣ  d-inǝbgiwǝn  n  \Rǝbbi\\
     \textsc{mod} make.\textsc{1pl} self \textsc{gen-1pl}  \textsc{d}-guests \textsc{gen}  lord\\
\glt ‘We shall pretend to be beggars (lit. guests of God).’
\z

Thus Berber \textit{d} is the best candidate for the origin of Jijel Arabic \textit{d}, though its usage in (Kabyle) Berber (where it is a primarily marker of syntactic organization) differs from that of Jijel Arabic (where it is mainly a marker of information structure). In a simplified scenario with a Berber variety as source language and Jijel Arabic as recipient, \textit{d} would likely have been imposed into Jijel Arabic with its exact Berber functions. As Kossmann notes, though, speech communities are full of variation and language contact is a “negotiation between the frequency of non-native speech and the prestige of the native way of speaking” \citep[138]{Kossmann2014}. Kossmann thus proposes a scenario in which larger groups of Berber speakers switched to a variety of Jijel Arabic and began imposing their own \textit{d}; the native Jijel Arabic speakers, fewer in number, began adopting \textit{d} but understood it differently and interpreted it as a focus marker, introducing it into new contexts; eventually the variety of Jijel Arabic with \textit{d} in all these functions became nativized. Per Kossmann (\citeyear[138\textendash139]{Kossmann2014}), two processes would have taken place: the transfer of a source-language feature by speakers dominant in the source language (Berber), followed by the borrowing of this feature by speakers dominant in the recipient language (Arabic), and its eventual regularization in that variety. Jijel Arabic is an excellent example of what may happen when large numbers of Berber speakers switch to Arabic.

\subsection{Lexicon}

Much work on contact and Maghrebi Arabic has focused on loanwords, the most salient effects of borrowing, with secondary attention to their phonological or morphological adaptation. The concept of social dominance has particular relevance for borrowing: in the North African context, the colonial languages, especially French, have high social status for both Arabic and Berber native speakers. One also must modify the idea of linguistic dominance to include those who acquire two languages natively (2L1 speakers; see \citealt[525]{Lucas2015}), definitely the case for certain speakers of Berber and Arabic in North Africa. 

  Unsurprisingly, we see firstly that the majority of words borrowed into Arabic varieties are nouns, and secondly that the lexical domains into which these borrowings fall are often restricted. Social dominance seems to play a role in the nature of the nouns borrowed.

  Berber loans are found in most Maghrebi varieties, though their number ranges from only a handful of words in the east to many more in the west (cf. §\ref{Morphology} above). Almost all Maghrebi varieties borrow the words \textit{ž(i)\Rāna} ‘frog’ and \textit{fakrūna} ‘turtle’, while in some oases Berber influence in agricultural terminology can be seen. Again, the documentation of the relevant varieties is often insufficient.

Several studies on contact between Maghrebi Arabic varieties and European languages exist. For French in Morocco, \citet{Heath1989} argues that code-switching and borrowing are essentially the same in a bilingual community which has established borrowing routines.\footnote{\citet[87]{VanCoetsem1988} notes that for bilingual speakers who have a balance in linguistic dominance between the two languages, the separation between the two transfer types (borrowing and imposition) will be weaker. Hence, either of the two dominant languages can serve as the recipient language in codeswitching behavior. \citeauthor{Winford2005} (2005, esp. 394–396), expanding on Van Coetsem’s framework, points out that code-switching is inherently linked to the borrowing transfer type. In the Maghreb, this scenario is possible for Berber\textendash Arabic bilinguals as well as for some French\textendash Arabic bilinguals. See \citet{Ziamari2008} for an insightful and more recent analysis of Moroccan Arabic in contact with French using a ``matrix language frame" analysis.} For French in Tunisia, \citet{Talmoudi1986} analyzes the phonological and morphological adaptation of French verbs into Arabic. \citet[127–151]{Sayahi2014} gives a broader view of lexical borrowing in diglossic/bilingual communities, focusing on French in Tunisia and Spanish in Morocco. \citet{Vicente2005} studies Arabic-Spanish code-switching in Ceuta, a Spanish enclave in northern Morocco. Italian in Tunisia is studied briefly by \citet{Cifoletti1994}. Studies of contact with Turkish are limited to discussion of loanwords: on Morocco see \citet{Procházka2012}; on Algeria, see \citet{BenCheneb1922}, to be read with the review by G.S. Colin \citep[21–30]{Colin1999}.

The remainder of this section will consider the influence of Turkish and Italian on Libyan Arabic (henceforth LA), a hitherto under-researched topic. Uniquely in the Maghreb region there is at present no superstratum language spoken widely by Arabic speakers in Libya, while there are also fewer Berber speakers than in Algeria or Morocco. As far as documented varieties of LA (Tripoli and Benghazi) go, contact situations are historical and not active.

  There seems to be an impression among dialectologists that LA varieties have the largest number of Turkish loans, though there is not a published basis for this. Procházka (\citeyear[191]{Procházka2005}) suggests that the number of (Ottoman) Turkish loans in a given Arabic dialect is proportional to the length and intensity of Ottoman rule. By this criterion Libya should have quite a few, as the regions now constituting Libya were under control of the Ottoman Empire from 1551 to 1911, but Procházka estimates that the dialect would show 200 to 500 surviving loans, fewer than in other dialects. Another important factor is likely to be that Libya’s population was very small during the period of Ottoman rule so that the long-term presence of even a few thousand Turkish speakers could have had a significant effect. However, I cannot yet offer a statistical analysis of Turkish words in Libyan Arabic.\footnote{The only study dedicated to Turkish loans in LA is \citet{Türkmen1988}, who lists 90 words. However, the basis for his wordlist seems unclear and  several items are either spurious or incorrect (e.g. there is no word \textit{kabak} ‘pumpkin’ in Benghazi Arabic but there is \textit{bkaywa} ‘pumpkin’, identified by \citet{Souag2013lexical} as a loan from Hausa). Turkish words in Libyan Arabic cited here are from the Benghazi variety, author’s data.} It is clear so far, though, that the effects of Turkish on LA can mainly be seen in the lexicon and, in my data, almost entirely in nouns. In terms of their semantic domains, Procházka (\citeyear[192]{Procházka2005}) points out that the majority of Turkish loans in Arabic dialects in general fall into three categories, roughly described as: private life; law, government, social classes; and army, war. By far the majority of surviving loans would belong to the first of these classes (such as \textit{šīšma} ‘tap’ < \textit{çeşme}, \textit{dizdān} ‘wallet’ < \textit{cüzdan}), or second (such as \textit{fayramān} ‘order’ < \textit{ferman}, \textit{ḥafð̣a} ‘week’ < \textit{hafte},) while I suspect that words from the third class are increasingly rarer. Outside of these, only a few words other than nouns seem to be present, such as \textit{du\.gri} ‘straight ahead’ and \textit{balki} ‘maybe’. The length of time since Turkish was last actively spoken in Libya no doubt means that the number of Turkish loans actively used by speakers has been decreasing.

  LA is unique among Maghrebi varieties in having had Italian as the main European contact language. Italian had a presence in what is now Libya from the 1800s, but this was mainly limited to the Tripolitanian Jewish community and wealthy merchant families. The Italian colonization of Libya officially began in 1911; though the majority of the region was not brought under Italian control until the early 1930s, large numbers of Italian colonists had begun to settle in Libya in the 1920s. From that period until 1970, when the remaining Italian citizens were expelled from the country, Italians made up 15\% or more of the population and the language was in widespread use. From the 1970s on, Italian was scarcely used in Libya, and the teaching of foreign languages was banned in 1984, not to return again until 2005.\footnote{For more information on the return of Italian instruction to Libya, see \citet{Danna2018phonetic}.} Many of the postwar generation spoke (and still speak) Italian, though they rarely use it anymore, but few Libyans of younger generations do. The 1920s to the 1970s can thus be regarded as the main period of contact between LA and Italian.\footnote{The Italian words in Yoda’s study of Tripoli Judeo-Arabic \citep{Yoda2005} need to be seen slightly differently than Italian words in non-Jewish dialects, owing to a different history of the Tripolitanian Jewish community with Italy.} However, the concentration of Italians differed from region to region and thus may have influenced local varieties differently. The primary study devoted to analyzing Italian loans in LA is that of \citet{Abdu1988} who, focusing on the variety of Tripoli, draws up a list of nearly 700 items (a few are misidentified), of which about 50\% were recognized by a majority of those surveyed. Some 93\% of these are nouns and the remainder are practically all derived from nouns or adjectives, such as \textit{bwōno} ‘well done!’ < \textit{buono} ‘good’ or \textit{faryaz} ‘to go out of order’ < Italian \textit{fuori} \textit{uso}.\footnote{See Abdu (\citeyear[271]{Abdu1988}) and \citet{Danna2018phonetic}. Some denominal verbs are cited by Abdu, but more extensive data might reveal several more in use: for example in the variety of Benghazi, I identified \textit{fu{\R}an}  ‘to brake (intransitive)’ < \textit{frayno} ‘brake’ < Italian \textit{freno}, not listed by Abdu.} Abdu’s study (\citeyear[248–268]{Abdu1988}) groups Italian loans into some 22 semantic categories, the vast majority of which relate to material culture. Examples of these from the Benghazi variety are \textit{byāmbu} ‘lead’ < \textit{piombo}, \textit{bōskō} ‘zoo’ < \textit{bosco} `wood', \textit{furkayta} ‘fork’ < \textit{forchetta}, \textit{maršabīdi} ‘sidewalk’ < \textit{marciapiede} (author’s data).

  As \citet{Danna2018phonetic} points out, the adaptation of Italian words to Libyan Arabic phonology varies: new phonemes, particularly  [v] and [č], sometimes occur but are sometimes adapted to the dialects’ pre-existing phonologies, an indication of “subsidiary phonological borrowing” (\citealt{VanCoetsem1988}: 98). Of course, the maintenance of new phonemes often depends on speakers continuing to have access to the source language; as this is no longer the case in Libya, Italian borrowings in Libyan Arabic are traversing a different trajectory than French borrowings in other Maghrebi varieties, where only the oldest borrowings have been phonologically integrated.

  The overwhelming majority of surviving Turkish and Italian loans in Libyan Arabic are nouns, widely acknowledged to be the most easily-borrowed word class due to their being the least disruptive of the recipient language’s argument structure \citep{Myers-Scotton2002}, though a few verbs derived dialect-internally do exist. Furthermore, almost all the nouns are cultural borrowings — “lexical content-words that denote an object or concept hitherto unfamiliar to the receiving society, terminology related to institutions that are the property of the neighboring [or colonizing] culture, and so on” \citep[210]{Matras2011universals}. Cultural borrowings are to be differentiated from core borrowings, the latter being words that more or less duplicate already existing words and which originate in a bilingual codeswitching context. These facts lead us to conclude that Turkish and Italian borrowings in Libyan varieties would be from (1) to (2) on the borrowing scale proposed by Thomason \& Kaufman (\citeyear[78–83]{ThomasonKaufman1988}). While (1) of the scale involves lexical borrowing of non-basic vocabulary only, (2) includes some function words as well as new phones appearing in those loanwords. Colonial language contact situations are typically ones of recipient-language agentivity, as the number of indigenous people learning the colonial language is many times more than the number of colonizers learning indigenous languages. Without a longer period of sustained bilingualism or language education motivated by continued contact with the metropole, Italian has affected Libyan Arabic to a much less degree than French has Libya's Maghrebi neighbours.

\section{Conclusion}

The general parameters of the Maghrebi linguistic landscape and contact situations are relatively well understood. However, more documentation of Maghrebi varieties is needed, and more specifically, of those where contact situations -- especially with Berber -- may have existed. Additionally, further research into the sociolinguistic factors affecting bilingualism in Berber and Arabic, or regarding the intersection of diglossia with bilingualism, will no doubt add to our knowledge of the parameters of contact-induced change more generally. Finally, inter-dialectal contact as well as the gradual rise of national or at least supra-local varieties certainly merits continuing attention.

\section*{Further reading}

\citet{Kossmann2013book} is the most extensive study so far of Berber\textendash Arabic contact, written from a Berberological point of view but important for Arabists.\\
\citet{Sayahi2014} studies the intersection of dialects, Standard Arabic, French and Spanish in Tunisia and Morocco.\\
\citet{Souag2016sahara} summarizes contact in the Saharan region among Arabic, Berber, Hausa, Songhay, Chadic, etc.\\
\citet{Ziamari2008} is the most up-to-date work discussing code-switching and borrowing strategies between Moroccan Arabic and French.
 

\section*{Abbreviations}

\begin{tabularx}{.45\textwidth}{lQ}
\textsc{1, 2, 3} & 1st, 2nd, 3rd person \\
\textsc{comp} & complementizer \\
\textsc{def} & definite \\
\textsc{deict} & deictic \\
\textsc{f} & feminine \\
\textsc{gen} & genitive \\
\textsc{impf} & imperfect \\
\textsc{m} & masculine \\
\end{tabularx}
\begin{tabularx}{.45\textwidth}{lQ}
\textsc{mod} & modal \\
\textsc{neg} & negative \\
\textsc{pl} & plural \\
\textsc{prf} & perfect \\
\textsc{prst} & presentative \\
\textsc{rel} & relative \\
\textsc{sg} & singular \\
\end{tabularx}


\section*{Acknowledgements} 
The writing of this article was supported by a grant from the Alexander von Humboldt Stiftung, to whom I am grateful for their support.

\sloppy
\printbibliography[heading=subbibliography,notkeyword=this] 
\end{document}