\documentclass[output=paper]{langsci/langscibook}
\author{Christopher Lucas\affiliation{SOAS University of London}}
\title{Contact and the expression of negation}
\abstract{This chapter presents an overview of developments in the expression of negation in Arabic and a number of its contact languages, focusing on clausal negation, with some remarks also on indefinites in the scope of negation. For most of the developments discussed in this chapter, it is not possible to say for certain that they are contact-induced. But evidence is presented which, cumulatively, points to widespread contact-induced change in this domain being the most plausible interpretation of the data.
}
\IfFileExists{../localcommands.tex}{
  % add all extra packages you need to load to this file 
\usepackage{graphicx}
\usepackage{tabularx}
\usepackage{amsmath} 
\usepackage{multicol}
\usepackage{lipsum}
\usepackage[stable]{footmisc}
\usepackage{adforn}
%%%%%%%%%%%%%%%%%%%%%%%%%%%%%%%%%%%%%%%%%%%%%%%%%%%%
%%%                                              %%%
%%%           Examples                           %%%
%%%                                              %%%
%%%%%%%%%%%%%%%%%%%%%%%%%%%%%%%%%%%%%%%%%%%%%%%%%%%%
% remove the percentage signs in the following lines
% if your book makes use of linguistic examples
\usepackage{./langsci/styles/langsci-optional} 
\usepackage{./langsci/styles/langsci-lgr}
\usepackage{morewrites} 
%% if you want the source line of examples to be in italics, uncomment the following line
% \def\exfont{\it}

\usepackage{enumitem}
\newlist{furtherreading}{description}{1}
\setlist[furtherreading]{font=\normalfont,labelsep=\widthof{~},noitemsep,align=left,leftmargin=\parindent,labelindent=0pt,labelwidth=-\parindent}
\usepackage{phonetic}
\usepackage{chronosys,tabularx}
\usepackage{csquotes}
\usepackage[stable]{footmisc} 

\usepackage{langsci-bidi}
\usepackage{./langsci/styles/langsci-gb4e} 

  \makeatletter
\let\thetitle\@title
\let\theauthor\@author 
\makeatother

\newcommand{\togglepaper}[1][0]{ 
  \bibliography{../localbibliography}
  \papernote{\scriptsize\normalfont
    \theauthor.
    \thetitle. 
    To appear in: 
    Christopher Lucas and Stefano Manfredi (eds.),  
    Arabic and contact-induced language change
    Berlin: Language Science Press. [preliminary page numbering]
  }
  \pagenumbering{roman}
  \setcounter{chapter}{#1}
  \addtocounter{chapter}{-1}
}

\newfontfamily\Parsifont[Script=Arabic]{ScheherazadeRegOT_Jazm.ttf} 
\newcommand{\arabscript}[1]{\RL{\Parsifont #1}}
\newcommand{\textarabic}[1]{{\arabicfont #1}}

\newcommand{\textstylest}[1]{{\color{red}#1}}

\patchcmd{\mkbibindexname}{\ifdefvoid{#3}{}{\MakeCapital{#3}
}}{\ifdefvoid{#3}{}{#3 }}{}{\AtEndDocument{\typeout{mkbibindexname could
not be patched.}}}

%command for italic r with dot below with horizontal correction to put the dot in the prolongation of the vertical stroke
%for some reason, the dot is larger than expected, so we explicitly reduce the font size (to \small)
%for the time being, the font is set to an absolute value. To be more robust, a relative reduction would be better, but this might not be required right now
\newcommand{\R}{r\kern-.05ex{\small{̣}}\kern.05ex}


\DeclareLabeldate{%
    \field{date}
    \field{year}
    \field{eventdate}
    \field{origdate}
    \field{urldate}
    \field{pubstate}
    \literal{nodate}
}

\renewbibmacro*{addendum+pubstate}{% Thanks to https://tex.stackexchange.com/a/154367 for the idea
  \printfield{addendum}%
  \iffieldequalstr{labeldatesource}{pubstate}{}
  {\newunit\newblock\printfield{pubstate}}
}

  %% hyphenation points for line breaks
%% Normally, automatic hyphenation in LaTeX is very good
%% If a word is mis-hyphenated, add it to this file
%%
%% add information to TeX file before \begin{document} with:
%% %% hyphenation points for line breaks
%% Normally, automatic hyphenation in LaTeX is very good
%% If a word is mis-hyphenated, add it to this file
%%
%% add information to TeX file before \begin{document} with:
%% %% hyphenation points for line breaks
%% Normally, automatic hyphenation in LaTeX is very good
%% If a word is mis-hyphenated, add it to this file
%%
%% add information to TeX file before \begin{document} with:
%% \include{localhyphenation}
\hyphenation{
affri-ca-te
affri-ca-tes
com-ple-ments
homo-phon-ous
start-ed
Meso-potam-ian
morpho-phono-logic-al-ly
morpho-phon-em-ic-s
Palestin-ian
re-present-ed
Ki-nubi
ḥawār-iyy-ūn
archa-ic-ity
fuel-ed
de-velop-ment
pros-od-ic
Arab-ic
in-duced
phono-logy
possess-um
possess-ive-s
templ-ate
spec-ial
espec-ial-ly
nat-ive
pass-ive
clause-s
potent-ial-ly
Lusignan
commun-ity
tobacco
posi-tion
Cushit-ic
Middle
with-in
re-finit-iz-ation
langu-age-s
langu-age
diction-ary
glossary
govern-ment
eight
counter-part
nomin-al
equi-valent
deont-ic
ana-ly-sis
Malt-ese
un-fortun-ate-ly
scient-if-ic
Catalan
Occitan
ḥammāl
cross-linguist-ic-al-ly
predic-ate
major-ity
ignor-ance
chrono-logy
south-western
mention-ed
borrow-ed
neg-ative
de-termin-er
European
under-mine
detail
Oxford
Socotra
numer-ous
spoken
villages
nomad-ic
Khuze-stan
Arama-ic
Persian
Ottoman
Ottomans
Azeri
rur-al
bi-lingual-ism
borrow-ing
prestig-ious
dia-lects
dia-lect
allo-phone
allo-phones
poss-ible
parallel
parallels
pattern
article
common-ly
respect-ive-ly
sem-antic
Moroccan
Martine
Harrassowitz
Grammatic-al-ization
grammatic-al-ization
Afro-asiatica
Afro-asiatic
continu-ation
Semit-istik
varieties
mono-phthong
mono-phthong-ized
col-loquial
pro-duct
document-ary
ex-ample-s
ex-ample
termin-ate
element-s
Aramaeo-grams
Centr-al
idioms
Arab-ic
Dadan-it-ic
sub-ordin-ator
Thamud-ic
difficult
common-ly
Revue
Bovingdon
under
century
attach
attached
bundle
graph-em-ic
graph-emes
cicada
contrast-ive
Corriente
Andalusi
Kossmann
morpho-logic-al
inter-action
dia-chroniques
islámica
occid-ent-al-ismo
dialecto-logie
Reichert
coloni-al
Milton
diphthong-al
linguist-ic
linguist-ics
affairs
differ-ent
phonetic-ally
kilo-metres
stabil-ization
develop-ments
in-vestig-ation
Jordan-ian
notice-able
level-ed
migrants
con-dition-al
certain-ly
general-ly
especial-ly
af-fric-ation
Jordan
counter-parts
com-plication
consider-ably
inter-dent-al
com-mun-ity
inter-locutors
com-pon-ent
region-al
socio-historical
society
simul-taneous
phon-em-ic
roman-ization
Classic-al
funeral
Kurmanji
pharyn-geal-ization
vocab-ulary
phon-et-ic
con-sonant
con-sonants
special-ized
latter
latters
in-itial
ident-ic-al
cor-relate
geo-graphic-al-ly
Öpengin
Kurd-ish
in-digen-ous
sunbul
Christ-ian
Christ-ians
sekin-în
fatala
in-tegration
dia-lect-al
Matras
morpho-logy
in-tens-ive
con-figur-ation
im-port-ant
com-plement
ḥaddād
e-merg-ence
Benjmamins
struct-ure
em-pir-ic-al
Orient-studien
Anatolia
American
vari-ation
Jastrow
Geoffrey
Yarshater
Ashtiany
Edmund
Mahnaz
En-cyclo-pædia
En-cyclo-paedia
En-cyclo-pedia
Leiden
dia-spora
soph-is-ic-ated
Sasan-ian
every-day
domin-ance
Con-stitu-tion-al
religi-ous
sever-al
Manfredi
re-lev-ance
re-cipi-ent
pro-duct-iv-ity
turtle
Morocco
ferman
Maghreb-ian
algérien
stand-ard
systems
Nicolaï
Mouton
mauritani-en
Gotho-burg-ensis
socio-linguist-ique
plur-al
archiv-al
Arab-ian
drop-ped
dihāt
de-velop-ed
ṣuḥbat
kitāba
kitābat
com-mercial
eight-eenth
region
Senegal
mechan-ics
Maur-itan-ia
Ḥassān-iyya
circum-cision
cor-relation
labio-velar-ization
vowel
vowels
cert-ain
īggīw
series
in-tegrates
dur-ative
inter-dent-als
gen-itive
Tuareg
tălămut
talawmāyət
part-icular
part-icular-ly
con-diment
vill-age
bord-er
polit-ical
Wiesbaden
Uni-vers-idad
Geuthner
typo-logie
Maur-itanie
nomades
Maur-itan-ian
dia-lecto-logy
Sahar-iennes
Uni-vers-ity
de-scend-ants
NENA-speak-ing
speak-ing
origin-al
re-captured
in-habit-ants
ethnic
minor-it-ies
drama-tic
local
long-stand-ing
regions
Nineveh
settle-ments
Ṣəndor
Mandate
sub-stitut-ing
ortho-graphy
re-fer-enced
origin-ate
twenti-eth
typ-ic-al-ly
Hobrack
never-the-less
character-ist-ics
character-ist-ic
masc-uline
coffee
ex-clus-ive-ly
verb-al
re-ana-ly-se-d
simil-ar-ities
de-riv-ation
im-pera-tive
part-iciple
dis-ambi-gu-ation
dis-ambi-gu-a-ing
phen-omen-on
phen-omen-a
traktar
com-mun-ity
com-mun-ities
dis-prefer-red
ex-plan-ation
con-struction
wide-spread
us-ual-ly
region-al
Bulut
con-sider-ation
afro-asia-tici
Franco-Angeli
Phono-logie
Volks-kundliche
dia-lectes
dia-lecte
select-ed
dis-appear-ance
media
under-stand-able
public-ation
second-ary
e-ject-ive
re-volu-tion
re-strict-ive
Gasparini
mount-ain
mount-ains
yellow
label-ing
trad-ition-al-ly
currently
dia-chronic
}
\hyphenation{
affri-ca-te
affri-ca-tes
com-ple-ments
homo-phon-ous
start-ed
Meso-potam-ian
morpho-phono-logic-al-ly
morpho-phon-em-ic-s
Palestin-ian
re-present-ed
Ki-nubi
ḥawār-iyy-ūn
archa-ic-ity
fuel-ed
de-velop-ment
pros-od-ic
Arab-ic
in-duced
phono-logy
possess-um
possess-ive-s
templ-ate
spec-ial
espec-ial-ly
nat-ive
pass-ive
clause-s
potent-ial-ly
Lusignan
commun-ity
tobacco
posi-tion
Cushit-ic
Middle
with-in
re-finit-iz-ation
langu-age-s
langu-age
diction-ary
glossary
govern-ment
eight
counter-part
nomin-al
equi-valent
deont-ic
ana-ly-sis
Malt-ese
un-fortun-ate-ly
scient-if-ic
Catalan
Occitan
ḥammāl
cross-linguist-ic-al-ly
predic-ate
major-ity
ignor-ance
chrono-logy
south-western
mention-ed
borrow-ed
neg-ative
de-termin-er
European
under-mine
detail
Oxford
Socotra
numer-ous
spoken
villages
nomad-ic
Khuze-stan
Arama-ic
Persian
Ottoman
Ottomans
Azeri
rur-al
bi-lingual-ism
borrow-ing
prestig-ious
dia-lects
dia-lect
allo-phone
allo-phones
poss-ible
parallel
parallels
pattern
article
common-ly
respect-ive-ly
sem-antic
Moroccan
Martine
Harrassowitz
Grammatic-al-ization
grammatic-al-ization
Afro-asiatica
Afro-asiatic
continu-ation
Semit-istik
varieties
mono-phthong
mono-phthong-ized
col-loquial
pro-duct
document-ary
ex-ample-s
ex-ample
termin-ate
element-s
Aramaeo-grams
Centr-al
idioms
Arab-ic
Dadan-it-ic
sub-ordin-ator
Thamud-ic
difficult
common-ly
Revue
Bovingdon
under
century
attach
attached
bundle
graph-em-ic
graph-emes
cicada
contrast-ive
Corriente
Andalusi
Kossmann
morpho-logic-al
inter-action
dia-chroniques
islámica
occid-ent-al-ismo
dialecto-logie
Reichert
coloni-al
Milton
diphthong-al
linguist-ic
linguist-ics
affairs
differ-ent
phonetic-ally
kilo-metres
stabil-ization
develop-ments
in-vestig-ation
Jordan-ian
notice-able
level-ed
migrants
con-dition-al
certain-ly
general-ly
especial-ly
af-fric-ation
Jordan
counter-parts
com-plication
consider-ably
inter-dent-al
com-mun-ity
inter-locutors
com-pon-ent
region-al
socio-historical
society
simul-taneous
phon-em-ic
roman-ization
Classic-al
funeral
Kurmanji
pharyn-geal-ization
vocab-ulary
phon-et-ic
con-sonant
con-sonants
special-ized
latter
latters
in-itial
ident-ic-al
cor-relate
geo-graphic-al-ly
Öpengin
Kurd-ish
in-digen-ous
sunbul
Christ-ian
Christ-ians
sekin-în
fatala
in-tegration
dia-lect-al
Matras
morpho-logy
in-tens-ive
con-figur-ation
im-port-ant
com-plement
ḥaddād
e-merg-ence
Benjmamins
struct-ure
em-pir-ic-al
Orient-studien
Anatolia
American
vari-ation
Jastrow
Geoffrey
Yarshater
Ashtiany
Edmund
Mahnaz
En-cyclo-pædia
En-cyclo-paedia
En-cyclo-pedia
Leiden
dia-spora
soph-is-ic-ated
Sasan-ian
every-day
domin-ance
Con-stitu-tion-al
religi-ous
sever-al
Manfredi
re-lev-ance
re-cipi-ent
pro-duct-iv-ity
turtle
Morocco
ferman
Maghreb-ian
algérien
stand-ard
systems
Nicolaï
Mouton
mauritani-en
Gotho-burg-ensis
socio-linguist-ique
plur-al
archiv-al
Arab-ian
drop-ped
dihāt
de-velop-ed
ṣuḥbat
kitāba
kitābat
com-mercial
eight-eenth
region
Senegal
mechan-ics
Maur-itan-ia
Ḥassān-iyya
circum-cision
cor-relation
labio-velar-ization
vowel
vowels
cert-ain
īggīw
series
in-tegrates
dur-ative
inter-dent-als
gen-itive
Tuareg
tălămut
talawmāyət
part-icular
part-icular-ly
con-diment
vill-age
bord-er
polit-ical
Wiesbaden
Uni-vers-idad
Geuthner
typo-logie
Maur-itanie
nomades
Maur-itan-ian
dia-lecto-logy
Sahar-iennes
Uni-vers-ity
de-scend-ants
NENA-speak-ing
speak-ing
origin-al
re-captured
in-habit-ants
ethnic
minor-it-ies
drama-tic
local
long-stand-ing
regions
Nineveh
settle-ments
Ṣəndor
Mandate
sub-stitut-ing
ortho-graphy
re-fer-enced
origin-ate
twenti-eth
typ-ic-al-ly
Hobrack
never-the-less
character-ist-ics
character-ist-ic
masc-uline
coffee
ex-clus-ive-ly
verb-al
re-ana-ly-se-d
simil-ar-ities
de-riv-ation
im-pera-tive
part-iciple
dis-ambi-gu-ation
dis-ambi-gu-a-ing
phen-omen-on
phen-omen-a
traktar
com-mun-ity
com-mun-ities
dis-prefer-red
ex-plan-ation
con-struction
wide-spread
us-ual-ly
region-al
Bulut
con-sider-ation
afro-asia-tici
Franco-Angeli
Phono-logie
Volks-kundliche
dia-lectes
dia-lecte
select-ed
dis-appear-ance
media
under-stand-able
public-ation
second-ary
e-ject-ive
re-volu-tion
re-strict-ive
Gasparini
mount-ain
mount-ains
yellow
label-ing
trad-ition-al-ly
currently
dia-chronic
}
\hyphenation{
affri-ca-te
affri-ca-tes
com-ple-ments
homo-phon-ous
start-ed
Meso-potam-ian
morpho-phono-logic-al-ly
morpho-phon-em-ic-s
Palestin-ian
re-present-ed
Ki-nubi
ḥawār-iyy-ūn
archa-ic-ity
fuel-ed
de-velop-ment
pros-od-ic
Arab-ic
in-duced
phono-logy
possess-um
possess-ive-s
templ-ate
spec-ial
espec-ial-ly
nat-ive
pass-ive
clause-s
potent-ial-ly
Lusignan
commun-ity
tobacco
posi-tion
Cushit-ic
Middle
with-in
re-finit-iz-ation
langu-age-s
langu-age
diction-ary
glossary
govern-ment
eight
counter-part
nomin-al
equi-valent
deont-ic
ana-ly-sis
Malt-ese
un-fortun-ate-ly
scient-if-ic
Catalan
Occitan
ḥammāl
cross-linguist-ic-al-ly
predic-ate
major-ity
ignor-ance
chrono-logy
south-western
mention-ed
borrow-ed
neg-ative
de-termin-er
European
under-mine
detail
Oxford
Socotra
numer-ous
spoken
villages
nomad-ic
Khuze-stan
Arama-ic
Persian
Ottoman
Ottomans
Azeri
rur-al
bi-lingual-ism
borrow-ing
prestig-ious
dia-lects
dia-lect
allo-phone
allo-phones
poss-ible
parallel
parallels
pattern
article
common-ly
respect-ive-ly
sem-antic
Moroccan
Martine
Harrassowitz
Grammatic-al-ization
grammatic-al-ization
Afro-asiatica
Afro-asiatic
continu-ation
Semit-istik
varieties
mono-phthong
mono-phthong-ized
col-loquial
pro-duct
document-ary
ex-ample-s
ex-ample
termin-ate
element-s
Aramaeo-grams
Centr-al
idioms
Arab-ic
Dadan-it-ic
sub-ordin-ator
Thamud-ic
difficult
common-ly
Revue
Bovingdon
under
century
attach
attached
bundle
graph-em-ic
graph-emes
cicada
contrast-ive
Corriente
Andalusi
Kossmann
morpho-logic-al
inter-action
dia-chroniques
islámica
occid-ent-al-ismo
dialecto-logie
Reichert
coloni-al
Milton
diphthong-al
linguist-ic
linguist-ics
affairs
differ-ent
phonetic-ally
kilo-metres
stabil-ization
develop-ments
in-vestig-ation
Jordan-ian
notice-able
level-ed
migrants
con-dition-al
certain-ly
general-ly
especial-ly
af-fric-ation
Jordan
counter-parts
com-plication
consider-ably
inter-dent-al
com-mun-ity
inter-locutors
com-pon-ent
region-al
socio-historical
society
simul-taneous
phon-em-ic
roman-ization
Classic-al
funeral
Kurmanji
pharyn-geal-ization
vocab-ulary
phon-et-ic
con-sonant
con-sonants
special-ized
latter
latters
in-itial
ident-ic-al
cor-relate
geo-graphic-al-ly
Öpengin
Kurd-ish
in-digen-ous
sunbul
Christ-ian
Christ-ians
sekin-în
fatala
in-tegration
dia-lect-al
Matras
morpho-logy
in-tens-ive
con-figur-ation
im-port-ant
com-plement
ḥaddād
e-merg-ence
Benjmamins
struct-ure
em-pir-ic-al
Orient-studien
Anatolia
American
vari-ation
Jastrow
Geoffrey
Yarshater
Ashtiany
Edmund
Mahnaz
En-cyclo-pædia
En-cyclo-paedia
En-cyclo-pedia
Leiden
dia-spora
soph-is-ic-ated
Sasan-ian
every-day
domin-ance
Con-stitu-tion-al
religi-ous
sever-al
Manfredi
re-lev-ance
re-cipi-ent
pro-duct-iv-ity
turtle
Morocco
ferman
Maghreb-ian
algérien
stand-ard
systems
Nicolaï
Mouton
mauritani-en
Gotho-burg-ensis
socio-linguist-ique
plur-al
archiv-al
Arab-ian
drop-ped
dihāt
de-velop-ed
ṣuḥbat
kitāba
kitābat
com-mercial
eight-eenth
region
Senegal
mechan-ics
Maur-itan-ia
Ḥassān-iyya
circum-cision
cor-relation
labio-velar-ization
vowel
vowels
cert-ain
īggīw
series
in-tegrates
dur-ative
inter-dent-als
gen-itive
Tuareg
tălămut
talawmāyət
part-icular
part-icular-ly
con-diment
vill-age
bord-er
polit-ical
Wiesbaden
Uni-vers-idad
Geuthner
typo-logie
Maur-itanie
nomades
Maur-itan-ian
dia-lecto-logy
Sahar-iennes
Uni-vers-ity
de-scend-ants
NENA-speak-ing
speak-ing
origin-al
re-captured
in-habit-ants
ethnic
minor-it-ies
drama-tic
local
long-stand-ing
regions
Nineveh
settle-ments
Ṣəndor
Mandate
sub-stitut-ing
ortho-graphy
re-fer-enced
origin-ate
twenti-eth
typ-ic-al-ly
Hobrack
never-the-less
character-ist-ics
character-ist-ic
masc-uline
coffee
ex-clus-ive-ly
verb-al
re-ana-ly-se-d
simil-ar-ities
de-riv-ation
im-pera-tive
part-iciple
dis-ambi-gu-ation
dis-ambi-gu-a-ing
phen-omen-on
phen-omen-a
traktar
com-mun-ity
com-mun-ities
dis-prefer-red
ex-plan-ation
con-struction
wide-spread
us-ual-ly
region-al
Bulut
con-sider-ation
afro-asia-tici
Franco-Angeli
Phono-logie
Volks-kundliche
dia-lectes
dia-lecte
select-ed
dis-appear-ance
media
under-stand-able
public-ation
second-ary
e-ject-ive
re-volu-tion
re-strict-ive
Gasparini
mount-ain
mount-ains
yellow
label-ing
trad-ition-al-ly
currently
dia-chronic
}
  \togglepaper[1]%%chapternumber
}{}

\begin{document}
\maketitle





\section{Overview of concepts and terminology}


\subsection{Jespersen’s cycle}


In the past few decades there has been steadily intensifying interest among linguists in historical developments in the expression of \isi{negation}, with particular attention given to the fact that these developments typically give the appearance of being cyclical in nature. We can date the beginning of this sustained interest to Dahl's (\citeyear{Dahl1979}) typological survey of \isi{negation} patterns in the world’s languages, in which he coined the term \textsc{Jespersen’s} \textsc{cycle}\footnote{The name was chosen in recognition of the early identification of this phenomenon by the Danish linguist Otto Jespersen in a (\citeyear{Jespersen1917}) \isi{article}, though others did identify the same set of changes earlier: \citet{Meillet1912}, for example, but also, significantly for the present work, \citet{Gardiner1904}, who observed a parallel set of changes in \ili{Coptic} and \ili{Arabic} as well as \ili{French} (cf. \citealt{Auwera2009}).} for what is by now the best-known set of developments in this domain: the replacement of an original negative morpheme with a newly grammaticalized alternative, after a period in which the two may co-occur, prototypically resulting in a word-order shift from preverbal to postverbal \isi{negation}. The best-known examples of \isi{Jespersen’s cycle} (both supplied, among others, by Jespersen himself in his \citeyear{Jespersen1917} work) come from the history of \ili{English} \REF{eng}, and \ili{French} \REF{fr}.

\ea\label{eng}
{{English} \citep[9]{Jespersen1917}}\\
\ea{Stage I – \ili{Old English}}\\
\gll ic \textbf{ne} secge\\
     \textsc{1sg}  \textsc{neg} \textsc{\textup{say}}.\textsc{prs.1sg}  \\
\glt ‘I do not say.’

\ex{Stage II – \ili{Middle English}}\\
\gll I \textbf{ne} seye \textbf{not}\\
     \textsc{1sg}  \textsc{neg} \textsc{\textup{say.}}\textsc{prs.1sg} \textsc{neg}  \\
\glt ‘I do not say.’


\ex{Stage III – \ili{Early Modern} English}\\
 I say \textbf{not}\\
\z
\z

\ea\label{fr}
{{French} \citep[7]{Jespersen1917}}\\
\ea{Stage I – \ili{Old French}}\\
\gll jeo \textbf{ne} di\\
     \textsc{1sg} \textsc{neg} \textsc{\textup{say}}.\textsc{prs.1sg}  \\
\glt  ‘I do not say.’

\ex{Stage II – contemporary written \ili{French}}\\
\gll Je \textbf{ne} dis \textbf{pas}\\
     \textsc{1sg} \textsc{neg} \textsc{\textup{say}}.\textsc{prs.1sg} \textsc{neg}  \\
\glt ‘I do not say.’

\ex{Stage III – contemporary colloquial \ili{French}}\\
\gll Je dis \textbf{pas}\\
     \textsc{1sg} \textsc{\textup{say}}.\textsc{prs.1sg} \textsc{neg}  \\
\glt ‘I do not say.’
\z
\z

More recently, \isi{Jespersen’s cycle} has come to be the subject of \isi{intensive} investigation, especially in the languages of Europe (e.g. \citealt{BerniniRamat1992}; \citeyear{BerniniRamat1996}; \citealt{WillisBreitbarth2013}), but also beyond (e.g. \citealt{Lucas2007}; \citeyear{Lucas2009,Lucas2013}; \citealt{LucasLash2010}; \citealt{DevosAuwera2013}; \citealt{AuweraVossen2015}; \citeyear{AuweraVossen2016,AuweraVossen2017}), with a picture emerging of a marked propensity for instances of \isi{Jespersen’s cycle} to be areally distributed, as we will see below in the discussion of \isi{Jespersen’s cycle} in \ili{Arabic} and its contact languages (§\ref{sec:dev}).

While \isi{Jespersen’s cycle} is the best known, best studied, and perhaps crosslinguistically most frequently occurring set of changes in the expression of \isi{negation}, two other important types of changes must also be mentioned here: \isi{Croft’s cycle}, and changes to \isi{indefinites} in the scope of \isi{negation}.



\subsection{Croft’s cycle}


In a typologically-oriented (\citeyear{Croft1991}) \isi{article}, Croft reconstructs from synchronic descriptions of a range of languages a recurring set of cyclical changes in the expression of \isi{negation}. Unlike \isi{Jespersen’s cycle}, in which the commonest sources of new negators are nominal elements expressing minimal quantities, such as ‘step’ or ‘crumb’, or generalizing pronouns like ‘(any)thing’, \textsc{Croft’s} \textsc{cycle} (named for Croft by \citealt{Kahrel1996}), involves the evolution of new markers of \isi{negation} developed from negative \isi{existential} particles. Croft (\citeyear[6]{Croft1991}) distinguishes the following three types of languages:

\begin{itemize}[noitemsep]
	\item[] Type A: the verbal negator is also used to negate \isi{existential} predicates
	\item[] Type B: there is a special negative \isi{existential} predicate distinct from the verbal negator
	\item[] Type C: there is a special negative \isi{existential} predicate, and this form is also used to negate verbs
\end{itemize}

For Type A, Croft (\citeyear[7]{Croft1991}) cites the example of \ili{Syrian} \ili{Arabic} \textit{mā} \textit{fī} ‘there is not’ and \textit{mā} \textit{baʕref} ‘I do not know’ among others. For Type B he cites (\citeyear[9]{Croft1991}), among other examples, the contrast between the \ili{Amharic} negative \isi{existential} \textit{yälläm} (affirmative \isi{existential} \textit{allä}) and regular verbal \isi{negation} \textit{a(l)…-əm}. For Type C he cites (\citeyear[11--12]{Croft1991}) \ili{Manam} (Oceanic) among other languages, giving the example in \REF{man}.


\ea\label{man}
{\ili{Manam} (\citealt{Croft1991}: 11–12; \citealt{Lichtenberk1983}: 385, 499)}\\
\ea{Verbal negation}\\
\gll tágo u-lóŋo\\
     \textsc{neg(.exs)} \textsc{1sg.real-}\textsc{\textup{hear}}  \\
\glt ‘I did not hear.’

\ex{Negative \isi{existential} predicate}\\
\gll anúa-lo tamóata tágo [*i-sóaʔi]\\
     village-in person \textsc{neg.exs} \textsc{[3sg.real-exs]}\\
\glt  ‘There is no one in the village.’
\z
\z

A number of languages also exhibit variation between two of the types: A {\textasciitilde} B, B {\textasciitilde} C, and C {\textasciitilde} A. This indicates a cyclical development A > B > C > A, in which a special negative \isi{existential} predicate arises in a language (A > B), comes to function also as a verbal negator (B > C), and is then felt to be the negator proper, requiring supplementation by a positive \isi{existential} predicate in \isi{existential} constructions (C > A).

While \isi{Croft’s cycle} is less common than \isi{Jespersen’s cycle}, and has not been shown to have occurred in its entirety in the recorded history of any language, we mention it here because recent work by Wilmsen (\citeyear[174–176]{Wilmsen2014}; \citeyear{Wilmsen2016}), discussed below in §\ref{sec:jesp}, argues for several instances of \isi{Croft’s cycle} in the history of \ili{Arabic}.



\subsection{Changes to indefinites in the scope of negation}\label{sec:ind}


The final major set of common changes to be dealt with here involve \isi{indefinite pronouns} and \isi{quantifiers} in the scope of \isi{negation}. Here too cyclical patterns are commonplace, and these changes have been labelled “the argument cycle” \citep{Ladusaw1993} or “the \isi{quantifier} cycle” \citep{Willis2011}. What we find is that certain items, typically \isi{quantifiers} such as ‘all’ or ‘one’ or generic nouns such as ‘person’ or ‘thing’, are liable to develop restrictions on the semantic contexts in which they can occur, namely what are referred to as either downward-entailing or non-veridical contexts (see \citealt{Giannakidou1998} for details and the distinction between the two). In essence, this means interrogative, \isi{conditional}, and negative clauses, as well as the complements of \isi{comparative} and \isi{superlative} adjectives, but not ordinary affirmative declarative clauses. Items that are restricted to appearing in such contexts, such as \ili{English} \textit{ever} (consider the ungrammaticality of, e.g., \textit{*I’ve} \textit{ever} \textit{been} \textit{to} \textit{Japan}), are generally termed \textsc{negative} \textsc{polarity} \textsc{items}. Often, however, we find negative polarity items whose appearance is restricted to a subset of these contexts, and much the most common restriction is to negative contexts only. Items with this narrower distribution, such as the \ili{English} degree-adverbial phrase \textit{one} \textit{bit}, are generally termed strong negative polarity items and those with the wider downward-entailing/non-veridical distribution may be termed weak negative polarity items in contrast.

A commonly recurring diachronic tendency of such items is that they become stronger over time. That is, an item goes from having no restrictions, to being a weak \isi{negative polarity item}, to being a strong \isi{negative polarity item}, to eventually being itself inherently negative. The best-known instance of this progression comes from \ili{French} \textit{personne} ‘nobody’ and \textit{rien} ‘nothing’. These derive from the ordinary, unrestricted \ili{Latin} generic nouns \textit{persona} ‘person’ and \textit{rem} ‘thing’ and still behaved as such in medieval \ili{French}, as in \REF{med}.


\ea\label{med}
{       {Medieval} \ili{French} (\citealt{Hansen2013}: 72; \citealt{Buridant2000}: 610)} \\
\gll Et si vous dirai une rien.\\
     and so \textsc{2pl} say.\textsc{fut.1sg} \textsc{indf.sg.f} thing\\
\glt ‘And so I’ll tell you a thing.’
\z

In later medieval \ili{French} they grammaticalized as \isi{indefinite pronouns} and began to acquire a weak negative polarity distribution, as in the interrogative example in \REF{c13}.

\ea\label{c13}
{Thirteenth-century \ili{French} (\citealt{Hansen2013}: 72; \citealt{Buridant2000}: 610)} \\
\gll As tu rien fet?\\
     \textsc{aux.2sg} \textsc{2sg} anything do.\textsc{ptcp.pst}\\
\glt ‘Have you done anything?’
\z

In present-day \ili{French} these items have become essentially inherently negative, as shown in \REF{cont}. They can no longer appear in interrogative, \isi{conditional} or main declarative clauses with an affirmative interpretation \citep[73]{Hansen2013}, though an affirmative interpretation remains possible in \isi{comparative} complements, albeit largely in frozen expressions, as in \textit{rien} \textit{au} \textit{monde} ‘anything in the world’ in \REF{cont1}.

\ea\label{cont}
{       Contemporary \ili{French} \citep[68]{Hansen2013}} \\
\gll Qui t’ a vu? Personne!\\
     who 2\textsc{sg.obj} \textsc{aux.3sg} see.\textsc{ptcp.pst} nobody\\
\glt ‘Who saw you? Nobody!’
\z

\ea\label{cont1}
{       Contemporary \ili{French} \citep[73]{Hansen2013}} \\
\gll J’ aime le vin mieux que rien au monde.\\
     \textsc{1sg} like.\textsc{prs} \textsc{def.sg.m} wine better than anything in+\textsc{def.sg.m} world\\
\glt ‘I like wine better than anything in the world.’
\z

Note that \ili{French} \textit{rien} ‘nobody’ and \textit{personne} ‘nothing’, like their equivalents in many other \ili{Romance} varieties (e.g. \ili{Italian} \textit{niente} and \textit{nessuno}), are not straightforward negative \isi{quantifiers} like \ili{English} \textit{nobody} and \textit{nothing}, even disregarding their behaviour in contexts such as \REF{cont1}. This is because \ili{French}, like many other languages but unlike {Standard} \ili{English}, {Standard} \ili{German}, \ili{Classical Latin} etc., exhibits \textsc{negative} \textsc{concord}. This refers to the fact that when two (or more) elements which express \isi{negation} on their own co-occur in a clause, the result is not logical double \isi{negation} (i.e. a positive) but a single logical negative, as illustrated in \REF{cont2}.

\ea\label{cont2}
{       Contemporary \ili{French} \citep[69]{Hansen2013}} \\
\gll Personne n’ a rien dit\\
     nobody \textsc{neg} \textsc{aux}.\textsc{prs.3sg} nothing say\textsc{.ptcp.pst}\\
\glt ‘Nobody said anything.’
\z

Items which have this unstable behaviour are distinguished from straightforwardly negative items by the term \textsc{n-word} (coined by \citealt{Laka1990}; see also \citealt{Giannakidou2006}). We will see in §\ref{sec:sco} that these distinctions and terminology are helpful in understanding developments in varieties of \ili{Arabic} and its contact languages that directly parallel those described above for \ili{French}.


\section{Developments in the expression of clausal negation}\label{sec:dev}


\subsection{Arabic}





\subsubsection{Synchronic description}



One of the most striking ways that a number of spoken \ili{Arabic} varieties differ from \ili{Classical} and \ili{Modern Standard} \ili{Arabic} is in the expression of \isi{negation}. In \ili{Classical} and \ili{Modern Standard} \ili{Arabic}, and in the majority of varieties spoken outside of North Africa, \isi{negation} is exclusively preverbal, with the basic verbal negator in the spoken varieties being \textit{mā}, as in the \ili{Damascus} \ili{Arabic} example in \REF{dam}.


\ea\label{dam}
{       \ili{Damascus} \ili{Arabic} \citep[328]{Cowell1964}}\\
\gll hayy masʔale \textbf{mā} bəḍḍaḥḥək\\
     \textsc{dem.f} matter \textsc{neg} laugh.\textsc{caus.impf.ind.3sg.m}\\
\glt ‘This is not a laughing matter.’ (lit. ‘does not cause laughter’)
\z

But in the varieties spoken across the whole of coastal North Africa and into the southwestern Levant, as well as in parts of the southern Arabian Peninsula (see \citealt{Diem2014}; \citealt{Lucas2018} for more precise details), \isi{negation} is bipartite, with preverbal \textit{mā} joined by an enclitic \textit{-š} which follows any direct or indirect pronominal object clitics, as in \REF{cai}.

\protectedex{
\ea\label{cai}
{       \ili{Cairo} \ili{Arabic} (advertising slogan)}\\
\gll banda \textbf{ma} yitʔal-lahā-\textbf{š} laʔ\\
     Panda \textsc{neg} say.\textsc{pass.impf.3sg.m-dat.3sg.f-neg} no\\
\glt ‘You don’t say “no” to Panda.’ (lit. ‘Panda, “no” is not said to it’)
\z
}
Finally, in a subset of the varieties that permit the bipartite construction in \REF{cai}, a purely postverbal construction is also possible, as in the \ili{Palestinian} \ili{Arabic} example in \REF{pal}.

\ea\label{pal}

{       \ili{Palestinian} \ili{Arabic} \citep[147]{Seeger2013grammar}}\\
\gll badaḫḫin\textsuperscript{i}-\textbf{š}\\
     smoke.\textsc{impf.ind.1sg-neg}\\
\glt ‘I don’t smoke.’
\z




\subsubsection{Jespersen or Croft?}\label{sec:jesp}



There is near unanimous \isi{agreement} among those who have considered the matter that the bipartite construction illustrated in \REF{cai} arose from the preverbal construction via \isi{grammaticalization}, phonetic reduction, and \isi{cliticization} of \textit{šayʔ} ‘thing’, and that the purely postverbal construction in \REF{pal} in turn arose from the bipartite construction via omission of the original negator \textit{mā}. As such, Lucas (\citeyear{Lucas2007,Lucas2009,Lucas2018}) and \citet{Diem2014}, among many others, view this as a paradigmatic case of \isi{Jespersen’s cycle}.

The only dissenting voice is that of Wilmsen (\citeyear{Wilmsen2013,Wilmsen2014}), who describes the parallels between the \ili{Arabic} data and that of well known cases of \isi{Jespersen’s cycle} such as \ili{French} as being “dutifully mentioned by all” (\citeyear[117]{Wilmsen2014}) who write on the topic. \citet{Wilmsen2014} turns the agreed etymology of negative \textit{{}-š} on its head by arguing: (i) that the original form in \ili{Arabic} was \textit{šī}, not \textit{šayʔ};\footnote{\citet{Wilmsen2014} also attempts to trace his etymology back further to the Proto-\ili{Semitic} third-person pronouns. Apart from the implausibility of the putative semantic shift from \isi{definite} pronoun to indefinite determiner, this reconstruction is untenable on phonological grounds (see \citealt{Al-Jallad2015review} for details).} (ii) that at an early stage this form had the full range of functions that we observe for it in different \ili{Arabic} dialects today (\isi{existential} predicate, indefinite determiner, interrogative particle; see \citealt{Wilmsen2014}: ch. 3, 122–123); (iii) that this element was then reanalysed as a negative particle; and (iv) \textit{šī/šayʔ} as a content word ‘thing’ is a later development of the function word – an instance of degrammaticalization. For a discussion of some of the numerous difficulties with these proposals, see \citet{Al-Jallad2015review}, \citet{Pat-El2016}, \citet{Souag2016review}, and \citet{Lucas2018}.

A specific element of Wilmsen’s proposals that we need to consider in some detail here before we proceed is his suggestion that, while in his view we should not see the developments in \ili{Arabic} as an instance of \isi{Jespersen’s cycle}, we can discern in them an instance of \isi{Croft’s cycle}. As we will see below, this suggestion involves a distortion or misunderstanding of both the \ili{Arabic} data and the sorts of patterns that constitute genuine instances of \isi{Croft’s cycle}, but the proposal has some prima facie plausibility, because of the existence in some dialects of the south and east of the Arabian Peninsula of an \isi{existential} predicate \textit{šī/šē/šay}, as in \REF{oma}.


\ea\label{oma}
{       \ili{Northern} \ili{Omani} \ili{Arabic} \citep[92]{Eades2009}}\\
\gll ḥmīr šē l-ḥmīr barra\\
     donkey.\textsc{pl} \textsc{exs} \textsc{def-}donkey.\textsc{pl} outside\\
\glt ‘There were donkeys… the donkeys were outside.’
\z

Note that a similar element \textit{śī} [ɬiː], with the same \isi{existential} function, is found in the Modern {South Arabian} languages (\ili{MSAL}) of \isi{Yemen} and {Oman}, as in \REF{meh}, from \ili{Mehri} of \isi{Yemen}.

\ea\label{meh}
{       \ili{Mehri} of \isi{Yemen} \citep[31]{Watson2011SAYemeni}}\\
\gll śī fśē \\
     \textsc{exs} lunch\\
\glt ‘Is there any lunch?’
\z

Though Wilmsen (\citeyear{Wilmsen2014}: 126; \citeyear{Wilmsen2017}: 298--301) seems to view \ili{Arabic} \textit{šī} and Modern {South Arabian} \textit{śī} as cognates, it is more likely that the presence of this item in the one set of varieties is the result of \isi{transfer} from the other (cf. \citealt{Al-Jallad2015review}). The direction of \isi{transfer} is unclear, however. At first glance, the fact that \textit{śī} as an affirmative \isi{existential} is found in essentially all of the \ili{MSAL} spoken on the Arabian Peninsula, which have a long history of \isi{intensive} contact with \ili{Arabic}, but not in \ili{Soqoṭri}, spoken on the island of Soqotra, where contact with \ili{Arabic} is more recent and less \isi{intensive} \citep{Simeone-Senelle2003}, would appear to suggest that this is an innovation within \ili{Arabic} originally, which was then transferred to just those \ili{MSAL} with which there was most contact. On the other hand, the precise situation in \ili{Soqoṭri} is perhaps instructive. Here the affirmative \isi{existential} predicate is a unique form \textit{ino}, while the negative \isi{existential} predicate is \textit{biśi} \citep[1108]{Simeone-Senelle2011}. It is conceivable that the latter is a borrowing from \ili{Arabic}, since affirmative existentials in \textit{b-} are widespread in the \ili{Arabic} dialects of \isi{Yemen}. But a negative \isi{existential} predicate \textit{bīši} or similar is completely unattested in the \ili{Yemeni} data provided by Behnstedt (\citeyear[346–348]{Behnstedt2016Yemen}). This suggests, therefore, that: (i) \isi{existential} \textit{śī} is an original feature of \ili{MSAL}; (ii) \ili{Soqoṭri} is an example of a Type B language in Croft’s typology, having innovated a new affirmative \isi{existential} predicate \textit{ino}, such that there is a special negative \isi{existential} predicate that is neither identical to the verbal negator, nor simply a combination of the verbal negator with the affirmative \isi{existential} predicate; and (iii) \textit{šī} as an \isi{existential} predicate in \ili{Arabic} dialects is the result of \isi{transfer} of \ili{MSAL} \textit{śī}.

This scenario is supported by the distribution of \isi{existential} \textit{šī} within \ili{Arabic} varieties: the only clear cases are in dialects of \isi{Yemen} and \ili{Oman} with a history of contact with \ili{MSAL}, and dialects of the \ili{Gulf} whose speakers are known to have migrated there from \isi{Yemen} or \ili{Oman} (such as \ili{Šiḥḥī}, §\ref{sec:key:kumz}). In various places Wilmsen tries to make a case for \isi{existential} uses of \textit{šī} outside this region, but this appears to be the result of confusion on his part between \textit{šī} as a \textit{bona} \textit{fide} \isi{existential} predicate and the \isi{existential} presupposition that will inevitably be associated with the use of \textit{šī} as an indefinite determiner (see, e.g., \citealt{Heim1988} on the semantics of indefinite noun phrases). For example, Wilmsen (\citeyear[123]{Wilmsen2014}) cites Caubet’s (\citeyear[123]{Caubet1993a}, \citeyear[280]{Caubet1993b}) \ili{Moroccan} \ili{Arabic} examples in \REF{mor} as evidence of an \isi{existential} use of \textit{šī} as far west as Morocco. But there is no justification for Wilmsen’s contradicting Caubet’s uncontroversial analysis of \textit{šī} as an indefinite determiner here: there are no \isi{existential} predicates in these examples – the existence of the referents of the indefinite noun phrases is presupposed, not asserted.

\ea\label{mor}
{\ili{Moroccan} \ili{Arabic} (\citealt[123]{Caubet1993a}, \citealt[280]{Caubet1993b})}\\
\ea\gll ši nās kayāklu-ha\\
     \textsc{indf} people eat.\textsc{impf.real.3pl-3sg.f}  \\
\glt ‘Some people eat it.’

\ex
\gll ši nās kaybɣēw əl-lbən\\
     \textsc{indf} people like.\textsc{impf.real.3pl} \textsc{def-}milk\\
\glt  ‘Some people like milk.’
\z
\z

Nevertheless, \textit{šī} does function as an \isi{existential} predicate in a few \ili{Arabic} varieties. The question, then, is whether a negated form of this predicate participates in a version of \isi{Croft’s cycle}, as Wilmsen maintains.

For the vast majority of \ili{Arabic} varieties the answer is a clear no: these varieties straightforwardly belong to Type A of Croft’s typology. The verbal negator (\textit{mā}, \textit{mā…-š}, or \textit{{}-š}) is also used to negate \isi{existential} predicates, as illustrated in \REF{cair} for \ili{Cairo} \ili{Arabic}.

\ea\label{cair}
{\ili{Cairo} \ili{Arabic}, personal knowledge} \\
\ea\gll ma ʕamalt\textsuperscript{i}{}-š ḥāga\\
     \textsc{neg} do.\textsc{prf.1sg-neg} thing  \\
\glt ‘I didn’t do anything.’

\ex\gll ma fī-š ḥāga\\
     \textsc{neg} \textsc{exs-neg} \textsc{\textup{thing}}\\
\glt  ‘There is nothing.’
\z
\z

Wilmsen (\citeyear[173–175]{Wilmsen2014}) suggests that Type B and Type C constructions can also be found, however. For Type B (“there is a special negative \isi{existential} predicate, distinct from the verbal negator”; \citealt{Croft1991}: 6), he cites \ili{Sana'a} \textit{māšī} and \ili{Moroccan} \textit{māši}. \ili{Sana'a} \textit{māšī} is certainly a negative \isi{existential} predicate. But there is nothing special about it – it is a paradigmatic Type A construction, with the \isi{negation} of the \isi{existential} predicate (\textit{šī}) performed by the verbal negator (\textit{mā}). \ili{Moroccan} \textit{māši}, on the other hand, is the negator for nominal predicates (equivalent to \textit{muš/miš/mū} in dialects east of Morocco). It is not a negative \isi{existential} predicate at all, and, as discussed above, the /ši/ component of this item does not function as an \isi{existential} in \ili{Moroccan}, unlike in \ili{Sana'a} and other southern Arabian varieties. The existence of \textit{m}\-\textit{āši} in \ili{Moroccan} \ili{Arabic} is thus irrelevant to the question of whether this constitutes a Type B variety.\footnote{Van Gelderen (\citeyear{VanGelderen2018}) argues that the definition of \isi{Croft’s cycle} should be expanded to encompass cases in which new negators arise from the univerbation of verbal negators with copulas and auxiliaries, as well as existentials. Wilmsen's (\citeyear{Wilmsen2014}) presentation of \isi{Croft’s cycle} makes no mention of any predicates other than existentials participating in the cycle, however.}  \ili{Moroccan} is a Type A variety: the positive \isi{existential} predicate is \textit{kāyn} and it is negated with the ordinary \ili{Moroccan} verbal negator \textit{ma…-š} \citep{Caubet2011}.

Wilmsen’s identification of \ili{Arabic} varieties of Type C (“there is a special negative \isi{existential} predicate, which is identical to the verbal negator”; \citealt{Croft1991}: 6) depends on the idea that the \ili{Arabic} predicate negator \textit{māši/muš/miš/mū} is a negative \isi{existential} predicate, which, as we have seen, it is not. If it were, it would be true that there are \ili{Arabic} varieties that are optionally of Type C, since in \ili{Cairo} \ili{Arabic}, among other varieties, it is possible to negate verbs with \textit{miš} instead of the usual \textit{ma…-š}, as \citet{Mughazy2003} and others have pointed out. But \ili{Cairo} \textit{miš} (and \ili{Moroccan} \textit{māši}) are not negative \isi{existential} predicates, and there is no evidence to suggest they ever were. Moreover, since the \ili{Sana'a} negative \isi{existential} predicate \textit{māšī} also does not seem to be able to function as a verbal negator, there is little apparent merit in Wilmsen's (\citeyear{Wilmsen2014}) attempt to recast the history of \isi{negation} in \ili{Arabic} as an instance of \isi{Croft’s cycle}.\footnote{This is not to deny, however, that some \ili{Arabic} dialects show some incipient Type B tendencies of a different kind. For example, Behnstedt (\citeyear[347]{Behnstedt2016Yemen}) cites the northern \ili{Yemeni} dialects of Rās Maḥall as-Sūdeh, Ḥammām ʿAlī and Afk, as varieties in which different morphemes are used in positive and negative existentials, albeit the negative construction used in each case is identical to that used for ordinary verbal \isi{negation}. In a different context, Stefano Manfredi (personal communication) points out that many urban speakers of \ili{Sudanese} \ili{Arabic} use the item \textit{māfīš}, borrowed from \ili{Egyptian Arabic}, as a negative \isi{existential}, while ordinary verbal \isi{negation} is performed with preverbal \textit{mā} alone (without postverbal \textit{{}-š}).}




\subsubsection{Internal or external?}



It is clear from the above discussion that there is no reason to doubt the majority view of the emergence of negative \textit{{}-š} as an instance of \isi{Jespersen’s cycle}. What is less clear and more controversial is the question of whether language contact played a role in triggering these developments, or whether this was a purely internal phenomenon (cf. \citealt{Diem2014}: 11–12). This is an issue about which it is impossible to be certain given our present state of knowledge. \citet{LucasLash2010} make the case that contact did play a triggering role, however, and also provide arguments against the widely held view that, in the words of Lass (\citeyear[209]{Lass1997}), “an endogenous explanation of a phenomenon is more parsimonious [than one invoking contact – CL], because endogenous change must occur in any case, whereas borrowing is never necessary” (cf. also \citealt{Lucas2009}: 38–43). Aside from this generalized reluctance to invoke contact in explanations of linguistic change unless absolutely necessary, another factor that is likely operative in the preference for seeing the \ili{Arabic} developments as a purely internal phenomenon is ignorance of the wider picture of negative developments in \ili{Arabic} and its contact languages. It is scarcely an exaggeration to say that everywhere an \ili{Arabic} variety with bipartite \isi{negation} is spoken, there is (or was) a contact language that also has bipartite \isi{negation}, and – just as importantly – wherever \ili{Arabic} dialects have only a single marker of \isi{negation}, the local contact languages do too. The picture is similar in Europe, Ethiopia \citep{Lucas2009}, Vietnam (\citealt{AuweraVossen2015}), and many other places besides. There can therefore be no doubt that negative constructions, and especially bipartite \isi{negation} (and hence \isi{Jespersen’s cycle} more generally), are particularly prone to diffusing through languages in contact. In the following sections I will briefly survey apparent instances of \isi{transfer} of bipartite or postverbal \isi{negation} in \ili{Arabic} and \ili{Coptic}, \ili{Arabic} and \ili{MSAL}, \ili{Arabic} and \ili{Kumzari}, \ili{Arabic} and \ili{Berber}, and \ili{Arabic} and \ili{Domari}. For more details see Lucas (\citeyear{Lucas2007,Lucas2009,Lucas2013}) and \citet{LucasLash2010}.



\subsection{Arabic and Coptic}


Based on an examination of evidence from Judaeo-\ili{Arabic} documents preserved in the Cairo Genizah, among other sources of evidence, \citet{Diem2014} comes to the conclusion that the \ili{Arabic} bipartite negative construction found across coastal North Africa originated in Egypt between the tenth and eleventh centuries. This chronology and point of origin conforms closely with the conclusions I have drawn on this point in my own work (\citealt{Lucas2007}; \citeyear{Lucas2009}; \citealt{LucasLash2010}), except that I have argued that what triggered the development of bipartite \isi{negation} in Egypt was contact with \ili{Coptic} (the name for the \ili{Egyptian} language from the first century CE onwards), which, at the relevant period, had a frequently occurring bipartite construction \textit{ən…an}, as illustrated in \REF{coptic}.

\ea\label{coptic}
{       \ili{Coptic} (\citealt{LucasLash2010}: 389)}\\
\gll \textbf{en} ti-na-tsabo-ou \textbf{an} e-amənte\\
     \textsc{neg} \textsc{1sg-fut-}teach-\textsc{3pl} \textsc{neg} on-hell\\
\glt ‘I will not teach them about hell.’
\z

The argument made in \citet{LucasLash2010} is that native speakers of \ili{Coptic} acquiring \ili{Arabic} as a second language must have encountered sentences negated with preverbal \textit{mā} only, but which also contained after the verb \textit{šī/šāy}, functioning either as an argument ‘(any)thing’ or an adverb ‘at all’,\footnote{\citet{Diem2014} makes the case that \textit{šī/šāy} had already developed an adverbial use at a very early stage, and that it is this adverbial use that should be seen as the form that was reanalysed as a negator.} and interpreted this as the second element of the bipartite negative construction that their first-language \ili{Coptic} predisposed them to expect. If this is correct, then the initial \isi{transfer} of bipartite \isi{negation} from \ili{Coptic} to \ili{Arabic} in Egypt should be understood as an instance of \isi{imposition} under \isi{source-language} agentivity, in the terms of Van Coetsem (\citeyear{VanCoetsem1988,VanCoetsem2000}), while the presence of bipartite \isi{negation} in the dialects spoken across the rest of coastal North Africa, and the southwestern Levant, should be understood as the result of contact between neighbouring dialects of \ili{Arabic}.



\subsection{Arabic and Modern South Arabian}


Diem (\citeyear[73]{Diem2014}) -- like Obler (\citeyear[148]{Obler1990}) and, following her, Lucas (\citeyear[416]{Lucas2007}) -- suggests that bipartite \isi{negation} in the southern Arabian Peninsula must have spread there from Egypt. This is conceivable, but historical evidence of significant early migration flows in this direction is lacking. The alternative explanation offered by \citet{LucasLash2010} is that bipartite \isi{negation} in the \ili{Arabic} dialects of this region is an independent parallel development, here triggered by contact with \ili{MSAL}, all mainland varieties of which have a bipartite negative construction of their own (or once had – some, such as \ili{Ḥarsūsi}, have largely progressed to stage III of \isi{Jespersen’s cycle} and lost the original preverbal negator), as illustrated in \REF{meh1} for \ili{Omani} \ili{Mehri}.

\ea\label{meh1}
{       \ili{Mehri} of \ili{Oman} \citep[23]{Johnstone1987}}\\
\gll \textbf{əl} təhɛləz b-ɛy \textbf{laʔ}\\
     \textsc{neg} nag.\textsc{impf.2sg.m} with-\textsc{1sg} \textsc{neg} \\
\glt ‘Don’t nag me!’
\z

If this is correct, then here too, exactly as with the \ili{Coptic}–\ili{Arabic} contact in the previous section, we must have had an instance of \isi{transfer} under \isi{source-language} agentivity, with \ili{MSAL}-dominant acquirers of \ili{Arabic} imposing a bipartite construction on their second-language \ili{Arabic} by reanalysing \textit{šī/šay} as a negator. The key point is that in all dialects in which \textit{šī/šay} functioned as an indefinite pronoun or adverb ‘at all’, the potential was there for reanalysis as the second element in a bipartite negative construction. But aside from in the dialects of Egypt and the southern Arabian Peninsula (and latterly dialects adjacent to Egyptian) this reanalysis never took place. Why the reanalysis did take place in Egypt and the southern Peninsula can be understood as being the result of the catalysing effect of contact with languages which themselves had a bipartite negative construction.\footnote{For further discussion of the details of these changes, including the issues of the semantics and positioning in the clause of the second negative element in each of the three languages, see Lucas \& Lash (\citeyear[395–401]{LucasLash2010}).}



\subsection{Arabic and Kumzari}\label{sec:key:kumz}


\ili{Kumzari} is an \ili{Iranian} language with heavy influence from both \ili{Arabic} and \ili{MSAL} that has only recently been described in detail (see \citealt{WalAnonbyforthcoming}). It is spoken on the Musandam Peninsula of northern \ili{Oman}, where its primary contact language of recent times has been the \ili{Šiḥḥī} variety of \ili{Arabic} (see \citealt{Bernabela2011} for a sketch grammar), which is clearly of the originally southern Arabian type described by Holes (\citeyear[18–32]{Holes2016}).

\ili{Šiḥḥī} \ili{Arabic} has no Jespersen stage-II (bipartite) negative construction, but it has both a typical eastern \ili{Arabic} stage-I construction with \textit{mā}, as in \REF{shi.a}, perhaps due to recent influence from other \ili{Gulf} \ili{Arabic} varieties, alongside a unique (for \ili{Arabic}) stage-III postverbal construction with \textit{{}-lu}, as in \REF{shi.b}. The latter construction is apparently a straightforward \isi{transfer} of the postverbal negator \textit{laʔ/lɔʔ} of \ili{MSAL} \REF{meh1}.


\ea
{\ili{Šiḥḥī} \ili{Arabic} \citep[87]{Bernabela2011}}\\
\ea\gll \textbf{mā} mšēt ḫaṣāb əl-yōm\label{shi.a}\\
     \textsc{neg} \textsc{\textup{go.}}\textsc{prf.1sg} Khasab \textsc{def-}day  \\
\glt ‘I didn’t go to Khasab today.’

\ex\gll yqōl-\textbf{lu} bass il-kilmatēn\label{shi.b}\\
     \textsc{\textup{say.}}\textsc{impf.3sg.m-}\textsc{neg} only \textsc{def-}\textsc{\textup{words.}}\textsc{du}\\
\glt ‘He doesn’t just say the two words.’
\z
\z

The \ili{Kumzari} negator is the typical \ili{Iranian} (and \ili{Indo-Iranian}) \textit{na}. What is less typical is that \textit{na} occurs postverbally in \ili{Kumzari}, as shown in \REF{kum}.

\ea\label{kum}
{       \ili{Kumzari} \citep[211]{WalAnonbyforthcoming}}\\
\gll mām-ō kōr bur \textbf{na}\\
     mother\textsc{{}-def} blind become.3\textsc{sg.real} \textsc{neg} \\
\glt ‘The mother didn’t become blind.’
\z

It seems very likely that contact with \ili{Šiḥḥī} \ili{Arabic} has played a role in this shift to postverbal \isi{negation}, though not enough is known about the historical sociolinguistics of these two speech communities to say with confidence which of the two languages the agents of this change were dominant in.



\subsection{Arabic and Berber}


\ili{Berber} languages are spoken from the oasis of \ili{Siwa} in western Egypt in the east, across to Morocco and as far south as Burkina Faso. The most southerly of the \ili{Berber} varieties – \ili{Tashelhiyt}, spoken in southern Morocco, \ili{Zenaga}, spoken in Mauritania, and \ili{Tuareg}, spoken in southern Algeria and Libya, Niger, {Mali} and Burkina Faso – have only preverbal \isi{negation}, as illustrated by the \ili{Tuareg} example in \REF{tua}.


\ea\label{tua}
{       \ili{Tuareg} \citep[10]{Chaker1996}}\\
\gll \textbf{ur} igle\\
     \textsc{neg} leave\textsc{.pfv.3sg.m}\\
\glt ‘He didn’t leave.’
\z

These languages have, until recently, either had little significant contact with \ili{Arabic}, or otherwise only with varieties such as \ili{Ḥassāniyya} that have only preverbal \isi{negation} with \textit{mā}. All other \ili{Berber} varieties which are in contact with \ili{Arabic} varieties with bipartite \isi{negation} also themselves have bipartite \isi{negation}, illustrated for \ili{Kabyle} (Algeria) in \REF{kab}, or, in a few cases, purely postverbal \isi{negation}, as in \ili{Awjila} (Libya), illustrated in \REF{awjneg}. The one exception is \ili{Siwa}, which negates with preverbal \textit{lā} alone – clearly a borrowing from a variety of \ili{Arabic}, though which variety is not clear (see \citealt{Souag2009} for further discussion).

\ea\label{kab}
{       \ili{Kabyle} \citep[25]{Rabhi1996}}\\
\gll \textbf{ul} ittaggad \textbf{kra}\\
     \textsc{neg} fear.\textsc{aor.3sg.m} \textsc{neg}\\
\glt ‘He is not afraid.’
\z

\ea\label{awjneg}
{       \ili{Awjila} \citep[82]{Paradisi1961}}\\
\gll akellim iššen-\textbf{ka} amakan \\
     servant know\textsc{.pfv.3sg.m-neg} place\\
\glt ‘The servant didn’t know the place.’
\z

\ea\label{}
{       \ili{Siwa} \citep[58]{Souag2009}}\\
\gll \textbf{lā} gā-nūsd-ak\\
     \textsc{neg} \textsc{fut-}come\textsc{.1pl-dat.2sg}\\
\glt ‘We won’t come to you.’
\z

{Different \ili{Berber} varieties have postverbal negators with a range of different forms, but in most cases they either derive from two apparently distinct Proto-\ili{Berber} items *kʲăra and *(h)ară(t), both meaning ‘thing’ \citep[332]{Kossmann2013book}, or are transparent loans of \ili{Arabic} \textit{šay/ši}. This fact, when combined with the respective geographical distributions of single preverbal and bipartite \isi{negation} in \ili{Arabic} and \ili{Berber} varieties, is sufficient to conclude that the presence of bipartite \isi{negation} in \ili{Berber} is in large part a result of \isi{calquing} the second element of the \ili{Arabic} construction, pace \citet{Brugnatelli1987} and \citet{Lafkioui2013reinventing} (see also \citealt{Kossmann2013book}: 334; and see \citealt{Lucas2007,Lucas2009} for more detailed discussion).\footnote{Another postverbal negator – \ili{Kabyle} \textit{ani} – derives from the word for ‘where’ \citep{Rabhi1992}, and so should perhaps be seen as more of an internal development, or at least less directly contact-induced. \ili{Tarifiyt} also has a postverbal negator \textit{bu}, whose etymology is uncertain, but which has also been transferred to the \ili{Moroccan} \ili{Arabic} dialect of Oujda \citep{Lafkioui2013bu}.} Given that, until recently, native speakers of \ili{Arabic} in the Maghreb acquiring \ili{Berber} as a second language will always have been greatly outnumbered by native speakers of \ili{Berber} learning \ili{Arabic} as a second language, we must assume that the agents of this change were \ili{Berber}-dominant speakers who made the change under \isi{recipient-language} agentivity in a process akin to what \citet{HeineKuteva2005} call polysemy copying and \isi{contact-induced grammaticalization} (see also, Leddy-Cecere, this volume; Manfredi, this volume; Souag, this volume).}\ia{Leddy-Cecere, Thomas@Leddy-Cecere, Thomas}\ia{Manfredi, Stefano@Manfredi, Stefano}\ia{Souag, Lameen@Souag, Lameen}


\subsection{Arabic and Domari}


The final instance of contact-induced changes to predicate \isi{negation} to be mentioned here concerns the \ili{Jerusalem} variety of the \ili{Indo-Aryan} language \ili{Domari}, as described by Matras (\citeyear{Matras1999,Matras2007Domari,Matras2012}; this volume).\ia{Matras, Yaron@Matras, Yaron}

Matras (\citeyear{Matras2012}: 350–351) describes two syntactic contexts in which negators borrowed from \ili{Palestinian} \ili{Arabic} are the only options in this variety of \ili{Domari}. The first is with \ili{Arabic}-derived \isi{modal} auxiliaries that take \ili{Arabic} suffix \isi{inflection}, as in \textit{bidd-} ‘want’ in \REF{domneg}. Here \isi{negation} is typically with the \ili{Palestinian} \ili{Arabic} stage-III construction \textit{{}-š} (without \textit{mā}), as it is would be also in \ili{Palestinian} \ili{Arabic}.


\ea\label{domneg}
{       \ili{Jerusalem} \ili{Domari} \citep[351]{Matras2012}}\\
\gll ben-om bidd-hā-\textbf{š} žawwiz-hōš-ar\\
     sister-\textsc{1sg} want\textsc{{}-3sg.f-neg} marry\textsc{-vitr.sbjv-3sg}\\
\glt ‘My sister doesn’t want to marry.’
\z

The second is when the negated predicate is nominal, as in \REF{jer.a}, or, to judge from Matras’s examples, when we have narrow focus of \isi{negation} with ellipsis, as in \REF{jer.b}. Here the negator that would be used in these contexts in \ili{Arabic} – \textit{miš} – is transferred to \ili{Domari} and functions in the same way.

\ea
{\ili{Jerusalem} \ili{Domari} \citep[350]{Matras2012}}\\
\ea\gll bay-os \textbf{mišš} kury-a-m-ēk\label{jer.a}\\
     \textsc{\textup{mother-}}\textsc{3sg} \textsc{neg} house\textsc{-obl.f-loc-pred.sg}  \\
\glt ‘His wife is not at home.’

\ex\gll day-om min ʕammān-a-ki \textbf{mišš} min ʕēl-oman-ki day-om\label{jer.b}\\
     \textsc{\textup{mother-1}}\textsc{sg} from \ili{Amman}-\textsc{obl.f-abl} \textsc{neg} from family-\textsc{1pl-abl} mother\textsc{-1sg}\\
\glt ‘My mother is from \ili{Amman}, she’s not from our family, my mother.’
\z
\z

In addition to these straightforward borrowings, \ili{Domari} has a bipartite negative construction in which both elements involve inherited lexical material, as illustrated in \REF{jer.c}.

\ea\label{jer.c}
{       \ili{Jerusalem} \ili{Domari} \citep[117]{Matras2012}}\\
\gll ʕašān ihne ama \textbf{n}-mang-am-san-\textbf{eʔ} l-ʕarab\\
     because thus \textsc{1sg} \textsc{neg-}want-\textsc{1sg-3pl-neg} \textsc{def-}Arabs\\
\glt ‘Because of this I don’t like the Arabs.’
\z

In Lucas (\citeyear{Lucas2013}: 413–414) I pointed out that the second element of this construction – -\textit{eʔ} – was apparently not attested in varieties of \ili{Domari} spoken outside of Palestine, and suggested that its presence in \ili{Jerusalem} \ili{Domari} could therefore be the result of influence from the \ili{Palestinian} bipartite negative construction. Herin (\citeyear{Herin2016,Herin2018}), however, has since convincingly shown that this is incorrect, and that the \ili{Jerusalem} \ili{Domari} bipartite construction is an internal development with cognates in more northerly varieties, the latter being in contact with \ili{Arabic} varieties that lack the bipartite negative construction. What is unique about the \ili{Jerusalem} variety of \ili{Domari} is that here a stage-III construction with \textit{{}-eʔ} alone is possible, omitting the original preverbal negator \textit{n(a)} that appears in \REF{jer.b}. Herin (\citeyear[32]{Herin2018}) argues that it is this stage-III construction, not the stage-II bipartite construction, that should be seen as the result of contact with \ili{Palestinian} \ili{Arabic}.

Overall, therefore, while the details naturally vary from one contact scenario to another, we see that negative constructions appear just as liable to be transferred between varieties of \ili{Arabic} and neighbouring languages as they are between the languages of Europe and beyond.


\section{Developments in indefinites in the scope of negation}\label{sec:sco}


\subsection{Loaned indefinites}


The organization and behaviour of \isi{indefinites} in the scope of \isi{negation} seem to be much more resistant to \isi{transfer} between languages than is the expression of clausal \isi{negation}, at least in the case of \ili{Arabic} and its contact languages.\footnote{Though for recent discussion of a related case – namely the acquisition of a determiner function by the \ili{Berber} indefinite \textit{kra} ‘something, anything’ via a \isi{calque} of the polyfunctionality of \ili{Maghrebi} \ili{Arabic} \textit{ši\-} – see \citet{Souag2018thing}.} Direct borrowing of individual indefinite items is rather common, however. I make no attempt at an exhaustive list here, but note the following two examples for illustrative purposes.

First, \ili{Berber} varieties stand out as frequent borrowers of \ili{Maghrebi} \ili{Arabic} \isi{indefinites}. The \isi{negative polarity item} \textit{ḥadd/ḥədd} ‘anyone’ is borrowed by at least \ili{Siwa} \citep[58]{Souag2009}, \ili{Kabyle}, \ili{Shawiya}, \ili{Mozabite} \citep[29]{Rabhi1996}, and \ili{Tashelhiyt} \citep[41]{Boumalk1996}. The \isi{n-word} \textit{walu} ‘nothing’ is borrowed by at least \ili{Tarifiyt} \citep[54]{Lafkioui1996}, \ili{Tashelhiyt}, and {Central} Atlas \ili{Tamazight} \citep[41]{Boumalk1996}. \textit{ḥətta}, in its function as an \isi{n-word} determiner, is borrowed by at least \ili{Tashelhiyt} \citep[41]{Boumalk1996}. \textit{qāʕ}, in its function as a negative polarity adverb ‘at all’, is borrowed by at least \ili{Tarifiyt} and {Central} Atlas \ili{Tamazight} \citep[42]{Boumalk1996}. And the negative polarity adverb *ʕumr ‘(n)ever’ (< ‘age, lifetime’) is borrowed by at least \ili{Kabyle}, \ili{Mozabite} \citep[30]{Rabhi1996}, and \ili{Tarifiyt} \citep[72]{Lafkioui1996}. Why these items should have been so freely borrowed, when each of them, with the possible exception of \textit{ḥətta}, have direct native equivalents, is unclear. But it is perhaps to be connected with the high degree of expressivity typically associated with negative statements containing \isi{indefinites}, which therefore creates a constant need for new and “extravagant” (in the sense of \citealt{Haspelmath2000}) means of expressing these meanings.

Second, while \ili{Arabic} itself seems to have been much more constrained in its borrowing of \isi{indefinites} from other languages, we can here point at least to the \isi{n-word} \textit{hīč} ‘nothing’, borrowed from \ili{Persian}, which Holes (\citeyear[549]{Holes2001}) includes in his glossary of pre-oil era \ili{Bahraini} \ili{Arabic}, citing also Blanc (\citeyear[159]{Blanc1964}) and Ingham (\citeyear[547]{Ingham1973}) for its occurrence in \ili{Baghdadi} and \ili{Khuzestan} \ili{Arabic} respectively. It remains in use in the latter (cf. Leitner, this volume), but consultations with present-day speakers of \ili{Baghdadi} \ili{Arabic} indicate that, in this variety at least, this item has since dropped out of use.\ia{Leitner, Bettina@Leitner, Bettina}



\subsection{The indefinite system of Maltese}


While most or perhaps all \ili{Arabic} varieties have at least some items that qualify as n-words according to the definition in §\ref{sec:ind}, it is only \ili{Maltese} that has developed into a straightforward negative-concord language with a full series of \isi{n-word} \isi{indefinites} in largely complementary distribution with a separate series of \isi{indefinites} that cannot appear in the scope of \isi{negation}, as is the situation in \ili{French}, described in §\ref{sec:ind}. These two series are shown in \tabref{tab:negation:malteseindefinites}, adapted from Haspelmath \& Caruana (\citeyear[215]{HaspelmathCaruana1996}).

\begin{table}
\begin{tabularx}{.6\textwidth}{XXX}
\lsptoprule
& n-words & non-n-words\\
\midrule
Determiner & ebda & xi\\
Thing & xejn & xi ħa\.ga\\
Person & ħadd & xi ħadd\\
Time & qatt & xi darba\\
Place & imkien & xi mkien\\
\lspbottomrule
\end{tabularx}
\caption{Maltese indefinites}
\label{tab:negation:malteseindefinites}
\end{table}


All the lexical material that makes up the \ili{Maltese} indefinite system illustrated in \tabref{tab:negation:malteseindefinites} is inherited from \ili{Arabic}, but the neat paradigm of n-words for determiner, ‘thing’, ‘person’, ‘time’, and ‘place’ is much more typical of European \ili{Romance} languages than of \ili{Arabic}. The extent to which, for example, \textit{xejn} ‘nothing’ (deriving from \textit{šayʔ} ‘thing’)\footnote{Contrary to what I suggested in Lucas (\citeyear[83--84]{Lucas2009}), the final segment of this item should not be understood as a fossilized retention of the indefinite suffix (so-called nunation or \textit{tanwīn}) of \ili{Classical} \ili{Arabic} (cf. \citealt{LucasSpagnol2019}).} is felt by \ili{Maltese} speakers to be inherently negative, is shown by the existence of the denominal verb \textit{xejjen} meaning ‘to nullify’, as illustrated in \REF{ex:mlt1}.\footnote{This is despite the fact that it may also occur in interrogatives with non-negative meaning (cf. \citealt{CamilleriSadler2017}). Compare the \ili{French} \isi{n-word} \textit{rien}, which, as illustrated in \REF{cont1}, retains a non-negative interpretation in a restricted set of negative-polarity contexts.}

\ea\label{ex:mlt1}
{       \ili{Maltese} \citep[441]{Lucas2013}}\\
\gll Iżda xejjen lil-u nnifs-u\\
     but nullify.\textsc{prf.3sg.m} \textsc{obj-3sg.m} self-\textsc{3sg.m}\\
\glt ‘But he made himself nothing.’
\z

As such, it seems likely that the \isi{intensive} contact that occurred over several centuries between \ili{Maltese} and the negative-concord languages \ili{Sicilian} and \ili{Italian} (cf. Lucas \& Čéplö, this volume)\ia{Lucas, Christopher@Lucas, Christopher}\ia{Čéplö, Slavomír@Čéplö, Slavomír} played a role in these developments in the \ili{Maltese} indefinite system. Precisely how this influence was mediated is hard to say, since both borrowing under \isi{recipient-language} agentivity and \isi{imposition} under \isi{source-language} agentivity were likely operative in the \ili{Maltese}–\ili{Romance} contact situation, and either are possible here. See Lucas (\citeyear{Lucas2013}: 439–444) for further discussion.

\section{Conclusion}

As we have seen, the overall areal picture of bipartite clausal \isi{negation} in \ili{Arabic} and its contact languages (and also, to a lesser extent, \isi{indefinites} in the scope of \isi{negation}) strongly suggests a series of contact-induced changes, and not a series of purely internally-caused independent parallel developments. What is required in \isi{future} research on this topic, to the extent that textual and other historical evidence becomes available, is a detailed, case-by-case examination of the linguistic and sociolinguistic conditions under which these constructions emerged in the languages in question. Such investigations would serve to either substantiate or undermine the contact-based explanations for these changes advanced in the course of this chapter. Ideally, they would also allow to understand in more detail the mechanisms of bilingual language use and acquisition that give rise to changes of this sort.

\section*{Further reading}

\citet{ChakerCaubet1996} is an edited volume providing a wealth of descriptive data on the expression of \isi{negation} in a number of \ili{Berber} and \ili{Maghrebi} \ili{Arabic} varieties.\\
\citet{Diem2014} is a detailed study of the \isi{grammaticalization} of \ili{Arabic} \textit{šayʔ} as a negator, with particular attention paid to early sources of textual evidence for this development.\\
\citet{WillisBreitbarth2013} and \citet{BreitbarthWillisLucasinpress} are two volumes of a work examining in detail the history of \isi{negation} in the languages of Europe and the Mediterranean.

\section*{Acknowledgements}

The research presented in this chapter was partly funded by a Leadership Fellows grant from the UK Arts and Humanities Research Council, whose support is hereby gratefully acknowledged. I am also very grateful to Stefano Manfredi, Lameen Souag and Bruno Herin for their comments on an earlier draft of the chapter. Responsibility for any failings that remain is mine alone.

\section*{Abbreviations}

\begin{tabularx}{.5\textwidth}{@{}lQ@{}}
\textsc{1, 2, 3} & 1st, 2nd, 3rd person \\
\textsc{abl} & ablative \\
\textsc{aor} & aorist \\
\textsc{aux} & auxiliary \\
\textsc{caus} & \isi{causative} \\
\textsc{dat} & dative \\
\textsc{def} & \isi{definite} \isi{article} \\
\textsc{dem} & demonstrative \\
\textsc{du} & dual \\
\textsc{exs} & \isi{existential} \\
\textsc{f} & feminine \\
\textsc{fut} & \isi{future} \\
\textsc{impf} & imperfect (prefix conjugation) \\
\textsc{ind} & indicative \\
\textsc{indf} & indefinite \\
\textsc{m} & masculine \\
\end{tabularx}%
\begin{tabularx}{.5\textwidth}{@{}lQ@{}}
\textsc{neg} & negative \\
\textsc{obj} & object \\
\ili{MSAL} & Modern {South Arabian} \\
\textsc{obl} & oblique \\
\textsc{pass} & \isi{passive} \\
\textsc{pfv} & perfective \\
\textsc{pst} & past \\
\textsc{pl} & plural \\
\textsc{ptcp} & \isi{participle} \\
\textsc{pred} & predicate \\
\textsc{prf} & perfect (suffix conjugation) \\
\textsc{prs} & present \\
\textsc{real} & realis \\
\textsc{sbjv} & subjunctive \\
\textsc{sg} & singular \\
\textsc{vitr} & intransitive marker \\
\end{tabularx}%



\sloppy
\printbibliography[heading=subbibliography,notkeyword=this]
\end{document}
