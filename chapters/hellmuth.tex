\documentclass[output=paper]{langsci/langscibook} 
\author{Sam Hellmuth\affiliation{University of York}}
\title{Contact and variation in Arabic intonation}
\abstract{Evidence is emerging of differences among Arabic dialects in their intonation patterns, along known parameters of variation in prosodic typology. Through a series of brief case studies, this chapter explores the hypothesis that variation in intonation in Arabic results from changes in the phonology of individual Arabic varieties, triggered by past (or present-day) speaker bilingualism. If correct, variation in intonation should reflect prosodic properties of the specific languages that a particular regional dialect has had contact with.}
\IfFileExists{../localcommands.tex}{
  % add all extra packages you need to load to this file 
\usepackage{graphicx}
\usepackage{tabularx}
\usepackage{amsmath} 
\usepackage{multicol}
\usepackage{lipsum}
\usepackage[stable]{footmisc}
\usepackage{adforn}
%%%%%%%%%%%%%%%%%%%%%%%%%%%%%%%%%%%%%%%%%%%%%%%%%%%%
%%%                                              %%%
%%%           Examples                           %%%
%%%                                              %%%
%%%%%%%%%%%%%%%%%%%%%%%%%%%%%%%%%%%%%%%%%%%%%%%%%%%%
% remove the percentage signs in the following lines
% if your book makes use of linguistic examples
\usepackage{./langsci/styles/langsci-optional} 
\usepackage{./langsci/styles/langsci-lgr}
\usepackage{morewrites} 
%% if you want the source line of examples to be in italics, uncomment the following line
% \def\exfont{\it}

\usepackage{enumitem}
\newlist{furtherreading}{description}{1}
\setlist[furtherreading]{font=\normalfont,labelsep=\widthof{~},noitemsep,align=left,leftmargin=\parindent,labelindent=0pt,labelwidth=-\parindent}
\usepackage{phonetic}
\usepackage{chronosys,tabularx}
\usepackage{csquotes}
\usepackage[stable]{footmisc} 

\usepackage{langsci-bidi}
\usepackage{./langsci/styles/langsci-gb4e} 

  \makeatletter
\let\thetitle\@title
\let\theauthor\@author 
\makeatother



\newcommand{\togglepaper}[1][0]{ 
  \bibliography{../localbibliography}
  \papernote{\scriptsize\normalfont
    \theauthor.
    \thetitle. 
    To appear in: 
    Christopher Lucas and Stefano Manfredi (eds.),  
    Arabic and contact-induced language change
    Berlin: Language Science Press. [preliminary page numbering]
  }
  \pagenumbering{roman}
  \setcounter{chapter}{#1}
  \addtocounter{chapter}{-1}
}

\newfontfamily\Parsifont[Script=Arabic]{ScheherazadeRegOT_Jazm.ttf} 
\newcommand{\arabscript}[1]{\RL{\Parsifont #1}}
\newcommand{\textarabic}[1]{{\arabicfont #1}}

\newcommand{\textstylest}[1]{{\color{red}#1}}

\patchcmd{\mkbibindexname}{\ifdefvoid{#3}{}{\MakeCapital{#3}
}}{\ifdefvoid{#3}{}{#3 }}{}{\AtEndDocument{\typeout{mkbibindexname could
not be patched.}}}

\newcommand{\R}{r\kern-.05ex{̣}\kern.05ex}

 
  %% hyphenation points for line breaks
%% Normally, automatic hyphenation in LaTeX is very good
%% If a word is mis-hyphenated, add it to this file
%%
%% add information to TeX file before \begin{document} with:
%% %% hyphenation points for line breaks
%% Normally, automatic hyphenation in LaTeX is very good
%% If a word is mis-hyphenated, add it to this file
%%
%% add information to TeX file before \begin{document} with:
%% %% hyphenation points for line breaks
%% Normally, automatic hyphenation in LaTeX is very good
%% If a word is mis-hyphenated, add it to this file
%%
%% add information to TeX file before \begin{document} with:
%% \include{localhyphenation}
\hyphenation{
Engels-zungen
affri-ca-te
affri-ca-tes
Bongor
Lotuho
Sprach-inseln
under-going
monkey
fortiori
kapaparát
báfura
imbíró
Bangala
Bettega
Bab-ylon-ian
Kouwenberg
Ammani
Amman
Ammanis
scen-arios
elabor-ate
elabor-ated
func-tion-al-ization
anthro-polo-gist
Iemmolo
table
Martin
gahawa
Tihāma
Sason
com-ple-ments
homo-phon-ous
Meso-potam-ian
Diyar-bakır
Tsiapera
Lucas
edu-ca-tion
cubit-ellu
corbata
morpho-phonology
morpho-phono-logic-al-ly
Volubilis
Mennen
Papa-constan-tinou
Vicente
re-inter-pre-ta-tion
megorashim
Malayalam
morpho-phon-em-ic-s
morpho-syn-tax
morpho-syn-tactic
Ki-nubi
ḥawār-iyy-ūn
archa-ic-ity
de-velop-ment
pro-so-dic
in-duced
pho-no-logy
possess-um
pos-ses-s-ive-s
Birgitt
es-pe-cial-ly
clause-s
po-ten-tial-ly
Lusignan
Cush-itic
re-fi-ni-ti-za-tion
lan-guage-s
lan-guage
dic-tion-ary
dic-tion-aries
socio-political
eight
counter-part
de-on-tic
Walter
Mifsud
ana-lys-is
re-ana-lys-is
re-ana-lyse
re-ana-lysed
re-ana-lyses
re-ana-lys-ing
Catalan
Occitan
Vigario
Sónia
ḥammāl
cross-linguist-ic-al-ly
south-western
bor-row-ed
ǧumruk
neg-ative
de-termin-er
Oxford
Socotra
Khuze-stan
Arama-ic
Otto-man
Otto-mans
Azeri
bi-lin-gual-ism
bor-row-ing
dia-lects
dia-lect
Ahwāzī
Catherine
allo-phone
allo-phones
phon-em-ic
hannaʔa
tubbaʕ
len-ition
Khorramshahr
poss-ible
Moroccan
Martine
Harrassowitz
gram-matica-lization
Wolfram
Afro-asiatica
Afro-asiatic
Semit-istik
mono-phthong
mono-phthong-ized
mono-phthong-izing
Heselwood
Hinrichs
Watson
Aramaeo-grams
Dadan-it-ic
sub-ordin-ator
Thamud-ic
Revue
Bovingdon
un-der
attach
attached
bundle
Arabia
graph-em-ic
graph-emes
ci-ca-da
Corriente
Andalusi
Kossmann
mor-pho-lo-gi-cal
dia-chroniques
islámica
occid-ent-al-ismo
idiom-a-ti-city
dia-lecto-logie
Reichert
co-lo-nial
Milton
diph-thong-al
lin-guis-tic
lin-guis-tique
lin-guis-tics
affairs
pho-net-ic-ally
kilo-metres
meta-typy
Jagiellonian
Arcodia
Hussein
Christopher
Giannakidou
Tashelhiyt
sta-bi-li-zation
de-vel-op-ments
in-vest-i-gation
Jor-dan-ian
af-fri-ca-tion
de-affri-ca-tion
Jordan
inter-dent-al
inter-locutors
socio-historical
roman-ization
Kurmanji
pha-ryn-geal-ization
con-sonant
con-sonants
phi-lo-logy
Öpengin
Kurd-ish
sunbul
sekin-în
fatala
dia-lect-al
Matras
mor-phol-o-gy
ḥaddād
emer-gence
Herzog
Benjmamins
struc-ture
Orient-studien
Ana-tolia
Ana-tolian
American
vari-ation
Jastrow
Geoffrey
Yarshater
Ḥarsūsi
Umberto
Ashtiany
Edmund
Mahnaz
En-cyclo-pædia
En-cyclo-paedia
En-cyclo-pedia
Leiden
dia-spora
Sasan-ian
every-day
domin-ance
Con-stitu-tion-al
Manfredi
pro-duct-iv-ity
Morocco
ferman
Maghreb-ian
algérien
Nicolaï
Mouton
maur-i-tani-en
Gotho-burg-ensis
socio-linguist-ique
socio-linguistic
dia-chronica
dihāt
de-velop-ed
ṣuḥbat
kitāba
kitābat
com-mercial
Senegal
mechan-ics
Maur-itan-ia
Ḥassān-iyya
circum-cision
cor-relation
labio-velar-ization
vowel
vowels
īggīw
series
in-tegrates
dur-ative
inter-dent-als
gen-itive
Tuareg
tălămut
talawmāyət
con-diment
Wiesbaden
Uni-vers-idad
Geuthner
typo-logie
Maur-itanie
nomades
Maur-itan-ian
dia-lecto-logy
dia-lecto-logical
Sahar-iennes
Uni-ver-sity
de-scend-ants
NENA-speak-ing
speak-ing
xaddaːm
Kahane
Tinniswood
Ven-etian
rabia
Cohen
dubi-ous
Al-geria
trans-humance
Marrakech
Bender
Munich
origin-al
re-captured
ethnic
minor-it-ies
drama-tic
local
long-stand-ing
regions
Nineveh
misti-linguismo
settle-ments
Ṣəndor
Mandate
sub-sti-tu-ting
or-thog-ra-phy
ref-er-enced
origin-ate
twenti-eth
typ-ic-al-ly
Hobrack
never-the-less
masc-uline
coffee
ex-clu-sive-ly
ver-bal
sim-i-lar-i-ties
der-i-va-tion
im-pera-tive
par-ti-ci-ple
dis-am-big-uation
dis-am-big-uating
phen-omen-on
phen-omen-a
traktar
Coetsem
mVCCūC
kān-at
Ing-ham
uni-versidad
Kerman-shah
Brin-cat
Tarifiyt
Vanhove
pre-verbal
Morris
Soqoṭri
dis-pre-fer-red
ex-pla-nation
con-struc-tion
Behnstedt
Andersen
wide-spread
usu-al-ly
re-gion-al
Bulut
con-sider-ation
afro-asia-tici
Franco-Angeli
Phono-logie
Volks-kundliche
dia-lectes
dia-lecte
dis-appear-ance
media
under-stand-able
pub-li-cation
second-ary
e-ject-ive
rev-o-lu-tion
re-stric-tive
Gasparini
moun-tain
moun-tains
yellow
label-ing
tra-di-tion-al-ly
cur-rent-ly
dia-chronic
Hebron
sub-se-quent-ly
doc-u-men-tation
Dom-ari
inter-actions
po-ten-tial
in-ven-tory
sim-pli-fi-cation
da-tive
pro-nom-in-al
or-i-gin
pre-po-si-tion
in-flec-tion-al
cat-e-gor-ic-al
in-di-cates
poss-ess-ive
de-termin-ation
rep-re-sent
e-lici-tation
typo-logy
ac-tiv-i-ty
ac-tiv-i-ties
fellows
in-ter-pre-tation
lin-guists
cycle
ne-ga-tion
sub-con-tinent
va-len-cy
ca-te-gor-ies
com-par-ison
gram-mat-i-cal
cor-res-pond-ing
ob-ser-vations
as-pec-tual
in-di-cation
ser-vice
iden-ti-fy-ing
utter-ances
no-ta-bly
for-mation
dis-tin-guish
mutand-is
muta-tis
in-di-vid-uals
lingua
natur-al
sub-sequent
re-cur-rence
rel-ative
rel-atives
mil-i-tary
ex-po-sure
spe-cif-ic-al-ly
rep-li-cate
rep-li-ca-ting
mod-i-fi-cations
gen-der
ṣəḥəttux
dra-ma-tic-al-ly
ac-cen-tuated
his-tor-i-cal
his-tor-i-cal-ly
event-ual-ly
pres-tige
dis-ap-pear
Romance
pen-insula
pen-insular
fea-ture
con-stitute
either
period
korufu
poly-sem-ous
Zammit
namrag
earli-er
pis-ellum
qillīd
Ta-rif-it
Reinhardt
situ-ation
Mohand
Zazaki
form-ación
Científicas
Roman-ica
butter-fly
roman-cismos
espec-ial-mente
Alexandrine
Ziamari
oriente
SELAF
cultur-elles
maroc-ain
medi-eval
medi-terranean
ɣesmat
Dordrecht
Trieste
}
\hyphenation{
Engels-zungen
affri-ca-te
affri-ca-tes
Bongor
Lotuho
Sprach-inseln
under-going
monkey
fortiori
kapaparát
báfura
imbíró
Bangala
Bettega
Bab-ylon-ian
Kouwenberg
Ammani
Amman
Ammanis
scen-arios
elabor-ate
elabor-ated
func-tion-al-ization
anthro-polo-gist
Iemmolo
table
Martin
gahawa
Tihāma
Sason
com-ple-ments
homo-phon-ous
Meso-potam-ian
Diyar-bakır
Tsiapera
Lucas
edu-ca-tion
cubit-ellu
corbata
morpho-phonology
morpho-phono-logic-al-ly
Volubilis
Mennen
Papa-constan-tinou
Vicente
re-inter-pre-ta-tion
megorashim
Malayalam
morpho-phon-em-ic-s
morpho-syn-tax
morpho-syn-tactic
Ki-nubi
ḥawār-iyy-ūn
archa-ic-ity
de-velop-ment
pro-so-dic
in-duced
pho-no-logy
possess-um
pos-ses-s-ive-s
Birgitt
es-pe-cial-ly
clause-s
po-ten-tial-ly
Lusignan
Cush-itic
re-fi-ni-ti-za-tion
lan-guage-s
lan-guage
dic-tion-ary
dic-tion-aries
socio-political
eight
counter-part
de-on-tic
Walter
Mifsud
ana-lys-is
re-ana-lys-is
re-ana-lyse
re-ana-lysed
re-ana-lyses
re-ana-lys-ing
Catalan
Occitan
Vigario
Sónia
ḥammāl
cross-linguist-ic-al-ly
south-western
bor-row-ed
ǧumruk
neg-ative
de-termin-er
Oxford
Socotra
Khuze-stan
Arama-ic
Otto-man
Otto-mans
Azeri
bi-lin-gual-ism
bor-row-ing
dia-lects
dia-lect
Ahwāzī
Catherine
allo-phone
allo-phones
phon-em-ic
hannaʔa
tubbaʕ
len-ition
Khorramshahr
poss-ible
Moroccan
Martine
Harrassowitz
gram-matica-lization
Wolfram
Afro-asiatica
Afro-asiatic
Semit-istik
mono-phthong
mono-phthong-ized
mono-phthong-izing
Heselwood
Hinrichs
Watson
Aramaeo-grams
Dadan-it-ic
sub-ordin-ator
Thamud-ic
Revue
Bovingdon
un-der
attach
attached
bundle
Arabia
graph-em-ic
graph-emes
ci-ca-da
Corriente
Andalusi
Kossmann
mor-pho-lo-gi-cal
dia-chroniques
islámica
occid-ent-al-ismo
idiom-a-ti-city
dia-lecto-logie
Reichert
co-lo-nial
Milton
diph-thong-al
lin-guis-tic
lin-guis-tique
lin-guis-tics
affairs
pho-net-ic-ally
kilo-metres
meta-typy
Jagiellonian
Arcodia
Hussein
Christopher
Giannakidou
Tashelhiyt
sta-bi-li-zation
de-vel-op-ments
in-vest-i-gation
Jor-dan-ian
af-fri-ca-tion
de-affri-ca-tion
Jordan
inter-dent-al
inter-locutors
socio-historical
roman-ization
Kurmanji
pha-ryn-geal-ization
con-sonant
con-sonants
phi-lo-logy
Öpengin
Kurd-ish
sunbul
sekin-în
fatala
dia-lect-al
Matras
mor-phol-o-gy
ḥaddād
emer-gence
Herzog
Benjmamins
struc-ture
Orient-studien
Ana-tolia
Ana-tolian
American
vari-ation
Jastrow
Geoffrey
Yarshater
Ḥarsūsi
Umberto
Ashtiany
Edmund
Mahnaz
En-cyclo-pædia
En-cyclo-paedia
En-cyclo-pedia
Leiden
dia-spora
Sasan-ian
every-day
domin-ance
Con-stitu-tion-al
Manfredi
pro-duct-iv-ity
Morocco
ferman
Maghreb-ian
algérien
Nicolaï
Mouton
maur-i-tani-en
Gotho-burg-ensis
socio-linguist-ique
socio-linguistic
dia-chronica
dihāt
de-velop-ed
ṣuḥbat
kitāba
kitābat
com-mercial
Senegal
mechan-ics
Maur-itan-ia
Ḥassān-iyya
circum-cision
cor-relation
labio-velar-ization
vowel
vowels
īggīw
series
in-tegrates
dur-ative
inter-dent-als
gen-itive
Tuareg
tălămut
talawmāyət
con-diment
Wiesbaden
Uni-vers-idad
Geuthner
typo-logie
Maur-itanie
nomades
Maur-itan-ian
dia-lecto-logy
dia-lecto-logical
Sahar-iennes
Uni-ver-sity
de-scend-ants
NENA-speak-ing
speak-ing
xaddaːm
Kahane
Tinniswood
Ven-etian
rabia
Cohen
dubi-ous
Al-geria
trans-humance
Marrakech
Bender
Munich
origin-al
re-captured
ethnic
minor-it-ies
drama-tic
local
long-stand-ing
regions
Nineveh
misti-linguismo
settle-ments
Ṣəndor
Mandate
sub-sti-tu-ting
or-thog-ra-phy
ref-er-enced
origin-ate
twenti-eth
typ-ic-al-ly
Hobrack
never-the-less
masc-uline
coffee
ex-clu-sive-ly
ver-bal
sim-i-lar-i-ties
der-i-va-tion
im-pera-tive
par-ti-ci-ple
dis-am-big-uation
dis-am-big-uating
phen-omen-on
phen-omen-a
traktar
Coetsem
mVCCūC
kān-at
Ing-ham
uni-versidad
Kerman-shah
Brin-cat
Tarifiyt
Vanhove
pre-verbal
Morris
Soqoṭri
dis-pre-fer-red
ex-pla-nation
con-struc-tion
Behnstedt
Andersen
wide-spread
usu-al-ly
re-gion-al
Bulut
con-sider-ation
afro-asia-tici
Franco-Angeli
Phono-logie
Volks-kundliche
dia-lectes
dia-lecte
dis-appear-ance
media
under-stand-able
pub-li-cation
second-ary
e-ject-ive
rev-o-lu-tion
re-stric-tive
Gasparini
moun-tain
moun-tains
yellow
label-ing
tra-di-tion-al-ly
cur-rent-ly
dia-chronic
Hebron
sub-se-quent-ly
doc-u-men-tation
Dom-ari
inter-actions
po-ten-tial
in-ven-tory
sim-pli-fi-cation
da-tive
pro-nom-in-al
or-i-gin
pre-po-si-tion
in-flec-tion-al
cat-e-gor-ic-al
in-di-cates
poss-ess-ive
de-termin-ation
rep-re-sent
e-lici-tation
typo-logy
ac-tiv-i-ty
ac-tiv-i-ties
fellows
in-ter-pre-tation
lin-guists
cycle
ne-ga-tion
sub-con-tinent
va-len-cy
ca-te-gor-ies
com-par-ison
gram-mat-i-cal
cor-res-pond-ing
ob-ser-vations
as-pec-tual
in-di-cation
ser-vice
iden-ti-fy-ing
utter-ances
no-ta-bly
for-mation
dis-tin-guish
mutand-is
muta-tis
in-di-vid-uals
lingua
natur-al
sub-sequent
re-cur-rence
rel-ative
rel-atives
mil-i-tary
ex-po-sure
spe-cif-ic-al-ly
rep-li-cate
rep-li-ca-ting
mod-i-fi-cations
gen-der
ṣəḥəttux
dra-ma-tic-al-ly
ac-cen-tuated
his-tor-i-cal
his-tor-i-cal-ly
event-ual-ly
pres-tige
dis-ap-pear
Romance
pen-insula
pen-insular
fea-ture
con-stitute
either
period
korufu
poly-sem-ous
Zammit
namrag
earli-er
pis-ellum
qillīd
Ta-rif-it
Reinhardt
situ-ation
Mohand
Zazaki
form-ación
Científicas
Roman-ica
butter-fly
roman-cismos
espec-ial-mente
Alexandrine
Ziamari
oriente
SELAF
cultur-elles
maroc-ain
medi-eval
medi-terranean
ɣesmat
Dordrecht
Trieste
}
\hyphenation{
Engels-zungen
affri-ca-te
affri-ca-tes
Bongor
Lotuho
Sprach-inseln
under-going
monkey
fortiori
kapaparát
báfura
imbíró
Bangala
Bettega
Bab-ylon-ian
Kouwenberg
Ammani
Amman
Ammanis
scen-arios
elabor-ate
elabor-ated
func-tion-al-ization
anthro-polo-gist
Iemmolo
table
Martin
gahawa
Tihāma
Sason
com-ple-ments
homo-phon-ous
Meso-potam-ian
Diyar-bakır
Tsiapera
Lucas
edu-ca-tion
cubit-ellu
corbata
morpho-phonology
morpho-phono-logic-al-ly
Volubilis
Mennen
Papa-constan-tinou
Vicente
re-inter-pre-ta-tion
megorashim
Malayalam
morpho-phon-em-ic-s
morpho-syn-tax
morpho-syn-tactic
Ki-nubi
ḥawār-iyy-ūn
archa-ic-ity
de-velop-ment
pro-so-dic
in-duced
pho-no-logy
possess-um
pos-ses-s-ive-s
Birgitt
es-pe-cial-ly
clause-s
po-ten-tial-ly
Lusignan
Cush-itic
re-fi-ni-ti-za-tion
lan-guage-s
lan-guage
dic-tion-ary
dic-tion-aries
socio-political
eight
counter-part
de-on-tic
Walter
Mifsud
ana-lys-is
re-ana-lys-is
re-ana-lyse
re-ana-lysed
re-ana-lyses
re-ana-lys-ing
Catalan
Occitan
Vigario
Sónia
ḥammāl
cross-linguist-ic-al-ly
south-western
bor-row-ed
ǧumruk
neg-ative
de-termin-er
Oxford
Socotra
Khuze-stan
Arama-ic
Otto-man
Otto-mans
Azeri
bi-lin-gual-ism
bor-row-ing
dia-lects
dia-lect
Ahwāzī
Catherine
allo-phone
allo-phones
phon-em-ic
hannaʔa
tubbaʕ
len-ition
Khorramshahr
poss-ible
Moroccan
Martine
Harrassowitz
gram-matica-lization
Wolfram
Afro-asiatica
Afro-asiatic
Semit-istik
mono-phthong
mono-phthong-ized
mono-phthong-izing
Heselwood
Hinrichs
Watson
Aramaeo-grams
Dadan-it-ic
sub-ordin-ator
Thamud-ic
Revue
Bovingdon
un-der
attach
attached
bundle
Arabia
graph-em-ic
graph-emes
ci-ca-da
Corriente
Andalusi
Kossmann
mor-pho-lo-gi-cal
dia-chroniques
islámica
occid-ent-al-ismo
idiom-a-ti-city
dia-lecto-logie
Reichert
co-lo-nial
Milton
diph-thong-al
lin-guis-tic
lin-guis-tique
lin-guis-tics
affairs
pho-net-ic-ally
kilo-metres
meta-typy
Jagiellonian
Arcodia
Hussein
Christopher
Giannakidou
Tashelhiyt
sta-bi-li-zation
de-vel-op-ments
in-vest-i-gation
Jor-dan-ian
af-fri-ca-tion
de-affri-ca-tion
Jordan
inter-dent-al
inter-locutors
socio-historical
roman-ization
Kurmanji
pha-ryn-geal-ization
con-sonant
con-sonants
phi-lo-logy
Öpengin
Kurd-ish
sunbul
sekin-în
fatala
dia-lect-al
Matras
mor-phol-o-gy
ḥaddād
emer-gence
Herzog
Benjmamins
struc-ture
Orient-studien
Ana-tolia
Ana-tolian
American
vari-ation
Jastrow
Geoffrey
Yarshater
Ḥarsūsi
Umberto
Ashtiany
Edmund
Mahnaz
En-cyclo-pædia
En-cyclo-paedia
En-cyclo-pedia
Leiden
dia-spora
Sasan-ian
every-day
domin-ance
Con-stitu-tion-al
Manfredi
pro-duct-iv-ity
Morocco
ferman
Maghreb-ian
algérien
Nicolaï
Mouton
maur-i-tani-en
Gotho-burg-ensis
socio-linguist-ique
socio-linguistic
dia-chronica
dihāt
de-velop-ed
ṣuḥbat
kitāba
kitābat
com-mercial
Senegal
mechan-ics
Maur-itan-ia
Ḥassān-iyya
circum-cision
cor-relation
labio-velar-ization
vowel
vowels
īggīw
series
in-tegrates
dur-ative
inter-dent-als
gen-itive
Tuareg
tălămut
talawmāyət
con-diment
Wiesbaden
Uni-vers-idad
Geuthner
typo-logie
Maur-itanie
nomades
Maur-itan-ian
dia-lecto-logy
dia-lecto-logical
Sahar-iennes
Uni-ver-sity
de-scend-ants
NENA-speak-ing
speak-ing
xaddaːm
Kahane
Tinniswood
Ven-etian
rabia
Cohen
dubi-ous
Al-geria
trans-humance
Marrakech
Bender
Munich
origin-al
re-captured
ethnic
minor-it-ies
drama-tic
local
long-stand-ing
regions
Nineveh
misti-linguismo
settle-ments
Ṣəndor
Mandate
sub-sti-tu-ting
or-thog-ra-phy
ref-er-enced
origin-ate
twenti-eth
typ-ic-al-ly
Hobrack
never-the-less
masc-uline
coffee
ex-clu-sive-ly
ver-bal
sim-i-lar-i-ties
der-i-va-tion
im-pera-tive
par-ti-ci-ple
dis-am-big-uation
dis-am-big-uating
phen-omen-on
phen-omen-a
traktar
Coetsem
mVCCūC
kān-at
Ing-ham
uni-versidad
Kerman-shah
Brin-cat
Tarifiyt
Vanhove
pre-verbal
Morris
Soqoṭri
dis-pre-fer-red
ex-pla-nation
con-struc-tion
Behnstedt
Andersen
wide-spread
usu-al-ly
re-gion-al
Bulut
con-sider-ation
afro-asia-tici
Franco-Angeli
Phono-logie
Volks-kundliche
dia-lectes
dia-lecte
dis-appear-ance
media
under-stand-able
pub-li-cation
second-ary
e-ject-ive
rev-o-lu-tion
re-stric-tive
Gasparini
moun-tain
moun-tains
yellow
label-ing
tra-di-tion-al-ly
cur-rent-ly
dia-chronic
Hebron
sub-se-quent-ly
doc-u-men-tation
Dom-ari
inter-actions
po-ten-tial
in-ven-tory
sim-pli-fi-cation
da-tive
pro-nom-in-al
or-i-gin
pre-po-si-tion
in-flec-tion-al
cat-e-gor-ic-al
in-di-cates
poss-ess-ive
de-termin-ation
rep-re-sent
e-lici-tation
typo-logy
ac-tiv-i-ty
ac-tiv-i-ties
fellows
in-ter-pre-tation
lin-guists
cycle
ne-ga-tion
sub-con-tinent
va-len-cy
ca-te-gor-ies
com-par-ison
gram-mat-i-cal
cor-res-pond-ing
ob-ser-vations
as-pec-tual
in-di-cation
ser-vice
iden-ti-fy-ing
utter-ances
no-ta-bly
for-mation
dis-tin-guish
mutand-is
muta-tis
in-di-vid-uals
lingua
natur-al
sub-sequent
re-cur-rence
rel-ative
rel-atives
mil-i-tary
ex-po-sure
spe-cif-ic-al-ly
rep-li-cate
rep-li-ca-ting
mod-i-fi-cations
gen-der
ṣəḥəttux
dra-ma-tic-al-ly
ac-cen-tuated
his-tor-i-cal
his-tor-i-cal-ly
event-ual-ly
pres-tige
dis-ap-pear
Romance
pen-insula
pen-insular
fea-ture
con-stitute
either
period
korufu
poly-sem-ous
Zammit
namrag
earli-er
pis-ellum
qillīd
Ta-rif-it
Reinhardt
situ-ation
Mohand
Zazaki
form-ación
Científicas
Roman-ica
butter-fly
roman-cismos
espec-ial-mente
Alexandrine
Ziamari
oriente
SELAF
cultur-elles
maroc-ain
medi-eval
medi-terranean
ɣesmat
Dordrecht
Trieste
} 
  \togglepaper[1]%%chapternumber
}{}

\begin{document}
\maketitle 
 
 

 \section{Introduction}


 \subsection{Contact-induced variation in intonation}


The hypothesis explored in this chapter is that observed synchronic variation in intonation across Arabic dialects is contact-induced. In this scenario, differences between dialects would result from changes in the intonational phonology of individual varieties triggered by speaker bilingualism in Arabic and one or more other languages \citep{Lucas2015}, either in the past, or up to and including the present day. To achieve this, \textcolor{red}{I} outline a framework for analysis of variation in intonation (§\ref{cross}), summarise recent research on the effects of bilingualism on the intonational phonology of bilingual individuals and the languages they speak (§\ref{var}), and sketch the types of language contact scenario which may be relevant for Arabic (§\ref{aravar}). In §\ref{araint} \textcolor{red}{I} present case studies of prosodic features which appear to be specific to a particular dialect, on current evidence at least, and discuss which of the potentially relevant contact languages might have served as the potential source of the feature in question, considering also possible endogenous (internal) sources of the change. The chapter closes (§\ref{closes}) with suggestions for future research.


 
 \subsection{Cross-linguistic variation in intonation} \label{cross}


Any attempt to delimit the nature and scope of variation in intonation depends on the model of intonational phonology adopted. The analyses explored in §\ref{araint} below are framed in the Autosegmental-Metrical (AM) theory of intonation \citep{Ladd2008}, and the parameters of intonational variation explored are thus influenced by this choice. 

A basic debate in analysis of intonation is whether the primitives of the system are whole contours (defined over an intonational phrase), or some sub-component of those contours \citep{Ladd2008}. In AM theory, intonation is modelled as interpolation of pitch between tonal targets; these tonal targets are the primitives of the system and are of two types: pitch accents are associated with the heads of metrical domains (e.g. stressed syllables), boundary tones are associated with the edges of metrical domains (e.g. prosodic phrases). In AM, tonal targets are transcribed using combinations of high (H) or low (L) targets, which reflect significant peaks and valleys, respectively, in the pitch contour of the utterance; association of these events to landmarks in the metrical structure is marked using ``*'' for pitch accents (associated with stressed syllables) and ``\%'' for boundary tones (associated with the right edge of prosodic phrases of different sizes). A typical AM analysis yields an inventory of the pitch accents and boundary tones needed to model the contours in a corpus of speech data, supported by a description of the observed contours (\citealt{JunFletcher2015}).

Ladd's (\citeyear{Ladd2008}) taxonomy of possible parameters of cross-linguistic variation in intonation (based on \citealt{Wells1982}) envisages four broad (inter-related) categories of variation: systemic (differences in the inventory of pitch accents or boundary tones); semantic (differences in the meaning or function associated with a particular contour, pitch accent or boundary tone); realisational (differences in the phonetic realisation of otherwise parallel pitch accents or boundary tones; and phonotactic (differences in the distribution of pitch accents and boundary tones, or in their association to metrical structure). 

Comparison of AM analyses across a typologically distinct set of languages (\citealt{Jun2005,Jun2015}) has highlighted systematic cross-linguistic variation of a systemic and/or phonotactic nature, in terms of prosodic phrasing (with relatively smaller or larger domains involved in structural organization of intonation patterns), the distribution of tonal events relative to prosodic constituents (marking either the edges or the metrical heads of phrases or both), and the size and composition of the inventory of tonal events regularly observed (pitch accents and boundary tones). There is also a large body of research on cross-linguistic variation in the phonetic realisation of pitch accents, in particular on peak alignment (\citealt{AttererLadd2004}; \citealt{Ladd2006}) and scaling (\citealt{LaddMorton1997}), confirming the existence of realisational cross-linguistic variation. The most advanced work on semantic variation to date has been on Romance languages, facilitated by a concerted effort to develop descriptions of these languages' intonation patterns within a common annotation system (\citealt{FrotaPrieto2015}). 

For Arabic, evidence is emerging of variation along similar lines. Recent review articles have highlighted clear differences in the size and composition of the inventory of pitch accents and boundary tones across Arabic dialects (\citealt{Chahal2011,ElZarka2017}), and in the association of pragmatic meanings with contours (cf. the case study in §\ref{tunis}). Initial evidence suggests a difference between Jordanian and Egyptian Arabic in the mapping of prosodic phrases to syntax \citep{Hellmuth2016} similar to that reported across Romance languages (\citealt{Dimperioetal2005}). Recent research suggests that Moroccan Arabic is a non-head-marking language in contrast to other Arabic dialects which are head-marking (see §\ref{moroc}), mirroring the cross-linguistic variation captured in Jun's (\citeyear{Jun2005}) typology, and among the head marking dialects, there appears to be variation in the density of distribution  of pitch accents (\citealt{ChahalHellmuth2015}; see §\ref{egypt}).


 
 \subsection{Contact-induced variation in intonation} \label{var}


A growing body of research has explored contact-induced prosodic change in the speech of bilingual communities and individuals. The initial focus of most studies was on second-language (L2) learners’ intonation patterns, or studies of individual bilinguals \citep{Queen2012}, and early L2 studies focused on realisational effects of a speaker’s L1 on their L2, and vice versa (\citealt{AttererLadd2004}; \citealt{Mennen2004}). More recent studies reveal a complex array of prosodic effects, both in terms of the features involved in the change (taking in all four of Ladd’s categories of possible variation), and also in the directionality of effects (L1 on L2, L2 on L1, or hybrid effects). 

\citet{Bullock2009} characterizes the general contact-induced language change literature (e.g. \citealt{WeinreichLabovHerzog1968,ThomasonKaufman1988}) as having made the assumption that segmental effects would ``precede'' prosodic effects, thus predicting prosodic effects would be seen only in contexts of widespread or sustained community bilingualism. As Bullock notes, however, there is no logical structural reason why this should be the case; her own study of English-like prosodic patterns in heritage French speakers in Pennsylvania confirms an effect of the dominant language in the prosodic domain (specifically in the realisation of focus) in speakers who in other respects maintain French segmental patterns.

Another example of prosodic properties of a dominant language affecting prosodic realisation of a heritage or second language, is that of immersion Gaelic learners in Scotland \citep{Nance2015}. Nance demonstrates a structural change in progress in Gaelic, from lexical pitch accent – still used by older English--Gaelic bilinguals – to a purely post-lexical system, used by younger bilinguals in immersion education who produce Gaelic with English-like intonation.  Similar effects of the dominant language on the non-dominant language are reported for Spanish in contact with Quechua (\citealt{ORourke2004}).  

The reverse effect has also been found in a number of studies, however, where prosodic properties of a non-dominant or heritage language have an effect on the prosodic realisation of the dominant language, in the speech of an individual or of the whole community. \citet{Fagyal2005} studied a group of bilingual French--Arabic adolescents in Paris; instead of a typical French phrase-final rise, these speakers produce a phrase-final rise--fall contour in declaratives, similar to the contour observed in Moroccan Arabic (MA) in parallel contexts. \citet{Simonet2011} shows that the steep (``concave'') final fall in Majorcan Catalan declaratives is now widely observed in Majorcan Spanish, replacing the typical gradual (``convex'') fall in Majorcan Spanish, but that each individual bilingual’s usage closely mirrors their reported language dominance. \citet{ColantoniGurlekian2004} observe patterns of peak alignment in pre-nuclear accents and pre-focal downstep in Buenos Aires Spanish which differ from neighbouring varieties of Spanish but resemble those in Italian, and ascribe them to high levels of Spanish--Italian bilingualism in the city in the late nineteenth and early twentieth centuries. In this last case, the period of community bilingualism which triggered the change is now long past, but the effect on the prosodic patterns of the dominant language (in this case, Spanish) persists.

Finally, Queen (\citeyear{Queen2001,Queen2012}) reports a case of ``fusion'': Turkish--German bilinguals in Germany display phrase-final intonation contours which are never used by monolinguals in either language, but are found only in the speech of bilinguals, in a new variety of German known as ``Türkisch-Deutsch''.

This emerging literature suggests that contact-induced prosodic change is a frequent phenomenon, arising in varied forms and across diverse contact situations. \citet{Bullock2009} suggests that prosody and intonation are especially prone to change for three reasons. First, because the acoustic parameters involved -- pitch, intensity and duration -- are part of the linguistic encoding of all languages, albeit in different constellations, and are thus readily adapted. Second, because, perceptually, all languages make use of prosodic parameters to convey some aspects of utterance-level meaning, thus the mapping of form to meaning is also readily adapted. Third, and perhaps most persuasively, because the form--meaning mapping in intonation is generally not fixed, but displays considerable inter- and intra-speaker variation (\citealt{CangemiGriceKrüger2015,CangemiEtAl2016}) as well as contextual variation (cf. \citealt{Walker2014}), and it is pockets of structural ‘indeterminacy’ of this kind which are prone to change in bilingual grammars \citep{Sorace2004}. Queen (\citeyear[57]{Queen2001}) also suggests that the intertwining of form and function in intonation makes it a fruitful sphere for investigation of contact-induced change, because “intonation is one of the few linguistic elements that comments simultaneously on grammar, context and culture”. Indeed, Simonet's (\citeyear{Simonet2011}) work shows that speakers are able to adopt the intonation of a contact language without actually being proficient in the source language. Finally, \citet{Matras2007borrowability} argues that the separate nature of prosody, which is processed separately from segmental phonology, and can be interpreted independently of the propositional content of the utterance, renders prosody more ``borrowable'' than other aspects of the grammar. 

In sum, there is strong evidence that intonation patterns are highly porous, being transferred between dominant and non-dominant languages in either direction; intonation is thus a fruitful area for investigation of contact-induced language change. Among the literature reviewed here, the paper by \citet{ColantoniGurlekian2004} most closely resembles the type of work which is needed in future for Arabic; they investigate present-day intonational variation in closely related varieties, and provide evidence from historical migration patterns to support the claim that the present-day variation can be ascribed to an earlier period of widespread bilingualism, in a language which is a plausible source of the feature in question. In the next section we outline a similar line of investigation for Arabic.


 
 \subsection{Contact-induced variation in Arabic} \label{aravar}


The time depth of descriptions of intonation patterns is shallow, due to the lack of historical audio recordings, and a general tendency that traditional grammars do not include detailed descriptions of prosody. It is thus difficult to reliably determine when changes in intonation may have happened, and the range of languages to be considered as the source of any putative intonational change is rather broad. 

One set of potential source languages is the substrate languages spoken in a particular region before the arrival of Arabic, for example Amazigh (Berber) in North Africa, and Egyptian--Coptic in Egypt. We might also see influence from the external languages which these indigenous languages were in contact with prior to the arrival of Arabic, such as Greek and Latin, or of other external languages whose influence was felt throughout the Arab world in later periods, such as Persian and Ottoman. Other possible source languages are European languages spoken along the northern coast of the Mediterranean, since large areas of southern Europe were under Arab rule for extended periods (7\textsuperscript{th}--15\textsuperscript{th} centuries CE), and contact through sea-borne trade is likely to have continued after that time. Conversely, large areas of the Arab world were under direct or indirect European control also (19\textsuperscript{th}--20\textsuperscript{th} centuries CE), and the influence of these languages is still felt today. Finally, we might also consider the potential effects of contact with global languages such as English, and with the L1 languages of migrant workers and long-term displaced language communities.

The decision to treat observed present-day variation as the result of change does not entail assuming that any one variety of Arabic was the ancestor of all dialects. Instead, the approach here will be to identify prosodic features which are seen in one Arabic dialect (or group of dialects), but not (yet) documented in any other dialects, as the most likely cases of potential contact-induced change. In each case study we evaluate the hypothesis by looking for evidence of the same feature in the relevant contact languages for the dialect in question, with comparison to possible language-internal sources of the change. 


 \section{Contact-induced variation in Arabic intonation} \label{araint}


 \subsection{Tunisian Arabic question marking} \label{tunis}


Tunisian Arabic polar questions are typically associated with a salient rise--fall pitch contour at the end of the utterance: speakers from southeast Tunisia produce a complete rise--fall in which pitch rises over the stressed syllable of the last word in the utterance to a peak, then falls to low; in contrast, speakers from Tunis produce a rise--plateau contour, in which, after the peak, pitch falls slightly then levels out. These patterns are illustrated in \figref{fig:key:1} \citep{BouchhiouaHellmuthAlmbark2019}. The rise--fall prosodic contour is frequently accompanied by a segmental question marker, in the form of a vowel added to the end of the last word in the utterance. The quality of the epenthesised vowel is influenced partly by vowels earlier in the word (in a form of vowel harmony) and partly by regional dialect, though these vowel quality patterns require further investigation.

  
\begin{figure}
\includegraphics[width=\textwidth]{figures/intonation-img1.png}
\caption{\label{fig:key:1}Pitch trace of yes/no questions from a Tunisian Arabic speaker from Tunis (left) and from the southeast (right).}

\begin{tabularx}{\textwidth}{llX}
\lsptoprule
si & naˈbil & /mawˈʒud/ [mawˈʒudə/u]\\
Mr & Nabil & present\\
\multicolumn{3}{l}{‘Is Mr Nabil there?’  (tuno-arc1-f1/tuse-arc1-f1)}\\
\lspbottomrule
\end{tabularx}
\end{figure}

The rise--fall yes/no-question contour in TA differs from the rise seen in yes/no questions in most Arabic dialects \citep{Hellmuthtoappearbook} and, in terms of distribution, from the rise--fall contour observed in Moroccan Arabic (MA) across all utterance types (not only in yes/no questions). The vowel epenthesis marker appears to be unique to TA, thus far.

A pattern of utterance-final vowel epenthesis has been observed in a number of Romance languages spoken along the northern edge of the Mediterranean, including Bari Italian \citep{GriceEtAl2015}, and different varieties of Portuguese \citep{FrotaEtAl2015}. These cases of utterance-final vowel epenthesis are interpreted as text--tune adjustment, where segmental material is added to accommodate a complex prosodic contour.  For example, in Standard European Portuguese, a more general rule of utterance-final vowel deletion is blocked in utterances bearing a complex prosodic contour, such as the fall--rise (H+L* LH\%) on yes/no questions \citep{FrotaEtAl2015}. In Bari Italian, epenthesis is seen on a range of utterance types, but – like Portuguese – its occurrence can be ascribed to tonal crowding (i.e. that the complex contour requires more segmental material to be realised). This is reflected in higher incidence of epenthesis on utterance-final monosyllables than on longer words, and on words in which the final sound is an obstruent than on words with a final sonorant \citep{GriceEtAl2015}.

Investigation of utterance-final vowel epenthesis in TA yes/no questions, in a corpus of data collected in Tunis, shows a very different pattern, however. In TA the incidence of epenthesis is not affected by the number of syllables in the utterance-final word nor by the type of final sound. In addition, whereas in the other Romance languages epenthesis occurs on a range of utterance types, in TA epenthesis occurs only in yes/no-questions, and predominantly in yes/no-questions which are produced with a complex rise--plateau or rise--fall contour \citep{Hellmuthforthcomingtunisianyesno}. The effects which in the Romance languages are taken as evidence of text--tune adjustment are lacking in TA, which appears to rule out a language-internal (endogenous) source of the TA pattern of vowel epenthesis.

The TA epenthetic vowel is in fact best characterized as an optional question marker comprising the vowel itself plus an accompanying fall in pitch. The segmental marker is well-known among Tunisian linguists, being described as “the pan-Tunisian question marker clitic [ā]” (\citealt{HerinZammit2017}: 141), but the accompanying prosodic contour has received little attention in the literature until recently. This traditional question-marking strategy may however be in decline, since it now alternates with realisation of a yes/no-question using a simple rise contour similar to that found in most other Arabic dialects, and without an utterance-final epenthetic vowel. The picture is complicated by the fact that there is a somewhat higher incidence of epenthesis among young female speakers, who might be expected to use the traditional form less, rather than more \citep{Hellmuthforthcomingtunisianyesno}. 

The epenthesis + complex contour strategy in TA yes/no-questions stands out from other Arabic dialects and may thus be due to contact-induced prosodic change. Italian was spoken in Tunisia more widely than French, in the late nineteenth century \citep{Sayahi2011}, and is thus a potential source of the contour, since rise--falls occur in yes/no questions in a number of Italian dialects \citep{GiliFivelaEtAl2015}. However, the conditioning environments of epenthesis reported for Bari Italian are very different, suggesting that contact with Italian is not a likely source of the epenthesis component of the TA pattern.

An alternative source of the vowel epenthesis pattern is French, since Tunisia has seen very high levels of bilingualism in TA and French from the late nineteenth century up to the present day, despite concerted efforts to reduce usage of French \citep{Daoud2007}. Utterance-final schwa epenthesis has been reported as an emerging phenomenon in French \citep{Hansen1997}, but its distribution is again much broader, being seen across a range of utterance types, and not restricted to yes/no questions. Despite clear evidence of contact-induced effects of French on TA in other domains, such as lexical borrowing, and a general trend towards use of French by female speakers \citep{Walters2011}, the different distribution of final epenthesis in French suggests it is not the most likely source of the TA segmental question-marking strategy. 

The other major contact language with TA is Tunisian Berber (TB). Although levels of TA--Berber bilingualism in Tunisia are now low, other than in certain regions \citep{Gabsi2011}, there was a sustained period of TA--TB bilingualism from the eleventh century, and TB is an important substrate of TA \citep{Daoud2007}. Although there are no studies of the prosody of TB, to our knowledge, a recent detailed study of Zwara Berber, spoken close to the Tunisian border in western Libya, documents a polar question marking clitic /a/ which is obligatorily accompanied by a rise--fall contour \citep{Gussenhoven2017}. The match of this description to the TA pattern is so close that it seems plausible that the TA question-marking pattern arose due to contact with TB during the period of sustained TA--TB bilingualism. The greater use of the epenthesis + contour strategy by female speakers than male speakers, as well as regional variation, makes this feature of TA ripe for further detailed sociolinguistic study.


 
 \subsection{Moroccan Arabic word prosody} \label{moroc}


Variation in word stress patterns across Arabic dialects has inspired much phonological investigation \citep{Watson2011stress}, but the Moroccan Arabic (MA) stress system has defied analysis until recently. Mitchell (\citeyear[202]{Mitchell1993}) notes that “in contrast with all the other vernaculars […], the place of prominence in a word in isolation is not carried over to its occurrence in the phrase and sentence”, and this characterization was confirmed experimentally by \citet{Boudlal2001}. A range of positions have emerged, with some authors claiming that MA does have word stress (\citealt{Benkirane1998}; \citealt{BurdinEtAl2014}), and others that it does not (\citealt{Maas2013}; \citealt{ElZarka2012}). 

It is now clear that MA is indeed typologically different from most other Arabic dialects in its word prosody. Whereas the majority of Arabic dialects have salient word-level stress and are thus clearly ``head-marking'' languages, in the typology proposed by \citet{Jun2005}, MA is a non-head-marking language in which tonal events mark the edges of prosodic phrases only. \citet{Bruggeman2018} provides acoustic evidence that there are no consistent cues to lexical prominence in MA, and perceptual evidence that MA listeners display the same type of ``deafness'' to stress as has been reported for listeners in languages which also lack head marking such as French \citep{DupouxPeperkamp2001} and Persian \citep{RahmaniRietveldGussenhoven2015}.

Can this stark variation in prosodic type between MA and other dialects of Arabic be attributed to contact-induced change? The Arabic language has been in sustained contact with Amazigh (Moroccan Berber, MB) since the seventh century, but also with Latin, French and Spanish\ia{Heath, Jeffrey@Heath, Jeffrey} (Heath, this volume). \citet{MaasProcházka2012} argue from corpus data that MA and MB share a common phonology, across a range of segmental and suprasegmental features. \citet{Bruggeman2018} confirms that there is no difference between MA and Tashlhiyt MB: both lack acoustic cues to word-level prominence in production and both groups of listeners display stress deafness. 

Since French is also an edge-marking language, without lexical stress, can we rule out French as an alternative source of this prosodic feature of MA? The main evidence comes from the fact that MA and MB also share other prosodic features which are not found in French, such as the shape of the tonal contour used to mark the edges of phrases, which is a rise in French \citep{Delais-Roussarieetal2015}, but a rise--fall in both Tashlhiyt MB (\citealt{GriceEtAl2015,BruggemanRoettgerGrice2017}) and MA (\citealt{Benkirane1998,Hellmuthtoappearbook}). The contrast is also exemplified in Faygal's (\citeyear{Fagyal2005}) study of French--MA bilinguals in Paris who use an MA rise--fall contour in French.  


 
 \subsection{Egyptian Arabic accent distribution} \label{egypt}


Cairene Egyptian Arabic (EA) displays a rich distribution of sentence accents, with a pitch accent typically observed on every content word. This has been noted independently by different authors (\citealt{Rifaat1991,Rastegar-ElZarka1997}), and is observed in both read and spontaneous speech styles \citep{Hellmuth2006}. Initial studies suggest that the same may be true also of some other dialects, such as Emirati \citep{BlodgettOwensRockwood2007} or Ḥiǧāzi \citep{Alzaidi2014}, but these observations await corroboration across different speech styles. 

Dense accent distribution has been noted in some languages on the northern coast of the Mediterranean also, including Spanish and Greek  \citep{Jun2005}, although, in Spanish, the rich accent distribution seen in laboratory speech is reduced in spontaneous speech \citep{Face2003}. Portuguese dialects vary in accent distribution: most varieties typically have an accent on every content word, but Standard European Portuguese shows an accent on the first and last words in an utterance only \citep{FrotaEtAl2015}. 

Rich accent distribution is not observed in Moroccan Arabic \citep{Benkirane1998}, nor in Tunisian Arabic \citep{Hellmuthtoappearbook}. If the EA accent distribution pattern were due to contact between EA and the southern European languages on the other side of the Mediterranean which share the tendency towards rich accent distribution, we might expect the pattern to be found all across North Africa.

There is strong documentary evidence from written sources of historical sustained multilingualism in Egypt. Greek arrived in Egypt in the fourth century BCE, serving as a formal administrative language alongside Egyptian for several centuries, and with the country reaching a state of “balanced societal bilingualism” in Greek and Egyptian in the sixth and seventh centuries CE \citep[6]{Papaconstantinou2010}. Egyptian evolved into Coptic, and its prestige continued to increase from the sixth century CE onwards. After the Arab conquest in the seventh century CE, Arabic began to take over from Greek as the language of administration, eventually replacing Coptic in daily use \citep{Papaconstantinou2012}. 

Is it possible that Egyptian--Coptic or Greek is the source of the rich accent distribution observed in EA (and indeed in Romance languages in southern Europe)? 

The distribution of full and long vowels in Coptic indicates that it had word-level prominence \citep[270]{Peust1999}, but it is not possible to determine from written texts the nature or distribution of any tonal contours which may have been associated with prominent syllables. Anecdotal evidence suggests that the intonation patterns used in surviving liturgical forms of Coptic are very different from those in EA \citep[32]{Peust1999}, though this difference may owe more to the liturgical setting than to properties of the languages in spoken form.

Ancient Greek is generally thought to have had a pitch accent system in which the primary marker of culminative accent in each word was pitch (\citealt{DevineStephens1985}). The Koine Greek dialect used in Egypt is thought to have lost pitch accent in favour of a stress accent system, however, by the fourth century BCE \citep{Benaissa2012}. 

Support for the hypothesis that Greek is the original source of the rich accent distribution would come from a match between the historical spread of Koine Greek around the Mediterranean with the location of languages in which rich accent distribution is also found. This would predict that eastern varieties of Libyan (Cyrenaican) Arabic might also be found to display rich accent distribution. If rich accent distribution is confirmed in dialects of Arabic (such as Emirati or Ḥiǧāzi) which did not have sustained contact with Greek, or with EA more recently, this would argue against Greek as the original source. Although Nubi \citep{Gussenhoven2006} and Juba Arabic \citep{Nakao2013} display hybrid properties between stress and lexical tone, the most likely explanation of their prosodic patterns is direct contact with local tonal languages. A potential endogenous trigger for development of rich accent distribution would be the absence of other forms of phonological marking of word domains, which are indeed somewhat reduced in EA, in comparison to other dialects \citep{Watson2002}, though the direction of causality of this correlation is not easily determined. 

Accent distribution has only recently been added to the parameters of variation explored in work on prosody \citep{Hellmuth2007}, and thus included in descriptions of the intonation systems of languages (e.g. \citealt{FrotaPrieto2015}). As further descriptions emerge of more dialects of Arabic it will be important to include documentation of accent distribution, across genres and speaking styles, in future research.

\section{Conclusion} \label{closes}

There is much that we do not yet know about variation in intonation in Arabic, which leaves scope for investigation of further potential cases of contact-induced prosodic change. One such case may be the Syrian Arabic utterance-final rising intonation, sometimes known as ``drawl'', which is found in yes/no questions but also across other utterance types \citep{Cowell1964}, and which is an identifiable feature of the Damascus dialect \citep{KulkOdéWoidich2003}. Although the full geographical range of the pattern has not been investigated in detail, and may be diffused to other dialects in the Levant, this rising declarative intonation pattern stands out from most other Arabic dialects, and is thus another potential case of contact-induced change. 

Another potential outlier pattern is the rise--fall intonation contour seen in yes/no questions in Yemeni Arabic from Ṣanʕāʔ \citep{Hellmuth2014}. The full areal reach of this prosodic question-marking strategy is also not yet fully known, and may extend into Ḥiǧāzi Arabic and western dialects of Oman. However, we do know that a rise--fall is seen in both Tunisia and Morocco, though in these places the pattern may be due to contact with varieties of Amazigh Berber. Nevertheless, it is tempting to speculate how a pattern found in Yemen might also be found in Tunisia and Morocco, and thus to explore the potential role of contact-induced variation due to ancient migrations between the eighth and fourteenth centuries \citep{Holes2018}.

Finally, the intonation of Modern Standard Arabic (MSA) and of other formal registers may prove to be a fruitful domain of future research. As our knowledge of the intonational phonology of spoken Arabic dialects improves, this will facilitate investigation of the extent to which the intonation patterns of a speaker’s regional dialect can be observed and/or perceived in their MSA speech, building on the findings of prior studies (\citealt{ElZarkaHellmuth2009}). An important goal would be to determine the extent to which a separate intonational system can be described for MSA, and to document the differential contribution to this system of specific genres of MSA discourse versus contact-induced influence due to widespread community mastery of multiple registers of the language.  

All these investigations would benefit from improved documentation of the time depth of present-day surface intonation patterns. For the quasi-unique features explored in §\ref{araint}, we do not know whether these are the result of recent or much more distant historical change. This situation might be rectified through analysis of archive audio materials, though dialect studies have often worked on oral narratives, which yield only a limited range of prosodic expression (i.e. usually few questions, and no information about turn-taking). A more viable strategy to gauge the time depth of contact-induced variation in Arabic intonation would be for future sociolinguistic studies to include prosodic features as variables in apparent time studies with participants in different age ranges, or for pre-existing corpora of apparent time data to be made available for prosodic analysis.

\section*{Further reading}

There are two key reference works, so far, on intonation in Arabic dialects, based on secondary analysis of prior published work: \citet{Chahal2011} and \citet{ElZarka2017}. Hellmuth (\citeyear{Hellmuth2019}) suggests prosodic variables for inclusion in studies of variation and change in Arabic. 

\section*{Acknowledgements}

The Intonational Variation in Arabic corpus (\citeyear{HellmuthAlmbark2017}) was funded by an award to the author by the UK Economic and Social Research Council (ES/I010106/1).

\section*{Abbreviations}

\begin{tabularx}{.45\textwidth}{lQ}
AM & Autosegmental-Metrical theory \\
BCE & before Common Era \\
CE & Common Era \\
EA & Egyptian Arabic \\
H & high tone \\
L & low tone \\
\end{tabularx}
\begin{tabularx}{.45\textwidth}{lQ}
L1, L2 & 1st, 2nd language \\
MA & Moroccan Arabic \\
MB & Moroccan Berber \\
MSA & Modern Standard Arabic \\
TA & Tunisian Arabic \\
TB & Tunisian Berber \\
YA & Yemeni Arabic \\
\end{tabularx}


\sloppy
\printbibliography[heading=subbibliography,notkeyword=this]
\end{document}