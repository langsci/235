\documentclass[output=paper]{langsci/langscibook} 
\author{Andrei Avram\affiliation{University of Bucharest}}
\title{Arabic pidgins and creoles}
\abstract{The chapter is an overview of eight Arabic-lexifier pidgins and creoles: Turku, Bongor Arabic, Juba Arabic, Kinubi, Pidgin Madame, Jordanian Pidgin Arabic, Romanian Pidgin Arabic, and Gulf Pidgin Arabic. The examples illustrate a number of selected features of these varieties. The focus is on two types of transfer, imposition and borrowing, within the framework outlined by Van Coetsem (\citeyear{VanCoetsem1988,VanCoetsem2000,VanCoetsem2003}) and Winford (\citeyear{Winford2005,Winford2008}).}
\IfFileExists{../localcommands.tex}{
  % add all extra packages you need to load to this file 
\usepackage{graphicx}
\usepackage{tabularx}
\usepackage{amsmath} 
\usepackage{multicol}
\usepackage{lipsum}
\usepackage[stable]{footmisc}
\usepackage{adforn}
%%%%%%%%%%%%%%%%%%%%%%%%%%%%%%%%%%%%%%%%%%%%%%%%%%%%
%%%                                              %%%
%%%           Examples                           %%%
%%%                                              %%%
%%%%%%%%%%%%%%%%%%%%%%%%%%%%%%%%%%%%%%%%%%%%%%%%%%%%
% remove the percentage signs in the following lines
% if your book makes use of linguistic examples
\usepackage{./langsci/styles/langsci-optional} 
\usepackage{./langsci/styles/langsci-lgr}
\usepackage{morewrites} 
%% if you want the source line of examples to be in italics, uncomment the following line
% \def\exfont{\it}

\usepackage{enumitem}
\newlist{furtherreading}{description}{1}
\setlist[furtherreading]{font=\normalfont,labelsep=\widthof{~},noitemsep,align=left,leftmargin=\parindent,labelindent=0pt,labelwidth=-\parindent}
\usepackage{phonetic}
\usepackage{chronosys,tabularx}
\usepackage{csquotes}
\usepackage[stable]{footmisc} 

\usepackage{langsci-bidi}
\usepackage{./langsci/styles/langsci-gb4e} 

  \makeatletter
\let\thetitle\@title
\let\theauthor\@author 
\makeatother

\newcommand{\togglepaper}[1][0]{ 
  \bibliography{../localbibliography}
  \papernote{\scriptsize\normalfont
    \theauthor.
    \thetitle. 
    To appear in: 
    Christopher Lucas and Stefano Manfredi (eds.),  
    Arabic and contact-induced language change
    Berlin: Language Science Press. [preliminary page numbering]
  }
  \pagenumbering{roman}
  \setcounter{chapter}{#1}
  \addtocounter{chapter}{-1}
}

\newfontfamily\Parsifont[Script=Arabic]{ScheherazadeRegOT_Jazm.ttf} 
\newcommand{\arabscript}[1]{\RL{\Parsifont #1}}
\newcommand{\textarabic}[1]{{\arabicfont #1}}

\newcommand{\textstylest}[1]{{\color{red}#1}}

\patchcmd{\mkbibindexname}{\ifdefvoid{#3}{}{\MakeCapital{#3}
}}{\ifdefvoid{#3}{}{#3 }}{}{\AtEndDocument{\typeout{mkbibindexname could
not be patched.}}}

%command for italic r with dot below with horizontal correction to put the dot in the prolongation of the vertical stroke
%for some reason, the dot is larger than expected, so we explicitly reduce the font size (to \small)
%for the time being, the font is set to an absolute value. To be more robust, a relative reduction would be better, but this might not be required right now
\newcommand{\R}{r\kern-.05ex{\small{̣}}\kern.05ex}


\DeclareLabeldate{%
    \field{date}
    \field{year}
    \field{eventdate}
    \field{origdate}
    \field{urldate}
    \field{pubstate}
    \literal{nodate}
}

\renewbibmacro*{addendum+pubstate}{% Thanks to https://tex.stackexchange.com/a/154367 for the idea
  \printfield{addendum}%
  \iffieldequalstr{labeldatesource}{pubstate}{}
  {\newunit\newblock\printfield{pubstate}}
}
 
  %% hyphenation points for line breaks
%% Normally, automatic hyphenation in LaTeX is very good
%% If a word is mis-hyphenated, add it to this file
%%
%% add information to TeX file before \begin{document} with:
%% %% hyphenation points for line breaks
%% Normally, automatic hyphenation in LaTeX is very good
%% If a word is mis-hyphenated, add it to this file
%%
%% add information to TeX file before \begin{document} with:
%% %% hyphenation points for line breaks
%% Normally, automatic hyphenation in LaTeX is very good
%% If a word is mis-hyphenated, add it to this file
%%
%% add information to TeX file before \begin{document} with:
%% \include{localhyphenation}
\hyphenation{
affri-ca-te
affri-ca-tes
com-ple-ments
homo-phon-ous
start-ed
Meso-potam-ian
morpho-phono-logic-al-ly
morpho-phon-em-ic-s
Palestin-ian
re-present-ed
Ki-nubi
ḥawār-iyy-ūn
archa-ic-ity
fuel-ed
de-velop-ment
pros-od-ic
Arab-ic
in-duced
phono-logy
possess-um
possess-ive-s
templ-ate
spec-ial
espec-ial-ly
nat-ive
pass-ive
clause-s
potent-ial-ly
Lusignan
commun-ity
tobacco
posi-tion
Cushit-ic
Middle
with-in
re-finit-iz-ation
langu-age-s
langu-age
diction-ary
glossary
govern-ment
eight
counter-part
nomin-al
equi-valent
deont-ic
ana-ly-sis
Malt-ese
un-fortun-ate-ly
scient-if-ic
Catalan
Occitan
ḥammāl
cross-linguist-ic-al-ly
predic-ate
major-ity
ignor-ance
chrono-logy
south-western
mention-ed
borrow-ed
neg-ative
de-termin-er
European
under-mine
detail
Oxford
Socotra
numer-ous
spoken
villages
nomad-ic
Khuze-stan
Arama-ic
Persian
Ottoman
Ottomans
Azeri
rur-al
bi-lingual-ism
borrow-ing
prestig-ious
dia-lects
dia-lect
allo-phone
allo-phones
poss-ible
parallel
parallels
pattern
article
common-ly
respect-ive-ly
sem-antic
Moroccan
Martine
Harrassowitz
Grammatic-al-ization
grammatic-al-ization
Afro-asiatica
Afro-asiatic
continu-ation
Semit-istik
varieties
mono-phthong
mono-phthong-ized
col-loquial
pro-duct
document-ary
ex-ample-s
ex-ample
termin-ate
element-s
Aramaeo-grams
Centr-al
idioms
Arab-ic
Dadan-it-ic
sub-ordin-ator
Thamud-ic
difficult
common-ly
Revue
Bovingdon
under
century
attach
attached
bundle
graph-em-ic
graph-emes
cicada
contrast-ive
Corriente
Andalusi
Kossmann
morpho-logic-al
inter-action
dia-chroniques
islámica
occid-ent-al-ismo
dialecto-logie
Reichert
coloni-al
Milton
diphthong-al
linguist-ic
linguist-ics
affairs
differ-ent
phonetic-ally
kilo-metres
stabil-ization
develop-ments
in-vestig-ation
Jordan-ian
notice-able
level-ed
migrants
con-dition-al
certain-ly
general-ly
especial-ly
af-fric-ation
Jordan
counter-parts
com-plication
consider-ably
inter-dent-al
com-mun-ity
inter-locutors
com-pon-ent
region-al
socio-historical
society
simul-taneous
phon-em-ic
roman-ization
Classic-al
funeral
Kurmanji
pharyn-geal-ization
vocab-ulary
phon-et-ic
con-sonant
con-sonants
special-ized
latter
latters
in-itial
ident-ic-al
cor-relate
geo-graphic-al-ly
Öpengin
Kurd-ish
in-digen-ous
sunbul
Christ-ian
Christ-ians
sekin-în
fatala
in-tegration
dia-lect-al
Matras
morpho-logy
in-tens-ive
con-figur-ation
im-port-ant
com-plement
ḥaddād
e-merg-ence
Benjmamins
struct-ure
em-pir-ic-al
Orient-studien
Anatolia
American
vari-ation
Jastrow
Geoffrey
Yarshater
Ashtiany
Edmund
Mahnaz
En-cyclo-pædia
En-cyclo-paedia
En-cyclo-pedia
Leiden
dia-spora
soph-is-ic-ated
Sasan-ian
every-day
domin-ance
Con-stitu-tion-al
religi-ous
sever-al
Manfredi
re-lev-ance
re-cipi-ent
pro-duct-iv-ity
turtle
Morocco
ferman
Maghreb-ian
algérien
stand-ard
systems
Nicolaï
Mouton
mauritani-en
Gotho-burg-ensis
socio-linguist-ique
plur-al
archiv-al
Arab-ian
drop-ped
dihāt
de-velop-ed
ṣuḥbat
kitāba
kitābat
com-mercial
eight-eenth
region
Senegal
mechan-ics
Maur-itan-ia
Ḥassān-iyya
circum-cision
cor-relation
labio-velar-ization
vowel
vowels
cert-ain
īggīw
series
in-tegrates
dur-ative
inter-dent-als
gen-itive
Tuareg
tălămut
talawmāyət
part-icular
part-icular-ly
con-diment
vill-age
bord-er
polit-ical
Wiesbaden
Uni-vers-idad
Geuthner
typo-logie
Maur-itanie
nomades
Maur-itan-ian
dia-lecto-logy
Sahar-iennes
Uni-vers-ity
de-scend-ants
NENA-speak-ing
speak-ing
origin-al
re-captured
in-habit-ants
ethnic
minor-it-ies
drama-tic
local
long-stand-ing
regions
Nineveh
settle-ments
Ṣəndor
Mandate
sub-stitut-ing
ortho-graphy
re-fer-enced
origin-ate
twenti-eth
typ-ic-al-ly
Hobrack
never-the-less
character-ist-ics
character-ist-ic
masc-uline
coffee
ex-clus-ive-ly
verb-al
re-ana-ly-se-d
simil-ar-ities
de-riv-ation
im-pera-tive
part-iciple
dis-ambi-gu-ation
dis-ambi-gu-a-ing
phen-omen-on
phen-omen-a
traktar
com-mun-ity
com-mun-ities
dis-prefer-red
ex-plan-ation
con-struction
wide-spread
us-ual-ly
region-al
Bulut
con-sider-ation
afro-asia-tici
Franco-Angeli
Phono-logie
Volks-kundliche
dia-lectes
dia-lecte
select-ed
dis-appear-ance
media
under-stand-able
public-ation
second-ary
e-ject-ive
re-volu-tion
re-strict-ive
Gasparini
mount-ain
mount-ains
yellow
label-ing
trad-ition-al-ly
currently
dia-chronic
}
\hyphenation{
affri-ca-te
affri-ca-tes
com-ple-ments
homo-phon-ous
start-ed
Meso-potam-ian
morpho-phono-logic-al-ly
morpho-phon-em-ic-s
Palestin-ian
re-present-ed
Ki-nubi
ḥawār-iyy-ūn
archa-ic-ity
fuel-ed
de-velop-ment
pros-od-ic
Arab-ic
in-duced
phono-logy
possess-um
possess-ive-s
templ-ate
spec-ial
espec-ial-ly
nat-ive
pass-ive
clause-s
potent-ial-ly
Lusignan
commun-ity
tobacco
posi-tion
Cushit-ic
Middle
with-in
re-finit-iz-ation
langu-age-s
langu-age
diction-ary
glossary
govern-ment
eight
counter-part
nomin-al
equi-valent
deont-ic
ana-ly-sis
Malt-ese
un-fortun-ate-ly
scient-if-ic
Catalan
Occitan
ḥammāl
cross-linguist-ic-al-ly
predic-ate
major-ity
ignor-ance
chrono-logy
south-western
mention-ed
borrow-ed
neg-ative
de-termin-er
European
under-mine
detail
Oxford
Socotra
numer-ous
spoken
villages
nomad-ic
Khuze-stan
Arama-ic
Persian
Ottoman
Ottomans
Azeri
rur-al
bi-lingual-ism
borrow-ing
prestig-ious
dia-lects
dia-lect
allo-phone
allo-phones
poss-ible
parallel
parallels
pattern
article
common-ly
respect-ive-ly
sem-antic
Moroccan
Martine
Harrassowitz
Grammatic-al-ization
grammatic-al-ization
Afro-asiatica
Afro-asiatic
continu-ation
Semit-istik
varieties
mono-phthong
mono-phthong-ized
col-loquial
pro-duct
document-ary
ex-ample-s
ex-ample
termin-ate
element-s
Aramaeo-grams
Centr-al
idioms
Arab-ic
Dadan-it-ic
sub-ordin-ator
Thamud-ic
difficult
common-ly
Revue
Bovingdon
under
century
attach
attached
bundle
graph-em-ic
graph-emes
cicada
contrast-ive
Corriente
Andalusi
Kossmann
morpho-logic-al
inter-action
dia-chroniques
islámica
occid-ent-al-ismo
dialecto-logie
Reichert
coloni-al
Milton
diphthong-al
linguist-ic
linguist-ics
affairs
differ-ent
phonetic-ally
kilo-metres
stabil-ization
develop-ments
in-vestig-ation
Jordan-ian
notice-able
level-ed
migrants
con-dition-al
certain-ly
general-ly
especial-ly
af-fric-ation
Jordan
counter-parts
com-plication
consider-ably
inter-dent-al
com-mun-ity
inter-locutors
com-pon-ent
region-al
socio-historical
society
simul-taneous
phon-em-ic
roman-ization
Classic-al
funeral
Kurmanji
pharyn-geal-ization
vocab-ulary
phon-et-ic
con-sonant
con-sonants
special-ized
latter
latters
in-itial
ident-ic-al
cor-relate
geo-graphic-al-ly
Öpengin
Kurd-ish
in-digen-ous
sunbul
Christ-ian
Christ-ians
sekin-în
fatala
in-tegration
dia-lect-al
Matras
morpho-logy
in-tens-ive
con-figur-ation
im-port-ant
com-plement
ḥaddād
e-merg-ence
Benjmamins
struct-ure
em-pir-ic-al
Orient-studien
Anatolia
American
vari-ation
Jastrow
Geoffrey
Yarshater
Ashtiany
Edmund
Mahnaz
En-cyclo-pædia
En-cyclo-paedia
En-cyclo-pedia
Leiden
dia-spora
soph-is-ic-ated
Sasan-ian
every-day
domin-ance
Con-stitu-tion-al
religi-ous
sever-al
Manfredi
re-lev-ance
re-cipi-ent
pro-duct-iv-ity
turtle
Morocco
ferman
Maghreb-ian
algérien
stand-ard
systems
Nicolaï
Mouton
mauritani-en
Gotho-burg-ensis
socio-linguist-ique
plur-al
archiv-al
Arab-ian
drop-ped
dihāt
de-velop-ed
ṣuḥbat
kitāba
kitābat
com-mercial
eight-eenth
region
Senegal
mechan-ics
Maur-itan-ia
Ḥassān-iyya
circum-cision
cor-relation
labio-velar-ization
vowel
vowels
cert-ain
īggīw
series
in-tegrates
dur-ative
inter-dent-als
gen-itive
Tuareg
tălămut
talawmāyət
part-icular
part-icular-ly
con-diment
vill-age
bord-er
polit-ical
Wiesbaden
Uni-vers-idad
Geuthner
typo-logie
Maur-itanie
nomades
Maur-itan-ian
dia-lecto-logy
Sahar-iennes
Uni-vers-ity
de-scend-ants
NENA-speak-ing
speak-ing
origin-al
re-captured
in-habit-ants
ethnic
minor-it-ies
drama-tic
local
long-stand-ing
regions
Nineveh
settle-ments
Ṣəndor
Mandate
sub-stitut-ing
ortho-graphy
re-fer-enced
origin-ate
twenti-eth
typ-ic-al-ly
Hobrack
never-the-less
character-ist-ics
character-ist-ic
masc-uline
coffee
ex-clus-ive-ly
verb-al
re-ana-ly-se-d
simil-ar-ities
de-riv-ation
im-pera-tive
part-iciple
dis-ambi-gu-ation
dis-ambi-gu-a-ing
phen-omen-on
phen-omen-a
traktar
com-mun-ity
com-mun-ities
dis-prefer-red
ex-plan-ation
con-struction
wide-spread
us-ual-ly
region-al
Bulut
con-sider-ation
afro-asia-tici
Franco-Angeli
Phono-logie
Volks-kundliche
dia-lectes
dia-lecte
select-ed
dis-appear-ance
media
under-stand-able
public-ation
second-ary
e-ject-ive
re-volu-tion
re-strict-ive
Gasparini
mount-ain
mount-ains
yellow
label-ing
trad-ition-al-ly
currently
dia-chronic
}
\hyphenation{
affri-ca-te
affri-ca-tes
com-ple-ments
homo-phon-ous
start-ed
Meso-potam-ian
morpho-phono-logic-al-ly
morpho-phon-em-ic-s
Palestin-ian
re-present-ed
Ki-nubi
ḥawār-iyy-ūn
archa-ic-ity
fuel-ed
de-velop-ment
pros-od-ic
Arab-ic
in-duced
phono-logy
possess-um
possess-ive-s
templ-ate
spec-ial
espec-ial-ly
nat-ive
pass-ive
clause-s
potent-ial-ly
Lusignan
commun-ity
tobacco
posi-tion
Cushit-ic
Middle
with-in
re-finit-iz-ation
langu-age-s
langu-age
diction-ary
glossary
govern-ment
eight
counter-part
nomin-al
equi-valent
deont-ic
ana-ly-sis
Malt-ese
un-fortun-ate-ly
scient-if-ic
Catalan
Occitan
ḥammāl
cross-linguist-ic-al-ly
predic-ate
major-ity
ignor-ance
chrono-logy
south-western
mention-ed
borrow-ed
neg-ative
de-termin-er
European
under-mine
detail
Oxford
Socotra
numer-ous
spoken
villages
nomad-ic
Khuze-stan
Arama-ic
Persian
Ottoman
Ottomans
Azeri
rur-al
bi-lingual-ism
borrow-ing
prestig-ious
dia-lects
dia-lect
allo-phone
allo-phones
poss-ible
parallel
parallels
pattern
article
common-ly
respect-ive-ly
sem-antic
Moroccan
Martine
Harrassowitz
Grammatic-al-ization
grammatic-al-ization
Afro-asiatica
Afro-asiatic
continu-ation
Semit-istik
varieties
mono-phthong
mono-phthong-ized
col-loquial
pro-duct
document-ary
ex-ample-s
ex-ample
termin-ate
element-s
Aramaeo-grams
Centr-al
idioms
Arab-ic
Dadan-it-ic
sub-ordin-ator
Thamud-ic
difficult
common-ly
Revue
Bovingdon
under
century
attach
attached
bundle
graph-em-ic
graph-emes
cicada
contrast-ive
Corriente
Andalusi
Kossmann
morpho-logic-al
inter-action
dia-chroniques
islámica
occid-ent-al-ismo
dialecto-logie
Reichert
coloni-al
Milton
diphthong-al
linguist-ic
linguist-ics
affairs
differ-ent
phonetic-ally
kilo-metres
stabil-ization
develop-ments
in-vestig-ation
Jordan-ian
notice-able
level-ed
migrants
con-dition-al
certain-ly
general-ly
especial-ly
af-fric-ation
Jordan
counter-parts
com-plication
consider-ably
inter-dent-al
com-mun-ity
inter-locutors
com-pon-ent
region-al
socio-historical
society
simul-taneous
phon-em-ic
roman-ization
Classic-al
funeral
Kurmanji
pharyn-geal-ization
vocab-ulary
phon-et-ic
con-sonant
con-sonants
special-ized
latter
latters
in-itial
ident-ic-al
cor-relate
geo-graphic-al-ly
Öpengin
Kurd-ish
in-digen-ous
sunbul
Christ-ian
Christ-ians
sekin-în
fatala
in-tegration
dia-lect-al
Matras
morpho-logy
in-tens-ive
con-figur-ation
im-port-ant
com-plement
ḥaddād
e-merg-ence
Benjmamins
struct-ure
em-pir-ic-al
Orient-studien
Anatolia
American
vari-ation
Jastrow
Geoffrey
Yarshater
Ashtiany
Edmund
Mahnaz
En-cyclo-pædia
En-cyclo-paedia
En-cyclo-pedia
Leiden
dia-spora
soph-is-ic-ated
Sasan-ian
every-day
domin-ance
Con-stitu-tion-al
religi-ous
sever-al
Manfredi
re-lev-ance
re-cipi-ent
pro-duct-iv-ity
turtle
Morocco
ferman
Maghreb-ian
algérien
stand-ard
systems
Nicolaï
Mouton
mauritani-en
Gotho-burg-ensis
socio-linguist-ique
plur-al
archiv-al
Arab-ian
drop-ped
dihāt
de-velop-ed
ṣuḥbat
kitāba
kitābat
com-mercial
eight-eenth
region
Senegal
mechan-ics
Maur-itan-ia
Ḥassān-iyya
circum-cision
cor-relation
labio-velar-ization
vowel
vowels
cert-ain
īggīw
series
in-tegrates
dur-ative
inter-dent-als
gen-itive
Tuareg
tălămut
talawmāyət
part-icular
part-icular-ly
con-diment
vill-age
bord-er
polit-ical
Wiesbaden
Uni-vers-idad
Geuthner
typo-logie
Maur-itanie
nomades
Maur-itan-ian
dia-lecto-logy
Sahar-iennes
Uni-vers-ity
de-scend-ants
NENA-speak-ing
speak-ing
origin-al
re-captured
in-habit-ants
ethnic
minor-it-ies
drama-tic
local
long-stand-ing
regions
Nineveh
settle-ments
Ṣəndor
Mandate
sub-stitut-ing
ortho-graphy
re-fer-enced
origin-ate
twenti-eth
typ-ic-al-ly
Hobrack
never-the-less
character-ist-ics
character-ist-ic
masc-uline
coffee
ex-clus-ive-ly
verb-al
re-ana-ly-se-d
simil-ar-ities
de-riv-ation
im-pera-tive
part-iciple
dis-ambi-gu-ation
dis-ambi-gu-a-ing
phen-omen-on
phen-omen-a
traktar
com-mun-ity
com-mun-ities
dis-prefer-red
ex-plan-ation
con-struction
wide-spread
us-ual-ly
region-al
Bulut
con-sider-ation
afro-asia-tici
Franco-Angeli
Phono-logie
Volks-kundliche
dia-lectes
dia-lecte
select-ed
dis-appear-ance
media
under-stand-able
public-ation
second-ary
e-ject-ive
re-volu-tion
re-strict-ive
Gasparini
mount-ain
mount-ains
yellow
label-ing
trad-ition-al-ly
currently
dia-chronic
} 
  \togglepaper[1]%%chapternumber
}{}

\begin{document}
\maketitle 


\section{Introduction}

This chapter aims to illustrate the emergence of Arabic-lexifier pidgins and creoles for which the contact situation – i.e. socio-historical context, the agents of change, and the languages involved – is at least relatively well known.

  The varieties considered can be classified into two groups, in geographical, historical and developmental terms: the Sudanic pidgins and creoles, and the immigrant pidgins in various Arab countries. Geographically, the Sudanic varieties developed in Africa – in present-day South Sudan, Chad, Uganda, and Kenya. Historically, the Sudanic varieties derive from a putative common ancestor, a pidgin that emerged in southern Sudan, in the first half of the nineteenth century. Various Turkish--Egyptian military expeditions between 1820 and 1840 opened southern Sudan for the slave trade. Permanent camps were set up soon after by slave traders in the White Nile Basin, Bahr el-Ghazal and Equatoria Province, inhabited by an Arabic-speaking minority and a huge majority of slaves from various ethnic and linguistic backgrounds. After 1850, the slave traders’ settlements were turned into military camps in which a military pidgin emerged, which is traditionally referred to as “Common Sudanic Pidgin Creole Arabic” (\citealt{ToscoManfredi2013}: 253). Two subgroups of Sudanic varieties are recognized: the western branch, consisting of Turku and Bongor Arabic (in Chad), and the eastern one, made up of Juba Arabic (in Sudan) and Kinubi (spoken in Uganda and Kenya). 

  Immigrant pidgins emerged in the eastern part of the Arab World, in Lebanon, Jordan, Iraq and the countries of the Arab Gulf. Historically, these do not go back more than 50 years. All these varieties are incipient pidgins.

  The contact situations illustrated presuppose: (i) a SL and a RL; (ii) agents of contact-induced change, who may be either SL or RL speakers; (iii) a psycholinguistically dominant language, which is not necessarily a socially dominant language (\citealt{VanCoetsem1988,VanCoetsem1995,VanCoetsem2000,VanCoetsem2003,Winford2005,Winford2008}). A distinction is made between two types of transfer: imposition and borrowing (\citealt{VanCoetsem1988,VanCoetsem2000,VanCoetsem2003}). Imposition involves SL-dominant speakers as agents (SL agentivity), is typical of second language acquisition, and induces changes mostly in phonology and syntax, although it may also include transfer of lexical items from the dominant SL into the non-dominant RL (\citealt[18]{VanCoetsem1995}; \citealt[376]{Winford2005}). Borrowing normally involves RL-dominant speakers as agents (RL agentivity), it typically targets lexical items, but may also include transfer of morphological material from a non-dominant SL into the dominant RL.

  In light of their sociolinguistic history, the varieties considered all emerged under conditions of untutored, short term, second language acquisition by adults dominant in their socially subordinate SLs. Second language acquisition, \textit{a} \textit{fortiori} with adults, triggers processes such as imposition via SL agentivity (i.e. substrate influence), simplification (\citealt{Trudgill2011}: 40, 101) – also known as restructuring \citep[529]{Lucas2015} – as well as language-internal (i.e. non-contact-induced) developments such as grammatical reanalysis \citep[415]{Winford2005}.

  As in \citet{Manfredi2018}, the focus of this chapter is on imposition and borrowing. It does not illustrate restructuring which does not involve any kind of transfer, but often involves a reduction in complexity \citep[529]{Lucas2015}. In the case of Arabic pidgins and creoles, restructuring is manifest in the domain of morphology, in, for example, the loss of the Arabic verbal affixes and of the nominal and verbal derivation strategies \citep{Miller1993}.

  The examples are illustrative only of selected contact-induced features of Arabic pidgins and creoles and their number has been kept to a reasonable minimum. The examples from Arabic and the pidgins and creoles considered appear in a uniform system of transliteration.

  The chapter is organized as follows. §\ref{sec:tur} and §\ref{sec:jub} are concerned with Sudanic pidgins and creoles. §\ref{sec:pid}, on the other hand, deals with Arabic immigrant varieties. §\ref{sec:conc} summarizes the findings and introduces issues for further research.


 \section{Turku and Bongor Arabic}\label{sec:tur}


 \subsection{Current state and historical development}


Turku is an extinct pidgin, formerly spoken in the Chari--Bagirmi region in western Chad \citep{Muraz1926}. After the abolition of slavery by the Turkish--Egyptian government in 1879, the Nile Nubian trader \iai{Rabeh} withdrew with his slave soldiers into Chad. From a sociolinguistic point of view, Turku was initially a military pidgin. However, it later became one of three trade languages in what was then French Equatorial Africa, along with Sango and Bangala (\citealt{ToscoOwens1993}: 183). Turku was a stable pidgin which does not appear to have creolized (\citealt{ToscoOwens1993}).

Bongor Arabic is spoken in southwestern Chad, in and around the town of Bongor, the capital of the Mayo--Kebbi Est region, close to the border with Cameroon \citep{Luffin2013}. Given the many structural features it shares with Turku, it is plausible to assume that Bongor Arabic developed from the former. Sociolinguistically, Bongor Arabic is a trade pidgin, used by the local Masa and Tupuri populations with Arabic-speaking traders. It is currently a stable pidgin, but it exhibits features indicative of depidginization under the influence of Chadian Arabic. No information about the number of speakers is available.


 
 \subsection{Contact languages}


The lexifier language of Turku and Bongor Arabic is Western Sudanic Arabic. The substratal input was provided by languages of various genetic affiliations: Nilo-Saharan – e.g. Bagirmi, Mbay, Ngambay, Sar, Sara (Central Sudanic), Kanuri (Western Saharan); Afro-Asiatic – Hausa (West Chadic); Niger-Congo – Fulfulde. In the case of Turku, an additional contributor was the creole language Sango. Both in Turku and in Bongor Arabic there is also adstratal input from French. The adstrate of Bongor Arabic additionally includes two languages: Masa (Nilo-Saharan, Western Chadic) and Tupuri (Niger-Congo).


 
 \subsection{Contact-induced changes}
 \subsubsection{Phonology}

The substrate languages do not have /ḫ/, which is generally replaced by [k]: Turku \textbf{\textit{k}}\textit{amsa} ‘five’ < ChA \textbf{\textit{ḫ}}\textit{amsa}; Bongor Arabic \textbf{\textit{k}}\textit{ídma} ‘work’ < ChA \textbf{\textit{ḫ}}\textit{idma}. Many of the substrate languages do not have /f/, which is substituted with [p] or perhaps [ɸ],\footnote{Given the transcription with <pf> in Muraz (\citeyear[168]{Muraz1926}).} e.g. Turku \textit{\textbf{pf}il} ‘elephant’ < ChA \textit{\textbf{f}īl}. In French loanwords, the reflexes of /v/ are either [b] or [w]: Bongor Arabic \textit{\textbf{b}oté} ‘to vote’ < French \textit{\textbf{v}oter}, \textit{\textbf{w}otír} ‘car’ < French \textit{\textbf{v}oiture}.  

  The consonants [ɲ] and [ŋ] occur only in loanwords: Turku \textit{konpa}\textit{\textbf{n\kern -1pty}e} ‘company’ < French \textit{compa}\textit{\textbf{gn}ie}, \textit{\textbf{ng}ari} ‘manioc’ < Mbay \textit{\textbf{ng}àrì}; Bongor Arabic \textit{\textbf{ng}ambáy} ‘Ngambay’ < Ngambay \textit{\textbf{ng}àmbáy}; [v] and [ʒ] occur only in phonologically non-integrated words of French origin: Turku \textit{si}\textit{\textbf{v}il} ‘civilian’ < French \textit{ci}\textit{\textbf{v}il}; Bongor Arabic \textit{\textbf{ž}urnalíst} `journalist' < French \textit{\textbf{j}ournaliste}. 

   Variation affects several consonants. For instance, [f] occurs in variation with [b] or [p]: Turku \textit{\textbf{f}išan} {\textasciitilde} \textit{\textbf{b}išan} ‘because’; Bongor Arabic \textit{má}\textit{\textbf{f}i} {\textasciitilde} \textit{má}\textit{\textbf{p}i} ‘\textsc{neg}’ < ChA \textit{mā} \textit{\textbf{f}ī}, \textit{so}\textit{\textbf{f}ér} {\textasciitilde} \textit{so}\textit{\textbf{p}ér} ‘driver’ < French \textit{chau}\textit{\textbf{ff}eur}. Most of the substrate languages do not have /š/, which accounts for [ʃ] {\textasciitilde} [s] variation, in words with either etymological /s/ or /ʃ/: Turku \textit{ga\textbf{s}i} {\textasciitilde} \textit{ga\textbf{š}i} ‘expensive’ < ChA \textit{gā\textbf{s}ī}, \textit{biri\textbf{š}} {\textasciitilde} \textit{biri\textbf{s}} ‘mat’ < ChA \textit{birī\textbf{š}}; Bongor Arabic \textit{má\textbf{š}i} {\textasciitilde} \textit{má\textbf{s}i} ‘go’. The usual reflexes of French /ʒ/, absent from the phonological inventories of the substrate languages, are [z], [ʤ] and [s] respectively: Turku \textit{\textbf{ǧ}inenal} ‘general’ < French \textit{\textbf{g}énéral}, \textit{\textbf{s}uska} ‘until < French \textit{\textbf{j}usqu’à}; Bongor Arabic \textit{\textbf{z}úska} `when, during' < French \textit{\textbf{j}usqu’à} `until'.

Finally, vowel length is not distinctive: Turku, Bongor Arabic \textit{kal\textbf{a}m} ‘speech; speak’ < ChA \textit{kal\textbf{ā}m} ‘speech’.


 \subsubsection{Morphology}

On current evidence (\citealt{Luffin2013}: 180–181), Bongor Arabic exhibits signs of depidginization under the influence of Chadian Arabic. The most striking instance of this is the use of pronominal suffixes, unique among Arabic-lexifier pidgins and creoles:

\ea
{Bongor Arabic \citep[180]{Luffin2013}}\\
\gll    índi gáy árifu úsum-\textbf{i} \\
        2\textsc{sg} \textsc{impf} know name-\textsc{poss}.1\textsc{sg}\\
\glt    `You know my name.'
\z
 
{  Also, verbal affixes are sporadically used:}\\
\ea
{Bongor Arabic \citep[181]{Luffin2013}}\\
\ea
\gll ána ma \textbf{n}-árfa \\
         1\textsc{sg} \textsc{neg} 1\textsc{sg}-know\\
\glt    `I don’t know.'

\ex
\gll  anína rikíb-\textbf{na} wotír da sáwa \\
         1\textsc{pl} ride.\textsc{prf}-1\textsc{pl} car \textsc{prox} together\\
\glt    `We took the car together.’
\z
\z
 
{These cases might be analyzed as borrowing under \textit{sui} \textit{generis} RL agentivity, whereby morphological material from a non-dominant SL is imported into a non-dominant (second) RL.} \\


 \subsubsection{Lexicon}

A part of the non-Arabic vocabulary of Turku can be traced back to its substrate languages (\citealt{Avram2019}). Most of the loanwords are from Sara-Bagirmi languages: \textit{adinbang} ‘eunuch’ < Bagirmi \textit{ádim} \textit{mbàŋ} ‘servant of the sultan’; \textit{gao} ‘hunter’ < Sar \textit{gáw}; \textit{ngari} ‘manioc’ < Mbay \textit{ngàrì}. The second most significant important contributor is Sango: \textit{kay} ‘paddle’ < Sango \textit{kâî}, \textit{tipoy} ‘carrying hammock’ < \textit{típóí}. A few words can be traced to Fulfulde and Kanuri: \textit{kelkelbanǧi} ‘golden beads’ < Fulfulde \textit{kelkel-banja}; \textit{wélik} ‘lightning’ < Kanuri \textit{wulak} ‘flash of lightning’. In a number of cases, the exact SL cannot be established: \textit{koporo} ‘0.10 Francs’ < Fulfulde, Sango, Sara \textit{koporo} ‘coin’; \textit{gurumba} ‘hat’ < Hausa \textit{gurúmba}, Kanuri \textit{gurumbá}. As for Bongor Arabic, its African adstrate languages have contributed only a few loanwords, such as \textit{bursdíya} ‘Monday’. There are also loanwords from French. In Turku most of these relate to the military (\citealt{ToscoOwens1993}: 262–263), e.g. Turku \textit{itenan} ‘lieutenant’ < French \textit{lieutenant}, \textit{permišon} ‘permission’ < French \textit{permission}. In addition to nouns, French loanwords include some verbs, such as Bongor Arabic \textit{komandé} ‘order’ < French \textit{commander}, and at least one function word, Turku \textit{suska}, Bongor Arabic \textit{zúska} ‘when, during’ < French \textit{jusqu’à} ‘until’.

  The substratal influence on Turku can also be seen in a number of compound calques (Avram \citeyear{Avram2019}; Manfredi, this volume).\ia{Manfredi, Stefano@Manfredi, Stefano} Some of these are modelled on Sara-Bagirmi languages: \textit{bahr} \textit{gum} ‘rising water’, cf. Ngambay \textit{màn-kà}\textit{w}, lit. ‘river goes’; \textit{nugra} \textit{ana} \textit{asal} ‘beehive’, cf. Ngambay \textit{bòlè-tǝnji}, lit. ‘hole (of) honey’. Other calques have equivalents in several SLs, such as \textit{nugra} \textit{haǧer} ‘cave’, lit. ‘hole mountain/stone’, cf. Kanuri \textit{kûl} \textit{kau-be} lit. ‘cavity mountain-of’, Ngambay \textit{bòlò-mbàl} lit. ‘hole mountain’, Sango \textit{dûtênë} lit. ‘hole stone’.


 \section{Juba Arabic and Kinubi}\label{sec:jub}


 \subsection{Current state and historical development}


Juba Arabic is mainly spoken in South Sudan; there are also diaspora communities, mostly in Sudan and Egypt. Two main reasons make it difficult to estimate its number of speakers. Firstly, while Juba Arabic is spoken as a primary language by 47\% of the population of Juba, the capital city of South Sudan, it is also used as a second or third language by the majority of the population of the country \citep[7]{Manfredi2017}. Secondly, the long coexistence of Juba Arabic with Sudanese Arabic, its main lexifier language, has led to the emergence of a continuum ranging from basilectal, through mesolectal, to acrolectal varieties; delimiting acrolectal Juba Arabic from Arabic is no easy task, particularly in the case of the large diaspora communities in Khartoum and Cairo.

  Juba Arabic emerged as a military pidgin. Sociolinguistically, it is today an inclusive identity marker for the ethnically and linguistically diverse population of South Sudan (\citealt{ToscoManfredi2013}: 507). Developmentally, Juba Arabic is a pidgincreole.\footnote{A pidgincreole is “a former pidgin that has become the main language of a speech community and/or a mother tongue for some of its speakers” \citep[131]{Bakker2008}.}

The Mahdist revolt, which started in 1881, eventually brought to an end Turkish--Egyptian control over Equatoria, in southern Sudan. Following an invasion by Mahdist rebels, the governor fled to Uganda, accompanied by slave soldiers loyal to the central government. These soldiers subsequently became the backbone of the British King’s African Rifles. While some of the troops remained in Uganda, others were moved to Kenya and Tanzania. This led to the dialectal division between Ugandan and Kenyan Kinubi. Like Juba Arabic, therefore, Kinubi started out as a military pidgin, then underwent stabilization and expansion. Today, however, Kinubi is the only Arabic-lexifier fully creolized variety, that is, a native language for its entire speech community.

Kinubi is spoken in Uganda and in Kenya. The number of speakers of Kinubi is a matter of debate. Ugandan Kinubi was spoken by some 15,000 people, according to the 1991 census, and Kenyan Kinubi by an estimated 10,000 in 2005. However, other estimates put the combined number of speakers at about 50,000. The largest communities of Kinubi speakers are in Bombo (Uganda), Nairobi (the Kibera neighbourhood) and Mombasa (Kenya).


 
 \subsection{Contact languages}


The main lexifier language of Juba Arabic is Sudanese Arabic, with some input from Egyptian Arabic and Western Sudanic dialects as well. The substrate is represented by a relatively large number of languages, belonging to super-phylums, Nilo-Saharan and Niger-Congo. The former includes Eastern Sudanic languages, such as Bari, Lotuho (Eastern Nilotic), Acholi, Belanda Bor, Dinka, Jur, Nuer, Päri, Shilluk (Western Nilotic), Didinga (Surmic), and Central Sudanic languages, such as Avokaya, Baka, Bongo, Ma'di, Moru; the Niger-Congo super-phylum is represented by, for example, Zande and Mundu. The main substrate language is considered to be Bari, including its dialects Kakwa, Kuku, Pojulu, and Mundari.\footnote{Sometimes considered to be separate languages \citep[207]{Wellens2003}.}

Given its sociolinguistic history, Kinubi shares much of its substrate with Juba Arabic. However, the substrate of Ugandan Kinubi additionally includes Eastern Sudanic languages, such as Alur, Luo (Western Nilotic), and Central Sudanic languages such as Mamvu, Lendu and Lugbara (\citealt{Owens1997}: 161; \citealt{Wellens2003}: 207), spoken in Uganda. Unlike Juba Arabic, Kinubi also exhibits the effects of the adstratal influence exerted by two Bantu languages, Luganda – particularly in Ugandan Kinubi – and Swahili – particularly in Kenyan Kinubi. One other language that should be mentioned is English, official both in Uganda and in Kenya.


 
 \subsection{Contact-induced changes}
 \subsubsection{Phonology}

A number of consonants found in Arabic, but absent from the phonological inventories of the substrate languages are either deleted or substituted. Consider the reflexes of pharyngeals: \textit{\textbf{h}áfla} ‘feast’ < SA \textit{\textbf{ḥ}afla}; \textit{\textbf{á}rabi} ‘Arabic’ < SA \textit{\textbf{ʕ}arabī}. The pharyngealized consonants are replaced by their plain counterparts: \textit{\textbf{t}owíl} ‘long’ < SA \textit{\textbf{ṭ}awīl}; \textit{\textbf{d}ul} ‘shadow’ < SA \textit{\textbf{ḍ}ull}; \textit{\textbf{s}úlba} ‘hip’ < SA \textit{\textbf{ṣ}ulba}; \textit{\textbf{z}úlum} ‘to anger’ < SA \textit{\textbf{ẓ}\kern -1ptulum}. The velar fricatives of Arabic are always replaced by velar stops: \textit{\textbf{k}ábara} ‘piece of news’ < SA \textit{\textbf{ḫ}abar}; \textit{šó\textbf{k}ol} ‘work’ < SA \textit{šo\textbf{ɣ}ol}, \textit{\textbf{g}árib} ‘west’ < SA \textit{\textbf{ɣ}ar(i)b}. 

As in Juba Arabic, the pharyngeals of Arabic are either replaced or lost in Kinubi (\citealt{Owens1985}: 10; \citealt{Wellens2003}: 209–212). The earliest records of Ugandan Kinubi\footnote{The main ones are: \citet{Cook1905}, \citet{Jenkins1909}, \citet{Meldon1913}, and \citet{OwenKeane1915}.} are replete with illustrative examples \citep{Avram2017talk}: \textit{\textbf{h}aǧa} ‘thing’ < SA \textit{\textbf{ḥ}āǧa}, \textit{aram} ‘thief’ < SA \textit{\textbf{ḥ}arāmi}, \textit{līb} < ‘to play’ < SA \textit{li\textbf{ʕ}ib}. The pharyngealized consonants are replaced by their plain counterparts, as in these examples from early Ugandan Kinubi: \textit{\textbf{t}owil} ‘long’ < SA \textit{\textbf{ṭ}awīl}; \textit{\textbf{d}ulu} ‘shadow’ < SA \textit{\textbf{ḍ}ull}, \textit{hi\textbf{s}iba} ‘measles’ < SA \textit{ḥi\textbf{ṣ}ba}; \textit{\textbf{z}ulm} ‘to anger’ < SA \textit{\textbf{ẓ}\kern -1ptulum}. Like Juba Arabic, Kinubi substitutes velar stops for the Arabic velar fricatives. Consider the following early Ugandan Kinubi forms: \textit{\textbf{k}idima} ‘work’ < SA \textit{\textbf{ḫ}idma}; \textit{šo\textbf{k}olo} ‘work’ < SA \textit{šo\textbf{ɣ}ol}, \textit{bala\textbf{g}o} ‘commandment’ < SA \textit{balā\textbf{ɣ}} ‘message’. Substratal influence also accounts for consonant degemination, given that the substrate languages “lack these in all but a few morphonologically determined contexts” \citep[162]{Owens1997}. 

Substratal influence can also be seen in the occurrence of certain consonants. Consider first /ɓ/ and /ɗ/: Juba Arabic \textit{\textbf{d'}éngele} ‘liver’ < Bari \textit{denggele}; Juba Arabic \textit{\textbf{b'}ónǧo} ‘pumpkin’ < Bongo \textit{\textbf{b'}onǧo}. The other consonants which occur only in loanwords from the substrate and/or adstrate languages are [p] [v], [ʧ], [ɲ], and [ŋ]: Kinubi \textit{lí\textbf{p}a} ‘to pay’ < Swahili \textit{-li\textbf{p}a}; Kinubi \textit{cam\textbf{p}} ‘camp’ < English \textit{cam\textbf{p}}; Kinubi \textit{\textbf{v}íta} ‘war’ < Swahili \textit{\textbf{v}ita}; Juba Arabic \textit{\textbf{č}am} ‘food’ < Acholi, Belanda Bor, Jur \textit{\textbf{č}ama}, Juba Arabic \textit{\textbf{č}ayniz} < English \textit{\textbf{Ch}inese}, Kinubi \textit{\textbf{č}ay} ‘tea’ < Swahili \textit{\textbf{ch}ai}; Juba Arabic \textit{\textbf{n\kern -1pty}ékem}, Kinubi \textit{\textbf{n\kern -1pty}ékem} ‘chin’ < Bari \textit{\textbf{n\kern -1pty}ékem}, Kinubi \textit{\textbf{n\kern -1pty}á\textbf{n\kern -1pty}a} ‘tomato’ < Swahili \textit{\textbf{n\kern -1pty}a\textbf{n\kern -1pty}a}; Juba Arabic \textit{\textbf{ŋ}\kern -1ptun} ‘divinity’ < Bari \textit{\textbf{ng}un}. The integration of these phonemes is thus a result of borrowing (under RL agentivity) rather than of imposition.

  The following instances of consonant variation are more common in Juba Arabic (\citealt{Manfredi2017}: 25–27). The most frequent is [ʃ] {\textasciitilde} [s]: \textit{geš} {\textasciitilde} \textit{ge\textbf{s}} ‘grass’. Further, [z] is in variation with [ʤ] before /o/ and /a/: \textit{\textbf{z}ówǧu} {\textasciitilde} \textit{\textbf{ǧ}ówǧu} ‘to marry’, \textit{\textbf{z}áman} {\textasciitilde} \textit{\textbf{ǧ}áman} ‘time; when’. There is also [p] {\textasciitilde} [f] variation in word-initial position, including in loanwords: \textit{\textbf{p}oǧúlu} {\textasciitilde} \textit{\textbf{f}oǧúlu} ‘Pojulu’, \textit{\textbf{p}rótestan} {\textasciitilde} \textit{\textbf{f}rótestan} ‘Protestant’. Finally, the phoneme /f/ may also be phonetically realized as [p]: \textit{nédi\textbf{f}u} {\textasciitilde} \textit{nédi\textbf{p}u} ‘to clean’. Of these cases of variation, the latter has been specifically attributed to substratal influence from Bari (\citealt{Miller1989}; \citealt{Manfredi2017}). It might be argued, however, that all these instances of consonant variation reflect the influence of the substrate languages, regardless of their genetic affiliations. The following do not have /š/: Acholi, Avokaya, Baka, Bari, Belanda Bor, Bongo, Dinka, Jur, Lotuho, Ma'di, Moru, Mundu, Nuer, Päri, Shilluk, Zande. Of these, Acholi, Belanda Bor, Bongo, Dinka, Jur, Nuer, Päri and Shilluk do not have /s/ either. A number of substrate languages do not have /z/: Acholi, Bongo, Belanda Bor, Dinka, Jur, Lotuho, Nuer, Päri, and Shilluk. All of these, however, have /ʤ/. Finally, /f/ is not part of the phonological inventory of Acholi, Bongo, Dinka, Jur, Nuer, Päri, and Shilluk, which do, however, have /p/. Given the intricacies of the distribution of /š/, /s/, /z/, /ʤ/, /f/, and /p/ across the substrate languages, the types of variation illustrated are not surprising. 

  As in Juba Arabic, [š] is in variation with [s] in Kinubi (\citealt{Owens1985}: 237; \citealt{Owens1997}: 161; \citealt{Wellens2003}: 38; \citealt{Luffin2005}: 62; \citealt{Avram2017talk}): early Ugandan Kinubi \textit{\textbf{š}abaka} {\textasciitilde} \textit{\textbf{s}abaka} ‘net’). Although it is etymological /š/ which is typically subject to variation, occasionally this also applies to etymological /s/: early Ugandan Kinubi \textit{\textbf{s}ikin} {\textasciitilde} \textit{\textbf{š}ekin} ‘knife’ < SA \textit{sikkīn} \citep{Avram2017talk} and modern Kenyan Kinubi \textit{flu\textbf{š}} {\textasciitilde} \textit{flu\textbf{s}} ‘money’ < SA \textit{fulūs} \citep[63]{Luffin2005}. Note that [š] {\textasciitilde} [s] variation also extends to loanwords from Swahili (\citealt{Wellens2003}: 80; \citealt{Luffin2005}: 63; \citealt{Avram2017talk}): early Ugandan Kinubi \textit{\textbf{š}amba} {\textasciitilde} \textit{\textbf{s}amba} ‘field’ < Swahili \textit{\textbf{sh}amba}. Like Juba Arabic, Kinubi exhibits [z] {\textasciitilde} [ʤ] variation (\citealt{Owens1985}: 235; \citealt{Owens1997}: 161; \citealt{Wellens2003}: 215; \citealt{Luffin2005}: 63; \citealt{Avram2017talk}): early Ugandan Kinubi \textit{\textbf{ǧ}alan} {\textasciitilde} \textit{\textbf{z}alan} ‘angry’ < SA \textit{\textbf{z}aʕlān}. However, unlike Juba Arabic, in Kinubi the [z] {\textasciitilde} [ʤ] variation also occurs before the two front vowels /i/ and /e/: \textit{\textbf{z}e} {\textasciitilde} \textit{\textbf{ǧ}e} ‘as’, early Ugandan Kinubi \textit{an\textbf{ǧ}il} {\textasciitilde} \textit{en\textbf{z}il} ‘descend’. According to \citet[161]{Owens1997}, this “is due perhaps to Bari substratal influence, since Bari has only \textit{j}, not \textit{z}.” In fact, as in the case of Juba Arabic, the same is true of several other substrate languages. Finally, there are instances of [l] {\textasciitilde} [r] variation (\citealt{Wellens2003}: 214; \citealt{Luffin2005}: 65), affecting both etymological /l/ and etymological /r/ in Arabic-derived words, e.g. \textit{tá\textbf{l}e} {\textasciitilde} \textit{tá\textbf{r}e} ‘go out’, \textit{ge\textbf{r}í} {\textasciitilde} \textit{ge\textbf{l}í} ‘near’, and in borrowings, e.g. Ugandan Kinubi \textit{čá\textbf{l}o} {\textasciitilde} \textit{čá\textbf{r}o} ‘village’ < Luganda \textit{e-kya\textbf{l}o}; Kenyan Kinubi \textit{tumbí\textbf{l}i} {\textasciitilde} \textit{tumbí\textbf{r}i} ‘monkey’ < Swahili \textit{tumbi\textbf{l}i}. This variation seems to reflect the influence of Luganda and Swahili. In the former, [l] and [r] are in complementary distribution, with [r] occurring after the front vowels /i/ and /e/, and [l] elsewhere \citep[214]{Wellens2003}, while in the latter [l] and [r] are in free variation \citep[79]{Luffin2014}. 

As in the substrate languages, there is no distinction between short and long vowels: Juba Arabic \textit{sud\textbf{á}ni} ‘Sudanese’ < SA \textit{sud\textbf{ā}nī}, Kinubi \textit{kab\textbf{í}r} ‘big’ < SA \textit{kab\textbf{ī}r}.


 \subsubsection{Morphology}

Apart from the Arabic-derived plural suffixes -\textit{at} and -\textit{in}, Juba Arabic uses the plural marker of Bari origin -\textit{ǧín} (\citealt{Nakao2012}: 131; \citealt{Manfredi2014plural}: 58), which is attached only to loanwords from local languages:


\ea
{Juba Arabic \citep[58]{Manfredi2014plural}}\\
\ea
\gll kɔrɔpɔ-ǧín (\textup{< Bari} \textit{kɔrɔpɔ})\\
    leaf-\textsc{pl}\\
\glt       `leaves' 

\ex
\gll beng-ǧín (< Dinka \textit{beng}) \\
         chief-\textsc{pl}\\
\glt       `chiefs'


\ex
\gll  b'angiri-ǧín (< \textup{Zande} \textit{b'angiri}) \\
         cheek\textsc{-pl}\\
\glt       `cheeks' 
\z
\z

Another phenomenon worth mentioning is the occurrence in the speech of young urban speakers of hybrid forms, which consist of the Bari relativizer \textit{lo-} and a noun either from Arabic or from one of the substrate/adstrate languages \citep[131]{Nakao2012}. Note, however, that there is a functional overlap between Bari \textit{lo}- and Sudanese Arabic \textit{abu}.

\ea\label{ex:key:lo}
{Juba Arabic \citep[46]{Manfredi2017}}\\
\ea\gll lo-beléde (\textup{< Bari} \textit{lo-} \textup{+ SA} \textit{beled})\\
\textsc{rel}-country\\
\glt       `peasant'

\ex
\gll lo-pómbe (< Bari \textit{lo-} + \textup{Swahili} \textit{pombe})\\
 \textsc{rel}-alcohol\\
\glt    `drunkard'
\z
\z

Given that a relatively large number of Bari-derived words contain \textit{lo-} (\citealt{Miller1989}; \citealt{Manfredi2017}: 46), the examples in \REF{ex:key:lo} confirm the fact that morphological innovations are typically introduced through lexical borrowings via RL agentivity, and subsequently become productive in the RL. 

Note, finally, that most of the speakers who use the plural marker -\textit{ǧín} and the relativizer \textit{lo}{}- are dominant in Juba Arabic. These cases therefore confirm the fact that RL monolinguals can be agents of borrowing (\citealt{VanCoetsem1988}: 10).

A small number of Kinubi nouns borrowed from Swahili exhibit the Bantu nominal classifiers:

\ea\label{ex:key:}

{Kinubi \citep[57]{Wellens2003}}\\
 
\ea\textbf{mu}-zé               \textbf{wa}-zé\\
 
{\textsc{nc}1-old man   \textsc{nc2}-old man}\\
 {`old man, old men'}\\
 
\ex
\gll  \textbf{mu}-zukú \textbf{wa}-zukú\\
          \textsc{nc}1-grandchild   \textsc{nc}2-grandchild\\
\glt       `grandchild, grandchildren'

\ex
\gll \textbf{m}-zúngu \textbf{wa}-zúngu\\
         \textsc{nc}1-European   \textsc{nc}2-European\\
\glt       `European, Europeans'
\z
\z

 \subsubsection{Syntax}

Bureng Vincent (\citeyear[77]{BurengVincent1986}) first noted the similarity between the prototypical passive construction in Juba Arabic and its Bari counterpart:


\ea
{Juba Arabic (\citealt{BurengVincent1986}: 77)}\\
\ea\gll  bágara áyinu \textbf{ma} Wáni\\
        cow see.\textsc{past} with Wani\\
\glt      `The cow has been seen by Wani.'

\ex\label{ex:key:}
{Bari (\citealt{BurengVincent1986}: 77)}\\
\gll             kítɜŋ a mɛtà kɔ̀ Wànì\\
                 cow \textsc{past} see with Wani\\
\glt     `The cow has been seen by Wani.'
\z
\z

As can be seen, in both Juba Arabic and Bari the agent is introduced by the comitative preposition ‘with’. This is a case of lexico-syntactic imposition via identification of SL and RL lexemes \citep[415]{Manfredi2018}: the Juba Arabic lexical entry \textit{ma} is derived from Sudanese Arabic \textit{maʕ}, but its semantics reflects the influence of Bari \textit{kɔ̀}. The same is true of Kenyan Kinubi:

\ea\label{ex:key:}
{       Kinubi \citep[230]{Luffin2005}}\\
\gll yal-á al akulú \textbf{ma} nas tomsá\\
     child-\textsc{pl} \textsc{rel} eat.\textsc{past.pass} with \textsc{pl} crocodile\\
\glt     `The children who were eaten by a crocodile.'
\z

Consider next the syntax of numerals in Kinubi (\citealt{Wellens2003}: 90; \citealt{Luffin2014}: 309). Their post-nominal placement is calqued on Swahili:

\ea\label{ex:key:}
{Kinubi \citep[309]{Luffin2014}}\\
\gll wéle \textbf{kámsa} ma baná \textbf{árba}\\
     boy five with girl.\textsc{pl} four\\
\glt     `Five boys and four girls.'
\z

\ea\label{ex:key:}
{Swahili \citep[309]{Luffin2014}}\\
\gll miti \textbf{mia} \textbf{tatu}\\
     tree hundred three\\
\glt     `Three hundred trees.`
\z

With cardinal numerals, the order is hundred + unit and thousand + unit respectively:

 

 \ea\label{ex:key:}
{Kinubi \citep[309]{Luffin2014}}\\

\gll   elf wáy\\
       thousand one\\
\glt      `one thousand'
\z

Kinubi thus follows the Swahili model:

\ea
{Swahili \citep[309]{Luffin2014}}\\
\gll            elfu moja \\
                thousand one \\
\glt     `one thousand'
\z

Consider also a case of syntactic change induced by lexical calquing. Juba Arabic \textit{(fu)wata} ‘ground’ functions as an impersonal subject in weather expressions:

\ea\label{ex:key:}
{Juba Arabic \citep[141]{Nakao2012}}\\
\gll   \textbf{(fu)watá} súkun\\
       ground hot\\
\glt     `It is hot.'
\z

Nakao (\citeyear[141]{Nakao2012}) shows that this is also the case in Acholi and Ma'di:

\ea\label{ex:key:}
{Acholi \citep[141]{Nakao2012}}\\
\gll \textbf{piiny} lyeet\\
     ground warm\\
\glt     `It is warm.'
\z

\ea\label{ex:key:}
{Ma'di \citep[141]{Nakao2012}}\\
\gll \textbf{vu} aci\\
     ground hot\\
\glt     `It is hot.'
\z

In fact, these types of sentences are widespread in Western Nilotic substrate languages, such as Dinka, Jur, Päri, and Shilluk:

\ea\label{ex:key:}
{Dinka \citep[202]{Nebel1979}}\\
\gll            \textbf{piny} a-tuc\\
                ground 3\textsc{sg}-warm\\
\glt     `It is warm.'
\z

  In both Juba Arabic and Kinubi \textit{ras} ‘head’ also occurs in the complex preposition \textit{fi} \textit{ras} ‘on’:

\ea\label{ex:key:}
\ea Juba Arabic \citep[141]{Nakao2012}\\
\gll     merísa fí fi \textbf{ras} terebéza\\
         beer \textsc{exs} on head table\\
\glt       `The beer is on the table.'

\ex\label{ex:key:}
Kinubi \citep[159]{Wellens2003}\\
\gll     fi \textbf{rá}\textbf{s} séder\\
         on head tree\\
\glt       `On top of the tree.'
\z
\z

Nakao (\citeyear[141]{Nakao2012}) attributes this function of \textit{ras} to substratal influence from Acholi and Ma'di:

\ea\label{ex:key:}
{Acholi \citep[141]{Nakao2012}}\\
\gll            cib \textbf{wi}-meja\\
                put head-table\\
\glt     `Put it on the table!'
\z
 
However, other possible sources include Western Nilotic languages such as Belanda Bor, Jur, Päri and Shilluk:

\ea\label{ex:key:}
{Jur  (\citealt{PozzatiPanza1993}: 342)}\\
\gll     kedh ŋo \textbf{wi} tarabesa\\
         put 3\textsc{sg} head table\\
\glt     `Put it on the table.'
\z

Moreover, a preposition ‘on’ derived from the noun ‘head’ is also attested in Bongo (Central Sudanic) and Zande (Niger-Congo):

\ea\label{ex:key:}
{Bongo (\citealt{Moietal2014}: 39)}\\
\gll            ba \textbf{do} mbaa\\
                3\textsc{sg} on car\\
\glt     `He is on a car.'
\z

\ea\label{ex:key:}
{Zande (\citealt{DeAngelis2002}: 288)}\\
\gll            mo mai he \textbf{ri} ngua\\
                2\textsc{sg} put 3\textsc{sg} on wood\\
\glt     `Put it on the wood.'
\z

The verb \textit{gal}/\textit{gale}/\textit{gali} ‘say’ is used in Juba Arabic and Ugandan Nubi as a complementizer, with \textit{verba} \textit{dicendi} and verbs of cognition:

\ea\label{ex:key:}
\ea Juba Arabic \citep[469]{Miller2001}\\
\gll     úwo kélem \textbf{gal} úwo bi-ǧa\\
         3\textsc{sg} speak \textsc{comp} 3\textsc{sg} \textsc{irr}-come\\
\glt       `He said that he would come.'

\ex\label{ex:key:}
Ugandan Kinubi \citep[204]{Wellens2003}\\
\gll     úmon áruf \textbf{gal} fí difan-á al gi-ǧá\\
         3\textsc{pl} know \textsc{comp} \textsc{exs} guest-\textsc{pl} \textsc{rel} \textsc{prog}-come\\
\glt    `They know that there are are guests who are coming.' 
\z
\z

The use of a \textit{verbum} \textit{dicendi} as a complementizer resembles the situation in Bari,\footnote{Unsurprisingly, in Juba Arabic “the use of \textit{adi} as in Bari [is] the most frequent […] in particular among speakers of Bari origin” (\citealt[470]{Miller2001}; author's translation).} where \textit{adi} ‘say’ introduces direct speech (\citealt{Owens1997}: 163; \citealt{Miller2001}: 469): 

\ea\label{ex:key:}

{Bari \citep[469]{Miller2001}}\\
\gll    mukungu a-kulya \textbf{adi} nan d'ad'ar kakitak merya-mukanat\\
                sub-chief \textsc{past}-say \textsc{comp} 1\textsc{sg} want worker fifty\\
\glt     `The sub-chief spoke saying: I want fifty workers.'
\z


 \subsubsection{Lexicon}

Since Bari is the main substrate language of Juba Arabic, unsurprisingly it contributes most of its African-derived words: \textit{gúgu} ‘granary’ < Bari \textit{gugu}, \textit{kení} ‘co-wife’ < Bari \textit{köyini}, \textit{loɲumé}\textit{g} < Bari \textit{lónyumöng}, \textit{tóŋga} ‘pinch’ < Bari \textit{toŋga}. In several cases, the Juba Arabic form can be traced to a specific dialect: \textit{d'oŋóŋ} ‘back of head’ < Pojulu \textit{doŋoŋ}, \textit{láŋa} ‘wander’ < Mundari \textit{laŋa} ‘travel’, \textit{nyéte} vs \textit{ŋéte} ‘black-eyed pea leaf’ < Bari \textit{nyete} vs Kakwa, Pojulu \textit{ŋete}. Moreover, “more Bari lexical items are being borrowed” in Youth Juba Arabic \citep[131]{Nakao2012}: \textit{kapaparát} ‘butterfly’ < Bari \textit{kapoportat}, \textit{lukulúli} ‘bat’ < Bari \textit{lukululi}. Several other substrate and adstrate languages have contributed to the lexicon of Juba Arabic (\citealt{Nakao2012,Nakao2015}): \textit{adúngú} ‘harp’ < Acholi \textit{aduŋu}; \textit{b'ónǧ}\textit{o} ‘pumpkin’ < Bongo \textit{b'onǧo}; \textit{báfura} ‘cassava’ < Dinka \textit{bafora} ‘manioc, (sweet) cassava’; \textit{káwu} ‘cowpea’ < Ma'di \textit{kau}; \textit{malangí} < bottle’ < Bangala/Lingala \textit{molangi}; \textit{kámba} ‘belt’ < Swahili \textit{kamba}; \textit{imbíró} ‘palm tree’ < Zande \textit{mbíró}. Some sixty lexical items found in the earliest records of Ugandan Kinubi can be traced back to various substrate languages \citep{Avram2017talk}: \textit{lawoti} ‘neighbours’ < Acholi \textit{lawoti} ‘fellow, friend’; \textit{korufu} ‘leaf’ < Bari \textit{korofo}/\textit{kɔrɔ}\textit{pɔ} ‘leaves’; \textit{lwar} ‘abscess’ < Dinka \textit{luär} ‘pain of a swelling’; \textit{seri} ‘fence’ < Lugbara \textit{seri} ‘plant used for fencing’; \textit{mukuta} ‘key’ < Päri \textit{mukuta}.

The influence of Luganda and Swahili as adstrate languages is already documented in early Ugandan Kinubi \citep{Avram2017talk}: Ugandan Kinubi \textit{kibra}/\textit{kibera} ‘forest’ < Luganda \textit{e-kibira}, \textit{nyinveza} ‘fix’ < Luganda \textit{nyweza} ‘make firm, hold firmly’; \textit{dirisa} ‘window’ < Swahili \textit{dirisha}; \textit{kibanda} ‘shed’ < Swahili \textit{kibanda} ‘small shed’. The lexicon of modern Ugandan Kinubi is characterized by a large number of loanwords from Luganda and Swahili (\citealt{Wellens2003}; \citealt{Nakao2012}: 133–134), such as: \textit{mé(é}\textit{)mvu} ‘banana’ < Luganda \textit{amaemvu} ‘bananas’, \textit{ntulége} ‘zebra’ < Luganda \textit{e-ntulege}; \textit{karibísha} ‘welcome’ < Swahili \textit{karibisha} ‘welcome’, \textit{sangá}/\textit{šangá} ‘be surprised’ < Swahili \textit{shangaa}. In some cases, these loanwords have replaced previously attested compounds consisting of Arabic-derived elements:\footnote{See also Tosco \& Manfredi (\citeyear[509]{ToscoManfredi2013}).} early Ugandan Kinubi \textit{mária} \textit{bitá} \textit{murhúm} ‘widow’, lit. ‘wife of the deceased’ vs. modern Ugandan Kinubi \textit{mamwándu} ‘widow’ < Luganda \textit{nnamuwandu}. As for the lexicon of modern Kenyan Kinubi, it is strongly influenced by Swahili. \citet{Luffin2004} lists some 170 loanwords from Swahili (out of approximately 1,400 words recorded), from a wide range of domains, for example: \textit{barabára} ‘highway’ < Swahili \textit{barabara}, \textit{serikáli} ‘government’ < Swahili \textit{serikali}, \textit{tafaúti} ‘difference’ < Swahili \textit{tafauti}, \textit{úza} ‘sell’ < Swahili \textit{ku-uza}. Swahili has also contributed several function words: \textit{badáye} ‘after’ < Swahili \textit{baadaye} ‘afterwards’, \textit{íle} ‘these’ < Swahili \textit{ile}, \textit{na} ‘and, with’ < Swahili \textit{na}. Kenyan Kinubi lexical items have occasionally undergone semantic shift or semantic extension under the influence of the meanings of their Swahili counterparts \citep[315]{Luffin2014}: \textit{destúr} ‘tradition’, cf. Swahili \textit{desturi} ‘tradition’; \textit{fáham} ‘to understand, to remember’, cf. Swahili -\textit{fahamu} ‘to understand, to remember’ 

In some cases, the exact origin of loanwords found in Juba Arabic cannot be established: \textit{búra} ‘cat’ < Acholi, Bongo, Dinka, Päri \textit{bura}, Didinga \textit{buura}; \textit{daŋá} ‘bow’ < Bari, Jur \textit{daŋ}, Didinga \textit{d'anga}, Dinka \textit{dhaŋ}; \textit{pondú} ‘cassava leaf’ < Bangala, Kakwa, Lingala \textit{pondu}, Pojulu \textit{pöndu}. The same holds for a number of loanwords attested in early Ugandan Kinubi \citep{Avram2017talk}: \textit{bongo} ‘cloth’ < Acholi, Lendu, Lugbara, Zande \textit{bongo}, Bari \textit{boŋgo}; \textit{godogodo} ‘thin from illness’ < Acholi, Avokaya, Bari, Baka, Lotuho, Moru, Zande \textit{godogodo} ‘very weak, thin, sick(ly)’; \textit{mukungu} ‘headman’ < Acholi, \textit{mukuŋu}, Bari \textit{mʊkʊ}\textit{ŋgʊ}, Lugbara \textit{mukungu}, Luganda \textit{o-mukungu} ‘(sub-) chief’. This is also true of several Kinubi words attested in more recent sources (\citealt{Wellens2003}; \citealt{Nakao2012}: 133–134): \textit{júju} ‘shrew’ < Bari \textit{juju}, Ma'di \textit{juju}; \textit{kingílo} ‘rhinoceros’ < Avokaya \textit{kiŋgili}, Moru \textit{kingile}. In some cases, the occurrence of alternative forms is due to their different SLs: \textit{ban\textbf{ǧ}a} ‘debt’ < Bari \textit{ban\textbf{j}a}, Lugbara \textit{ban\textbf{j}a}, Luganda \textit{e-bban\textbf{j}a} vs. \textit{ban\textbf{y}a} ‘debt’ < Acholi \textit{ban\textbf{y}a}. 

Under the influence of the substrate and adstrate languages, some Arabic-derived lexical items have undergone semantic extension, thereby becoming polysemous in Juba Arabic \citep[136]{Nakao2012}, e.g. \textit{gówi} ‘hard; difficult’, cf. Acholi \textit{tek}, Bari \textit{logo’}, Lotuho \textit{gol}, Ma'di \textit{okpo}, Swahili \textit{kali}.

Juba Arabic “compensates its lexical gaps through the lexification of Arabic morphosyntactic sequences”  (\citealt{ToscoManfredi2013}: 509). A case in point are Juba Arabic compounds, formed via juxtaposition or with their two members linked by the possessive particle \textit{ta} (\citealt{Manfredi2014relex}: 308–309). These include calques after several substrate languages \citep[136]{Nakao2012}, e.g. \textit{ída} \textit{ta} \textit{fil} ‘elephant trunk’, cf. Acholi \textit{ciŋ} \textit{lyec}, Bari \textit{könin} \textit{lo} \textit{tome}, Dinka \textit{ciin} \textit{akɔɔn}, Jur \textit{ciŋ} \textit{lyec}, Lotuho \textit{naam} \textit{tome}, Shilluk \textit{bate} \textit{lyec}, lit. ‘arm (of) elephant’. Kinubi also exhibits a number of calques (\citealt{Nakao2012}; \citealt{Avram2017talk}; Manfredi, this volume).\ia{Manfredi, Stefano@Manfredi, Stefano} Some of these compounds and phrases can be traced to several SLs, as in the following early Ugandan Kinubi examples \citep{Avram2017talk}: \textit{gata} \textit{kalam} ‘decide, judge’, cf. Acholi \textit{ŋɔlɔ} \textit{kop} ‘decide, give judgment’, Bongo \textit{ad'oci} \textit{kudo}, Jur \textit{ŋɔl} \textit{lubo}, Päri \textit{ŋondi} \textit{lubo}, lit. ‘cut word/speech’; Dinka \textit{wèt} \textit{tèm} ‘decide, give the sentence’, lit. ‘word cut’; \textit{jua} \textit{bita} \textit{ter} ‘nest’, cf. Acholi \textit{ot} \textit{winyo}, Bari \textit{kadi-na-kwen}, Belanda Bor \textit{kwɔt} \textit{winy}, Shilluk \textit{wot} \textit{winyo}, Zande \textit{dumô} \textit{zirê}, lit. ‘house (of) bird’. Other calques, presumably more recent ones, reflect the growing influence of Swahili on Kenyan Kinubi \citep[315]{Luffin2014}: \textit{bakán} \textit{wá}\textit{y} ‘together’, cf. Swahili \textit{pamoja} ‘together’, lit. ‘place one’, \textit{mára} \textit{wá}\textit{y} \textit{wáy} ‘seldom’, cf. Swahili \textit{mara} \textit{moja} \textit{moja} ‘seldom’, lit. ‘time one one’.

To conclude, SL agentivity accounts for the small number of loanwords and calques recorded in the earliest stage (i.e. pidginization) of Juba Arabic and Kinubi. At a later stage (i.e. after nativization), the larger number of loanwords and calques is a result of borrowing under RL agentivity.


 \section{Arabic-lexifier pidgins in the Middle East}\label{sec:pid}


 \subsection{Current state and historical development}


Several Arabic-lexifiers pidgins have emerged in the Middle East. These include Romanian Pidgin Arabic, Pidgin Madame, Jordanian Pidgin Arabic, and Gulf Pidgin Arabic. The first three can be classified as work force pidgins.\footnote{These are pidgins which “came into being in work situations” \citep[28]{Bakker1995}.} Gulf Pidgin Arabic also started out as work force pidgin \citep[83]{Smart1990}, but it is now an interethnic contact language \citep[13]{Avram2014Pidgin}.\footnote{That is, one which is “used not just for trade, but also in a wide variety of other domains” \citep[28]{Bakker1995}.}

Romanian Pidgin Arabic \citep{Avram2010} was a short-lived pidgin, formerly used on Romanian well sites in Iraq, in locations in the vicinity of Ammara, Basra, Kut, Nassiriya, Rashdiya and Rumaila. Romanian Pidgin Arabic emerged after 1974, when Romanian well sites started operating in Iraq. Romanians typically made up two thirds of the oil crews, with Arabs making up the final third. The first Gulf War and the subsequent withdrawal of the Romanian oil rigs put an end to the use of Romanian Pidgin Arabic. 

Immigration of Sri Lankan women to Arabic-speaking countries is reported to have started in 1976 \citep[16]{Bizri2010}, but the large influx into Lebanon came later, in the early 1990s. Pidgin Madame is spoken in Lebanon by Sri Lankan female domestic workers and their Arab employers, mostly in the urban centres of the country. 

  Jordanian Pidgin Arabic (\citealt{Al-Salman2013}) is used in the city of Irbid, in the Ar-Ramtha district in the north of Jordan, in interactions between Jordanians and Southeast Asian migrant workers of various linguistic backgrounds. However, only Jordanian Pidgin Arabic as spoken by Bengalis is documented.  

Gulf Pidgin Arabic is a blanket term designating the varieties of pidginized Arabic used in Saudi Arabia and the countries on the western coast of the Arab Gulf, i.e. Kuwait, the United Arab Emirates, Oman, Bahrain, and Qatar. 


 
 \subsection{Contact languages}


The main languages involved in the emergence of Romanian Pidgin Arabic are Romanian, Egyptian Arabic (spoken by immigrant workers), and Iraqi Arabic. A small minority of the participants in the language-contact situation had some knowledge of English.

The other pidginized varieties of Arabic in the Middle East share the characteristic of having various Asian languages as their substrate.\footnote{Bizri (\citeyear[385]{Bizri2014}) therefore suggests the cover term “Asian Migrant Arabic pidgins”.} For Pidgin Madame, the main contact languages are Lebanese Arabic, as the lexifier language, and Sinhalese. Another language, with a much smaller contribution, is English. In the case of Jordanian Pidgin Arabic, the contact languages are mainly Jordanian Arabic and Bengali. The contribution of English is very limited. As for Gulf Pidgin Arabic, it emerged in a contact situation of striking complexity. On the one hand, Arabic, the lexifier language, is represented by several dialects, which are not all subsumed under what is known as Gulf Arabic, in spite of what the name of the pidgin suggests. On the other hand, the number of languages spoken by the immigrant workers is staggering: for instance, in the United Arab Emirates the 200 nationalities and 150 ethnic groups speak some 150 languages. Adding to the complexity of the language-contact situation is the fact that these languages are typologically diverse. Last but not least, English also plays a role in interethnic communication, particularly in the service sector.


 
 \subsection{Contact-induced changes}
 \subsubsection{Phonology}

The phonology of all the pidginized varieties of Arabic in the Middle East exhibits the outcomes of SL agentivity, which also accounts for the occurrence of considerable intra- and  inter-speaker variation (\citealt{Avram2010}: 21–22; \citealt{Bizri2014}: 393; \citealt{Avram2017article}: 133).

Consider first Romanian Pidgin Arabic. The following are features characteristic of speakers with Romanian as L1. The phrayngeals are either replaced or deleted: \textit{\textbf{h}abib} ‘friend’ < IA/EA \textit{\textbf{ḥ}abīb}, \textit{mufta} ‘key’ < IA/EA \textit{muftā\textbf{ḥ}}; \textit{saa} ‘hour’ < IA/EA \textit{sā\textbf{ʕ}a}. Plain consonants are substituted for pharyngealized ones: \textit{hala\textbf{s}} ‘ready’ < IA/EA \textit{ḫalā\textbf{ṣ}}. Both velar fricatives are replaced: \textit{\textbf{h}amsa} ‘five’ < IA/EA \textit{\textbf{ḫ}amsa}; \textit{šo\textbf{g}ol} ‘work (\textsc{n})’ < IA \textit{šu\textbf{ɣ}(u)l}. Geminate consonants are degeminated: \textit{si\textbf{t}a} ‘six’ < IA/EA \textit{si\textbf{tt}a}. There is no distinction between short and long vowels, either in lexical items of Arabic origin or in those from English: \textit{l\textbf{a}zim} ‘must’ < IA/EA \textit{l\textbf{ā}zim}; \textit{sl\textbf{i}p} ‘sleep’ < English \textit{sleep}. A feature typical of speakers with Egyptian or Iraqi Arabic as L1 is the substitution of /b/ for Romanian or English /p/ and /v/: \textit{\textbf{b}i\textbf{b}ul} ‘people, men’ < English \textit{\textbf{p}eo\textbf{p}le}; \textit{gi\textbf{b}} ‘give, bring’ < English \textit{gi\textbf{v}e}.

Consider next a number of selected features, generally typical of Pidgin Madame, Jordanian Pidgin Arabic, and Gulf Pidgin Arabic.  Pharyngeals are either replaced: Pidgin Madame \textit{\textbf{h}areb} ‘war’ < LA \textit{\textbf{ḥ}areb}; Jordanian Pidgin Arabic \textit{bisalli\textbf{h}} ‘repair’ < JA \textit{biṣalli\textbf{ḥ}} ‘repair.\textsc{impf}.\textsc{3sg.m}’; Gulf Pidgin Arabic \textit{a\textbf{k}san} ‘best’ < GA \textit{a\textbf{ḥ}san}, \textit{\textbf{h}ut} ‘put’ < GA \textit{\textbf{ḥ}uṭṭ} ‘put.\textsc{imp.2sg.m}’; or deleted: Pidgin Madame \textit{\textbf{ē}ki} ‘cry’ < LA \textit{ə\textbf{ḥ}ki} ‘cry.\textsc{imp}.2\textsc{sg}.\textsc{f}’; Jordanian Pidgin Arabic \textit{\textbf{a}rabi} ‘Arabic’ < JA \textit{\textbf{ʕ}arabi}; Gulf Pidgin Arabic \textit{araf} ‘know’ < GA \textit{\textbf{ʕ}araf}. The pharyngealized consonants are replaced by plain counterparts:  Pidgin Madame \textit{\textbf{s}arep} ‘envelope’ < LA \textit{\textbf{ẓ}aref}; Jordanian Pidgin Arabic \textit{bandora} ‘tomato’ < JA \textit{ban\textbf{ḍ}ōra}; Gulf Pidgin Arabic \textit{hala\textbf{s}} < GA \textit{ḫalā\textbf{ṣ}}; or they are realized as retroflex ones: Pidgin Madame \textit{\textbf{ʈ}awīle} ‘long’ < LA \textit{\textbf{ṭ}awīle} ‘long.\textsc{f}.\textsc{sg}’. The velar stops are replaced by velar stops or, less frequently, by /h/: Pidgin Madame \textit{so\textbf{k}on} ‘warm’ < LA \textit{su\textbf{ḫ}un} ‘warm’, \textit{so\textbf{g}ol} < LA \textit{šə\textbf{ɣ}əl} ‘work’; Jordanian Pidgin Arabic \textit{\textbf{k}amsa} ‘five’ < JA \textit{\textbf{ḫ}amsa}, \textit{su\textbf{k}ul} ‘work (\textsc{n})’ < JA \textit{šu\textbf{ɣ}l}, \textit{za\textbf{g}īr} ‘small’ < JA \textit{ṣa\textbf{ɣ}īr}; Gulf Pidgin Arabic \textit{\textbf{k}ubus} ‘bread’ < GA \textit{\textbf{ḫ}ubz}, \textit{\textbf{h}alas} ‘finish’ < GA \textit{\textbf{ḫ}alaṣ}; \textit{yisto\textbf{k}ol} ‘work’ < GA \textit{yištu\textbf{ɣ}ul} ‘work.\textsc{impf.3sg.m}’, \textit{šu\textbf{g}l} ‘work’ < GA \textit{šu\textbf{ɣ}l}. Geminate consonants generally undergo degemination (\citealt{Næss2008}: 36; \citealt{Avram2014Pidgin}: 15): Jordanian Pidgin Arabic \textit{si\textbf{t}in} ‘sixty’ < JA \textit{si\textbf{tt}īn}; Gulf Pidgin Arabic \textit{si\textbf{t}a} ‘six’ < GA \textit{si\textbf{tt}a}.

  Moreover, consonants not found in the L1s of the users of Gulf Pidgin Arabic may also be replaced. For instance, Indonesian, Javanese, Sinhalese and Tagalog speakers may substitute [p] for /f/: Pidgin Madame \textit{\textbf{p}alē\textbf{p}il} ‘falafel’ < LA \textit{\textbf{f}alē\textbf{f}il}; Jordanian Pidgin Arabic \textit{\textbf{p}i} ‘in’ < JA \textit{fī} Gulf Pidgin Arabic \textit{na\textbf{p}ar} ‘person’ < GA \textit{na\textbf{f}ar}; Indonesian and Sinhalese speakers may realize /z/ as [s] or [ʤ]: Pidgin Madame \textit{e\textbf{s}a} ‘if’ < LA \textit{iza}; Gulf Pidgin Arabic \textit{\textbf{s}ēn} {\textasciitilde} \textit{\textbf{ʤ}ēn} ‘good’ < GA \textit{\textbf{z}ēn} (\citealt{Bizri2014}: 393; \citealt{Avram2017article}: 133). Bengali and Sinhalese speakers may replace /š/ with [s]: Pidgin Madame \textit{\textbf{s}ū} ‘what’ < LA \textit{\textbf{š}ū}; Jordanian Pidgin Arabic \textit{\textbf{s}u} ‘what’ < JA \textit{\textbf{š}ū}.

Finally, although phonetically long vowels do occur, vowel length is not distinctive, as shown by the occurrence of variation, e.g. Gulf Pidgin Arabic \textit{b\textbf{a}d\textbf{e}n} {\textasciitilde} \textit{b\textbf{a}d\textbf{ē}n} ‘then’ < GA \textit{baʕd\textbf{ē}n}. 


 \subsubsection{Syntax} 

There is relatively little that can be attributed to SL agentivity in the syntax of the Arabic-lexifier pidgins in the Middle East (\citealt{Almoaily2013}; \citealt{Al-Salman2013}; \citealt{Avram2014Pidgin}; \citealt{Bizri2014}; \citealt{Avram2017article}; \citealt{Bakir2017}).

Since the substrate of these varieties, with the exception of Romanian Pidgin Arabic, consists of many SOV languages, e.g. Bengali, Hindi/Urdu, Malayalam, Punjabi, Persian, Sinhalese, Tamil, this word order is occasionally attested (\citealt{Avram2017article}: 133–134; \citealt{Bizri2014}: 403). For instance, direct objects may occur in pre-verbal position:                                                              


\ea
\ea Pidgin Madame \citep[227]{Bizri2010}\\
\gll     misʈer kilot sīli\\
         mister underwear take off\\
\glt       `Mister takes off his underwear.'     

\ex Gulf Pidgin Arabic \citep[133]{Avram2017article}\\
\gll     ana čiko sūp\\
         1\textsc{sg} child see\\
\glt       `I will see my children.'                                        
\z                                             
\z

In attributive possession constructions the order of constituents is possessor–possessee:                                                                                        
\ea\label{ex:key:}
\ea Pidgin Madame \citep[198]{Bizri2010} \\
\gll     kullu māmā benet \\
         all mother girl    \\
\glt `All mother’s girls.'

\ex Gulf Pidgin Arabic (\citealt{Næss2008}: 87)\\
\gll     ana jawd bādēn ysīr Jakarta stokol\\
         1\textsc{sg} husband then go Jakarta work\\
\glt       `Then my husband went to work in Jakarta.' 
\z
\z

Adjectives generally precedes the nouns they modify:

\ea\label{ex:key:}
{ Pidgin Madame \citep[119]{Bizri2010}}\\
\gll   \textbf{bī}\textbf{r} bēt\\
       big house\\
\glt     `A big house.'
\z

Similarly, adverbs precede the adjectives they modify:

\ea\label{ex:key:}
\ea Pidgin Madame \citep[119]{Bizri2010}\\
\gll     \textbf{ʈīr} gūɖ\\
         very good\\
\glt       `very good'

\ex\label{ex:key:}
Gulf Pidgin Arabic \citep[25]{Avram2014Pidgin}\\
\gll     \textbf{sem}-\textbf{sem} kalām\\
         same speak \\
\glt       `They speak in the same way.'
 \z
 \z

  Occasional instances of postpositions are attested:
\ea\label{ex:key:}
\ea Pidgin Madame \citep[132]{Bizri2010}\\
\gll     mister \textbf{mayik} masārē\\
         mister with money\\
\glt    `Mister has the money.'

\ex\label{ex:key:}
Gulf Pidgin Arabic \citep[25]{Avram2014Pidgin}\\
\gll     zamal \textbf{fok} \\
         camel above\\
\glt       `Above the camel.'
\z
\z

Interestingly, Pidgin Madame has a focalized negative copula, derived etymologically from English \textit{no}:

\ea\label{ex:key:}
{ Pidgin Madame \citep[133]{Bizri2010}}\\
\gll   māmā bīrūt \textbf{no}\\
       mother Beirut \textsc{neg}.\textsc{foc}\\
\glt     `It’s not in Beirut that my mother is.'
\z

This resembles the Sinhalese negator \textit{nemiyi}, which “is used only in focalized phrases” \citep[69]{Bizri2010}:

\ea\label{ex:key:}
{ Pidgin Madame \citep[69]{Bizri2010}}\\
\gll   bat kāve mama \textbf{nemeyi}\\
       rice ate 1\textsc{sg} \textsc{neg}.\textsc{foc}\\
\glt     `It is not I who ate the rice.'
\z

 \subsubsection{Lexicon}

Imposition under SL agentivity accounts for the fact that there are few instances of transfer of lexical items from the various SLs into the non-dominant RL (i.e. the pidgin). 

The lexicon of Romanian Pidgin Arabic includes words of Romanian and English origin \citep[32]{Avram2010}: \textit{mašina} ‘car’ < Romanian \textit{maşină}, \textit{sonda} ‘oil rig’ < Romanian \textit{sonda}; \textit{spik} ‘speak, say, tell’ < English \textit{speak}, \textit{tumač} ‘much, many’ < English \textit{too} \textit{much}. Occasionally, non-Arabic words undergo semantic extension under the influence of phonetically similar Arabic words \citep[32]{Avram2010}: \textit{gib} ‘give; bring’ < English \textit{give}, cf. EA \textit{gīb} ‘bring.\textsc{imp.2sg.m}’. 

  The lexicon of all the other pidginized varieties of Arabic spoken in the Middle East includes loanwords from English: Pidgin Madame \textit{ambasi} < English \textit{embassy}; \textit{go} < English \textit{go}, \textit{kam} < English \textit{come}, \textit{no} \textit{gūɖ} ‘bad’ < English \textit{no} \textit{good}, \textit{oké} < English \textit{OK}; Jordanian Pidgin Arabic \textit{bēbi} ‘child’ < English \textit{baby}, \textit{finiš} ‘finish’ < English \textit{finish}, \textit{fisa} ‘visa’ < English \textit{visa}; Gulf Pidgin Arabic \textit{hazband} < English \textit{husband}, \textit{pēšent} ‘patient’ < English \textit{patient}. However, as noted by \citet[113]{Smart1990} concerning Gulf Pidgin Arabic, “it is difficult to say […] whether they are a true part of the pidgin” or rather nonce borrowings.  

Given the extreme diversity of the substrate, it is not surprising that only a few words from the SLs have made it into the lexicon of Gulf Pidgin Arabic (\citealt{Avram2017article}: 134–135): \textit{ača} ‘fine’ < Urdu \textit{achā} ‘good, very well’, \textit{ǧaldi}/\textit{ǧeldi}/\textit{ǧeldi} < Urdu/Hindi \textit{jaldī} ‘quick’. 

Jordanian Pidgin Arabic and Gulf Pidgin Arabic exhibit light verb constructions which may well be calques after Bengali (noun/adjective + \textit{kara} ‘make’) and/or Hindi/Urdu (noun/adjective + \textit{karnā} ‘make’) and/or Persian – noun/adjective + \textit{kardan} ‘make’): Jordanian Pidgin Arabic \textit{sawwi} \textit{zadīd} ‘renew’, lit. ‘make new’; Gulf Pidgin Arabic \textit{sawwi} \textit{suāl} ‘ask’, lit. ‘make a question’, \textit{sawwi} \textit{zalān} ‘upset’, lit. ‘make angry’. 

\section{Conclusion}\label{sec:conc}

This chapter has shown that Arabic-lexifier contact languages emerged primarily through imposition under SL agentivity, in line with the typology of contact languages (\citealt{Winford2005}: 396; 2008: 128). 

The effects of imposition are most obvious in the phonology, syntax and the syntax-semantics interface, and to a lesser extent in the morphology and the lexicon. In the phonology, SL agentivity induces the loss or replacement of certain phonemes not found in the SLs. However, there are also instances of imposition in the sense of transfer from the SLs. As seen, for example, in Turku and Bongor Arabic, some consonants occur only in loanwords from the substrate languages. The occurrence of such loanwords confirms the fact that imposition under SL agentivity may include transfer of lexical items into the RL. Borrowing under RL agentivity has generally played a far less significant role in the development of Arabic pidgins and creoles. As expected, it mostly involves transfer of lexical items; these may lead to the borrowing of certain consonant phonemes, as seen in, for example, Juba Arabic and Kinubi. Finally, borrowing has been shown to include transfer of morphological material as well.

A notable difference between Juba Arabic and Kinubi on the one hand, and the Arabic-lexifiers in the Middle East on the other hand, resides in the relative weight of imposition under SL agentivity and borrowing under RL agentivity. As we have seen, Juba Arabic and Kinubi exhibit the effects of both imposition in their earliest stage (i.e. pidginization), and of borrowing in their latest stage (i.e. nativization). In contrast, imposition is pervasive in the Arabic-lexifier pidgins in the Middle East, given that these varieties have not undergone nativization.

  There are still a number of issues awaiting resolution. For instance, the identification of the SLs is rendered difficult by their number and typological diversity. This difficulty is further compounded by the fact that some substrate languages are still under- researched. This is particularly the case of the substrate languages of Juba Arabic and Kinubi. Also, the distinction between substrate and adstrate languages is blurred \citep[132]{Nakao2012}, particularly when varieties emerge and develop \textit{in} \textit{situ}, as, for example, with Juba Arabic. Further research also needs to consider the effects of the existence of a creole continuum in Juba Arabic as well as of bilingual and monolingual speakers of the language on the relative importance of restructuring, imposition and borrowing. The extent of restructuring and imposition, for instance, is presumably much greater in basilectal and L2 varieties, as opposed to acrolectal and monolingual varieties of the language. The same holds for Bongor Arabic, which, as shown, appears to be undergoing depidginization. Last but not least, further investigations are necessary to establish whether Gulf Pidgin Arabic is evolving towards stabilization, possibly becoming closer to its lexifier, via borrowing of morphological material or is rather undergoing constant repidginization, essentially via imposition.  

\section*{Further reading}

\citet{Miller1993}, \citet{Nakao2012}, and \citet{Luffin2014} illustrate in detail substratal and adstratal influence on Juba Arabic and Kinubi.\\
\citet{Avram2019} analyzes the substratal input in the lexicon of Turku.\\
\citet{Avram2017article} and \citet{Bakir2017} discuss the various sources of Gulf Pidgin Arabic.\\

\section*{Abbreviations}
\begin{tabularx}{.45\textwidth}{lQ}
\textsc{1, 2, 3} & 1st, 2nd, 3rd person \\
ChA & Chadian Arabic \\
EA & Egyptian Arabic \\
\textsc{exs} & existential \\
\textsc{foc} & focus \\
GA & Gulf Arabic \\
IA & Iraqi Arabic \\
\textsc{impf} & imperfect \\
JA & Jordanian Arabic \\
LA & Lebanese Arabic \\
lit. & literally \\
\textsc{n} & noun \\
\textsc{nc} & noun class \\
\end{tabularx}
\begin{tabularx}{.45\textwidth}{lQ}
\textsc{neg} & neg \\
\textsc{obj} & object \\
\textsc{pass} & passive \\
\textsc{past} & past \\
\textsc{pl} & plural \\
\textsc{poss} & possessive \\
\textsc{prf} & perfect \\
\textsc{prox} & proximal \\
\textsc{rel} & relative \\
RL & recipient language \\
SA & Sudanese Arabic \\
SL & source language \\
\textsc{sg} & singular \\
\textsc{v} & verb \\
\end{tabularx}



\sloppy\printbibliography[heading=subbibliography,notkeyword=this]\end{document}