\documentclass[output=paper]{langsci/langscibook} 
\author{Lameen Souag\affiliation{CNRS, Lacito}}
\title{Berber}
\abstract{Arabic has influenced Berber at all levels – not just lexically, but phonologically, morphologically, and syntactically – to an extent varying from region to region. Arabic influence is especially prominent in smaller northern and eastern varieties, but is substantial even in the largest varieties; only in Tuareg has Arabic influence remained relatively limited.  This situation is the result of a long history of large-scale asymmetrical bilingualism often accompanied by language shift.}
\IfFileExists{../localcommands.tex}{
  % add all extra packages you need to load to this file 
\usepackage{graphicx}
\usepackage{tabularx}
\usepackage{amsmath} 
\usepackage{multicol}
\usepackage{lipsum}
\usepackage[stable]{footmisc}
\usepackage{adforn}
%%%%%%%%%%%%%%%%%%%%%%%%%%%%%%%%%%%%%%%%%%%%%%%%%%%%
%%%                                              %%%
%%%           Examples                           %%%
%%%                                              %%%
%%%%%%%%%%%%%%%%%%%%%%%%%%%%%%%%%%%%%%%%%%%%%%%%%%%%
% remove the percentage signs in the following lines
% if your book makes use of linguistic examples
\usepackage{./langsci/styles/langsci-optional} 
\usepackage{./langsci/styles/langsci-lgr}
\usepackage{morewrites} 
%% if you want the source line of examples to be in italics, uncomment the following line
% \def\exfont{\it}

\usepackage{enumitem}
\newlist{furtherreading}{description}{1}
\setlist[furtherreading]{font=\normalfont,labelsep=\widthof{~},noitemsep,align=left,leftmargin=\parindent,labelindent=0pt,labelwidth=-\parindent}
\usepackage{phonetic}
\usepackage{chronosys,tabularx}
\usepackage{csquotes}
\usepackage[stable]{footmisc} 

\usepackage{langsci-bidi}
\usepackage{./langsci/styles/langsci-gb4e} 

  \makeatletter
\let\thetitle\@title
\let\theauthor\@author 
\makeatother



\newcommand{\togglepaper}[1][0]{ 
  \bibliography{../localbibliography}
  \papernote{\scriptsize\normalfont
    \theauthor.
    \thetitle. 
    To appear in: 
    Christopher Lucas and Stefano Manfredi (eds.),  
    Arabic and contact-induced language change
    Berlin: Language Science Press. [preliminary page numbering]
  }
  \pagenumbering{roman}
  \setcounter{chapter}{#1}
  \addtocounter{chapter}{-1}
}

\newfontfamily\Parsifont[Script=Arabic]{ScheherazadeRegOT_Jazm.ttf} 
\newcommand{\arabscript}[1]{\RL{\Parsifont #1}}
\newcommand{\textarabic}[1]{{\arabicfont #1}}

\newcommand{\textstylest}[1]{{\color{red}#1}}

\patchcmd{\mkbibindexname}{\ifdefvoid{#3}{}{\MakeCapital{#3}
}}{\ifdefvoid{#3}{}{#3 }}{}{\AtEndDocument{\typeout{mkbibindexname could
not be patched.}}}

\newcommand{\R}{r\kern-.05ex{̣}\kern.05ex}

 
  %% hyphenation points for line breaks
%% Normally, automatic hyphenation in LaTeX is very good
%% If a word is mis-hyphenated, add it to this file
%%
%% add information to TeX file before \begin{document} with:
%% %% hyphenation points for line breaks
%% Normally, automatic hyphenation in LaTeX is very good
%% If a word is mis-hyphenated, add it to this file
%%
%% add information to TeX file before \begin{document} with:
%% %% hyphenation points for line breaks
%% Normally, automatic hyphenation in LaTeX is very good
%% If a word is mis-hyphenated, add it to this file
%%
%% add information to TeX file before \begin{document} with:
%% \include{localhyphenation}
\hyphenation{
Engels-zungen
affri-ca-te
affri-ca-tes
Bongor
Lotuho
Sprach-inseln
under-going
monkey
fortiori
kapaparát
báfura
imbíró
Bangala
Bettega
Bab-ylon-ian
Kouwenberg
Ammani
Amman
Ammanis
scen-arios
elabor-ate
elabor-ated
func-tion-al-ization
anthro-polo-gist
Iemmolo
table
Martin
gahawa
Tihāma
Sason
com-ple-ments
homo-phon-ous
Meso-potam-ian
Diyar-bakır
Tsiapera
Lucas
edu-ca-tion
cubit-ellu
corbata
morpho-phonology
morpho-phono-logic-al-ly
Volubilis
Mennen
Papa-constan-tinou
Vicente
re-inter-pre-ta-tion
megorashim
Malayalam
morpho-phon-em-ic-s
morpho-syn-tax
morpho-syn-tactic
Ki-nubi
ḥawār-iyy-ūn
archa-ic-ity
de-velop-ment
pro-so-dic
in-duced
pho-no-logy
possess-um
pos-ses-s-ive-s
Birgitt
es-pe-cial-ly
clause-s
po-ten-tial-ly
Lusignan
Cush-itic
re-fi-ni-ti-za-tion
lan-guage-s
lan-guage
dic-tion-ary
dic-tion-aries
socio-political
eight
counter-part
de-on-tic
Walter
Mifsud
ana-lys-is
re-ana-lys-is
re-ana-lyse
re-ana-lysed
re-ana-lyses
re-ana-lys-ing
Catalan
Occitan
Vigario
Sónia
ḥammāl
cross-linguist-ic-al-ly
south-western
bor-row-ed
ǧumruk
neg-ative
de-termin-er
Oxford
Socotra
Khuze-stan
Arama-ic
Otto-man
Otto-mans
Azeri
bi-lin-gual-ism
bor-row-ing
dia-lects
dia-lect
Ahwāzī
Catherine
allo-phone
allo-phones
phon-em-ic
hannaʔa
tubbaʕ
len-ition
Khorramshahr
poss-ible
Moroccan
Martine
Harrassowitz
gram-matica-lization
Wolfram
Afro-asiatica
Afro-asiatic
Semit-istik
mono-phthong
mono-phthong-ized
mono-phthong-izing
Heselwood
Hinrichs
Watson
Aramaeo-grams
Dadan-it-ic
sub-ordin-ator
Thamud-ic
Revue
Bovingdon
un-der
attach
attached
bundle
Arabia
graph-em-ic
graph-emes
ci-ca-da
Corriente
Andalusi
Kossmann
mor-pho-lo-gi-cal
dia-chroniques
islámica
occid-ent-al-ismo
idiom-a-ti-city
dia-lecto-logie
Reichert
co-lo-nial
Milton
diph-thong-al
lin-guis-tic
lin-guis-tique
lin-guis-tics
affairs
pho-net-ic-ally
kilo-metres
meta-typy
Jagiellonian
Arcodia
Hussein
Christopher
Giannakidou
Tashelhiyt
sta-bi-li-zation
de-vel-op-ments
in-vest-i-gation
Jor-dan-ian
af-fri-ca-tion
de-affri-ca-tion
Jordan
inter-dent-al
inter-locutors
socio-historical
roman-ization
Kurmanji
pha-ryn-geal-ization
con-sonant
con-sonants
phi-lo-logy
Öpengin
Kurd-ish
sunbul
sekin-în
fatala
dia-lect-al
Matras
mor-phol-o-gy
ḥaddād
emer-gence
Herzog
Benjmamins
struc-ture
Orient-studien
Ana-tolia
Ana-tolian
American
vari-ation
Jastrow
Geoffrey
Yarshater
Ḥarsūsi
Umberto
Ashtiany
Edmund
Mahnaz
En-cyclo-pædia
En-cyclo-paedia
En-cyclo-pedia
Leiden
dia-spora
Sasan-ian
every-day
domin-ance
Con-stitu-tion-al
Manfredi
pro-duct-iv-ity
Morocco
ferman
Maghreb-ian
algérien
Nicolaï
Mouton
maur-i-tani-en
Gotho-burg-ensis
socio-linguist-ique
socio-linguistic
dia-chronica
dihāt
de-velop-ed
ṣuḥbat
kitāba
kitābat
com-mercial
Senegal
mechan-ics
Maur-itan-ia
Ḥassān-iyya
circum-cision
cor-relation
labio-velar-ization
vowel
vowels
īggīw
series
in-tegrates
dur-ative
inter-dent-als
gen-itive
Tuareg
tălămut
talawmāyət
con-diment
Wiesbaden
Uni-vers-idad
Geuthner
typo-logie
Maur-itanie
nomades
Maur-itan-ian
dia-lecto-logy
dia-lecto-logical
Sahar-iennes
Uni-ver-sity
de-scend-ants
NENA-speak-ing
speak-ing
xaddaːm
Kahane
Tinniswood
Ven-etian
rabia
Cohen
dubi-ous
Al-geria
trans-humance
Marrakech
Bender
Munich
origin-al
re-captured
ethnic
minor-it-ies
drama-tic
local
long-stand-ing
regions
Nineveh
misti-linguismo
settle-ments
Ṣəndor
Mandate
sub-sti-tu-ting
or-thog-ra-phy
ref-er-enced
origin-ate
twenti-eth
typ-ic-al-ly
Hobrack
never-the-less
masc-uline
coffee
ex-clu-sive-ly
ver-bal
sim-i-lar-i-ties
der-i-va-tion
im-pera-tive
par-ti-ci-ple
dis-am-big-uation
dis-am-big-uating
phen-omen-on
phen-omen-a
traktar
Coetsem
mVCCūC
kān-at
Ing-ham
uni-versidad
Kerman-shah
Brin-cat
Tarifiyt
Vanhove
pre-verbal
Morris
Soqoṭri
dis-pre-fer-red
ex-pla-nation
con-struc-tion
Behnstedt
Andersen
wide-spread
usu-al-ly
re-gion-al
Bulut
con-sider-ation
afro-asia-tici
Franco-Angeli
Phono-logie
Volks-kundliche
dia-lectes
dia-lecte
dis-appear-ance
media
under-stand-able
pub-li-cation
second-ary
e-ject-ive
rev-o-lu-tion
re-stric-tive
Gasparini
moun-tain
moun-tains
yellow
label-ing
tra-di-tion-al-ly
cur-rent-ly
dia-chronic
Hebron
sub-se-quent-ly
doc-u-men-tation
Dom-ari
inter-actions
po-ten-tial
in-ven-tory
sim-pli-fi-cation
da-tive
pro-nom-in-al
or-i-gin
pre-po-si-tion
in-flec-tion-al
cat-e-gor-ic-al
in-di-cates
poss-ess-ive
de-termin-ation
rep-re-sent
e-lici-tation
typo-logy
ac-tiv-i-ty
ac-tiv-i-ties
fellows
in-ter-pre-tation
lin-guists
cycle
ne-ga-tion
sub-con-tinent
va-len-cy
ca-te-gor-ies
com-par-ison
gram-mat-i-cal
cor-res-pond-ing
ob-ser-vations
as-pec-tual
in-di-cation
ser-vice
iden-ti-fy-ing
utter-ances
no-ta-bly
for-mation
dis-tin-guish
mutand-is
muta-tis
in-di-vid-uals
lingua
natur-al
sub-sequent
re-cur-rence
rel-ative
rel-atives
mil-i-tary
ex-po-sure
spe-cif-ic-al-ly
rep-li-cate
rep-li-ca-ting
mod-i-fi-cations
gen-der
ṣəḥəttux
dra-ma-tic-al-ly
ac-cen-tuated
his-tor-i-cal
his-tor-i-cal-ly
event-ual-ly
pres-tige
dis-ap-pear
Romance
pen-insula
pen-insular
fea-ture
con-stitute
either
period
korufu
poly-sem-ous
Zammit
namrag
earli-er
pis-ellum
qillīd
Ta-rif-it
Reinhardt
situ-ation
Mohand
Zazaki
form-ación
Científicas
Roman-ica
butter-fly
roman-cismos
espec-ial-mente
Alexandrine
Ziamari
oriente
SELAF
cultur-elles
maroc-ain
medi-eval
medi-terranean
ɣesmat
Dordrecht
Trieste
}
\hyphenation{
Engels-zungen
affri-ca-te
affri-ca-tes
Bongor
Lotuho
Sprach-inseln
under-going
monkey
fortiori
kapaparát
báfura
imbíró
Bangala
Bettega
Bab-ylon-ian
Kouwenberg
Ammani
Amman
Ammanis
scen-arios
elabor-ate
elabor-ated
func-tion-al-ization
anthro-polo-gist
Iemmolo
table
Martin
gahawa
Tihāma
Sason
com-ple-ments
homo-phon-ous
Meso-potam-ian
Diyar-bakır
Tsiapera
Lucas
edu-ca-tion
cubit-ellu
corbata
morpho-phonology
morpho-phono-logic-al-ly
Volubilis
Mennen
Papa-constan-tinou
Vicente
re-inter-pre-ta-tion
megorashim
Malayalam
morpho-phon-em-ic-s
morpho-syn-tax
morpho-syn-tactic
Ki-nubi
ḥawār-iyy-ūn
archa-ic-ity
de-velop-ment
pro-so-dic
in-duced
pho-no-logy
possess-um
pos-ses-s-ive-s
Birgitt
es-pe-cial-ly
clause-s
po-ten-tial-ly
Lusignan
Cush-itic
re-fi-ni-ti-za-tion
lan-guage-s
lan-guage
dic-tion-ary
dic-tion-aries
socio-political
eight
counter-part
de-on-tic
Walter
Mifsud
ana-lys-is
re-ana-lys-is
re-ana-lyse
re-ana-lysed
re-ana-lyses
re-ana-lys-ing
Catalan
Occitan
Vigario
Sónia
ḥammāl
cross-linguist-ic-al-ly
south-western
bor-row-ed
ǧumruk
neg-ative
de-termin-er
Oxford
Socotra
Khuze-stan
Arama-ic
Otto-man
Otto-mans
Azeri
bi-lin-gual-ism
bor-row-ing
dia-lects
dia-lect
Ahwāzī
Catherine
allo-phone
allo-phones
phon-em-ic
hannaʔa
tubbaʕ
len-ition
Khorramshahr
poss-ible
Moroccan
Martine
Harrassowitz
gram-matica-lization
Wolfram
Afro-asiatica
Afro-asiatic
Semit-istik
mono-phthong
mono-phthong-ized
mono-phthong-izing
Heselwood
Hinrichs
Watson
Aramaeo-grams
Dadan-it-ic
sub-ordin-ator
Thamud-ic
Revue
Bovingdon
un-der
attach
attached
bundle
Arabia
graph-em-ic
graph-emes
ci-ca-da
Corriente
Andalusi
Kossmann
mor-pho-lo-gi-cal
dia-chroniques
islámica
occid-ent-al-ismo
idiom-a-ti-city
dia-lecto-logie
Reichert
co-lo-nial
Milton
diph-thong-al
lin-guis-tic
lin-guis-tique
lin-guis-tics
affairs
pho-net-ic-ally
kilo-metres
meta-typy
Jagiellonian
Arcodia
Hussein
Christopher
Giannakidou
Tashelhiyt
sta-bi-li-zation
de-vel-op-ments
in-vest-i-gation
Jor-dan-ian
af-fri-ca-tion
de-affri-ca-tion
Jordan
inter-dent-al
inter-locutors
socio-historical
roman-ization
Kurmanji
pha-ryn-geal-ization
con-sonant
con-sonants
phi-lo-logy
Öpengin
Kurd-ish
sunbul
sekin-în
fatala
dia-lect-al
Matras
mor-phol-o-gy
ḥaddād
emer-gence
Herzog
Benjmamins
struc-ture
Orient-studien
Ana-tolia
Ana-tolian
American
vari-ation
Jastrow
Geoffrey
Yarshater
Ḥarsūsi
Umberto
Ashtiany
Edmund
Mahnaz
En-cyclo-pædia
En-cyclo-paedia
En-cyclo-pedia
Leiden
dia-spora
Sasan-ian
every-day
domin-ance
Con-stitu-tion-al
Manfredi
pro-duct-iv-ity
Morocco
ferman
Maghreb-ian
algérien
Nicolaï
Mouton
maur-i-tani-en
Gotho-burg-ensis
socio-linguist-ique
socio-linguistic
dia-chronica
dihāt
de-velop-ed
ṣuḥbat
kitāba
kitābat
com-mercial
Senegal
mechan-ics
Maur-itan-ia
Ḥassān-iyya
circum-cision
cor-relation
labio-velar-ization
vowel
vowels
īggīw
series
in-tegrates
dur-ative
inter-dent-als
gen-itive
Tuareg
tălămut
talawmāyət
con-diment
Wiesbaden
Uni-vers-idad
Geuthner
typo-logie
Maur-itanie
nomades
Maur-itan-ian
dia-lecto-logy
dia-lecto-logical
Sahar-iennes
Uni-ver-sity
de-scend-ants
NENA-speak-ing
speak-ing
xaddaːm
Kahane
Tinniswood
Ven-etian
rabia
Cohen
dubi-ous
Al-geria
trans-humance
Marrakech
Bender
Munich
origin-al
re-captured
ethnic
minor-it-ies
drama-tic
local
long-stand-ing
regions
Nineveh
misti-linguismo
settle-ments
Ṣəndor
Mandate
sub-sti-tu-ting
or-thog-ra-phy
ref-er-enced
origin-ate
twenti-eth
typ-ic-al-ly
Hobrack
never-the-less
masc-uline
coffee
ex-clu-sive-ly
ver-bal
sim-i-lar-i-ties
der-i-va-tion
im-pera-tive
par-ti-ci-ple
dis-am-big-uation
dis-am-big-uating
phen-omen-on
phen-omen-a
traktar
Coetsem
mVCCūC
kān-at
Ing-ham
uni-versidad
Kerman-shah
Brin-cat
Tarifiyt
Vanhove
pre-verbal
Morris
Soqoṭri
dis-pre-fer-red
ex-pla-nation
con-struc-tion
Behnstedt
Andersen
wide-spread
usu-al-ly
re-gion-al
Bulut
con-sider-ation
afro-asia-tici
Franco-Angeli
Phono-logie
Volks-kundliche
dia-lectes
dia-lecte
dis-appear-ance
media
under-stand-able
pub-li-cation
second-ary
e-ject-ive
rev-o-lu-tion
re-stric-tive
Gasparini
moun-tain
moun-tains
yellow
label-ing
tra-di-tion-al-ly
cur-rent-ly
dia-chronic
Hebron
sub-se-quent-ly
doc-u-men-tation
Dom-ari
inter-actions
po-ten-tial
in-ven-tory
sim-pli-fi-cation
da-tive
pro-nom-in-al
or-i-gin
pre-po-si-tion
in-flec-tion-al
cat-e-gor-ic-al
in-di-cates
poss-ess-ive
de-termin-ation
rep-re-sent
e-lici-tation
typo-logy
ac-tiv-i-ty
ac-tiv-i-ties
fellows
in-ter-pre-tation
lin-guists
cycle
ne-ga-tion
sub-con-tinent
va-len-cy
ca-te-gor-ies
com-par-ison
gram-mat-i-cal
cor-res-pond-ing
ob-ser-vations
as-pec-tual
in-di-cation
ser-vice
iden-ti-fy-ing
utter-ances
no-ta-bly
for-mation
dis-tin-guish
mutand-is
muta-tis
in-di-vid-uals
lingua
natur-al
sub-sequent
re-cur-rence
rel-ative
rel-atives
mil-i-tary
ex-po-sure
spe-cif-ic-al-ly
rep-li-cate
rep-li-ca-ting
mod-i-fi-cations
gen-der
ṣəḥəttux
dra-ma-tic-al-ly
ac-cen-tuated
his-tor-i-cal
his-tor-i-cal-ly
event-ual-ly
pres-tige
dis-ap-pear
Romance
pen-insula
pen-insular
fea-ture
con-stitute
either
period
korufu
poly-sem-ous
Zammit
namrag
earli-er
pis-ellum
qillīd
Ta-rif-it
Reinhardt
situ-ation
Mohand
Zazaki
form-ación
Científicas
Roman-ica
butter-fly
roman-cismos
espec-ial-mente
Alexandrine
Ziamari
oriente
SELAF
cultur-elles
maroc-ain
medi-eval
medi-terranean
ɣesmat
Dordrecht
Trieste
}
\hyphenation{
Engels-zungen
affri-ca-te
affri-ca-tes
Bongor
Lotuho
Sprach-inseln
under-going
monkey
fortiori
kapaparát
báfura
imbíró
Bangala
Bettega
Bab-ylon-ian
Kouwenberg
Ammani
Amman
Ammanis
scen-arios
elabor-ate
elabor-ated
func-tion-al-ization
anthro-polo-gist
Iemmolo
table
Martin
gahawa
Tihāma
Sason
com-ple-ments
homo-phon-ous
Meso-potam-ian
Diyar-bakır
Tsiapera
Lucas
edu-ca-tion
cubit-ellu
corbata
morpho-phonology
morpho-phono-logic-al-ly
Volubilis
Mennen
Papa-constan-tinou
Vicente
re-inter-pre-ta-tion
megorashim
Malayalam
morpho-phon-em-ic-s
morpho-syn-tax
morpho-syn-tactic
Ki-nubi
ḥawār-iyy-ūn
archa-ic-ity
de-velop-ment
pro-so-dic
in-duced
pho-no-logy
possess-um
pos-ses-s-ive-s
Birgitt
es-pe-cial-ly
clause-s
po-ten-tial-ly
Lusignan
Cush-itic
re-fi-ni-ti-za-tion
lan-guage-s
lan-guage
dic-tion-ary
dic-tion-aries
socio-political
eight
counter-part
de-on-tic
Walter
Mifsud
ana-lys-is
re-ana-lys-is
re-ana-lyse
re-ana-lysed
re-ana-lyses
re-ana-lys-ing
Catalan
Occitan
Vigario
Sónia
ḥammāl
cross-linguist-ic-al-ly
south-western
bor-row-ed
ǧumruk
neg-ative
de-termin-er
Oxford
Socotra
Khuze-stan
Arama-ic
Otto-man
Otto-mans
Azeri
bi-lin-gual-ism
bor-row-ing
dia-lects
dia-lect
Ahwāzī
Catherine
allo-phone
allo-phones
phon-em-ic
hannaʔa
tubbaʕ
len-ition
Khorramshahr
poss-ible
Moroccan
Martine
Harrassowitz
gram-matica-lization
Wolfram
Afro-asiatica
Afro-asiatic
Semit-istik
mono-phthong
mono-phthong-ized
mono-phthong-izing
Heselwood
Hinrichs
Watson
Aramaeo-grams
Dadan-it-ic
sub-ordin-ator
Thamud-ic
Revue
Bovingdon
un-der
attach
attached
bundle
Arabia
graph-em-ic
graph-emes
ci-ca-da
Corriente
Andalusi
Kossmann
mor-pho-lo-gi-cal
dia-chroniques
islámica
occid-ent-al-ismo
idiom-a-ti-city
dia-lecto-logie
Reichert
co-lo-nial
Milton
diph-thong-al
lin-guis-tic
lin-guis-tique
lin-guis-tics
affairs
pho-net-ic-ally
kilo-metres
meta-typy
Jagiellonian
Arcodia
Hussein
Christopher
Giannakidou
Tashelhiyt
sta-bi-li-zation
de-vel-op-ments
in-vest-i-gation
Jor-dan-ian
af-fri-ca-tion
de-affri-ca-tion
Jordan
inter-dent-al
inter-locutors
socio-historical
roman-ization
Kurmanji
pha-ryn-geal-ization
con-sonant
con-sonants
phi-lo-logy
Öpengin
Kurd-ish
sunbul
sekin-în
fatala
dia-lect-al
Matras
mor-phol-o-gy
ḥaddād
emer-gence
Herzog
Benjmamins
struc-ture
Orient-studien
Ana-tolia
Ana-tolian
American
vari-ation
Jastrow
Geoffrey
Yarshater
Ḥarsūsi
Umberto
Ashtiany
Edmund
Mahnaz
En-cyclo-pædia
En-cyclo-paedia
En-cyclo-pedia
Leiden
dia-spora
Sasan-ian
every-day
domin-ance
Con-stitu-tion-al
Manfredi
pro-duct-iv-ity
Morocco
ferman
Maghreb-ian
algérien
Nicolaï
Mouton
maur-i-tani-en
Gotho-burg-ensis
socio-linguist-ique
socio-linguistic
dia-chronica
dihāt
de-velop-ed
ṣuḥbat
kitāba
kitābat
com-mercial
Senegal
mechan-ics
Maur-itan-ia
Ḥassān-iyya
circum-cision
cor-relation
labio-velar-ization
vowel
vowels
īggīw
series
in-tegrates
dur-ative
inter-dent-als
gen-itive
Tuareg
tălămut
talawmāyət
con-diment
Wiesbaden
Uni-vers-idad
Geuthner
typo-logie
Maur-itanie
nomades
Maur-itan-ian
dia-lecto-logy
dia-lecto-logical
Sahar-iennes
Uni-ver-sity
de-scend-ants
NENA-speak-ing
speak-ing
xaddaːm
Kahane
Tinniswood
Ven-etian
rabia
Cohen
dubi-ous
Al-geria
trans-humance
Marrakech
Bender
Munich
origin-al
re-captured
ethnic
minor-it-ies
drama-tic
local
long-stand-ing
regions
Nineveh
misti-linguismo
settle-ments
Ṣəndor
Mandate
sub-sti-tu-ting
or-thog-ra-phy
ref-er-enced
origin-ate
twenti-eth
typ-ic-al-ly
Hobrack
never-the-less
masc-uline
coffee
ex-clu-sive-ly
ver-bal
sim-i-lar-i-ties
der-i-va-tion
im-pera-tive
par-ti-ci-ple
dis-am-big-uation
dis-am-big-uating
phen-omen-on
phen-omen-a
traktar
Coetsem
mVCCūC
kān-at
Ing-ham
uni-versidad
Kerman-shah
Brin-cat
Tarifiyt
Vanhove
pre-verbal
Morris
Soqoṭri
dis-pre-fer-red
ex-pla-nation
con-struc-tion
Behnstedt
Andersen
wide-spread
usu-al-ly
re-gion-al
Bulut
con-sider-ation
afro-asia-tici
Franco-Angeli
Phono-logie
Volks-kundliche
dia-lectes
dia-lecte
dis-appear-ance
media
under-stand-able
pub-li-cation
second-ary
e-ject-ive
rev-o-lu-tion
re-stric-tive
Gasparini
moun-tain
moun-tains
yellow
label-ing
tra-di-tion-al-ly
cur-rent-ly
dia-chronic
Hebron
sub-se-quent-ly
doc-u-men-tation
Dom-ari
inter-actions
po-ten-tial
in-ven-tory
sim-pli-fi-cation
da-tive
pro-nom-in-al
or-i-gin
pre-po-si-tion
in-flec-tion-al
cat-e-gor-ic-al
in-di-cates
poss-ess-ive
de-termin-ation
rep-re-sent
e-lici-tation
typo-logy
ac-tiv-i-ty
ac-tiv-i-ties
fellows
in-ter-pre-tation
lin-guists
cycle
ne-ga-tion
sub-con-tinent
va-len-cy
ca-te-gor-ies
com-par-ison
gram-mat-i-cal
cor-res-pond-ing
ob-ser-vations
as-pec-tual
in-di-cation
ser-vice
iden-ti-fy-ing
utter-ances
no-ta-bly
for-mation
dis-tin-guish
mutand-is
muta-tis
in-di-vid-uals
lingua
natur-al
sub-sequent
re-cur-rence
rel-ative
rel-atives
mil-i-tary
ex-po-sure
spe-cif-ic-al-ly
rep-li-cate
rep-li-ca-ting
mod-i-fi-cations
gen-der
ṣəḥəttux
dra-ma-tic-al-ly
ac-cen-tuated
his-tor-i-cal
his-tor-i-cal-ly
event-ual-ly
pres-tige
dis-ap-pear
Romance
pen-insula
pen-insular
fea-ture
con-stitute
either
period
korufu
poly-sem-ous
Zammit
namrag
earli-er
pis-ellum
qillīd
Ta-rif-it
Reinhardt
situ-ation
Mohand
Zazaki
form-ación
Científicas
Roman-ica
butter-fly
roman-cismos
espec-ial-mente
Alexandrine
Ziamari
oriente
SELAF
cultur-elles
maroc-ain
medi-eval
medi-terranean
ɣesmat
Dordrecht
Trieste
} 
  \togglepaper[1]%%chapternumber
}{}

\begin{document}
\maketitle 
  


 \section{Current state and contexts of use}


 \subsection{Introduction}


Berber, or Tamazight, is the indigenous language family of northwestern Africa, distributed discontinuously across an area ranging from western Egypt to the Atlantic, and from the Mediterranean to the Sahel. Its range has been expanding in the Sahel within recent times, as Tuareg speakers move southwards, but in the rest of this area, Berber has been present since before the classical period (\citealt{MúrciaSánchez2010}). Its current discontinuous distribution is largely the result of language shift to Arabic over the past millennium.

At present, the largest concentrations of Berber speakers are found in the highlands of Morocco (Tashelhiyt, Tamazight, Tarifiyt) and northeastern Algeria (Kabyle, Chaoui). Tuareg, in the central Sahara and Sahel, is more diffusely spread over a large but relatively sparsely populated zone. Across the rest of this vast area, Berber varieties constitute small islands – in several cases, single towns – in a sea of Arabic.

This simplistic map, however, necessarily leaves out the effects of mobility – not limited to the traditional practice of nomadism in the Sahara and transhumance in parts of the Atlas mountains. The rapid urbanisation of North Africa over the past century has brought large numbers of Berber speakers into traditionally Arabic-speaking towns, occasionally even changing the town's dominant language. The conquests of the early colonial period created small Berber-speaking refugee communities in the Levant and Chad, while more recent emigration has led to the emergence of urban Berber communities in western Europe and even Quebec.


 
 \subsection{Sociolinguistic situation of Berber}


In North Africa proper, the key context for the maintenance of Berber is the village. Informal norms requiring the use of Berber with one's relatives and fellow villagers, or within the village council, encourage its maintenance not only there but in cities as well, depending on the strength of emigrants' (often multigenerational) ties to their hometowns. In some areas, such as Igli in Algeria \citep{Mouili2013}, the introduction of mass education in Arabic has disrupted these norms, encouraging parents to speak to their children in Arabic to improve their educational chances; in others, such as Siwa in Egypt \citep{Serreli2017}, it has had far less impact. Beyond the village, in wider rural contexts such as markets, communication is either in Berber or in Arabic, depending on the region; where it is in Arabic, it creates a strong incentive for bilingualism independent of the state's influence.  For centuries, Berber-speaking villages in largely Arabic-speaking areas have sporadically been shifting to Arabic, as in the Blida region of Algeria (\citealt{ElArifi2014}); the opposite is also more rarely attested, as near Tizi-Ouzou in Algeria \citep[258]{Gautier1913}.

In urban contexts, on the other hand, norms enforcing Berber have no public presence – quite the contrary.  There one addresses a stranger in Arabic, or sometimes French, but rarely in Berber, except perhaps in a few Berber-majority cities such as Tizi{}-Ouzou \citep{Tigziri2008}.  Even within the family, Arabic takes on increasing importance; in a study of Kabyle Berbers living in Oran (Algeria), Ait Habbouche (\citeyear[79]{AitHabbouche2013}) found that 54\% said they mostly spoke Arabic to their siblings, and 10\% even with their grandparents. In the Sahel, Arabic is out of the picture, but there too family language choice is affected; 13\% of the Berber speakers interviewed by Jolivet (\citeyear[146]{Jolivet2008}) in Niamey (Niger) reported speaking no Tamasheq at all with their families, using Hausa or, less frequently, Zarma instead.

Bilingualism is widespread but strongly asymmetrical. Almost all Berber speakers learn dialectal Arabic (as well as standard Arabic, taught at school), whereas Arabic speakers almost never learn Berber.  There are exceptions: in some contexts, Arabic-speaking women who marry Berber-speaking men need to learn Berber to speak with their in-laws (the author has witnessed several Kabyle examples), while Arabic speakers who settle in a strongly Berber-speaking town – and their children – sometimes end up learning Berber, as in Siwa (Egypt). Nevertheless, most Arabic speakers place little value on the language, and some openly denigrate it; in Bechar (Algeria), anyone expressing interest in Berber can expect frequently to hear the contemptuous saying \textit{əš-šəlḥa} \textit{ma-hu} \textit{klam} \textit{wə-d-dhən} \textit{ma} \textit{hu} \textit{l-idam} `Shilha (Berber) is no more speech than vegetable oil is animal fat'. To further complicate the situation, French remains an essential career skill (except in Libya and Egypt), since it is still the working language of many ministries and companies; in some middle-class families, it is the main home language spoken with children.

On paper, Berber (Tamazight) is now an official language of Morocco (since 2011) and Algeria (since 2016), while Tuareg (Tamasheq/Tamajeq) is a recognised national language of Mali and Niger. In practice, “official language” remains a misleading term.  Official documents are rarely, if ever, provided in Berber, and there is no generalized right to communicate with the government in Berber.  However, Berber is taught as a school subject in selected Algerian, Moroccan, and (since 2012 or so) Libyan schools, while some Malian and Nigerien ones even use it as a medium of education. It is also used in broadcast media, including some TV and radio channels. Both Morocco and Algeria have established language planning bodies to promote neologisms and encourage publishing, with a view towards standardisation. The latter poses difficult problems, given that each country includes major varieties which are not inherently mutually intelligible.

Berber varieties have been written since before the second century BC \citep{Pichler2007} – although the language of the earliest inscriptions is substantially different from modern Berber and decipherable only to a limited extent – and southern Morocco has left a substantial corpus of pre-colonial manuscripts \citep{Boogert1997}; many other examples could be cited from long before people such as \citet{Mammeri1976} attempted to make Berber a printed language. Nevertheless, writing seems to have had very little impact on the development of Berber as yet. Awareness of the existence of a Berber writing system – Tifinagh – is widespread, and often a matter of pride. However, most Berber speakers have never studied Berber, and do not habitually read or write in it in any script – with the increasingly important exception of social media and text messages, typically in Latin or Arabic script depending on the region. Efforts to create a standard literary Berber language have not so far been successful enough to exert a unifying influence on its dispersed varieties. In the North African context, this is often understood as implying that Berber is not a language at all – “language” (Arabic \textit{luɣa}) being popularly understood in the region as “standardized written language”.


 
 \subsection{Demographic situation of Berber}


No reliable recent estimate of the number of Berber speakers exists; relevant data is both scarce and hotly contested. The estimates brought together by Kossmann (\citeyear[1]{Kossmann2011}; \citeyear[29--36]{Kossmann2013book}) suggest a range of 30–40\% for Morocco, 20–30\% for Algeria, 8\% for Niger, 7\% for Mali, about 5\% for Libya, and less than 1\% for Tunisia, Egypt, and Mauritania.  Selecting the midpoint of each range, and substituting in the mid-2017 populations of each of these countries (\citealt{CIA2017}) would yield a total speaker population of about 25 million, 22 million of them divided almost evenly between Morocco and Algeria.


 \section{Contact languages and historical development}


 \subsection{Across North Africa} \label{na}


Berber contact with Arabic began in the seventh century with the Islamic conquests. For several centuries, language shift seems to have been largely confined to major cities and their immediate surroundings, probably affecting Latin speakers more than Berber speakers. The invasion of the Banū Hilāl and Banū Sulaym in the mid-eleventh century is generally identified as the key turning point: it made Arabic a language of pastoralism, rapidly reshaping the linguistic landscape of Libya and southern Tunisia, then over the following centuries slowly transforming the High Plateau and the northern Sahara in general.  This rural expansion further reinforced the role of Arabic as a lingua franca, while the recruitment of Arabic-speaking soldiers from pastoralist tribes encouraged its spread further west to the Moroccan Gharb.

The resulting linguistic divide between rural groups and towns remained a key theme of Maghrebi sociolinguistics until the twentieth century. In several cases, a town spoke a different language than its hinterland; in much of the Sahara, Berber-speaking oasis towns such as Ouargla or Igli formed linguistic islands in regions otherwise populated by Arabic speakers, and in the north, towns such as Bejaia or Cherchell constituted small Arabic-speaking communities surrounded by a sea of Berber-speaking villages. Even in larger cities such as Algiers or Marrakech, the dominance of Arabic was counterbalanced by substantial regular immigration from Berber-speaking regions further afield.

Today all Berber communities are more or less multilingual, usually in Arabic and often also in French; outside of the most remote areas, monolingual speakers are quite difficult to find. Even in the nineteenth century, however, monolingual Berber speakers were considerably more numerous \citep[41]{Kossmann2013book}.

Alongside the coexistence of colloquial Maghrebi Arabic with Berber, Classical Arabic also had a role to play as the primary language of learning and in particular religious studies.  Major Berber-speaking areas such as Kabylie (northern Algeria) and the Souss (southern Morocco) developed extensive systems of religious education, whose curricula consisted primarily of Arabic books (\citealt{Boogert1997}; \citealt{Mechehed2007}). The restriction of Classical Arabic to a limited range of contexts, and the relatively small proportion of the population pursuing higher education, gave it a comparatively small role in the contact situation; even in the lexicon, its influence is massively outweighed by that of colloquial Arabic, and it appears to have had no structural influence at all.


 
 \subsection{In Siwa}


Examples of contact-induced change in this chapter are often drawn from Siwi, the Berber language of the oasis of Siwa in western Egypt. Sporadic long-distance contact with Arabic there presumably began in the seventh or eighth century with the Islamic conquests, and increased gradually as Cyrenaica and Lower Egypt became Arabic-speaking and as the trade routes linking Egypt to West Africa were re-established.  During the eleventh century, the Banū Sulaym, speaking a Bedouin Arabic dialect, established themselves throughout Cyrenaica.

In the twelfth century, al-Idrīsī\ia{al-Idrīsī, Muḥammad@al-Idrīsī, Muḥammad} reports Arab settlement within Siwa itself, alongside the Berber population. Later geographers make no mention of an Arab community there, suggesting that these early immigrants were integrated into the Berber majority. Several core Arabic loans in Siwi, such as the negative copula \textit{qačči} < \textit{qaṭṭ} \textit{šayʔ} and the noon prayer \textit{luli} < \textit{al-ʔūlē}, are totally absent from surrounding Arabic varieties today; such archaisms are likely to represent founder effects dating back to this period \citep{Souag2009}.  

The available data gives nothing close to an adequate picture of the linguistic environment of medieval Siwa. We may assume that, throughout these centuries, most Siwis – or at least the dominant families – would have spoken Berber as their first language, and more mobile ones – especially traders – would have learned Arabic (but whose Arabic?) as a second language. Alongside these, however, we must envision a fluctuating population of Arabic-speaking immigrants and West African slaves learning Berber as a second language. In such a situation, both Berber-dominant and Arabic-dominant speakers should be expected to play a part in bringing Arabic influences into Siwi.

The oasis was integrated into the Egyptian state by Muhammad Ali in 1820, but large-scale state intervention in the linguistic environment of the oasis only took effect in the twentieth century; the first government school was built in 1928, and television was introduced in the 1980s. An equally important development during this period was the rise of labour migration, taking off in the 1960s as Siwi landowners recruited Upper Egyptian labourers, and Siwi young men found jobs in Libya's booming oil economy. It has then grown further since the 1980s with the rise of tourism and the growth of tertiary education. The effects of this integration into a national economy include a conspicuous generation gap in local second-language Arabic: older and less educated men speak a Bedouin-like dialect with *q > \textit{g}, while younger and more educated ones speak a close approximation of Cairene Arabic.


 \section{Contact-induced changes in Berber}


 \subsection{Introduction}


As noted above, bilingualism in North Africa has been asymmetrical for many centuries, with Berbers much more likely to learn Arabic than vice versa. This suggests the plausible general assumption that the agents of contact-induced change were typically dominant in the (Berber) recipient language rather than in Arabic.  However, closer examination of individual cases often reveals a less clear-cut situation; as seen above in §\ref{na}, the history of Siwi suggests that Berber- and Arabic-dominant speakers both had a role to play, and \textit{post} \textit{facto} analysis of the language's structure seems to confirm this assumption.  The loss of feminine plural agreement, for example (§\ref{morph} below), can more easily be attributed to Arabic-dominant speakers adopting Berber than to Berber-dominant speakers.  In the absence of clear documentary evidence, caution is therefore called for in the application of Van Coetsem's (\citeyear{VanCoetsem1988,VanCoetsem2000}) model to Berber.


 
 \subsection{Phonology}


The influence of Arabic on Berber phonology is conspicuous; in general, every phoneme used in a given region's dialectal Arabic is found in nearby Berber varieties. Almost all Northern Berber varieties have adopted from Arabic at least the pharyngeals /ʕ/ and /ḥ/, a series of voiceless emphatics: /ṣ/, /ḫ/, non-geminate /q/, and either /ḍ/ or /ṭ/. These phonemes presumably reached Berber through loanwords from Arabic, but have been extended to inherited vocabulary as well, through reinterpretation of emphatic spread or through their use in “expressive formations” \citep[199]{Kossmann2013book}, e.g. Kabyle \textit{θi-ḥəðmər-θ} `breast of a small animal' < \textit{iðmar-ən} `breast'.

In Siwi (\citealt{Souag2013book}: 36–39; \citealt{SouagvanPutten2016}), at least nine phonemes were clearly introduced from Arabic.  The pharyngealised coronals /ṣ/, /ḷ/, /\R/ and /ḍ/ have no regular source in Berber, and occur in inherited vocabulary almost exclusively as a result of secondary emphasis spread (with the isolated exception of \textit{ḍəs} `to laugh').  The order of borrowing appears to be \textit{ḷ,} \textit{ṛ} > \textit{ṣ} > \textit{ḍ}; in a few older loans, Arabic \textit{ṣ} is borrowed as \textit{ẓ} (e.g. \textit{ẓəffaṛ} `to whistle' < \textit{ṣaffar}), and in all but the most recent strata of loans, Arabic \textit{ḍ/ð̣} is borrowed as \textit{ṭ} (e.g. \textit{a-ʕṛiṭ} `broad' < \textit{ʕarīḍ}).  The pharyngeals /ḥ/ and /ʕ/ (e.g. \textit{ḥəbba} `a little' < \textit{ḥabba} `a grain', \textit{ʕammi} `paternal uncle' < \textit{ʕamm-ī} `uncle-\textsc{obl.1sg}') likewise have no regular source in Berber, although 1\textsc{sg} -\textit{ɣ}{}- has become \textit{{}-ʕ}{}- for some speakers (an irregular sound change specific to this morpheme). \textit{ʕ} is lost in a number of older loans (e.g. \textit{annaš} `bier' < \textit{an-naʕš}), but \textit{ḥ} is always retained as such rather than being dropped or adapted (unlike Tuareg, where it is typically adapted to \textit{ḫ}).  This suggests that Siwi continued to adapt Arabic loans to its phonology by dropping \textit{ʕ} up to some stage well after the beginning of significant borrowing from Arabic, but started accepting Arabic loans with \textit{ḥ} too early for any adapted to survive, implying an order of borrowing \textit{ḥ} > \textit{ʕ}. Among the glottals, /h/ (e.g. \textit{ddhan} `oil' < \textit{dihān} `oils') appears in inherited vocabulary only in the distal demonstratives, where comparison to Berber languages that do have \textit{h} suggests that it is excrescent, while /ʔ/ only rarely appears even in recent loanwords (e.g. \textit{ʔəǧǧəṛ} `to rent' < \textit{ʔaǧǧar}). The mid vowel /o/ has been integrated into Siwi phonology as a result of borrowing from Arabic; having been established as a phoneme, however, it went on to emerge by irregular change from original \textit{*u} in two inherited words (\textit{allon} `window', \textit{agṛoẓ} `palm heart'), and from irregular simplification of \textit{*aɣu} in some demonstratives (e.g. \textit{wok} `this\textsc{.sg.m}' < *wa ɣuṛ-ək `this.\textsc{sg.m} at-\textsc{2sg.m')}. The interdentals /θ/ and /ð̣/ have a more marginal status, but are used by some speakers even in morphologically well-integrated loans, e.g. \textit{a-θqil} or \textit{a-tqil} `heavy' < \textit{θaqīl}. Arabic influence may also be responsible for the treatment of [ʒ] and [dʒ] as free variants of the same phoneme /ǧ/ \citep{Vycichl2005}, so that e.g. /taǧlaṣt/ `spider' is variously realized as [tʰæʒlˤɑsˤt] {\textasciitilde} [tʰædʒlˤɑsˤt] (\citealt{Naumann2012}: 152); other Berber languages with phonemic \textit{ž} normally have [dʒ] as a conditioned allophone (e.g. when geminated) or as a cluster.

Arabic influence has also massively affected the frequency of some phonemes. /q/ and /ḫ/ were marginal in Siwi before Arabic influence, while *e had nearly disappeared due to regular sound changes, but all three are now quite frequent. Conversely, the influx of Arabic loans has helped make labiovelarised phonemes such as \textit{gʷ} and \textit{qʷ} rare.


 
 \subsection{Morphology} \label{morph}


Berber offers numerous examples of the borrowing of Arabic words together with their original Arabic inflectional morphology, a case of what \citet{Kossmann2010} calls Parallel System Borrowing. This phenomenon is most prominent for nominal number marking, but sometimes attested in other contexts too.

In Berber, most nouns are consistently preceded by a prefix marking gender (masculine/feminine), number (singular/plural), and often case/state. Nouns borrowed from Arabic normally either get assigned a Berber prefix, or fill the prefix slot with an invariant reflex of the Arabic definite article: compare Figuig \textit{a-gʕud} vs. Siwi \textit{lə-gʕud} `young camel' (< \textit{qaʕūd}). The Berber plural marking system prior to Arabic influence was already rather complex, combining several different types of affixal marking with internal ablaut strategies; many Arabic loans are integrated into this system, e.g. Kabyle \textit{a-bellar} `crystal' > pl. \textit{i-bellar-en} (< \textit{billawr}), Siwi \textit{a-kəddab} `liar' > pl. \textit{i-kəddab-ən} (< \textit{kaððāb}). However, in most Berber varieties, Arabic loans have further complicated the system by frequently retaining their original plurals, e.g. Kabyle \textit{l-kaɣeḍ} `paper' > \textit{le-kwaɣeḍ} (< \textit{kāɣid}), Siwi \textit{əl-gənfud} `hedgehog' > pl. \textit{lə-gnafid} (< \textit{qunfuð}). (The difference correlates fairly well with the choice in the singular between a Berber prefix and an Arabic article, but not perfectly; contrast e.g. Siwi \textit{a-fruḫ} `chick, bastard' < \textit{farḫ}, which takes the Arabic-style plural \textit{lə-fraḫ}.)  Berber has no inherited system of dual marking, instead using analytic strategies. Nevertheless, for a limited number of measure words, duals too are borrowed, e.g. Kabyle \textit{yum-ayen} `two days' < \textit{yawm-ayn} (although `day' remains \textit{ass}!), Siwi \textit{s-sən-t} `year' > \textit{sən-t-en} `two years' < \textit{san-at-ayn}.  Arabic number morphology may sporadically spread to inherited terms as well, e.g. Kabyle \textit{berdayen} `twice' < \textit{a-brid} `road, time', Siwi \textit{lə-gʷrazən} `dogs' < \textit{a-gʷərzni} `dog' \citep{Souag2013book}.

Whereas nouns are often borrowed together with their original inflectional morphology, verbs almost never are. The only attested exception is Ghomara, a heavily mixed variety of northern Morocco.  In Ghomara, many (but not all) verbs borrowed from Arabic are systematically conjugated in Arabic in otherwise monolingual utterances, a phenomenon which seems to have remained stable over at least a century: thus `I woke up' is consistently \textit{faq-aḫ}, but `I fished' is equally consistently \textit{ṣṣað-iθ} (\citealt{Mourigh2016}: 6, 137, 165). However, the borrowing of Arabic participles to express progressive aspect is also attested in Zuwara, if only for the two verbs of motion \textit{mašəy} `going' (pl. \textit{mašy-in}) and \textit{žay} `coming' (pl. \textit{žayy-in}), contrasting with inherited \textit{fəl} `go', \textit{asəd} `come' (\citealt{Kossmann2013book}: 284–285).

Prepositions are less frequently borrowed; in some cases where they are borrowed, however – including Igli \textit{mənɣir-} `except', Ghomara \textit{bin} `between' \citep[293]{Kossmann2013book} – they too occasionally retain Arabic pronominal markers, e.g. Siwi \textit{msabb-ha} `for her' < \textit{min} \textit{sababi-hā}  `from reason.\textsc{obl}{}-\textsc{obl.3sg.f}' \citep[48]{Souag2013book}.  In Awjila, more unusually, two inherited prepositions somewhat variably take Arabic pronominal markers, e.g. \textit{dit-ha} `in front of her' (\citealt{vanPutten2014}: 113).

A rarer but more spectacular example of morpheme borrowing is the borrowing of productive templates from Arabic. Such cases include the elative template əCCəC in Siwi, used to form the comparative degree of triliteral adjectives irrespective of etymology – thus \textit{əmləl} `whiter' < \textit{a-məllal} alongside \textit{əṭwəl} `taller' < \textit{a-ṭwil} \textit{<} Arabic \textit{ṭawīl} \citep{Souag2009} – and the diminutive template CCiCəC in Ghomara \citep{Mourigh2016}, e.g. \textit{aẓwiyyəṛ} `little root' < \textit{aẓaṛ} alongside \textit{ləmwiyyəs} `little knife' < \textit{l-mus} < Arabic \textit{al-mūsā} `razor' (gemination of \textit{y} is automatic in the environment i\_V). As the latter example illustrates, borrowed derivational morphology sometimes becomes productive.

The effects of Arabic on Berber morphology are by no means limited to the borrowing of morphemes. There is reason to suspect Arabic influence of having played a role in processes of simplification attested mainly in peripheral varieties, such as the loss of case marking in many areas. In Siwi, where Arabic influence appears on independent grounds to be unusually high, the verbal system shows a number of apparent simplifications targeting categories absent in sedentary Arabic varieties: the loss of distinct negative stems, the near-complete merger of perfective with aorist, the fixed postverbal position of object clitics, and so on. It is tempting to explain such losses as arising from imperfect acquisition of Siwi by Arabic speakers.

Structural calquing in morphology is also sporadically attested. Siwi has lost distinct feminine plural agreement on verbs, pronouns, and demonstratives, extending the inherited masculine plural forms to cover plural agreement irrespective of gender. Within Berber, this is unprecedented; plural gender agreement is extremely well conserved across the family. However, it perfectly replicates the usual sedentary Arabic system found in Egypt and far beyond.


 
 \subsection{Syntax}


Syntactic influence is often difficult to identify positively.  Nevertheless, Berber offers a number of examples, of which relative clause formation is one of the clearest (\citealt{Souag2013book}: 151–156; \citealt{Kossmann2013book}: 369–407). Relative clauses in Berber are normally handled with a gap strategy combined with fronting of any stranded prepositions, as in ‎(\ref{awjila}).

\ea
{Awjila \citep[79]{Paradisi1961}}\\ \label{awjila}
\gll ərrafəqa-nnəs wi ižin-an-a nettin id-sin ksum\\
     friend\textsc{.pl}{}-\textsc{gen.}\textsc{3sg} \textsc{rel.pl.m} divide-\textsc{3pl.m-prf} 3\textsc{sg.m} with-\textsc{obl.}3\textsc{pl}.\textsc{m} meat\\
\glt `his friends with whom he divided the meat'
\z

In subject relativisation, a special form of the verb not agreeing in person (the so-called “participle”) is used, as in (\ref{awj}); such a form is securely reconstructible for proto-Berber \citep{Kossmann2003}. 

\ea \label{awj}
{Awjila \citep[162]{Paradisi1960}}\\
\gll amədən wa tarəv-ən nettin ʕayyan\\
     man \textsc{rel.sg.m} write.\textsc{ipfv-ptcp} \textsc{3sg.m} ill\\
\glt `The man who is writing is ill.'
\z

In several smaller easterly varieties apart from Awjila, however, both of these traits have been lost. The strategy found in varieties such as Siwi – resumptive weak (affixal) pronouns throughout, and regular finite agreement for subject relativisation – perfectly parallels Arabic: 

\ea\label{ex:key:}
\langinfo{Siwi}{}{\citealt{Souag2013book}: 151–152}\\
\gll tálti tən dəzz-ɣ{}-as ǧǧəwab\\
woman \textsc{rel.sg.f} send-\textsc{1sg-dat.3sg} letter\\
\glt `the woman to whom I sent the letter' \\
     \z
     
\ea\label{ex:key:}
{ Siwi (field data)}\\
\gll ággʷid wənn i-ʕəṃṃaṛ iməǧran\\
man \textsc{rel.sg.m} \textsc{3sg.m}{}-make.\textsc{ipfv} sickle.\textsc{pl}\\
\glt `the man who makes sickles' 
\z

In the case of verbal negation, an originally syntactic calque has often been morphologised in parallel in Arabic and Berber. A number of varieties – especially the widespread Zenati subgroup of Berber, ranging from eastern Morocco to northern Libya – have developed a postverbal negative clitic \textit{š} \textit{/} \textit{ša} from \textit{*\'{k}ăra} `thing', apparently a calque on Arabic \textit{š} \textit{/} \textit{ši} from \textit{šayʔ}; however, some instead use the direct borrowings \textit{ši} or \textit{šay} (\citealt{Lucas2007}; \citealt{Kossmann2013book}: 332–334).


 
 \subsection{Lexicon}


Lexical borrowing from Arabic is pervasive in Berber. Out of 41 languages around the world compared in the Loanword Typology Project \citep{Tadmor2009}, Tarifiyt Berber was second only to (Selice) Romani in the percentage of loanwords – more than half (51.7\%) of the concepts compared. More than 90\% of loanwords examined in Tarifiyt were from Arabic, almost all from dialectal Maghrebi Arabic.  There is little reason to suppose that Tarifiyt is exceptional in this respect among Northern Berber languages; to the contrary, Kossmann (\citeyear[110]{Kossmann2013book}) finds its rate of basic vocabulary borrowing to be typical of Northern Berber, whereas Siwi and Ghomara go much higher. The rate of borrowing from Arabic, however, is considerably lower further south and west; on a 200-word list of basic vocabulary, Chaker (\citeyear{Chaker1984}: 225–226) finds 38\% Arabic loans in Kabyle (north-central Algeria) vs. 25\% in Tashelhiyt (southern Morocco) and only 5\% in Tahaggart Tuareg (southern Algeria).

This borrowing is pervasive across the languages concerned, rather than being restricted to particular domains. Every semantic field examined for Tarifiyt, including body parts, contained at least 20\% loanwords, and verbs or adjectives were about as frequently borrowed as nouns were \citep{Kossmann2009}. Numerals stand out for particularly massive borrowing; most Northern Berber varieties have borrowed all numerals from Arabic above a number ranging from `one' to `three' \citep{Souag2007}.

The effects of this borrowing on the structure of the lexicon remain insufficiently investigated, but appear prominent in such domains as kinship terminology. Throughout Northern Berber, a basic distinction between paternal kin and maternal kin is expressed primarily with Arabic loanwords (\textit{ʕammi} `paternal uncle' vs. \textit{ḫali} `maternal uncle' etc.), whereas in Tuareg that distinction is not strongly lexicalised. Nevertheless, borrowing does not automatically entail lexical restructuring; Tashelhiyt, for example, kept its vigesimal system even after borrowing the Arabic word for `twenty' (\textit{ʕšrin}), cf. Ameur (\citeyear{Ameur2008}: 77).

The borrowing of analysable multi-word phrases – above all, numerals followed by nouns – stands out as a rather common outcome of Berber contact with Arabic. Usually this is limited to the borrowing of numerals in combination with a limited set of measure words, such as `day'; thus in Siwi we find forms like \textit{sbaʕ-t} \textit{iyyam} `seven days' rather than the expected regular formation *\textit{səbʕa} \textit{n} \textit{nnhaṛ-at }\citep[114]{Souag2013book}. In Beni Snous (western Algeria), the phenomenon seems to have gone rather further: Destaing (\citeyear[212]{Destaing1907}) reports that numerals above `ten' systematically select for Arabic nouns.  \citet{SouagKherbache2016}, however, explain this as a codeswitching effect, rather than a true case of one language's grammar requiring shifts into another.

\section{Conclusion}

The influence of Arabic on Berber has come to be better understood over the past couple of decades, but much remains to be done.  Synchronically, Berber–Arabic codeswitching remains virtually unresearched; rare exceptions include \citet{Hamza2007} and \citet{Kossmann2012}. Sociolinguistic methods could help us better understand the gradual integration of new Arabic loanwords; the early efforts of \citet{Brahimi2000} have hardly been followed up on. Diachronically, it remains necessary to move beyond the mere identification of loanwords and contact effects towards a chronological ordering of different strata, an approach explored for some peripheral varieties by \citet{Souag2009} and \citet{vanPuttenBenkato2017}. While linguists are belatedly beginning to take advantage of earlier manuscript data to understand the history of Berber (\citealt{Boogert1997}; \citealt{Boogert1998}; \citealt{Brugnatelli2011}; \citealt{Meouak2015}), this data has not yet been used in any systematic way to help date the effects of contact at different periods. For many smaller varieties, especially in the Sahara, basic documentation and description are still necessary before the influence of Arabic can be explored. The unprecedented degree of Arabic influence revealed in Ghomara by recent work \citep{Mourigh2016}, extending to the borrowing of full verb paradigms, suggests that such descriptive work may yet yield dividends in the study of contact.

Despite all these gaps, the work done so far is more than sufficient to establish a general picture of Arabic influence on Berber. Throughout Northern Berber, Arabic influence on the lexicon is substantial and pervasive, bringing with it significant effects on phonology and morphology. Structural effects of Arabic on morphology, and Arabic influence on Berber syntax, are less conspicuous but nevertheless important, especially in smaller varieties such as Siwi. Looking at these results through Van Coetsem's framework, this suggests that RL-dominant speakers have had an especially dominant role in Arabic–Berber contact in larger varieties, whereas SL-dominant speakers' role is more visible in smaller varieties.  However, this \textit{a} \textit{priori} conclusion should be tested against directly attested historical data wherever possible.

\section*{Further reading}

The key reference for Arabic influence on Northern Berber is \citet{Kossmann2012}, frequently cited above; this covers all levels of influence including the lexicon, phonology, nominal and verbal morphology, borrowing of morphological categories, and syntax.

The most extensive in-depth study of Arabic influence on a specific Berber variety is \citet{Souag2013book}, effectively a contact-focused grammatical sketch of Siwi Berber.

\citet{Mourigh2016} is a thorough synchronic description of by far the most strongly Arabic-influenced Berber variety, Ghomara, giving a uniquely clear picture of just how far the process can go without resulting in language shift.

\section*{Acknowledgements}

The author thanks his consultants in Siwa, especially the late Sherif Bougdoura, for their help with studying Siwi.

\section*{Abbreviations}
\begin{tabularx}{.5\textwidth}{@{}lQ@{}}
\textsc{1, 2, 3} & 1st, 2nd, 3rd person \\
\textsc{dat} & dative \\
\textsc{f} & feminine \\
\textsc{gen} & genitive \\
\textsc{ipfv} & imperfective  \\
\textsc{m} & masculine \\
\end{tabularx}%
\begin{tabularx}{.5\textwidth}{@{}lQ@{}}
\textsc{obl} & oblique \\
\textsc{pl}/pl. & plural \\
\textsc{prf} & perfect (suffix conjugation) \\
\textsc{ptcp} & participle \\
\textsc{sg} & singular \\
\end{tabularx}%
 

\sloppy
\printbibliography[heading=subbibliography,notkeyword=this]
\end{document}