\documentclass[output=paper,modfonts,nonflat]{langsci/langscibook} 
\author{Ángeles Vicente \affiliation{University of Zaragoza}}
\title{Andalusi Arabic}
\abstract{This chapter covers an ancient contact language situation: Andalusi Arabic with another two languages – the Romance varieties spoken by the local population, and the Berber varieties brought by different Berber speakers arriving in al-Andalus during its existence. The situation of bilingualism whereby the Romance language was sociolinguistically dominant for most of the population over the course of several centuries resulted in numerous contact-induced changes in all areas of grammar. In addition, interaction between Arabic-speaking and Berber-speaking populations constituted a second locus of language contact with consequences for Andalusi Arabic.}

\IfFileExists{../localcommands.tex}{
  % add all extra packages you need to load to this file 
\usepackage{graphicx}
\usepackage{tabularx}
\usepackage{amsmath} 
\usepackage{multicol}
\usepackage{lipsum}
\usepackage[stable]{footmisc}
\usepackage{adforn}
%%%%%%%%%%%%%%%%%%%%%%%%%%%%%%%%%%%%%%%%%%%%%%%%%%%%
%%%                                              %%%
%%%           Examples                           %%%
%%%                                              %%%
%%%%%%%%%%%%%%%%%%%%%%%%%%%%%%%%%%%%%%%%%%%%%%%%%%%%
% remove the percentage signs in the following lines
% if your book makes use of linguistic examples
\usepackage{./langsci/styles/langsci-optional} 
\usepackage{./langsci/styles/langsci-lgr}
\usepackage{morewrites} 
%% if you want the source line of examples to be in italics, uncomment the following line
% \def\exfont{\it}

\usepackage{enumitem}
\newlist{furtherreading}{description}{1}
\setlist[furtherreading]{font=\normalfont,labelsep=\widthof{~},noitemsep,align=left,leftmargin=\parindent,labelindent=0pt,labelwidth=-\parindent}
\usepackage{phonetic}
\usepackage{chronosys,tabularx}
\usepackage{csquotes}
\usepackage[stable]{footmisc} 

\usepackage{langsci-bidi}
\usepackage{./langsci/styles/langsci-gb4e} 

  \makeatletter
\let\thetitle\@title
\let\theauthor\@author 
\makeatother



\newcommand{\togglepaper}[1][0]{ 
  \bibliography{../localbibliography}
  \papernote{\scriptsize\normalfont
    \theauthor.
    \thetitle. 
    To appear in: 
    Christopher Lucas and Stefano Manfredi (eds.),  
    Arabic and contact-induced language change
    Berlin: Language Science Press. [preliminary page numbering]
  }
  \pagenumbering{roman}
  \setcounter{chapter}{#1}
  \addtocounter{chapter}{-1}
}

\newfontfamily\Parsifont[Script=Arabic]{ScheherazadeRegOT_Jazm.ttf} 
\newcommand{\arabscript}[1]{\RL{\Parsifont #1}}
\newcommand{\textarabic}[1]{{\arabicfont #1}}

\newcommand{\textstylest}[1]{{\color{red}#1}}

\patchcmd{\mkbibindexname}{\ifdefvoid{#3}{}{\MakeCapital{#3}
}}{\ifdefvoid{#3}{}{#3 }}{}{\AtEndDocument{\typeout{mkbibindexname could
not be patched.}}}

\newcommand{\R}{r\kern-.05ex{̣}\kern.05ex}

 
  %% hyphenation points for line breaks
%% Normally, automatic hyphenation in LaTeX is very good
%% If a word is mis-hyphenated, add it to this file
%%
%% add information to TeX file before \begin{document} with:
%% %% hyphenation points for line breaks
%% Normally, automatic hyphenation in LaTeX is very good
%% If a word is mis-hyphenated, add it to this file
%%
%% add information to TeX file before \begin{document} with:
%% %% hyphenation points for line breaks
%% Normally, automatic hyphenation in LaTeX is very good
%% If a word is mis-hyphenated, add it to this file
%%
%% add information to TeX file before \begin{document} with:
%% \include{localhyphenation}
\hyphenation{
Engels-zungen
affri-ca-te
affri-ca-tes
Bongor
Lotuho
Sprach-inseln
under-going
monkey
fortiori
kapaparát
báfura
imbíró
Bangala
Bettega
Bab-ylon-ian
Kouwenberg
Ammani
Amman
Ammanis
scen-arios
elabor-ate
elabor-ated
func-tion-al-ization
anthro-polo-gist
Iemmolo
table
Martin
gahawa
Tihāma
Sason
com-ple-ments
homo-phon-ous
Meso-potam-ian
Diyar-bakır
Tsiapera
Lucas
edu-ca-tion
cubit-ellu
corbata
morpho-phonology
morpho-phono-logic-al-ly
Volubilis
Mennen
Papa-constan-tinou
Vicente
re-inter-pre-ta-tion
megorashim
Malayalam
morpho-phon-em-ic-s
morpho-syn-tax
morpho-syn-tactic
Ki-nubi
ḥawār-iyy-ūn
archa-ic-ity
de-velop-ment
pro-so-dic
in-duced
pho-no-logy
possess-um
pos-ses-s-ive-s
Birgitt
es-pe-cial-ly
clause-s
po-ten-tial-ly
Lusignan
Cush-itic
re-fi-ni-ti-za-tion
lan-guage-s
lan-guage
dic-tion-ary
dic-tion-aries
socio-political
eight
counter-part
de-on-tic
Walter
Mifsud
ana-lys-is
re-ana-lys-is
re-ana-lyse
re-ana-lysed
re-ana-lyses
re-ana-lys-ing
Catalan
Occitan
Vigario
Sónia
ḥammāl
cross-linguist-ic-al-ly
south-western
bor-row-ed
ǧumruk
neg-ative
de-termin-er
Oxford
Socotra
Khuze-stan
Arama-ic
Otto-man
Otto-mans
Azeri
bi-lin-gual-ism
bor-row-ing
dia-lects
dia-lect
Ahwāzī
Catherine
allo-phone
allo-phones
phon-em-ic
hannaʔa
tubbaʕ
len-ition
Khorramshahr
poss-ible
Moroccan
Martine
Harrassowitz
gram-matica-lization
Wolfram
Afro-asiatica
Afro-asiatic
Semit-istik
mono-phthong
mono-phthong-ized
mono-phthong-izing
Heselwood
Hinrichs
Watson
Aramaeo-grams
Dadan-it-ic
sub-ordin-ator
Thamud-ic
Revue
Bovingdon
un-der
attach
attached
bundle
Arabia
graph-em-ic
graph-emes
ci-ca-da
Corriente
Andalusi
Kossmann
mor-pho-lo-gi-cal
dia-chroniques
islámica
occid-ent-al-ismo
idiom-a-ti-city
dia-lecto-logie
Reichert
co-lo-nial
Milton
diph-thong-al
lin-guis-tic
lin-guis-tique
lin-guis-tics
affairs
pho-net-ic-ally
kilo-metres
meta-typy
Jagiellonian
Arcodia
Hussein
Christopher
Giannakidou
Tashelhiyt
sta-bi-li-zation
de-vel-op-ments
in-vest-i-gation
Jor-dan-ian
af-fri-ca-tion
de-affri-ca-tion
Jordan
inter-dent-al
inter-locutors
socio-historical
roman-ization
Kurmanji
pha-ryn-geal-ization
con-sonant
con-sonants
phi-lo-logy
Öpengin
Kurd-ish
sunbul
sekin-în
fatala
dia-lect-al
Matras
mor-phol-o-gy
ḥaddād
emer-gence
Herzog
Benjmamins
struc-ture
Orient-studien
Ana-tolia
Ana-tolian
American
vari-ation
Jastrow
Geoffrey
Yarshater
Ḥarsūsi
Umberto
Ashtiany
Edmund
Mahnaz
En-cyclo-pædia
En-cyclo-paedia
En-cyclo-pedia
Leiden
dia-spora
Sasan-ian
every-day
domin-ance
Con-stitu-tion-al
Manfredi
pro-duct-iv-ity
Morocco
ferman
Maghreb-ian
algérien
Nicolaï
Mouton
maur-i-tani-en
Gotho-burg-ensis
socio-linguist-ique
socio-linguistic
dia-chronica
dihāt
de-velop-ed
ṣuḥbat
kitāba
kitābat
com-mercial
Senegal
mechan-ics
Maur-itan-ia
Ḥassān-iyya
circum-cision
cor-relation
labio-velar-ization
vowel
vowels
īggīw
series
in-tegrates
dur-ative
inter-dent-als
gen-itive
Tuareg
tălămut
talawmāyət
con-diment
Wiesbaden
Uni-vers-idad
Geuthner
typo-logie
Maur-itanie
nomades
Maur-itan-ian
dia-lecto-logy
dia-lecto-logical
Sahar-iennes
Uni-ver-sity
de-scend-ants
NENA-speak-ing
speak-ing
xaddaːm
Kahane
Tinniswood
Ven-etian
rabia
Cohen
dubi-ous
Al-geria
trans-humance
Marrakech
Bender
Munich
origin-al
re-captured
ethnic
minor-it-ies
drama-tic
local
long-stand-ing
regions
Nineveh
misti-linguismo
settle-ments
Ṣəndor
Mandate
sub-sti-tu-ting
or-thog-ra-phy
ref-er-enced
origin-ate
twenti-eth
typ-ic-al-ly
Hobrack
never-the-less
masc-uline
coffee
ex-clu-sive-ly
ver-bal
sim-i-lar-i-ties
der-i-va-tion
im-pera-tive
par-ti-ci-ple
dis-am-big-uation
dis-am-big-uating
phen-omen-on
phen-omen-a
traktar
Coetsem
mVCCūC
kān-at
Ing-ham
uni-versidad
Kerman-shah
Brin-cat
Tarifiyt
Vanhove
pre-verbal
Morris
Soqoṭri
dis-pre-fer-red
ex-pla-nation
con-struc-tion
Behnstedt
Andersen
wide-spread
usu-al-ly
re-gion-al
Bulut
con-sider-ation
afro-asia-tici
Franco-Angeli
Phono-logie
Volks-kundliche
dia-lectes
dia-lecte
dis-appear-ance
media
under-stand-able
pub-li-cation
second-ary
e-ject-ive
rev-o-lu-tion
re-stric-tive
Gasparini
moun-tain
moun-tains
yellow
label-ing
tra-di-tion-al-ly
cur-rent-ly
dia-chronic
Hebron
sub-se-quent-ly
doc-u-men-tation
Dom-ari
inter-actions
po-ten-tial
in-ven-tory
sim-pli-fi-cation
da-tive
pro-nom-in-al
or-i-gin
pre-po-si-tion
in-flec-tion-al
cat-e-gor-ic-al
in-di-cates
poss-ess-ive
de-termin-ation
rep-re-sent
e-lici-tation
typo-logy
ac-tiv-i-ty
ac-tiv-i-ties
fellows
in-ter-pre-tation
lin-guists
cycle
ne-ga-tion
sub-con-tinent
va-len-cy
ca-te-gor-ies
com-par-ison
gram-mat-i-cal
cor-res-pond-ing
ob-ser-vations
as-pec-tual
in-di-cation
ser-vice
iden-ti-fy-ing
utter-ances
no-ta-bly
for-mation
dis-tin-guish
mutand-is
muta-tis
in-di-vid-uals
lingua
natur-al
sub-sequent
re-cur-rence
rel-ative
rel-atives
mil-i-tary
ex-po-sure
spe-cif-ic-al-ly
rep-li-cate
rep-li-ca-ting
mod-i-fi-cations
gen-der
ṣəḥəttux
dra-ma-tic-al-ly
ac-cen-tuated
his-tor-i-cal
his-tor-i-cal-ly
event-ual-ly
pres-tige
dis-ap-pear
Romance
pen-insula
pen-insular
fea-ture
con-stitute
either
period
korufu
poly-sem-ous
Zammit
namrag
earli-er
pis-ellum
qillīd
Ta-rif-it
Reinhardt
situ-ation
Mohand
Zazaki
form-ación
Científicas
Roman-ica
butter-fly
roman-cismos
espec-ial-mente
Alexandrine
Ziamari
oriente
SELAF
cultur-elles
maroc-ain
medi-eval
medi-terranean
ɣesmat
Dordrecht
Trieste
}
\hyphenation{
Engels-zungen
affri-ca-te
affri-ca-tes
Bongor
Lotuho
Sprach-inseln
under-going
monkey
fortiori
kapaparát
báfura
imbíró
Bangala
Bettega
Bab-ylon-ian
Kouwenberg
Ammani
Amman
Ammanis
scen-arios
elabor-ate
elabor-ated
func-tion-al-ization
anthro-polo-gist
Iemmolo
table
Martin
gahawa
Tihāma
Sason
com-ple-ments
homo-phon-ous
Meso-potam-ian
Diyar-bakır
Tsiapera
Lucas
edu-ca-tion
cubit-ellu
corbata
morpho-phonology
morpho-phono-logic-al-ly
Volubilis
Mennen
Papa-constan-tinou
Vicente
re-inter-pre-ta-tion
megorashim
Malayalam
morpho-phon-em-ic-s
morpho-syn-tax
morpho-syn-tactic
Ki-nubi
ḥawār-iyy-ūn
archa-ic-ity
de-velop-ment
pro-so-dic
in-duced
pho-no-logy
possess-um
pos-ses-s-ive-s
Birgitt
es-pe-cial-ly
clause-s
po-ten-tial-ly
Lusignan
Cush-itic
re-fi-ni-ti-za-tion
lan-guage-s
lan-guage
dic-tion-ary
dic-tion-aries
socio-political
eight
counter-part
de-on-tic
Walter
Mifsud
ana-lys-is
re-ana-lys-is
re-ana-lyse
re-ana-lysed
re-ana-lyses
re-ana-lys-ing
Catalan
Occitan
Vigario
Sónia
ḥammāl
cross-linguist-ic-al-ly
south-western
bor-row-ed
ǧumruk
neg-ative
de-termin-er
Oxford
Socotra
Khuze-stan
Arama-ic
Otto-man
Otto-mans
Azeri
bi-lin-gual-ism
bor-row-ing
dia-lects
dia-lect
Ahwāzī
Catherine
allo-phone
allo-phones
phon-em-ic
hannaʔa
tubbaʕ
len-ition
Khorramshahr
poss-ible
Moroccan
Martine
Harrassowitz
gram-matica-lization
Wolfram
Afro-asiatica
Afro-asiatic
Semit-istik
mono-phthong
mono-phthong-ized
mono-phthong-izing
Heselwood
Hinrichs
Watson
Aramaeo-grams
Dadan-it-ic
sub-ordin-ator
Thamud-ic
Revue
Bovingdon
un-der
attach
attached
bundle
Arabia
graph-em-ic
graph-emes
ci-ca-da
Corriente
Andalusi
Kossmann
mor-pho-lo-gi-cal
dia-chroniques
islámica
occid-ent-al-ismo
idiom-a-ti-city
dia-lecto-logie
Reichert
co-lo-nial
Milton
diph-thong-al
lin-guis-tic
lin-guis-tique
lin-guis-tics
affairs
pho-net-ic-ally
kilo-metres
meta-typy
Jagiellonian
Arcodia
Hussein
Christopher
Giannakidou
Tashelhiyt
sta-bi-li-zation
de-vel-op-ments
in-vest-i-gation
Jor-dan-ian
af-fri-ca-tion
de-affri-ca-tion
Jordan
inter-dent-al
inter-locutors
socio-historical
roman-ization
Kurmanji
pha-ryn-geal-ization
con-sonant
con-sonants
phi-lo-logy
Öpengin
Kurd-ish
sunbul
sekin-în
fatala
dia-lect-al
Matras
mor-phol-o-gy
ḥaddād
emer-gence
Herzog
Benjmamins
struc-ture
Orient-studien
Ana-tolia
Ana-tolian
American
vari-ation
Jastrow
Geoffrey
Yarshater
Ḥarsūsi
Umberto
Ashtiany
Edmund
Mahnaz
En-cyclo-pædia
En-cyclo-paedia
En-cyclo-pedia
Leiden
dia-spora
Sasan-ian
every-day
domin-ance
Con-stitu-tion-al
Manfredi
pro-duct-iv-ity
Morocco
ferman
Maghreb-ian
algérien
Nicolaï
Mouton
maur-i-tani-en
Gotho-burg-ensis
socio-linguist-ique
socio-linguistic
dia-chronica
dihāt
de-velop-ed
ṣuḥbat
kitāba
kitābat
com-mercial
Senegal
mechan-ics
Maur-itan-ia
Ḥassān-iyya
circum-cision
cor-relation
labio-velar-ization
vowel
vowels
īggīw
series
in-tegrates
dur-ative
inter-dent-als
gen-itive
Tuareg
tălămut
talawmāyət
con-diment
Wiesbaden
Uni-vers-idad
Geuthner
typo-logie
Maur-itanie
nomades
Maur-itan-ian
dia-lecto-logy
dia-lecto-logical
Sahar-iennes
Uni-ver-sity
de-scend-ants
NENA-speak-ing
speak-ing
xaddaːm
Kahane
Tinniswood
Ven-etian
rabia
Cohen
dubi-ous
Al-geria
trans-humance
Marrakech
Bender
Munich
origin-al
re-captured
ethnic
minor-it-ies
drama-tic
local
long-stand-ing
regions
Nineveh
misti-linguismo
settle-ments
Ṣəndor
Mandate
sub-sti-tu-ting
or-thog-ra-phy
ref-er-enced
origin-ate
twenti-eth
typ-ic-al-ly
Hobrack
never-the-less
masc-uline
coffee
ex-clu-sive-ly
ver-bal
sim-i-lar-i-ties
der-i-va-tion
im-pera-tive
par-ti-ci-ple
dis-am-big-uation
dis-am-big-uating
phen-omen-on
phen-omen-a
traktar
Coetsem
mVCCūC
kān-at
Ing-ham
uni-versidad
Kerman-shah
Brin-cat
Tarifiyt
Vanhove
pre-verbal
Morris
Soqoṭri
dis-pre-fer-red
ex-pla-nation
con-struc-tion
Behnstedt
Andersen
wide-spread
usu-al-ly
re-gion-al
Bulut
con-sider-ation
afro-asia-tici
Franco-Angeli
Phono-logie
Volks-kundliche
dia-lectes
dia-lecte
dis-appear-ance
media
under-stand-able
pub-li-cation
second-ary
e-ject-ive
rev-o-lu-tion
re-stric-tive
Gasparini
moun-tain
moun-tains
yellow
label-ing
tra-di-tion-al-ly
cur-rent-ly
dia-chronic
Hebron
sub-se-quent-ly
doc-u-men-tation
Dom-ari
inter-actions
po-ten-tial
in-ven-tory
sim-pli-fi-cation
da-tive
pro-nom-in-al
or-i-gin
pre-po-si-tion
in-flec-tion-al
cat-e-gor-ic-al
in-di-cates
poss-ess-ive
de-termin-ation
rep-re-sent
e-lici-tation
typo-logy
ac-tiv-i-ty
ac-tiv-i-ties
fellows
in-ter-pre-tation
lin-guists
cycle
ne-ga-tion
sub-con-tinent
va-len-cy
ca-te-gor-ies
com-par-ison
gram-mat-i-cal
cor-res-pond-ing
ob-ser-vations
as-pec-tual
in-di-cation
ser-vice
iden-ti-fy-ing
utter-ances
no-ta-bly
for-mation
dis-tin-guish
mutand-is
muta-tis
in-di-vid-uals
lingua
natur-al
sub-sequent
re-cur-rence
rel-ative
rel-atives
mil-i-tary
ex-po-sure
spe-cif-ic-al-ly
rep-li-cate
rep-li-ca-ting
mod-i-fi-cations
gen-der
ṣəḥəttux
dra-ma-tic-al-ly
ac-cen-tuated
his-tor-i-cal
his-tor-i-cal-ly
event-ual-ly
pres-tige
dis-ap-pear
Romance
pen-insula
pen-insular
fea-ture
con-stitute
either
period
korufu
poly-sem-ous
Zammit
namrag
earli-er
pis-ellum
qillīd
Ta-rif-it
Reinhardt
situ-ation
Mohand
Zazaki
form-ación
Científicas
Roman-ica
butter-fly
roman-cismos
espec-ial-mente
Alexandrine
Ziamari
oriente
SELAF
cultur-elles
maroc-ain
medi-eval
medi-terranean
ɣesmat
Dordrecht
Trieste
}
\hyphenation{
Engels-zungen
affri-ca-te
affri-ca-tes
Bongor
Lotuho
Sprach-inseln
under-going
monkey
fortiori
kapaparát
báfura
imbíró
Bangala
Bettega
Bab-ylon-ian
Kouwenberg
Ammani
Amman
Ammanis
scen-arios
elabor-ate
elabor-ated
func-tion-al-ization
anthro-polo-gist
Iemmolo
table
Martin
gahawa
Tihāma
Sason
com-ple-ments
homo-phon-ous
Meso-potam-ian
Diyar-bakır
Tsiapera
Lucas
edu-ca-tion
cubit-ellu
corbata
morpho-phonology
morpho-phono-logic-al-ly
Volubilis
Mennen
Papa-constan-tinou
Vicente
re-inter-pre-ta-tion
megorashim
Malayalam
morpho-phon-em-ic-s
morpho-syn-tax
morpho-syn-tactic
Ki-nubi
ḥawār-iyy-ūn
archa-ic-ity
de-velop-ment
pro-so-dic
in-duced
pho-no-logy
possess-um
pos-ses-s-ive-s
Birgitt
es-pe-cial-ly
clause-s
po-ten-tial-ly
Lusignan
Cush-itic
re-fi-ni-ti-za-tion
lan-guage-s
lan-guage
dic-tion-ary
dic-tion-aries
socio-political
eight
counter-part
de-on-tic
Walter
Mifsud
ana-lys-is
re-ana-lys-is
re-ana-lyse
re-ana-lysed
re-ana-lyses
re-ana-lys-ing
Catalan
Occitan
Vigario
Sónia
ḥammāl
cross-linguist-ic-al-ly
south-western
bor-row-ed
ǧumruk
neg-ative
de-termin-er
Oxford
Socotra
Khuze-stan
Arama-ic
Otto-man
Otto-mans
Azeri
bi-lin-gual-ism
bor-row-ing
dia-lects
dia-lect
Ahwāzī
Catherine
allo-phone
allo-phones
phon-em-ic
hannaʔa
tubbaʕ
len-ition
Khorramshahr
poss-ible
Moroccan
Martine
Harrassowitz
gram-matica-lization
Wolfram
Afro-asiatica
Afro-asiatic
Semit-istik
mono-phthong
mono-phthong-ized
mono-phthong-izing
Heselwood
Hinrichs
Watson
Aramaeo-grams
Dadan-it-ic
sub-ordin-ator
Thamud-ic
Revue
Bovingdon
un-der
attach
attached
bundle
Arabia
graph-em-ic
graph-emes
ci-ca-da
Corriente
Andalusi
Kossmann
mor-pho-lo-gi-cal
dia-chroniques
islámica
occid-ent-al-ismo
idiom-a-ti-city
dia-lecto-logie
Reichert
co-lo-nial
Milton
diph-thong-al
lin-guis-tic
lin-guis-tique
lin-guis-tics
affairs
pho-net-ic-ally
kilo-metres
meta-typy
Jagiellonian
Arcodia
Hussein
Christopher
Giannakidou
Tashelhiyt
sta-bi-li-zation
de-vel-op-ments
in-vest-i-gation
Jor-dan-ian
af-fri-ca-tion
de-affri-ca-tion
Jordan
inter-dent-al
inter-locutors
socio-historical
roman-ization
Kurmanji
pha-ryn-geal-ization
con-sonant
con-sonants
phi-lo-logy
Öpengin
Kurd-ish
sunbul
sekin-în
fatala
dia-lect-al
Matras
mor-phol-o-gy
ḥaddād
emer-gence
Herzog
Benjmamins
struc-ture
Orient-studien
Ana-tolia
Ana-tolian
American
vari-ation
Jastrow
Geoffrey
Yarshater
Ḥarsūsi
Umberto
Ashtiany
Edmund
Mahnaz
En-cyclo-pædia
En-cyclo-paedia
En-cyclo-pedia
Leiden
dia-spora
Sasan-ian
every-day
domin-ance
Con-stitu-tion-al
Manfredi
pro-duct-iv-ity
Morocco
ferman
Maghreb-ian
algérien
Nicolaï
Mouton
maur-i-tani-en
Gotho-burg-ensis
socio-linguist-ique
socio-linguistic
dia-chronica
dihāt
de-velop-ed
ṣuḥbat
kitāba
kitābat
com-mercial
Senegal
mechan-ics
Maur-itan-ia
Ḥassān-iyya
circum-cision
cor-relation
labio-velar-ization
vowel
vowels
īggīw
series
in-tegrates
dur-ative
inter-dent-als
gen-itive
Tuareg
tălămut
talawmāyət
con-diment
Wiesbaden
Uni-vers-idad
Geuthner
typo-logie
Maur-itanie
nomades
Maur-itan-ian
dia-lecto-logy
dia-lecto-logical
Sahar-iennes
Uni-ver-sity
de-scend-ants
NENA-speak-ing
speak-ing
xaddaːm
Kahane
Tinniswood
Ven-etian
rabia
Cohen
dubi-ous
Al-geria
trans-humance
Marrakech
Bender
Munich
origin-al
re-captured
ethnic
minor-it-ies
drama-tic
local
long-stand-ing
regions
Nineveh
misti-linguismo
settle-ments
Ṣəndor
Mandate
sub-sti-tu-ting
or-thog-ra-phy
ref-er-enced
origin-ate
twenti-eth
typ-ic-al-ly
Hobrack
never-the-less
masc-uline
coffee
ex-clu-sive-ly
ver-bal
sim-i-lar-i-ties
der-i-va-tion
im-pera-tive
par-ti-ci-ple
dis-am-big-uation
dis-am-big-uating
phen-omen-on
phen-omen-a
traktar
Coetsem
mVCCūC
kān-at
Ing-ham
uni-versidad
Kerman-shah
Brin-cat
Tarifiyt
Vanhove
pre-verbal
Morris
Soqoṭri
dis-pre-fer-red
ex-pla-nation
con-struc-tion
Behnstedt
Andersen
wide-spread
usu-al-ly
re-gion-al
Bulut
con-sider-ation
afro-asia-tici
Franco-Angeli
Phono-logie
Volks-kundliche
dia-lectes
dia-lecte
dis-appear-ance
media
under-stand-able
pub-li-cation
second-ary
e-ject-ive
rev-o-lu-tion
re-stric-tive
Gasparini
moun-tain
moun-tains
yellow
label-ing
tra-di-tion-al-ly
cur-rent-ly
dia-chronic
Hebron
sub-se-quent-ly
doc-u-men-tation
Dom-ari
inter-actions
po-ten-tial
in-ven-tory
sim-pli-fi-cation
da-tive
pro-nom-in-al
or-i-gin
pre-po-si-tion
in-flec-tion-al
cat-e-gor-ic-al
in-di-cates
poss-ess-ive
de-termin-ation
rep-re-sent
e-lici-tation
typo-logy
ac-tiv-i-ty
ac-tiv-i-ties
fellows
in-ter-pre-tation
lin-guists
cycle
ne-ga-tion
sub-con-tinent
va-len-cy
ca-te-gor-ies
com-par-ison
gram-mat-i-cal
cor-res-pond-ing
ob-ser-vations
as-pec-tual
in-di-cation
ser-vice
iden-ti-fy-ing
utter-ances
no-ta-bly
for-mation
dis-tin-guish
mutand-is
muta-tis
in-di-vid-uals
lingua
natur-al
sub-sequent
re-cur-rence
rel-ative
rel-atives
mil-i-tary
ex-po-sure
spe-cif-ic-al-ly
rep-li-cate
rep-li-ca-ting
mod-i-fi-cations
gen-der
ṣəḥəttux
dra-ma-tic-al-ly
ac-cen-tuated
his-tor-i-cal
his-tor-i-cal-ly
event-ual-ly
pres-tige
dis-ap-pear
Romance
pen-insula
pen-insular
fea-ture
con-stitute
either
period
korufu
poly-sem-ous
Zammit
namrag
earli-er
pis-ellum
qillīd
Ta-rif-it
Reinhardt
situ-ation
Mohand
Zazaki
form-ación
Científicas
Roman-ica
butter-fly
roman-cismos
espec-ial-mente
Alexandrine
Ziamari
oriente
SELAF
cultur-elles
maroc-ain
medi-eval
medi-terranean
ɣesmat
Dordrecht
Trieste
} 
  \togglepaper[1]%%chapternumber
}{}

\begin{document}
\maketitle 




\section{Historical development of Andalusi Arabic}


A dialect of the western Neo-Arabic type, Andalusi Arabic is currently a dead language. It was spoken from the eighth to the seventeenth century in a changing territory following historical vicissitudes. 

Arabic arrived in the Iberian Peninsula in the eighth century with Arabic-speaking tribes coming from different zones at various stages.\footnote{Historians have long argued for the ethnic variety of the Arabs who invaded the Iberian Peninsula, particularly referring to the presence of Syrian and Yemeni tribes. See \citet{TerésSádaba1957}, \citet{Al-Wasif1990} and \citet{Guichard1995}.} According to historical sources, the number of Muslims initially arriving was small, most of them probably partially arabized Berber-speakers from North Africa.\footnote{Historians agree that it is extremely difficult, if not impossible, to establish what the level of arabization of this population was. According to Manzano Moreno (\citeyear{ManzanoMoreno1990}: 399), it seems that linguistic arabization was not widespread among Andalusi Berbers at least during the eighth century.} Over time, the society of al-Andalus (the name given to the territory in the Iberian Peninsula under different Muslim–Arab systems of rule for eight centuries) would eventually come to use a distinctive variety of Maghrebi Arabic known as Andalusi Arabic.\footnote{Andalusi Arabic features the only common discriminating trait of Maghrebi varieties, that is, the \textit{n}{}- and \textit{n-…-u} desinences for the first person singular and plural of the imperfect (cf. Benkato, this volume).\ia{Benkato, Adam@Benkato, Adam}} This variety evolved through dialectal levelling and changes resulting from contact with other languages present in the zone, and became a reasonably unified variety by the tenth century. The political success of the Umayyad dynasty and the establishment of their caliphate in the year 929 CE may have contributed to language levelling, though dialect variation continued to exist in the form of diatopical variants from various regions; scholars thus refer to the existence of an Andalusi ``dialect bundle" (e.g. \citealt[6]{Corriente1977}, \citeyear[446]{Corriente1992chapter}). For instance, the Granadian variety seems to have been more conservative than those spoken in other regions.\footnote{According to Corriente (\citeyear[56]{Corriente1998stress}), \textcolor{red}{this is because Granada was relatively isolated from the Andalusi mainstream, and played a secondary political role. An example that Corriente gives of this conservatism is the retention of strong \textit{imāla} (raising of originally low front vowels) found in Granadian Arabic, since this feature was eliminated or reduced in other Andalusi varieties with written attestation.}} The regional Andalusi variety spoken in Valencia was the last to disappear with the expulsion of the \textit{moriscos} \textcolor{red}{(descendants of North Africans)} in the seventeenth century (\citealt{BarcelóLabarta2009}: 117).

Even though Andalusi Arabic was a vernacular variety, the few extant sources are always written, and therefore reflect a higher register than that of the language used for daily communication. In fact, hardly any material reflecting the everyday dialectal level is available, since most of the sources consist of texts written in Middle Arabic (i.e. a written form intermediate between Classical and spoken dialectal Arabic; see \citealt{Lentin2011Middle}). Furthermore, complications arise due to the use of Arabic script to record dialect variants.\footnote{An overview of sources for the description of Andalusi Arabic can be found in Corriente et al. (\citeyear{CorrientePereiraVicente2015}: xxiii--xxiv).} 

Consequently, a comprehensive view of all the periods and places where this language was spoken is lacking. For instance, sources are scarce regarding the use of the language in the eighth and ninth centuries. As Wasserstein (\citeyear[3]{Wasserstein1991}) put it: “A linguistic map of Islamic Spain for any period between the middle of the eighth century and the middle of the thirteenth century would be extremely difficult to draw.”

Notwithstanding this, written documents in Andalusi Arabic are available from the tenth century until the expulsion of the \textit{moriscos} in the seventeenth century. The oldest documented and preserved Andalusi text is an early form of \textit{zaǧal} poetry dating from 913 CE, illustrated in \REF{laban}.\footnote{It consists of a verse by one of the supporters of ʕUmar ibn Ḥafsūn,\ia{ʕUmar ibn Ḥafsūn} insulting the caliph ʕAbd ar-Raḥmān III.\ia{ʕAbd ar-Raḥmān III} It appears in the historical chronicle \textit{al-Muqtabis} \textit{V}, by Ibn Ḥayyān.\ia{Ibn Ḥayyān}} 

\ea\label{laban}
Tenth-century Andalusi Arabic (\citealt{CorrientePereiraVicente2015}: 237).\footnote{Acute accents on vowels in transcription of Andalusi Arabic represent stress rather than vowel length. See §\ref{bkm:Ref12959774} for further details.} \\
\ea \gll labán úmm-u fi fúmm-u\\
     milk mother-\textsc{3sg.m} in mouth-\textsc{3sg.m}\\
\glt ‘His mother’s milk is in his mouth.’

\ex\label{ex:key:}
\gll rás ban ḥafṣún fi ḥúkm-u \\
head Ban Ḥafṣún in power-\textsc{3sg.m}\\
\glt ‘Ban Ḥafṣún’s head is at his disposal.’
\z
\z

The latest attestations of this language consist of private documents written by \textit{moriscos} from Valencia from seventeenth century, in which interesting instances of Romance dialectalisms and influence of Catalan, the Romance language spoken in the region, Aragonese and Castilian can be observed (\citealt{BarcelóLabarta2009}: 119).

Andalusi Arabic continued to be spoken in the Iberian Peninsula after the end of al-Andalus as a Muslim–Arab state in 1492 CE, as some of the Arabic-speaking population remained in certain regions up until the seventeenth century, when the last \textit{moriscos} were expelled. This language was therefore taken by the migrant population to various places in North Africa in different periods from the Middle Ages up to the Modern Era.\footnote{This is the reason why Andalusi Arabic has played a very important role in the formation of the Moroccan Arabic (cf. \citealt{Vicente2010}; Heath, this volume).\ia{Heath, Jeffrey@Heath, Jeffrey}} 

Initially the second language (L2) of most of the population, after a two-century gestation process (approximately from the conquest in 711 to the beginning of the caliphate in 929), Andalusi Arabic gradually became the first language (L1) of the majority of the population, overtaking the Romance dialect spoken by the original local population. The main reason for this was the growing social prestige attached to Arabic in an Islamic society, in contrast to the lower social status of Andalusi Romance, which became the local L2 and eventually disappeared.\footnote{Mixed marriages between Muslims and Christian women constituted a significant factor in the propagation of Andalusi Arabic amongst \textcolor{red}{Christians until it also became their L1} (\citealt{Guichard1989}: 82–83, \citeyear{Guichard1995}: 456–457; \citealt{Chalmeta2003}).}  

Andalusi Arabic became the dominant language (regardless of religion) thanks to the political and social situation of al-Andalus. Furthermore, the advent of an Arabic-speaking population from the east, especially in the Umayyad caliphate (929--1031), played a major role in the expansion of arabization. According to some scholars such as Fiero Bello (\citeyear{FierroBello2001}) and Corriente (\citeyear[104]{Corriente2008}), al-Andalus became a society largely monolingual in Andalusi Arabic around the eleventh century, though communities using other languages did exist, especially in rural areas.\footnote{Other scholars disagree, for instance Galmés de Fuentes (\citeyear{GalmésdeFuentes1994}: 81–88) and Menéndez Pidal \& Galmés de Fuentes (\citeyear{MenéndezPidalGalmésdeFuentes2001}), who affirmed that the Romance language still had a considerable social and literary status even in the period dominated by the Almoravid and the Almohad dynasties (twelfth and thriteenth centuries).}  

The vernacular Arabic variety spoken in al-Andalus even reached the status of a literary language, appropriating part of the domain of Classical Arabic through proverbs and a number of stanza-based poetic forms (including some \textit{ḫaraǧāt} and the \textit{azǧāl}). Andalusi Arabic poetry reached the circles of the court and the palaces of Taifa Kings. Such social and cultural prestige reveals the extent to which Andalusi Arabic had become the dominant language in this society, and it is for this reason that it is the best-documented vernacular Arabic variety of all those spoken in the Middle Ages. 

Andalusi Arabic does not conform neatly to either the bedouin or the pre-Hilali sedentary type of dialect in the classification usually applied nowdays to Maghrebi Arabic dialects (cf. Benkato, this volume).\ia{Benkato, Adam@Benkateo, Adam} It shares features of both types of dialects. For instance, in the phonological system the three interdental phonemes are the same as those in Old Arabic as happens in bedouin-type Maghrebi dialects;\footnote{In sedentary-type Maghrebi dialects these are typically pronounced as occlusives. The data do show that the occlusive pronunciation of interdentals was known in Andalusi Arabic, though it was considered vulgar and was repressed (\citealt{CorrientePereiraVicente2015}: 29).}  however, /q/ is realized using the voiceless variant [q] as in sedentary-type dialects rather than the voiced variant [g], as in bedouin-type dialects.\footnote{That said, /q/ may have been realized as a voiced [g] in some registers, regions or periods in Andalusi Arabic, see (\citealt{CorrientePereiraVicente2015}: 64).}

According to Corriente (\citeyear[34]{Corriente1992}), the largest number of speakers was reached between the eleventh – a time when the Andalusi koine reached maturity – and the twelfth century. 


\section{Contact languages}


Andalusi Arabic developed in the Iberian Peninsula through the interaction of various Arabic dialects \footnote{Besides oriental Neo-Arabic varieties brought by invaders in the eighth century, from which Andalusi Arabic emerged, this language continued to evolve and interact with Maghrebi dialects, particularly with Moroccan Arabic. Owing to this it is possible to find intra-Arabic contact-induced language change, for instance in the Andalusi variety of Granada. Some instances of transfer from Moroccan are the verbs \textit{šāf} ‘to see’ and \textit{žāb} ‘to bring’, and the second element in the negative \textit{ma} \textit{šāf} \textit{ši} ‘he did not see’, see \citep[57]{Corriente1998a}. For instance, the particle \textit{lás} or its variant with \textit{imala} \textit{lís} was the most used for negation in Andalusi Arabic, while this construction \textit{ma} +verb+ \textit{ši} was exceptional {in older sources but not in the work of a}š-Šustarī, a Granadian{ author, due to his travels to North Africa, according to} (Corriente, \citealt{PereiraVicente2015}: 212–215){.}In addition, Classical Arabic had an influence, especially on the lexicon. The migration of the bedouin population into North Africa, however, did not have an influence on the evolution of Andalusi Arabic.} alongside two contact languages. This situation spanned a long period of time, resulting in a significant amount of transfer. This has been analysed by various authors \citep{Ferrando1995}, \citep{Ferrando1997}, \citep{Vicente2006}, and particularly by Corriente, see for instance \citep{Corriente1981}, \citep{Corriente1992}, \citep{Corriente2000}, \citep{Corriente2002}. 

The languages Andalusi Arabic was in contact with were the Romance varieties spoken by the Andalusi population and the Berber varieties brought by different Berber speakers arriving in al-Andalus during its existence. 



\subsection{Andalusi Romance}



Andalusi Romance is a dialect bundle originating in the Iberian proto-Romance language spoken in the Iberian Peninsula when the Islamic invasion occurred in 711,\footnote{This Iberian proto-Romance originated in Hispanic Vulgar Latin with transfers from pre-Romance languages and Visigoth lexical borrowings.}  which underwent a particular evolution through interaction with Arabic. This Ibero-Romance dialect was the L1 of a large proportion of Andalusi society regardless of their religion. It is also the oldest documented Ibero-Romance variety, according {to Corriente, the language of the \textit{xaraǧāt} reflects the Romance dialectal bunddle used in al-Andalus between the ninth and eleventh centuries,} {\citep{Corriente1995}}{,} {\citep{Corriente1997a}}{,} {\citep{Corriente2000}}{.} 

The language is not well known: only a few written sources are available, transmitted by copyists who may have had limited knowledge of the language. These sources are written both in Arabic and Latin scripts. 

Sources in Arabic script consist of bilingual dictionaries and botanical, agronomical and medical glossaries. These evidence a limited amount of Romance lexical borrowing, constituting less than 5\% of the lexicon according to \citep[142]{Corriente1992}.

Another source in Arabic script are \textit{xaraǧāt}, the final refrains of each stanza of the \textit{muwaššaḥāt}, one of the two types of Andalusi strophic poetry. A few of these refrains were partially written in Andalusi Romance.\footnote{Up to 68 \textit{xaraǧāt} in Andalusi Romance have been found, 42 in Arabic script and 26 in Hebrew script with one or more words in this language, see (\citealt{Corriente1997a}: 268–323), all of them dating from the tenth-eleventh centuries, see \citep[343]{Corriente1997a}.} In addition to these \textit{xaraǧāt}, Romance borrowings were also transmitted in the \textit{zaǧal} poems of Ibn Quzman. 

Latin script sources also exist, in toponymy, for instance, as well as in loanwords from Andalusi Romance in northern Romance dialects, though the data these contribute need to be treated with caution, since adaptation to other Romance dialects blurs the features of the source language, making them of limited use from a linguistic point of view. 

This language has been analysed by Corriente (see for instance \citep{Corriente1995}, \citep{Corriente2000}, \citep{Corriente2012} who compiles lexical borrowings from botanical glossaries and from \textit{xaraǧāt} poetry.  

In the first centuries of the history of al-Andalus, Andalusi Romance was the L1 used by the majority of Andalusi society, even by some Muslims, such as the \textit{muwalladūn} (converted Muslims), who would learn Arabic as their L2 for self-promotion in society. In time, however, as an Arabic variety became the dominant language, diastratic differences become noticeable. Thus, Andalusi Romance was the L1 used by the rural population and the lower classes, whereas the urban Andalusi population underwent more rapid arabization due to increased exposure to Arabic through mosques, schools, trade, pilgrimages, and so on. Thus, the inhabitants of cities and, above all, leading members of society always had Andalusi Arabic as their L1. 

No concrete evidence exists as to when monolingualism in Andalusi Arabic became established. The most commonly accepted date for the disappearance of Romance as a common means of communication in al-Andalus is the late twelfth century, under Almoravid rule. This period saw migrations north out of al-Andalus of the Christian Mozarabs, although most of these were in fact Arabic-speakers, as instances of lexical borrowings from Andalusi Arabic in Romance languages from the north reveal. Corriente suggests that bilingualism no longer existed by the thirteenth century and that in the eleventh and twelfth century was merely vestigial \citep{Corriente1997a}, \citep[443]{Corriente1992}, \citep{Corriente2005}. In contrast, Galmés de Fuentes and Menéndez Pidal have defended the existence of bilingualism in Andalusi society up until the thirteenth century, see (Galmés de \citealt{Fuentes1994}: 81–88), (Menéndez Pidal \& Galmés de \citealt{Fuentes2001}).\footnote{While some Romance-speaking communities may indeed have lasted up until the thirteenth century, note that this circumstance does not imply the existence of a wider bilingual Andalusi society.}  



\subsection{Berber}



The arrival of a Berber-speaking population in al-Andalus took place in the eighth and thirteenth centuries, first as auxiliary troops and later as conquerors, though many of them may have been already arabized and used an early form of North African Arabic as L2 or even as L1 in the case of those arriving later. 

Modern historiography (Manzano \citealt{Moreno1990}), \citep{Guichard1995}, \citep{Chalmeta2003} reveals that a significant number of Berbers played a major role in the conquest of al-Andalus, a population which grew larger with the later arrival of the Almoravid and Almohad dynasties in the twelfth and thirteenth centuries. Interaction between Arabic-speaking and Berber-speaking populations on both sides of the Strait of Gibraltar facilitated lasting language contact. 

The role of Berber in the language development of al-Andalus has not been analysed in depth, however. This is due to data being scarce regarding not only the state of Berber varieties at the time, but also their impact on Andalusi Arabic and the speed of their disappearance from the language scene in the Iberian Peninsula. No sources exist written directly in Berber, plus interpretation issues arise due to the transmission of Berber loanwords in Arabic or Latin script, as the phonological systems do not fully coincide. Their transmission in Arabic script by botanical glossaries authors has been the most common method applied for the transfer of the Andalusi Berberisms.  

Berber varieties had no social prestige and corresponded to lower registers and non-urban areas, a fact which had an obvious repercussion on the direction of transfers in contact-induced changes. According to scholars \citep[160]{Chalmeta2003}, \citep{Guichard1995} the reason behind this could be the Berbers’ social organization, who tended to settle in rural zones. 

As a result of all of the above plus the fact that the number of local Romance-speakers was much higher, there is far less transfer from Berber than from Romance in Andalusi Arabic. 

These transfers basically consist of lexical borrowings and have been analysed by various authors: \citep{Ferrando1997}, which includes an unpublished analysis conducted by Colin, \citep{Corriente1981}, \citep{Corriente1998b}, \citep{Corriente2002}, (Corriente, \citealt{PereiraVicente2017}) and (Corriente, Pereira \& Vicente Forthcoming).


\section{Contact-induced changes in Andalusi Arabic} 

 

\section{Contact with Andalusi Romance}



A special feature of the linguistic history of al-Andalus is that within a few centuries a situation of bilingualism, whereby the Romance language was the L1 for most of the population while Andalusi Arabic was L2, was reversed, eventually leading to a third phase of monolingualism using only Andalusi Arabic. 

Transfers from Romance to Andalusi Arabic probably took place during the first of the bilingualism phases, a situation which, according to Corriente, see \citep{Corriente2005} and \citep{Corriente2008}, must have lasted two hundred years, from the eighth to the tenth century. 

It is complex to diagnose what type of transfer took place in such an ancient contact situation. When the agents of change used Romance (the source language; SL) as L1 and Andalusi Arabic (the recipient language; RL) as L2, the type of change was imposition, according to the framework of \citep{Coetsem1988} and \citep{Coetsem2000}. As we have seen, however, this situation would evolve and the agents of change would come to have Andalusi Arabic (the RL) as their L1 and Romance (the SL) as their L2, meaning that transfer in this situation would be classified as borrowing in Van Coetsem’s framework. 

However, in cases such as this where the precise sociolinguistic situation at a given time is impossible to judge, it is difficult to establish whether the agents of change had two L1s or one L1 and one L2. Thus, the possibility exists that the contact-induced language changes taking place are a convergence type of transfer (in the terms of \citep{Lucas2015}. 




\subsection{\label{bkm:Ref12959774}Phonology}




A contact-induced language change from Romance concerned the prosodic rhythm of Andalusi Arabic. The quantitative rhythm of Old Arabic was replaced by the intense stress system of proto-Romance languages in the Iberian Peninsula.\footnote{A change which had taken place in Latin about one-thousand years earlier. This language evolved from a quantitative stress system to an intense stress system in its daughter languages. The same process took place later in Andalusi Arabic.} Thus, while all Old Arabic and Neo-Arabic varieties feature a prosodic rhythm that distinguishes long and short syllables, Andalusi Arabic is the only variety where this quantitative rhythm was replaced by a system where there is no phonemic vowel length \citep{Corriente1977}, \citep{Corriente1992} and (Corriente, \citealt{PereiraVicente2015}: 75–78).

In this case, the agents of change were presumably speakers of Andalusi Romance as an L1, making the transfer a case of imposition on the L2, in this case Andalusi Arabic. 

The altered use of the \textit{matres} \textit{lectionis} in the Arabic script constitutes graphemic evidence of this change in prosodic rhythm. Thus, in Andalusi sources, the graphemes which traditionally mark the Old Arabic long vowels are sometimes used mark etymologically short vowels, to indicate that these are stressed. For instance: {\arabscript{مقاص}} \textit{{muqāṣ}} = /muqáṣṣ/ ‘pair of scissors’ (from \textit{miqaṣṣ}), {\arabicfont{أسقوف}} \textit{usqūf} = /usqúf/ ‘bishop’ (from \textit{usquf}), {\arabicfont {قنفود }} \textit{qunfūd} = /qunfúd/ ‘hedgehog’ (from \textit{qunfud}).  

Moreover, historically long vowels that were not stressed are often represented without the regular \textit{matres} \textit{lectionis}, for instance: {\arabicfont{ فران}} \textit{firān} = /firán/ ‘mice’, {\arabicfont {عم}} \textit{ʕam} = /ʕam/ ‘year’.

Another instance is the very name \textit{al-Andalus}, pronounced by its inhabitants as /alandalús/, a fact known due to the \textit{matres} \textit{lectionis} for /ū/ which appears in the final syllable, indicating that the vowel \textit{u} is stressed: {\arabicfont{ الاندلوس }} al-andalūs = /alandalús/.

In addition, lexical borrowings from Andalusi Arabic currently found in Ibero-Romance languages also attest to this change of prosodic rhythm. For instance, the Spanish word \textit{andaluz} (stressed on the last syllable) can only originate in the Andalusi word /alandalús/. While the Spanish word \textit{azahar} ‘orange blossom’ (also stressed on the last syllable) comes from the Andalusi word /azzahár/, rather than directly from Old Arabic \textit{zahr} ‘flower’.

The use of \textit{matres} \textit{lectionis} in this way was by no means systematic, since {less cultivated scribes inserted or suppressed them arbitrarily}; a fact which could be interpreted as indicative of an incipient evolution towards the loss of phonological value of the stress in Andalusi prosody, (Corriente, \citealt{PereiraVicente2015}: 76, n. 213), a phenomenon that characterizes nowadays Moroccan Arabic, maybe the last step of this evolution in Maghrebi Arabic dialects.

{In some cases, a graphic gemination of the following consonant instead of the grapheme of the vocal quantity indicates also an accentuated vowel, for instance: \textit{usquff} = \textit{usqúf} ‘bishop’, \textit{ṯiqqa} = \textit{ṯíqa} ‘trust’,} {(Corriente, \citealt{PereiraVicente2015}: 77)}{.} 

In terms of phonology, Andalusi Arabic also features the appearance of three marginal phonemes /p/, /g/ and /č/ as a transfer from Andalusi Romance, which, however, may not have existed in some Andalusi sub-dialects. Bearing in mind that these phonemes were incorporated through lexical borrowings \citep{Corriente1978}, we can assume that the agents of change had Andalusi Arabic as L1 and that therefore this is a borrowing type of transfer. For instance: \textit{čípp} ‘trap’, \textit{čiqála} ‘cicada’, \textit{čírniya} ‘blackbird’ (Corriente, \citealt{PereiraVicente2015}: 57). As these phonemes exist even in late toponymy it may be concluded that they were part of the Andalusi phonological system. 

Another contact-induced phonological change was the partial loss of contrastive velarization in some phonemes. As velarization does not exist in Romance languages, we can assume that this was a case of phonological imposition by L1 Romance speakers on their L2 Andalusi Arabic. 

The effects of this change are visible, for instance, in the frequent interchangeability of /s/ and /ṣ/. Recurrent permutations between both realizations exist and pseudo-corrections are also in evidence. For example: /sūr, ṣūr/ ‘wall’, /nāqūs, nāqūṣ/ ‘bell’, /qaswa, qaṣwa/ ‘cruelty’. This is not, however, a very common feature and took place only in the early stages of the arabization process (Corriente, \citealt{PereiraVicente2015}: 82). 

The fricatization of the occlusive is another contact-induced change. According to Romanists this phenomenon was common in Romance languages since the Latin period. It consists of a phonological imposition of the SL, Andalui Romance, on RL, Andalusi Arabic\footnote{The fricatization of the occlusives is also a feature of some Arabic varieties spoken in Morocco especially, though not exclusively, in the North. In this case, the agents of change are Arabic-Berber bilingual speakers who incorporate the borrowing into their L1 and RL, from their L2 and SL. This may have also happened in Andalusi society, though data to corroborate it is insufficient.} . 

The fricatization of /d/ / [ð] can be observed. Authors of Andalusi Arabic would write {\arabicfont { ذ }} /ð/ rather than{\arabicfont { د }}/d/ for both *d and *ð because they considered both sounds to be allophones of /d/, particularly in the postvocalic position.\footnote{This fricatization is also realized in other positions, however.}  The realization of the /d/ phoneme clearly changed through contact with Andalusi Romance. This is a widespread feature noted in various authors, regions, ages and social groups. For instance: {\arabicfont {جذول }} /ǧaðwal/ ‘creek’ / \textit{ǧadwal}, {\arabicfont {حفيذ }} /ḥafīð/ ‘nephew’ / \textit{ḥafīd}, {\arabicfont {ألحذ} } /al-ḥaðð/ ‘Sunday’ / \textit{al-ḥadd}, {\arabicfont{ سيذي }} /sīði/ ‘my lord’ / \textit{sīdi}. This phenomenon seems to have been more common in lower and middle registers of Andalusi. 

Another example is the fricatized allophone of /b/ / [β], which could constitute a borrowing from Romance or Zenati Berber. This may be confirmed by the replacement of /b/ with /f/, as in /qasfūrā/ / \textit{kuzbara} ‘coriander’, /fiš/ / \textit{baš/biš} ‘in order to’, or confusion between both phonemes: /baysāra/, /faysāra/ ‘a dish of cooked beans’ (Corriente, \citealt{PereiraVicente2015}: 19).
 

\subsubsection{Morphology}




A noteworthy contact-induced morphological change concerns the elimination of a gender distinction in the 2nd person singular of both pronouns and verbs. For instance: \textit{taqtúl} ‘you kill’, \textit{tikassár} ‘you break’, \textit{taḥtarám} ‘you respect’, \textit{taxriǧ} ‘you throw’, see (Corriente, \citealt{PereiraVicente2015}: 154–155).

The addition of Romance suffixes to Arabic words to produce hybrid terms was another example of a borrowing. These suffixes are numerous. For instance, the augmentative suffix -\textit{ūn}:\footnote{For more details, see (Corriente, \citealt{PereiraVicente2015}: 230–231).} /ǧurrún/ ‘big jar’ /\textit{ǧarra/} ‘jar’, /raqadún/ ‘sleepyhead’ /\textit{rāqid/} ‘asleep’, the diminutive suffix -\textit{éḻ(a)}: /ḥár+allah/ ‘small neighbourhood’, and the agentive suffix \textit{-áyr}: /ǧawabáyr/ ‘cheeky’, for more Romance suffixes see (\citealt{Corriente1992}: 126–131). 




\subsubsection{Syntax}




Changes in gender agreement also arguably result from contact-induced change: \textit{ʕáyn} ‘eye’, \textit{šáms} ‘sun’, and \textit{nár} ‘fire’ are feminine in Arabic but were occasionally treated as masculine in Andalusi Arabic, as their translation equivalents are in Romance. Likewise, \textit{má} ‘water’ and \textit{dwá} ‘medicine’ are masculine in Arabic but were sometimes considered feminine in Andalusi Arabic, again on a Romance model (Corriente, \citealt{PereiraVicente2015}: 232). This was presumably a case of imposition, where the agents of change were L2 speakers of Andalusi Arabic. 

There are cases of concordless determination constructions in qualifying syntagms following the Romance construction, for instance: \textit{alʕaqd} \textit{ṯānī} ‘the second contract’ instead of more typical \textit{al-ʕaqd} \textit{aṯ-ṯānī}, \textit{albáhar} \textit{maâcut} ‘the freezing sea’ (Corriente, \citealt{PereiraVicente2015}: 186). These examples come from texts written by bilingual Mozarabs from Toledo; since they were either dominant in Andalusi Arabic or had both Andalusi Arabic and Andalusi Romance as L1s, this change must have been either an instance of borrowing or of convergence. 

There are instances of a construction using the analytic genitive with the preposition \textit{min} ‘of’ as well as innovative uses of \textit{li} ‘for’. These are found particularly in late texts with a strong influence from Andalusi Romance, as in \citep{Corriente2012}. As in the previous case, we are dealing here with agents of change who are either dominant in Andalusi Arabic and thus borrowing from Andalusi Romance, or this is an instance of convergence brought about by speakers of both languages as L1s.


\ea\label{bkm:Ref13069050}
Late Andalusi Arabic (Corriente, \citealt{PereiraVicente2015}: 233–234)\\
\ea \gll mudda min ʕām-ayn\\
     period from year-\textsc{du}\\
\glt ‘a two-year period’



\ex\label{ex:key:}
\gll min ʕām\\
     from year\\
\glt ‘one year old’



\ex\label{ex:key:}
\gll naḫruǧ li wild-ī\\
  go\_out.\textsc{ipfv.1sg} to father-\textsc{1sg}\\
\glt ‘I look like my father.’
\z
\z

The examples in (\ref{bkm:Ref13069050}) are clearly calqued on Romance expressions: \textit{un} \textit{periodo} \textit{de} \textit{dos} \textit{años,} \textit{de} \textit{un} \textit{año} and \textit{salgo} \textit{a} \textit{mi} \textit{padre}, respectively. 
 
\subsubsection{Lexicon}




Lexical borrowings from Romance in Andalusi Arabic constitute less than 5\%, according to \citep[142]{Corriente1992}.\footnote{The number of lexical borrowings from Andalusi Arabic into Romance languages spoken in Spain is larger, {according to Corriente, its number is close to 2000, not counting lexical derivation and place names included by the authors who proposed the quantity of 4000 or even 5000.} Many of the terms in question are nowadays obsolete, see (\citealt{Corriente2005}: 203, n. 59). We must not forget that these languages had a different social status during the period of bilingualism, a major element in contact-induced language changes. In such situations, less prestigious languages always receive a larger number of transfers. See the last publication about this borrowings (Corriente, \citealt{PereiraVicente2019}).} 

The most common semantic fields are botanical terms of species endemic to the Iberian Peninsula: \textit{ūliyā} ‘olive’, \textit{amindāl} ‘almand’, \textit{blāṭur} ‘nenuphar, water lily’, \textit{bulmuš} ‘elm tree’, and zoological terms: \textit{burrays} ‘lamb’, \textit{poxóta} ‘whiting’, \textit{buṭrah} ‘mule’, \textit{ṭābaraš} ‘capers’, for more examples see (Corriente, \citealt{PereiraVicente2017}). Other semantic fields are parts of the body: \textit{imlīq}  ‘navel’, \textit{muǧǧa} ‘breast’, family relations: \textit{šuqru} ‘father-in-law’, \textit{šubrīn} ‘nephew’, and household items and technicalities of various professions: \textit{šuqūr} ‘axe’, \textit{šayra} ‘basket’, (Corriente, \citealt{PereiraVicente2015}: 224). 

Some words even adapted to the pattern of broken plural in Andalusi Arabic, for instance \textit{š(u)nyūr} ‘sir’, pl. \textit{šanānīr}, though most used the suffix of the regular plural -\textit{āt.}

\subsection{Contact with Berber}

As with the Arabic–Romance contact situation, lack of information regarding the sociolinguistic status of Berber speakers in al-Andalus in the relevant period makes it difficult to classify the relevant changes according to the types of agentivity involved. That said, since we have no reason to think that significant numbers of native Arabic speakers would have acquired Berber languages as L2s, the changes described here seem most likely to be the result of imposition by L1 Berber speakers.

\subsubsection{Phonology}

Available data is always written sources and it is therefore hard to conjecture on the existence of phonological contact-induced changes. 

The realization k / /ḫ/ has been considered a Zeneti Berber influence \citep[7]{Corriente1981}. For instance: \textit{aḫθar} ‘plus’, \textit{aḫṭubar} ‘October’ (Corriente, \citealt{PereiraVicente2015}: 61).  

The replacement of /l/ with /r/, as in Tarifit Berber, could be another instance of transfer from Berber. Thus, the following spellings in documents written in Latin script could be instances of possible assimilation-induced allophones: \textit{Huaraç,} \textit{Hurad,} \textit{Uarat} / \textit{walad} ‘boy’. The late source where these spellings are found, documents written by Valencian \textit{moriscos} in the second half of the 16th century \citep{Labarta1987}, reveals that this change could have been introduced through contact with the last Berber immigration waves into al-Andalus (13th century). Nonetheless, this trait may not have been generalised in the community speech and could merely consist of an idiolectal variation or even misspelling. 

\subsubsection{Lexicon}

While contact-induced changes in Andalusi Arabic from Berber were initially considered very scarce, more comprehensive analyses of the sources have revealed that changes may not have been so insignificant.\footnote{For instance, linguistic analyses of some sources such as the botanical glossaries written in al-Andalus, have yielded a larger number of Berberisms in Andalusi Arabic, see (Abū l-Ḫayr al-Išbīlī et al. 2004), (Abū l-Ḫayr al-Išbīlī et al. 2007), \citep{Corriente2012}.} In fact, the list compiled by Corriente in 1981 contained 15 Berber loanwords in Andalusi Arabic (\citealt{Corriente1981}: 28–29), the list in his dictionary of 1997 listed 62 \citep[590]{Corriente1997b}, whereas the compilation made by Ferrando the same year included 82, of which 39 corresponded to an unpublished study by Georges Colin and 43 were compiled from the proposals made by various scholars \citep[133]{Ferrando1997}. The most recent list contains 115 Berber loanwords (Corriente, \citealt{PereiraVicente2017}: 1432–1433).

According to Ferrando, these borrowings appear mostly in earlier sources and their number considerably decreases in later sources \citep[140]{Ferrando1997}. This fact could be put down to the social and cultural prestige Andalusi Arabic reached, even contributing to social cohesion and, therefore, linguistic cohesion. Most lexical transfers must have taken place in the early centuries of the existence of al-Andalus, prior to the arrival of new Berber speakers, the Almoravids and the Almohads. For obvious geographical reasons it is quite likely that the Berber-speaking Muslims (already arabized) who reached the Iberian Peninsula with the first Muslim troops came from an area in modern north-western Morocco, the region known as Jbala. Ghomari and Senhaja are the vernacular Berber varieties from this region. These non-Zenati varieties are different from those spoken in the Rif \citep{Kossmann2017}. It is therefore probable that Ghomari and Senhaja were the sources of a good deal of these borrowings, though any attempt at classifying them is hindered by the lack of detailed phonetic or morphological data. 

Semantically, most of these lexical borrowings correspond to phytonyms and zoological terms, socio-political symbols and names of weapons, household goods, clothing and food. The number of Berber loanwords habitually used by the Andalusi population is not easily determined, as many are the names of plants probably existing only in Berber botanical treatises. 

The following are some examples from (Corriente, \citealt{PereiraVicente2017}):\footnote{The Berber origin of some of the lexical borrowings from these lists is only probable, not certain. Due to the characteristics of the sources, written in Arabic or Romance by possibly non Berber-speaking scribes, the available information does not allow in some occasions to go beyond mere working hypotheses. It is also difficult to decide which Berber variety they belong to: Tarifit, Taqbaylit and Tashelhiyt have all been found. Note also that all Arabic items in this section are rendered as transliterations of their Arabic-script orthography, rather than transcription of their (assumed) phonology.} 

\textit{azarūd} / \textit{azrud/aẓrud} ‘sweet clover’, \textit{aṭṭifu} / \textit{ǝṭṭǝf} ‘take him’, \textit{āwurmī} / \textit{awurmi/iwurmi} ‘garden street’, \textit{aɣlāl} / \textit{aɣlal} ‘snails’, \textit{tamaɣra} / \textit{tamǝɣra} (wedding party) ‘banquet’, \textit{zuɣzal} / \textit{ugzal} ({with the agglutination of the preposition} \textit{s} ‘with’) ‘half-pike (Berber weapon)’, \textit{tāqra} / \textit{tagra} (wooden dish to make couscous) ‘terrine’, \textit{Ɂqrūn} / \textit{aɣrum} ‘bread’ (only probable) ‘pancakes cut into squares and eaten with honey’.\footnote{This item exists in Taqbaylit with the meaning ‘unleavened cooked pasta cookie’ \citep{Dallet1982}. The ending -\textit{um} becomes -\textit{un} due to a metanalysis that associates it with the Romance suffix -\textit{on}, which is highly productive in Andalusi Romance.} 

Some of these loans present a chronological problem. The problematic items are those which have an ungeminated /š/ or /q/, phonemes that were transferred to the Berber varieties through contact with Arabic.\footnote{I thank Maarten Kossmann for this and other valuable comments on the section of this work dealing with contact between the Andalusi Arabic and Berber varieties.} These would appear, therefore, to be later loans that arrived with the Berber already arabized or through Moroccan Arabic (see \textit{supra}), for instance: \textit{išir} / \textit{iššir} (only probable) ‘boy’, \textit{finniš} / \textit{afǝnniš} ‘snub-nosed’ (only probable) ‘mule’,\footnote{In Moroccan Arabic \textit{fǝnnīš/fənnūš}, (\citealt{Prémare1998}: Vol 10, 167).}  \textit{barqī} / \textit{abǝrqi} ‘slap’.\footnote{According to Prémare (\citealt{Prémare1993}: Vol 1, 5), the Moroccan Arabic word \textit{ābā{\R}ǝq} ‘slap’ is also a loanword of Berber origin.}

Some of these loans do not appear in modern dictionaries of Berber varieties, such as \textit{arɣīs} / \textit{arɣis} ‘barberry,’\footnote{The Berber origin of this item has nevertheless been affirmed by Colin and Ferrando, since this circumstance is clear thanks to the data provided by Ibn al-Bayṭār (\citealt{Ferrando1997}: 110–111). It is documented in Moroccan Arabic, \textit{ārɣīs} ‘barberry’, (\citealt{Prémare1995}: Vol 5, 151) and in Spanish it has become \textit{alargue} and \textit{alguese}, and in Portuguese \textit{largis} (Corriente, Pereira \& Vicente Forthcoming). A fall into disuse in the SL is maybe the reason of its absence from the current dictionaries.}\textit{āðiqal} / \textit{adigal} ‘watermelon’, \textit{maqaqūn} / \textit{amaka} ‘stallion’.\footnote{The last two lexical borrowings are documented in the Andalusi source \textit{kitābu} \textit{ʕumdati} \textit{ṭ-ṭabīb}, by Abu l-Ḫayr al-ʔIšbīlī, a botanist of the eleventh century/ However, their Berber origin is quite doubtful for M. Kossmann (personal communication).} 

In some cases we have loans that come from Vulgar Latin to Andalusi Arabic via Berber, for instance: \textit{fullūs} / \textit{afǝllus} (Berber) / \textit{pullus} (Vulgar Latin) ‘chicken’, \textit{bāqya} ‘large clay dish’ / \textit{tabaqit} / \textit{θabǝqqišθ} (Tarifit\footnote{See Ibáñez \citep[272]{Ibáñez1949} whose transcription is \textit{zabeqqixz}.}  ‘great dish of superior quality’) / \textit{bacchia} (Vulgar Latin ‘goblet, water jug’), \textit{hirkāsa} / \textit{arkasǝn} (Taqbaylit) / \textit{arkas}, \textit{ahǝrkus} (Tarifit) / \textit{calcĕus} (Vulgar Latin; but not certainly) ‘rustic leather shoe’, \textit{tirfās} / \textit{tǝrfas} (Berber) / \textit{tuferas} (Low Latin) ‘truffles’, \textit{zabzīn} / \textit{zabazin} (Berber, {with the agglutination of the preposition} \textit{s} ‘with’) / \textit{pisellum} (Low Latin, diminutive of \textit{pisum} ‘pea’) ‘low-quality couscous’.\footnote{The word exists in Moroccan Arabic, \textit{ābāzīn} (\citealt{Prémare1993}: Vol 1, 5), and in Taqbaylit Berber \textit{tabazint} (augmentative of \textit{abazin}).}  These transfers are very likely to have first taken place in North Africa (the northern part of present-day Morocco) since we know that some variety of Vulgar Latin was in contact there with the Berber variety of the region before the arrival of Muslim troops (cf. Heath, this volume). The Berber-speaking Andalusians would have then later transferred these items to Andalusi Arabic.\footnote{A number of these Berber loans have then gone on to reach the Romance languages through Andalusi Arabic. The most recent list includes forty of these borrowings in Romance languages, see (Corriente, \citealt{PereiraVicente2019}).}  

Some of these lexical borrowings have certain characteristics that demonstrate greater integration than others in Andalusi Arabic. 

\begin{itemize}
\item Various morphophonemic adaptations.


Phonemic adaptation to Arabic, although this may simply be a problem of orthography, since the Arabic script lacks a means of representing the Berber phonemes /g/ and /ẓ/. Either way, /g/ is represented as /k/, /q/ or /ǧ/: \textit{akzal/aqzal} / \textit{agzal} ‘pike (characteristic weapon of the Berbers)’,\footnote{Andalusi Arabic seems to have had a diminutive form of this item: \textit{tagzalt} (modern dictionaries give the diminutive \textit{tagǝzzalt} ‘small stick’; \citep{Taïfi1991}. This could then be the source of Castilian \textit{tragacete} and Portuguese \textit{tragazeite} ‘dart’ (Corriente, Pereira \& Vicente Forthcoming).}  \textit{āðiqal} / \textit{adigal} ‘watermelon’, \textit{arǧān} / \textit{argan} ‘argan tree’, \textit{qillīd} / \textit{agǝllid} ‘Berber prince’. /ẓ/ is represented as /z/: \textit{zawzana} / \textit{aẓiẓun} ‘mutism’, \textit{lazāz} / \textit{aẓẓaẓ} ‘werewolf’. 



Elimination of typically Berber morphemes. For instance, the loss of prefix \textit{a}{}- of masculine names: \textit{bāzīn} / \textit{abazin} ‘a dish of couscous, meat and vegetables’, \textit{dād} \textit{abyaḍ} / \textit{addad} ‘white chameleon’, \textit{mizwār} ‘manager, commander’ / \textit{amǝzwaru} ‘first’, \textit{finniš} / \textit{afǝnniš} (snub-nosed) ‘mule’, \textit{mazad} / \textit{amzad} ‘Quranic school’. Likewise the loss of prefix and suffix \textit{tv-…-t} of feminine nouns: \textit{zaɣnaz} / \textit{tisǝɣnǝst} (Tarifit) ‘brooch, buckle’,\footnote{This is a noun of instrument derived from the verb \textit{ɣnǝs} ‘to tie with a brooch’. Corriente makes it derive from \textit{asǝgnǝs} ‘needle’, see (Corriente, Pereira \& Vicente Forthcoming), but the phoneme /ɣ/ makes the first option more likely (personal communication by Maarten Kossmann).} \textit{muzūra} / \textit{tamzurt} (Tarifit and Taqbaylit) ‘horse braid’, \textit{sarɣant} / \textit{tasǝrɣint} ‘root of the orpine plant’; as well as elimination of prefix \textit{t-}: \textit{abɣā} / \textit{tab\.ga} ‘wild bramble’. 



\item Another process for the integration of lexical borrowing involves fitting Berber words to Arabic patterns, for instance: \textit{zawzana} / \textit{aẓiẓun} ‘mutism’ (with the Arabic pattern CawCaCa), \textit{harkama} / \textit{urkimen} ‘tripe stew’ (with the Arabic pattern CaCCaCa), \textit{hirkāsa} / Taqbaylit \textit{arkasǝn} or Tarifit \textit{arkas}, \textit{ahǝrkus} (Tarifit) ‘rustic leather shoe’ (with the Arabic pattern CiCāCa).
\end{itemize}


\section{Conclusion}


Andalusi Arabic developed in the Iberian Peninsula through intra-Arabic levelling and contact with two other language types: Romance and Berber. This situation spanned a long period of time and resulted in a good deal of contact-induced change. 

Initially the L2 of most of the population, after a two-century gestation process, Andalusi Arabic gradually became the dominant language, overtaking the Romance dialect spoken by the local population. The main reason was the growing social prestige attached to Arabic in an Islamic society, in contrast to the lower social status of Andalusi Romance, which first became an L2, before the bilingualism situation eventually disappeared. This contact situation resulted in numerous contact-induced changes in all areas of grammar, but it is often complex to diagnose what type of transfer took place in such an ancient contact situation. 

Concerning Berber varieties, modern historiography reveals that the interaction between Arabic-speaking and Berber-speaking populations on both sides of the Strait of Gibraltar facilitated lasting language contact. The role of Berber in the language development of al-Andalus, however, has not yet been analysed in depth. The nature of the available data is such that lexical borrowings are the only transfers that have been well described at present. 

Future research would be particularly desirable with regard to contact-induced changes in Andalusi Arabic due to the presence of Berber varieties in the Iberian Peninsula. This should involve collaboration between scholars of Berber and of Arabic. 

\section*{Further reading}

\citep{Corriente1997a} it{ contains the linguistic analysis of Andalusian strophic poetry.}

 \citep{Corriente2005} offers valuable information concerning the impact of Andalusi Arabic on Ibero-Romance. 

(Corriente, \citealt{PereiraVicente2015}) is the most up-to-date book-length description of Andalusi Arabic grammar. It contains a section dealing with transfer from Romance and Berber. 

\citep{Ferrando1997} offers an etymological description of some Berber loanwords in Andalusi Arabic. 

\citep{Vicente2010} this work describes the Andalusi influence on Northern Morocco. 

\section*{Abbreviations}

\begin{tabularx}{.45\textwidth}{lQ}
\textsc{1, 2, 3} & 1st, 2nd, 3rd person \\
\textsc{du} & dual \\
\textsc{ipfv} & imperfective \\
L1 & first language \\
L2 & second language \\
\end{tabularx}
\begin{tabularx}{.45\textwidth}{lQ}
\textsc{m} & masculine \\
RL & recipient language \\
\textsc{sg} & singular \\
SL & source language \\
\end{tabularx}
 
\sloppy\printbibliography[heading=subbibliography,notkeyword=this]
\end{document}
