\documentclass[output=paper]{langsci/langscibook} 
\author{Ergin Öpengin\affiliation{University of Kurdistan-Hewlêr}}
\title{Kurdish}
% \keywords{} 
\abstract{This chapter provides an overview of the influence of Arabic on Kurdish, especially on its Northern and Central varieties spoken mainly in Turkey–Syria–Iraq and Iraq–Iran, respectively. It summarizes and critically assesses the limited research on the contact-induced changes in the phonology and syntax of Kurdish, and proposes several new dimensions in the morphology and syntax, in addition to providing a first treatment of lexical convergence in Kurdish through borrowings from Arabic.}

\begin{document}
\maketitle

\section{Kurdish and its speech community} 

Kurdish is a Northwestern \ili{Iranian} language spoken by 25 to 30 million speakers in a contiguous area of western Iran, northern Iraq, eastern Turkey and northeastern Syria. There are also scattered enclaves of Kurdish speakers in central \isi{Anatolia}, the Caucasus, northeastern Iran (Khorasan province) and {Central} Asia, with a large European \isi{diaspora} population. The three major varieties of Kurdish are: (i) Southern Kurdish, spoken under various names near the city of Kermanshah in Iran and across the border in Iraq; (ii) \ili{Central Kurdish} (also known as \ili{Sorani}), one of the official languages of the autonomous Kurdish region in Iraq, also spoken by a large population in western Iran along the Iraqi border; (iii) \ili{Northern Kurdish} (also known as \ili{Kurmanji}), spoken by the \isi{Kurds} of Turkey, Syria and the northwestern perimeter of Iraq, in the province of West Azerbaijan in northwestern Iran and in pockets in the west of Armenia (cf. \citealt{HaigÖpengin2014} for a discussion on defining “Kurdish”). Of these three, the largest group in terms of speaker numbers is \ili{Northern Kurdish}. The Kurdish population in respective states is difficult to reliably determine since none of the sovereign countries make the relevant census information available. \tabref{tab:opengin:1} provides some cautious estimates based on various sources (especially \citealt{Sirkeci2005,ZeyneloğluEtAl2016}; and Ethnologue).\footnote{See \url{https://www.ethnologue.com/language/kur} (accessed 31/01/2020; \citealt{Ethnologue}).}\textsuperscript{,}\footnote{The population figures should not be taken as equivalent to “number of speakers”, since especially in Turkey a significant portion of the Kurdish population grow up with no or very limited knowledge of Kurdish (cf. \citealt{Öpengin2012,ZeyneloğluEtAl2016}).} 

\begin{table}
\begin{tabular}{lr}
\lsptoprule
Country & Population size\\\midrule
{Turkey} & c. 15,000,000\\
{Iraq}   & c. 6,000,000\\
{Iran}   & c. 8,000,000\\
{Syria}  & c. 2,000,000\\
\lspbottomrule
\end{tabular}
\caption{Estimates of Kurdish population numbers}
\label{tab:opengin:1}
\end{table}


\section{The history of Kurdish–Arabic contact} 

In\isi{formation} about the pre-Islamic history of the \isi{Kurds} and their language is scarce. According to early Islamic sources, at the time of the Islamic conquest of the Near East (Upper Mesopotamia, Iran, and Armenia) in the seventh century (\citealt{BoisEtAl2012}: 451), the communities designated with the term \textit{Kurd} were already living in most of the present-day Kurdish-inhabited areas, namely from \ili{Mosul} to the north of Lake Van, and from Hamadan to the Jazira region situated around the intersection of present-day Syria, Iraq and Turkey \citep[111]{James2007}. The \isi{Kurds} have thus been living in contact with various \ili{Aramaic}-speaking Christian and Jewish communities as well as \ili{Arabic}-speaking communities since at least the early Islamic period, though the contact of \ili{Iranian}-speaking populations with \ili{Aramaic} dates back to the fifth century BCE (cf. \citealt[69]{Utas2005}, citing also \citealt{Folmer1995} and \citealt{Kent1953}). Kurdish differs from other \ili{Iranian} languages such as \ili{Persian} in sharing the same or close geographical spaces with \ili{Arabic}-speaking populations, especially in Upper Mesopotamia. The historical socio-cultural contact between Kurdish and \ili{Arabic}-speaking communities requires a more refined treatment than is currently possible, but there are a number of medieval \ili{Arabic} sources which attest to the interaction and mobility of Kurdish and \ili{Arabic} communities in some regions (e.g. Erbil, \ili{Mosul}), as well as \isi{language shift} of some Kurdish communities to \ili{Arabic} and vice versa (cf. \citealt{BoisEtAl2012}: 449, 452, 456; \citealt{James2007}: 115–120). 

Given the unquestionably \isi{prestigious} status of \ili{Arabic} in administration and sciences in the Islamicized Near East, consolidated especially under Abbasid rule (which included most of the Kurdish-inhabited areas), Kurdish was heavily dominated by \ili{Arabic}. Even in several of the important medieval Kurdish dynasties such as that of the Marwānids (10th–11th centuries), \ili{Arabic} enjoyed the high status of being the administrative and literary language (cf. \citealt{James2007}: 112), since the coins bore \ili{Arabic} script, while \textit{qaṣīda} reading ceremonies or contests would feature primarily \ili{Arabic}, but to a limited extent also \ili{Persian} pieces (\citealt{Ripper2012}: 507–528). With the conquest of the Kurdish-inhabited regions by \ili{Turkic} peoples and Mongols from tenth century onwards, which led also to the final overthrowing of the Abbasid state in 1248 by the Mongols, the \ili{Arabic}-speaking populations may have started to diminish and retreat. Although at this stage \ili{Persian} attained a firm status as the literary language in the Islamic East (\citealt{Perry2012}: 73), \ili{Arabic} preserved its higher status in administration and, later on, especially in education, well into the end of the nineteenth century. Thus, Kurdish developed a literary tradition only starting from the sixteenth century, but its limited usage was largely restricted to writing verse throughout the following several centuries. The literature in this period is heavily dominated by the vocabulary and literary formulas and metaphors of the two dominant languages, \ili{Arabic} and \ili{Persian} (cf. \citealt{Öpenginforthcoming}). 

In the early twentieth century, with the dissolution of the Ottoman Empire, Kurdish in Iraq and Syria again came into primary direct contact with \ili{Arabic}. In Iraq, up until 1991, with the establishment of a Kurdish autonomous region, the language configuration was one in which \ili{Arabic} was the \isi{prestigious} language of higher domains. Not being in possession of any official status, the \isi{Kurds} in Syria have been in a highly asymmetric language-contact situation with \ili{Arabic}. In Turkey, especially in \ili{Mardin} and \ili{Siirt} provinces, \isi{Kurds} have been in contact with \ili{Arabic}-speaking communities, but as the lingua franca of the communities of cultural–historical Kurdistan (cf. \citealt{Edwards1851}: 121), Kurdish must have been the dominant language of interaction between these communities (cf. \citealt{Lentin2012}), and it is indeed possible to observe important influences from Kurdish on the local \ili{Arabic} dialects (cf. \citealt{Jastrow2011Kurdish} and §\ref{bkm:Ref14688184} below.). 

As a result of these differing degrees and modalities of contact with \ili{Arabic}, the influence of \ili{Arabic} should be viewed as consisting of at least two layers, and viewed separately for different country contexts where Kurdish is spoken. Of the two layers, there should be assumed a deeper contact influence, shared in larger portions of Kurdish-speaking areas, dating to before the twentieth century; and a more shallow layer that is the result of the more recent societal \isi{bilingualism} in Iraq and Syria. Likewise, while in Syria and Iraq the \ili{Arabic} influence on Kurdish continues, this influence is largely replaced by influence from the dominant state languages in Turkey and Iran.~Naturally, the intensity of \ili{Arabic} influence on Kurdish shows a great deal of variation across Kurdish varieties and dialects within varieties. Accordingly, the historically deeper-layer \ili{Arabic} influence on Kurdish is characterized by its being restricted mostly to lexicon and being shared in the majority of Kurdish dialects. This has been the result of borrowing under \isi{recipient-language} agentivity in the sense of Van Coetsem (\citeyear{VanCoetsem1988,VanCoetsem2000}). On the other hand, the relatively advanced \ili{Arabic} influence on the Kurdish spoken in the historical Jazira region (including \ili{Mosul}, northeast Syria, and \ili{Mardin} province in southeast Turkey), as well as the more recent \ili{Arabic} influence on the Kurdish spoken in Syria, but also – albeit more restrictedly – in Iraq, concerns also grammatical constructions and at least some of that contact influence could be due to \isi{imposition} under \isi{source-language} agentivity.    

\section{Contact-induced changes in Kurdish} 

\subsection{\label{bkm:Ref14688184}Phonology}

The consonant inventory of \ili{Kurmanji} is given in \tabref{tab:opengin:2}.\footnote{Kurdish data are transcribed in the standard Kurdish \ili{Latin} alphabet with some additions for emphatics and \isi{pharyngeals}, mostly consonant with the Library of Congress approach for the romanization of Kurdish: \url{https://www.loc.gov/catdir/cpso/romanization/kurdish.pdf} (accessed 31/01/2020).} 

\begin{table}
\begin{tabularx}{\textwidth}{lXXXXXXXX} 
\lsptoprule
& \rotatebox{66}{Bilabial} & \rotatebox{66}{Labio-dental} & \rotatebox{66}{Alveolar} & \rotatebox{66}{Palatal} & \rotatebox{66}{Velar} & \rotatebox{66}{Uvular} & \rotatebox{66}{Pharyngeal} & \rotatebox{66}{Glottal}\\\midrule
{Plosive} & {p’ p b} & & {t’ t d} & & {k’ k  g} & {q} &  & {’}\\
{Fricative} &  & {f v} & {s z} & {ş j} & {x ẍ} &  & {ḥ  ʿ} & {h}\\
{Affricate} &  & & & {ç’  ç  c} &  &  &  & \\
{Nasal} & {m} & & {n} &  &  &  &  & \\
{Trill} &  & & {\={r}} &  &  &  &  & \\
{Flap} &  & & {r} &  &  &  &  & \\
{Approximant} & {w} &  &  & {y} &  &  &  \\
{Lateral} &  & & {l} &  &  &  &  & \\
\lspbottomrule
\end{tabularx}
\caption{\label{bkm:Ref14688948}\label{tab:opengin:2}Consonant phonemes in Kurmanji}
\end{table}

In cells of doublets/triplets, the voiceless phonemes come first. The apostrophe on plosive and fricative phonemes indicates aspiration, which marks a phonemic distinction in \ili{Kurmanji}. In addition to these consonants with indisputable phonemic status, there are the so-called \isi{emphatic} or \isi{pharyngealized} variants of the obstruents /p, b, t, d, s, z/. These variants are transcribed in the text with a dot beneath the characters. 

The consonant inventory of \ili{Sorani} is basically identical with \tabref{tab:opengin:2}, except: (i) it does not have unaspirated stop phonemes; and (ii) it has velar nasal and velarized lateral phonemes \citep[27]{Öpengin2016}. 

\ili{Arabic} (or more generally \ili{Semitic}) influence on the phonology of Kurdish is most clearly observed in the presence of the two \isi{pharyngeal} phonemes \textit{ḥ} [ħ] and \textit{ʿ} [ʕ] (cf. \citealt{Kahn1976,Haig2007,Anonbyforthcoming,Barryforthcoming}), as well as the series of \isi{emphatic} obstruants \textit{ṭ,} \textit{ḍ,} \textit{ṣ,} and \textit{ẓ} (Haig \& Öpengin 2018), respectively. The precise \ili{Semitic} \isi{source language} for these sounds cannot be determined, since Kurdish (or rather its ancestor languages) must have been in close contact with various \ili{Semitic} languages for more than two millennia \citep[69]{Utas2005}. However, these phonemes set the consonant inventory of Kurdish clearly apart from other West \ili{Iranian} languages such as \ili{Persian}, with the only other West \ili{Iranian} languages possessing both \isi{pharyngeals} and \isi{emphatic consonants} being \ili{Zazaki}, and the \ili{Kumzari} language spoken mainly in Oman \citep{Anonbyforthcoming}. In what follows, I illustrate the presence and interactions of the \isi{pharyngeal} and \isi{emphatic consonants} in Kurdish, and provide a brief discussion of their paths of development.\footnote{The \ili{Kurmanji} lexical items presented in this section are based on my native-speaker knowledge of the Şemdînan (Şemdinli) dialect, and my knowledge of \ili{Kurmanji}-internal dialectal variation, drawing also on \citep{Chyet2003}, \citep{ÖpenginHaig2014}, and the  Manchester Database of Kurdish Dialects presented in \citet{MatrasKoontz-Garboden2016}. The \ili{Sorani} lexical items are from \citet{Öpengin2016} and the popular press.}  

The \isi{pharyngeal} phonemes are found in varying degrees in both \ili{Central Kurdish} and \ili{Northern Kurdish}. They are retained in most of the \ili{Arabic} \isi{loanwords} originally bearing them, a list of which is given in \tabref{tab:opengin:3}.\footnote{Note that all through the \isi{article}, unless stated otherwise, the \ili{Arabic} data represents \ili{Classical} \ili{Arabic}, giving an approximation of the ultimate \ili{Arabic} etyma of the items without necessarily implying that these are the immediate source of the Kurdish items (as they may have been borrowed from local \ili{Arabic} dialects as well as through the intermediary languages such as \ili{Persian} or \ili{Ottoman}). Furthermore, the glosses in tables are for Kurdish items, as sometimes the meanings of the \ili{Arabic} etyma are not completely identical.} 

\begin{table}
\begin{tabular}{llll}
\lsptoprule
{Arabic} & \ili{Northern Kurdish} & \ili{Central Kurdish} & {Gloss}\\\midrule
\textit{ʕarab}     & \textit{ʿereb}        & \textit{ʿereb}                    & ‘Arab’\\
\textit{maʕlūm}    & \textit{meʿlûm}       & \textit{meʿlûm}    & ‘evident’\\
\textit{ʕadāla(t)} & \textit{ʿedalet}      & \textit{ʿedaḷet}                  & ‘justice’\\
\textit{ṭābiʕ}     & \textit{tabiʿ/ṭabiʿ}  & \textit{ṭabiʿ}                    & ‘dependent’\\
\textit{maḥall}    & \textit{miḥele}       & \textit{meḥel}                    & ‘neighborhood’\\
\textit{maḥšar}    & \textit{meḥşer}       & \textit{meḥşer}                   & ‘resurrection (day)’\\
\textit{ḥākim}     & \textit{ḥakim}        & \textit{ḥakim}                    & ‘judge, governor’\\
\textit{ḥammām}    & \textit{ḥemam}        & \textit{ḥemam}                    & ‘bath’\\
\textit{baḥr}      & \textit{beḥr}         & \textit{beḥr}                     & ‘sea’\\
\lspbottomrule
\end{tabular}
\caption{\label{bkm:Ref14691447}Loanwords with retained pharyngeals in Kurdish}
\label{tab:opengin:3}
\end{table}

Some \isi{loanwords} with original \isi{pharyngeals} are reanalysed as containing their non-\isi{pharyngeal} counterparts. Such is the word \textit{haq} from \ili{Arabic} \textit{ḥaqq} ‘right’, or the \ili{Arabic} word \textit{ṭaʕm} ‘taste’ that is seen in eastern dialects of \ili{Northern Kurdish} and in \ili{Central Kurdish} without the voiced \isi{pharyngeal} as \textit{ṭam} and \textit{tam}, respectively. 

Furthermore, an original \isi{pharyngeal} in a \isi{loanword} may be substituted with the alternative \isi{pharyngeal} sound, so, for example, the voiced \isi{pharyngeal} of the \ili{Arabic} \textit{ṭamaʕ} ‘greed’ may be realized as either of the \isi{pharyngeals} in different Kurdish dialects. Such indeterminate or alternative use of \isi{pharyngeals} may exist within a single dialect (cf. \citealt{Kahn1976}: 25). For instance, in the Mukri dialect of \ili{Central Kurdish}, (\citealt{Öpengin2016}: 41–42) the following \ili{Arabic}-origin words can be found in both of the form pairs: \textit{saʿib} {\textasciitilde} \textit{saḥib} ‘owner’, \textit{ʿerz} {\textasciitilde} \textit{ḥerz} ‘honour’, \textit{cemaʿet} {\textasciitilde} \textit{cemaḥet} ‘community’. 

Finally, a \isi{pharyngeal} may develop in \isi{loanwords} that have no \isi{pharyngeal} in the \isi{source language}. Thus, in most of \ili{Northern Kurdish} the \ili{Arabic} word \textit{ʔarḍ} ‘earth’ appears with a non-etymological \isi{pharyngeal} as \textit{ʿerd}, while the \ili{Arabic} word \textit{ǧāhil} ‘naïve, young’ is seen with a \isi{pharyngeal} as \textit{caḥêl} (but also \textit{cahil}).    

Although the \isi{pharyngeals} in Kurdish occur mostly in \ili{Arabic} \isi{loanwords}, they have expanded also into inherited native \ili{Iranian} lexicon, especially in \ili{Northern Kurdish}. However, unlike in \ili{Arabic} \isi{loanwords}, fluctuation between \isi{pharyngeal} and non-\isi{pharyngeal} uses of such words among the dialects (sometimes in immediate geographic proximity) is readily apparent. \tabref{tab:opengin:4} presents some native \ili{Iranian} words of this kind. Where relevant, the non-\isi{pharyngeal} forms are also noted, while \ili{Persian} cognates are included for comparison.  

\begin{table}
\begin{tabular}{llll}
\lsptoprule
{Persian} & {\ili{Northern Kurdish}} & {\ili{Central Kurdish}} & {Gloss}\\\midrule
\textit{abr} & \textit{ʿewr} & \textit{hewr} & ‘cloud’\\
\textit{zabān} & \textit{ʿezman {\textasciitilde} ziman} & \textit{ziman} & ‘language’\\
\textit{āsemān} & \textit{ʿesman} & \textit{asman {\textasciitilde} hasman} & ‘sky’\\
\textit{ḫošk} & \textit{ḥişk {\textasciitilde} hişk}  & \textit{wişk} & ‘dry, hard’\\
\textit{haft} & \textit{ḥeft {\textasciitilde} heft} & \textit{ḥewt}  & ‘seven’\\
\textit{hašt} & \textit{ḥeşt {\textasciitilde} heşt} & \textit{heşt} & ‘eight’\\
\textit{bahašt} & \textit{biḥeşt {\textasciitilde} bihişt} & \textit{beheşt} & ‘paradise’\\
\lspbottomrule
\end{tabular}
\caption{\label{bkm:Ref14705119}Pharyngeal sounds in native Iranian lexical items}
\label{tab:opengin:4}
\end{table}

More striking, however, is the emergence of a voiced \isi{pharyngeal} in a subset of words with similar structure in the northern dialects of \ili{Northern Kurdish} that are geographically farthest from direct \ili{Arabic}/\ili{Semitic} contact but close to Caucasian languages which also possess \isi{pharyngeals}. Thus, the native words such as \textit{masî} ‘fish’, \textit{çav} ‘eye’, \textit{mar} ‘snake’ (in \ili{Central Kurdish} and in central and southern dialects of \ili{Northern Kurdish}) appear in the northern dialects of \ili{Northern Kurdish} with a \isi{pharyngeal}, as \textit{meʿsî,} \textit{çeʿv,} \textit{meʿr}. These are obviously the result of language-internal processes, though nested in an initial introduction of the phonemes into the language via contact with either \ili{Arabic} or Caucasian languages, or both.

As for their distribution, the \isi{pharyngeal} phonemes are most robustly present in the central areas of the {Northern} and {Central Kurdish} speech zones. Their presence in \ili{Arabic} \isi{loanwords} is weakened towards the extreme northern and southern peripheries in heavy contact with \ili{Turkish} and \ili{Persian} (cf. Map 1.27 in the Manchester Kurdish Database, which illustrates such weakening of \isi{pharyngeals} at the peripheries through the distribution of the \ili{Arabic} \isi{loanword} \textit{ḥeywan} ‘animal’).\footnote{\begin{flushleft}See \url{http://kurdish.humanities.manchester.ac.uk/pharyngeal-retentionloss-animal/} (accessed 31/01/2020).\end{flushleft}}

We turn now to the series of \isi{emphatic} (\isi{pharyngealized}) obstruents \textit{ṭ, ḍ} and \textit{ṣ, ẓ}. \tabref{tab:opengin:5} gives a list of \ili{Arabic} \isi{loanwords} in which the original \isi{emphatic consonant} is retained in Kurdish. 

\begin{table}
\begin{tabular}{lll}
\lsptoprule
\ili{Arabic} & \ili{Northern Kurdish} & Gloss\\\midrule
\textit{ṭaʕm}   & \textit{ṭam} {\textasciitilde} \textit{ṭeʿm} & ‘taste’\\ 
\textit{ṭāʔir}  & \textit{ṭeyr}                               & ‘bird’\\ 
\textit{baṭṭāl} & \textit{beṭal}                              & ‘empty, cancel’\\ 
\textit{ð̣ulm}  & \textit{ẓulm}                               & ‘oppression’\\ 
\textit{ḍābit}  & \textit{ẓabit}                              & ‘clerk’\\ 
\textit{ṣūfī}   & \textit{ṣofî}                               & ‘devotee, Sufi’\\ 
\textit{ṣāfī}   & \textit{ṣaf} {\textasciitilde} \textit{ṣafî}                           & ‘clear’\\
\lspbottomrule
\end{tabular}
\caption{Arabic loanwords with emphatic consonants in Kurdish\label{tab:opengin:5}}
\end{table}

In the deeper-layer \isi{loanwords}, the \ili{Arabic} interdental and voiced alveolar emphatics are \isi{merged} into the voiced \isi{emphatic} alveolar \isi{phoneme} \textit{ẓ} in Kurdish. But in present-day Iraqi and {Syrian} Kurdish speech, especially those speakers with formal education may also pronounce the interdental \isi{phoneme}, especially in the case of nonce borrowings and code-mixing.  

On the other hand, quite a number of \ili{Arabic} \isi{loanwords} are pronounced without their original \isi{emphatic consonants}, and thus reanalysed as the corresponding plain phonemes (similarly to \ili{Persian}), as in the items in \tabref{tab:opengin:6}. 

\begin{table}
\begin{tabular}{llll}
\lsptoprule
{Arabic} & {\ili{Northern Kurdish}} & {\ili{Central Kurdish}} & {Gloss}\\\midrule
\textit{ḫāṭir}  & \textit{xatir}                           & \textit{xatir}  & ‘mind’\\
\textit{ṭaraf}  & \textit{teref}                           & \textit{teref}  & ‘side, direction’\\
\textit{šayṭān} & \textit{şeytan}                          & \textit{şeytan} & ‘devil’\\
\textit{ḍaʕīf}  & \textit{zeʿîf}                           & \textit{zeʿîf}  & ‘weak’\\
\textit{ḥāḍir}  & \textit{ḥazir}                           & \textit{ḥazir}  & ‘ready’  \\
\textit{qaṣṣāb} & \textit{qesab}                           & \textit{qesab}  & ‘butcher’\\
\textit{fasīḥ}  & \textit{fesîḥ}                           & \textit{fesîḥ}  & ‘clear’\\
\textit{ṣabr}   & \textit{sebr {\textasciitilde} ṣebr}       & \textit{sebr}   & ‘patience’\\
\lspbottomrule
\end{tabular}
\caption{\label{bkm:Ref14707500}Arabic loanwords with lost emphatics in Kurdish\label{tab:opengin:6}}
\end{table}

On the reverse side, some \ili{Arabic} \isi{loanwords} with no original \isi{emphatic consonants} are pronounced with \isi{emphatic consonants} in Kurdish, such as \textit{ẓełał} ({\textasciitilde} \textit{ẓelal} and \textit{zelal}) from \ili{Arabic} \textit{zulāl} ‘clear’ (dialectal \textit{zalāl}), or \textit{ẓelam} ‘man’ from \ili{Syrian} \ili{Arabic} \textit{zalame}.    

Finally, just as with the \isi{pharyngeal} consonants, \isi{emphatic} sounds also appear in inherited native \ili{Iranian} words, as illustrated in Table \ref{tab:opengin:7}. 

\begin{table}
\begin{tabular}{lll}
\lsptoprule
\ili{Northern Kurdish} & \ili{Central Kurdish} & Gloss\\\midrule
\textit{meẓin} & - & ‘big’\\
\textit{ẓiman} & \textit{ziman} & ‘language’\\
\textit{ẓik} \textit{{\textasciitilde} zik} & \textit{zig} & ‘stomach’\\
\textit{aẓad}  & \textit{azad} & ‘free’\\
\textit{ẓava} & \textit{zawa} & ‘groom’\\
\textit{beẓîn} & - & ‘to run’\\
\textit{peẓ} & \textit{pez} & ‘sheep’\\
\textit{ṣal} & \textit{ṣał} & ‘year’\\
\textit{ṣed} \textit{{\textasciitilde} sed} & \textit{ṣed} & ‘hundred’\\
\textit{ṣe} & \textit{ṣeg} & ‘dog’\\
\textit{beṣ} \textit{{\textasciitilde} bes} & \textit{bes} & ‘enough’\\
\textit{ṣawa} & \textit{sawa} & ‘very young, newborn’\\
\textit{ṣotin} & \textit{sûtan} & ‘to burn’\\
\textit{ṣiṣt} & - & ‘loose’\\
\textit{ṭarî} & \textit{tarîk} & ‘dark’\\
\textit{ṭezî} & \textit{tezî} & ‘cold’\\
\textit{ṭeng} & \textit{teng} & ‘narrow’\\
\textit{ṭerm} & \textit{term} & ‘dead body’\\
\textit{ṭirş} & \textit{tirş} & ‘sour’\\
\textit{ḍaṣ(ik)} & \textit{das} & ‘sickle’\\
\textit{ḍiṙî} & - & ‘blackberry bush’\\
\lspbottomrule
\end{tabular}
\caption{Emphatic consonants in native Iranian lexical items}
\label{tab:opengin:7}
\end{table}

Of the \isi{emphatic} obstruents, the fricative pair (\textit{ṣ,} \textit{ẓ}) are found both in {Northern} and {Central Kurdish} (though less often in the latter), while the stops (\textit{ṭ,} \textit{ḍ}) are found only in \ili{Northern Kurdish}, with the voiced counterpart being extremely rare. The fact that the voiceless \isi{emphatic} stop is widespread only in \ili{Northern Kurdish} most probably has to do with the presence of two series of aspirated and unaspirated voiceless stops in the language (cf. \tabref{tab:opengin:2}). The unaspirated stops are probably intermediary in the development of emphatics. This is further reinforced by the fact that in \ili{Northern Kurdish} the bilabial voiceless stop \textit{p} also has an \isi{emphatic} version, as in the native words \textit{ṗeẓ} ‘sheep’ and \textit{ṗenîr} ‘cheese’ (in some dialects; cf. \citealt{Kahn1976}: 27). Within \ili{Northern Kurdish}, they are found in more southerly dialects, and are noted to be particularly frequent in both the Kurdish and \ili{Neo-Aramaic} of Duhok and Hakkari provinces  \citep[329]{Blau1989}. They tend to be less present moving northwards (Erzurum–Kars) while MacKenzie (\citeyear[43]{MacKenzie1961}) notes that they are altogether absent in the Yerevan dialect. This distribution is of course consistent with a language-contact scenario, in the sense that in the northern dialects away from \ili{Semitic} influence the language either did not develop emphatics or lost them as a result of contact with and \isi{bilingualism} in \ili{Armenian}, \ili{Turkic} and Caucasian languages that do not possess such emphatics.   

Given the shallow history of written Kurdish, it is not possible to determine the historical period of the introduction of the emphatics and \isi{pharyngeals} into the language. However, they are found abundantly even in the earliest Kurdish texts, especially in the \ili{Arabic} component, but also in inherited lexical items, such as \textit{ṣal} ‘year’, \textit{ṣar} ‘cold’, \textit{ṣed} ‘hundred’, \textit{meẓin} ‘big’, \textit{ḥemyan} ‘all of them’ (items taken from \textit{Şêxê} \textit{Senʿaniyan} by the early seventeenth-century poet Feqiyê Teyran, cf. \citealt{Teyran2011}). 

Three studies have treated the \isi{pharyngeals} and emphatics in Kurdish, namely \citet{Kahn1976}, \citet{Anonbyforthcoming} and \citet{Barryforthcoming}. \citet{Barryforthcoming} suggests that the \isi{pharyngeal} sounds (including emphatics) in Kurdish are the result of contact influence from \ili{Arabic} with a phonetic basis. The phonetic basis consists in the recategorization of vowels and the \textit{h} sound within syllables with “flat” consonants (including \isi{pharyngeals}, rhotics, grooved fricatives, and labials). Thus, initially, through extensive language contact with and \isi{bilingualism} in \ili{Arabic}, the speakers attained an active category of \isi{pharyngeals}. Then the (inherited or loan) vocabulary with sounds that have pharyngeal-like effects on neighbouring vowels led to the reanalysis of the given vocabulary items as \isi{pharyngeal}. In this account, the whole syllable is \isi{pharyngeal} rather than individual sound segments. This account is particularly appropriate since, while it acknowledges the role of language contact with \ili{Arabic} in the initial stage, it posits a phonetic mechanism of language-internal development of pharyngealization that captures an expansion of \isi{pharyngeals} into historically non-\isi{pharyngeal} lexical items that would be impossible to explain on purely language-contact grounds. It is, for instance, consistent with the fact that, in the above-presented data, the emphatics, but not \isi{pharyngeals} in loan words, are restricted to the environment of more open vowels: \textit{e}, \textit{a}, \textit{o}, and \textit{i} [ɪ]. Furthermore, although not stated in the source study, the assumed subsequent development of a phonetic basis for the propagation of the \isi{pharyngeals} into items originally without \isi{pharyngeal} sounds is consonant with the facts of different stages or layers of borrowing. For instance, from the \ili{Arabic} \isi{root} \textit{√ǧmʕ} we have three forms in \ili{Kurmanji}: \textit{civat} ‘community, company’, \textit{cimat} ‘the assembly of prayers in a funeral’, and \textit{cemaʕet} ‘community’. The first form is probably the result of an early borrowing right after the  initial Islamicization of the \isi{Kurds}, as the fricativization of the bilabial nasal was active then (as seen also in \textit{silav} ‘greeting’ from \ili{Arabic} \textit{salām}; \citealt{Paul2008}). The second form with a slightly specialized semantic difference may have originated in a dialect where the mentioned fricativization did not occur. In any case, the first two forms, which are clearly early borrowings, did not retain the original \isi{pharyngeal}, whereas in a later borrowing from the same \isi{root}, when one can assume that the \isi{pharyngeals} were better tolerated in the language (and that the fricativization of bilabial nasal was not active), the \isi{pharyngeal} sound did survive.   

However, this account fails to explain why, in the great majority of the vocabulary with the relevant phonetic environment (syllables with “flat” consonants and low and back vowels), pharyngealization has not occurred. If the phonetic mechanism is integrated into the phonological system of the language, then pharyngealization would be expected in all relevant contexts. In this sense, although there is a phonetic ground to the propagation of the \isi{pharyngeals} and emphatics in Kurdish, it may be safer not to postulate it as integrated into the phonological system of the language. Rather, the \isi{pharyngeals} and emphatics should still be considered as peripheral to the phonological system (cf. \citealt{Haig2007,Anonbyforthcoming}), since, as noted by Haig (\citeyear[167]{Haig2007}), they are restricted to individual lexical items, their functional load is very limited, and there is considerable cross-speaker and cross-dialectal variability in the extent of their presence. 

Although it is not the main focus of this chapter, a note on the reverse direction of contact influence is in order at this point. The \ili{Arabic} dialects of \isi{Anatolia} or Upper Mesopotamia (\ili{Mardin}, \ili{Siirt}, Kozluk, Sason, and the plain of Muş) have adopted some consonant and vowel phonemes via \isi{loanwords} from Kurdish and \ili{Turkish}, which do not exist in mainstream \ili{Arabic} dialects (\citealt[84]{Jastrow2011Kurdish}; Akkuş, this volume: §3.1.1).\ia{Akkuş, Faruk@Akkuş, Faruk} The phonemes and example words with their sources are given in Table \ref{tab:opengin:8}.

\begin{table}
\begin{tabular}{ll}
\lsptoprule 
Phoneme & Example\\\midrule
bilabial stop /p/   &   \textit{parčāye} ‘piece’ < \ili{Tr.} \textit{parça}\\
voiced labio-dental fricative /v/  & \textit{davare} ‘ramp’  < Kr. \textit{dever} (\textsc{f)} ‘place’\\
voiceless affricate /č/     &  \textit{č\kern 0.75ptǝ\kern -1ptqmāq} ‘lighter’ < \ili{Tr.} \textit{çakmak}\footnote{It is more probable that this word (and others attributed to \ili{Turkish}) is borrowed via Kurdish, since the uvularization (/k/ > /q/) in \isi{loanwords} and the change in the vowel of the first syllable (cf. also \textit{qeymaẍ} ‘cream’, from \ili{Tr.} \textit{kaymak}) are typical of \ili{Kurmanji} spoken in the region.}\\
voiced palatal fricative /ž/\footnote{Note that the reflex of \ili{Arabic} 〈\kern .75pt{\arabscript{ج}}〉 in this variety is /ǧ/, not /ž/.}    & \textit{ṭāžī} ‘greyhound’ < Kr. \textit{ṭajî}\\
voiced velar stop /g/      & \textit{gōmlak} ‘shirt (modern)’ < \ili{Tr.} \textit{gömlek}\\
mid long front vowel /ē/    & \textit{tēl} ‘wire’ < \ili{Tr.} \textit{tel} (via Kr. \textit{têl})\\
mid long back vowel /ō/\footnote{Note also that the original \ili{Arabic} \isi{diphthongs} *ay and *aw are preserved in this variety, not monophthongized to /ē/ and /ō/.}  &  \textit{ḫōrt} ‘young man’ < Kr. \textit{xort}\\
\lspbottomrule
\end{tabular}
\caption{Borrowed phonemes in Arabic dialects of Anatolia}
\label{tab:opengin:8}
\end{table}

These additions into the \isi{phoneme} inventory of the \ili{Anatolian} \ili{Arabic} are evidently the result of contact with Kurdish and \ili{Turkish}. The introduction of these new phonemes has, as noted by Jastrow (\citeyear[84]{Jastrow2011Kurdish}), on the one hand re-established the lacking symmetry caused by historical sound changes in \ili{Old} \ili{Arabic}, while on the other hand causing further sound shifts in the inherited \ili{Arabic} vocabulary. 

\subsection{Morphology}

It is usually assumed that \ili{Arabic} influence on Kurdish is absent in the grammar (e.g. \citealt{Edwards1851}), being largely restricted to phonology and lexicon. This is indeed to a large extent true. There are, however, several potential grammatical features that may be related to such contact influence. 

Matras (\citeyear[75]{Matras2010}) suggests that the presence of aspect--mood prefixes in the languages of the Eastern Anatolian linguistic zone, namely \ili{Persian}, Kurdish, \ili{Neo-Aramaic}, \ili{Arabic} and Western \ili{Armenian}, is an outcome of language contact. Accordingly, all of these languages have a progressive–indicative aspectual prefix (in turn: \textit{mī-,} \textit{di-,} \textit{gǝ-}, \textit{ko-,} \textit{ba-/-a-}), while subjunctive is marked either by the absence of the indicative prefix (\ili{Armenian} and \ili{Neo-Aramaic}) or by a specialized subjunctive prefix (\ili{Persian}, Kurdish, \ili{Arabic}). Since such aspect--mood prefixes are considered typical of \ili{Iranian} languages of the region, they would have diffused from Kurdish and \ili{Persian} into the other languages of the zone, including \ili{Arabic} (which in its standard grammar does not have such forms; cf. \citealt{Ryding2014}: 46–47). However, assessing the validity of Eastern \isi{Anatolia} being a linguistic area, Haig (\citeyear{HaigÖpengin2014}: 20–25) casts doubt on this claimed contact scenario, primarily since: (i) the feature exists in \ili{Arabic} dialects outside the region; and (ii) it is absent in the two major languages of \isi{Anatolia}, namely \ili{Turkish} and \ili{Zazaki}. Jastrow (\citeyear[92]{Jastrow2011Kurdish}), on the other hand, although acknowledging the source of such verbal prefixes grammaticalizing from \ili{Old} \ili{Arabic} verb forms, hypothesizes – though without providing supporting arguments – that they may have developed under \ili{Turkish} and Kurdish influence. Assessing also the \isi{grammaticalization} of such formatives in various languages and rejecting a contact scenario behind their frequent occurrence in the languages of \isi{Anatolia}, Haig (\citeyear[26]{Haig2014}) concludes that the present indicative prefixes found in \ili{Kurmanji}, and in certain varieties of \ili{Aramaic} and \ili{Arabic} in \isi{Anatolia}, could be interpreted as reflexes of an inherited morphological template, which is well-attested in the related Northwest \ili{Iranian} and \ili{Semitic} languages outside \isi{Anatolia}.

Another (not previously discussed) candidate for \ili{Arabic} influence on \ili{Kurmanji} Kurdish relates to \isi{gender assignment} in more recent \isi{loanwords} from European languages. In \ili{Kurmanji}, like \ili{Arabic}, nouns are assigned to feminine and masculine genders. The \isi{gender} of inanimate nouns is largely arbitrary, with limited morpho-phonological basis in both languages. In \ili{Arabic}, words carrying the \textit{{}-a} ending are feminine, while in \ili{Kurmanji} abstract nouns ending in \textit{-î} are feminine, while the rest may be of either \isi{gender}. Now, when \ili{Arabic} borrows modern vocabulary items from European languages, items ending in \textit{{}-a} are assigned to feminine \isi{gender}, while the rest are assigned to masculine \isi{gender} \citep[5]{Ibrahim2015}. The default \isi{gender} assigned to new lexical borrowings is masculine in \ili{Arabic}. There is as yet no research on the \isi{gender assignment} of borrowings in \ili{Kurmanji}. However, it is easily observed that \ili{Kurmanji} spoken in Turkey mostly favors feminine, while the \ili{Kurmanji} of Iraq uses masculine \isi{gender} for integrating modern vocabulary items into the language. The modern lexical borrowings (boldface) in \REF{bkm:Ref14712415} are all assigned to masculine \isi{gender} in \ili{Badini} \ili{Kurmanji} of Iraq. Note that the \isi{gender} of the nouns is visible in the \textit{ezāfe} (see §\ref{bkm:Ref520275931}) and oblique case suffixes. 

%%1st subexample: change \ea\label{...} to \ea\label{...}\ea; remove \z  
%%further subexamples: change \ea to \ex; remove \z  
%%last subexample: change \z to \z\z 
\ea{\ili{Badini} dialect of \ili{Kurmanji} in Iraq (from media outlets)}\label{bkm:Ref14712415}\\
\ea 
\gll \textbf{sîstem}-ê endroyd-ê\\
     system-\textsc{ez.m} android-\textsc{obl.f}\\
\glt ‘Android system’
\ex\label{parl}
\gll serok-ê \textbf{parleman}-î\\
     president-\textsc{ez.m} parliament-\textsc{obl.m}\\
\glt ‘the president of the parliament’
\ex \gll \textbf{form}ê têgehiştin-ê\\
     form-\textsc{ez.m} understanding\textsc{{}-obl.f}\\
\glt ‘the form of understanding’
\ex \gll \textbf{moral}-ê diyalog\\
     moral-\textsc{ez.m} dialogue\\
\glt ‘the moral of dialogue’
\ex \gll \textbf{proj(e)}-ê av-ê\\
     project-\textsc{ez.m} water-\textsc{obl.f}\\
\glt ‘the water project’
\ex
\gll \textbf{prensîp}-ê hevwelatîbûn-ê\\
     principle-\textsc{ez.m} citizenship-\textsc{obl.m}\\
\glt ‘the principle of citizenship’
\z
\z

All of these lexical borrowings exist also in \ili{Kurmanji} as spoken in Turkey, but they are systematically used with feminine \isi{gender}. For instance the phrase in (\ref{parl}) would be realized as \textit{serok-ê} \textit{parleman-ê} (president-\textsc{ez.m} parliament-\textsc{obl.f}), with the feminine form of the oblique case suffix. 

As was stated above, the majority of such modern lexical borrowings in \ili{Arabic} are assigned to masculine \isi{gender}. The masculine \isi{gender assignment} in \ili{Kurmanji} in Iraq is thus most probably motivated by the \ili{Arabic} \isi{gender assignment} pattern. This is all the more plausible when we consider that \ili{Arabic}, as the dominant state language for the Iraqi \isi{Kurds} for almost a century, serves also as the intermediary language via which such lexical items are normally borrowed into \ili{Kurmanji} in Iraq. However, this contact influence must have been established relatively recently, since earlier technical borrowings in \ili{Kurmanji} in Iraq such as \textit{têlevizyon} and \textit{radyo} are treated as feminine nouns, despite being masculine in \ili{Arabic}.  

\subsection{\label{bkm:Ref520275931} Syntax}

Although several studies have dealt with the outcomes of language contact between Kurdish and (Neo-)\ili{Aramaic} in the grammar of these languages – especially on such topics as alignment \citep{Coghill2016}, \isi{word order} \citep{Haig2014}, and \isi{noun phrase} morphology \citep{Noorlander2014} – as far as I am aware, the only study on \ili{Arabic}–Kurdish contact in grammar is the short note of \citet{Tsabolov1994} about the distinctive position of the possessor in a multiple-modifier \isi{noun phrase} in \ili{Northern Kurdish}. 

As is well known, a number of West \ili{Iranian} languages (Middle and contemporary \ili{Persian}, Kurdish, \ili{Zazaki}, etc.) employ a bound morpheme for linking post-head modifiers in a \isi{noun phrase}, called \textit{ezāfe} or \textit{izāfe} (from \ili{Arabic} \textit{ʔiḍāfa} ‘joining, addition’), as in (\ref{bkm:Ref14712641}) and (\ref{bkm:Ref14712647}). 

\ea\label{bkm:Ref14712641}\ili{Persian} (personal knowledge)\\
\gll ḫāna-e bozorg\\
     house-\textsc{ez} big\\
\glt ‘(the) big house’
\ex \label{bkm:Ref14712647}\ili{Northern Kurdish} (personal knowledge)\\
\gll xanî-yê mezin\\
     house-\textsc{ez.m} big\\
\glt ‘the big house’
\z

The \textit{ezāfe} in \ili{Northern Kurdish} differs from its cognates in, for instance, \ili{Central Kurdish} and \ili{Persian}, as it inflects for \isi{gender} (masculine \textit{{}-ê} vs. feminine \textit{{}-a}) and number (singular \textit{{}-ê/-a} and plural \textit{{}-ên/-êd}), in addition to having secondary or pronominal forms used in chain \textit{ezāfe} constructions with multiple modifiers (and some other predicative functions; cf. \citealt{Haig2011,HaigÖpengin2018}). In most West \ili{Iranian} languages, noun phrases with multiple modifiers have their head noun first, followed by qualitative then possessive modifiers, as in (\ref{bkm:Ref14712778}) and (\ref{bkm:Ref14712785}). This is also the order in Middle \ili{Persian}, as in (\ref{bkm:Ref14712798}), where Tsabolov (\citeyear[122]{Tsabolov1994}) considers such constructions may be regarded as prototypes of the \textit{ezāfe} constructions of modern West \ili{Iranian} languages. 

\ea\label{bkm:Ref14712778}\ili{Persian} (personal knowledge)\\
\gll ḫāna-e bozorg-e Malek\\
     house-\textsc{ez} big-\textsc{ez} \textsc{pn}\\
\glt ‘Malek’s big house’
\ex \label{bkm:Ref14712785}\label{bkm:Ref14771780}\ili{Central Kurdish}  (personal knowledge)\\
\gll kurr-î gewre-y Karwan\\
     son-\textsc{ez} big-\textsc{ez} \textsc{pn}\\
\glt ‘my friend’s beautiful daughter’
\ex \label{bkm:Ref14712798}Middle \ili{Persian} \citep[122]{Tsabolov1994}\\
\gll pus ī mas ī Artavān\\
     son \textsc{ez} big \textsc{ez} \textsc{pn}\\
\glt ‘Artavan’s elder son’
\z

However, in \ili{Northern Kurdish} the order of modifiers is reversed, such that a possessor of the head noun in the \isi{noun phrase} comes before attributive modifiers, as in (\ref{bkm:Ref14712924}), where the secondary linking element is glossed as \textsc{sec}.  

\protectedex{
\ea\label{bkm:Ref14712924}\ili{Northern Kurdish} (personal knowledge)\\
\gll xanî-yê Malik-î (y)ê mezin\\
     house-\textsc{ez.m} \textsc{pn-obl.m} \textsc{ez.m.sec} big\\
\glt ‘Malik’s big house’
\z
}

Tsabolov observes that these syntactic particularities of \ili{Northern Kurdish} have no parallels in other Kurdish varieties and \ili{Iranian} languages as a whole, but that they correspond to the \isi{word order} in noun phrases in \ili{Arabic}, as can be seen in the comparison of (\ref{bkm:Ref14712996}) and (\ref{bkm:Ref14713006}). 

\ea\label{bkm:Ref14712996}\label{bkm:Ref14771925}\ili{Arabic} \citep[123]{Tsabolov1994}\\
\gll miḥfað̣atu ṭ-ṭālibi l-ǧadīdatu\\
     bag\textsc{.nom} \textsc{def}{}-student.\textsc{gen} \textsc{def}{}-new.\textsc{f.nom} \\
\glt ‘the student’s new bag’
\ex \label{bkm:Ref14713006}\ili{Northern Kurdish} \citep[123]{Tsabolov1994}\\
\gll çent-ê şagirt-î taze\\
     bag-\textsc{ez.m} student-\textsc{ez.m.sec} new\\
\glt ‘the student’s new bag’
\z

Note that although in standard \ili{Kurmanji} (\ili{Northern Kurdish}) the forms of the primary and secondary \textit{ezāfes} are identical, with the difference being in the latters’ status either as enclitics or independent particles, in the northern dialect of \ili{Northern Kurdish} considered by Tsabolov, the singular forms of the secondary \textit{ezāfe} are different (with masculine \textit{{}-î} and feminine \textit{{}-e}). In Tsabolov’s view, the centuries-old close contacts between Kurdish and \ili{Semitic} dialects, especially \ili{Arabic}, have not only resulted in the above-described change of noun-phrase-internal \isi{word order} (syntactic) but also in the development of secondary forms of \textit{ezāfe} through the “weakening” of the primary ones (morphological), because, he argues, such distinct forms “were necessary for correlating each attribute in an [\textit{ezāfe}] chain with the ruling noun they refer to” (\citeyear{Tsabolov1994}: 123). 

On closer scrutiny, however, the motivation Tsabolov puts forward for the morphological change may not be entirely correct, since, on the one hand, \textit{ezāfe} forms in \ili{Northern Kurdish} distinguish \isi{gender} and number, which already correlate the modifiers with their head nouns, and on the other hand, in the majority of \ili{Northern Kurdish} dialects the primary and secondary \textit{ezāfes} are formally identical. The change in form is an instance of \isi{vowel raising} (\textit{a} > \textit{e}, \textit{ê} > \textit{î}) that is also observed elsewhere in the morphology of \isi{noun phrase} (cf. \citealt{HaigÖpengin2018}). 

As for Tsabolov’s main claim regarding word-order change leading to the initial positioning of a possessor modifier in the \isi{noun phrase}, here too the role of language contact might require revision, since it might have more to do with language-internal organization of morphological material: \ili{Zazaki} (geographically contiguous with \ili{Kurmanji} but from a separate historical source to Kurdish), which, like \ili{Kurmanji}, has \isi{gender}/number-marking \textit{ezāfe} forms and a case distinction in its nominal system, follows precisely the same \isi{word order} pattern as \ili{Kurmanji} in the \isi{noun phrase} (cf. \citealt{Todd2002}: 95), while \ili{Sorani}, which has lost \isi{gender}/number-marking in \textit{ezāfes} and case distinctions in its nominal system, differs from them and instead follows the \ili{Persian} and Middle \ili{Persian} pattern (cf. \citealt{Öpengin2016}: 61–64). That is, the determining factor seems to be the presence or absence of \isi{gender}/number-marking \textit{ezāfe} forms, which enable reference tracking between heads and dependents in a \isi{noun phrase} independently of \isi{word order}.   

Despite the scepticism one may have towards Tsabolov’s hyopthesis, there is a rather parallel more recent syntactic change in progress stemming from the \ili{Arabic} influence on the Kurdish of Iraq. This change concerns especially the naming of institutions, such as schools and airports. Recall that in \ili{Central Kurdish} the possessor in a chain \textit{ezāfe} construction is positioned at the end of the \isi{noun phrase}, as illustrated in (\ref{bkm:Ref14771780}). However, in the case of these examples, the proper name occurs right after the head noun and before the qualitative modifier, as in (\ref{bkm:Ref14771794}) and (\ref{bkm:Ref14771799}). 

\ea\label{bkm:Ref14771794}\ili{Central Kurdish} (official signage)\\
\gll qutabxane-y Qemeryan-î seretayî\\
     school-\textsc{ez} \textsc{pn-ez} primary\\
\glt ‘Qamaryan primary school’
\ex \label{bkm:Ref14771799}\ili{Central Kurdish} (official signage)\\
\gll firokexane-y Hewlêr-î nêwdewletî\\
     airport-\textsc{ez} \textsc{pn-ez} international\\
\glt ‘Hawler international airport’
\z

If the proper name is understood as having the function of possessor here, this is an order that is rather different from the typical \ili{Central Kurdish} syntax of chain \textit{ezāfe} constructions. But this is precisely the order described for multiple modifier noun phrases of \ili{Arabic}, as in (\ref{bkm:Ref14771925}). Thus the order in (\ref{bkm:Ref14771799}) is the exact \isi{replication} of the \ili{Arabic} version of the same name illustrated in (\ref{bkm:Ref14772143}).

\protectedex{
\ea\label{bkm:Ref14772143}\ili{Arabic} (official signage)\\
\gll maṭār arbīl ad-dawlī\\
     airport \textsc{pn} \textsc{def}{}-international\\
\glt ‘Erbil international airport’
\z
}

This is clearly a recent \isi{imposition} from \ili{Arabic} which does not seem to have gone much beyond naming institutions, especially official signage: the \ili{Arabic}-like ordering of the name of the airport appears only half as frequently as the inherited order in a Google search. Furthermore, there is no trace of such a \isi{word order} pattern in the use of \ili{Central Kurdish} in Iran.  

\subsection{Lexicon}

\ili{Arabic} influence on Kurdish and all other Near Eastern languages is observed most clearly and abundantly in the vast number of \isi{loanwords}. According to Perry (\citeyear[97]{Perry2005}), the process of lexical \isi{convergence} initially took place in \ili{Persian} between the ninth and thirteenth centuries, when a large number of learned terms were borrowed into literary \ili{Persian}, and thence transmitted to the other languages of the region. This scenario explains some of the similarities of \isi{loanword} integration in the two languages (e.g. the borrowing of \textit{tāʔ} \textit{marbūṭa} as \textit{{}-at/-et} (rather than \textit{a}) in a number of words, such as \textit{hukūmat} ‘government’, \ili{Persian} \textit{hokūmat}, and \textit{quwet} ‘strength’, \ili{Persian} \textit{qovvat}). However, being spoken in a region that is closer geographically to \ili{Arabic}-speaking communities, and having had its own educational and religious institutions where \ili{Arabic} served as the high literary language, Kurdish must have also followed its own course of contact with \ili{Arabic}. Despite this, there are no studies of lexical borrowing from \ili{Arabic} into Kurdish. Given the vastness of the topic, with its layers of time-depth and subsantial extra-linguistic aspects, I can only propose here to sketch the major lexical domains of borrowing, and note some observations on the word class and morpho-phonological integration of the borrowings. 

The three major varieties differ in their proportions of borrowing from \ili{Arabic}. Impressionistically, \ili{Northern Kurdish} seems to have borrowed most extensively. There is, however, a deeper layer of lexical borrowings shared throughout Kurdish (some of which are common to all or most of the Near Eastern languages), such as the following (cited in their \ili{Northern Kurdish} forms):\footnote{The main source for the lexical items in this section, together with the information regarding their \ili{Arabic} origin, is \citet{Chyet2003}. However, I have supplied the interpretation and the discussion of the material and as such only I am responsible for any shortcomings.} 

\ea
\begin{tabbing}
\textit{xiyal} ‘thought, grief’ \hspace{1em} \= Ar. \textit{ḫayāl} ‘imagination’\kill
\textit{xerab} ‘bad’     \> < Ar. \textit{ḫarāb} ‘ruins’\\
\textit{xelk/xelq} ‘people’  \> < Ar. \textit{ḫalq} (\textit{√xlq} ‘to create’)\\
\textit{xiyanet} ‘betrayal’  \> < Ar. \textit{ḫiyāna} \\
\textit{xizêm} ‘nose-ring’  \> < Ar. \textit{ḫizām}\\
\textit{xizmet} ‘service’  \> < Ar. \textit{ḫidma} \\
\textit{ʿeql}/\textit{aqil} ‘reason’  \> < Ar. \textit{ʕaql} (\textit{qəltu} Ar. \textit{ʕaqəl})\\
\textit{qelem} ‘pen’    \> < Ar. \textit{qalam}\\
\textit{quwet} ‘strength’  \> < Ar. \textit{quwwa}\\
\textit{kitêb} ‘book’    \> < Ar. \textit{kitāb}\\
\textit{xiyal} ‘thought, grief’  \> < Ar. \textit{ḫayāl} ‘imagination’\\
\textit{hevîr} ‘dough’    \> < Ar. \textit{ḫamīr} \\
\textit{fikr} ‘thought, idea’  \> < Ar. \textit{fikr}\\
\textit{fêkî/fêqî} ‘fruit’  \> < Ar. \textit{fākiḥa}\\
\textit{ḥal} ‘condition’  \> < Ar. \textit{ḥāl} \\
\textit{ḥazir} ‘ready’    \> < Ar. \textit{ḥāḍir}\\
\textit{şol/şuẍul} ‘work’  \> < Ar. \textit{šuɣl} \\
\textit{terk} ‘abandonment’  \> < Ar. \textit{√trk} ‘to abandon’
\end{tabbing}
\z

Within varieties too, the dialect zones where the communities have had historically closer contact with \ili{Arabic}-speaking areas show greater \ili{Arabic} influence in vocabulary. Thus, the dialect of \ili{Northern Kurdish} named as Southern \ili{Kurmanji} by  \citet{ÖpenginHaig2014}, spoken around \ili{Mardin} and Diyarbekir provinces in Turkey, the Jazira province of northeast Syria, and the Sinjar region of Iraq, is the dialect with most extensive \ili{Arabic} lexical borrowings. Thus, the following items are restricted to this dialect of \ili{Northern Kurdish}: \textit{tefa-ndin} ‘extinguish-\textsc{tr.inf}’ (from dialectal Ar. \textit{ṭafa} or standard \textit{ṭafiʔa}), \textit{şiteẍl-în} ‘speak-\textsc{intr.inf}’ (from dialectal Ar. \textit{ištaɣal} ‘to work’), \textit{hersim} ‘unripe and sour grapes’ (from Ar. \textit{ḥiṣrim}), \textit{siʿûd} ‘good luck’ (Ar. \textit{suʕūd}, pl. of \textit{saʕd}), \textit{şîret} and \textit{şêwr} ‘advice, counsel’ (Ar. \textit{√šwr}).     

\ili{Arabic} \isi{loanwords} in Kurdish belong to various semantic fields, such as kinship, body parts, animals, agriculture, basic tools, temporal concepts and religion. Regarding kinship terms, while the terms for the members of the nuclear family are all inherited, the four second-degree kin terms are all borrowed from \ili{Arabic}: \textit{met} ‘paternal aunt’ (cf. Ar. \textit{ʕamma(t)}; this item does not exist in \ili{Sorani}), \textit{xalet/xaltî} ‘maternal aunt’ (Ar. \textit{ḫāla}), \textit{mam} \textit{{\textasciitilde} am} ‘paternal uncle’ (Ar. \textit{ʕamm}), \textit{xal} ‘maternal uncle’ (Ar. \textit{ḫāl}). Considering that the language had its own kin terms before its contact with \ili{Arabic}, the borrowing of such kin terms constitutes a case of \isi{prestige} borrowing, probably motivated by the use of such kin words as address forms in the cultures of the region (cf. \citealt{HaigÖpengin2015}).

Similarly, while words for basic animals are inherited, the animals not indigenous to the mountainous region of core Kurdistan are borrowed from \ili{Arabic}, such as \textit{tîmseḥ} ‘crocodile’ (Ar. \textit{timsāḥ}), \textit{fîl} ‘elephant’ (Ar. \textit{fīl}), \textit{xezal} ‘gazelle, deer’ (Ar. \textit{ɣazāl}). Likewise, the generic term for ‘bird’ or ‘large birds’ is the \ili{Arabic} \isi{loanword} \textit{ṭeyr} (Ar. \textit{ṭayr}), while the category word \textit{ferx} ‘young bird/chicken’ is also from \ili{Arabic} \textit{farḫ}. Several agricultural terms are also borrowed from \ili{Arabic}, such as \textit{ẓad} ‘grain, food’ (Ar. \textit{zād} `provisions'), \textit{simbil} ‘spike (of corn or wheat)’ (Ar. \textit{sunbul}), \textit{xox} ‘peach’ (Ar. \textit{ḫawḫ}), \textit{dims} ‘grape molasses’ (Ar. \textit{dibs}). Various terms for spaces and tools of daily life are also borrowed from \ili{Arabic}, such as \textit{saʿet} ‘hour’ (Ar. \textit{sāʕa}), \textit{sifre} ‘tablecloth’ (Ar. \textit{sufra} ‘dining table’), \textit{qefes} ‘cage’ (Ar. \textit{qafaṣ}), \textit{ḥubr} ‘ink’ (Ar. \textit{ḥibr}), \textit{ḥemam} ‘bath’ (Ar. \textit{ḥammām}), \textit{ḥewş} ‘yard’ (Ar. \textit{ḥawš)}, \textit{meẍmer} ‘velvet’ (Ar. \textit{muḫmal}). Some occupational terms from \ili{Arabic} are \textit{neqş} ‘embroidery’ (Ar. \textit{naqš} ‘painting, drawing’), \textit{ḥedad} ‘blacksmith’ (Ar. \textit{ḥaddād}), \textit{ʿesker} ‘soldier’ (Ar. \textit{ʕaskar} ‘army’), \textit{tucar} and its older form \textit{têcirvan} (Ar. \textit{tuǧǧār} ‘traders’, sg. \textit{tāǧir}).  

The older layer of administrative and legal terms are predominantly derived from \ili{Arabic} – though they may have mostly entered via \ili{Persian} and \ili{Ottoman} {Turkish} – such as \textit{sultan} ‘monarch’ (Ar. \textit{sulṭān}), \textit{walî} ‘provincial governor’ (Ar. \textit{wālī}), \textit{muxtar} ‘village chief’ (Ar. \textit{muḫtār}), \textit{ḥukûmet} ‘government’ (Ar. \textit{ḥukūma}), \textit{meḥkeme} ‘court’ (Ar. \textit{maḥkama}), \textit{deʿwā} ‘request, court case’ (Ar. \textit{daʕwa} ‘request, invitation’ and \textit{daʕwā} ‘court case’), \textit{qanûn} ‘law’ (Ar. \textit{qānūn}), \textit{mekteb} ‘school’ (Ar. \textit{maktab} ‘office, desk’).  

As for religious terms, similar to the \ili{Persian} case (cf. \citealt{Perry2012}: 72), a number of basic Islamic concepts are inherited, such as the words for god, prophet, angel, devil, heaven, purgatory, prayer, fasting, and sin. In some instances, the \ili{Arabic} equivalents of these terms exist alongside the inherited ones, restricting the use of the latter, as in the cases of \textit{şeytan} ‘devil’ and \textit{cehnem} ‘hell’, from \ili{Arabic} \textit{šayṭān} and \textit{ǧahannam}, replacing the \ili{Iranian} \textit{dêw} and \textit{dojeh}. Many other basic and more peripheral concepts are borrowed from \ili{Arabic}, such as the following: \textit{xêr} ‘good’ (Ar. \textit{ḫayr}), \textit{xezeb} ‘wrath’ (Ar. \textit{ɣaḍab}), \textit{civat/cemaʿet} ‘society, gathering’ (Ar. \textit{ǧamāʿa}), \textit{ḥec} ‘pilgrimage’ (Ar. \textit{ḥaǧǧ}), \textit{şeytan} ‘devil’ (Ar. \textit{šayṭān}), \textit{weʿz} ‘(Islamic) sermon’ (Ar. \textit{waʿð̣}), \textit{ḥelal} ‘permitted’ (Ar. \textit{ḥalāl}), \textit{ḥeram} ‘forbidden’ (Ar. \textit{ḥarām}), \textit{ruḥ} ‘soul, spirit’ (Ar. \textit{rūh}), \textit{tizbî} (\ili{Sorani} \textit{tezbêḥ}) ‘prayer beads’ (Ar. \textit{tasbīḥ}).  

Finally, there are also a large number of concepts (temporal, moral, cosmological) that originate from \ili{Arabic} \isi{roots}, such as \textit{sibe(h)} ‘morning, tomorrow’ (Ar. \textit{ṣabāḥ}), \textit{heyam} ‘period’ (Ar. \textit{ayyām} ‘days’), \textit{hêsîr} ‘prisoner’ (Ar. \textit{ʔasīr}), \textit{dinya} ‘world’ (Ar. \textit{dunyā}), \textit{ḥesab} ‘count, calculation’ (Ar. \textit{ḥisāb}), \textit{ḥîle} ‘trick, ruse’ (Ar. \textit{ḥīla}), \textit{ḥel}  ‘solution’ (Ar. \textit{ḥall)}, \textit{eşq} ‘love’ (Ar. \textit{ʕišq}), \textit{ʿerz} ‘honor, esteem’ (Ar. \textit{ʕirḍ}). Note also that the word \textit{dinya} is used corresponding to the English expletive subject \textit{it} in time and weather expressions, as in \textit{dinya} \textit{esr} \textit{e} ‘it is late afternoon’ or \textit{dinya} \textit{ewr} \textit{e} ‘it is cloudy’. This usage is noted to exist also in colloquial \ili{Arabic} \citep[155]{Chyet2003}.

Some other interesting developments with \ili{Arabic} material in Kurdish lexicon may be noted here. The \ili{Arabic} \textit{daʕwa} ‘invitation’ has resulted in two related but different concepts: \textit{dawet/dewat} ‘wedding ceremony’ and \textit{deʿwet} ‘invitation’. While the latter meaning is shared in \ili{Ottoman}/{Turkish} and \ili{Persian}, the former is a Kurdish-internal semantic expansion of the source meaning. The Kurdish (in all three varieties) word for ‘home’ \textit{mal}, in the sense of family and familial belongings, rather than the house as a structure, is probably derived from the \ili{Arabic} word \textit{māl} ‘goods, property’. The generic term in Kurdish that designates \isi{Christians} regardless of their ethnicity and confession is \textit{fileh/file} which derives from \ili{Arabic} \textit{fallāḥ} ‘peasant, farmer’. Finally, there is the word \textit{mixaletî} ‘the son of the maternal uncle or aunt’ in the southern \ili{Kurmanji} dialect of \ili{Northern Kurdish} that can probably be analysed as \textit{mi} (< \textit{ben} ‘son’) + \textit{xalet} ‘aunt’ (< Ar. \textit{ḫāla}) + \textit{î} ‘my’.

Turning now to the word class categories of the \isi{loanwords}, as has been seen from the presentation of semantic domains above, most \ili{Arabic} \isi{loanwords} in Kurdish are nouns. However, many \ili{Arabic} noun loans are incorporated into Kurdish verb forms. This takes place either through morphological integration or syntactic composition. In morphological integration, the \ili{Arabic} \isi{root} (whether nominal or verbal) is taken as the \isi{stem} onto which the Kurdish verbal suffixes \textit{-în/-îyan} for instransitives and \textit{{}-andin} for transitives are added. Thus the \ili{Arabic} noun \textit{ʕilm} ‘knowledge’, apart from being used in its nominal sense, serves as the \isi{stem} for the \isi{derivation} of the intransitive \textit{ʿelimîn} (\textit{ʿelim-în}) ‘to learn’ and transitive \textit{ʿelimandin} ‘to teach, educate’. The following verbs are further examples of using \ili{Arabic} \isi{roots} (whether the original borrowings are nouns or verbs is not always clear) in the \isi{derivation} of verbs in Kurdish: \textit{tefandin} ‘to extinguish’ (Ar. \textit{ṭafa/ṭafiʔa}), \textit{fetisandin} ‘to suffocate’ (Ar. \textit{faṭṭas}), \textit{fetilîn} ‘to turn around’ (?Ar. \textit{fatala} ‘to twist together’), \textit{qulibîn} ‘to be overturned’ (Ar. \textit{qalaba} ‘to overturn’), \textit{sekinîn} ‘to stand, stop’ (Ar. \textit{√skn} ‘calm, rest’), \textit{fikirîn} ‘to think; to look at’ (Ar. \textit{fikr} ‘thought’).\footnote{Kurdish possesses a number of preverbs such as \textit{ve-} and \textit{ra-}. When inflected with tense--aspect--mood prefixes, these preverbs are detached from the verb \isi{stem}, as with the verb \textit{ve-kirin} ‘to open’ in \textit{ve-di-ki-m} (\textsc{pvb-ind}{}-do.\textsc{prs-1sg}) ‘I open (it)’. Now, the initial syllable of the verbs \textit{sekinîn} and \textit{fekirîn}, which are based on \ili{Arabic} loan \isi{roots}, resemble such Kurdish preverbs. As a result, in some dialects, they are treated as preverbal elements detaching from the verb \isi{stem}, as with \textit{fe-di-ki-m-ê} ‘I look at it’ (own data, Şirnak area) or \textit{se-di-kin-e} ‘s/he stands’ (own data, from Gevaş), where the initial syllables of originally \ili{Arabic} \isi{roots} are reanalysed as preverbs.}  The verb \textit{qelandin} ‘to roast; to uproot’ has two sources as Ar. \textit{qalā} and \textit{qalaʕa,} respectively, which explains its polysemy in Kurdish. 

In syntactic composition, on the other hand, a compound verb\footnote{Here the term \textit{compound} \textit{verb} is employed in a pre-theoretical sense, regardless of whether or not the given complex verb is considered to form a compound. See \citet{Haig2002} for a discussion of complex verbs in Kurdish.}  is obtained by combining an \ili{Arabic} \isi{root} with an inherited auxiliary \isi{light verb}, such as \textit{kirin} ‘do’ or \textit{dan} ‘give’ for transitives, and \textit{bûn} ‘to be’ for instansitives. Thus, the combination of \ili{Arabic} adjective \isi{loanword} \textit{xerab} ‘bad’ (< Ar. \textit{ḫarāb} ‘ruin’) with \textit{kirin} yields the verbal meaning ‘to destroy’ while its combination with \textit{bûn} means ‘to go bad, be spoiled’. Some example compound verbs with \ili{Arabic} \isi{roots} are given in \REF{compound}.

\ea\label{compound}
\textit{qedr} ‘respect’              (Ar. \textit{qadr})                +   \textit{girtin} ‘to hold’    =  ‘to respect’     \\
\textit{silav} ‘greeting’            (Ar. \textit{salām} ‘peace’)       +   \textit{dan} ‘to give’       =   ‘to greet’      \\
\textit{teʿn\slash ṭan} ‘scolding’   (Ar. \textit{ṭaʕn} ‘piercing’)     + \textit{dan}                   =   ‘to criticize’  \\
\textit{qedeẍe} ‘forbidden’          (Ar. \textit{qadaḥa} ‘to rebuke’)  +   \textit{kirin} ‘to do’       =   ‘to forbid’     \\
\textit{qesd} ‘intention’            (Ar. \textit{qaṣd})                +   \textit{kirin}               =   ‘to head for’   \\
\textit{zeʿîf} ‘weak’                (Ar. \textit{ḍaʕīf})               +   \textit{bûn} ‘to be’         =   ‘to become slim’
\z

What motivates the choice between the morphological versus syntactic technique in the integration of \ili{Arabic} loan \isi{roots} in forming verbs in Kurdish is not yet clear. While a few such verbs are found to be used in both synthetic and analytic forms, such as \textit{ceribandin} and \textit{cerebe} \textit{kirin} ‘to try’ (Ar. \textit{< ǧarraba}), most verbs are used in just one of the two forms. However, there is a great deal of dialectal differentiation as to whether a verb is analytically or synthetically integrated. Thus, the morphologically integrated verbs of most {Northern} \ili{Kurmanji} dialects such as \textit{emilandin} ‘to use’ (dialectal Ar. \textit{ʕimil} ‘to do’), \textit{şuẍulîn} (Ar. \textit{šuɣl} ‘work’), \textit{fikirîn} (Ar. \textit{fikr} ‘thought’) are seen in the southeastern \ili{Badini} dialect in synthetic form, with a nominal base combining with a \isi{light verb}, as \textit{emel} \textit{kirin}, \textit{şol} \textit{kirin}, \textit{fikr} \textit{kirin}.   

There are also various function words (discourse markers, conjunctions, adverbs) which are either borrowed from \ili{Arabic} or developed in Kurdish based on material borrowed from \ili{Arabic}. Thus, the conjunction \textit{xeyrî} (also seen as \textit{xeyr} \textit{ji} and \textit{xêncî}) ‘apart from, besides’ is based on \ili{Arabic} \textit{ɣayr} ‘other than’, while the adversative \textit{emā} ‘but’ is dervied from \ili{Arabic} \textit{ʔammā} ‘however’. The similative \textit{şibî} (also \textit{şubhetî} and \textit{şitî}) is derived from the \ili{Arabic} \isi{root} \textit{√šbh} ‘resemblance’. The classifiers \textit{ḥeb} (and the adverbial \textit{hebekî} ‘a little’) and \textit{lib} are derived from \ili{Arabic} \textit{ḥabb} ‘grain(s)’ and \textit{lubb} ‘kernels’, respectively. Finally, some discourse and verbal adverbs resulting from \ili{Arabic} sources are as follows: \textit{meselen} ‘for instance’ and \textit{helbet} ‘of course’ are from \ili{Arabic} \textit{maθalan} and \textit{al-batta}; in the eastern section of the \ili{Badini} dialect of \ili{Kurmanji}, there is the use of the discourse marker \textit{seḥî} ‘apparently, that means’, which is derived from the \ili{Arabic} \textit{aṣaḥḥ} ‘more correct’ – which separately exists in wider Kurdish as \textit{esseḥ} ‘certainly’; while, finally, the \ili{Arabic} adjective \textit{qawī} ‘strong’ has evolved into an adverb \textit{qewî} ‘very; very much’ (though this is more literary than spoken).      

All of these lexical borrowings illustrate matter \isi{transfer} (in the sense of \citealt{MatrasSakel2007}). In the following we have two instances of pattern \isi{transfer}. First, there is a particular adverbial form \textit{nema} ‘no longer’, found only in the southeastern dialect of \ili{Kurmanji}, spoken in the \ili{Mardin} region of Turkey and Jazira region of northeast Syria. This can be analysed as \textit{ne-ma}, consisting of the negative prefix \textit{ne-} and the past \isi{tense} 3\textsc{sg} conjugation of the verb \textit{man} ‘to stay’, as in (\ref{bkm:Ref14778636}).

\ea
Southern dialect of \ili{Northern Kurdish} (Media)\label{bkm:Ref14778636}\footnote{From a poem by an author from Syria, available online at: \url{http://avestakurd.net/blog/2016/10/26/romanivs-kurd-jan-dost-lal-b-ye-vdyo/} (accessed 31/01/2020).} \\
\gll nema di-kar-im veger-im welêt\\
     no.longer \textsc{ind}{}-be.able.\textsc{prs-1sg} return.\textsc{prs.sbjv-1sg} country.\textsc{obl}\\
\glt `I can no longer return to the homeland'\z

There is an immediately-corresponding adverbial form \textit{mā} \textit{ʕād} ‘no longer’ in \ili{Arabic}, which is based on the negative form of the semantically similar verb \textit{ʕād} ‘to return, keep doing’. This is obviously not a very recent development as it is shared in the whole dialect area across country borders, but seemingly not so deep either as to be shared by all Kurdish varieties, not even by all \ili{Northern Kurdish} dialects, further strengthening the particular status of the Jazira region in \ili{Arabic}--Kurdish language contact. 

Second, there is a particular lexical construction \textit{bi} \textit{X} \textit{rabûn} ‘to do; to complete; to achieve’ in \ili{Northern Kurdish} and \textit{hellsan} \textit{be} \textit{X} in \ili{Central Kurdish}, where \textit{X} stands for any activity or task (usually in the form of an \isi{infinitive} verb). The construction is based on the verb for ‘to rise, stand’ and a \isi{preposition} in both varieties, as illustrated in (\ref{bkm:Ref14779023}) and \REF{17}.

\ea \ili{Central Kurdish} (Media)\label{bkm:Ref14779023}\footnote{URL of \isi{article}: \url{http://www.kurdistan24.net/so/news/5ca67132-7a7f-4840-bfb4-dea5bf25ea2e} (accessed 31/01/2020).}\\
\gll polîs hellsa be kokirdinewe-y zanyarî\\
     police rise.\textsc{pst.3sg} with collecting-\textsc{ez} information\\
\glt ‘The police undertook (the task of) collecting information.’
\ex \label{17} \ili{Northern Kurdish} (Media)\footnote{URL of \isi{article}: \url{http://portal.netewe.com/mir-celadet-bedirxan-bi-tene-sere-xwe-bi-kare-dewleteke-rabu/} (accessed 31/01/2020).}\\
\gll Mîr {Celadet (…)} bi kar-ê dewlet-ek-ê rabû\\
     Emir Celadet with work-\textsc{ez.m} state-\textsc{indf-obl.f} rise.\textsc{pst.3sg}       \\
\glt ‘Emir Celadet undertook the work of a state.’\z
This lexical construction also has a parallel in \ili{Modern Standard} \ili{Arabic}, based on the verb \textit{qāma} ‘to stand (up)’ and the \isi{preposition} \textit{bi} ‘with’, with the collocation \textit{qāma} \textit{bi} meaning ‘to undertake’. This is obviously a recent influence on Kurdish, as it is seen only in Iraq and Syria, and in a manner cross-cutting the broad variety borders between \ili{Sorani} and \ili{Kurmanji}. 

\section{Conclusion}
Contact with \ili{Arabic}, which started in the early medieval period (approx. 7th–8th centuries) with the arrival of Islam in the Near East, has had a profound impact on Kurdish, particularly on its lexicon and phonology. Given the total absence of any substantial previous study on the matter, the present chapter provides a first assessment of the influence of \ili{Arabic} on Kurdish, primarily as represented in Kurdish phonology and lexicon but also, albeit more restrictedly, in morphology and syntax. \ili{Kurmanji} Kurdish seems to be the variety that is most affected by contact with \ili{Arabic}, which is understandable considering the geographical continuity of the Kurdish and \ili{Arabic} communities, especially in the historical Jazira region and more widely in Upper Mesopotamia (in \ili{Mardin}–Diyarbekir, \ili{Mosul}–Sinjar, and Haseke province). There are thus areas which show more \isi{intensive} \ili{Arabic} influence within the speech zones of major Kurdish varieties, while the outcomes of the contact reflect different layers in terms of time depth. Accordingly, the deeper-layer influence comes in the form of lexical \isi{convergence} with \ili{Arabic}, sometimes through the intermediary of \ili{Persian} and/or \ili{Ottoman} {Turkish}. This contact has repercussions in the expansion of the phonological inventory of the language, and is shared across most Kurdish varieties. There are no unquestionably demonstrated changes in the morphosyntax resulting from contact with \ili{Arabic} at this layer. At the relatively shallower layer, the influence is mainly seen in Syria and Iraq, and in the form of further expansion of the phonological inventory and a vocabulary heavily lexified by \ili{Arabic} \isi{roots} incorporated also into the verbal domain. There are also several cases illustrating morphosyntactic and lexicosyntactic change, such as the default \isi{gender assignment} and \isi{word order} in complex noun phrases, as well as certain phrasal and adverbial lexical items. 

In terms of “cognitive dominance”, in the sense of \citet{VanCoetsem1988,VanCoetsem2000} and \citet{Lucas2015}, in these instances of contact influence, the deeper-layer influence, which is restricted to, or related to, lexical borrowing, takes place with the speakers being cognitively dominant in the \isi{recipient language}, Kurdish. The more recent instances of heavy lexification, and morphosyntactic and lexicosyntactic changes may, however, be the result of \isi{imposition}, where the speakers are dominant in the \isi{source language}. 

These outcomes may also be related to \isi{bilingualism} and language configuration in historical perspective. That is, the absence of \isi{imposition} (in the form of morphosyntactic changes) in the deeper historical layer, and the restriction of the influence to lexicon, point to the absence of widespread \ili{Arabic}–Kurdish \isi{bilingualism} among the speakers of Kurdish at those historical stages. Some \isi{imposition} of this kind is observed in the \ili{Kurmanji} of the Jazira region, which is known to have had greatest speaker contact between Kurdish and \ili{Arabic} speech communities. By contrast, the widespread \isi{bilingualism} and \ili{Arabic}-dominant linguistic configuration in Syria and Iraq for at least a century has led to instances of \isi{imposition} where the morphosyntactic and lexical patterns of \ili{Arabic} are replicated in Kurdish. These outcomes are also mostly consonant with the predictions of Van Coetsem’s (\citeyear{VanCoetsem1988,VanCoetsem2000}) “\isi{stability gradient}”, which argues that lexicon is less stable than syntax and phonology, which require dominance in the \isi{source language} in order to be affected by contact-induced change.    

Given the limitations of a first attempt, much is yet to be explored regarding Kurdish–\ili{Arabic} language contact. In particular, the precise paths of development of \isi{pharyngeals} and emphatics in Kurdish should be analysed through fieldwork-based \isi{comparative} dialect data, while, in the domain of lexicon, it is important to analyse the morphophonological integration of borrowings into Kurdish. It is also of interest to be able to develop diagnostics to disentangle direct \ili{Arabic} influence on Kurdish from influence via other major languages such as \ili{Persian} and \ili{Ottoman} {Turkish}. Finally, a detailed account of the history of Kurdish–\ili{Arabic} socio-political and cultural contact is required in order to complement the linguistic data and enable a more fine-grained analysis of the agentivity of contact-induced change in Kurdish.      

\section*{Further reading}
\citet{Barryforthcoming} is a comprehensive and theoretically grounded treatment of the introduction and further propagation of \isi{pharyngeal} sounds in Kurdish.\\ 
\citet{Chyet2003} is the most comprehensive Kurdish–English dictionary,  providing information on the \isi{source language} of most \isi{loanwords} in Kurdish, including those from \ili{Arabic}.\\
\citet{Tsabolov1994} is the only work published so far on \ili{Arabic} influence on the grammar of Kurdish. 

\section*{Acknowledgements}
I would like to thank the editors of the volume and an anonymous reviewer for their helpful and detailed feedback. Thanks also to Adam Benkato for his help with \ili{Arabic} data. Only I am responsible for any remaining shortcomings and errors. 

\section*{Abbreviations}
\begin{multicols}{2}
\begin{tabbing}
\textsc{ipfv} \hspace{1em} \= before common era\kill
Ar.              \> {Arabic}\\
BCE             \> before Common Era\\
ca.             \> circa\\
\textsc{def}    \> {definite} \\
\textsc{drct}   \> directional\\
dial.           \> dialectal \\
\textsc{ez}     \> \textit{ezāfe} \\
\textsc{f}      \> feminine \\
\textsc{gen}    \> genitive\\
Kr.             \> Kurdish\\
\textsc{ind}    \> indicative \\
\textsc{indf}   \> indefinite \\
\textsc{inf}    \> {infinitive}\\
\textsc{intr}   \> intransitive\\
\textsc{ipfv}   \> imperfective \\
\textsc{m}      \> masculine \\
\textsc{neg}    \> negative\\
\textsc{nom}    \> nominative\\
\textsc{obl}    \> oblique\\
\textsc{pl/}pl. \> plural \\
\textsc{pn}     \> proper noun\\
\textsc{poss}   \> possessive \\
\textsc{prs}    \> present \\
\textsc{pst}    \> past \\
\textsc{pvb}    \> preverb\\
\textsc{sbjv}   \> subjunctive \\
\textsc{sec}    \> secondary or pronominal \\ \> \textit{ezāfe}/linking element\\
\textsc{sg/}sg. \> singular\\
\textsc{tr}     \> transitive\\
Tr.             \> {Turkish}
\end{tabbing}
\end{multicols}


{\sloppy\printbibliography[heading=subbibliography,notkeyword=this]}
\end{document}
