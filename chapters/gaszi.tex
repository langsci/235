\documentclass[output=paper]{langsci/langscibook} 
\author{Dénes Gazsi}
\title{Iranian languages}
\abstract{Iranian languages, spoken from Turkey to Chinese Turkestan, have been in language contact with Arabic since pre-Islamic times. Arabic as a source language has provided phonological and morphological elements, as well as a plethora of lexical items, to numerous Iranian languages under recipient-language agentivity. New Persian, the most significant member of this group, has been a prominent recipient of Arabic language elements. This study provides an overview of the historical development of this contact, before analyzing Arabic elements in New Persian and other New Iranian languages. It also discusses how Arabic has influenced Modern Persian dialects, and how Persian vernaculars in the Persian Gulf region of Iran have incorporated Arabic lexemes from Gulf Arabic dialects.}


\begin{document}
\maketitle 

\section{Current state and historical development}

\subsection{Iranian languages}

Iranian languages, along with Indo-Aryan and Nuristani languages, constitute the group of Indo-Iranian languages, which is a sizeable branch of the Indo-European language family. The term ``Iranian language'' has historically been applied to any language that descended from a proto-Iranian parent language spoken in Central Asia in the late third to early second millennium BCE (\citealt{Skjærvø2012}).

Iranian languages are known from three chronological stages: Old, Middle, and New Iranian. Persian is the only language attested in all three historical stages. New Persian, originally spoken in Fārs province, descended from Middle Persian, the language of the Sasanian empire (third–seventh centuries CE), which is the progeny of Old Persian, the language of the Achaemenid empire (sixth–fourth centuries BCE). New Persian is divided into Early Classical (ninth–twelfth centuries CE), Classical (thirteenth–nineteenth centuries) and Modern Persian (from the nineteenth century onward), the latter considered to be based on the dialect of Tehran \citep[427]{Jeremiás2003}. 

Today, Iranian languages are spoken from the Caucasus, Turkey and Iraq in the west to Pakistan and Chinese Turkestan in the east, as well as in a large diaspora in Europe and the Americas. New Iranian languages are divided into two main groups: Western and Eastern Iranian languages. The focus of this study is New Persian, the most significant member among Iranian languages, but a brief overview of Arabic influence on other New Iranian languages will also be provided. Below is a list of the most important members and their geographical distribution \citep[246]{Schmitt1989}.


\subsubsection{Western Iranian languages}



\subsubsubsection{Southwestern group}
Persian (\textit{Fārsī}) (spoken throughout Iran and adjacent areas), Tajik (the variety of New Persian in Central Asia), Darī Persian (Afghanistan), Kumzārī (Musandam Peninsula). Persian dialects in this group include Dizfūlī (Khuzestan province), Lurī (ethnic group along the Zagros mountain range), Baḫtiārī (nomadic tribe in the Zagros mountains), Fārs dialects (Fārs province), Lāristānī dialects (Lāristān region of Fārs province), Bandarī (dialects spoken around Bandar ʕAbbās in the Persian Gulf region, to which Fīnī also belongs).



\subsubsubsection{Northwestern group}
Kurdish, Zazaki (in eastern Turkey), Gurānī (in eastern Iraq and western Iran), Balūčī (Balochi, spoken chiefly in Iranian and Pakistani Baluchistan, and parts of Oman). Non-literary languages and dialects: Tātī, Tālišī and Gīlakī (on the shores of the Caspian Sea), Central dialects (spoken in a vast area between Hamadān, Kāšān and Iṣfahān), Kirmānī (south of the Dašt-i Kawīr).



\subsubsection{Eastern Iranian languages}

\subsubsubsection{Southeastern group}
Pashto (Afghanistan, Pakistan, eastern border region of Iran), Pamir languages (Pamir Mountains along the Pānj River).

\subsubsubsection{Northeastern group}
Yaɣnōbi (Zarafšān region of Tajikistan), Ossetic (central Caucasus).

\subsection{Historical development of Arabic–Persian language contact} %1.2. /

Language contact between Arabic and Persian has been a reciprocal process for the past 1500 years. During the pre-Islamic and early Islamic era (sixth–seventh centuries CE), Middle Persian, being embedded in the well-established and sophisticated Iranian culture, provided many loanwords to pre-Classical and Classical Arabic (\citealt{Gazsi2011}: 1015; see also van Putten, this volume)\ia{van Putten, Marijn@van Putten, Marijn} under RL (recipient-language) agentivity (\citealt{VanCoetsem1988,VanCoetsem2000}). With the collapse of the Sasanian Empire and expansion of Islam and the Arabic language over vast territories outside Arabia, Classical Arabic began to exercise an unprecedented impact on the emerging New Persian language. Arabic never took root in the everyday communication of the ethnically Persian population, although it gained some dominance as a written vehicle in the administrative, theological, literary and scientific domains in the eastern periphery of the Abbasid Caliphate. Instead, spoken Middle Persian (\textit{Darī}) flourished as a vernacular language. In the middle of the ninth century CE, it was in this part of Iran, specifically in Fārs province, that \textit{Darī} emerged in a new form as it repositioned itself in the culture and literature of the local populace. This new literary language, the revitalization of the Persian linguistic heritage, would be called New Persian. Since its earliest phase, New Persian has borrowed a staggering number of loanwords. Initially, these loanwords were borrowed from various northwestern and eastern Iranian languages, such as Parthian and Sogdian. Despite this relatively large group of loans, the most versatile lenders were the Arabs. Whereas in the pre-Islamic era Arabic had almost exclusively taken lexical items from Middle Persian (in the fields of religion, botany, science and bureaucracy among others), New Persian also incorporated Arabic morphosyntactic elements.

The first Arabic loanwords began to permeate New Persian in the ninth–tenth centuries CE (20–30\%). This process was not even diminished by the Iranian \textit{šuʕūbiyya} movement, the major output of which was all conducted in Arabic. In subsequent centuries, Persian continued to absorb an ever-expanding set of Arabic lexemes. By the turn of the twelfth century, the proportion of Arabic loans increased to approximately 50\%. The majority of Arabic loans had already been integrated into New Persian by that time and have shown a remarkable steadiness until recently.

After the fall of Baghdad in 1258 CE, Arabic lost its foothold in the eastern provinces of the Caliphate, thereby drawing the final boundary between the use of Arabic and Persian \citep{Danner2000}. The Mongol Ilkhānids, who as non-Muslims were not dependent on Arabic, introduced Persian as the language of education and administration in Iran and Anatolia. Despite the significant destruction the Mongols caused to northern Iran during their conquest, this period (thirteenth and foureenth centuries CE) is considered to be the zenith of Persian literature. This is also the epoch when literary Persian is, in an excessive way, inundated with Arabic language elements. This phenomenon is easily detectable in the works of one of the most significant personalities in Classical Persian literature, and a pre-eminent poet of thirteenth-century Persia, Saʕdī of Shiraz. Following the norms of Persian prose writing and poetry of his time, Saʕdī flooded his writings with a bewildering array of Arabic language elements. To illustrate this, here is a typical sentence from Saʕdī’s \textit{Gulistān} ‘Rose Garden’ (completed in 1258 CE), where words of Arabic origin are highlighted in boldface (\citealt{Yūsifī2004}: 77).


\ea
\RL{\textarabic{
اگر راى عزيز فلان ، أحسن الله خلاصه ، به جانب ما التفات كند در رعايت خاطرش هرچه تمامترسعى كرده شود واعيان اين مملكت به ديدار او مفتقرند و جواب اين حروف را منتظر
}}\\
{\itshape agar \textbf{rāy}-i \textbf{ʕazīz}-i \textbf{fulān}, \textbf{aḥsana allāhu ḫalāṣahu}, ba \textbf{ǧānib}-i mā \textbf{iltifāt} kunad dar \textbf{riʕāyat}-i \textbf{ḫāṭir}aš har či \textbf{tamām}tar \textbf{saʕī}  karda  šawad wa \textbf{aʕyān}-i īn \textbf{mamlakat} ba dīdār-i ū \textbf{muftaqir}and wa \textbf{ǧawāb}-i īn \textbf{ḥurūf} rā \textbf{muntaẓir}.}\\
\glt ‘If the precious mind of that person, may God make the end of his affairs prosperous, were to look in our direction, the utmost efforts would be made to please him, because the nobles of this realm would consider it an honor to see him, and are waiting for a reply to this letter.’\footnote{Persian transcription in this chapter follows the Arabic phonological conventions to avoid using two disparate systems.}
\z

It is easy to ascertain that, apart from verbs and adverbs, almost every lexical item in the sentence is of Arabic origin. But writers of this era, such as Saʕdī, not only inundated their works with Arabic elements, but even used Arabic morphology and semantics freely by coining new and innovative meanings, e.g. \textit{ṣaʕqa} ‘lightning’ < MSA/MSP \textit{ṣāʕiqa} or \textit{baṭṭāl} ‘liar’ < MSA/MSP ‘inactive, unemployed person’,\footnote{In the chapter, references are made to Modern Standard Arabic (MSA) and Modern Standard Persian (MSP) as a comparison to dialectal forms in both languages. This seemed more straightforward as it is not always feasible to ascertain at what point in time a lexeme was borrowed from Arabic into Persian.} < MSA \textit{mubṭil} ‘liar’. The Persian and Arabic language use of Saʕdī and other literary figures in the Classical Persian period came closest to what \citet{Lucas2015} calls convergence under the language dominance principle. As reflected in the purely Arabic and Arabic-infused Persian segments of his oeuvre, Saʕdī was equally dominant in both Classical Arabic and Classical Persian along with the dialect of Shiraz.

Modern Persian is still deeply rooted in Arabic. Arabic loanwords constitute more than 50\% of its vocabulary, but in elevated styles (religious, scientific, literary) Arabic loans may exceed 80\% \citep{Jeremiás2011}. Although the proportion of these loanwords fluctuates according to age, genre, social context or idiolect, any style in Modern Persian deprived of Arabic influence is almost impossible. An endeavor similar to \name[Atatürk]{Atatürk}{}’s to purge Turkish of foreign language elements would be unrealistic in Modern Persian, even with recurring efforts by linguistic purists and the Academy of Persian Language and Literature (\textit{Farhangistān-i} \textit{zabān} \textit{wa} \textit{adab-i} \textit{fārsī}).\footnote{An example of their activity is the publication (by \citealt{Rāzī2004}) of a dictionary that lists “pure” Persian words.} It is noteworthy that when the need arose for new terminology to describe fledgling political concepts in Iran, for instance during the Constitutional Revolution in the early twentieth\textsuperscript{} century, as \citet{Elwell-Sutton2000} phrased it, “politicians and journalists instinctively turned to Arabic rather than Persian”. Frequently, however, these “Arabic” words were new coinages in the recipient language, e.g. \textit{mašrūṭa} ‘constitution’, \textit{mawqiʕiyyat} ‘situation, position’. After the Islamic Revolution in 1979, another wave of Arabic lexemes related to the new religious governing system surfaced, e.g. \textit{mustaẓʕifīn} ‘the needy, the enfeebled’ (< MSA \textit{mustaḍʕafūna}\kern 0.5pt/\kern -1pt\textit{mustaḍʕafīna}).

Primary and secondary schools in contemporary Arabic-speaking countries do not offer language education in Persian. In Iran, compulsory Classical Arabic instruction is part of the curriculum. However, the language is taught for religious purposes only, with no intention to utilize MSA as a means of acquiring communication skills.

\section{Contact languages}

This section briefly describes the linguistic impact of Standard Arabic on several New Iranian languages. A more detailed analysis of contact-induced language change in New Persian (\textit{Fārsī}) will follow in §\ref{dial}.

\subsection{Arabic influence on New Iranian languages}

\subsubsection{Tajik (\textit{Tōǧīkī})}

Tajik, written in a modified Cyrillic script, is the variety of New Persian spoken throughout Central Asia, most notably in Tajikistan, Uzbekistan, and northern Afghanistan. Similarly to all varieties of Persian, Arabic borrowings constitute the earliest layer of foreign vocabulary in Tajik \citep{Perry2009}. This lexicon was transferred under RL agentivity. Although Arabic lexical items have a firm hold in Tajik, their pattern of distribution differs from that of New Persian. For instance, Tajik uses \textit{pēš} ‘before’ and \textit{pas} ‘after’ rather than MSA/MSP \textit{qabl} and \textit{baʕd}, but \textit{ōid} \textit{ba}{}-/\textit{ōid-i} (< MSA \textit{ʕāʔid} ‘returning’) ‘concerning, relating to’ in lieu of MSP \textit{rāǧiʕ} \textit{ba}{}- (< MSA \textit{rāǧiʕ} ‘recurring’). Also, \textit{madaniyyat} ‘civilization’ (< MSA \textit{madaniyya} ‘civilization’; cf. MSA/MSP \textit{tamaddun} ‘civilization’), \textit{hōzir} ‘now’ (< MSA \textit{ḥāḍir} ‘present; ready’, MSP \textit{ḥāẓir} ‘present’), \textit{ittifōq} ‘(labor) union’ (< MSA \textit{ittifāq} ‘agreement; contract’; cf.  MSP \textit{ittiḥād} ‘[labor] union’).

Arabic plural forms, both sound feminine plural and broken plural, were lexicalized with collective or singular meanings: \textit{hašarōt} ‘insect’, with regular plural ending \textit{hašarōthō} ‘insects’ (< MSA/MSP \textit{ḥašarāt} ‘insects’), \textit{talaba} ‘student’, pl. \textit{talabagōn} (< MSA/MSP \textit{ṭalaba} ‘students’), \textit{šarōit} ‘condition, stipulation’ (< MSA/MSP \textit{šarāʔiṭ} ‘conditions’).


\subsubsection{Kurdish}

A characteristic feature of Kurdish, the change of postvocalic /m/ > /v/ or /w/, also occurs frequently in words of Arabic origin: \textit{silāv} ‘greeting’ (< MSA/MSP \textit{salām}; \citealt{Paul2008}).

\subsubsection{Gurānī}

The phonological system of Gurānī dialects is similar to Kurdish in the occurrence of Arabic pharyngeal and emphatic sounds /ʕ/, /ḥ/, /ṣ/ \citep{MacKenzie2012}.

\subsubsection{Ossetic}
Ossetic has incorporated terms related to Islam from Arabic and Persian through neighboring Caucasian languages \citep{Thordarson2009}.

\subsection{Arabic-speaking communities in Iran}
Arabic-speaking communities are known to be present within the boundaries of the Islamic Republic of Iran, but their exact number is not readily discernible from official statistics. It is estimated that 3\% of Iran’s 80 million citizens are Arabs, which would put the Arab population at approximately 2.5 million. The majority of Arabs live in the western parts of Khuzestan province (see Leitner, this volume),\ia{Leitner, Bettina@Leitner, Bettina} but also along Iran’s Persian Gulf coast and parts of Khorasan in eastern Iran \citep{Oberling2011}. Already during the Sasanian era, several Arab tribes, including the Bakr ibn Wāʔil and Banū Tamīm, settled in the area stretching from the Šaṭṭ al-ʕArab to the Zagros Mountains \citep{Daniel2011}. At the end of the sixteenth century, the Banū Kaʕb, originating from present-day Kuwait, settled in Khuzestan. During subsequent centuries, more Arab tribes moved from southern Iraq to the province. As a result, Khuzestan, which until 1925 was called ʕArabistān, became extensively Arabized. Members of these Arab tribes live on either side of the Iran–Iraq border. In the same way as Iraqi Arabic vernaculars, Khuzestan Arabic has been influenced by Persian. However, Khuzestan Arabic can most easily be distinguished from Iraqi dialects by its wide-ranging transfer of Persian lexemes (\citealt{Ingham1997}: 25; see also Leitner, this volume).\ia{Leitner, Bettina@Leitner, Bettina}

Arab presence has a well-documented history on the Iranian coastline of the Persian Gulf, in what now constitutes Būšihr and Hurmuzgān province. According to travelogues from the eighteenth to twentieth centuries CE, as well as British archival materials dating back to the British Residency of the Persian Gulf, Arab tribes inhabited most fishing and pearling villages, as well as islands and coastal towns with strategic importance (e.g. Bandar ʕAbbās). The most significant tribes in this area were, and in some cases still are, the Qawāsim, Marāzīq, Āl Ḥaram, Āl ʕAlī, Āl Naṣūr, Banī Tamīm, Banī Ḥammād, Banī Bišr, among others. In contrast to most Persians and Khuzestani Arabs who are primarily Shiite, these tribes are Sunni Muslims. A widespread exonym to designate Arabs on the Iranian coast, but shunned by the local population, is \textit{hōla} (variously referred to as \textit{hula}, \textit{huwala} or \textit{hawala}). Local tribes prefer the endonym ‘Arabs of the Coast’ (\textit{ʕarab} \textit{as-sāḥil}) \citep[110]{Gazsi2017}.

Most Khuzestani and Iranian Persian Gulf Arabs are bilingual, speaking Arabic as their mother tongue and Persian as a second language. Although Khuzestan and the two Persian Gulf provinces are geographically part of Iran, linguistically their Arab populations form a continuum with the southern Mesopotamian Muslim \textit{gilit}{} dialects, and the dialects of the eastern coast of the Arabian Peninsula, respectively. In the spoken and written code, ‘Arabs of the Coast’ often engage in tetra-glossic switching between MSA, Gulf Arabic (GA), MSP, Colloquial Persian and one of its local dialects such as Bandarī. In their speech, Persian phonological and lexical elements are borrowed into GA under RL agentivity.

\section{Contact-induced changes in New Persian and modern Persian dialects} \label{dial}
Language contact between Arabic and New Persian is most evidently detectable in the New Persian lexicon, and to a lesser extent in phonology and morphosyntax. This section summarizes the characteristics of this contact. In addition to standard New Persian, and its contemporary variant MSP, Arabic has also influenced modern Persian dialects. This influence is slightly different, and in several ways more far-reaching, particularly in the realm of phonology and lexicon.

Persian dialects developed separately from and parallel to Classical Persian and MSP. Modern Persian dialects retain several Early Classical and Classical Persian phonological and morphosyntactic features that are not present in MSP. Additionally, they were in direct contact with the Arabic language through Arab tribes that settled across Persia immediately after the Islamic conquest or in later centuries. Although most Arab tribes have long been integrated into the Persian-speaking population, the Arabic language in the areas currently dominated by ethnic Arabs is still in contact with the surrounding Persian dialects. Unlike Arabic influence on the standard version of New Persian, Arabic influence on modern Persian dialects is an understudied field that does not allow for providing an exhaustive list of contact-induced changes at this point. Instead, below is a preliminary description of salient examples of how Arabic phonological and lexical elements were transferred to New Persian, both its standard and dialectal variations.

\subsection{Phonology}


\subsubsection{New Persian}

The initial step in the adoption of Arabic lexemes was the adoption of the Arabic script. New Persian began to use a modified Arabic script in the ninth century CE; it has 32 letters, 28 taken over from Arabic and 4 new letters added to represent Persian phonemes (/p/, /č/, /ž/, /g/). Arabic /θ/ and /ṣ/ collapse to Persian /s/, Arabic /ð/, /ḍ/, /ð̣/ collapse to Persian /z/, and Arabic /ṭ/ becomes Persian /t/. The phonemic inventory of Early Classical Persian was augmented with the glottal stop, which originated in the two separate Arabic phonemes /ʔ/ and /ʕ/.

\subsubsection{Modern Persian dialects}

This section highlights phonological features of modern Persian dialects that were the result of contact-induced language change under RL agentivity, either with Arabic or with Classical Persian and subsequently MSP.

\subsubsubsection{Adoption of Arabic pharyngeal sounds}

The two Arabic pharyngeal sounds undergo phonological integration in New Persian: the voiceless pharyngeal fricative /ḥ/ is pronounced as a voiceless glottal fricative /h/, and the voiced pharyngeal fricative /ʕ/ as a glottal stop /ʔ/. The dialects of Dizfūl and Šūštar have acquired pharyngeal sounds directly from Arabic, which occur in Arabic loanwords: \textit{ʕaǧīb} ‘strange’, \textit{baʕd} ‘after’ \citep{MacKinnon2015}. The dialect of Jarkūya shares this feature: \textit{ḥasüd} ‘jealous’, \textit{ǧimʕa} ‘Friday’ \citep{Borjian2008}.

The dialect of Kulāb in Tajikistan also borrows Arabic pharyngeal sounds in words of Arabic origin: \textit{ʕaib} ‘flaw’, \textit{daʕvō} ‘claim’, \textit{mıʕalim} ‘teacher’, \textit{ḥıkımat} ‘wisdom’, \textit{sōḥib} ‘owner’. Arabic pharyngeal sounds also occur in a few Persian/Tajiki words (\textit{ʕasp} ‘horse’, \textit{ḥamsōya} ‘neighbor’). Interestingly, the pharyngealized form for ‘horse’ occurs far and wide within the Iranian linguistic domain, as \textit{ʕasb} in the Lurī dialect of Šūštar, in Ḫānsāri and Caucasian Tātī. In the Arab Gulf states, the \textit{ʕAǧam}, ethnic Persians holding Kuwaiti, Emirati and other Gulf citizenship, pronounce Arabic loanwords in their Persian speech with pharyngeal sounds.

\subsubsubsection{Dropping of Arabic pharyngeal sounds}

In several modern Persian dialects, the voiceless pharyngeal fricative /ḥ/ is absent. The preceding vowel is lengthened or the subsequent vowel disappears too, e.g. \textit{mūtāǧ} ‘in need, destitute’ < MSA/MSP \textit{muḥtāǧ} (\citealt{Īzadpanāh2001}: 190), \textit{ṣārā} ‘desert’ < MSA/MSP \textit{ṣaḥrā} \citep[15]{Sarlak2002}, \textit{ṣāb} ‘owner’ < MSA/MSP \textit{ṣāḥib} (\citealt{Ṣarrāfī1996}: 135), \textit{mulāẓa} ‘consideration, observation’ < MSP \textit{mulāḥiẓa}, cf. MSA \textit{mulāḥað̣a} (\citealt{Ṣarrāfī1996}: 188), \textit{ṣul} ‘peace’ < MSA/MSP \textit{ṣulḥ} \citep{Stilo2001}, \textit{ēsās} ‘feeling’ < MSA/MSP \textit{iḥsās} (\citealt{Salāmī2004}: 160–161). In Kirmān, the sound change /uḥ/ > /ā/ is attested, e.g. \textit{fāš} ‘insult’ < MSA/MSP \textit{fuḥš} \citep{Borjian2017}.

    The voiced pharyngeal fricative /ʕ/, pronounced as a glottal stop in MSP, can also be dropped. This may result in vowel lengthening: \textit{māṭal} ‘idle’ < MSA/MSP \textit{muʕaṭṭal} (\citealt{Ṣarrāfī1996}: 184), \textit{māmila} ‘transaction’ < NewP \textit{muʕāmila}, cf. MSA \textit{muʕāmala} (\citealt{Ṣarrāfī1996}: 184; \citealt{Sarlak2002}: 15), \textit{rubbi} \textit{sāt} ‘quarter hour’ < MSP \textit{rubʕ} \textit{sāʕat}, cf. MSA \textit{rubʕ} \textit{sāʕa} (\citealt{Ṣarrāfī1996}: 108), \textit{mānī} ‘meaning’ < MSP \textit{maʕnī}, cf. MSA \textit{maʕnā} \citep[15]{Sarlak2002}, \textit{mōǧiza} ‘miracle’ < MSA/MSP \textit{muʕǧiza} (\citealt{Īzadpanāh2001}: 190), \textit{tāǧub} \~{} \textit{tāǧuv} ‘surprise, wonder’ < MSA/MSP \textit{taʕaǧǧub} (\citealt{Salāmī2004}: 162–163), \textit{rāyat} ‘regard’ < MSP \textit{riʕāyat}, cf. MSA \textit{riʕāya} (\citealt{Ṣarrāfī1996}: 107).

\subsubsubsection{Dropping of the Arabic voiceless glottal fricative /h/}

The voiceless glottal fricative disappears in closed syllables in many Persian dialects, resulting in occasional vowel lengthening: \textit{ṭārat} ‘cleanliness’ < MSP \textit{ṭahārat}, cf. MSA \textit{ṭahāra} \citep[76]{Sarlak2002}, \textit{nāal} ‘impolite’ < MSP \textit{nāahl} (\citealt{Īzadpanāh2001}: 192).

\subsubsubsection{Miscellaneous sound changes}

A range of additional consonant developments and shifts can be attested in Persian dialects. Some of these developments include:

\begin{itemize}
\item[] 
/ʕ/ > /ḥ/: In Lurī and the dialect of Jarkūya, a shift occurs from the voiced to the voiceless pharyngeal: \textit{ḥilāǧ} ‘cure’ < MSA/MSP \textit{ʕilāǧ} (\citealt{Īzadpanāh2001}: 207), \textit{ṭaḥna} ‘sarcasm’ < MSA/MSP \textit{ṭaʕna} \citep{Borjian2008}.

\item[]
/ḥ/ > /ʔ/ occurring with occasional metathesis: \textit{ṭaʔr} ‘plan’ < MSA/MSP \textit{ṭarḥ} (\citealt{Ṣarrāfī1996}: 137), \textit{maʔla} ‘city quarter’ < MSA/MSP \textit{maḥalla} (\citealt{Ṣarrāfī1996}: 188), \textit{maʔala} ‘city quarter’ (\citealt{NaǧībiFīni2002}: 133).

\item[]
/h/ > /ʔ/: \textit{muʔlat} ‘deadline, respite’ < MSP \textit{muhlat}, cf. MSA \textit{muhla} (\citealt{Ṣarrāfī1996}: 190).

\item[]
/θ/ > /t/: This shift is also common in several Arabic dialects, e.g. in Egypt and Morocco: \textit{mīrāt} ‘heritage’ < MSP \textit{mīrās}, cf. MSA \textit{mīrāθ} (\citealt{Īzadpanāh2001}: 190).

\item[]
Word-final /b/ and /f/ > /m/: \textit{naǧīm} ‘noble’ < MSA/MSP \textit{naǧīb} (\citealt{Īzadpanāh2001}: 193), \textit{niṣm} ‘half’ < MSA/MSP \textit{niṣf} (\citealt{Īzadpanāh2001}: 195).

\item[]
/r/ > /l/: in Kirmān, \textit{zilar} \~{} \textit{zilal} ‘damage, loss’ < MSP \textit{ẓarar}, cf. MSA \textit{ḍarar} (\citealt{Ṣarrāfī1996}: 136; \citealt{Dānišgar1995}: 163), \textit{ḥaṣīl} ‘straw mat’ < MSA/MSP \textit{ḥaṣīr} (\citealt{Ṣarrāfī1996}: 85), \textit{qulfa} ‘small room for summer resting’ < MSA \textit{ɣurfa} ‘room’ (\citealt{Fāẓilī2004}: 151).

\item[]
Arabic voiceless dental emphatic /ṭ/ > /d/: \textit{mudbaḫ} \~{} \textit{madbaḫ} ‘kitchen’ < MSA \textit{maṭbaḫ} (\citealt{Ṣarrāfī1996}: 186; not attested in MSP), \textit{mudbaq} in Baḫtiārī \citep[251]{Sarlak2002}.

\item[]
/b/ > /f/: \emph{muftilā} ‘afflicted’ < MSP \textit{mubtilā}, cf. MSA  \textit{mubtalā} \citep{Borjian2017}.

\item[]
Medial and word-final /b/ > /v/: in Baḫtiārī, \textit{ādāv} ‘customs’ < MSA/MSP \textit{ādāb} \citep[15]{Sarlak2002}, \textit{ʕajīv} ‘strange’ < MSA/MSP \textit{ʕajīb} \citep[25]{Sarlak2002}, \textit{qavīla} ‘tribe’ < MSA/MSP \textit{qabīla} \citep[199]{Sarlak2002}.

\item[]
Word-initial /ḫ/ > /h/: in northern Lurī and Baḫtiārī, \textit{hāla} ‘aunt’ < MSA/MSP \textit{ḫāla} (\citealt{Īzadpanāh2001}: 204).

\item[]
/q/ > /k/: \textit{kabīla} ‘tribe’ < MSA/MSP \textit{qabīla} (\citealt{NaǧībiFīni2002}: 21).

\item[]
/ɣ/ > /q/: \textit{šuql} ‘occupation’ < MSP/MSA \textit{šuɣl} \citep{Stilo2001}.

\item[]
/ǧ/ > /y/: direct borrowing from Khuzestan Arabic dialects, \textit{mailis} ‘council’ < MSA/MSP \textit{maǧlis} (\citealt{Sarlak2002}: 260; \citealt{Fāẓilī2004}: 165).

\item[]
Metathesis: \textit{qulf} ‘lock’ < MSA/MSP \textit{qufl} (\citealt{Salāmī2004}: 84–85; \citealt{ImāmAhwāzī2000}: 146), \textit{ṣuḥb} ‘morning’ < MSA/MSP \textit{ṣubḥ} (\citealt{Dānišgar1995}: 161; \citealt{NaǧībiFīni2002}: 23).
\end{itemize}

The full /t/ of the \textit{tāʔ} \textit{marbūṭa} appears on words where it is absent in MSP: \textit{ḥalmat} ‘attack’ < MSA/MSP \textit{ḥamla} (\citealt{Īzadpanāh2001}: 207), \textit{ḥaǧāmat} ‘cupping’ < MSA/MSP \textit{ḥaǧāma} (\citealt{Salāmī2004}: 92–93). This was a typical feature of Classical Persian literature.

\subsection{Morphosyntax}

Several Arabic morphosyntactic features were transferred to New Persian in the realm of nominal morphology under RL agentivity. These features encompass sound and broken plural forms (\textit{musāfirīn} ‘passengers’, \textit{tablīɣāt} ‘propaganda’, \textit{dihāt} ‘villages’, \textit{ḥuqūq} ‘rights’), possessive constructions (\textit{fāriɣ} \textit{ut-taḥṣīl} ‘graduate’, \textit{wāǧib} \textit{ul-iǧrā} ‘peremptory’) and occasional gender agreement in lexicalized expressions (\textit{quwwa-yi} \textit{darrāka} ‘perceptive power’). Word formation has been an active method of transferring Arabic lexical elements into New Persian from early on, either by way of derivation (\textit{diḫālat} ‘interference’ < MSA \textit{mudāḫala}, \textit{awlā-tar} ‘superior’ < MSA \textit{awlā}, \textit{raqṣīdan} ‘to dance’, \textit{aks̱aran} ‘most, generally’ < MSA \textit{akθar} ‘more, most’) or compounding. Compounding is a highly developed process of enlarging the New Persian vocabulary. It is manifest in lexical compounds (\textit{taɣẕia-šinās} ‘nutritionist’, \textit{ḫiānat-kārāna} ‘perfidiously’) and phrasal compounds (\textit{iṭāʕat} \textit{kardan} ‘to obey’, \textit{ʕadam-i} \textit{wuǧūd} ‘non-existence’, \textit{ʕala} \textit{l-ḫuṣūṣ} ‘particularly’).

\subsection{Lexicon}

\subsubsection{Arabic lexicon in New Persian}

Contact-induced language change manifests itself most strikingly in the lexicon transferred from Arabic to New Persian under RL agentivity. The earliest loanwords entered New Persian during the ninth–tenth centuries. This process occurred smoothly, as the phonological inventory of Early Classical Persian was likely close to that of Middle Persian and also close to that of Classical Arabic.\footnote{In Early Classical Persian, short vowels were likely pronounced as /u/ and /i/, and the \textit{alif} as /ā/. In MSP, the pronunciation is /o/, /e/ and /ɒ/.} The influx of Arabic loanwords has unabatedly continued over the centuries until now. To showcase a recent example of Arabic vocabulary in Modern Persian, below are titles of articles from \textit{Hamšahrī} ‘fellow citizen’, a major Iranian national newspaper, taken from its 29th January 2018 edition. Arabic words are highlighted in boldface:

\ea
\ea 
\gll \textbf{kulliyyāt}-i \textbf{lāyiḥa}-yi būdǧa-yi sāl-i 97-i \textbf{kull}-i kišwar \textbf{radd} šud\\
total.\textsc{pl-gen} bill-\textsc{gen} budget-\textsc{gen} year-\textsc{gen} 97-\textsc{gen} whole-\textsc{gen} country reject be.\textsc{pst.3sg}\\
    \glt ‘The total budget bill of the year 1997 for the whole country was rejected.’

\ex 
\gll \textbf{daʕwat} {az} {tihrānīhā} barā-yi \textbf{ihdā}-yi ḫūn \textbf{asāmī}-yi \textbf{marākiz}-i \textbf{faʕʕāl}\\
call from Tehrani.\textsc{pl} for-\textsc{gen} donation-\textsc{gen} blood name.\textsc{pl-gen} center.\textsc{pl-gen} active\\
    \glt ‘Calling the residents of Tehran to donate blood. Names of active centers.’

\ex 
\gll \textbf{iḥrāz}-i \textbf{huwiyyat} dar \textbf{muʕāmilāt}-i \textbf{milkī} {bā} {kārt-i} hūšmand-i \textbf{millī} {anǧām} {mī-šaw-ad}\\
authentication-\textsc{gen} identity in transaction.\textsc{pl-gen} proprietary with card-\textsc{gen} smart-\textsc{gen} national complete \textsc{prs}-be-\textsc{3sg}\\ 
    \glt‘Personal authentication in real estate transactions is done with the national smart card.’
\z
\z


In the Arabic lexicon of New Persian, further characteristics can be observed, such as phonetic changes (NewP \textit{maʔnī} ‘meaning’ < MSA \textit{maʕnā}, NewP \textit{madrisa} ‘school’ < MSA \textit{madrasa}, NewP \textit{šikl} ‘shape, form’ < SA \textit{šakl}), where in some cases the Persian pronunciation may follow Arabic dialectal forms, semantic changes (NewP \textit{kitābat} ‘writing’ and \textit{kitāba} ‘inscription’ < MSA \textit{kitāba} ‘writing’, NewP \textit{ṣuḥbat} ‘speech’ < MSA \textit{ṣuḥba} ‘companionship’), and occasional \textit{imāla} in elevated or poetic style (NewP \textit{ḥiǧīz} < MSA \textit{ḥiǧāz}).

\subsubsection{Arabic lexicon in Persian dialects}

Arabic loanwords affect Persian dialects in two ways that differ from MSP: i) semantic changes, where Arabic lexemes assume new meanings unattested in both MSA and MSP: in Kirmān \textit{ðāt} ‘age’   (\citealt{Ṣarrāfī1996}: 106) < MSA/MSP ‘self, soul, essence, nature’, \textit{ðātī} ‘old’ < MSA/MSP ‘own, personal’; ii) lexemes and expressions directly borrowed from Arabic, and not attested in MSP: in Šūštar, \textit{ḥaya} ‘snake’ < MSA \textit{ḥayya}, MSP \textit{mār} (\citealt{Fāẓilī2004}: 140), \textit{ṭayyāra} ‘airplane’ < Arabic dialects \textit{ṭayyāra}, MSA \textit{ṭāʔira}, MSP \textit{hawāpaimā} (\citealt{Fāẓilī2004}: 150), \textit{ṣaḥn} ‘bowl, dish’ < MSA \textit{ṣaḥn}, MSP \textit{bušqāb} (\citealt{Fāẓilī2004}: 150), \textit{ṭabaq} ‘plate, tray’ < MSA \textit{ṭabaq}, MSP \textit{sīnī} (\citealt{Fāẓilī2004}: 150), in Fīn, \textit{mismāl} ‘nail’ < MSA \textit{mismār}, MSP \textit{mīḫ} ‘nail’ (\citealt{NaǧībiFīni2002}: 133), in Kirmān, \textit{aḥad} \textit{un-nās} ‘nobody, somebody’ < MSA \textit{aḥad} \textit{un-nās}, MSP \textit{hīčkas} ‘nobody’, \textit{kasī} ‘somebody’ (\citealt{Ṣarrāfī1996}: 33).

On the Persian Gulf coast of Iran, due to linguistic, economic and commercial connections with the Arabian Peninsula, Persian dialects have incorporated a number of Arabic technical terms relating to pearling, fishing and traditional shipbuilding from Gulf Arabic: \textit{muḥār} ‘shellfish, oysters’ (cf. MSA \textit{maḥār}), \textit{giyās} ‘measure, gauge’ (< GA \textit{giyās}, cf. MSA \textit{qiyās}), \textit{mīdāf} ‘helm (boat)’ (< GA \textit{mīdāf}, cf. MSA \textit{miǧdāf}), \textit{māčila} ‘meal (on a boat)’ (< GA \textit{māčila}, cf. MSA \textit{maʔkūl}). Two neighborhoods in the town of Bandar Linga (opposite Dubai, 180 km west of Bandar ʕAbbās) are called \textit{Maḥalla-yi} \textit{Baḥrainī} ‘Bahraini Quarter’ and \textit{Maḥalla-yi} \textit{Sammāčī} ‘Fishers’ Quarter’ (< GA \textit{sammāč}, cf. MSA \textit{sammāk}) (\citealt{Baḫtiyārī1990}: 137–138).

\section{Conclusion}

Although Arabic–Persian language contact has been a well-known phenomenon for centuries, academic research dedicated to this topic is far from abundant. Throughout the centuries, Persian writers and poets used Arabic lexical elements in new meanings or coined non-standard Perso-Arabic lexemes based on Arabic derivational patterns. Idiosyncratic features of individual Persian writers should be examined separately before compiling a comprehensive review of this contact-induced language change. Substantial fieldwork needs to be conducted to describe the bilingualism of ethnic Arab communities of Iran and ethnic Persians in Arabic-speaking countries. Additionally, it is essential for linguists to look into Arabic influence on Modern Persian dialects and Iranian languages other than New Persian. This will help scholars understand the scale and depth of how Arabic has shaped Iranian languages for the past thousand years.

Contact-induced language change in New Iranian languages primarily transpired under RL agentivity. It should be noted, however, that medieval Persian literati were so well-versed in Arabic due to its prestige and dominance, that their bilingualism may have enabled convergence in Arabic–Persian language contact.

\section*{Further reading}

\citet{Asbaghi1987} gathers eight hundred Persian words of Arabic origin in twenty-three groups and analyzes the semantic changes they underwent when transferred from Arabic to New Persian.\\
\citet{Gazsi2011} gives an overview of Arabic–Persian language contact from pre-Islamic times up to the modern era, also touching on Arabic dialects in Iran. It additionally provides a brief analysis of Arabic morphosyntactic features in New Persian.\\
\citet{Ṣādiqī2011} discusses a range of Arabic phonological, grammatical and semantic elements in New Persian.

\section*{Acknowledgements}

I would like to express my gratitude to Prof. Éva Jeremiás, Prof. Werner Arnold and Prof. Ali Ashraf Sadeghi for their support while I was working on Arabic–Persian language contact. I am thankful to members of the ‘Arabs of the Coast’ in the UAE, especially Sheikh Ibrahim, Sheikh Abdulrahman, Dr. Abdullah, Walid, Ahmed, and many others for providing language data in GA.

\section*{Abbreviations}
\begin{tabularx}{.5\textwidth}{@{}lQ@{}}
BCE & before Common Era\\
CE & Common Era\\
GA & Gulf Arabic\\
\textsc{gen}   &  genitive\\
MSA &  Modern Standard Arabic\\
MSP  &  Modern Standard Persian\\
NewP   &  New Persian\\
\textsc{pl}   &  plural\\
\textsc{prs}   &  past\\
\textsc{pst}   &  past\\
RL &  recipient language \\
\end{tabularx}%



{\sloppy\printbibliography[heading=subbibliography,notkeyword=this]}
\end{document}
