\documentclass[output=paper]{langsci/langscibook} 
\author{Stefano Manfredi\affiliation{CNRS, SeDyL}}
\title{Contact and calquing}
\abstract{The notion of calquing refers to the transfer of semantic and syntactic patterns deprived of morphophonological matter. By providing examples of lexical and grammatical calques in a number of Arabic dialects and Arabic-based contact languages, this chapter identifies ways to relate the process of calquing to Van Coetsem’s psycholinguistic principle of language dominance.}
\IfFileExists{../localcommands.tex}{
  % add all extra packages you need to load to this file 
\usepackage{graphicx}
\usepackage{tabularx}
\usepackage{amsmath} 
\usepackage{multicol}
\usepackage{lipsum}
\usepackage[stable]{footmisc}
\usepackage{adforn}
%%%%%%%%%%%%%%%%%%%%%%%%%%%%%%%%%%%%%%%%%%%%%%%%%%%%
%%%                                              %%%
%%%           Examples                           %%%
%%%                                              %%%
%%%%%%%%%%%%%%%%%%%%%%%%%%%%%%%%%%%%%%%%%%%%%%%%%%%%
% remove the percentage signs in the following lines
% if your book makes use of linguistic examples
\usepackage{./langsci/styles/langsci-optional} 
\usepackage{./langsci/styles/langsci-lgr}
\usepackage{morewrites} 
%% if you want the source line of examples to be in italics, uncomment the following line
% \def\exfont{\it}

\usepackage{enumitem}
\newlist{furtherreading}{description}{1}
\setlist[furtherreading]{font=\normalfont,labelsep=\widthof{~},noitemsep,align=left,leftmargin=\parindent,labelindent=0pt,labelwidth=-\parindent}
\usepackage{phonetic}
\usepackage{chronosys,tabularx}
\usepackage{csquotes}
\usepackage[stable]{footmisc} 

\usepackage{langsci-bidi}
\usepackage{./langsci/styles/langsci-gb4e} 

  \makeatletter
\let\thetitle\@title
\let\theauthor\@author 
\makeatother

\newcommand{\togglepaper}[1][0]{ 
  \bibliography{../localbibliography}
  \papernote{\scriptsize\normalfont
    \theauthor.
    \thetitle. 
    To appear in: 
    Christopher Lucas and Stefano Manfredi (eds.),  
    Arabic and contact-induced language change
    Berlin: Language Science Press. [preliminary page numbering]
  }
  \pagenumbering{roman}
  \setcounter{chapter}{#1}
  \addtocounter{chapter}{-1}
}

\newfontfamily\Parsifont[Script=Arabic]{ScheherazadeRegOT_Jazm.ttf} 
\newcommand{\arabscript}[1]{\RL{\Parsifont #1}}
\newcommand{\textarabic}[1]{{\arabicfont #1}}

\newcommand{\textstylest}[1]{{\color{red}#1}}

\patchcmd{\mkbibindexname}{\ifdefvoid{#3}{}{\MakeCapital{#3}
}}{\ifdefvoid{#3}{}{#3 }}{}{\AtEndDocument{\typeout{mkbibindexname could
not be patched.}}}

%command for italic r with dot below with horizontal correction to put the dot in the prolongation of the vertical stroke
%for some reason, the dot is larger than expected, so we explicitly reduce the font size (to \small)
%for the time being, the font is set to an absolute value. To be more robust, a relative reduction would be better, but this might not be required right now
\newcommand{\R}{r\kern-.05ex{\small{̣}}\kern.05ex}


\DeclareLabeldate{%
    \field{date}
    \field{year}
    \field{eventdate}
    \field{origdate}
    \field{urldate}
    \field{pubstate}
    \literal{nodate}
}

\renewbibmacro*{addendum+pubstate}{% Thanks to https://tex.stackexchange.com/a/154367 for the idea
  \printfield{addendum}%
  \iffieldequalstr{labeldatesource}{pubstate}{}
  {\newunit\newblock\printfield{pubstate}}
}
 
  %% hyphenation points for line breaks
%% Normally, automatic hyphenation in LaTeX is very good
%% If a word is mis-hyphenated, add it to this file
%%
%% add information to TeX file before \begin{document} with:
%% %% hyphenation points for line breaks
%% Normally, automatic hyphenation in LaTeX is very good
%% If a word is mis-hyphenated, add it to this file
%%
%% add information to TeX file before \begin{document} with:
%% %% hyphenation points for line breaks
%% Normally, automatic hyphenation in LaTeX is very good
%% If a word is mis-hyphenated, add it to this file
%%
%% add information to TeX file before \begin{document} with:
%% \include{localhyphenation}
\hyphenation{
affri-ca-te
affri-ca-tes
com-ple-ments
homo-phon-ous
start-ed
Meso-potam-ian
morpho-phono-logic-al-ly
morpho-phon-em-ic-s
Palestin-ian
re-present-ed
Ki-nubi
ḥawār-iyy-ūn
archa-ic-ity
fuel-ed
de-velop-ment
pros-od-ic
Arab-ic
in-duced
phono-logy
possess-um
possess-ive-s
templ-ate
spec-ial
espec-ial-ly
nat-ive
pass-ive
clause-s
potent-ial-ly
Lusignan
commun-ity
tobacco
posi-tion
Cushit-ic
Middle
with-in
re-finit-iz-ation
langu-age-s
langu-age
diction-ary
glossary
govern-ment
eight
counter-part
nomin-al
equi-valent
deont-ic
ana-ly-sis
Malt-ese
un-fortun-ate-ly
scient-if-ic
Catalan
Occitan
ḥammāl
cross-linguist-ic-al-ly
predic-ate
major-ity
ignor-ance
chrono-logy
south-western
mention-ed
borrow-ed
neg-ative
de-termin-er
European
under-mine
detail
Oxford
Socotra
numer-ous
spoken
villages
nomad-ic
Khuze-stan
Arama-ic
Persian
Ottoman
Ottomans
Azeri
rur-al
bi-lingual-ism
borrow-ing
prestig-ious
dia-lects
dia-lect
allo-phone
allo-phones
poss-ible
parallel
parallels
pattern
article
common-ly
respect-ive-ly
sem-antic
Moroccan
Martine
Harrassowitz
Grammatic-al-ization
grammatic-al-ization
Afro-asiatica
Afro-asiatic
continu-ation
Semit-istik
varieties
mono-phthong
mono-phthong-ized
col-loquial
pro-duct
document-ary
ex-ample-s
ex-ample
termin-ate
element-s
Aramaeo-grams
Centr-al
idioms
Arab-ic
Dadan-it-ic
sub-ordin-ator
Thamud-ic
difficult
common-ly
Revue
Bovingdon
under
century
attach
attached
bundle
graph-em-ic
graph-emes
cicada
contrast-ive
Corriente
Andalusi
Kossmann
morpho-logic-al
inter-action
dia-chroniques
islámica
occid-ent-al-ismo
dialecto-logie
Reichert
coloni-al
Milton
diphthong-al
linguist-ic
linguist-ics
affairs
differ-ent
phonetic-ally
kilo-metres
stabil-ization
develop-ments
in-vestig-ation
Jordan-ian
notice-able
level-ed
migrants
con-dition-al
certain-ly
general-ly
especial-ly
af-fric-ation
Jordan
counter-parts
com-plication
consider-ably
inter-dent-al
com-mun-ity
inter-locutors
com-pon-ent
region-al
socio-historical
society
simul-taneous
phon-em-ic
roman-ization
Classic-al
funeral
Kurmanji
pharyn-geal-ization
vocab-ulary
phon-et-ic
con-sonant
con-sonants
special-ized
latter
latters
in-itial
ident-ic-al
cor-relate
geo-graphic-al-ly
Öpengin
Kurd-ish
in-digen-ous
sunbul
Christ-ian
Christ-ians
sekin-în
fatala
in-tegration
dia-lect-al
Matras
morpho-logy
in-tens-ive
con-figur-ation
im-port-ant
com-plement
ḥaddād
e-merg-ence
Benjmamins
struct-ure
em-pir-ic-al
Orient-studien
Anatolia
American
vari-ation
Jastrow
Geoffrey
Yarshater
Ashtiany
Edmund
Mahnaz
En-cyclo-pædia
En-cyclo-paedia
En-cyclo-pedia
Leiden
dia-spora
soph-is-ic-ated
Sasan-ian
every-day
domin-ance
Con-stitu-tion-al
religi-ous
sever-al
Manfredi
re-lev-ance
re-cipi-ent
pro-duct-iv-ity
turtle
Morocco
ferman
Maghreb-ian
algérien
stand-ard
systems
Nicolaï
Mouton
mauritani-en
Gotho-burg-ensis
socio-linguist-ique
plur-al
archiv-al
Arab-ian
drop-ped
dihāt
de-velop-ed
ṣuḥbat
kitāba
kitābat
com-mercial
eight-eenth
region
Senegal
mechan-ics
Maur-itan-ia
Ḥassān-iyya
circum-cision
cor-relation
labio-velar-ization
vowel
vowels
cert-ain
īggīw
series
in-tegrates
dur-ative
inter-dent-als
gen-itive
Tuareg
tălămut
talawmāyət
part-icular
part-icular-ly
con-diment
vill-age
bord-er
polit-ical
Wiesbaden
Uni-vers-idad
Geuthner
typo-logie
Maur-itanie
nomades
Maur-itan-ian
dia-lecto-logy
Sahar-iennes
Uni-vers-ity
de-scend-ants
NENA-speak-ing
speak-ing
origin-al
re-captured
in-habit-ants
ethnic
minor-it-ies
drama-tic
local
long-stand-ing
regions
Nineveh
settle-ments
Ṣəndor
Mandate
sub-stitut-ing
ortho-graphy
re-fer-enced
origin-ate
twenti-eth
typ-ic-al-ly
Hobrack
never-the-less
character-ist-ics
character-ist-ic
masc-uline
coffee
ex-clus-ive-ly
verb-al
re-ana-ly-se-d
simil-ar-ities
de-riv-ation
im-pera-tive
part-iciple
dis-ambi-gu-ation
dis-ambi-gu-a-ing
phen-omen-on
phen-omen-a
traktar
com-mun-ity
com-mun-ities
dis-prefer-red
ex-plan-ation
con-struction
wide-spread
us-ual-ly
region-al
Bulut
con-sider-ation
afro-asia-tici
Franco-Angeli
Phono-logie
Volks-kundliche
dia-lectes
dia-lecte
select-ed
dis-appear-ance
media
under-stand-able
public-ation
second-ary
e-ject-ive
re-volu-tion
re-strict-ive
Gasparini
mount-ain
mount-ains
yellow
label-ing
trad-ition-al-ly
currently
dia-chronic
}
\hyphenation{
affri-ca-te
affri-ca-tes
com-ple-ments
homo-phon-ous
start-ed
Meso-potam-ian
morpho-phono-logic-al-ly
morpho-phon-em-ic-s
Palestin-ian
re-present-ed
Ki-nubi
ḥawār-iyy-ūn
archa-ic-ity
fuel-ed
de-velop-ment
pros-od-ic
Arab-ic
in-duced
phono-logy
possess-um
possess-ive-s
templ-ate
spec-ial
espec-ial-ly
nat-ive
pass-ive
clause-s
potent-ial-ly
Lusignan
commun-ity
tobacco
posi-tion
Cushit-ic
Middle
with-in
re-finit-iz-ation
langu-age-s
langu-age
diction-ary
glossary
govern-ment
eight
counter-part
nomin-al
equi-valent
deont-ic
ana-ly-sis
Malt-ese
un-fortun-ate-ly
scient-if-ic
Catalan
Occitan
ḥammāl
cross-linguist-ic-al-ly
predic-ate
major-ity
ignor-ance
chrono-logy
south-western
mention-ed
borrow-ed
neg-ative
de-termin-er
European
under-mine
detail
Oxford
Socotra
numer-ous
spoken
villages
nomad-ic
Khuze-stan
Arama-ic
Persian
Ottoman
Ottomans
Azeri
rur-al
bi-lingual-ism
borrow-ing
prestig-ious
dia-lects
dia-lect
allo-phone
allo-phones
poss-ible
parallel
parallels
pattern
article
common-ly
respect-ive-ly
sem-antic
Moroccan
Martine
Harrassowitz
Grammatic-al-ization
grammatic-al-ization
Afro-asiatica
Afro-asiatic
continu-ation
Semit-istik
varieties
mono-phthong
mono-phthong-ized
col-loquial
pro-duct
document-ary
ex-ample-s
ex-ample
termin-ate
element-s
Aramaeo-grams
Centr-al
idioms
Arab-ic
Dadan-it-ic
sub-ordin-ator
Thamud-ic
difficult
common-ly
Revue
Bovingdon
under
century
attach
attached
bundle
graph-em-ic
graph-emes
cicada
contrast-ive
Corriente
Andalusi
Kossmann
morpho-logic-al
inter-action
dia-chroniques
islámica
occid-ent-al-ismo
dialecto-logie
Reichert
coloni-al
Milton
diphthong-al
linguist-ic
linguist-ics
affairs
differ-ent
phonetic-ally
kilo-metres
stabil-ization
develop-ments
in-vestig-ation
Jordan-ian
notice-able
level-ed
migrants
con-dition-al
certain-ly
general-ly
especial-ly
af-fric-ation
Jordan
counter-parts
com-plication
consider-ably
inter-dent-al
com-mun-ity
inter-locutors
com-pon-ent
region-al
socio-historical
society
simul-taneous
phon-em-ic
roman-ization
Classic-al
funeral
Kurmanji
pharyn-geal-ization
vocab-ulary
phon-et-ic
con-sonant
con-sonants
special-ized
latter
latters
in-itial
ident-ic-al
cor-relate
geo-graphic-al-ly
Öpengin
Kurd-ish
in-digen-ous
sunbul
Christ-ian
Christ-ians
sekin-în
fatala
in-tegration
dia-lect-al
Matras
morpho-logy
in-tens-ive
con-figur-ation
im-port-ant
com-plement
ḥaddād
e-merg-ence
Benjmamins
struct-ure
em-pir-ic-al
Orient-studien
Anatolia
American
vari-ation
Jastrow
Geoffrey
Yarshater
Ashtiany
Edmund
Mahnaz
En-cyclo-pædia
En-cyclo-paedia
En-cyclo-pedia
Leiden
dia-spora
soph-is-ic-ated
Sasan-ian
every-day
domin-ance
Con-stitu-tion-al
religi-ous
sever-al
Manfredi
re-lev-ance
re-cipi-ent
pro-duct-iv-ity
turtle
Morocco
ferman
Maghreb-ian
algérien
stand-ard
systems
Nicolaï
Mouton
mauritani-en
Gotho-burg-ensis
socio-linguist-ique
plur-al
archiv-al
Arab-ian
drop-ped
dihāt
de-velop-ed
ṣuḥbat
kitāba
kitābat
com-mercial
eight-eenth
region
Senegal
mechan-ics
Maur-itan-ia
Ḥassān-iyya
circum-cision
cor-relation
labio-velar-ization
vowel
vowels
cert-ain
īggīw
series
in-tegrates
dur-ative
inter-dent-als
gen-itive
Tuareg
tălămut
talawmāyət
part-icular
part-icular-ly
con-diment
vill-age
bord-er
polit-ical
Wiesbaden
Uni-vers-idad
Geuthner
typo-logie
Maur-itanie
nomades
Maur-itan-ian
dia-lecto-logy
Sahar-iennes
Uni-vers-ity
de-scend-ants
NENA-speak-ing
speak-ing
origin-al
re-captured
in-habit-ants
ethnic
minor-it-ies
drama-tic
local
long-stand-ing
regions
Nineveh
settle-ments
Ṣəndor
Mandate
sub-stitut-ing
ortho-graphy
re-fer-enced
origin-ate
twenti-eth
typ-ic-al-ly
Hobrack
never-the-less
character-ist-ics
character-ist-ic
masc-uline
coffee
ex-clus-ive-ly
verb-al
re-ana-ly-se-d
simil-ar-ities
de-riv-ation
im-pera-tive
part-iciple
dis-ambi-gu-ation
dis-ambi-gu-a-ing
phen-omen-on
phen-omen-a
traktar
com-mun-ity
com-mun-ities
dis-prefer-red
ex-plan-ation
con-struction
wide-spread
us-ual-ly
region-al
Bulut
con-sider-ation
afro-asia-tici
Franco-Angeli
Phono-logie
Volks-kundliche
dia-lectes
dia-lecte
select-ed
dis-appear-ance
media
under-stand-able
public-ation
second-ary
e-ject-ive
re-volu-tion
re-strict-ive
Gasparini
mount-ain
mount-ains
yellow
label-ing
trad-ition-al-ly
currently
dia-chronic
}
\hyphenation{
affri-ca-te
affri-ca-tes
com-ple-ments
homo-phon-ous
start-ed
Meso-potam-ian
morpho-phono-logic-al-ly
morpho-phon-em-ic-s
Palestin-ian
re-present-ed
Ki-nubi
ḥawār-iyy-ūn
archa-ic-ity
fuel-ed
de-velop-ment
pros-od-ic
Arab-ic
in-duced
phono-logy
possess-um
possess-ive-s
templ-ate
spec-ial
espec-ial-ly
nat-ive
pass-ive
clause-s
potent-ial-ly
Lusignan
commun-ity
tobacco
posi-tion
Cushit-ic
Middle
with-in
re-finit-iz-ation
langu-age-s
langu-age
diction-ary
glossary
govern-ment
eight
counter-part
nomin-al
equi-valent
deont-ic
ana-ly-sis
Malt-ese
un-fortun-ate-ly
scient-if-ic
Catalan
Occitan
ḥammāl
cross-linguist-ic-al-ly
predic-ate
major-ity
ignor-ance
chrono-logy
south-western
mention-ed
borrow-ed
neg-ative
de-termin-er
European
under-mine
detail
Oxford
Socotra
numer-ous
spoken
villages
nomad-ic
Khuze-stan
Arama-ic
Persian
Ottoman
Ottomans
Azeri
rur-al
bi-lingual-ism
borrow-ing
prestig-ious
dia-lects
dia-lect
allo-phone
allo-phones
poss-ible
parallel
parallels
pattern
article
common-ly
respect-ive-ly
sem-antic
Moroccan
Martine
Harrassowitz
Grammatic-al-ization
grammatic-al-ization
Afro-asiatica
Afro-asiatic
continu-ation
Semit-istik
varieties
mono-phthong
mono-phthong-ized
col-loquial
pro-duct
document-ary
ex-ample-s
ex-ample
termin-ate
element-s
Aramaeo-grams
Centr-al
idioms
Arab-ic
Dadan-it-ic
sub-ordin-ator
Thamud-ic
difficult
common-ly
Revue
Bovingdon
under
century
attach
attached
bundle
graph-em-ic
graph-emes
cicada
contrast-ive
Corriente
Andalusi
Kossmann
morpho-logic-al
inter-action
dia-chroniques
islámica
occid-ent-al-ismo
dialecto-logie
Reichert
coloni-al
Milton
diphthong-al
linguist-ic
linguist-ics
affairs
differ-ent
phonetic-ally
kilo-metres
stabil-ization
develop-ments
in-vestig-ation
Jordan-ian
notice-able
level-ed
migrants
con-dition-al
certain-ly
general-ly
especial-ly
af-fric-ation
Jordan
counter-parts
com-plication
consider-ably
inter-dent-al
com-mun-ity
inter-locutors
com-pon-ent
region-al
socio-historical
society
simul-taneous
phon-em-ic
roman-ization
Classic-al
funeral
Kurmanji
pharyn-geal-ization
vocab-ulary
phon-et-ic
con-sonant
con-sonants
special-ized
latter
latters
in-itial
ident-ic-al
cor-relate
geo-graphic-al-ly
Öpengin
Kurd-ish
in-digen-ous
sunbul
Christ-ian
Christ-ians
sekin-în
fatala
in-tegration
dia-lect-al
Matras
morpho-logy
in-tens-ive
con-figur-ation
im-port-ant
com-plement
ḥaddād
e-merg-ence
Benjmamins
struct-ure
em-pir-ic-al
Orient-studien
Anatolia
American
vari-ation
Jastrow
Geoffrey
Yarshater
Ashtiany
Edmund
Mahnaz
En-cyclo-pædia
En-cyclo-paedia
En-cyclo-pedia
Leiden
dia-spora
soph-is-ic-ated
Sasan-ian
every-day
domin-ance
Con-stitu-tion-al
religi-ous
sever-al
Manfredi
re-lev-ance
re-cipi-ent
pro-duct-iv-ity
turtle
Morocco
ferman
Maghreb-ian
algérien
stand-ard
systems
Nicolaï
Mouton
mauritani-en
Gotho-burg-ensis
socio-linguist-ique
plur-al
archiv-al
Arab-ian
drop-ped
dihāt
de-velop-ed
ṣuḥbat
kitāba
kitābat
com-mercial
eight-eenth
region
Senegal
mechan-ics
Maur-itan-ia
Ḥassān-iyya
circum-cision
cor-relation
labio-velar-ization
vowel
vowels
cert-ain
īggīw
series
in-tegrates
dur-ative
inter-dent-als
gen-itive
Tuareg
tălămut
talawmāyət
part-icular
part-icular-ly
con-diment
vill-age
bord-er
polit-ical
Wiesbaden
Uni-vers-idad
Geuthner
typo-logie
Maur-itanie
nomades
Maur-itan-ian
dia-lecto-logy
Sahar-iennes
Uni-vers-ity
de-scend-ants
NENA-speak-ing
speak-ing
origin-al
re-captured
in-habit-ants
ethnic
minor-it-ies
drama-tic
local
long-stand-ing
regions
Nineveh
settle-ments
Ṣəndor
Mandate
sub-stitut-ing
ortho-graphy
re-fer-enced
origin-ate
twenti-eth
typ-ic-al-ly
Hobrack
never-the-less
character-ist-ics
character-ist-ic
masc-uline
coffee
ex-clus-ive-ly
verb-al
re-ana-ly-se-d
simil-ar-ities
de-riv-ation
im-pera-tive
part-iciple
dis-ambi-gu-ation
dis-ambi-gu-a-ing
phen-omen-on
phen-omen-a
traktar
com-mun-ity
com-mun-ities
dis-prefer-red
ex-plan-ation
con-struction
wide-spread
us-ual-ly
region-al
Bulut
con-sider-ation
afro-asia-tici
Franco-Angeli
Phono-logie
Volks-kundliche
dia-lectes
dia-lecte
select-ed
dis-appear-ance
media
under-stand-able
public-ation
second-ary
e-ject-ive
re-volu-tion
re-strict-ive
Gasparini
mount-ain
mount-ains
yellow
label-ing
trad-ition-al-ly
currently
dia-chronic
} 
  \togglepaper[1]%%chapternumber
}{}

\begin{document}
\maketitle 
  

\section{Introduction}

In its simplest definition, \textsc{calquing} is a type of contact-induced change in which a word or sentence structure is transferred without actual morphemes \citep[260]{Thomason2001}. Calques are sometimes called loan translations as they typically represent a word-by-word (or morpheme-by-morpheme) translation of a lexeme or a sentence from another language. Heath (\citeyear[367]{Heath1984}) labels this process “pattern transfer” and distinguishes it from “matter borrowing” which is instead linked to the integration of morphophonological material. \cite{Ross2007}, for his part, points out that that calquing can also have important grammatical effects, and he considers it a necessary precondition for contact-induced morphosyntactic restructuring (what Ross calls “metatypy”).  

Broadly speaking, we can distinguish two types of calquing: lexical calquing, which entails the transfer of semantic properties of lexical items, and grammatical calquing, which instead implies the transfer of the functional properties of morphemes and syntactic constructions. Using Ross’s words (\citeyear[126]{Ross2007}), lexical calquing consists of remodelling lexical “ways of saying things”, whereas grammatical calquing consists of remodelling grammatical “ways of saying things”. Despite this fundamental difference, lexical and grammatical calquing share a single cause: bilingual speakers’ need to express the same meaning in two languages (\citealt[32]{Sasse1992}). This also means that everything that expresses meaning (i.e. morphemes, lexemes, and constructions) can, in principle, be a source of calquing. 

Focusing mainly on the transfer of linguistic matter, Van Coetsem (\citeyear{VanCoetsem1988}) does not overtly mention the possibility of transferring lexical and grammatical meanings through calquing. This chapter thus aims at relating contact induced changes produced by calquing to the principle of language dominance as postulated by Van Coetsem. 


 \section{Contact-induced changes and calquing}


 \subsection{Lexical calquing}\label{sec:lex}

According to Haspelmath (\citeyear[39]{Haspelmath2009}), a lexical calque is a lexical unit that was created by an item-by-item translation of the source unit. This type of contact-induced change occurs as bilingual speakers reorganise the lexicon of one of their languages to match the semantic organisation of the other (\citealt[132]{Ross2007}). Adopting the psycholinguistic standpoint of language dominance, Winford (\citeyear[345]{Winford2003}) regards lexical calquing as a subtype of lexical borrowing, which is a combination of recipient language (RL) lexemes in imitation of source language (SL) semantic patterns. In contrast, I will show that, though lexical calquing can easily be triggered by RL-dominant speakers, it can also be a product of imposition via SL agentivity. In order to do this, I will mainly focus on calquing of compound nouns. A compound noun is here defined as a series of two or more lexemes, which is semantically conceived as a single unit. Each component of the compound can function as a lexeme independent from the other(s), and may show some phonological and/or morphological constraints within the compound when compared to its isolated syntactic usage \citep{Bauer2001}. Against this backdrop, I will specifically discuss noun–noun compounds as they represent the more uniform phenomenon of nominal compounding in the world’s languages \cite{Pepperforthcoming}. As we will see, the transfer of the semantics of compound nouns does not imply any morphosyntactic change in Arabic, as calqued compounds are typically adjusted to fit RL morphosyntactic patterns.  

Generally speaking, lexical calquing through borrowing can occur in indirect contact situations characterized by a very low degree of bilingualism. This is because RL monolinguals can also be agents of lexical borrowing (\citealt[10]{VanCoetsem1988}) Typical instances of lexical calquing via RL agentivity are related to the transfer of the semantic patterns of English compound nouns in modern Arabic dialects. This kind of transfer is linked to the expansion of the non-core Arabic lexicon for expressing previously unknown concepts. A prime example is the English calque \textit{lōḥit} \textit{il-mafatīḥ} `keyboard' (lit. ‘the board of keys’) in Egyptian Arabic \citep[9]{WilmsenWoidich2011}. Here, it can be clearly seen that the transfer of the semantic organization of the SL compound noun does not affect the morphosyntax of the RL, as the word order of the English nominal juxtaposition is reversed to fit the Arabic construct state.  

Lexical calquing can also take place in prolonged contact situations, as testified by numerous Italian compounds in Maltese. A singular case of mixed calquing is that of \textit{wi\.c\.c} \textit{tost} ‘shameless person’ (lit. ‘tough face’) deriving from the Italian compound \textit{faccia} \textit{tosta} ‘shameless person’ (lit. ‘tough face’) \citep{Aquilina1987}. On the one hand, the first lexical item of the compound presents an Arabic phonological form while expressing semantic properties associated with the lexeme ‘face’ in Italian. On the other hand, the second lexical item clearly results from the borrowing of the adjective \textit{tosto} ‘hard, tough’ retaining both the Italian phonological matter and semantic properties. The mixed nature of this compound brings to the fore the complementary relationship between RL and SL agentivity and shows that it is not always a trivial matter to distinguish between imposition and borrowing. However, Maltese also gives evidence of genitive compounds in which both lexical components have an Arabic phonological form coupled with Italian semantic properties. This is the case of the compound nouns \textit{saba'} \textit{ta'} \textit{sieq} ‘toe’ calqued on the Italian \textit{dito} \textit{del} \textit{piede} ‘toe’ \citep{Pepperforthcoming}. Such instances of lexical calquing clearly mirror semantic properties of SL lexemes and they most plausibly result from borrowing via RL agentivity (cf. Lucas \& Ćeplö, this volume).\ia{Lucas, Christopher@Lucas, Christopher}\ia{Čéplö, Slavomír@Čéplö, Slavomír} 

Ḥassāniyya Arabic, for its part, presents many compound nouns that are traditionally analysed in terms of substratum interference from Zenaga Berber \citep{Taine-Cheikh2008chapter,Taine-Cheikh2012}. Also in this case, the transfer of the semantic properties of the SL does not produce any morphosyntactic change in Arabic, as we can see in the pairs of examples in \REF{ex:crow} and \REF{ex:ankle}.

\ea\label{ex:crow}
\ea{Ḥassāniyya Arabic \citep[126]{Taine-Cheikh2008chapter}}\\
\gll   k{\R}aʕ lә-ɣ{\R}ab  \\
       foot \textsc{def}-crow\\
\glt   `aquatic herbaceous plant' (lit.  ‘crow’s foot’)

\ex{Zenaga Berber \citep[126]{Taine-Cheikh2008chapter}}\\
\gll   að̣aʔ\R әn tayyaḷ \\
       foot \textsc{gen} crow \\
\glt   `aquatic herbaceous plant' (lit. ‘crow’s foot’)
\z
\z

\ea\label{ex:ankle}
\ea{Ḥassāniyya Arabic \citep[126]{Taine-Cheikh2008chapter}}\\
\gll   sayllāl lә-ʕrāgib \\
       ripper \textsc{def}-ankle.\textsc{pl}\\
\glt   `honey badger' (lit. ‘ripper of ankles’)

\ex{Zenaga Berber \citep[126]{Taine-Cheikh2008chapter}}\\
\gll   amәssäf әn ūržan \\
       ripper \textsc{gen} ankle.\textsc{pl} \\
\glt   `honey badger' (lit. ‘ripper of ankles’)
\z
\z

Taine-Cheikh (\citeyear[126]{Taine-Cheikh2008chapter}) stresses that it is somewhat difficult to trace back the origin of these compounds. Accordingly, she speaks of a process of convergence between the two languages, rather than determining the direction of the semantic transfer. However, it should be observed that these compound nouns are not attested in other spoken varieties of Arabic. Furthermore, since at least the mid-twentieth century, Berbers in Mauritania have been gradually loosing competence in Zenaga, in favour of Arabic \citep[100]{Taine-Cheikh2012}, while Zenaga is rarely acquired as second language by Ḥassāniyya Arabic speakers. In such a context, the most probable agents of contact-induced change were former Berber-dominant speakers who gradually shifted to Arabic. Thus, it seems plausible that the transfer of the semantic properties of Zenaga compounds has been achieved through imposition, rather than through borrowing.

Nigerian Arabic also shows interesting instances of lexical calquing as a consequence of a longstanding contact with Kanuri, a Nilo-Saharan language widely spoken in the Lake Chad area. Owens (\citeyear{Owens2015,Owens2016idioms}) gives evidence of the transfer of the semantic properties of numerous compound nouns including the lexeme \textit{{\R}ās} ‘head’. Similar to the previous instances of compound calquing, the integration of Kanuri semantic patterns does not affect the Arabic morphosyntax, as we can see in the  pairs of examples in \REF{ex:roof} and \REF{ex:tassel}. 

\ea\label{ex:roof}
\ea{Nigerian Arabic \citep[69]{Owens2016idioms}}\\
\gll   {\R}ās al-bēt \\
       head \textsc{def}-house\\
\glt   `roof' (lit. ‘head of house’)

\ex{Kanuri \citep[69]{Owens2016idioms}}\\
\gll   kǝla fato-be   \\
       head house-\textsc{gen}\\
\glt   `roof' (lit. ‘head of house’)
\z
\z

\ea\label{ex:tassel}
\ea{Nigerian Arabic \citep[65]{Owens2016idioms}}\\
\gll   {\R}ās al-qalla  \\
       head \textsc{def}-corn\\
\glt   `tassel' (lit. ‘head of corn')

\ex{Kanuri \citep[65]{Owens2016idioms}}\\
\gll   kǝla argǝm-be \\
       head corn-\textsc{gen} \\
\glt   `tassel' (lit. ‘head of corn')
\z
\z

According to Owens (\citeyear[65]{Owens2016idioms}), Kanuri–Arabic bilingualism, with Arabic being a minority language, would have been the foremost factor underlying the transfer of these compound nouns into Nigerian Arabic. He further stresses that Kanuri is the main source of compound nouns in a number of other minority languages in the area (e.g. Kotoko, Glayda, and Fulfulde) and that there is little evidence of Kanuri to Arabic shift in the region \citep[147]{Owens2014}. However, the fact that Kanuri represents the majority language of northeastern Nigeria, does not shed light on the transfer mechanism lying behind lexical calquing in Nigerian Arabic. This is because speakers can be linguistically dominant in a socially subordinate language \citep[376]{Winford2005}. In fact, such contact settings are closely tied to SL agentivity, as the youngest bilingual generations tend to impose semantic features from their dominant language (i.e. Kanuri) onto the ancestral language (i.e. Arabic). It is only at a later stage that these innovations are borrowed by older bilingual speakers who are still dominant in Arabic. 

The fact that Nigerian Arabic speakers have gradually developed a high bilingual proficiency in Kanuri is also testified by the transfer of a number of idiomatic expressions. In this regard, Ross (\citeyear[122]{Ross2007}) observes that calquing of meaning is not only reflected in word compounding, but also in lexical collocations of idiomatic expressions. These are combinations of lexical items that are semantically idiosyncratic as they have a pairing of form and meaning that cannot be predicted from the rest of the grammar. The pair of examples in \REF{ex:key:distract} provide evidence of an idiomatic Kanuri calque in Nigerian Arabic. 

\ea\label{ex:key:distract}
\ea{Nigerian Arabic \citep[77]{Ritt-Benmimounetc2017}}\\
\gll   šuqul šāl {\R}ās-i\\
       something carry.\textsc{prf.3sg.m} head-\textsc{obl.1sg}\\
\glt   `Something distracted me.' (Lit. ‘Something carried my head.’)

\ex{Kanuri \citep[77]{Ritt-Benmimounetc2017}}\\
\gll   awode kǝla   gō-zǝ-na \\
       something head carry-\textsc{3sg-prf}\\
\glt   `Something distracted me.' (Lit. ‘Something carried head.’)
\z
\z

Given that idiomatic expressions are syntactically compositional (i.e. their lexical components behave syntactically as they do in non-idiomatic expressions), it is not only the meanings expressed by the lexeme ‘head’ which correspond between Nigerian Arabic and Kanuri, but also their idiomatic collocations, which align between the two languages \citep[157]{Owens2014}. Besides, it is worth noting that also idiomatic expressions are adjusted to fit RL morphosyntactic patterns. This is evidenced by the inalienable possession of body parts in Nigerian Arabic (\textit{{\R}ās-i} ‘my head’), which is instead unattested in the SL (\textit{kǝla} ‘head’). Even if we cannot exclude the possibility that these kinds of calques are a product of borrowing, it is evident that their integration needs a high proficiency in the SL for individuating the single idiomatic collocations of lexical items. Furthermore, differently from borrowed calques, imposed idiomatic expressions can significantly affect the lexical semantics of the RL created by SL-dominant bilinguals, and thus produce grammatical changes in the long run. 

Finally, lexical calquing via SL-agentivity can also take place in extreme contact situations such as creolization. For instance, Juba Arabic, the Arabic-based pidgincreole spoken in South Sudan, shows numerous calques in which Arabic-derived lexemes are compounded according to the semantic patterns of Bari, the main substrate language of Juba Arabic (\citealt{Nakao2012}; \citealt[50]{Manfredi2017}; Avram, this volume).\ia{Avram, Andrei@Avram, Andrei} As we can see in \REF{ex:key:tree} and \REF{ex:key:eleph}, the word order in Juba Arabic compounds follows the order of Bari compounds. However, this cannot be seen as an innovative morphosyntactic development, as the possessed–possessor order matches also with the Arabic lexifier. 

\ea\label{ex:key:tree}
\ea{Juba Arabic \citep[136]{Nakao2012}} \\
\gll   éna ta séjera  \\
       eye \textsc{gen} tree \\
\glt   `fruit’ (lit. ‘eye of tree’)

\ex{Bari \citep[136]{Nakao2012}}\\
\gll   koŋe lo-ködini\\
       eye \textsc{gen}-tree\\
\glt   `fruit’ (lit. ‘eye of tree’)
\z
\z

\ea\label{ex:key:eleph}
\ea{Juba Arabic \citep[137]{Nakao2012}} \\
\gll   ída ta fil\\
       hand \textsc{gen} elephant \\
\glt   `trunk’ (lit. ‘hand of elephant’)

\ex{Bari \citep[137]{Nakao2012}}\\
\gll   könin lo-tome\\
       hand \textsc{gen}-elephant \\
\glt   `trunk’ (lit. ‘hand of elephant’)
\z
\z

Given that the asymmetric contact situation leading to creole formation limits access to the superstrate language (i.e. Sudanese Arabic), the semantic patterns of substrate languages (i.e. Bari) can be easily carried over into the creole in ways peculiar to imposition via SL-agentivity. 

All things considered, unlike lexical borrowing, lexical calquing allows for a semantic overlapping of RL and SL lexical entries and it can also produce important structural changes. 


 
 \subsection{Grammatical calquing}\label{sec:gra}

Grammatical calquing brings about a match between the grammatical categories of two languages and the memberships of these categories \citep[132]{Ross2007}. \citet{HeineKuteva2005} suggest that the grammatical changes induced by calquing can be better analysed in terms of contact-induced grammaticalization (see also Leddy-Cecere, this volume).\ia{Leddy-Cecere, Thomas@Leddy-Cecere, Thomas} In fact, the calquing of the semantic properties of lexical and grammatical items may lead to the grammaticalization of innovative syntactic structures in the RL matching with those of the SL. From the traditional sociohistorical perspective of contact-induced change (\citealt{ThomasonKaufman1988}), grammatical calquing is basically seen as a product of language shift. In contrast, Ross (\citeyear[131]{Ross2007}) argues that grammatical calques can widely occur in situations of language maintenance. Actually, the different grammatical outputs of calquing mainly depend on the way in which they are transferred from the SL into the RL and, by extension, on different kinds and degrees of bilingualism.   

For the purposes of this chapter, I distinguish between three different types of grammatical calquing: 

\begin{itemize} 
\item Calquing of polyfunctionality of lexical items without syntactic change;
\item Calquing of polyfunctionality of grammatical items leading to syntactic change; 
\item Narrow syntactic calquing (without calquing of polyfunctionality of lexical/grammatical items).
\end{itemize}

Being lexical in nature, the first of these three types of grammatical calquing can be triggered by both imposition via SL agentivity and borrowing via RL agentivity, whereas the two latter types are likely to result only from imposition via SL agentivity. 

Calquing of polyfunctionality patterns of lexical items is by far the most common type of grammatical calquing, and it can be exemplified by the comparison of reflexive anaphors in different Arabic dialects. As is well known, Classical and Standard Arabic express a reflexive meaning either by means of agent-oriented derived verbs lacking an overtly expressed patient (e.g. \textit{istaḥamma} ‘he washed himself’) or by anaphoric constructions in which the syntagm \textit{nafs-}\textsc{pro.poss} ‘soul-\textsc{pro.poss’} marks coreferentiality between the agent and the patient of the predicate (e.g. \textit{qatala} \textit{nafsa-hu} ‘he killed himself’). Nevertheless, as a result of contact with different languages, a number of modern Arabic dialects have grammaticalized other lexical sources for expressing a reflexive meaning. Western Maghrebi dialects are a case in point. As we can see in \REF{ex:key:15}--\REF{ex:key:16}, both Moroccan and Ḥassāniyya Arabic have grammaticalized the nominal syntagm \textit{{\R}āṣ=}\textsc{pro.poss} ‘head-\textsc{pro.poss’} as default reflexive anaphor.

\ea\label{ex:key:15}
{Moroccan Arabic (D. Caubet, personal communication)} \\
\gll   qtәl {\R}ās-o\\
       kill.\textsc{prf.3sg.m} head-\textsc{3sg.m}\\
\glt   `He killed himself.' (Lit. ‘He killed his head.’)
\z

\ea\label{ex:key:16}
{Ḥassāniyya Arabic \citep[16]{Taine-Cheikh2008chapter}}\\
\gll   ktәl {\R}āṣ-u\\
       kill.\textsc{prf.3sg.m} head-\textsc{3sg.m}\\
\glt   `He killed himself.' (Lit. ‘He killed his head.’)
\z

This reflexive use of the lexeme ‘head’ has generally been interpreted as substrate interference from Berber languages \citep[197]{ElAissati2011}, in which the same grammaticalization path is attested, as shown in the following examples from Tarifit and Zenaga: 

\ea\label{ex:key:}
{Tarifit Berber (\citealt[95]{Kossmann2000})}\\
\gll   yәtšaθ iḫәf nnәs\\
       beat.\textsc{prf.3sg.m} head \textsc{poss.3sg.m}\\
\glt   `He beats himself.' (Lit. ‘He beats his head.’)
\z

\ea\label{ex:key:}
{Zenaga Berber \citep[126]{Taine-Cheikh2008chapter}}\\
\gll   yәʔna iʔf-әn-š\\
       kill.\textsc{prf.3sg.m} head-\textsc{gen-3sg.m}\\
\glt   `He killed himself.' (Lit. ‘He killed his head.’)
\z

The lexeme for ‘head’ is the second most common source of grammaticalization of reflexive anaphors worldwide \citep{KönigTöpper2013} and its occurrence is particularly common in West Africa (\citealt[50]{Heine2011}). In this scenario, it should be stressed that the reflexive function of the lexeme ‘head’ is an innovative feature of both Arabic and Berber varieties of northwestern Africa. Other Berber languages typically use the reflexive anaphor \textit{iman-}\textsc{poss} ‘soul-\textsc{poss}’, as we can see in the following example from Kabyle.

\ea\label{ex:key:}
{Kabyle \citep{Mettouchi2012}}\\
\gll   n-səlk-dd       iman-ntəɣ\\
       \textsc{1pl}-spare.\textsc{prf-prox} soul.\textsc{abs.sg.m-poss.1pl.f}\\
\glt   `We saved ourselves.'\\
\z

In addition, the known Arabic–Berber contact situation, in which second language learners of Berber only played a marginal role in triggering contact-induced change in Arabic, suggests that the contact induced grammaticalization of ‘head’ in westernmost Arabic dialects resulted from an imposition enacted by former Berber-dominant speakers.  

A similar instance of calquing in the domain of anaphoric reflexive constructions is found in Kordofanian Baggara Arabic, a western Sudanic dialect spoken in the Nuba Mountains area, in central Sudan. In this case, the source of the reflexive anaphor is the lexeme for ‘neck’, as we can see in \REF{ex:key:20}.  

\ea\label{ex:key:20}
{Kordofanian Baggara Arabic \citep[176]{Manfredi2010}}\\
\gll  abrahīm gaṣṣa ragabt-a\\
       Ibrahim cut.\textsc{prf.3sg.m} neck-\textsc{3sg.m}\\
\glt   `Ibrahim cut himself.' (Lit. `Ibrahim cut his neck.')
\z

Different from ‘head’, the grammaticalization of ‘neck’ as a reflexive anaphor is quite rare in Africa (\citealt[50]{Heine2011}), but it is attested in a number of Niger-Kordofanian languages spoken in the same region. Such is the case of Tagoi \REF{ex:key:21} and Koalib \REF{ex:key:22}. 

\ea\label{ex:key:21}
{Tagoi \citep[26]{Alamin2015}}\\
\gll   t-áɡám t-ùrúŋ ínní\\
       \textsc{nc}-neck \textsc{nc}-\textsc{poss.3} kill.\textsc{prf.3}\\
\glt   `He killed himself.' (Lit. ‘He killed his neck.’)
\z

\ea\label{ex:key:22}
{Koalib (N. Quint, personal communication)} \\
\gll   ɛ̀ɽnyɛ́ r-ɔ́kwɽɔ̀ r-ùŋwún\\
       kill \textsc{nc}-neck \textsc{nc-poss.3}\\
\glt   `to kill oneself' (lit. ‘to kill one's neck’)
\z

Similarly to the situation described with reference to western Maghrebi dialects, Arabic-speaking groups in the Nuba Mountains have hardly developed any bilingual competence in local Niger-Kordofanian languages. Therefore, it seems likely that the calquing of the polyfunctionality patterns of ‘neck’ has been imposed by Arabized populations who were dominant in the SL. 

Maltese also provides remarkable examples of calquing of polyfunctionality of lexical items. This is particularly evident in the domain of auxiliary verbs (\citealt{Vanhove1993}; \citealt{VanhoveCaubet2009}). A well-known example is that of the lexical verb \textit{\.gie} ‘come’ used as an auxiliary for expressing a dynamic passive \REF{ex:key:23} in the same way as Italian \REF{ex:key:24}.

\ea\label{ex:key:23}
{Maltese \citep[214]{BorgAzzopardi-Alexander1997}}\\
\gll   it-tabib \.gie afdat bi-l-każ\\
       \textsc{def}-doctor come.\textsc{prf.3sg.m} trust.\textsc{ptcp.pass} with-\textsc{def}-case\\
\glt   `The doctor was entrusted with the case.' (Lit. ‘The doctor came entrusted with the case.’)
\z

\ea\label{ex:key:24}
{Italian (own knowledge)} \\
\gll   non venne creduto\\
       \textsc{neg} come.\textsc{past.3sg.m} trust.\textsc{ptcp.past}\\
\glt   `He was not trusted.' (Lit. ‘He did not come trusted.’)
\z

Even if imposition played a role in the emergence of Maltese (Lucas \& Ćeplö, this volume),\ia{Lucas, Christopher@Lucas, Christopher}\ia{Čéplö, Slavomír@Čéplö, Slavomír} it is generally accepted that intertwined languages emerge mainly from a widespread process of borrowing in Van Coetsem’s terminology (\citealt[397]{Winford2005}; \citealt{Manfredi2018}). This suggests that, unlike the aforementioned grammaticalization of reflexive anaphors in Arabic dialects, the calquing of polyfunctionality of lexical verb ‘come’ in Maltese was most likely triggered by agentivity of RL dominant speakers. 

Regardless of the different contact situations, what holds all the previous instances of grammatical calquing together is the fact that the transfer of patterns of grammaticalization did not produce any syntactic change in Arabic. In contrast to the above, the calquing of polyfunctionality of grammatical items can be accompanied by important typological changes. This is the case of the grammaticalization of prototypical passive constructions in Juba Arabic (\citealt[92]{Manfredi2017}; \citeyear[415]{Manfredi2018}). As we can see in \REF{ex:key:passive}, the South Sudanese pidgincreole presents an innovative passive construction in which the patient occupies the syntactic slot of a preverbal subject, whereas the oblique-marked agent is introduced by the comitative preposition \textit{ma-} ‘with’.  

\ea\label{ex:key:passive}
{Juba Arabic (\citealt[86]{Manfredi2017})}\\
\gll   bab de kasurú ma-jón \\
       door \textsc{prox.sg} break.\textsc{pass} with-John\\
\glt   `This door has been broken by John.' (Lit. ‘This door has been broken with John.’)
\z
 
Interestingly, this prototypical passive construction is not attested in the lexifier language of Juba Arabic (i.e. Sudanese Arabic), which instead makes use of impersonal passive constructions with a default \textsc{3pl.m} subject.

\ea\label{ex:key:}
{Sudanese Arabic (own knowledge)}\\
\gll   kassaru-hu\\
       break.\textsc{prf.3pl.m-3sg.m}\\
\glt   `It got broken.' (Lit. ‘They have broken it.’) 
\z

Indeed, the grammaticalization of this complex syntactic structure is the result of the calquing of the functional properties associated with the comitative preposition of the main substrate language, Bari. Bari presents the same kind of prototypical passive construction in which an oblique-marked agent is introduced by the preposition \textit{ko-} ‘with’.

\ea\label{ex:key:}

{Bari (\citealt[65]{Owen1909})}\\
\gll   niena  wuret   a-wur-ö ko-nan \\
       \textsc{prox.sg} book \textsc{3sg.past}-write-\textsc{pass} with-\textsc{1sg}\\
\glt   `This book has been written by me.' (Lit. ‘This book has been written with me.’)
\z

If we assume that the emergence of creole languages is always induced by the disruption of the transmission of the lexifier language \citep{Comrie2011}, we can conclude that Bari speakers have imposed the semantics of their dominant language on a grammatical item derived from Arabic, and thus induced profound changes in the word order of the creole when compared to its lexifier language.\footnote{This kind of syntactic change accompanied by the calquing of semantic properties of substrate items in creole languages is traditionally labelled ``relexification'' \citep{Lefebvre1998}.}  In light of the above, the contact dynamics lying behind the calquing of polyfunctionality of grammatical items are quite restrictive as they are most likely a product of imposition via SL agentivity.  

The third kind of grammatical calquing is linked to the transfer of syntactic patterns without transfer of polyfunctionality of either lexical or grammatical items. This narrow type of syntactic calquing can be exemplified by possessor doubling in Central Asian Arabic \citep{Ratcliffe2005}. Clitic doubling is a construction in which a clitic co-occurs with a full nominal phrase in argument position, forming a discontinuous constituent with it. Various forms of clitic doubling have arisen in a number of Arabic varieties as a result of contact with different substrate/adstrate languages \citep{Souag2017clitic}. In regard to possessive constructions, Arabic typically presents a possessed–possessor order. In contrast, Central Asian Arabic \REF{ex:key:28} gives evidence of the opposite order with obligatory possessor doubling in the same way as Tajik \REF{ex:key:29}. 

\ea\label{ex:key:28}
{Central Asian Arabic (\citealt{Ratcliffe2005}; \citealt[56]{Souag2017clitic})}\\
\gll   amīr wald-u\\
        prince son-\textsc{3sg.m}\\
\glt   `the prince’s son'
\z

\ea\label{ex:key:29}
{Tajik (\citealt[56]{Souag2017clitic})}\\
\gll   buḫoro universitet-aš\\
       Bukhara university-\textsc{3sg}\\
\glt   `Bukhara University'
\z

Souag (\citeyear[157]{Souag2017clitic}) states that double possessor constructions in Central Asian Arabic are instances of grammatical calquing, accommodated through the reinterpretation of pre-existing topicalized constructions. This means that, unlike the syntactic changes induced by the calquing of polyfunctionality of morphemes, the emergence of double possessor constructions in Bukhara Arabic would have been favoured by a formal congruence between SL and RL syntactic structures. As such, this instance of contact-induced morphosyntactic restructuring (i.e. metatypy) does not derive from a direct copying of a double possessor construction. Rather, it consists in speakers expressing a possessive meaning in Arabic by using a construction which they equate with the construction in adstratal languages (\citealt[128]{Ross2007}). If we consider that the youngest speakers of Central Asian Arabic are gradually losing competence in their ancestral language in favour of socially dominant languages (\citealt[128]{Chikovani2005}), it is plausible to think that such kind of syntactic restructuring can only be a result of imposition via SL-agentivity. Still, given our limited diachronic knowledge, we cannot exclude the hypothesis of an early process of borrowing enacted by former Arabic-dominant speakers.   

\section{Conclusions}

Van Coetsem (\citeyear[20]{VanCoetsem1988}) suggests that the variable outcomes of language contact are primarily a reflex of the ``stability gradient'' of language, which induces speakers to preserve the domains of their dominant language that are less affected by change. As lexicon is the most unstable linguistic domain, it is likely to be transferred via RL-agentivity. In contrast, morphosyntax and phonology are considered to be relatively stable domains and they are expected to be transferred only via SL-agentivity. Against this background, it is unclear how the transfer of semantic features deprived of morphophonological matter should be understood in relation to the linguistic dominance of the agents of contact-induced change. 

If we look at the previously analysed instances of lexical calquing (§\ref{sec:lex}), it is evident that the transfer of the semantic features of nominal compounds can take place within speech communities with a very low degree of bilingualism, as in the case of Egyptian-Arabic-dominant speakers borrowing the semantics of English compounds. But it is also true that compound calquing can be a product of imposition resulting from ongoing language shift or pidginization, and the transfer of semantic features of single lexical items within idiomatic expressions always requires a widespread proficiency in the SL, as in the case of Arabic–Kanuri bilingualism in northern Nigeria. 

As far as grammatical calquing is concerned (§\ref{sec:gra}), I have shown that calquing of the polyfunctionality of lexical items can be triggered either by imposition, as in the case of substrate interference in Ḥass\-āniyya and Baggara dialects, or by borrowing in the emergence of intertwined languages such as Maltese. Calquing of polyfunctionality of grammatical items, for its part, requires a higher degree of linguistic abstraction for the identification of a functional overlap between morphemes. Accordingly, this type of transfer will typically occur via imposition by SL-dominant speakers in deep contact situations such as creolization. In the same manner, narrow syntactic calquing requires high bilingual proficiency, as it necessitates the recognition of some formal congruence between the SL and the RL, as shown by the emergence of possessor doubling in Central Asian Arabic.  

To stay somewhat in line with the stability gradient principle, we could argue that, in absence of the transfer of linguistic matter, the semantic properties of morphemes and syntactic constructions are more stable than those of lexical items. However, such a generalization would be misleading without an in-depth knowledge of the sociolinguistic circumstances underlying a specific instance of second language acquisition (i.e. symmetric bilingualism, asymmetric bilingualism, multilingualism, pidginization/creolization). Thus, it becomes evident that the recognition of different patterns of bilingualism within the same community remains the only way to identify the transfer type at play in a given contact situation, regardless of its different structural outputs.  

Drawing on the available literature, this chapter has surveyed only a few instances of  calquing induced by contact between Arabic and other languages. This is mainly because we lack information about calquing in dialect contact situations. Indeed, it is regrettable that studies dealing with dialect contact and new dialect formation are still exclusively focused on the diffusion of few lexical and morphophonological features, while disregarding the transfer of semantic and syntactic patterns. Fine-grained analyses of semantic changes induced by dialect contact thus remain a major desideratum for the development an aggregate variationist Arabic dialectology.  

\section*{Further reading}

\citet{Keesing1988} adopts the notion of calquing and describes the transfer of semantic properties of Oceanic morphemes in Melanesian Pidgin.\\
\citet{Meyerhoff2009}, by focusing on the notions of~replication,~transfer, and~calquing, strengthens connections between variationist sociolinguistics and contact linguistics.\\
\citet{Zuckermann2009} provides numerous instances of calquing in Modern Hebrew and analyses them in the light of the Congruence Principle.

\section*{Abbreviations}

\begin{tabularx}{.5\textwidth}{@{}lQ@{}}
\textsc{1, 2, 3} & 1st, 2nd, 3rd person \\
\textsc{def} & definite article \\
\textsc{f} & feminine \\
\textsc{gen} & genitive \\
\textsc{m} & masculine \\
\textsc{nc} & noun class  \\
\textsc{obl} & oblique \\
\textsc{pass} & passive \\
\textsc{past} & past \\
\textsc{pl} & plural \\
\end{tabularx}%
\begin{tabularx}{.5\textwidth}{@{}lQ@{}}
\textsc{poss} & possessive pronoun \\
\textsc{prg} & pragmatic marker \\
\textsc{prf} & perfect (prefix conjugation) \\
\textsc{pro} & pronoun \\
\textsc{prox} & proximal \\
RL & recipient language \\
\textsc{refl} & reflexive \\
\textsc{sg} & singular \\
SL & source language \\
\textsc{abs} & absolute state \\
\end{tabularx}%






\sloppy
\printbibliography[heading=subbibliography,notkeyword=this]
\end{document}