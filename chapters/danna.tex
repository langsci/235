\documentclass[output=paper]{langsci/langscibook} 
\author{Luca D'Anna\affiliation{Università degli Studi di Napoli `L'Orientale'}}
\title{Arabic in the diaspora}
\abstract{This paper offers an overview of contact-induced change in diasporic Arabic. It provides a socio-historical description of the Arab diaspora, followed by a sociolinguistic profile of Arabic-speaking diasporic communities. Language change is analyzed at the phonological, morphological, syntactic and lexical level, distinguishing between contact-induced change and internal developments caused by reduced input and weakened monitoring. In the course of the description, parallels are drawn between diasporic Arabic and other contemporary or extinct contact varieties, such as Arabic-based pidgins and Andalusi Arabic.}
\IfFileExists{../localcommands.tex}{
  % add all extra packages you need to load to this file 
\usepackage{graphicx}
\usepackage{tabularx}
\usepackage{amsmath} 
\usepackage{multicol}
\usepackage{lipsum}
\usepackage[stable]{footmisc}
\usepackage{adforn}
%%%%%%%%%%%%%%%%%%%%%%%%%%%%%%%%%%%%%%%%%%%%%%%%%%%%
%%%                                              %%%
%%%           Examples                           %%%
%%%                                              %%%
%%%%%%%%%%%%%%%%%%%%%%%%%%%%%%%%%%%%%%%%%%%%%%%%%%%%
% remove the percentage signs in the following lines
% if your book makes use of linguistic examples
\usepackage{./langsci/styles/langsci-optional} 
\usepackage{./langsci/styles/langsci-lgr}
\usepackage{morewrites} 
%% if you want the source line of examples to be in italics, uncomment the following line
% \def\exfont{\it}

\usepackage{enumitem}
\newlist{furtherreading}{description}{1}
\setlist[furtherreading]{font=\normalfont,labelsep=\widthof{~},noitemsep,align=left,leftmargin=\parindent,labelindent=0pt,labelwidth=-\parindent}
\usepackage{phonetic}
\usepackage{chronosys,tabularx}
\usepackage{csquotes}
\usepackage[stable]{footmisc} 

\usepackage{langsci-bidi}
\usepackage{./langsci/styles/langsci-gb4e} 

  \makeatletter
\let\thetitle\@title
\let\theauthor\@author 
\makeatother

\newcommand{\togglepaper}[1][0]{ 
  \bibliography{../localbibliography}
  \papernote{\scriptsize\normalfont
    \theauthor.
    \thetitle. 
    To appear in: 
    Christopher Lucas and Stefano Manfredi (eds.),  
    Arabic and contact-induced language change
    Berlin: Language Science Press. [preliminary page numbering]
  }
  \pagenumbering{roman}
  \setcounter{chapter}{#1}
  \addtocounter{chapter}{-1}
}

\newfontfamily\Parsifont[Script=Arabic]{ScheherazadeRegOT_Jazm.ttf} 
\newcommand{\arabscript}[1]{\RL{\Parsifont #1}}
\newcommand{\textarabic}[1]{{\arabicfont #1}}

\newcommand{\textstylest}[1]{{\color{red}#1}}

\patchcmd{\mkbibindexname}{\ifdefvoid{#3}{}{\MakeCapital{#3}
}}{\ifdefvoid{#3}{}{#3 }}{}{\AtEndDocument{\typeout{mkbibindexname could
not be patched.}}}

%command for italic r with dot below with horizontal correction to put the dot in the prolongation of the vertical stroke
%for some reason, the dot is larger than expected, so we explicitly reduce the font size (to \small)
%for the time being, the font is set to an absolute value. To be more robust, a relative reduction would be better, but this might not be required right now
\newcommand{\R}{r\kern-.05ex{\small{̣}}\kern.05ex}


\DeclareLabeldate{%
    \field{date}
    \field{year}
    \field{eventdate}
    \field{origdate}
    \field{urldate}
    \field{pubstate}
    \literal{nodate}
}

\renewbibmacro*{addendum+pubstate}{% Thanks to https://tex.stackexchange.com/a/154367 for the idea
  \printfield{addendum}%
  \iffieldequalstr{labeldatesource}{pubstate}{}
  {\newunit\newblock\printfield{pubstate}}
}
 
  %% hyphenation points for line breaks
%% Normally, automatic hyphenation in LaTeX is very good
%% If a word is mis-hyphenated, add it to this file
%%
%% add information to TeX file before \begin{document} with:
%% %% hyphenation points for line breaks
%% Normally, automatic hyphenation in LaTeX is very good
%% If a word is mis-hyphenated, add it to this file
%%
%% add information to TeX file before \begin{document} with:
%% %% hyphenation points for line breaks
%% Normally, automatic hyphenation in LaTeX is very good
%% If a word is mis-hyphenated, add it to this file
%%
%% add information to TeX file before \begin{document} with:
%% \include{localhyphenation}
\hyphenation{
affri-ca-te
affri-ca-tes
com-ple-ments
homo-phon-ous
start-ed
Meso-potam-ian
morpho-phono-logic-al-ly
morpho-phon-em-ic-s
Palestin-ian
re-present-ed
Ki-nubi
ḥawār-iyy-ūn
archa-ic-ity
fuel-ed
de-velop-ment
pros-od-ic
Arab-ic
in-duced
phono-logy
possess-um
possess-ive-s
templ-ate
spec-ial
espec-ial-ly
nat-ive
pass-ive
clause-s
potent-ial-ly
Lusignan
commun-ity
tobacco
posi-tion
Cushit-ic
Middle
with-in
re-finit-iz-ation
langu-age-s
langu-age
diction-ary
glossary
govern-ment
eight
counter-part
nomin-al
equi-valent
deont-ic
ana-ly-sis
Malt-ese
un-fortun-ate-ly
scient-if-ic
Catalan
Occitan
ḥammāl
cross-linguist-ic-al-ly
predic-ate
major-ity
ignor-ance
chrono-logy
south-western
mention-ed
borrow-ed
neg-ative
de-termin-er
European
under-mine
detail
Oxford
Socotra
numer-ous
spoken
villages
nomad-ic
Khuze-stan
Arama-ic
Persian
Ottoman
Ottomans
Azeri
rur-al
bi-lingual-ism
borrow-ing
prestig-ious
dia-lects
dia-lect
allo-phone
allo-phones
poss-ible
parallel
parallels
pattern
article
common-ly
respect-ive-ly
sem-antic
Moroccan
Martine
Harrassowitz
Grammatic-al-ization
grammatic-al-ization
Afro-asiatica
Afro-asiatic
continu-ation
Semit-istik
varieties
mono-phthong
mono-phthong-ized
col-loquial
pro-duct
document-ary
ex-ample-s
ex-ample
termin-ate
element-s
Aramaeo-grams
Centr-al
idioms
Arab-ic
Dadan-it-ic
sub-ordin-ator
Thamud-ic
difficult
common-ly
Revue
Bovingdon
under
century
attach
attached
bundle
graph-em-ic
graph-emes
cicada
contrast-ive
Corriente
Andalusi
Kossmann
morpho-logic-al
inter-action
dia-chroniques
islámica
occid-ent-al-ismo
dialecto-logie
Reichert
coloni-al
Milton
diphthong-al
linguist-ic
linguist-ics
affairs
differ-ent
phonetic-ally
kilo-metres
stabil-ization
develop-ments
in-vestig-ation
Jordan-ian
notice-able
level-ed
migrants
con-dition-al
certain-ly
general-ly
especial-ly
af-fric-ation
Jordan
counter-parts
com-plication
consider-ably
inter-dent-al
com-mun-ity
inter-locutors
com-pon-ent
region-al
socio-historical
society
simul-taneous
phon-em-ic
roman-ization
Classic-al
funeral
Kurmanji
pharyn-geal-ization
vocab-ulary
phon-et-ic
con-sonant
con-sonants
special-ized
latter
latters
in-itial
ident-ic-al
cor-relate
geo-graphic-al-ly
Öpengin
Kurd-ish
in-digen-ous
sunbul
Christ-ian
Christ-ians
sekin-în
fatala
in-tegration
dia-lect-al
Matras
morpho-logy
in-tens-ive
con-figur-ation
im-port-ant
com-plement
ḥaddād
e-merg-ence
Benjmamins
struct-ure
em-pir-ic-al
Orient-studien
Anatolia
American
vari-ation
Jastrow
Geoffrey
Yarshater
Ashtiany
Edmund
Mahnaz
En-cyclo-pædia
En-cyclo-paedia
En-cyclo-pedia
Leiden
dia-spora
soph-is-ic-ated
Sasan-ian
every-day
domin-ance
Con-stitu-tion-al
religi-ous
sever-al
Manfredi
re-lev-ance
re-cipi-ent
pro-duct-iv-ity
turtle
Morocco
ferman
Maghreb-ian
algérien
stand-ard
systems
Nicolaï
Mouton
mauritani-en
Gotho-burg-ensis
socio-linguist-ique
plur-al
archiv-al
Arab-ian
drop-ped
dihāt
de-velop-ed
ṣuḥbat
kitāba
kitābat
com-mercial
eight-eenth
region
Senegal
mechan-ics
Maur-itan-ia
Ḥassān-iyya
circum-cision
cor-relation
labio-velar-ization
vowel
vowels
cert-ain
īggīw
series
in-tegrates
dur-ative
inter-dent-als
gen-itive
Tuareg
tălămut
talawmāyət
part-icular
part-icular-ly
con-diment
vill-age
bord-er
polit-ical
Wiesbaden
Uni-vers-idad
Geuthner
typo-logie
Maur-itanie
nomades
Maur-itan-ian
dia-lecto-logy
Sahar-iennes
Uni-vers-ity
de-scend-ants
NENA-speak-ing
speak-ing
origin-al
re-captured
in-habit-ants
ethnic
minor-it-ies
drama-tic
local
long-stand-ing
regions
Nineveh
settle-ments
Ṣəndor
Mandate
sub-stitut-ing
ortho-graphy
re-fer-enced
origin-ate
twenti-eth
typ-ic-al-ly
Hobrack
never-the-less
character-ist-ics
character-ist-ic
masc-uline
coffee
ex-clus-ive-ly
verb-al
re-ana-ly-se-d
simil-ar-ities
de-riv-ation
im-pera-tive
part-iciple
dis-ambi-gu-ation
dis-ambi-gu-a-ing
phen-omen-on
phen-omen-a
traktar
com-mun-ity
com-mun-ities
dis-prefer-red
ex-plan-ation
con-struction
wide-spread
us-ual-ly
region-al
Bulut
con-sider-ation
afro-asia-tici
Franco-Angeli
Phono-logie
Volks-kundliche
dia-lectes
dia-lecte
select-ed
dis-appear-ance
media
under-stand-able
public-ation
second-ary
e-ject-ive
re-volu-tion
re-strict-ive
Gasparini
mount-ain
mount-ains
yellow
label-ing
trad-ition-al-ly
currently
dia-chronic
}
\hyphenation{
affri-ca-te
affri-ca-tes
com-ple-ments
homo-phon-ous
start-ed
Meso-potam-ian
morpho-phono-logic-al-ly
morpho-phon-em-ic-s
Palestin-ian
re-present-ed
Ki-nubi
ḥawār-iyy-ūn
archa-ic-ity
fuel-ed
de-velop-ment
pros-od-ic
Arab-ic
in-duced
phono-logy
possess-um
possess-ive-s
templ-ate
spec-ial
espec-ial-ly
nat-ive
pass-ive
clause-s
potent-ial-ly
Lusignan
commun-ity
tobacco
posi-tion
Cushit-ic
Middle
with-in
re-finit-iz-ation
langu-age-s
langu-age
diction-ary
glossary
govern-ment
eight
counter-part
nomin-al
equi-valent
deont-ic
ana-ly-sis
Malt-ese
un-fortun-ate-ly
scient-if-ic
Catalan
Occitan
ḥammāl
cross-linguist-ic-al-ly
predic-ate
major-ity
ignor-ance
chrono-logy
south-western
mention-ed
borrow-ed
neg-ative
de-termin-er
European
under-mine
detail
Oxford
Socotra
numer-ous
spoken
villages
nomad-ic
Khuze-stan
Arama-ic
Persian
Ottoman
Ottomans
Azeri
rur-al
bi-lingual-ism
borrow-ing
prestig-ious
dia-lects
dia-lect
allo-phone
allo-phones
poss-ible
parallel
parallels
pattern
article
common-ly
respect-ive-ly
sem-antic
Moroccan
Martine
Harrassowitz
Grammatic-al-ization
grammatic-al-ization
Afro-asiatica
Afro-asiatic
continu-ation
Semit-istik
varieties
mono-phthong
mono-phthong-ized
col-loquial
pro-duct
document-ary
ex-ample-s
ex-ample
termin-ate
element-s
Aramaeo-grams
Centr-al
idioms
Arab-ic
Dadan-it-ic
sub-ordin-ator
Thamud-ic
difficult
common-ly
Revue
Bovingdon
under
century
attach
attached
bundle
graph-em-ic
graph-emes
cicada
contrast-ive
Corriente
Andalusi
Kossmann
morpho-logic-al
inter-action
dia-chroniques
islámica
occid-ent-al-ismo
dialecto-logie
Reichert
coloni-al
Milton
diphthong-al
linguist-ic
linguist-ics
affairs
differ-ent
phonetic-ally
kilo-metres
stabil-ization
develop-ments
in-vestig-ation
Jordan-ian
notice-able
level-ed
migrants
con-dition-al
certain-ly
general-ly
especial-ly
af-fric-ation
Jordan
counter-parts
com-plication
consider-ably
inter-dent-al
com-mun-ity
inter-locutors
com-pon-ent
region-al
socio-historical
society
simul-taneous
phon-em-ic
roman-ization
Classic-al
funeral
Kurmanji
pharyn-geal-ization
vocab-ulary
phon-et-ic
con-sonant
con-sonants
special-ized
latter
latters
in-itial
ident-ic-al
cor-relate
geo-graphic-al-ly
Öpengin
Kurd-ish
in-digen-ous
sunbul
Christ-ian
Christ-ians
sekin-în
fatala
in-tegration
dia-lect-al
Matras
morpho-logy
in-tens-ive
con-figur-ation
im-port-ant
com-plement
ḥaddād
e-merg-ence
Benjmamins
struct-ure
em-pir-ic-al
Orient-studien
Anatolia
American
vari-ation
Jastrow
Geoffrey
Yarshater
Ashtiany
Edmund
Mahnaz
En-cyclo-pædia
En-cyclo-paedia
En-cyclo-pedia
Leiden
dia-spora
soph-is-ic-ated
Sasan-ian
every-day
domin-ance
Con-stitu-tion-al
religi-ous
sever-al
Manfredi
re-lev-ance
re-cipi-ent
pro-duct-iv-ity
turtle
Morocco
ferman
Maghreb-ian
algérien
stand-ard
systems
Nicolaï
Mouton
mauritani-en
Gotho-burg-ensis
socio-linguist-ique
plur-al
archiv-al
Arab-ian
drop-ped
dihāt
de-velop-ed
ṣuḥbat
kitāba
kitābat
com-mercial
eight-eenth
region
Senegal
mechan-ics
Maur-itan-ia
Ḥassān-iyya
circum-cision
cor-relation
labio-velar-ization
vowel
vowels
cert-ain
īggīw
series
in-tegrates
dur-ative
inter-dent-als
gen-itive
Tuareg
tălămut
talawmāyət
part-icular
part-icular-ly
con-diment
vill-age
bord-er
polit-ical
Wiesbaden
Uni-vers-idad
Geuthner
typo-logie
Maur-itanie
nomades
Maur-itan-ian
dia-lecto-logy
Sahar-iennes
Uni-vers-ity
de-scend-ants
NENA-speak-ing
speak-ing
origin-al
re-captured
in-habit-ants
ethnic
minor-it-ies
drama-tic
local
long-stand-ing
regions
Nineveh
settle-ments
Ṣəndor
Mandate
sub-stitut-ing
ortho-graphy
re-fer-enced
origin-ate
twenti-eth
typ-ic-al-ly
Hobrack
never-the-less
character-ist-ics
character-ist-ic
masc-uline
coffee
ex-clus-ive-ly
verb-al
re-ana-ly-se-d
simil-ar-ities
de-riv-ation
im-pera-tive
part-iciple
dis-ambi-gu-ation
dis-ambi-gu-a-ing
phen-omen-on
phen-omen-a
traktar
com-mun-ity
com-mun-ities
dis-prefer-red
ex-plan-ation
con-struction
wide-spread
us-ual-ly
region-al
Bulut
con-sider-ation
afro-asia-tici
Franco-Angeli
Phono-logie
Volks-kundliche
dia-lectes
dia-lecte
select-ed
dis-appear-ance
media
under-stand-able
public-ation
second-ary
e-ject-ive
re-volu-tion
re-strict-ive
Gasparini
mount-ain
mount-ains
yellow
label-ing
trad-ition-al-ly
currently
dia-chronic
}
\hyphenation{
affri-ca-te
affri-ca-tes
com-ple-ments
homo-phon-ous
start-ed
Meso-potam-ian
morpho-phono-logic-al-ly
morpho-phon-em-ic-s
Palestin-ian
re-present-ed
Ki-nubi
ḥawār-iyy-ūn
archa-ic-ity
fuel-ed
de-velop-ment
pros-od-ic
Arab-ic
in-duced
phono-logy
possess-um
possess-ive-s
templ-ate
spec-ial
espec-ial-ly
nat-ive
pass-ive
clause-s
potent-ial-ly
Lusignan
commun-ity
tobacco
posi-tion
Cushit-ic
Middle
with-in
re-finit-iz-ation
langu-age-s
langu-age
diction-ary
glossary
govern-ment
eight
counter-part
nomin-al
equi-valent
deont-ic
ana-ly-sis
Malt-ese
un-fortun-ate-ly
scient-if-ic
Catalan
Occitan
ḥammāl
cross-linguist-ic-al-ly
predic-ate
major-ity
ignor-ance
chrono-logy
south-western
mention-ed
borrow-ed
neg-ative
de-termin-er
European
under-mine
detail
Oxford
Socotra
numer-ous
spoken
villages
nomad-ic
Khuze-stan
Arama-ic
Persian
Ottoman
Ottomans
Azeri
rur-al
bi-lingual-ism
borrow-ing
prestig-ious
dia-lects
dia-lect
allo-phone
allo-phones
poss-ible
parallel
parallels
pattern
article
common-ly
respect-ive-ly
sem-antic
Moroccan
Martine
Harrassowitz
Grammatic-al-ization
grammatic-al-ization
Afro-asiatica
Afro-asiatic
continu-ation
Semit-istik
varieties
mono-phthong
mono-phthong-ized
col-loquial
pro-duct
document-ary
ex-ample-s
ex-ample
termin-ate
element-s
Aramaeo-grams
Centr-al
idioms
Arab-ic
Dadan-it-ic
sub-ordin-ator
Thamud-ic
difficult
common-ly
Revue
Bovingdon
under
century
attach
attached
bundle
graph-em-ic
graph-emes
cicada
contrast-ive
Corriente
Andalusi
Kossmann
morpho-logic-al
inter-action
dia-chroniques
islámica
occid-ent-al-ismo
dialecto-logie
Reichert
coloni-al
Milton
diphthong-al
linguist-ic
linguist-ics
affairs
differ-ent
phonetic-ally
kilo-metres
stabil-ization
develop-ments
in-vestig-ation
Jordan-ian
notice-able
level-ed
migrants
con-dition-al
certain-ly
general-ly
especial-ly
af-fric-ation
Jordan
counter-parts
com-plication
consider-ably
inter-dent-al
com-mun-ity
inter-locutors
com-pon-ent
region-al
socio-historical
society
simul-taneous
phon-em-ic
roman-ization
Classic-al
funeral
Kurmanji
pharyn-geal-ization
vocab-ulary
phon-et-ic
con-sonant
con-sonants
special-ized
latter
latters
in-itial
ident-ic-al
cor-relate
geo-graphic-al-ly
Öpengin
Kurd-ish
in-digen-ous
sunbul
Christ-ian
Christ-ians
sekin-în
fatala
in-tegration
dia-lect-al
Matras
morpho-logy
in-tens-ive
con-figur-ation
im-port-ant
com-plement
ḥaddād
e-merg-ence
Benjmamins
struct-ure
em-pir-ic-al
Orient-studien
Anatolia
American
vari-ation
Jastrow
Geoffrey
Yarshater
Ashtiany
Edmund
Mahnaz
En-cyclo-pædia
En-cyclo-paedia
En-cyclo-pedia
Leiden
dia-spora
soph-is-ic-ated
Sasan-ian
every-day
domin-ance
Con-stitu-tion-al
religi-ous
sever-al
Manfredi
re-lev-ance
re-cipi-ent
pro-duct-iv-ity
turtle
Morocco
ferman
Maghreb-ian
algérien
stand-ard
systems
Nicolaï
Mouton
mauritani-en
Gotho-burg-ensis
socio-linguist-ique
plur-al
archiv-al
Arab-ian
drop-ped
dihāt
de-velop-ed
ṣuḥbat
kitāba
kitābat
com-mercial
eight-eenth
region
Senegal
mechan-ics
Maur-itan-ia
Ḥassān-iyya
circum-cision
cor-relation
labio-velar-ization
vowel
vowels
cert-ain
īggīw
series
in-tegrates
dur-ative
inter-dent-als
gen-itive
Tuareg
tălămut
talawmāyət
part-icular
part-icular-ly
con-diment
vill-age
bord-er
polit-ical
Wiesbaden
Uni-vers-idad
Geuthner
typo-logie
Maur-itanie
nomades
Maur-itan-ian
dia-lecto-logy
Sahar-iennes
Uni-vers-ity
de-scend-ants
NENA-speak-ing
speak-ing
origin-al
re-captured
in-habit-ants
ethnic
minor-it-ies
drama-tic
local
long-stand-ing
regions
Nineveh
settle-ments
Ṣəndor
Mandate
sub-stitut-ing
ortho-graphy
re-fer-enced
origin-ate
twenti-eth
typ-ic-al-ly
Hobrack
never-the-less
character-ist-ics
character-ist-ic
masc-uline
coffee
ex-clus-ive-ly
verb-al
re-ana-ly-se-d
simil-ar-ities
de-riv-ation
im-pera-tive
part-iciple
dis-ambi-gu-ation
dis-ambi-gu-a-ing
phen-omen-on
phen-omen-a
traktar
com-mun-ity
com-mun-ities
dis-prefer-red
ex-plan-ation
con-struction
wide-spread
us-ual-ly
region-al
Bulut
con-sider-ation
afro-asia-tici
Franco-Angeli
Phono-logie
Volks-kundliche
dia-lectes
dia-lecte
select-ed
dis-appear-ance
media
under-stand-able
public-ation
second-ary
e-ject-ive
re-volu-tion
re-strict-ive
Gasparini
mount-ain
mount-ains
yellow
label-ing
trad-ition-al-ly
currently
dia-chronic
} 
  \togglepaper[1]%%chapternumber
}{}

\begin{document}
\maketitle 

\section{Current state and historical development}

The terms \textsc{\ili{Arabic} in the diaspora} and \textsc{\ili{Arabic} as a minority language} have been used to designate two distinct linguistic entities, namely \ili{Arabic} \textit{Sprachinseln} outside the \ili{Arabic}-speaking world and \ili{Arabic} in contemporary migration settings. The two situations correspond to the two major social processes that give rise to language contact: conquest and migration. In the former case, speakers of \ili{Arabic} were isolated from the central area in which the \ili{Arabic} language is spoken, exposed to a different dominant language, and consequently underwent a slow process of \isi{language erosion} (and eventually shift) usually spanning across several generations. This situation often gives rise to long periods of relatively stable \isi{bilingualism}, where contact-induced change is more noticeable \citep[641]{Sankoff2001}. In migration contexts, on the contrary, \isi{language shift} occurs at a faster pace, sometimes within the lifespan of the first generation and usually no later than the third \citep[151]{Canagarajah2008}. 

This chapter analyzes contact-induced change in migration contexts. Arab migration to the West started in the late nineteenth century, with the first wave of migrants who left Greater Syria to settle in the United States and \ili{Latin} America. The first migrants were mostly \ili{Christian} unskilled workers, followed by more educated \ili{Lebanese}, Palestinians, Yemenis and Iraqis after World War II. During the 1950s and 1960s, more migrants continued to settle in the US, while the unstable political situations in Palestine, Lebanon and Iraq resulted in a fourth wave in the 1970s and 1980s \citep[17--18]{Rouchdy_introduction_1992}. Because of the events that took place during the last two decades and that resulted in a further destabilization of the entire Middle East, immigration toward the US has never stopped, even though recent American policies have considerably reduced the intake of refugees and immigrants. In 2016, however, 84,995 refugees were resettled in the US, with two \ili{Arabic}-speaking countries (Syria and Iraq) featuring among the top five states that make 70\% of the total intake.\footnote{Data come from the US Department of State. \url{https://www.state.gov/j/prm/releases/factsheets/2017/266365.htm}, accessed April 2, 2019.}

Large-scale migration to western Europe from \ili{Arabic}-speaking countries began in the wake of the decolonization process during the 1960s and mainly involved speakers from North Africa (Morocco, Algeria and Tunisia). Following a common trend in labor migration, men arrived first, followed by their wives and children. In 1995, a total of 1,110,545 Moroccans, 655,576 Algerians and 279,813 Tunisians lived in Europe, mostly in France, the Netherlands, Belgium, Germany and Italy \citep[259--260]{BoumansdeRuiter2002}. The socioeconomic profile of the first immigrants mainly consisted of unskilled laborers, usually with low education rates. After six decades from the first wave of immigration, however, most communities consist today of a first, second and third generation, while the political upheaval which started at the end of 2010 resulted in a new wave of young immigrants. Both old and new immigrants had to face the economic crisis that hit Europe in the early 1990s and, again, in 2007, with particularly harsh consequences for the immigrant population \citep[261]{BoumansdeRuiter2002}.

The sociolinguistic profile of \ili{Arabic}-speaking communities in the \isi{diaspora} is quite diverse in different parts of the world and can be analyzed using the ethnolinguistic vitality framework, according to which status, demographics, and institutional support shape the vitality of a linguistic minority \citep{GilesTaylor1977,Ehala2015}. \ili{Arabic}-speaking immigrants do not usually enjoy a particularly high status, while the level of institutional support is variable. The first waves of immigration to the US, for instance, had to face an environment that was generally hostile to foreign languages. The \ili{English}-only movement actively worked to impose the exclusive employment of \ili{English} in public places, while the immigrants themselves committed to learning and using \ili{English} to integrate into mainstream American life. Only in the aftermath of 9/11 did American policymakers begin to re-evaluate the importance of \ili{Arabic} (and other heritage languages), considering it a resource for homeland security \citep[319--320]{Albirini2016}. Other countries, such as the Netherlands, provided higher levels of formal institutional support, including \ili{Arabic} in school curricula. These efforts did not achieve the desired goals, however, mostly because the great linguistic diversity of the \ili{Moroccan} community living in the Netherlands cannot be adequately represented in the teaching curricula. Moroccans in the Netherlands, in fact, speak different \ili{Arabic} dialects, alongside three main varieties of \ili{Berber}, namely \ili{Tashelhiyt}, \ili{Tamazight} and \ili{Tarifiyt} \citep[160--161]{ExtraDeRuiter1994}. The voluntary home language instruction program, however, provides instruction in \ili{Modern Standard} \ili{Arabic}, even though writing skills are only taught starting from third grade \citep[163--165]{ExtraDeRuiter1994}. This is not, of course, the language students are exposed to at home, but attempts to introduce \ili{Moroccan} dialect or \ili{Berber} are generally opposed by parents, who value \ili{Classical} \ili{Arabic} for its religious and cultural relevance. Similar Home Language Instruction programs are found in most European countries, even though their implementation is sometimes carried out by local governments (in the Netherlands and Germany), private organizations (in Spain) or even by the governments of the origin country (in France) \citep[264--265]{BoumansdeRuiter2002}. The \ili{Italian} town of \isi{Mazara del Vallo} in \isi{Sicily} represents an extreme case, since the members of the \ili{Tunisian} community obtained from the \ili{Tunisian} government the opening of a \ili{Tunisian} school, where a complete \ili{Arabic} curriculum is offered and \ili{Italian} is not even taught as a second language. Until the end of the 1990s, this school, opened in 1981, was the first choice for \ili{Tunisian} families, who hoped for a possible return to Tunisia. When it eventually became clear that this was unlikely to happen, enrollments consequently declined, which means that \ili{Arabic} teaching is no longer available to the community in any form \citep[73--77]{Danna2017book}. Issues of \isi{diglossia} and language diversity thus undermine Home Language Instruction programs, which usually occupy a marginal position within school curricula.

Given the generally low status of, and insufficient institutional support for, \ili{Arabic}-speaking communities in the \isi{diaspora}, demographic factors are often decisive in determining the ethnolinguistic vitality of the community. While speakers of \ili{Arabic} are usually scattered in large areas where the dominant language is prevalently spoken, in some \ili{Dutch} towns \ili{Moroccan} youth make up 50\% of the population of certain neighborhoods \citep[50]{Boumans2004}. At the other end of the continuum, we find closely-knit communities, living in the same neighborhood, such as in \isi{Mazara del Vallo}, where Tunisians hailing from the two neighboring towns of Mahdia and Chebba constitute up to 70\% of the population of the old town \citep[27]{Danna2017book}. All things being equal, given the low status of the \ili{Tunisian} community and the mediocre institutional support they receive, it is primarily demographic factors which have resulted in the preservation of \ili{Arabic} in this community beyond the threshold of the third generation.\footnote{Other factors also played a minor role in the preservation of \ili{Arabic} in \isi{Mazara del Vallo} \citep[80--81]{Danna2017book}.}

In the light of what has been said above, and despite some notable exceptions, \ili{Arabic} diasporic communities are characterized by relatively rapid processes of \isi{language shift}, both in the US \citep[29]{Daher1992} and in Europe \citep[282]{BoumansdeRuiter2002}. This means that the processes of contact-induced change observed in diasporic communities of \ili{Arabic} are generally the prelude to language loss. The importance of studying \isi{language change} in migrant languages, however, also resides in the fact that the same changes usually take place, at a much slower rate, in the standard spoken in the homeland. Internally motivated change in diasporic varieties, from this perspective, often represent an accelerated version of \isi{language change} in the homeland. Contact-induced change, on the other hand, sometimes suggests parallels with the socially different process of pidginization \citep[194--195]{GonzoSaltarelli1983}. The study of \ili{Arabic}-speaking diasporic communities, thus, can help us shed light on the more general evolution of the language, with regard to both contact-induced and internally-motivated change.

\section{Contact languages}

Contact languages for diasporic \ili{Arabic}-speaking communities include, but are not restricted to, \ili{American} \citep{Rouchdy_arabic_1992} and \ili{British} \ili{English} \citep{AbuHaidar2012}, \ili{Portuguese} in Brazil \citep[292]{Versteegh2014book}, \ili{French}  \citep{BoumansCaubet2000}, \ili{Dutch} \citep{Boumans2000,Boumans2004,Boumans2007,BoumansCaubet2000,BoumansdeRuiter2002}, \ili{Spanish} \citep{Vicente2005,Vicente2007} and \ili{Italian} \citep{Danna2017book,Danna2018Chebba}. Some contact situations are better described than others, as in the case of \ili{English}, \ili{French} and \ili{Dutch}. At the other end of the continuum, research on the outcome of contact between \ili{Italian} and \ili{Arabic} is extremely recent, and data on \ili{Portuguese} are scarce.

In the following sections, we will draw from the sources so far cited to describe the main phenomena of \isi{language change} occurring in diasporic \ili{Arabic} at the phonological, morphological, syntactic and lexical level, highlighting possible parallels with comparable changes in other non-diasporic varieties of \ili{Arabic}.

\section{Contact-induced changes in diasporic Arabic}

Despite the great variety of contact languages, it is possible to individuate a number of phenomena that predictably occur in diasporic \ili{Arabic}-speaking communities. It is not always easy, however, to assess whether an individual phenomenon is due to contact or whether it is, on the contrary, the result of internal development \citep[377]{Romaine1989}. Gonzo \& Saltarelli (\citeyear[177]{GonzoSaltarelli1977}) put the matter as follows:

\begin{quote}While it seems clear that some types of changes are due to interference from the dominant language, and others may be attributable to sociological and other external pressures, there are some changes which are language-internal. The latter type is in accordance with a principle of regularization and code reduction which one might expect when the language is acquired in a weakly monitored sociolinguistic environment.\end{quote}

The concept of \textsc{weakened} \textsc{monitoring}, a situation in which a generally accepted standard and the reinforcement of correct norms are lacking, is an effective tool of analysis when investigating \isi{language change} in diasporic communities \citep{GonzoSaltarelli1977,GonzoSaltarelli1983}. In a situation of weakened monitoring, processes of \isi{language change} that are occurring slowly in other varieties of the language can be sped up.

In the following sections, interference between languages will be referred to as \textsc{transfer}, which occurs from the \textsc{source} \textsc{language} (\isi{SL}) to the \textsc{recipient} \textsc{language} (\isi{RL}). If the speaker is dominant in the \isi{SL}, \isi{transfer} is more specifically defined as \textsc{imposition}. If, on the contrary, the speaker is dominant in the \isi{RL}, \isi{transfer} is defined as \textsc{borrowing} \citep{VanCoetsem1988, VanCoetsem2000,Lucas2015}. While the concept of linguistic \textsc{dominance} will be extensively used in this paper, one final caveat concerns the difficulty of individuating the dominant language (which may actually shift) in second-generation speakers. Lucas identifies a category of 2L1 speakers, who undergo the simultaneous acquisition of two distinct native languages \citep[525]{Lucas2015}. The linguistic trajectory of most second-generation speakers, however, usually involves two consecutive stages in which first the heritage and then the socially dominant language function as the dominant language. While the heritage language is almost exclusively spoken at home during early childhood, in fact, second-generation speakers gradually shift to the socially dominant language when they start school and consequently expand their social network.


 
 \subsection{Phonology}


In the domain of phonology, diasporic varieties of \ili{Arabic} generally go in the direction of the loss of marked phonemes \citep[293]{Versteegh2014book}. It is generally the \isi{emphatic} and post-velar phonemes that undergo erosion, though the loss is usually not systematic, featuring a great deal of inter and intra-individual variation. In non-diasporic communities, adults, peers and institutions provide corrective feedback to children during their process of language acquisition, while in immigrant communities, due to the weakened monitoring mentioned above, the chain of intergenerational \isi{transmission} is less secure. Some phenomena of phonetic loss thus have a developmental origin, and are equally common in pidgins and dying languages \citep[372–373]{Romaine1989}. Consider the following example:

\ea\label{ex:key:maz}
{\ili{Tunisian} \ili{Arabic}, \isi{Mazara del Vallo} \citep[85]{Danna2017book}}\\
\gll ʕala ḫāṭr-i ʕarbi u nnəžžəm naʕrəf aktər wāəd mia lingua\\
     on thought-\textsc{obl.1sg} Arab and can.\textsc{impf.1sg} know.\textsc{impf.1sg} more one \textsc{poss.1sg.f} language\\
\glt `Because I’m an Arab and I can know above all my language.'
\z

The speaker in sample  \REF{ex:key:maz} realizes the voiced \isi{pharyngeal} fricative /ʕ/, one of the phonemes that are usually lost, but then fails to realize its voiceless counterpart /ḥ/ in \textit{wāəd} < \textit{wāḥəd} ‘one’.\footnote{Similar phenomena of phonetic \isi{simplification} occur in peripheral varieties of \ili{Arabic} and \textit{Sprachinseln}, such as \ili{Nigerian} \ili{Arabic} (\citealt[19--20]{Owens1993}; this volume), \ili{Cypriot Maronite} \ili{Arabic} (\citealt{Borg1985}; Walter, this volume), \ili{Uzbekistan} \ili{Arabic}  (\citealt{Seeger2013article}) and \ili{Maltese} (\citealt[299]{BorgAzzopardi-Alexander1997}; Lucas \& Čéplö, this volume).\ia{Lucas, Christopher@Lucas, Christopher}\ia{Čéplö, Slavomír@Čéplö, Slavomír}\ia{Walter, Mary Ann@Walter, Mary Ann}\ia{Owens, Jonathan@Owens, Jonathan} The single varieties here mentioned vary with regard to the phonological \isi{simplification} they underwent.} Similar phenomena also occur, as noted above, in \ili{Arabic}-based pidgins and creoles, such as \ili{Juba} \ili{Arabic} (\citealt[17, 21]{Manfredi2017}; cf. Avram, this volume).\ia{Avram, Andrei@Avram, Andrei} 

In the process of phonological erosion, therefore, contact languages seem to have a limited impact. If the dominant language does not feature, in its phonemic inventory, the \isi{phoneme} that is being eroded, it fails to reinforce whatever input young bilingual speakers receive in the other L1 in the contexts of primary socialization. Reduced input and weakened monitoring, however, play a bigger role, allowing forms usually observed in the earliest stages of language acquisition by \isi{monolingual} children to survive and spread. It is relatively common, for instance, to observe the presence of shortened or reduced forms, such as \textit{qe} < \textit{lqe} ‘he found’, \textit{ḥal} < \textit{nḥal} ‘bees’, \textit{ləd} < \textit{uləd} ‘kid’, which sometimes give rise to phenomena of compensation, such as in \textit{uləd} > \textit{ləd} > \textit{lədda} ‘kid’ (\ili{Tunisian} diasporic \ili{Arabic}, \isi{Mazara del Vallo}, Italy; \citealt[85]{Danna2017book}). In diasporic communities, reduced forms are more easily allowed to survive and spread, occurring in the speech of teenagers, as in the examples reported here. Once again, the same phenomenon also occurs in \isi{pidgin} and dying languages: 

In the case of dying and \isi{pidgin} languages it may be that children have greater scope to act as norm-makers due to the fact that a great deal of variability exists among the adult community \citep[372–373]{Romaine1989}.

In conclusion, the phonology of diasporic \ili{Arabic} does not seem to be heavily influenced by borrowing from contact languages. The combined action of reduced input and weakened monitoring, on the other hand, is responsible for the unsystematic loss of marked phonemes and for the survival and spread of reduced forms.


 
 \subsection{Morphology}


The complex mixture of concatenative and \isi{non-concatenative} morphology in the domain of \ili{Arabic} plural \isi{formation} has been one of the main focuses of research in situations of language contact resulting from migration. Once again, borrowing from contact languages and independent developments occur side by side.

In \ili{Arabic}, both concatenative and \isi{non-concatenative} morphology contribute to plural \isi{formation}. Concatenative morphology, which consists in attaching a suffix to the singular noun, yields the so-called sound plurals, that is, in spoken \ili{Arabic}, the plural suffixes \textit{{}-īn} and \textit{{}-āt} respectively. It has been argued that sound feminine plural is the default plural form according to the morphological underspecification hypothesis, even though masculine is the default \isi{gender} in all other domains of plural morphology \citep[855–856]{AlbiriniBenmamoun2014}. While sound masculine plural is specified for [+human], in fact, sound feminine plural has the semantic feature [±human]. Non-concatenative, or broken, plurals require a higher cognitive load, since they involve the mapping of a vocalic template onto a consonantal \isi{root}.\footnote{The notion of \isi{root and pattern}, which has long been at the core of the morphology of \ili{Arabic}, has recently been criticized \citep{Ratcliffe2013}, even though psycholinguistic studies seem to confirm the existence of the \isi{root} in the mental lexicon of native speakers \citep{Boudelaa2013}.} Sound feminine plurals are acquired by children by the age of three, while broken plurals involving geminate and defective \isi{roots} are not mastered until beyond the age of six \citep[857–858]{AlbiriniBenmamoun2014}.  After the age of five, however, \isi{heritage speakers} of \ili{Arabic} become increasingly exposed to their L2, which encroaches upon their acquisition of broken plurals. It has thus been convincingly demonstrated that \isi{heritage speakers} display a better command of sound plurals and that, in the domain of broken plurals, some are more affected by \isi{language erosion} than others \citep[858–859]{AlbiriniBenmamoun2014}. Across different varieties of diasporic \ili{Arabic}, therefore, plural morphology displays both contact phenomena due to borrowing and internal developments that are akin to what might be called \textsc{restructuring}, that is:

\begin{quote}changes that a speaker makes to an L2 that are the result not of \isi{imposition} but of interpreting the L2 input in a way that a child acquiring an Ll would not \citep[525]{Lucas2015}.\footnote{In this case, of course, the speaker would not be re-interpreting an L2, but an L1 learned under reduced input conditions and subject to \isi{language erosion}.}\end{quote}

Borrowing from the contact languages can take two forms. In rare cases, the suffix plural morpheme of the contact language is directly borrowed, as in the examples \textit{ḥuli-s} ‘sheep-PL’, \textit{ḥmar-s} ‘donkeys’ and \textit{l-ʕud-s} ‘the horses’\footnote{The target form here is \textit{ʕewd-an}, so that also vowel quality is not standard.} collected from one \ili{Moroccan} informant in the Netherlands \citep[274]{BoumansdeRuiter2002}. Sometimes, however, \isi{transfer} works in a subtler way, which consists in the generalization of the sound masculine plural suffix \textit{{}-īn},\footnote{The suffix for masculine plural \textit{–īn} is realized with a short vowel in the diasporic \ili{Moroccan} varieties that are being discussed.} by \isi{analogy} with the default form of the contact language, yielding \textit{ḥul-in} ‘sheep-PL’, \textit{ḥmār-in} ‘donkeys’, \textit{ʕewd-in} ‘horses’ \citep[274]{BoumansdeRuiter2002}. A study conducted by Albirini \& Benmamoun (\citeyear[866–867]{AlbiriniBenmamoun2014}) shows that L2 learners of \ili{Arabic} usually tend to overgeneralize the sound masculine plural, wrongly perceived as a default form, while \isi{heritage speakers} more often resort to the \ili{Arabic}-specific default, i.e. sound feminine plural. The cases of borrowing reported above, therefore, represent an idiosyncratic exception.

On the other hand, the non-optimal circumstances under which \ili{Arabic} is learned in diasporic communities often result in overgeneralization processes that cannot be directly attributed to contact. One of them is, as noted above, the generalization of the sound feminine plural \textit{{}-āt}. In the domain of broken plurals, moreover, not all patterns are equally distributed. The iambic pattern, consisting of a light syllable followed by one with two moras (CVCVVC), is the most common among \ili{Arabic} broken plurals \citep[857]{AlbiriniBenmamoun2014}. As a consequence, it is often generalized by \isi{heritage speakers} of \ili{Levantine} varieties (\ili{Syrian}, \ili{Lebanese}, \ili{Palestinian} and \ili{Jordanian}) living in the US, yielding forms such as: \textit{fallāḥ} ‘farmer’, pl. \textit{aflāḥ}/\textit{fulūḥ} (target plural \textit{fallāḥ-īn}); \textit{šubbāk} ‘window’, pl. \textit{šubūk} (target plural \textit{šabābīk}); \textit{ṭabbāḫ} ‘cook’, pl. \textit{ṭabāʔiḫ} (target plural \textit{ṭabbāḫ-īn}) \citep[865]{AlbiriniBenmamoun2014}.\footnote{The overgeneralization of some \isi{broken plural} patterns indicates that the \isi{root and pattern} system is still productive in \isi{heritage speakers}, as opposed, for instance, to speakers of \ili{Arabic}-based pidgins and creoles. Recent studies, however, have advanced the hypothesis that the iambic pattern involves operations below the level of the word, but without necessarily entailing the mapping of a template onto a consonantal \isi{root} \citep[112]{AlbiriniSaadah2014}.} 

Borrowing does not involve plural morphemes only, but other classes as well. In \isi{Mazara del Vallo}, for instance, young speakers occasionally use the \ili{Sicilian} \isi{diminutive} morpheme \textit{{}-eddru} with \ili{Arabic} names, creating morphological hybrids of the kind illustrated in  \REF{ex:key:maz1} : 

\ea\label{ex:key:maz1}
{\ili{Tunisian} \ili{Arabic}, \isi{Mazara del Vallo} \citep[107]{Danna2017book}}\\
\gll Grazie safwani-ceddruu\footnotemark\\
     thanks Safwan-\textsc{dim}\\
\footnotetext{The utterance appeared as a Facebook post in the timeline of one of my informants and was transcribed verbatim.}
\glt `Thanks little Safwan.'
\z

This type of borrowing, quite widespread among young speakers, seems to replicate another instance of contact-induced change that occurred in an extinct variety of \ili{Arabic}. \ili{Andalusi} \ili{Arabic}, in fact, borrowed from \ili{Romance} the \isi{diminutive} morpheme -\textit{el} (e.g. \textit{tarabilla} ‘mill-clapper’ < \textit{ṭarab+ella} ‘little music’), incidentally etymologically \isi{cognate} with the \ili{Sicilian} \textit{{}-eddru} (\ili{Latin} \textit{-ellum} > \ili{Sicilian} \textit{-eddru/-eddu}) \citep[60]{Andalusi2013}. The behavior of the young \ili{Tunisian} speakers of \isi{Mazara del Vallo}, who use these \ili{Sicilian} diminutives in a playful mode, might represent the first stage of the same process that resulted in in the \isi{transfer} of this morpheme into \ili{Andalusi} \ili{Arabic} \citep[108]{Danna2017book}.

While plurals represent one of the most common areas of change in diasporic \ili{Arabic}, morpheme borrowing is a much rarer phenomenon, which probably occurs in situations of more pronounced \isi{bilingualism}. The above two examples, however, provide a representative exemplification of the effect of language contact in the domain of morphology.


 
 \subsection{Syntax}


Borrowing and restructuring also happen in the domain of syntax. As has been noted both for Moroccans in the Netherlands \citep[99]{deRuiter1989} and Tunisians in Italy (personal research), second-generation speakers tend to use simpler clauses than \isi{monolingual} speakers, namely main or subordinate clauses to which no other clause is attached, as evident from the following sample:


\ea
{\ili{Tunisian} \ili{Arabic}, \isi{Mazara del Vallo} (personal research)}\\
\gll m-baʕd əl-uləyyəd rqad u l-kaləb zāda u l-žrāna ḫaržət mən əl-wāḥəd ēh dabbūsa\\
 from-after \textsc{def}-boy.\textsc{dim} sleep\textsc{.prf.3sg.m} and \textsc{def-}dog also and \textsc{def-}frog exit\textsc{.prf.3sg.f} from \textsc{def-}one \textsc{hesit} bottle\\
\glt {`Then the little boy slept and also the dog and the frog escaped from the hum bottle.'}\\
\z

Accordingly, they also display the effects of \isi{language erosion} in establishing long-distance dependencies typical of more complex clauses \citep[305]{Albirini2016}.

\ili{Palestinian} and \ili{Egyptian} speakers born in the US have also been found to realize overt pronouns in sentences that opt for the pro-drop strategy in the speech of monolinguals, which is probably due to the influence of \ili{English} \citep[283]{AlbiriniSaadah2014}. Preliminary observations on second-generation Tunisians in Italy, in fact, do not show the same phenomenon. Since \ili{Italian} is, like \ili{Arabic}, a pro-drop language, the use of overt pronouns in American diasporic \ili{Arabic} can be considered as a case of syntactic borrowing or \isi{convergence} \citep{Lucas2015}, depending on the speakers’ degree of \isi{bilingualism}.

The syntax of \isi{negation} is another area in which \isi{language erosion} triggers phenomena that seem to be happening, albeit at a slower rate, in non-diasporic communities. \ili{Egyptian} speakers in the US, for instance, seem to overgeneralize the monopartite negatior \textit{miš} / \textit{muš} at the expense of the default discontinuous verbal negator \textit{ma…-š}: 

\ea\label{ex:key:egy}
{\ili{Egyptian} \ili{Arabic} in the US \citep[482]{AlbiriniBenmamoun2015}}\\
\gll huwwa miš rāḥ l-kaftiria\\
     \textsc{3sg.m} \textsc{neg} go.\textsc{prf.3sg.m} to-cafeteria\\
\glt `He didn’t go to the cafeteria.'
\z

Example \REF{ex:key:egy} represents a deviation from the standard \ili{Cairene} dialect spoken by monolinguals. In Egypt, however, the negative \isi{copula} \textit{miš{\textasciitilde}muš} represents a pragmatically marked possibility to negate the \textit{b-} imperfect \citep[302]{Brustad2000}, while in \ili{Cairo} it is now the standard \isi{negation} for \isi{future} \isi{tense} (\textit{miš} \textit{ḥa-…}, contrasting with \textit{ma-ḥa-…-š} in some areas of Upper Egypt \citep[285]{Brustad2000}. More generally, therefore,  \textit{miš{\textasciitilde}muš} is gaining ground at the expense of the discontinuous \isi{negation} \citep[285]{Brustad2000}, so that what we observe in diasporic \ili{Egyptian} \ili{Arabic} might just be an accelerated instance of the same process.

Another major area of \isi{language change}, documented in most diasporic languages, is the erosion of complex \isi{agreement} systems \citep[192]{GonzoSaltarelli1983}. In diasporic \ili{Arabic}, \isi{heritage speakers} show relatively few problems with subject–verb \isi{agreement}, but struggle with the subtleties of noun–adjective \isi{agreement} \citep[8]{AlbiriniChakrani2013}. While subject--verb \isi{agreement} involves a verbal paradigm with a relatively large number of cells, it is nevertheless simpler than noun–adjective \isi{agreement}, since plural nouns can trigger adjective \isi{agreement} in the sound or \isi{broken plural} or in the feminine singular, depending on factors involving humanness, individuation, and the morphological shape of both the noun and the adjective, with marked dialectal variation \citep[103–104]{Danna2017article}. Heritage speakers thus perform significantly better when default \isi{agreement} in the masculine singular is required \citep[8]{AlbiriniChakrani2013}, but display evident signs of \isi{language erosion} when more complex structures are involved: 

\ea\label{ex:key:usa}
{\ili{Egyptian} \ili{Arabic} in the US \citep[740]{Albirini2014}}\\
\gll wi-kamān baḥibb arūḥ l-Detroit ʕašān ʕinda-ha maṭāʕim *mumtaz-īn\\
     and-also love.\textsc{impf.ind.1sg} go.\textsc{impf.1sg} to-Detroit because at-\textsc{3sg.f} restaurant.\textsc{pl} excellent-\textsc{pl.m}\\
\glt `And I also like to go to Detroit because it has excellent restaurants.'
\z

In \REF{ex:key:usa}, the speaker selects the sound masculine plural, while non-human plural nouns require either the \isi{broken plural} or the feminine singular in \ili{Egyptian} \ili{Arabic}. Once again, \isi{language change} in diasporic \ili{Arabic}, where the language is learned under reduced input conditions, tends to replicate processes of \isi{language change} that happened or are happening in the \ili{Arabic}-speaking world. In the case of \isi{agreement}, the standardization that the \isi{agreement} system underwent in the transition from pre-\ili{Classical} to \ili{Classical} \ili{Arabic} has been convincingly explained as emerging from the overgeneralization of frequent patterns by L2 learners \citep{Belnap1999}.

Finally, isolated cases show syntactic borrowing or \isi{convergence}\footnote{Once again, considering this phenomenon as syntactic borrowing or \isi{convergence} depends on the speaker’s language dominance, which is not clear from the source and is not easily ascertained in second-generation speakers, whose dominant language is often subject to shift.} at the level of \isi{word order}, which is usually preserved in diasporic contexts, as in the example in \REF{neth}.

\protectedex{
\ea\label{neth}
{\ili{Moroccan} \ili{Arabic} in the Netherlands \citep[105]{Boumans2001}}\\
\gll u ʕṭat l-u dyal-u l-lḥem\\
     and give.\textsc{prf.3sg.f} to-\textsc{3sg.m} \textsc{gen-3sg.m} \textsc{def}{}-meat\\
\glt `And she gave it [i.e. the dog] its meat.'
\z
}

This example illustrates an extreme case of \isi{word order} change, in which the possessive \textit{dyal-u} ‘its’ precedes the head. Overgeneralization of permissible (but sometimes pragmatically marked) word orders, however, occur much more frequently. \ili{Egyptian} \isi{heritage speakers} in the US, for instance, use SVO order 77.65\% of the time, vs 52.64\% for \ili{Egyptian} native speakers \citep[280–281]{AlbiriniSaadah2011}.

In situations of stable \isi{bilingualism}, such as in some \ili{Arabic} \textit{Sprachinseln}, \isi{convergence} with contact languages can result in permanent alterations to \isi{word order}. In Buxari \ili{Arabic}, for instance, transitive verbs feature a mandatory SOV \isi{word order}, with optional resumptive pronoun after the verb. Cleft sentences such as the following one are quite common in all \ili{Arabic} dialects:

\ea\label{ex:key:}
{\ili{Egyptian} \ili{Arabic} \citep[145]{Ratcliffe2005}}\\
\gll il-fustān  gibt-u\\
     \textsc{def-}dress  get.\textsc{prf.1sg-3sg.m}\\
\glt ‘I got the dress.’
\z

In Bukhari \ili{Arabic}, which has long been in contact with SOV languages (such as \ili{Persian} and \ili{Tajik}), this structure became the standard for transitive verbs, so that the resumptive pronoun can also be dropped, as in the following sample:

\ea\label{ex:key:}
{Bukhari \ili{Arabic}  \citep[144]{Ratcliffe2005}}\\
\gll   fāt ʕūd ḫada\\
       \textsc{indef} stick take\textsc{.prf.3sg.m}\\
\glt ‘He took a stick.’
\z


 
 \subsection{Lexicon}


In the domain of lexical borrowing, which has attracted considerable interest among scholars, the situation of \isi{bilingualism} in diasporic contexts poses some methodological issues in the individuation of actual \isi{loanwords}. The production of \isi{heritage speakers}, in fact, is inevitably marked by frequent phenomena of \isi{code-switching}, which makes difficult to distinguish between nonce-borrowings \citep{Poplack1980} and \isi{code-switching}. If we define lexical borrowing as “the diachronic process by which languages enhance their vocabulary” \citep[106]{Matras2009}, in fact, it is not clear which language is here enhancing its vocabulary, since diasporic varieties of \ili{Arabic} are not discrete varieties and feature the highest degree of internal variability. A possible solution to this impasse consists in looking exclusively at the linguistic properties of the alleged \isi{loanword}. In this vein, Adalar \& Tagliamonte (\citeyear[156]{AdalarTagliamonte1998}) have shown that, when foreign-origin nouns appear in contexts in which they are completely surrounded by the other language, they are treated like borrowings (in this case, nonce-borrowings) at the phonological, morphological and syntactic level. When, on the contrary, they appear in bilingual (or multilingual) utterances, they represent cases of \isi{code-switching}, patterning with the language of their etymology. The domain of lexical borrowing in diasporic varieties of \ili{Arabic}, however, is an area that needs further research.

\section{Conclusion}

This chapter has offered an overview of the main phenomena of contact-induced change observed in \ili{Arabic} diasporic communities, distinguishing them from internal developments due to reduced input and weakened monitoring. Diasporic communities rarely feature situations of stable \isi{bilingualism}, so that \isi{language change} usually corresponds to language attrition and is followed by the complete shift to the dominant language. The study of \isi{language change} in diasporic communities, however, constitutes an interesting field of investigation, both in itself and for the insight it can give us into \isi{language change} in \isi{monolingual} communities. Change at the phonological, morphological and syntactic level finds parallels in comparable phenomena that have occurred in the history of \ili{Arabic} (such as in the case of \isi{agreement}) or that are occurring as we speak (such as in the case of the spread of the negator \textit{miš} in \ili{Egyptian} \ili{Arabic}). Not by chance, similar phenomena also occur(red) in the \ili{Arabic}-based pidgins of East Africa, such as \ili{Juba} \ili{Arabic}. Various scholars, in fact, have maintained that the mechanisms of change differ in the degree of intensity, but not in their intrinsic nature, from those operating in less extreme situations of contact (e.g. \citealt[8]{Miller2003}; \citealt[528]{Lucas2015}).

On the other hand, the analysis of contact phenomena in diasporic communities poses some methodological issues with regard to the categories of borrowing, \isi{imposition} and \isi{convergence} \citep{VanCoetsem1988,VanCoetsem2000}. These categories, in fact, imply the possibility to define clearly the speaker’s dominant language or, at least, to define him as a stable 2L1 speaker. This is rarely the case with \isi{heritage speakers}, whose repertoires follow trajectories in which language dominance shifts, usually from the heritage language to the socially dominant one. This process is usually concomitant with the beginning of school education, but we lack theoretical and methodological tools to determine with accuracy the speaker’s position on the trajectory.

Further avenues of research on this topic thus include a more rigorous investigation of emerging and shifting repertoires and the analysis of the complex relation between diasporic languages, pidginization and creolization, which has already been the object of a number of contributions (e.g. \citealt{GonzoSaltarelli1983,Romaine1989}).

\section*{Further reading}

 \citet{Rouchdy_arabic_1992} is the first description of \ili{Arabic} in the US.\\
 \citet{Rouchdy2002} analyzes more broadly language contact and conflict, with a section devoted to \ili{Arabic} in the \isi{diaspora}.\\
 \citet{Owens2000editor} collects essays on \ili{Arabic} as a minority language, \isi{focusing} on both \textit{Spracheninseln} and diasporic \ili{Arabic}, but introducing also historical and cross-ethnic perspectives.
 
 \section*{Acknowledgements}
I am grateful to the University of Mississippi, which generously funded this research and my fieldwork in \isi{Mazara del Vallo}. To Adam Benkato, for reading the manuscript and providing, as always, his valuable feedback. To all my informants in \isi{Mazara del Vallo}, whose patience during the interviews was only matched by their warm hospitality.
 
 \section*{Abbreviations}

\begin{tabularx}{.55\textwidth}{@{}lQ@{}}
\textsc{1, 2, 3} & 1st, 2nd, 3rd person \\
\textsc{def} & \isi{definite} \isi{article} \\
\textsc{dim} & \isi{diminutive} \\
\textsc{f} & feminine \\
\textsc{hesit} & hesitation \\
\textsc{gen} & genitive \\
\textsc{impf} & imperfect (prefix conjugation) \\
\textsc{ind} & indicative \\
\end{tabularx}%
\begin{tabularx}{.5\textwidth}{@{}lQ@{}}
\textsc{indef} & indefinite \\
\textsc{m} & masculine \\
\textsc{neg} & negative \\
\textsc{obl} & oblique \\
\textsc{pl} & plural \\
\textsc{poss} & possessive pronoun \\
\textsc{prf} & perfect (suffix conjugation) \\
\textsc{sg} & singular \\
\end{tabularx}%


\sloppy\printbibliography[heading=subbibliography,notkeyword=this]\end{document}