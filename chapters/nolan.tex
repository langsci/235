\documentclass[output=paper]{langsci/langscibook} 
\title{Mediterranean Lingua Franca}
\author{Joanna Nolan\affiliation{SOAS University of London}}
% \chapterDOI{} %will be filled in at production

% % % \epigram{Change epigram in chapters/01.tex or remove it there}

\abstract{This chapter explores the effect of Arabic contact on Lingua Franca, an almost exclusively oral pidgin spoken across the Mediterranean and along the North African coastline from the seventeenth to the nineteenth centuries. The chapter highlights the phonological and lexical impact Arabic appears to have had on Lingua Franca.}

\maketitle
\begin{document}
	
\section{Overview and historical development}
	
	Today, \textsc{lingua franca} is a term describing a language used by two or more linguistic groups as a means of communication, often for economic motives. Typically, none of the groups speak the chosen language as their native tongue. The original and eponymous Lingua Franca, however, was a trading language, used among and between Europeans and Arabs across the Mediterranean \citep{KahaneKahane1976}. Its exact historical and geographical \isi{roots} – as well as its precise \isi{lexifier} languages – prove elusive. \citet{Hall1966} dates Lingua Franca’s birth to the era of the crusades, while other linguists \citep{Cifoletti2004,Minervini1996} suggest that it took \isi{root} on the North African Barbary Coast (in the Regencies of Algiers, Tunis and Tripoli) at the close of the sixteenth century. 
	
	Contention extends to its very name. There are several discrete etymological suggestions for the term \textit{Franca}.  Some linguists interpret \textit{Franca} as meaning ‘\ili{French}’ (e.g \citealt[3]{Hall1966}). Hall claims that France’s regional significance in the medieval era meant that its languages, specifically \ili{Provençal}, were adopted across the Mediterranean, and were a key constituent of the original Lingua Franca. In their etymological study of Lingua Franca, Kahane \& Kahane (\citeyear[25]{KahaneKahane1976}), on the other hand, assert that the name \textit{Lingua Franca} is rooted in the East and the Byzantine tradition, stemming from the \ili{Greek} word \textit{phrangika}, which denoted \ili{Venetian} as much as \ili{Italian}, or indeed, as any Western language \citep[31]{KahaneKahane1976}. An alternative etymology for \textit{Lingua Franca}, espoused by Schuchardt (\citeyear[74]{Schuchardt1909}) among others, is from the \ili{Arabic}, \textit{lisān al-faranǧ} ‘language of the Franks’. This initially referred to \ili{Latin} and then to describe a trading language employed largely by \isi{Jews} across the Mediterranean. It later came to encompass the languages of all Europeans, but particularly Italians \citep[26]{KahaneKahane1976}.  
	
	Evolving from its maritime origins, by the late sixteenth century Lingua Franca was the language of pirates of the North African Barbary coast and their captured slaves, and, as such, the subject of legend and myth. Indeed, the variation found in the accounts of Lingua Franca, and descriptions of its linguistic makeup lead some linguists \citep{Minervini1996,Mori2016} to suggest that there may have been multiple Lingua Francas or that it was simply second-language \ili{Italian}. As Schuchardt (\citeyear[88]{Schuchardt1909}), identified, Lingua Franca – perhaps above all in its resistance to theoretical classification – adheres to Heraclitus’ philosophy of \textit{panta rei}  ‘everything is in flux’. 
	
	Contemporaneous descriptions of Lingua Franca detailing its lexifiers and, in some cases, its salient features, come mostly from the North African Barbary Regencies and from the Levant. While the writers of these descriptions often identify \ili{Italian} and \ili{Spanish} as lexifiers, there are also, if fewer, mentions of \ili{Portuguese}, \ili{French}, \ili{Provençal}, \ili{Arabic}, \ili{Turkish} and \ili{Greek} (see below for further detail). This speaks to the hypothesis that there were multiple Lingua Francas, or perhaps more appropriately lingua francas. It also raises the frequent subjectivity of the source’s writer and their consequent interpretation of Lingua Franca. Their native language appears to have a bearing on the makeup of the Lingua Franca recorded. It may influence the lexicon they hear, as well as the \isi{orthography} they employ in their account. Equally, there is the subjectivity of the researcher to bear in mind. The assumption that a \ili{French} source, for example, has represented Lingua Franca in a particular manner overlooks the fact that the European residents, particularly of port cities across the Mediterranean, most likely would have been multilingual, with an ability to adapt their lexicon to maximize understanding and communication with their interlocutor. 
	
	The most widespread documentation of Lingua Franca comes from the Levant and northwest Africa. Algiers, and to a lesser extent, Tunis and Tripoli, had long been the crucible of Mediterranean piracy, and as the slave trade of Barbary pirates increased – with over a million European slaves held there between the sixteenth and nineteenth centuries \citep[23]{Davis2004} – so too did the domains and usage of Lingua Franca. The sixteenth–seventeenth century Spanish Abbot Diego del Haedo described it as follows: 
	
	\begin{quote}
	\textit{La que los Moros e Turcos llaman Franca… siendo todo una mexcla de lenguas cristianas y de vocablos, que son por la mayor parte Italianos e espanoles y algunos portugueses… Este hablar Franco es tan general que non hay casa do no se use}
	\end{quote}
	
	\begin{quote}
	‘that which the Arabs and Turks call Franca… being a mix of Christian languages and words, which are in the majority \ili{Italian} and \ili{Spanish} and some \ili{Portuguese}, this Franca speech is so widespread that there isn’t a house [in Algiers] where it isn’t spoken’  (\citealt[24]{Haedo1612}; author’s translation).
	\end{quote}
	
	Despite its alleged profusion in the Barbary coast, and numerous references in various contemporary sources, the corpus of Lingua Franca is remarkably limited. The exclusively European documentary sources (from diplomats, travellers, priests and slaves) provide mostly phrases and individual words and a handful of short dialogues. The most fulsome examples come from literature, and, as such, only provide indirect, and potentially less authentic, evidence of the contact vernacular. 
	
	For example, the alleged earliest record of Lingua Franca comes from an anonymous poem, \textit{Contrasto della Zerbitana,} found by Grion (\citeyear{Grion1890}) in a fourteenth-century Florentine codex, and apparently written in the late thirteenth or early fourteenth century on the island of Djerba, off the coast of Tunisia. The sixteenth century \textit{La Zingana} \citep{Giancarli1545} has an eponymous \ili{Arabic}-speaking heroine whose language features hallmarks of Lingua Franca, while the speeches of the Turkish characters in Molière’s \textit{Le Sicilien} (\citeyear{Molière1667}) and \textit{Le Bourgeois gentilhomme} (\citeyear{Molière1798} [1670]) also appear to share a number of its defining linguistic traits.
	
	The first detailed description and documentation of Lingua Franca comes in Haedo’s (\citeyear{Haedo1612}) \textit{Topographia}, a comprehensive study of Algiers, with a chapter devoted to the languages spoken there. Haedo spent several years in Algiers at the close of the sixteenth century and was even imprisoned for a number of months. His \textit{Topographia}  details the urban features, social makeup and linguistic mix of Algiers, creating an impression of Lingua Franca’s ubiquity across multiple domains, and indispensability to daily commercial, and even domestic, life.
	
    Other early sources are predominantly \ili{French}. A Trinitarian priest, Pierre Dan, was almost contemporary with Haedo in Algiers; in the mid-seventeenth century the diplomat Savary de Brèves travelled to Tripoli; and Chevalier D’Arvieux, King Louis XIV’s envoy to the region, and advisor to Molière on Turkish and \ili{Arabic} matters, visited both Algiers and Tunis. All these men offered excerpts of Lingua Franca in their writings, as well as descriptions of its character and lexifiers \citep{SavarydeBrèves1628,Dan1637,D’Arvieux1735}.
    
	Certainly seventeenth-century Algiers and the other two Barbary Regencies of Tunis and Tripoli provided the conditions for what had previously been a pre-\isi{pidgin} – with limited lexicon and a lack of stability – to evolve into the language of daily life, permeating all echelons of society and facilitating contact among and between the plurilingual populations of the urban centres. The Barbary states were, from the late sixteenth century, under the \textit{de jure} but not \textit{de facto} control of the Ottoman empire, whose support was needed to shore up the rule of the Greek \iai{Barbarossa} brothers who had ousted Spanish forces from North Africa. The two brothers (named for their red beards), \iai{Aruj} and \iai{Hizir}, gradually brought much of North Africa under Turkish sovereignty through a series of naval challenges and, later, city sieges securing power over coastal areas. The indigenous population rallied to the brothers’ cry and although the elder, \iai{Aruj}, was slain, \iai{Hizir} assumed control of Algiers in the early sixteenth century (\citealt[8]{Tinniswood2010}; \citealt[10]{Weiss2011}). He immediately offered the Ottoman Empire control over the brothers’ conquests in order to bolster his own position and ward off threats from Spain. Ottoman rule was compounded over the following decades \citep[27]{Clissold1977}. 
	
	In reality, however, the Regencies had unstable political systems with local elites vying for power. Their economy was driven by corsairing, and the real power lay in the hands of the mostly European renegades who carried out raids on land and at sea, seizing cargo and most importantly human booty, sold as slaves on their return \citep{Plantet1889}.
	
	The huge influx of captured Europeans swelled the urban population and created multinational, multidenominational and notably multilingual societies. The Flemish diplomat D’Aranda, imprisoned in Algiers in the 1660s, wrote of hearing twenty-two languages in the slave quarters of the city \citep{D’Aranda1662}. Lingua Franca emerged as a contact language accessible to the majority of slaves (though not all), given its \ili{Romance}-influenced lexicon. Although the elites were predominantly \ili{Arabic}-speaking, Europeans permeated the upper levels of Barbary society through their economic sway as corsairs and high levels of inter-marriage of Arabs and Europeans. Lingua Franca quickly became the default language within the slave quarters, known locally as \textit{bagnios}, seemingly a Lingua Franca term, and in master--slave relationships. Authors who detail the use of Lingua Franca across more than 250 years and throughout the regencies, including Pananti (\citeyear{Pananti1841}) and Broughton (\citeyear{Broughton1839}), report the regular use of Lingua Franca by \ili{Arabic}-speaking slave owners, including the Pashas, Beys and various dignitaries of the ruling households. 
	
	As noted above, Lingua Franca also elicits various opinions regarding its key lexifiers, though one common point of \isi{agreement} among its contemporary witnesses and speakers is that \ili{Italian}\footnote{Italian is a catch-all term, used by contemporary authors in Barbary, as well as linguists today, as identified by Trivellato (\citeyear[178]{Trivellato2009}): ``I write “\ili{Italian}”, “\ili{Portuguese}” and “\ili{Spanish}”, but recall that European written languages in the epoch were not fully standardized.'' \ili{Venetian} and Tuscan were both described as \ili{Italian}, for example.} and \ili{Spanish} are mentioned repeatedly as principal sources. Descriptions mostly include at least three lexifying languages, though not always the same three: while \ili{Italian} and \ili{Spanish} are consistently named, \ili{Provençal} also features (\citealt{D’Arvieux1735}: vol. 5, p.235) as does \ili{Portuguese} (\citealt[22]{D’Aranda1662}; \citealt[24]{Haedo1612}). A much later account from the Italian merchant Pananti, briefly imprisoned in Algiers, mentions \ili{Arabic} as a \isi{lexifier}: \textit{la Lingua franca è una mistura d’italiano, di arabo e di spagnolo} ‘Lingua Franca is a mix of \ili{Italian}, \ili{Arabic} and \ili{Spanish}’ \citep[201]{Pananti1841}. Within the same memoir, however, Pananti refers to African rather than \ili{Arabic} as one of three lexifiers of Lingua Franca (the other two being \ili{Italian} and \ili{Spanish}). Such inconsistency is the hallmark of many of the sources, compounding an already confused and often contradictory picture of the language. As \citet[18]{Selbach2008} observes, ``lexical variants were as much a part of the language as variant lexifiers.''
	
	The contribution of multiple languages to Lingua Franca is borne out by its lexicon: there are often several alternatives for a single meaning, as listed in the \textit{Dictionnaire} \citep{Anonymous1830}, the sole comprehensive lexical record of the language. For example, ‘to do’ is translated by \textit{far} (from the \ili{Italian} \textit{fare}), \textit{fazir} (from \ili{Portuguese} \textit{fazer}), and \textit{counchiar} (likely from \ili{Sicilian} \textit{cunzari}; \citealt[316]{Cifoletti2004}). Lingua Franca – as substantiated by its corpus of written attestations – is always labelled as such by a non-speaker. The European authors of descriptions of Lingua Franca, its \isi{diffusion} and its usage, present it as, if not foreign, certainly removed and remote from their own languages, and attribute the speaking of Lingua Franca to the Arabs and Turks (even if it is clearly a language that lexically is much closer to European, and specifically \ili{Romance}, languages).\footnote{See, for example, the exchange between Louis Bonaparte and Hyde Clarke across several issues of the periodical Athenaeum in 1877.} In her comprehensive if subjective account and an  analysis of Lingua Franca, Dakhlia highlights how citations, often expressing insults and aggressive warnings, and usually introduced into the texts by witnesses to Lingua Franca in direct speech, punctuated with exclamation marks, underline what she terms \textit{le choc linguistique de l’altérité et de la barbarie} ‘the linguistic shock (or jolt) of otherness and barbarism’ (\citealt[351]{Dakhlia2008}). This choice of words is important. Dakhlia’s association of otherness and barbarism suggests that Lingua Franca, according to the European documentary sources, was the language of the Arab oppressor. While this may apply to the corsairs and slave-masters, speaking Lingua Franca to their European captives, there are other instances within the corpus of written attestations where \ili{Arabic}-speaking elites use Lingua Franca in diplomatic, even philosophical, exchanges. For example, Louis Frank, the Bey of Tunis’ doctor, comments on the deemed impropriety of the Bey speaking formal \ili{Italian}, and his consequent use of Lingua Franca, which permeated all levels of society \citep[70]{Frank1850}:
	
	\begin{quote}
		\textit{la langue franque, c’est à dire cet italien ou provençal corrompu qu’on parle dans le Levant, lui est également familière; il avait meme voulu essayer d’ap\-prendre à lire et à écrire l’italien pur-toscan: mais les chefs de la religion l’ont détourné de cette etude, qu’ils prétendaient être indigne d’un prince musulman.}
	\end{quote}
	
	\begin{quote}
		‘Lingua Franca, or rather this bad \ili{Italian} or \ili{Provençal} spoken in the Levant, is equally familiar to him; he had actually wanted to learn to read and write  pure Tuscan \ili{Italian}; but his religious chiefs had warned him off such study, which they claimed was unworthy of a Muslim prince’ (author's translation).
	\end{quote}
	
	Frank wrote of the intriguing linguistic, socio-political, cultural and even religious conflation evidenced in Lingua Franca, describing an encounter with a Muslim beggar, who implored: “\textit{Donar mi meschino la carità d’una carrouba}\footnote{Until 1891 a \textit{carrouba} was worth 1/16 of a {Tunisian} \textit{piaster}, according to \citet{Rossetti1999}.} \textit{per l’amor della Santissima Trinità e dello gran Bonaparte}”, ‘Please to give miserable me the charity of a penny for the love of the most holy Trinity and the great Bonaparte’ (\citealt[101]{Frank1850}; author's translation). In just this one Lingua Franca sentence, multiple lexifiers are represented: \textit{meschino} is from the \ili{Arabic}, \textit{miskīn}, and there is \ili{Spanish} in \textit{carrouba} ‘penny’, \textit{donar} ‘give’, and \textit{amor} ‘love’, with the \ili{Italian} Catholic reference of \textit{Santissima Trinità} ‘most holy Trinity’, and \ili{French} \textit{Bonaparte}. The latter would have still been Emperor and possibly at the height of his power. Bonaparte is qualified by \textit{gran} ‘great’, from the \ili{Italian} or even \ili{Venetian}. It suggests how cosmopolitan, multicultural and multilingual Tunis and its population had become that a beggar should speak this way. Even Frank was struck by the incongruity of the beggar’s words: ``\textit{sa supplique en ces termes, bien étranges dans la bouche d’un Musulman}'', ‘his petition in these terms, very odd in the mouth of a Muslim’ (\citealt[101]{Frank1850}; author's translation).
	
	Lingua Franca’s demise dates from 1830, as a consequence of the outlawing of slavery and the start of the French colonization of North Africa. Lingua Franca became known alternatively as Sabir (HSA 1882: Letter 6-7473), with a later incarnation which Schuchardt dubbed Judeo-Sabir (\citealt[87]{Schuchardt1909}). Residual elements seem to persist, however, in other contemporary jargons and languages. The \isi{pidgin} spoken in Algeria, Pataouète, while largely lexified by \ili{French} and \ili{Arabic}, also features a significant number of words that are identified by \citet{Lanly1962} as Lingua Franca in origin. Duclos’ (\citeyear{Duclos1992}) Pataouète dictionary enumerates at least thirty words whose etymology she specifies as Lingua Franca. These include \textit{baroufa} ‘quarrel', \textit{fantasia} ‘pride, delusion’, \textit{mercanti} ‘merchant’, and \textit{rabia} ‘rage’.
	
 \section{Contact with Arabic}
	
	As mentioned above, a substantial proportion of Lingua Franca speakers appear to have had \ili{Arabic} as their first language. Inevitably, there would have been \isi{transfer} when they spoke Lingua Franca, although, given the shared history of \ili{Arabic} and European cultures in \isi{Sicily}, Spain and other parts of the Mediterranean, it is perhaps hard to state unequivocally whether lexical influences \isi{stem} from the contact of \ili{Arabic} with Lingua Franca directly, or its earlier contact with \ili{Romance} languages. \citet{Pellegrini1972} identified the many \ili{Arabic} \isi{loanwords} integrated into \ili{Italian}, particularly in the realms of trade, conflict and exploration. A number of these are included in the \textit{Dictionnaire} (\citeyear{Anonymous1830}), including \textit{magazino} ‘shop’ from the \ili{Arabic}, \textit{maḫāzīn} ‘storage facility’, and \textit{fondaco} ‘trading post’ from the \ili{Arabic} \textit{funduq} ‘hotel, inn’. Both would already have been in use in \ili{Italian}, thus complicating further an etymological study of Lingua Franca’s lexicon.
	

\section{Contact-induced changes in Lingua Franca}
	
	Contact-induced changes in Lingua Franca with regard to \ili{Arabic} are relatively limited, evident predominantly in its lexicon but also, to some extent, in its phonology. It is perhaps even an overstatement to consider \ili{Arabic}’s influence as contact-induced change; rather it might be viewed simply as an additional \isi{lexifier}.
	
\subsection{Phonology}
	
	The \isi{relative} lack of written record and potential unreliability of the sources’ excerpts of Lingua Franca makes the identification of a definitive phonemic inventory both difficult, and at times, inconclusive. Overall, Lingua Franca follows the phonology of \ili{Romance} languages, predominantly Tuscan \ili{Italian}, though with elements of \ili{Venetian} and \ili{Spanish}. \ili{Venetian} influence is also evident in the Lingua Franca tendency to omit final vowels following sonorants /l/, /n/ and /r/, as in \textit{colazion} instead of \textit{colazione} ‘breakfast’. Both \ili{Venetian} and Lingua Franca exhibit examples of \isi{degemination} (e.g. \textit{tuto} `all' vs. Tuscan \textit{tutto}) and voicing of intervocalic stops – \textit{segredo} `secret' rather than \textit{segreto} \citep{Ursini2011}. The voicing of intervocalic *t is also consistent with \ili{Spanish}, which appears also to have had an influence on elements of Lingua Franca phonology. An example from the \textit{Dictionnaire} (\citeyear[63]{Anonymous1830}) that illustrates both the plosive voicing and the final vowel omission is \textit{padron} ‘master’, an epithet that recurs throughout the corpus of attestations. However, in terms of the language’s vocalic system, \ili{Arabic} appears to exert some influence. \citet{Cifoletti2004} suggests that \ili{Arabic} influence on the realization of vocalic elements can be seen in the \textit{Dictionnaire}: \textit{bonou} from the \ili{Italian} \textit{buono} ‘good’ evidences a \isi{simplification} of the \isi{diphthong} \textit{uo}. \citet[444]{Bergareche1993} confirms this \isi{simplification}; for example \ili{Italian} \textit{uovo} ‘egg’, \textit{duole} ‘hurt’, \textit{buono} ‘good’ are reduced in their Lingua Franca counterparts: \textit{obo, dole, bono}. 
	
	There is some evidence in the \textit{Dictionnaire} of a reduction in the number of qualities for short vowels in Lingua Franca from a typical five-vowel \ili{Romance} system to the more impoverished systems found in North African \ili{Arabic} varieties, seen, for example in the frequent realization of final /e/ as 〈a〉, as in \textit{scoura} from \textit{scure} ‘axe’ or \textit{gratzia} from \textit{grazie} ‘thanks’, or 〈i〉, as in \textit{sempri} from \textit{sempre} ‘always’ or \textit{grandi} from \textit{grande} ‘big’. \citet[444]{Bergareche1993} reinforces this, citing the Lingua Franca words \textit{mouchou} ‘much, many’, \textit{poudir}, ‘to be able’, and \textit{inglis} ‘English’, with their \isi{roots} in \ili{Spanish} (\textit{mucho}, \textit{poder}, \textit{inglés}) as evidence of a reduction in vowel qualities as a result of contact with \ili{Arabic}. 
	
	Unlike most (non-sonorant-final) words in Lingua Franca that have a typical \ili{Romance} vowel ending, \ili{Arabic}-derived words generally retain their consonant ending, as in \textit{rouss} from \textit{ruzz} ‘rice’ and \textit{maboul} from \textit{mahbūl} ‘stupid’ \citep[38]{Cifoletti2004}. Non-\ili{Romance} influence on Lingua Franca is also evidenced in the regular substitution of /b/ for /p/, which is lacking in the phonemic inventory of most \ili{Arabic} varieties. The \textit{Dictionnaire} features a number of replacements of this type \citep[38]{Cifoletti2004}, as in \textit{esbinac} ‘spinach’ and \textit{nabolitan} ‘Neapolitan’.  Minervini (\citeyear[257–60]{Minervini1996}) analyses the speech of the eponymous heroine of Giancarli’s (\citeyear{Giancarli1545}) \textit{La Zingana}, and comments on the frequent substitution of /b/ for /p/ and /v/, offering examples such as \textit{cattiba} (\textit{cattiva} ‘nasty’ in \ili{Italian}), \textit{bericola (pericolo} ‘danger’ in \ili{Italian}), the native \ili{Arabic} of the character allegedly influencing her pronunciation. Given, however, that these examples come from a work of fiction, they do not provide conclusive evidence of influence. 
	
	Perhaps given that Lingua Franca as attested is replete with abbreviation, ellipsis and omissions, it predictably features examples of aphaeresis. For example, many \ili{Romance}-derived items beginning with a syllable that resembles the \ili{Arabic} \isi{definite} \isi{article} see this omitted in Lingua Franca. Examples include \textit{bassiador} for \textit{ambasciatore} ‘ambassador’, \textit{bastantza} for \textit{abbastanza} ‘enough’, and \textit{rigar} for \textit{irrigare} ‘to water’. As with many other linguistic features, however, the similarity between Lingua Franca and \ili{Venetian} dialect must be considered, as some of these words exist in a similarly abbreviated form in \ili{Venetian} \citep{Schuchardt1909}.
	
\subsection{Lexicon}
	
	From my quantitative analysis of the material in the \textit{Dictionnaire} and other available sources, it is apparent that there were very few \ili{Arabic} lexemes in Lingua Franca’s lexicon. Of the more than 2,100 entries in the \textit{Dictionnaire}, 32 are of \ili{Arabic} origin, and of the 176 additional Lingua Franca lexemes identified in the corpus of attestations, only nine have an \ili{Arabic} etymology. However, a number of the individual items are regularly repeated in the corpus, and, as such, \ili{Arabic} appears a more influential \isi{lexifier} on a token than on a type basis.
	
	\ili{Romance}/non-\ili{Romance} (often \ili{Arabic}) doublets feature particularly in terms of place names, and officialdom within the Regencies. The port of Tunis was known by its \ili{French}, \ili{Italian} and \ili{Arabic} names, seemingly interchangeably: \textit{La Goulette, La Goletta, and Wādi l-Ḥalq} ‘the gullet’. In his manual for \isi{future} consuls, the outgoing English consul of Tripoli, \iai{Knecht}, enumerates the hierarchies within the Pasha’s household and city administration. Many of these involve a combination of \ili{Arabic} or \ili{Turkish} and \ili{Italian}, or perhaps Lingua Franca. Key positions include a \textit{hasnadawr Grande ed un hasndawr Piccolo} – ‘a senior treasurer and a junior treasurer (\citealt[97]{Pennell1982}; author's translation).  \textit{Hasnadawr} comes from the \ili{Ottoman} {Turkish} \textit{hazinedar} or \textit{haznadar} (from \ili{Arabic} \textit{ḫāzin ad-dār}) ‘Lord treasurer of the household’ \citep{Gilson1987}.
	
	Another example is \textit{Kecchia Grande} and \textit{Kecchia Piccolo} ‘chief administrator and assistant administrator’ \citep[104]{Pennell1982}. In the letters written by members of the household of Richard Tully, British Consul to Tripoli at the close of the eighteenth century, there is a reference to the ``Great Chiah and the Little Chiah'' \citep[70]{Tully1819}, surely an anglicization of the title. \textit{Kecchia} appears to derive from the \ili{Tunisian} \ili{Arabic} \textit{kāhiya} ‘chief officer of an administrative district’ – \textit{kecchia} is an italianised (or, again, possibly Lingua Franca-influenced) \isi{orthography} and pronunciation. Similarly, \textit{sotto rais} (from the \ili{Italian} and \ili{Arabic} literally meaning ‘under captain’) denoted the second in command of the harbour (\citealt[97, 100]{Pennell1982}). The commander is referred to separately as the \textit{rays de la marina} ‘chief of the port’ \citep[92]{Pennell1982}. Again, one finds the combination of \ili{Arabic} and \ili{Italian}. (\textit{Rays} is spelt in two different ways,\footnote{The standard \ili{Arabic} form is \textit{raʔīs}.} which highlights how, pre-standardisation of European languages, \isi{orthography} was erratic, even within a single document.) Another example is a proverb regarding the eradication of plague by the Day of St. John. Two variations on the proverb are cited – by Poiret (\citeyear{Poiret1802}) and Rehbinder (\citeyear{Rehbinder1800}): 
	
	\ea
	(\citealt{Poiret1802})\\
	\textit{Saint Jean venir, Gandouf andar} 
	\glt ‘[The day of] St. John comes, the plague leaves.’
	\ex 
	(\citealt{Rehbinder1800})\\
	\textit{Saint Jean venir, buba andar} 
	\glt ‘[The day of] St. John comes, the plague leaves.’ 
	\z
	
	Selbach (\citeyear[44]{Selbach2008}) remarks on how such varied nomenclature in Lingua Franca ``allowed for much room to manoeuver, and for speakers to mark their religious, political and cultural \isi{identity}''. \textit{Buba} ‘plague’ appears to come originally from the \ili{Greek}, \textit{βουβών} (\textit{boubōn}) ‘groin’, suggesting yet another potential lexifying influence on Lingua Franca, while \textit{Gandouf} plausibly derives from \ili{Arabic} \textit{ɣunduba {\textasciitilde} ɣundūb} ‘swollen tonsils’ (\citealt[72]{Schuchardt1909}; cf. \citealt[45]{Selbach2008}). 
	
	This example raises the issue of words already common to \ili{Arabic} and \ili{Romance} languages, since contact between them, as discussed above, had been prolonged and extensive. Similarly, the \ili{French} (and possibly \ili{Italian} or even Lingua Franca) word \textit{avanie} ‘fine, insult, affront’ occurs often in the corpus of attestations (e.g. \citealt{Pananti1841} and \citealt{Grandchamp1920}). Grandchamp (\citeyear[xiii]{Grandchamp1920}) defines it thus: 
	
	\begin{quote}
		les avanies étaient des sommes d'argent que les pachas réclamaient aux marchands des échelles sous les prétextes les plus divers, prétextes la plupart du temps injustes, parfois extrêmement bizarres
	\end{quote}
	
	\begin{quote}
		‘the fines were sums of money the Pashas demanded of the Levant merchants on various pretexts, pretexts that were for the most part unfair, and at times extremely strange’ (author's translation).
	\end{quote}
	
	Although this word would appear to be derived from \ili{French}, or at least a \ili{Romance} language, given that it was the creation of Ottoman elites, it seems more likely that its origins are \ili{Turkish}. This is confirmed by \citet{Pihan1847} who suggests that it actually originally derives from the \ili{Arabic} \textit{hawān} ‘contempt’,\footnote{This etymology is also favoured by \textit{Le Trésor de la langue francaise informatisé} \citep{Dendien1994}.} but who also states \citep[46]{Pihan1847}:
	
	\begin{quote}
		se dit également des impôts énormes que les Turcs font peser sur les Chrétiens dans le but de les humilier
	\end{quote}
	
	\begin{quote}
		‘it applies equally to the enormous taxes the Turks impose on \isi{Christians} with the goal of humiliating them’ (author's translation).
	\end{quote}
	
	Additionally, there are words that appear to have etymologies in multiple languages that are rarely translated, at least by English sources such as \citet{Tully1819}. Two terms with similar meanings, \textit{firman} ‘pass, decree’ and \textit{teschera} ‘pass, edict’, both issued by Ottoman or \ili{Arabic} rulers, also bear remarkable similarity to \ili{Italian} words with comparable meanings. \textit{Firman} is from \ili{Arabic}: \textit{firmān {\textasciitilde} faramān}, though originally \ili{Persian}, and would have come into \ili{Arabic} through \ili{Ottoman} {Turkish}, once again reinforcing how the languages spoken in the region were far from discrete entities. However, \textit{firmare} in \ili{Italian} means `to sign', and the Lingua Franca translation of the \ili{French} \textit{seing} ‘signature’ and \textit{signature} ‘signature’ in the \textit{Dictionnaire} is \textit{firmar}. A decree or pass (allowing free passage or safe conduct) would necessarily require an official signature. \textit{Teschera} ‘pass, edict’ might appear to come from the \ili{Italian} \textit{tessera} ‘pass, ticket’ but there is also the \ili{Arabic} word \textit{taðkira {\textasciitilde} taðkara {\textasciitilde} tazkira {\textasciitilde} tazakara} (all variant realizations of the same item), which also means ‘permit’ or ‘ticket’. Both words seem integral to Barbary life, and are not translated. \citet[258]{Tully1819} writes: 
	
	\begin{quote}
		It is still affirmed that he has a teskerra, or firman, with him for this unfortunate Bashaw. A teskerra is a written order from the Grand Signior, and is held so sacred that every Musulman who receives it must obey its mandate, even to death.
	\end{quote}
	
	Perhaps the most iconic word in Lingua Franca is \textit{fantasia}. It is mentioned by multiple sources spanning more than two centuries (e.g. \citealt{Haedo1612,Broughton1839}). Although it appears to be a \ili{Romance} word, Schuchardt (\citeyear[71]{Schuchardt1909}) points out that it is used in the \ili{Arabic} sense of pride, arrogance, as in, for example, \ili{Egyptian Arabic} \textit{itfanṭaz} ‘to give oneself airs’.
	
	Collections in the UK National Archives provide some limited evidence of borrowings from Lingua Franca (as opposed to other \ili{Romance} languages) into \ili{Arabic}. Hopkins' (\citeyear{Hopkins1982}) research in these archives focuses on two sets of English (and later British) state papers relating to the Barbary regencies and Morocco, including correspondence in \ili{Arabic}, from the late sixteenth to late eighteenth centuries. 
	
	Hopkins adds a glossary to his translations that demonstrates the extent to which the \ili{Arabic} letters from the Barbary States disproportionately feature loans from \ili{Romance} languages. Hopkins (\citeyear[x]{Hopkins1982}) comments that “[f]oreign words are very common and seem to be used quite unselfconsciously”, and many of those words Hopkins isolates are also listed in the \textit{Dictionnaire} (\citeyear{Anonymous1830}) as Lingua Franca words. Three such items -- \textit{justisiya} ‘justice’, \textit{markānti} ‘merchant’ and \textit{zabantut} (\textit{sbendout} in Lingua Franca, presumably from the \ili{Italian} \textit{bandito}) ‘pirate’ -- occur in a single letter of unspecified provenance but written to the King of England in 1730 by a man claiming to be an Algerine trader in Tripoli (TNA: SP 71/23/51). The incidence of three non-\ili{Arabic}, \ili{Romance} words, is noteworthy and it seems plausible that these were Lingua Franca terms in such common usage that they would be borrowed as a native language alternative, an example of \textit{mot juste} switching \citep[32]{Gardner-Chloros2009}.
	
\section{Conclusion}
	
	As demonstrated, the available corpus of writing in Lingua Franca – both documentary texts written by European visitors to the Barbary states and the dramatic works produced by contemporary authors – offers limited evidence of both lexicon and grammar. This makes description of Lingua Franca challenging, and, likewise, any concrete and substantiated analysis of its relationship with other, particularly non-European, languages.
	
	Nevertheless, this chapter has suggested how \ili{Arabic} and \ili{Romance} languages influenced the emergence of Lingua Franca, specifically in terms of its lexicon and phonology. Authors throughout the era of Lingua Franca’s existence, from \citet{Haedo1612} to \citet{Broughton1839} and \citet{Frank1850} reiterate that, despite its overwhelming \ili{Romance} base, Lingua Franca was spoken predominantly by the often \ili{Arabic}-speaking slavemasters and rulers of the Barbary states. The plurilingual character of the population of this region, both collectively and individually, \isi{compounds} an already unclear picture, however, as the fluidity of Barbary society led to European (and often \ili{Romance}-language speaking) corsairs and diplomats alike permeating its upper echelons (\citealt{Haedo1612}: 9; \citealt{Garcès2011}: 129).
	
	The lexical influence of \ili{Arabic} is most evident in the \ili{Romance}/\ili{Arabic} doublets used in official terms and place names. These are often compound terms, such as \textit{ra’ïs de la marina} ‘captain of the port’. \iai{Warrington}, English Consul to Tripoli in the late eighteenth century, uses an Anglicised version of the phrase, \textit{rays marina}, suggesting the ubiquity of such doublets (TNA: FO 161/9) Further evidence from the National Archives \citep{Hopkins1982} demonstrates that Lingua Franca words were borrowed in correspondence from \ili{Arabic}-speaking dignitaries in the Barbary Regencies to the English Secretary of State, showing evidence of Lingua Franca borrowings in the written as well as the oral domain. 
	
	In terms of phonology, there seems to be evidence offered by the \textit{Dictionnaire} (\citeyear{Anonymous1830}) and the observation of \citet{Haedo1612} of the influence of \ili{Arabic} on Lingua Franca. \citet[24]{Haedo1612} also stated, with regard to the ``Moors and Turks'' that ``\textit{no saben ellos variar los modos, tiempos y casos}'', ‘they don’t know about \isi{gender}, tenses and cases’ (author's translation). Given that Lingua Franca lacks for the most part any verbal \isi{inflection} and an absence of cases, this might be, as Haedo says, a result of contact, but it is typical of most pidgins and cannot be attributed solely to contact with \ili{Arabic}. Such lack of certainty applies more generally. \ili{Arabic} evidently exerted some influence on the evolution of Lingua Franca in North Africa, but not to the extent that it can straightforwardly be classified as contact-induced change.
	
\section*{Further reading}
	
	\citet{Nolan2018} gives a comprehensive English-language introduction to Lingua Franca.
	
	\citet{Corré2005} compiles a substantial number of texts featuring Lingua Franca, offering translations from the original European languages and a selection of articles analysing and contexualising Lingua Franca. He also provides a glossary, largely derived from the \textit{Dictionnaire} (\citeyear{Anonymous1830}) and Schuchardt’s seminal (\citeyear{Schuchardt1909}) work.
	
	\citet{Schuchardt1909} offers the earliest detailed historical and linguistic analysis of Lingua Franca. More recent in-depth texts include Dakhlia's (\citeyear{Dakhlia2008}) rather subjective book, Minervini's (\citeyear{Minervini1996}) comprehensive \isi{article} that focuses predominantly on \ili{Italian} and \ili{French} sources, and Cifoletti's (\citeyear{Cifoletti2004}) forensic biography and analysis of Lingua Franca, again based almost exlusively on \ili{Romance} language sources. 
	
	The key source authors are inevitably \citet{Haedo1612}, \citet{Broughton1839}, \citet{Pananti1841}, and \citet{Frank1850}.

\section*{Abbreviations}	

\begin{tabularx}{.5\textwidth}{@{}lQ@{}}
HSA & Hugo Schuchardt Archiv\\
TNA & The National Archive \\ 
\end{tabularx}%

	{\sloppy\printbibliography[heading=subbibliography,notkeyword=this]}

\end{document}
