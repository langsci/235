\documentclass[output=paper,nonflat]{langsci/langscibook} 
\author{Bettina Leitner\affiliation{University of Vienna}}
\title{Khuzestan Arabic}
\abstract{Khuzestan Arabic is an Arabic variety spoken in the southwestern Iranian province of Khuzestan. It has been in contact with (Modern) Persian since the arrival of Arab tribes in the region before the rise of Islam. Persian is the socio-politically dominant language in the modern state of Iran and has influenced the grammar of Khuzestan Arabic on different levels. The present article discusses phenomena of contact-induced change in Khuzestan Arabic and considers their limiting factors.
}
\IfFileExists{../localcommands.tex}{
  % add all extra packages you need to load to this file 
\usepackage{graphicx}
\usepackage{tabularx}
\usepackage{amsmath} 
\usepackage{multicol}
\usepackage{lipsum}
\usepackage[stable]{footmisc}
\usepackage{adforn}
%%%%%%%%%%%%%%%%%%%%%%%%%%%%%%%%%%%%%%%%%%%%%%%%%%%%
%%%                                              %%%
%%%           Examples                           %%%
%%%                                              %%%
%%%%%%%%%%%%%%%%%%%%%%%%%%%%%%%%%%%%%%%%%%%%%%%%%%%%
% remove the percentage signs in the following lines
% if your book makes use of linguistic examples
\usepackage{./langsci/styles/langsci-optional} 
\usepackage{./langsci/styles/langsci-lgr}
\usepackage{morewrites} 
%% if you want the source line of examples to be in italics, uncomment the following line
% \def\exfont{\it}

\usepackage{enumitem}
\newlist{furtherreading}{description}{1}
\setlist[furtherreading]{font=\normalfont,labelsep=\widthof{~},noitemsep,align=left,leftmargin=\parindent,labelindent=0pt,labelwidth=-\parindent}
\usepackage{phonetic}
\usepackage{chronosys,tabularx}
\usepackage{csquotes}
\usepackage[stable]{footmisc} 

\usepackage{langsci-bidi}
\usepackage{./langsci/styles/langsci-gb4e} 

  \makeatletter
\let\thetitle\@title
\let\theauthor\@author 
\makeatother

\newcommand{\togglepaper}[1][0]{ 
  \bibliography{../localbibliography}
  \papernote{\scriptsize\normalfont
    \theauthor.
    \thetitle. 
    To appear in: 
    Christopher Lucas and Stefano Manfredi (eds.),  
    Arabic and contact-induced language change
    Berlin: Language Science Press. [preliminary page numbering]
  }
  \pagenumbering{roman}
  \setcounter{chapter}{#1}
  \addtocounter{chapter}{-1}
}

\newfontfamily\Parsifont[Script=Arabic]{ScheherazadeRegOT_Jazm.ttf} 
\newcommand{\arabscript}[1]{\RL{\Parsifont #1}}
\newcommand{\textarabic}[1]{{\arabicfont #1}}

\newcommand{\textstylest}[1]{{\color{red}#1}}

\patchcmd{\mkbibindexname}{\ifdefvoid{#3}{}{\MakeCapital{#3}
}}{\ifdefvoid{#3}{}{#3 }}{}{\AtEndDocument{\typeout{mkbibindexname could
not be patched.}}}

%command for italic r with dot below with horizontal correction to put the dot in the prolongation of the vertical stroke
%for some reason, the dot is larger than expected, so we explicitly reduce the font size (to \small)
%for the time being, the font is set to an absolute value. To be more robust, a relative reduction would be better, but this might not be required right now
\newcommand{\R}{r\kern-.05ex{\small{̣}}\kern.05ex}


\DeclareLabeldate{%
    \field{date}
    \field{year}
    \field{eventdate}
    \field{origdate}
    \field{urldate}
    \field{pubstate}
    \literal{nodate}
}

\renewbibmacro*{addendum+pubstate}{% Thanks to https://tex.stackexchange.com/a/154367 for the idea
  \printfield{addendum}%
  \iffieldequalstr{labeldatesource}{pubstate}{}
  {\newunit\newblock\printfield{pubstate}}
}
 
  %% hyphenation points for line breaks
%% Normally, automatic hyphenation in LaTeX is very good
%% If a word is mis-hyphenated, add it to this file
%%
%% add information to TeX file before \begin{document} with:
%% %% hyphenation points for line breaks
%% Normally, automatic hyphenation in LaTeX is very good
%% If a word is mis-hyphenated, add it to this file
%%
%% add information to TeX file before \begin{document} with:
%% %% hyphenation points for line breaks
%% Normally, automatic hyphenation in LaTeX is very good
%% If a word is mis-hyphenated, add it to this file
%%
%% add information to TeX file before \begin{document} with:
%% \include{localhyphenation}
\hyphenation{
affri-ca-te
affri-ca-tes
com-ple-ments
homo-phon-ous
start-ed
Meso-potam-ian
morpho-phono-logic-al-ly
morpho-phon-em-ic-s
Palestin-ian
re-present-ed
Ki-nubi
ḥawār-iyy-ūn
archa-ic-ity
fuel-ed
de-velop-ment
pros-od-ic
Arab-ic
in-duced
phono-logy
possess-um
possess-ive-s
templ-ate
spec-ial
espec-ial-ly
nat-ive
pass-ive
clause-s
potent-ial-ly
Lusignan
commun-ity
tobacco
posi-tion
Cushit-ic
Middle
with-in
re-finit-iz-ation
langu-age-s
langu-age
diction-ary
glossary
govern-ment
eight
counter-part
nomin-al
equi-valent
deont-ic
ana-ly-sis
Malt-ese
un-fortun-ate-ly
scient-if-ic
Catalan
Occitan
ḥammāl
cross-linguist-ic-al-ly
predic-ate
major-ity
ignor-ance
chrono-logy
south-western
mention-ed
borrow-ed
neg-ative
de-termin-er
European
under-mine
detail
Oxford
Socotra
numer-ous
spoken
villages
nomad-ic
Khuze-stan
Arama-ic
Persian
Ottoman
Ottomans
Azeri
rur-al
bi-lingual-ism
borrow-ing
prestig-ious
dia-lects
dia-lect
allo-phone
allo-phones
poss-ible
parallel
parallels
pattern
article
common-ly
respect-ive-ly
sem-antic
Moroccan
Martine
Harrassowitz
Grammatic-al-ization
grammatic-al-ization
Afro-asiatica
Afro-asiatic
continu-ation
Semit-istik
varieties
mono-phthong
mono-phthong-ized
col-loquial
pro-duct
document-ary
ex-ample-s
ex-ample
termin-ate
element-s
Aramaeo-grams
Centr-al
idioms
Arab-ic
Dadan-it-ic
sub-ordin-ator
Thamud-ic
difficult
common-ly
Revue
Bovingdon
under
century
attach
attached
bundle
graph-em-ic
graph-emes
cicada
contrast-ive
Corriente
Andalusi
Kossmann
morpho-logic-al
inter-action
dia-chroniques
islámica
occid-ent-al-ismo
dialecto-logie
Reichert
coloni-al
Milton
diphthong-al
linguist-ic
linguist-ics
affairs
differ-ent
phonetic-ally
kilo-metres
stabil-ization
develop-ments
in-vestig-ation
Jordan-ian
notice-able
level-ed
migrants
con-dition-al
certain-ly
general-ly
especial-ly
af-fric-ation
Jordan
counter-parts
com-plication
consider-ably
inter-dent-al
com-mun-ity
inter-locutors
com-pon-ent
region-al
socio-historical
society
simul-taneous
phon-em-ic
roman-ization
Classic-al
funeral
Kurmanji
pharyn-geal-ization
vocab-ulary
phon-et-ic
con-sonant
con-sonants
special-ized
latter
latters
in-itial
ident-ic-al
cor-relate
geo-graphic-al-ly
Öpengin
Kurd-ish
in-digen-ous
sunbul
Christ-ian
Christ-ians
sekin-în
fatala
in-tegration
dia-lect-al
Matras
morpho-logy
in-tens-ive
con-figur-ation
im-port-ant
com-plement
ḥaddād
e-merg-ence
Benjmamins
struct-ure
em-pir-ic-al
Orient-studien
Anatolia
American
vari-ation
Jastrow
Geoffrey
Yarshater
Ashtiany
Edmund
Mahnaz
En-cyclo-pædia
En-cyclo-paedia
En-cyclo-pedia
Leiden
dia-spora
soph-is-ic-ated
Sasan-ian
every-day
domin-ance
Con-stitu-tion-al
religi-ous
sever-al
Manfredi
re-lev-ance
re-cipi-ent
pro-duct-iv-ity
turtle
Morocco
ferman
Maghreb-ian
algérien
stand-ard
systems
Nicolaï
Mouton
mauritani-en
Gotho-burg-ensis
socio-linguist-ique
plur-al
archiv-al
Arab-ian
drop-ped
dihāt
de-velop-ed
ṣuḥbat
kitāba
kitābat
com-mercial
eight-eenth
region
Senegal
mechan-ics
Maur-itan-ia
Ḥassān-iyya
circum-cision
cor-relation
labio-velar-ization
vowel
vowels
cert-ain
īggīw
series
in-tegrates
dur-ative
inter-dent-als
gen-itive
Tuareg
tălămut
talawmāyət
part-icular
part-icular-ly
con-diment
vill-age
bord-er
polit-ical
Wiesbaden
Uni-vers-idad
Geuthner
typo-logie
Maur-itanie
nomades
Maur-itan-ian
dia-lecto-logy
Sahar-iennes
Uni-vers-ity
de-scend-ants
NENA-speak-ing
speak-ing
origin-al
re-captured
in-habit-ants
ethnic
minor-it-ies
drama-tic
local
long-stand-ing
regions
Nineveh
settle-ments
Ṣəndor
Mandate
sub-stitut-ing
ortho-graphy
re-fer-enced
origin-ate
twenti-eth
typ-ic-al-ly
Hobrack
never-the-less
character-ist-ics
character-ist-ic
masc-uline
coffee
ex-clus-ive-ly
verb-al
re-ana-ly-se-d
simil-ar-ities
de-riv-ation
im-pera-tive
part-iciple
dis-ambi-gu-ation
dis-ambi-gu-a-ing
phen-omen-on
phen-omen-a
traktar
com-mun-ity
com-mun-ities
dis-prefer-red
ex-plan-ation
con-struction
wide-spread
us-ual-ly
region-al
Bulut
con-sider-ation
afro-asia-tici
Franco-Angeli
Phono-logie
Volks-kundliche
dia-lectes
dia-lecte
select-ed
dis-appear-ance
media
under-stand-able
public-ation
second-ary
e-ject-ive
re-volu-tion
re-strict-ive
Gasparini
mount-ain
mount-ains
yellow
label-ing
trad-ition-al-ly
currently
dia-chronic
}
\hyphenation{
affri-ca-te
affri-ca-tes
com-ple-ments
homo-phon-ous
start-ed
Meso-potam-ian
morpho-phono-logic-al-ly
morpho-phon-em-ic-s
Palestin-ian
re-present-ed
Ki-nubi
ḥawār-iyy-ūn
archa-ic-ity
fuel-ed
de-velop-ment
pros-od-ic
Arab-ic
in-duced
phono-logy
possess-um
possess-ive-s
templ-ate
spec-ial
espec-ial-ly
nat-ive
pass-ive
clause-s
potent-ial-ly
Lusignan
commun-ity
tobacco
posi-tion
Cushit-ic
Middle
with-in
re-finit-iz-ation
langu-age-s
langu-age
diction-ary
glossary
govern-ment
eight
counter-part
nomin-al
equi-valent
deont-ic
ana-ly-sis
Malt-ese
un-fortun-ate-ly
scient-if-ic
Catalan
Occitan
ḥammāl
cross-linguist-ic-al-ly
predic-ate
major-ity
ignor-ance
chrono-logy
south-western
mention-ed
borrow-ed
neg-ative
de-termin-er
European
under-mine
detail
Oxford
Socotra
numer-ous
spoken
villages
nomad-ic
Khuze-stan
Arama-ic
Persian
Ottoman
Ottomans
Azeri
rur-al
bi-lingual-ism
borrow-ing
prestig-ious
dia-lects
dia-lect
allo-phone
allo-phones
poss-ible
parallel
parallels
pattern
article
common-ly
respect-ive-ly
sem-antic
Moroccan
Martine
Harrassowitz
Grammatic-al-ization
grammatic-al-ization
Afro-asiatica
Afro-asiatic
continu-ation
Semit-istik
varieties
mono-phthong
mono-phthong-ized
col-loquial
pro-duct
document-ary
ex-ample-s
ex-ample
termin-ate
element-s
Aramaeo-grams
Centr-al
idioms
Arab-ic
Dadan-it-ic
sub-ordin-ator
Thamud-ic
difficult
common-ly
Revue
Bovingdon
under
century
attach
attached
bundle
graph-em-ic
graph-emes
cicada
contrast-ive
Corriente
Andalusi
Kossmann
morpho-logic-al
inter-action
dia-chroniques
islámica
occid-ent-al-ismo
dialecto-logie
Reichert
coloni-al
Milton
diphthong-al
linguist-ic
linguist-ics
affairs
differ-ent
phonetic-ally
kilo-metres
stabil-ization
develop-ments
in-vestig-ation
Jordan-ian
notice-able
level-ed
migrants
con-dition-al
certain-ly
general-ly
especial-ly
af-fric-ation
Jordan
counter-parts
com-plication
consider-ably
inter-dent-al
com-mun-ity
inter-locutors
com-pon-ent
region-al
socio-historical
society
simul-taneous
phon-em-ic
roman-ization
Classic-al
funeral
Kurmanji
pharyn-geal-ization
vocab-ulary
phon-et-ic
con-sonant
con-sonants
special-ized
latter
latters
in-itial
ident-ic-al
cor-relate
geo-graphic-al-ly
Öpengin
Kurd-ish
in-digen-ous
sunbul
Christ-ian
Christ-ians
sekin-în
fatala
in-tegration
dia-lect-al
Matras
morpho-logy
in-tens-ive
con-figur-ation
im-port-ant
com-plement
ḥaddād
e-merg-ence
Benjmamins
struct-ure
em-pir-ic-al
Orient-studien
Anatolia
American
vari-ation
Jastrow
Geoffrey
Yarshater
Ashtiany
Edmund
Mahnaz
En-cyclo-pædia
En-cyclo-paedia
En-cyclo-pedia
Leiden
dia-spora
soph-is-ic-ated
Sasan-ian
every-day
domin-ance
Con-stitu-tion-al
religi-ous
sever-al
Manfredi
re-lev-ance
re-cipi-ent
pro-duct-iv-ity
turtle
Morocco
ferman
Maghreb-ian
algérien
stand-ard
systems
Nicolaï
Mouton
mauritani-en
Gotho-burg-ensis
socio-linguist-ique
plur-al
archiv-al
Arab-ian
drop-ped
dihāt
de-velop-ed
ṣuḥbat
kitāba
kitābat
com-mercial
eight-eenth
region
Senegal
mechan-ics
Maur-itan-ia
Ḥassān-iyya
circum-cision
cor-relation
labio-velar-ization
vowel
vowels
cert-ain
īggīw
series
in-tegrates
dur-ative
inter-dent-als
gen-itive
Tuareg
tălămut
talawmāyət
part-icular
part-icular-ly
con-diment
vill-age
bord-er
polit-ical
Wiesbaden
Uni-vers-idad
Geuthner
typo-logie
Maur-itanie
nomades
Maur-itan-ian
dia-lecto-logy
Sahar-iennes
Uni-vers-ity
de-scend-ants
NENA-speak-ing
speak-ing
origin-al
re-captured
in-habit-ants
ethnic
minor-it-ies
drama-tic
local
long-stand-ing
regions
Nineveh
settle-ments
Ṣəndor
Mandate
sub-stitut-ing
ortho-graphy
re-fer-enced
origin-ate
twenti-eth
typ-ic-al-ly
Hobrack
never-the-less
character-ist-ics
character-ist-ic
masc-uline
coffee
ex-clus-ive-ly
verb-al
re-ana-ly-se-d
simil-ar-ities
de-riv-ation
im-pera-tive
part-iciple
dis-ambi-gu-ation
dis-ambi-gu-a-ing
phen-omen-on
phen-omen-a
traktar
com-mun-ity
com-mun-ities
dis-prefer-red
ex-plan-ation
con-struction
wide-spread
us-ual-ly
region-al
Bulut
con-sider-ation
afro-asia-tici
Franco-Angeli
Phono-logie
Volks-kundliche
dia-lectes
dia-lecte
select-ed
dis-appear-ance
media
under-stand-able
public-ation
second-ary
e-ject-ive
re-volu-tion
re-strict-ive
Gasparini
mount-ain
mount-ains
yellow
label-ing
trad-ition-al-ly
currently
dia-chronic
}
\hyphenation{
affri-ca-te
affri-ca-tes
com-ple-ments
homo-phon-ous
start-ed
Meso-potam-ian
morpho-phono-logic-al-ly
morpho-phon-em-ic-s
Palestin-ian
re-present-ed
Ki-nubi
ḥawār-iyy-ūn
archa-ic-ity
fuel-ed
de-velop-ment
pros-od-ic
Arab-ic
in-duced
phono-logy
possess-um
possess-ive-s
templ-ate
spec-ial
espec-ial-ly
nat-ive
pass-ive
clause-s
potent-ial-ly
Lusignan
commun-ity
tobacco
posi-tion
Cushit-ic
Middle
with-in
re-finit-iz-ation
langu-age-s
langu-age
diction-ary
glossary
govern-ment
eight
counter-part
nomin-al
equi-valent
deont-ic
ana-ly-sis
Malt-ese
un-fortun-ate-ly
scient-if-ic
Catalan
Occitan
ḥammāl
cross-linguist-ic-al-ly
predic-ate
major-ity
ignor-ance
chrono-logy
south-western
mention-ed
borrow-ed
neg-ative
de-termin-er
European
under-mine
detail
Oxford
Socotra
numer-ous
spoken
villages
nomad-ic
Khuze-stan
Arama-ic
Persian
Ottoman
Ottomans
Azeri
rur-al
bi-lingual-ism
borrow-ing
prestig-ious
dia-lects
dia-lect
allo-phone
allo-phones
poss-ible
parallel
parallels
pattern
article
common-ly
respect-ive-ly
sem-antic
Moroccan
Martine
Harrassowitz
Grammatic-al-ization
grammatic-al-ization
Afro-asiatica
Afro-asiatic
continu-ation
Semit-istik
varieties
mono-phthong
mono-phthong-ized
col-loquial
pro-duct
document-ary
ex-ample-s
ex-ample
termin-ate
element-s
Aramaeo-grams
Centr-al
idioms
Arab-ic
Dadan-it-ic
sub-ordin-ator
Thamud-ic
difficult
common-ly
Revue
Bovingdon
under
century
attach
attached
bundle
graph-em-ic
graph-emes
cicada
contrast-ive
Corriente
Andalusi
Kossmann
morpho-logic-al
inter-action
dia-chroniques
islámica
occid-ent-al-ismo
dialecto-logie
Reichert
coloni-al
Milton
diphthong-al
linguist-ic
linguist-ics
affairs
differ-ent
phonetic-ally
kilo-metres
stabil-ization
develop-ments
in-vestig-ation
Jordan-ian
notice-able
level-ed
migrants
con-dition-al
certain-ly
general-ly
especial-ly
af-fric-ation
Jordan
counter-parts
com-plication
consider-ably
inter-dent-al
com-mun-ity
inter-locutors
com-pon-ent
region-al
socio-historical
society
simul-taneous
phon-em-ic
roman-ization
Classic-al
funeral
Kurmanji
pharyn-geal-ization
vocab-ulary
phon-et-ic
con-sonant
con-sonants
special-ized
latter
latters
in-itial
ident-ic-al
cor-relate
geo-graphic-al-ly
Öpengin
Kurd-ish
in-digen-ous
sunbul
Christ-ian
Christ-ians
sekin-în
fatala
in-tegration
dia-lect-al
Matras
morpho-logy
in-tens-ive
con-figur-ation
im-port-ant
com-plement
ḥaddād
e-merg-ence
Benjmamins
struct-ure
em-pir-ic-al
Orient-studien
Anatolia
American
vari-ation
Jastrow
Geoffrey
Yarshater
Ashtiany
Edmund
Mahnaz
En-cyclo-pædia
En-cyclo-paedia
En-cyclo-pedia
Leiden
dia-spora
soph-is-ic-ated
Sasan-ian
every-day
domin-ance
Con-stitu-tion-al
religi-ous
sever-al
Manfredi
re-lev-ance
re-cipi-ent
pro-duct-iv-ity
turtle
Morocco
ferman
Maghreb-ian
algérien
stand-ard
systems
Nicolaï
Mouton
mauritani-en
Gotho-burg-ensis
socio-linguist-ique
plur-al
archiv-al
Arab-ian
drop-ped
dihāt
de-velop-ed
ṣuḥbat
kitāba
kitābat
com-mercial
eight-eenth
region
Senegal
mechan-ics
Maur-itan-ia
Ḥassān-iyya
circum-cision
cor-relation
labio-velar-ization
vowel
vowels
cert-ain
īggīw
series
in-tegrates
dur-ative
inter-dent-als
gen-itive
Tuareg
tălămut
talawmāyət
part-icular
part-icular-ly
con-diment
vill-age
bord-er
polit-ical
Wiesbaden
Uni-vers-idad
Geuthner
typo-logie
Maur-itanie
nomades
Maur-itan-ian
dia-lecto-logy
Sahar-iennes
Uni-vers-ity
de-scend-ants
NENA-speak-ing
speak-ing
origin-al
re-captured
in-habit-ants
ethnic
minor-it-ies
drama-tic
local
long-stand-ing
regions
Nineveh
settle-ments
Ṣəndor
Mandate
sub-stitut-ing
ortho-graphy
re-fer-enced
origin-ate
twenti-eth
typ-ic-al-ly
Hobrack
never-the-less
character-ist-ics
character-ist-ic
masc-uline
coffee
ex-clus-ive-ly
verb-al
re-ana-ly-se-d
simil-ar-ities
de-riv-ation
im-pera-tive
part-iciple
dis-ambi-gu-ation
dis-ambi-gu-a-ing
phen-omen-on
phen-omen-a
traktar
com-mun-ity
com-mun-ities
dis-prefer-red
ex-plan-ation
con-struction
wide-spread
us-ual-ly
region-al
Bulut
con-sider-ation
afro-asia-tici
Franco-Angeli
Phono-logie
Volks-kundliche
dia-lectes
dia-lecte
select-ed
dis-appear-ance
media
under-stand-able
public-ation
second-ary
e-ject-ive
re-volu-tion
re-strict-ive
Gasparini
mount-ain
mount-ains
yellow
label-ing
trad-ition-al-ly
currently
dia-chronic
} 
  \togglepaper[1]%%chapternumber
}{}

\begin{document}
\maketitle  

\section{Current state and historical development} 
\subsection{Historical development}
Arab settlement in Iran preceded the Arab destruction of the Sasanian empire with the rise of Islam. Various tribes, such as the Banū Tamīm, had settled in \ili{Khuzestan} prior to the arrival of the Arab Muslim armies \citep[211]{Daniel1986}. In the centuries after the spread of Islam in the region, large groups of nomads from the Ḥanīfa, Tamīm, ʕAbd-al-Qays, and other tribes crossed the \ili{Persian} \ili{Gulf} and occupied some of the territories of southwestern Iran \citep[215]{Oberling1986}. The Kaʕb, still an important tribe in the area,\footnote{Cf. Oberling (\citeyear[218]{Oberling1986}) for an overview of the Arab tribes in \ili{Khuzestan}.} settled there at the end of the sixteenth century \citep[216]{Oberling1986}. During the succeeding centuries many more tribes moved from southern Iraq into \ili{Khuzestan}. This has led to a considerable increase of \ili{Arabic} speakers in the region, which until 1925 was called Arabistan (see \citealt{Gazsi2011}: 1020; Gazsi, this volume).\ia{Gazsi, Dénes@Gazsi, Dénes} Today \ili{Khuzestan} is one of the 31 provinces of the Islamic Republic of Iran, situated in the southwest, at the border with Iraq. 

There has been considerable movement to and from Iraq, to Kuwait, Bahrain, and Syria, and from villages into towns. Many of these migrations were a consequence of the Iran–Iraq war (1980–1988), but some were due to socio-economic reasons. The settlement of Persians in the region over the past decades \citep[1020]{Gazsi2011} is another important factor in its demographic history. From the early twentieth century on, \ili{Khuzestan} has attracted international, especially \ili{British}, interest because of its oil resources. 

\subsection{Current situation of Arabs in Khuzestan}

In\isi{formation} about the exact number of \ili{Arabic}-speaking people in Iran, and in \ili{Khuzestan} in particular, is hard to find. Estimates in the 1960s of the \ili{Arabic}-speaking population in Iran ranged from 200,000 to 650,000 \citep[216]{Oberling1986}. Today it is estimated that around 2 to 3 million Arabs live in \ili{Khuzestan} (\citealt{MatrasShabibi2007}: 137; \citealt{Gazsi2011}: 1020). 

Many Arabs and Persians living in \ili{Khuzestan} work in the sugar cane or oil industries, but few of the former hold white-collar or managerial positions (\citealt{DePlanhol1986}: 55–56). This is one of the reasons why many Arabs in \ili{Khuzestan} feel strongly disadvantaged in society and politics in comparison to their \ili{Persian} neighbours.\footnote{The most common \ili{Khuzestan} \ili{Arabic} terms for the \ili{Persian} people and their language are \textit{ʕaǧam} ‘\ili{Persian}' (people and language; lit. `non-Arab'), and \textit{əl-ǧamāʕa} ‘Persians’ (lit. `group of people'). Both are often used pejoratively.} 

\section{Language contact in Khuzestan}

Currently, the main and most influential language in contact with \ili{Khuzestan} \ili{Arabic} (KhA) is the \ili{Western} \ili{Iranian} language \ili{Persian}. Among the other (partly historically) influential languages in the region the most prominent are \ili{English}, \ili{Turkish}/Ottoman (cf. \citealt{Ingham2005}), and \ili{Aramaic} (see Procházka, this volume).\ia{Procházka, Stephan@Procházka, Stephan}

\ili{Persian} and different forms of \ili{Arabic} share a long history of contact in the region of \ili{Khuzestan}, implying a long exchange of language material in both directions. 

KhA belongs to the \ili{Bedouin}-type south \ili{Mesopotamian} \textit{gələt}-dialects.\footnote{There is as yet no comprehensive grammar of the dialects of \ili{Khuzestan}. The main source of information on these dialects is the collection of data made in the 1960s by the Arabist and linguist Bruce Ingham (\citeyear{Ingham1973,Ingham1976,Ingham2011khuz}). The \isi{article} by Yaron Matras and Maryam Shabibi, “Grammatical borrowing in Khuzistani \ili{Arabic}” \citep{MatrasShabibi2007}, is based on Shabibi’s unpublished dissertation “Contact-induced grammatical changes in Khuzestani \ili{Arabic}” \citep{Shabibi2006}.} Therefore, it shows great similarity to \ili{Iraqi} dialects such as Baṣra \ili{Arabic}, as well as to other dialects in the \ili{Gulf}, such as \ili{Bedouin} \ili{Bahraini} \ili{Arabic} – that is, the \ili{Arabic} spoken by the Sunni Arab population descended from Najd. 

The dialects of \ili{Khuzestan} can be considered “peripheral” dialects of \ili{Arabic} because they are spoken in a country where \ili{Arabic} is not the language of the majority population and is not used in education or administration. Therefore, there is practically no influence of \ili{Modern Standard} \ili{Arabic}. However, because it shares a long geographically-open border with Iraq, \ili{Khuzestan} is not isolated from the \ili{Arabic}-speaking world. Moreover, since around 2000 it has had access to \ili{Arabic} news, soaps, etc. via satellite TV. Intra-\ili{Arabic} contact is limited to the linguistically very similar (southern) \ili{Iraqi} dialects\footnote{KhA is often differentiated from its neighboring \ili{Iraqi} dialects by the number of \ili{Persian} borrowings that are employed \citep[1020]{Gazsi2011}. Although the greatest influence has occurred in lexicon, \ili{Persian} influence also extends to grammar (see below).} through, for example, religious visits to Kerbala.

\ili{Persian} is the only official language in Iran, it is the only language used in education, and is sociolinguistically and culturally dominant, especially in the domains of business and administration. \ili{Persian} consequently enjoys high \isi{prestige} in society. For \ili{Persian} speakers, and sometimes also for KhA speakers, the KhA varieties have very low \isi{prestige} and are not associated with the highly \isi{prestigious} \ili{Arabic} of the \isi{Quran}, which is taught in schools. KhA speakers who acquire KhA as a first language usually acquire \ili{Persian} at school. Later, the opportunities for KhA speakers to use \ili{Persian} are restricted to certain social settings outside the family, e.g. school, work (employment in a large company would probably require communication in \ili{Persian}), contact with \ili{Persian} friends, or through the \ili{Persian} media.

Accordingly, the command of \ili{Persian} or the degree of \isi{bilingualism} among KhA speakers varies greatly due to such factors as level of education, affiliation, age, \isi{gender}, and urban or rural environment. The older generation and women have far less access to education and jobs and consequently less contact with people outside the family, which implies less exposure to contact situations and a lower degree of \isi{bilingualism}. Among some members of the younger generation we may notice a certain intentional reinforcement of \ili{Arabic} words alongside a resistance to recognizable \ili{Persian} lexical borrowings, plus a preference for the \ili{Arabic} over the \ili{Persian} names for the cities in \ili{Khuzestan}. This is of course consistent with nationalist ideas and the separatist movement taking place in present-day \ili{Khuzestan}, and also shows the impact of intentionality in language contact situations. 

In sum, one might find very different degrees of \ili{Persian} influence among the speakers of KhA (cf. \citealt{MatrasShabibi2007}: 147). For that reason, all statements on \ili{Persian}–KhA contact phenomena must be seen in relation to the above factors, which are decisive for any speaker’s command of \ili{Persian}. 
\largerpage

\section{Contact-induced changes in KhA} 
\subsection{General remarks\footnotemark}
The main aim of the present chapter is to highlight the most striking phenomena and trends in KhA \isi{language change} due to contact with \ili{Persian}.

\footnotetext{The data used for the present analysis was collected mainly in Aḥwāz, Muḥammara (Khorramshahr), Ḥamīdiyye and Ḫafaǧiyye (Susangerd) in 2016. The male and female informants were bilingual as well as \isi{monolingual} KhA speakers from 25 to over 70 years old.}

All phenomena of contact-induced change in KhA can be considered as \isi{transfer} of patterns or matter\footnote{Sakel (\citeyear[15]{Sakel2007}) defines matter \isi{replication} as the \isi{replication} of “morphological material and its phonological shape”.} from the \isi{source language} (\isi{SL}) \ili{Persian} to the \isi{recipient language} (\isi{RL}) KhA under \isi{RL} agentivity (i.e. borrowing rather than \isi{imposition}).{} The agents of \isi{transfer} are cognitively dominant in the \isi{RL} KhA, the agents’ L1. Even though \ili{Persian} is generally acquired during childhood and today is spoken by most speakers, it usually is the speakers’ L2. Cases of \isi{convergence} (cf. \citealt{Lucas2015}: 530–531) are possible in the present contact situation among speakers with a very high (L1-like) command of \ili{Persian}, for example university students. But of course it is hard to draw an exact line between L1 and L2 proficiency and thus between \isi{convergence} and borrowing (cf. \citealt{Lucas2015}: 531). 



\subsection{Phonology}
As in other \ili{Bedouin} \ili{Arabic} dialects, the presence of the phonemes /č/ and /g/ is ultimately the result of internal development from original *k and *q, rather than borrowing from \ili{Persian} (see Procházka, this volume).\ia{Procházka, Stephan@Procházka, Stephan}

The \isi{phoneme} /p/, e.g. \textit{perde} ‘curtain’ < Pers. \textit{parde},\footnote{For convenience, and due to the lack of sources on other spoken varieties of \ili{Persian}, in this and all following lexical references ``\ili{Persian}'' refers to Contemporary \ili{Standard} \ili{Persian}. This should not be taken to suggest that the relevant form in KhA was necessarily borrowed from this variety of \ili{Persian}. The transcription and translation of all \ili{Persian} lexical items is based on the forms as given by Junker \& Alavi (\citeyear{JunkerAlavi2002}) and/or information provided by native speakers.} is also common in all \ili{Iraqi} dialects and probably emerged in this region due to contact with \ili{Persian} and \ili{Kurdish} (see Procházka, this volume).\ia{Procházka, Stephan@Procházka, Stephan} 

An interesting phonological feature of KhA is that /ɣ/ often reflects etymological *q,\footnote{This phenomenon is also documented for the \ili{Arabic} dialects of Kuwait, Qatar, and the United \ili{Arabic} Emirates (\citealt{Holes2016}: 54, fn. 5).} which is otherwise realized as /g/ and /ǧ/. It is most likely that the shift /ɣ/ < *q first occurred in KhA forms borrowed from \ili{Persian} but ultimately of \ili{Arabic} origin, e.g. \textit{ɣisma} ‘part, section’ (cf. Pers. \textit{ɣesmat}), \textit{taṣdīɣ} ‘driving licence’ (cf. Pers. \textit{taṣdīɣ} `approval'), \textit{taɣrīban} ‘approximately’ (cf. Pers. \textit{taɣrīban} `idem'), \textit{bəɣri} ‘electronic’ (cf. Pers. \textit{barɣi} with the same meaning but ultimately going back to \ili{CA} \textit{barq} ‘lightning’). This feature is either an internal development,\footnote{Cf. Holes (\citeyear{Holes2016}: 53–54), who explains the /ɣ/–/q/ \isi{merger} among the Najd-descendent \ili{Bahraini} \ili{Arabic} speakers as an internal development.} or a \isi{transfer} from \ili{Persian}, in which both *q and *ɣ in \ili{Arabic} \isi{loanwords} are always pronounced /ɣ/ (\citealt{MatrasShabibi2007}: 138).\footnote{In \ili{Modern Standard} \ili{Persian} with Tehran ``standard'' pronunciation (cf. \citealt{Paul2018}: 581) the \isi{phoneme} /ɣ/ (corresponding to \ili{CA} /q/) has two allophones, [ɢ] and [ɣ] (\citealt{Majidi1986}: 58–60). There are, however, some varieties of Spoken Modern \ili{Persian}, for instance Yazdi \ili{Persian}, that maintain a difference between *q and *ɣ (Chams Bernard \ia{Bernard, Chams@Bernard, Chams} personal communication; cf. \citealt{Paul2018}: 582).} Later, this phonological change further affected lexemes which have no \isi{cognate} forms in \ili{Persian}, e.g. \textit{baɣra} `cow', a borrowing from \ili{Modern Standard} \ili{Arabic} (the KhA dialectal form being \textit{hāyša} `cow'). There are, however, certain lexemes, especially those that do not have a \isi{cognate} form in \ili{Persian}, which are not affected by this rule, e.g. \textit{gāl} `he said', \textit{gēð̣} ‘summer’, or \textit{marag} ‘sauce’. Other lexemes show free variation in the pronunciation of /g/, e.g. \textit{gabul} \~{} \textit{ɣabul} `formerly, before'.

Lexical borrowings are often adapted to \ili{Arabic} phonology. For example, speakers of the older generation usually pronounce the \isi{phoneme} /p/ as [b], e.g. \textit{berde} ‘curtain’ < Pers. \textit{parde}. 

Negative structures bear \isi{stress} on the first syllable,\footnote{Ingham (\citeyear[724]{Ingham1991}) describes this phenomenon also for KhA \textit{wh}-interrogatives and \isi{prepositions}.} e.g. KhA \textit{mā́} \textit{arūḥ} ‘I don’t go’. This is a feature shared with some \ili{Persian} and \ili{Turkish} varieties and other North East Arabian dialects (\citealt{Ingham2005}: 178–179). This common phonological characteristic therefore seems to be a Sprachbund phenomenon of the \ili{Mesopotamian} region, which reflects the long history of contact and migration across language boundaries due to trade, war, shared cultural practices, nomadism, etc. (\citealt{Winford2003}: 70–74). Though the directions and mechanisms of borrowing within the languages of a Sprachbund are often hard to categorize \citep[74]{Winford2003}, we can probably assume that KhA, being spoken by a minority group, has borrowed and adapted this phonological \isi{stress} pattern under \isi{RL} agentivity.



\subsection{Morphosyntax}
\subsubsection{Replication of Persian phrasal verbs}

The \isi{replication} of phrasal verbs is a contact phenomenon also found in the \ili{Arabic} varieties of Turkey (\citealt{Grigore2007book}: 157–159; Procházka, this volume).\ia{Procházka, Stephan@Procházka, Stephan} As shown in examples (\ref{muxx})--(\ref{tala}), KhA replicates \ili{Persian} phrasal verbs by substituting the \ili{Persian} \isi{light verbs} with KhA equivalents and directly replicating the \ili{Persian} nouns (cf. \citealt{MatrasShabibi2007}: 142). The noun in example (\ref{muxx}) is \ili{Arabic} in its origins but its usage in a phrasal verb construction with a new meaning is a \ili{Persian} innovation.

\ea \label{muxx}
\ea Aḥwāz, \ili{Khuzestan}, male, 26 years (own data)\\
\gll ṭəgg muḫḫ\\
     hit.\textsc{prf}.3\textsc{sg}.\textsc{m} brain\\ 
\ex{Persian}\\
\gll muḫḫ zadan\\
     brain hit.\textsc{inf} \\
\glt ‘to brainwash, convince someone’\footnote{All \ili{Persian} translations are given in the modern spoken Tehrani variety of \ili{Persian}, and were provided by Hooman Mehdizadehjafari, a native speaker of this variety. They are presented in a broad phonemic transcription.} 
\z
\z

\ea \label{irad}
\ea{Aḥwāz, \ili{Khuzestan}, male, 39 years (own data)}\\
\gll kað̣ð̣ īrād\\
     take.\textsc{prf}.3\textsc{sg}.\textsc{m} nagging\\ 
\ex{Persian}\\
\gll īrād gereftan\\
     nagging take.\textsc{inf}\\
\glt ‘to find fault with someone’ 
\z\z

 As examples (\ref{omade}) and (\ref{tala}) show, \ili{Persian} nouns are sometimes adapted morphophonologically.

\ea \label{omade}
\ea{Aḥwāz, \ili{Khuzestan}, male, 50 years (own data)}\\
\gll sawwa ʔōmāde\\
     make.\textsc{prf}.3\textsc{sg}.\textsc{m} ready\\ 
\ex{Persian}\\
\gll āmāde kardan \\
     ready make.\textsc{inf}\\
\glt ‘to prepare sth.’
\z\z

\ea \label{tala}
\ea{Aḥwāz, \ili{Khuzestan}, male, 26 years (own data)}\\
\gll ṭalaʕ ɣabūli\footnotemark \\
     emerge.\textsc{prf}.3\textsc{sg}.\textsc{m} acceptance\\
\ex{Persian}\\
\gll ɣabūl šodan\\
     acceptance become.\textsc{inf} \\
\glt ‘to pass (an exam), be accepted’
\z\z
\footnotetext{The final \textit{-i} in \textit{ɣabūli} probably originates from the \ili{Persian} indefiniteness marker \textit{-i} (see \citealt{Majidi1990}: 309–314) and has become part of this word in KhA, so that \textit{ɣabūli} is monomorphemic.}
The pattern for phrasal verbs – transferred into the \isi{RL} KhA under \isi{RL} agentivity – provides KhA with an easy way to convert foreign nouns into verbs.

As illustrated in examples \REF{taggi} and \REF{tamate}, the pattern is adapted according to \ili{Arabic} syntactic rules: (i) the verb is moved into the initial position; and (ii) a direct object is introduced between verb and nominal element (post-verbally). In \ili{Persian}, however, the verb always remains in final position following the nominal element and a direct object would be introduced before both elements (see e.g. \citealt{Majidi1990}: 447–448).

\ea \label{taggi}
\ea{Aḥwāz, \ili{Khuzestan}, male, 26 years (own data)}\\
\gll ṭəggi ɣandart-i wāks\\
     hit.\textsc{imp.2sg.f} shoe-\textsc{obl}.1\textsc{sg} wax\\
 
\ex{Persian}\\
\gll kafš-am-o vāks be-zan\\
     shoe-\textsc{obl.}1\textsc{sg-obj} wax \textsc{imp-}hit\textsc{.prs}\\
\glt ‘Polish my shoes!’
\z\z

\ea \label{tamate}
\ea{Aḥwāz, \ili{Khuzestan}, female, 35 years (own data)}\\
\gll yṭəggūn əṭ-ṭamāṭe rande\\
     hit.\textsc{impf.3pl.m} \textsc{def}{}-tomato grater \\
 
\ex{Persian}\\
\gll gūǧe\_farangi-ro rande mī-zanan\\
     tomato-\textsc{obj} grater \textsc{ind}-hit\textsc{.prs.3pl} \\
\glt ‘They grate some tomato.’
%\check glosses 
\z\z

This structure has become productive in KhA. For example, in the phrasal verb \textit{ṭəgg} \textit{dabbe} ‘to cheat’ (lit. ‘to hit a water canister’) both the verb and noun are taken from KhA and only the construction’s syntactic pattern is taken from \ili{Persian}. 

\subsection{Syntax}
\subsubsection{Definiteness marking}

Matras \& Shabibi (\citeyear{MatrasShabibi2007}: 141–142) see KhA \isi{relative} clauses without \isi{definite} heads as evidence for the decline of overt \isi{definiteness} marking in KhA, based on a \ili{Persian} model with generally unmarked \isi{definiteness}, e.g. \textit{mara} \textit{lli} \textit{šiftū-ha} \textit{ḫābarat} ‘The woman that you saw called’ (\citeyear{MatrasShabibi2007}: 142). However, this pattern is also documented in \ili{Arabic} dialects which have had no contact with \ili{Persian} (\citealt{Pat-El2017}: 454–455; cf. \citealt{Procházka2018Fertile}: 269).

Matras \& Shabibi (\citeyear[140]{MatrasShabibi2007}) further postulate that the \ili{Persian} \textit{ezāfe} pattern in adjectival attribution is replicated in KhA.\footnote{See e.g. Ahadi (\citeyear[103–109]{Ahadi2001}) for the usage of the \ili{Persian} \textit{ezāfe}.} According to their theory, the construct state marker -\textit{t} (with an indefinite head) and/or the \isi{definite} \isi{article} (of the attribute) are reanalysed as markers of attribution matching the \ili{Persian}  \textit{ezāfe} marker \textit{-(y)e}, as in \REF{island}. 

\ea\label{island}
\ea
{KhA (\citealt{MatrasShabibi2007}: 140)}\\
\gll ǧazīra-t l-ḫað̣ra \\
     island-\textsc{con} \textsc{def}-green.\textsc{f}\\
 
\ex{Persian}\\
\gll ǧazīre-ye sabz\\
     island-\textsc{ez} green\\
\glt ‘the green island’ 
\z\z

However, this pattern is also observed in other modern \ili{Arabic} dialects which have not been exposed to \ili{Persian} influence as well as in older forms of \ili{Arabic}.\footnote{See Pat-El (\citeyear{Pat-El2017}: 445–449) for numerous examples from different modern \ili{Arabic} dialects, Middle \ili{Arabic}, Pre-Islamic \ili{Arabic}, \isi{Quran} \ili{Arabic}, and other \ili{Central} \ili{Semitic} languages. See also Retsö (\citeyear{Retsö2009}: especially 21–22) and Procházka (\citeyear{Procházka2018Fertile}: 267–269), who also proves that this is an old feature already found in \ili{Old} \ili{Arabic} and points out that it is mainly found among dialects which are spoken in regions with no or only marginal influence from \ili{Modern Standard} \ili{Arabic}.} 

Consequently, it is highly unlikely that this phenomenon has developed due to \ili{Persian} influence, although it cannot be ruled out that contact with \ili{Persian} has fostered the preservation of this apparently old feature. 


\subsubsection{Word order changes} \label{woc}

KhA shows no changes due to contact in basic \isi{word order}.\footnote{Ingham (\citeyear[715]{Ingham1991}) states that in KhA neither VSO nor SVO \isi{word order} is particularly dominant. Matras \& Shabibi (\citeyear[147]{MatrasShabibi2007}) postulate that the usage of OV order in KhA is increasing as “the beginning of a shift in \isi{word order}” on the basis of the \ili{Persian} type, where OV prevails. In both of their examples the objects are topicalized (with pronominal resumption), which is a common phenomenon in spoken \ili{Arabic} (\citealt{Brustad2000}: 330–333; 349), and as such not obviously the result of \ili{Persian} influence (cf. El Zarka \& Ziagos \citeyear{ElZarkaZiagos2019}, who in their recent description of the beginnings of \isi{word order} changes in some \ili{Arabic} dialects spoken in southern Iran, show that these dialects, like KhA, have still retained VO as their basic \isi{word order} despite the strong influence of \ili{Persian}).}  The only attested \isi{word order} changes concern the position of the verbs \textit{čān} ‘to be’ and \textit{ṣār} ‘to become’, both of which can appear in final position as an unmarked construction. This sentence-final position in no case functions as the default, and is in fact less frequent than its non-final position.\footnote{In my data, \textit{čān} appears 23 of 152 times in sentence-final position, \textit{ṣār} 11 of 165 times. The additional examples are taken from my questionnaire.} \textit{čān} or \textit{ṣār} in final position are never stressed.

The sentence-final position of \textit{čān} or \textit{ṣār} (see examples \REF{port}--\REF{cold}) is likely a pattern \isi{replication} of the \ili{Persian} model, i.e. sentences with final \textit{būdan} ‘to be’ or \textit{šodan} ‘to become’.

\ea\label{port}
\ea
{ʕAbbādān, \ili{Khuzestan}, male, 35 years (own data)}\\
\gll šuɣul-hum b-əl-bandar čān\\
     work-\textsc{3pl}.\textsc{m} in-\textsc{def}-port be.\textsc{prf.3sg.m}\\
 
\ex{Persian}\\
\gll kār-ešūn tū-ye bandar būd\\
     job-\textsc{obl}.3\textsc{pl} in-\textsc{ez} port be.\textsc{pst.3sg}\\
\glt ‘Their job was at the port.’
\z\z

\ea
\ea {Muḥammara, \ili{Khuzestan}, male, 30 years
 (own data)}\\
\gll əǧdād-i mallāk-īn čānaw\\
     grandparents-\textsc{obl.1sg} owner-\textsc{pl} be.\textsc{prf.3pl.m}\\
 
\ex{Persian}\\
\gll aǧdād-am mālek būdan\\
     grandparents-\textsc{obl.1sg} owner be.\textsc{pst.3pl}\\
\glt ‘My grandparents were owners [of land].’
\z\z
\newpage
\ea\label{cold}
\ea {Aḥwāz, \ili{Khuzestan}, female, 40 years (own data)}\\
\gll hassa šway l-māy bārəd ṣār\\
     now a\_bit \textsc{def}{}-water cold become.\textsc{prf.3sg.m}\\
 
\ex{Persian}\\
\gll alʔān yekam ʔāb sard šod\\
     now a\_bit water cold become.\textsc{pst}.\textsc{3sg}\\
\glt ‘The water has become a bit cold now.’
\z\z

The next example might show a tendency to use a present-\isi{tense} \isi{copula} with human subjects, expressed with the verb \textit{ṣār} ‘to become’: 
\ea
\ea {Aḥwāz, \ili{Khuzestan}, female, 35 years
 (own data)}\\
\gll əhya mart uḫū-y əṣṣīr\\
     3\textsc{sg.f} wife brother-\textsc{obl.1sg} \textsc{cop.impf.3sg.f}\\
 
\ex{Persian}\\
\gll ūn zan-dādāš-am-e\\
     3\textsc{sg} wife-brother-\textsc{obl.1sg}-\textsc{cop.prs.3sg}\\
\glt ‘She is the wife of my brother.’
\z\z


In the KhA construction for pluperfect \isi{tense}, \textit{čān} can also appear in sentence-final position, after the active \isi{participle}. This construction, although not very frequent, is very likely a direct \isi{transfer} of the \ili{Persian} structure, in which the auxiliary \textit{būdan} also follows the \isi{participle}.\footnote{Matras and Shabibi (\citeyear{MatrasShabibi2007}: 142–143) describe the use of this construction as a change in the KhA \isi{tense} system. However, the pattern \textit{kān} + active \isi{participle} is also commonly used in other \ili{Arabic} dialects to express pluperfect meaning or to describe completed actions which have an impact on the present, see for example Denz (\citeyear{Denz1971}: 92–94; 115–116) for \ili{Iraqi} (Kwayriš) and Grotzfeld (\citeyear[88]{Grotzfeld1965}) for \ili{Syrian} \ili{Arabic}.} 

\ea \ea {Aḥwāz, \ili{Khuzestan}, male, 26 years (own data)}\\
\gll lamman əyēna l-əl-bīət, əhma mākl-īn čānaw\\
     when come.\textsc{prf.1pl} to-\textsc{def}-house 3\textsc{pl}.\textsc{m} eat.\textsc{ptcp-pl.m} be.\textsc{prf}.\textsc{3pl.m}\\

\ex{Persian}\\
\gll vaɣti-ke mā bargaštīm ḫūne, ūnhā ɣazā-ro ḫorde būdan  \\
     when-\textsc{rel} \textsc{1pl} come\_back.\textsc{pst}.\textsc{1pl} home 3\textsc{pl} food-\textsc{obj} eat.\textsc{ptcp} be.\textsc{pst}.\textsc{3pl}\\
\glt ‘When we came home, they had (already) eaten.’
\z
\z

This \isi{word order} change has probably been triggered by the high \isi{frequency} in speech of \ili{Persian} sentences with forms of \textit{būdan} in final position. Lucas (\citeyear[295]{Lucas2012}) explains the usage of foreign patterns as the result of the human cognitive tendency to minimize the high processing efforts associated with the extensive use of two languages.\footnote{{Connections between units of a neural network associated with certain syntactic patterns can be strengthened from repeated exposure to and use of that pattern} {\citep[291]{Lucas2012}. Hence, the employment of a \ili{Persian} syntactic structure in KhA needs less processing effort because the same strengthened neural network is activated.}}

\textit{čān} is also used in sentence-final positions after the main verb in the imperfect in KhA constructions expressing the continuous past. In spoken \ili{Persian}, the continuous past is formed without a sentence-final \textit{būdan}.\footnote{The Modern \ili{Iranian} \ili{Persian} continuous past is formed with the particle \textit{mī} prefixed to the simple past of the respective main verb and can (for the progressive form) be preceded by the simple past of \textit{dāštan} ‘to have’: e.g. (\textit{dāšt}) \textit{mī-raft} ‘he was going’ (\citealt{Majidi1990}: 232, 235).} This case is not a direct \isi{transfer} of the \ili{Persian} pattern, but perhaps a construction analogous to the pluperfect and other \ili{Persian} forms with \textit{būdan} in final position. 

\ea
\ea {Aḥwāz, \ili{Khuzestan}, male, 55 years (own data)}\\
\gll hāda ham mən zuɣur yəštəɣəl čān  \\
     \textsc{dem}.\textsc{sg}.\textsc{m} also from childhood work.\textsc{impf.3sg.m} be.\textsc{prf.3sg.m}\\
 
\ex{Persian}\\
\gll īn-am az kūdaki kār mī-kard \\
     \textsc{dem.sg}-also from childhood work \textsc{ind}{}-do.\textsc{pst.3sg}\\
\glt ‘This one has also been working from childhood on.’
\z\z

Example \REF{mother} shows both syntactic variants in one sentence, i.e. \textit{čān} before and after the main verb.

\ea\label{mother} \ea {Muḥammara, \ili{Khuzestan}, female, 40 years (own data)}\\
\gll umm-i čānat tətḥaǧǧab, eh, əb-zamān əš-šāh, bass tətbawwaš čānat\\
     mother-\textsc{obl.1sg} be\textsc{.prf.3sg.f} veil.\textsc{impf.3sg.f} yes in-time \textsc{def}{}-shah only veil.\textsc{impf.3sg.f} be\textsc{.prf.3sg.f}\\
\newpage 
\ex{Persian}\\
\gll mādar-am (dāšt) neqāb mī-zad, āre, dar zamān-e šāh, hamīše neqāb mī-zad \\
     mother-\textsc{1sg} (have.\textsc{pst.3sg}) veil \textsc{ind}-hit.\textsc{pst.3sg} yes in time-\textsc{ez} shah always veil \textsc{ind}-hit.\textsc{pst.3sg}\\
\glt  ‘My mother used to veil her face (with a \textit{būšiyye}),\footnote{\textit{būšiyye} or \textit{pūšiyye} ‘veil’ is also documented for \ili{Iraqi} \ili{Arabic} (\citealt{WoodheadEtAl1967}: 53).} yes, during the times of the shah, she always used to veil her face.’ 
\z\z

Because all the above examples equally work with \textit{čān/ṣār} in non-final position, this process of word-order-related pattern \isi{replication} in KhA is still ongoing. Indeed, all informants, when asked for the correct structure in the above examples, preferred the verb \textit{čān} in non-final position.\footnote{My informants from Baghdad considered all constructions with \textit{čān} in final position to be wrong. However, this structure is used in Baṣra \ili{Arabic} (Qasim Hassan, personal communication, January 2018).\ia{Hassan, Qasim@Hassan, Qasim}}

Lucas (\citeyear{Lucas2015}: 530–531) explains the basic \isi{word order} changes (from VSO to SOV) in Bukhara \ili{Arabic} (cf. \citealt{Ratcliffe2005}: 143–144; and \citealt{Versteegh2010}: 639) as a result of \isi{convergence} with Uzbek.\footnote{Lucas (\citeyear[525]{Lucas2015}) defines \isi{convergence} as changes made to a language under the agentivity of speakers who are native speakers of both the \isi{SL} and the \isi{RL}.} Although a clear division between \isi{convergence} and borrowing is hard to make, I consider the contact-induced \isi{word order} changes that occur in KhA to be instances of borrowing because most speakers are clearly native speakers of, and therefore dominant in, KhA only.


\subsubsection{\textit{ḫōš} preceding verbs and nouns} \label{xosh}

In \ili{Persian}, \textit{ḫoš} ‘good, well’ is used as a prefixed (lexicalized) element preceding some nouns and verbs to coin compound adjectives, nouns, and verbs (\citealt{Majidi1990}: 411, 413): e.g. Pers. \textit{ḫoš-andām} ‘handsome’ (< \textit{andām} ‘shape; body’), \textit{ḫoš-nevīs} ‘calligrapher’ (< present \isi{stem} \textit{nevīs-} ‘to write’).

KhA has borrowed some of these \ili{Persian} compound adjectives: e.g. KhA \textit{ḫōš-bū} ‘nice-smelling’ (< Pers. \textit{bū} ‘smell, scent’), \textit{ḫōš-tīp} ‘handsome’ (< Pers. \textit{tīp} ‘type’), and \textit{ḫōš-aḫlāq} ‘(with) good manners’ (< Pers. \textit{aḫlāq} ‘decency; ethics, morality’, pl. of \textit{ḫolq} ‘character, nature’). However, in KhA the use of this element has been further developed. It is productively used as an attributive adjective preceding nouns, but not agreeing in \isi{gender} or number with them, e.g. \textit{ḫōš} \textit{walad} ‘a good boy’, \textit{ḫōš} \textit{əbnayya} ‘a good girl’, \textit{ḫōš} \textit{banāt} ‘good girls’, \textit{ḫōš} \textit{əwlād} ‘good kids’.\footnote{This construction is also found in \ili{Iraqi} \ili{Arabic} (cf. \citealt{Erwin1963}: 256), which might prove that the element \textit{ḫōš} is an older borrowing.}



\subsection{Lexicon}
\subsubsection{Lexical transfer}

The greatest influence from \ili{Persian} on KhA has occurred in lexicon. Many \ili{Persian} lexemes were borrowed generations ago. The most frequently borrowed elements are nouns denoting cultural or technological innovations which have filled lexical gaps in the \isi{RL} KhA. Verbs, adverbs, adjectives, and many discourse particles have also been borrowed from the \isi{SL} \ili{Persian}.

The majority of the examples below are cases of \isi{transfer} of morphophonological material (matter) and semantic meaning (pattern) under \isi{RL} agentivity. 

Many of the \ili{Persian} borrowings have been phonologically and morphologically integrated into the \isi{RL}. For instance, for many borrowed \ili{Persian} nouns \ili{Arabic} \isi{internal plural} forms are created, e.g. \textit{ḫətākīr} ‘ball-point pens’ (sg. \textit{ḫətkār} < Pers. \textit{ḫod-kār} ‘ball-point pen’), or \textit{banādər} ‘ports’ (sg. \textit{bandar} < Pers. \textit{bandar} ‘port’).

Again, the borrowing of foreign (L2) elements into the speakers’ L1 might be explained by the human cognitive tendency to minimize the processing effort in lexical selection between two languages (\citealt{Lucas2012}: 291; see §\ref{woc}). So if a certain \ili{Persian} word is frequently used and often heard (for example at school), the connections of a neural network associated with this word are strengthened \citep[291]{Lucas2012}, which makes it easier to employ the word in one’s L1. 


\subsubsection{Semantic fields}

The following illustrative list of \ili{Persian} loans in KhA shows the most important semantic fields of lexical borrowing.

\begin{itemize} 
\item[] Administration and military: \textit{čārra} ‘crossroad’ < Pers. \textit{čahār-rāh}; \textit{sarbāz} \~{} \textit{šarbāz} ‘soldier’ < Pers. \textit{sarbāz}; \textit{farmāndāri} ‘governorship’ < Pers. \textit{farmān\-dāri}.
\item[] Agriculture: \textit{kūd} ‘dung’ < Pers. \textit{kūd}; \textit{ʕalafkoš} ‘pesticide' (lit. weed-killer) < Pers. \textit{ʔalaf-koš}. 
\item[] Dress and textiles: \textit{dāmen} ‘skirt’ < Pers. \textit{dāman}; \textit{šīəla} ‘head covering’ < Pers. \textit{šāl} ‘Kashmir shawl’ \citep[174]{Ingham2005}.
\item[] Education:  \textit{klāṣ} ‘class, grade’ < Pers. \textit{kelās}; \textit{ḫətkār} ‘ball-point pen’ < Pers. \textit{ḫod-kār}; \textit{dānišga} ‘university’\textit{<} Pers. \textit{dānišgāh}.
\item[] Food: \textit{ǧaʕfari} ‘parsley’ < Pers. \textit{ǧaʔfari}; \textit{češmeš} ‘raisins’ < Pers. \textit{kešmeš}; \textit{serke} ‘vinegar’ < Pers. \textit{serke}; \textit{šalɣam} ‘turnip’ < Pers. \textit{šalɣam}.
\item[] Material culture: \textit{šīše} ‘bottle’ < Pers. \textit{šīše}; \textit{ǧām} ‘(window) glass’ < Pers. \textit{ǧām} ‘(window) glass; goblet, cup'; \textit{tīɣe} ‘blade’ < Pers. \textit{tīɣe}; \textit{yəḫčāle} ‘refrigerator’ < Pers. \textit{yahčāl}; \textit{sīm} \textit{buksel} ‘towrope’ < Pers. \textit{sīm-e} \textit{boksol}; \textit{perde} \~{} \textit{berde} ‘curtain’ < \textit{parde}; \textit{gīre} ‘hair barrette’ < Pers. \textit{gīre-ye} \textit{sar/mūy}; \textit{mīz} ‘table’ < Pers. \textit{mīz}; \textit{darīše} ‘window’ < Pers. \textit{darīče}; \textit{pənǧara} ‘window’ < Pers. \textit{panǧare}.
\item[] Other: \textit{ɣīme} ‘price’ < Pers. \textit{ɣīmat}; \textit{bandar} ‘port’ < Pers. \textit{bandar}; \textit{nāmard} ‘brute’ < Pers. \textit{nāmard} `coward; brute, rascal'.
\end{itemize}

Some items ultimately of \ili{Arabic} origin have been re-borrowed into KhA from \ili{Persian}, preserving the \ili{Persian} meaning, e.g. KhA \textit{bərɣi} ‘electronic’ < Pers. \textit{barɣ} ‘electricity; lightning’ < \ili{Arabic} \textit{barq} ‘lightning’. 


\subsubsection{Verbs and adverbs} 

KhA verbs and adverbs resulting from language contact are always morphologically integrated. These are either directly borrowed \ili{Persian} verbs, e.g. \textit{bannad} ‘to close (e.g. the tap)’ < Pers. imperfect and present \isi{stem} \textit{band-} ‘close’;\footnote{Also common in the \ili{Gulf} region and in \isi{Yemen} (\citealt{BehnstedtWoidich2014}: 290).} \textit{gayyər} ‘to get stuck’ <  Pers. \textit{gīr} \textit{šodan} ‘to get stuck’; \textit{ʕamm{әr}} ‘to repair’ < Pers. \textit{taʕmīr} \textit{kardan} ‘to repair’; \textit{čass{әb}} ‘to glue’ < Pers. \textit{časb} \textit{zadan} ‘to glue’; \textit{gəzar} ‘to pass (time)’ < Pers. present \isi{stem} \textit{gozar-} ‘to pass (time)’ (see example \REF{ca} below);\footnote{The verb \textit{gəzar} is used only in phrases that refer to the “passing by” of life.} \textit{zaḥəm} ‘to bother’ (transitive) < Pers. \textit{zahmat dādan} ‘to bother, cause trouble’ (transitive) (see examples \REF{zah} and \REF{mumkin} below);\footnote{The KhA noun \textit{zaḥme} `shame' is also used for a rebuke, e.g. \textit{zaḥme} \textit{ʕalīək!} ‘Shame on you!’, which would be expressed in a different way in \ili{Persian}: \textit{ḫeǧālat} \textit{ne-mī-keši?} ‘Shame on you!' (lit. ‘Are you not ashamed?’).} or \ili{Persian} nouns turned into KhA (ad)verbs, e.g. \textit{əb-zūr} ‘by force’ < Pers. \textit{zūr} ‘power; violence; force’.

\ea
{Aḥwāz, \ili{Khuzestan}, male, 26 years (own data)}\\ \label{ca}
\gll čā hāy əl-ḥayāt lō la? təgzar baʕad, təmši\\
     \textsc{dm} \textsc{dem}.\textsc{f} \textsc{def}{}-life or no pass.\textsc{impf}.\textsc{3sg.f} after\_all go.\textsc{impf.3sg.f}\\
\glt ‘See, that is how life is, right? It passes by (quickly), it goes.’
\z
\protectedex{
\ea \label{zah}
{Aḥwāz, \ili{Khuzestan}, male, 26 years (own data)}\\
\gll zaḥmīət-kum, ʕafwan\\
     bother.\textsc{prf}.\textsc{1sg}{}-\textsc{2pl.m} sorry\\
\glt ‘Sorry, I must have bothered you.’\footnotemark
\z
}
\ea \label{mumkin}
{Aḥwāz, \ili{Khuzestan}, male, 25 years (own data)}\\
\gll mumkin azaḥm-ək əb-šuɣla\\
     possible bother.\textsc{impf}.\textsc{1sg}{}-\textsc{2sg.m} with-issue\\
\glt ‘May I bother you with something (i.e. ask you a favour)?’
\z

\footnotetext{A phrase often used when leaving, for example after an invitation for dinner, cf. Pers. \textit{ḫeyli zahmat dādīm} lit. `We have caused (you) a lot of trouble'.}

\subsubsection{Discourse elements}

A range of \ili{Persian} discourse elements have been borrowed by KhA (cf. \citealt{MatrasShabibi2007}: 143–145),\footnote{Matras \& Shabibi (\citeyear[144]{MatrasShabibi2007}) claim that the \ili{Persian} conjunctions \textit{agarče} and \textit{bāīnke}, both meaning ‘although, even though’, and the \ili{Persian} factual \isi{complementizer} \textit{ke} ‘that’ have also been borrowed by KhA. However, I have found no evidence for their usage in my data.} e.g. KhA \textit{ham} \~{} \textit{hamme} ‘also, as well’ < Pers. \textit{ham} and KhA \textit{ham}…\textit{ham} ‘(both)…and’ < Pers. \textit{ham}…\textit{ham};\footnote{This discourse element is also known for Iraq \citep[36]{Malaika1963} and, like KhA \textit{hast} \~{} \textit{hassət} ‘there is’ < Pers. \textit{hast} (\citealt{Ingham1973}: 25, fn.27), is probably an older borrowing.} or KhA \textit{hīč} ‘nothing; no(t)… at all’ < Pers. \textit{hīč}.\footnote{Shabibi (\citeyear{Shabibi2006}: 176–177) further derives KhA \textit{balkət} ‘maybe, hopefully’ from Pers. \textit{balke} \textit{ham}, which can mean ‘maybe’. A \ili{Turkish} origin of this word seems more likely: cf. Aksoy (\citeyear[620]{Aksoy1963}) for the existence of \textit{belke} \~{} \textit{belkit} in \ili{Eastern} \ili{Turkish} dialects. Malaika (\citeyear[35]{Malaika1963}) also derives the \ili{Baghdadi} \ili{Arabic} \textit{belki} ‘rather, maybe’ from \ili{Turkish}, as does Seeger (\citeyear[28]{Seeger2009}) for \textit{balki,} \textit{balkīš,} \textit{balkin} ‘maybe; possibly; probably’ in Ramallah \ili{Arabic}.} 

The KhA discourse elements \textit{ḫō}/\textit{ḫōš} ‘well; okay’ < Pers. \textit{ḫo(b)}/\textit{ḫoš} are often used phrase-initially, (\ref{okay}).\footnote{According to my informants and data, the form \textit{ḫōb} is not used in KhA (contrast \citealt{MatrasShabibi2007}: 143).} They are of \ili{Persian} origin, but have partly adopted a different form and function in KhA.\footnote{In \ili{Persian}, \textit{ḫob} is a discourse particle and related to the adjective and adverb \textit{ḫūb}, \textit{ḫo} is also a discourse particle used in less formal situations (Mehrdad Meshkinfam, Erik Anonby and Mortaza Taheri-Ardali, personal communication), and \textit{ḫoš} is an adjective (see §\ref{xosh}; \citealt{Shabibi2006}: 160; \citealt{Mohammadi2018}: 104--105). Thus the \ili{Persian} adjective \textit{ḫoš} has been desemanticized in KhA to function as a discourse particle with the meaning ‘well, okay’ \citep[160]{Shabibi2006}.}

\protectedex{
\ea\label{okay}
{Aḥwāz, \ili{Khuzestan}, male, 55 years (own data)}\\
\gll ḫōš, š-ʕəd-na, taʕay əhna baba\\
     \textsc{dm} what-at-\textsc{1pl} come.\textsc{imp}.\textsc{sg.}\textsc{f} here father\\
\glt ‘Okay, what (else) do we have, come here, dear!’
\z
}

Both \textit{ḫō} and \textit{ḫōš} are also often used in stories following the verb \textit{gāl} ‘to say’.

\ea
{Aḥwāz/Fəllāḥiyya, \ili{Khuzestan}, female, 50 years (own data)}\\
\gll lamman ɣada mən ʕəd-hum, gāl-la ḫō, hāy ər-rummānāt š-asawwi bī-hən\\
     when leave.\textsc{prf}.\textsc{3sg.m} from at-\textsc{3pl.m} say.\textsc{prf}.\textsc{3sg.m}-\textsc{dat.3sg.m} \textsc{dm} \textsc{dem}.\textsc{sg}.\textsc{f} \textsc{def}-pomegranate.\textsc{pl} what-make.\textsc{impf}.\textsc{1sg} with-\textsc{3pl.f}\\
\glt ‘When he left them, he said to him, ``Well, what shall I do with these pomegranates?”’
\z

\section{Conclusion}

Because of the dominance of \ili{Persian} in the \ili{Iranian} educational system and work environment, the lack of influence from \ili{Modern Standard} \ili{Arabic}, and the long period of geographical proximity, the \ili{Persian}-speaking society of southwest Iran has left many linguistic traces in the language of the \ili{Arabic}-speaking community of \ili{Khuzestan}. 

Van Coetsem (\citeyear{VanCoetsem2000}: 59; cf. \citealt{Lucas2015}: 532) suggests that lexical, but not syntactic and phonological \isi{transfer} is to be expected under \isi{RL} agentivity. However, KhA phonology and syntax have been influenced by the \isi{SL} \ili{Persian} under \isi{RL} agentivity, albeit to a much lesser extent than the lexicon.

KhA does not show \isi{transfer} of patterns from \ili{Persian} in either \isi{inflectional} or \isi{derivational} morphology. However, we do find an adapted pattern \isi{replication} of \ili{Persian} phrasal verbs (with preservation of the \ili{Arabic} \isi{word order}). 

As for syntax and contact-induced \isi{word order} changes, the alternative sentence construction with \textit{čān} in sentence-final position can be explained as a result of \ili{Persian} influence on KhA. This change might have been triggered by the similar and very frequent \ili{Persian} constructions with sentence-final \textit{būdan}. Thus, we do have some syntactic change due to \isi{transfer} under \isi{RL} agentivity, which Van Coetsem considered to be unexpected (see above).

\ili{Persian} lexical items have often been borrowed in KhA for novel concepts (lexical gaps), which is why semantic fields relating to technical or cultural innovations, education, and administration show the greatest amount of \ili{Persian} borrowing. This also explains why nouns are generally more often transferred than verbs (cf. \citealt{Lucas2015}: 532). \ili{Persian} words are regularly integrated into KhA phonology and morphology, for example the \ili{Arabic} \isi{internal plural} is formed for \ili{Persian} nouns. Also, many discourse particles have been transferred from \ili{Persian} into KhA. Some of them, e.g. \textit{ham} ‘also’, had been in use generations ago among \ili{Arabic} speakers in \ili{Khuzestan} and beyond (Iraq, \ili{Gulf}).

Of course, contact between KhA and \ili{Persian} has always been limited to certain social contexts (outside the family), especially for women, who had and still have much less access to education and employment and thus to the \ili{Persian}-speaking world. This fact, and some structural differences between the languages, explain the limits of contact-induced \isi{language change} in KhA, especially in morphology and syntax.

Hopefully, \isi{future} research on the dialects of \ili{Khuzestan} will provide more empirical data on instances of contact-induced change. An enlarged database should especially provide further evidence concerning the development and extent of \isi{word order} changes.

\section*{Further reading}


\citet{Ingham2011khuz} provides a sketch grammar of KhA.\\
\citet{Ingham2005} discusses \ili{Turkish} and \ili{Persian} borrowings in KhA and northeastern Arabian dialects.\\
\citet{MatrasShabibi2007} is an \isi{article} on contact-induced changes in KhA based on \citet{Shabibi2006}.\\
\citet{Shabibi2006} is an unpublished doctoral dissertation on contact-induced change in KhA.



\section*{Acknowledgements}

I would like to thank Stephan Procházka and Dina El-Zarka for their critical remarks and bibliographical suggestions on this chapter. I would additionally like to thank my informant and good friend Majed Naseri for all his help on the transcription and translation of my recordings.

\section*{Abbreviations}

\begin{tabularx}{.55\textwidth}{@{}lQ@{}}
\textsc{1, 2, 3} & 1st, 2nd, 3rd person \\
\ili{CA} & \ili{Classical} \ili{Arabic}\\
\textsc{cop} & \isi{copula}\\
\textsc{dat} & dative\\
\textsc{def} & \isi{definite}\\
\textsc{dem} & demonstrative\\
\textsc{dm} & discourse marker\\
\textsc{ez} & \ili{Persian} \textit{ezāfe}\\
\textsc{f} & feminine\\
\textsc{imp} & imperative\\
\textsc{impf} & imperfect (prefix conjugation)\\
\textsc{ind} & indicative\\
\textsc{inf} & \isi{infinitive}\\
\end{tabularx}%
\begin{tabularx}{.5\textwidth}{@{}lQ@{}}
KhA & \ili{Khuzestan} \ili{Arabic}\\
\textsc{m} & masculine\\
\textsc{obj} & object\\
\textsc{obl} & oblique\\
Pers. & \ili{Persian}\\
\textsc{pl}/pl. & plural\\
\textsc{ptcp} & \isi{participle}\\
\textsc{prf} & perfect(suffix conjugation)\\
\textsc{prog} & progressive\\
\textsc{prs} & present\\
\textsc{pst} & past\\
\textsc{rel} & \isi{relative} particle\\
\textsc{sg}/sg. & singular\\
\end{tabularx}%




\sloppy\printbibliography[heading=subbibliography,notkeyword=this]
\end{document}