\documentclass[output=paper]{langsci/langscibook} 

% \chapterDOI{} %will be filled in at production

\title{Anatolian Arabic}  

\author{Faruk Akkuş\affiliation{University of Pennsylvania}}

\abstract{This chapter investigates contact-induced changes in Anatolian Arabic varieties. The study first gives an overview of the current state and historical development of Anatolian Arabic. This is followed by a survey of changes Anatolian Arabic varieties have undergone as a result of language contact with primarily Turkish and Kurdish. The chapter demonstrates that the extent of the change varies from one dialect to another, and that this closely correlates with the degree of contact a dialect has had with the surrounding languages.}



\begin{document}
\maketitle

\section{Current state and historical development} 
     
Anatolian Arabic is part of the so-called \textit{qəltu}-dialect branch of the larger Mesopotamian Arabic, and essentially refers to the Arabic dialects spoken in eastern Turkey.\footnote{This group represents an older linguistic stratum of Mesopotamia as compared to the \textit{gələt} dialects. The terms \textit{qəltu} vs. \textit{gələt} dialects are due to \cite{Blanc1964}, who distinguished between the Arabic dialects spoken by three religious communities,
Muslim, Jewish, and Christian, in Baghdad. He classified the Jewish and
Christian dialects as \textit{qəltu} dialects and the Muslim dialect as a \textit{gələt} dialect, on the basis of their respective reflexes of Classical Arabic \textit{qultu} `I said'.} In three provinces of Turkey -- Hatay, Mersin and Adana -- Syrian sedentary Arabic is spoken (see Proch\'{a}zka, this volume, for discussion of these dialects).\ia{Procházka, Stephan@Procházka, Stephan} Other than these dialects, in Jastrow's (\citeyear{Jastrow1978}) classification of Mesopotamian \textit{qəltu} dialects, Anatolian Arabic dialects are subdivided into five groups:
Diyarbak{\i}r dialects (spoken by a Jewish and Christian minority, now almost extinct); Mardin dialects; Siirt dialects; Kozluk dialects; and Sason dialects.
In his later work, \citet{Jastrow2011anatolian} classifies Kozluk and Sason dialects under one group along with Mu\c{s} dialects -- investigated primarily by Talay (\citeyear{Talay2001,Talay2002}). The two larger cities where Arabic is spoken are Mardin and Siirt, although in the latter Arabic is gradually being replaced by Turkish. 

The linguistic differences between these various Arabic-speaking groups are quite considerable. Thus, given the low degree of mutual intelligibility, speakers of different varieties resort to the official language, Turkish, to communicate. \citet{Jastrow2006} reports an anecdote, wherein high school students from Mardin and Siirt converse in Turkish, since they find it difficult to understand each other's dialects. Expectedly, mutual intelligibility is at a considerably higher level among different varieties of a single group, despite certain differences. For instance,  speakers of Kozluk and Mu\c{s} Arabic have no difficulty in communicating with one another in Arabic. 

The existence of Anatolian dialects closely relates to the question of the Arabicization of the greater Mesopotamian area. Although the details largely remain obscure, a commonly-held view is that it took place in two stages: the first stage concerns the emergence of urban varieties of Arabic around the military centers, such as Ba\d{s}ra or K\={u}fa, during the early Arab conquests. Later, the migration of Bedouin dialects of tribes added another layer to the urban dialects (see e.g. \citealt{Blanc1964,Versteegh1997,Jastrow2006} for discussion). According to \cite{Blanc1964}, the \textit{qəltu} dialects are a continuation of the medieval vernaculars that were spoken in the sedentary centers of ʿAbb\={a}sid Iraq. \cite{Blanc1964} also noted that the \textit{qəltu} dialects did not stop at the Iraqi--Turkish border, but in fact continued into Turkish territory. He mentioned the towns of Mardin and Siirt as places where \textit{qəltu} dialects were still spoken.

Despite being a continuation of Mesopotamian dialects, Anatolian dialects of Arabic have been cut off from the mainstream of Arabic dialects. How exactly this cut-off and separation between dialects happened, given the lack of specific barriers, is largely unknown and remains at a speculative level. Regarding this topic, Proch\'{a}zka (this volume)\ia{Procházka, Stephan@Procházka, Stephan} suggests ``the foundation of nation states after World War One entailed significant decrease in contact between the different dialect groups and an almost complete isolation of the Arabic dialects spoken in Turkey''.

Like Central Asian Arabic and Cypriot Maronite Arabic (Walter, this volume),\ia{Walter, Mary Ann@Walter, Mary Ann} Anatolian Arabic dialects are characterized by: (i) separation from the Arabic-speaking world; (ii) contact with regional languages, which has affected them strongly; and (iii) multilingualism of speakers. %These properties extend to 

The Anatolian dialects have diverged much more from the Standard type of Arabic compared to the other \textit{qəltu} dialects, such as the Tigris or Euphrates groups (\citealt{Jastrow2011iraq}). One of the hallmarks of Anatolian Arabic is the suffix \textit{-n} instead of \textit{-m} in the second and third person plural (e.g. in Mardin Arabic \textit{baytkən} `your (\textsc{pl}) house', \textit{baytən} `their house') and the negation \textit{m\={o}} with the imperfect. In addition to many interesting properties like the ones just mentioned, Anatolian Arabic has acquired a large number of interesting contact-induced patterns.

These dialects are spoken as minority languages by speakers belonging to different ethnic or religious groups. As noted by \cite{Jastrow2006}, not all of the Anatolian Arabic varieties are spoken \textit{in situ} however, and in fact some may no longer be spoken at all. Jastrow notes that some of the dialects were exclusively spoken by Christians and almost died out during World War One as a result of the massacres of the Armenians and other Christian groups. A few thousand speakers of these dialects survive to this day, most of whom have migrated to big cities, starting from the mid-1980s, particularly Istanbul. Some speakers of these dialects also live in Europe. Nevertheless, these dialects are very likely to face extinction in a few decades. 

The Jews who spoke Anatolian Arabic varieties (mainly in Diyarbak{\i}r, but also in Urfa and Siverek; cf. \citealt{Nevo1999}) migrated to Israel after the foundation of the State of Israel in 1948. These dialects also face a serious threat of extinction. 

Today Anatolian Arabic dialects are predominantly spoken by Muslims (although there are a few hundred Arabic-speaking Christians, particularly in some parts of Istanbul, such as Samatya). These dialects are still found \textit{in situ}, however they are also subject to constant linguistic pressure from Turkish (the official language) and Kurdish (the dominant regional Indo-Iranian language), and social pressure to assimilate. The quote from Grigore (\citeyear[27]{Grigore2007book}) summarizes the overall context of Anatolian Arabic: ``il se situe dans un microcontexte kurde, situé à son tour dans un macrocontexte turc, étant isolé de la sorte de la grande masse des dialectes arabes contemporains.''\footnote{``It is situated in a Kurdish microcontext, which is in turn situated in a Turkish macrocontext, thus being isolated from the vast majority of contemporary Arabic dialects''.} 

The total number of speakers is around 620,000 (\citealt[162]{Procházka2018Anatolian}), most of whom are bi- or trilingual in
Arabic, Kurdish and Turkish. As Jastrow (\citeyear[88]{Jastrow2011anatolian}) points out, the phenomenon of diglossia is not observed in Anatolia; instead Turkish occupies the position of the `High variety', and
Anatolian Arabic, the `Low variety', occupies a purely dialectal position. In addition, speakers of
different dialects may speak other minority languages as well. For instance, a considerable number of Sason Arabic speakers know the local variety of the Iranian language Zazaki, and those of Armenian origin speak an Armenian dialect.

Anatolian Arabic varieties are in decline among the speakers of these varieties, and public life is dominated primarily by Turkish (and Kurdish). The presence of Arabic in Turkey has increased due to Syrian refugees who fled to Turkey, yet this increased presence primarily concerns Syrian Arabic, rather than Anatolian Arabic (see Proch\'{a}zka, this volume).\ia{Procházka, Stephan@Procházka, Stephan} In addition to the absence of awareness about Anatolian Arabic dialects in the Arab states, the Anatolian dialects also suffer from a more general lack of interest. The speakers generally do not attribute any prestige to their languages, calling it ``broken Arabic'', and often making little effort to pass it on to the next generations. It should, however, be noted that there has been increasing interest in these dialects in recent years, especially at the academic level. To this end, several workshops have been organized at universities in the relevant regions, aimed at promoting these dialects and discussing possible strategies for their preservation.  

The data referenced in this chapter come from various Anatolian Arabic dialects. The name of each variety and its source(s) are as follows: \={A}zəḫ \citep{Wittrich2001}; Darag\"{o}z\"{u} \citep{Jastrow1973}; \d{H}apəs \citep{Talay2007}; Hask\"{o}y \citep{Talay2001,Talay2002}; Kinderib \citep{Jastrow1978};  Mardin  \citep{Jastrow2006,Grigore2007article,GrigoreBituna2012}; Mutki-Sason \citep{Akkus2016,Akkus2017,Isaksson2005}; Siirt \citep{Bituna2016,GrigoreBituna2012}; Tillo \citep{Lahdo2009}.

\section{Contact languages}

\subsection{Overview}
Anatolia, especially the (south)eastern part, has been home to many distinct linguistic groups (as well as ethnic and religious groups). Up until the beginning of the twentieth century, speakers of the largest Anatolian languages -- Kurdish, Zazaki, Armenian, Aramaic and Arabic -- had been co-existing for almost a thousand years. This has naturally resulted in extensive contact among these languages.  

Contact influence on Anatolian Arabic has arisen mainly through long-term bi- and multi-lingualism rather than through language shift (in which speakers of other languages shifted to Arabic; \citealt{Thomason2001}).\footnote{But note also the case of the Mhallamiye near Midyat, who most likely were Aramaic speakers and shifted to Arabic after adopting Islam as their religion (thanks to Stephan Procházka for bringing this to my attention).} As a result, when applicable, the changes seem to be primarily through borrowing, rather than imposition (in the sense of \citealt{VanCoetsem1988,VanCoetsem2000}). 

\subsection{Turkish}
Turkish, as the official language of Turkey, currently dominates public life in most Arabic-speaking areas. However, as noted by Haig (\citeyear[14]{Haig2014}), ``the current omnipresent influence of Turkish in the region is in fact a relatively recent phenomenon,
fueled by compulsory Turkish-language state education, the mass-media, and large-scale military operations carried out by the Turkish army in the conflict against militant Kurdish groups. But prior to the twentieth century, the influence of Turkish in many parts of rural east Anatolia was negligible.''

Although Turkish is the dominant language in the public sphere, there are still many people, particularly in rural parts of (south-)eastern Turkey, who do not speak Turkish, including speakers of Anatolian Arabic varieties. It is usually women over forty years old that fall into this category. They tend to speak the local Arabic variety along with the dominant language in that geographic area. 

Moreover, the amount of Turkish influence is greater on the Arabic speakers who have migrated to bigger cities such as Istanbul, compared to those who still speak their dialects \textit{in situ}.

\subsection{Kurdish and Zazaki}
Anatolian Arabic has been in intensive contact with two western Iranian languages: Kurmanji Kurdish and Zazaki. These languages have influenced each other on different levels. As noted by Proch\'{a}zka (this volume)\ia{Procházka, Stephan@Procházka, Stephan} and Öpengin (this volume), Kurdish and Arabic, including the region of south-eastern Anatolia, have experienced extensive contact since at least the tenth century.\ia{Öpengin, Ergin@Öpengin, Ergin}


Due to the multi-ethnic (and to a lesser extent multi-religious) nature of the regions, bilingualism between Arabic and Kurdish (or Zazaki) is very widespread. The speakers of the non-dominant languages tend to have a stronger command of the dominant languages than the reverse situation. For instance, in Mutki, Bitlis province, where Kurdish is the dominant regional language, Arabic speakers have a native-like command of Kurdish, whereas not many Kurdish speakers speak the local Arabic variety. In some parts of Sason, Batman province, on the other hand, Arabic is the dominant language, and Kurdish speakers learn Arabic as a second language. 

\subsection{Aramaic}
Aramaic and Arabic have for centuries lived side by side, so that it is possible to speak of both substrates (from Syriac/Neo-Aramaic to Arabic), and of adstrates, or rather, of superstrates (from Arabic to Aramaic). In the context of Anatolian Arabic, Aramaic has been in contact mainly with the Mardin dialect group. 

These two languages have influenced each other in many ways. For instance, the many dialects constituting Modern Eastern Aramaic show considerable diversity as to choice of verbal particles. Some dialects use particles similar in form and function to those of the \textit{qəltu}-dialects (see e.g. \citealt{Jastrow1985}; as well as Coghill, this volume, for North-Eastern Neo-Aramaic dialects).\ia{Coghill, Eleanor@Coghill, Eleanor}

Finally, it is worth mentioning that, given the existence of Arabic speakers of Armenian origin, Armenian might have influenced certain Anatolian Arabic varieties. However, the influence of Armenian is hardly known, apart from the fact that many villages in the further eastern part of Anatolia, in which Arabic was spoken or is still spoken, bear Armenian names. This requires further investigation in its own right.

\section{Contact-induced changes in Anatolian Arabic}

Anatolian Arabic dialects manifest considerable variation, and have also come to exhibit interesting patterns due to language contact in every linguistic aspect. This section surveys these changes and features in turn.

\subsection{Phonology}

Anatolian Arabic has undergone significant changes in its consonant and vowel inventories due to language contact (as well as language internal developments). These changes include the introduction of new consonantal phonemes, loss or weakening of emphatic consonants, and introduction of new vowels. In addition to these changes, it is possible to count word-final devoicing as a contact-induced change.

This section first introduces the consonant inventory in varieties of Anatolian Arabic. It should be noted that not all consonants are present in every variety, but the chart serves as the sum of consonants available across Anatolian Arabic varieties. For instance, the phonology of Sason Arabic (and other varieties of the Kozluk--Sason--Mu\c{s} group) is characterized by the (near) absence of pharyngeal and emphatic (pharyngealized) consonants,\footnote{These sounds, whose emphatic quality is indicated in Table \ref{tab:1:consonants} and throughout with a subscript dot are only nearly absent for two reasons: (i) it is possible to detect them in the speech of elderly speakers in some lexical items, while the younger generations have lost them, (ii) \cite{Talay2001} reports their availability in Hask\"{o}y, Mu\c{s} province to a certain extent.} which have fused with their plain counterparts, e.g. \textit{pasal} `onions' in
Sason $<$ Old Arabic (OA) \textit{baṣal}.\footnote{Compare Cypriot Maronite Arabic (Walter, this volume), Maltese (Lucas \& Čéplö, this volume) and Nigerian Arabic (Owens, this volume).}\ia{Walter, Mary Ann@Walter, Mary Ann}\ia{Lucas, Christopher@Lucas, Christopher}\ia{Čéplö, Slavomír@Čéplö, Slavomír}\ia{Owens, Jonathan@Owens, Jonathan}


\begin{table} 
\begin{tabularx}{\textwidth}{ l X X X X X X X X X }
\lsptoprule
& \rotatebox{66}{Labial} & \rotatebox{66}{Interdental} & \rotatebox{66}{Dental} & \rotatebox{66}{Postalveolar} & \rotatebox{66}{Palatal} & \rotatebox{66}{Velar} & \rotatebox{66}{Uvular} & \rotatebox{66}{Pharyngeal} & \rotatebox{66}{Glottal} \\\midrule
Plosive
& p & & t \d{t}  &  & & k & q & & (ʔ)\\
& b & & d \d{d} &  & & ɡ &  & & \\
Affricate
& & & & \v{c}  &  & & & & \\
& & & &  \v{g} &  & & & & \\
Fricative
& f & θ  & s \d{s} & \v{s}  &  & ḫ & & \d{h} ʕ & h\\
& v&ð  \d{ð} %\b{d}\hspace{0em}\raisebox{-.3ex}{\d{}{\hbox{}}}     \b{ð}\hspace{0em}\raisebox{-.3ex}{\d{}{\hbox{}}} 
& z &  \v{z} &   & ɣ & & & \\
Nasal
& m & & n &  & & & & & \\
Vibrant
& & & r \d{r} &  & & & & & \\
Lateral 
&  && l & & & &  & & \\
Approximant
& w & & & & y & & & & \\\lspbottomrule
\end{tabularx}
\caption{Inventory of consonants. Marginal or doubtful phonemes within parentheses}
\label{tab:1:consonants}
\end{table}

Table \ref{tab:1:consonants}, with information largely taken from \cite{Jastrow2011anatolian}, demonstrates that Anatolian Arabic has several consonants that were originally alien to Arabic (see §\ref{newsounds} for discussion). With respect to the inventory of vowels, the noteworthy development is the introduction of /\={e}/ and /\={o}/ for some lexical items. Note that the Old Arabic diphthongs *ay and *aw have largely been preserved in these varieties: Jastrow (\citeyear[89]{Jastrow2011anatolian}) notes that one of the processes by means of which these mid long vowels entered the inventory of Anatolian Arabic is via loanwords from Turkish and Kurdish, e.g. commonly used items, \textit{\v{c}\={o}l} `desert', \textit{t\={e}l} `wire' (Turkish, probably through the intermediary of Kurdish), \textit{ḫ\={o}rt} `young man' (Kurdish).  


\subsubsection{New phonemes /p, \v{c}, ž, g, v/}\label{newsounds}
The Anatolian Arabic varieties, as well as the varieties in (northern) Syria and Iraq, have certain phonemes that were not originally familiar to these varieties of Arabic. These phonemes include the voiceless bilabial stop /p/, the voiceless affricate /\v{c}/, the voiced post-alveolar fricative /ž/,\footnote{Cf. \cite{Jastrow2011anatolian} and \cite{GrigoreBituna2012} regarding the status of /ž/: this sound is largely restricted to borrowed words. The reflex of Arabic 〈{\arabscript{ج}}〉 in Anatolian Arabic is /\v{g}/.} the voiced velar stop /g/, and the voiced labiodental fricative /v/.\footnote{Blanc (\citeyear[6--7]{Blanc1964}) considers /p/ and /\v{c}/ as characteristic of Mesopotamian varieties.} The emergence of these phonemes is most likely due to the massive contact with Turkish, Kurdish and Aramaic. That is, the most likely scenario is that the centuries-long borrowing of words which contained these sounds ultimately resulted in them getting incorporated into the phonemic inventory. 


With regard to /v/, it is likely that there are two paths of emergence: (i) as an internal evolution of the voiced interdental fricative /ð/ and (ii) via loan-words from Turkish and Kurdish. The forms \textit{vīp} and \textit{zīp} `wolf' (cf. OA \textit{ðiʔb} `wolf') represent a language internal development, whereby the interdental fricatives have shifted to sibilants in Kozluk-Sason-Mu\c{s}, and to labiodental fricatives in \={A}zəx (\c{S}{\i}rnak province, \citealt{Wittrich2001}), whereas they have been retained in most Mardin group dialects.\footnote{For more discussion, see \citet{Wittrich2001}, \citet{Jastrow2011anatolian}, \citet{Grigore2007article}, \citet{Talay2011}, \citet{Akkus2017}, and \cite{Bituna2016} among others. } 

In many cases, it is impossible to pinpoint which language these sounds were (initially) borrowed from. However, as also noted in Proch\'{a}zka (this volume),\ia{Procházka, Stephan@Procházka, Stephan} /p/ was probably introduced via contact with Kurdish, followed by influence from Ottoman and modern Turkish.\footnote{For further illustrations and discussion, see e.g. \citet{VockeWaldner1982}, \citet{Jastrow2011anatolian}, Talay (\citeyear{Talay2002,Talay2007}) and \citet{GrigoreBituna2012}.} Some illustrations are as follows:

\ea 
\noindent \textit{p\={i}s} `dirty', cf. Kurdish/Turkish \textit{p\^{i}s, pis}\\
\textit{par\v{c}\={a}ye} `piece', cf. Turkish \textit{par\c{c}a}\\
\textit{p\={u}z} `nose' (\d{H}apəs), cf. Kurdish \textit{poz}\\ 
\textit{davare} `ramp', cf. Kurdish \textit{dever} fem. `place'\\
\textit{\v{c}uv\={a}le} `sack', cf. Turkish \textit{\c{c}uval} \\
\textit{pēlāv} (\d{H}ask\"{o}y) `shoe', cf. Kurdish \textit{p\^{e}lav}\\
\textit{\v{c}\={a}y} `tea', cf. Turkish \textit{\c{c}ay}\\
\textit{\v{c}aqm\={a}q} `lighter', cf. Turkish \textit{\c{c}akmak}\\
\textit{rēnčbarī} (Hasköy), \textit{rē\v{z}barī} (Sason) `husbandry', cf. Kurdish \textit{rêncberî}\\
\textit{žīžo} (Āzəḫ) `hedgehog', cf. Kurdish \textit{jîjo} \\
\textit{ṭāži} `greyhound', cf. Kurdish \textit{tajî} \\
\textit{g\={o}mlak} `shirt', cf. Turkish \textit{g\"{o}mlek}\\
\textit{magz\={u}n}, \textit{mazg\={u}n} (in Sason) `sickle', cf. Syriac \textit{magz\={u}n\={a}}; \d{T}uroyo \textit{magz\={u}no}\\
\z

\cite{Talay2007} suggests that the loss of the phonemic status of the emphatic consonants and the weakness of the pharyngeal in Kozluk--Sason--Mu\c{s} group is likely due to the influence of Turkish, which does not have them. Examples are from the Hask\"{o}y dialect, and are taken from Talay (\citeyear{Talay2007}: 181):

\ea
\noindent \textit{ata} `he gave' ($<$ *ʔaʕ\d{t}ā), cf. \textit{ad\={a}} in Sason\\
\textit{sēbi} `boy' ($<$ *\d{s}abiyy)\\
\textit{zarab} `he hit' ($<$ *\d{ð}arab $<$ *\d{d}arab)\\
\z

Thus, changes of this kind can be seen as a quasi-adaptation of the consonant inventory to that of the superstrate and adstrate languages.

\subsubsection{Word-final devoicing}

Certain voiced stops in Anatolian Arabic /b, d, ǧ, g/ have a tendency to become devoiced [p, t, č, k] when they occur word-finally, probably due to Turkish influence, which is well-known for this property.

For instance, /b/ is mainly realized as the voiceless [p] in final pre-pausal position, e.g.: \textit{anep} `grape(s)',
cf. OA \textit{ʕinab}; \textit{ɣarīp} `stranger', cf. OA \textit{ɣarīb}. This might reflect a change in progress, as \cite{Lahdo2009} points out that the incidence of devoicing in other Anatolian dialects is also increasing over time. Note that the devoicing process does not take place in all instances, supporting the claim that the language is
undergoing a transition in this regard. Moreover, the lack of a written form removes a possible brake on this process. Further illustrations are as follows:

\ea
\noindent \textit{axa[θ]} `he took’\\
\textit{kata[p]} `he wrote’ (Mardin; \citealt[90]{Jastrow2011anatolian})\\
\textit{kt\={e}[p]} `book' (Mardin), cf. OA \textit{kit\={a}b}\\
\textit{baʕī[t]} `far' (\={A}zəḫ), cf. OA \textit{baʕ\={i}d}\\
\textit{a\d{t}ya[p]} `nicer' (Tillo), cf. OA \textit{ʔa\d{t}yab} \\
\textit{azya[t]} `more', cf. OA \textit{ʔazyad} (\citealt[106]{Lahdo2009})\\
\z

\noindent Devoicing is not limited to word-final position, however, but is also attested before voiceless consonants, e.g. \textit{haps} `prison', cf. OA \textit{ḥabs}. %Other illustrations are \textit{təšreb} `you (\textsc{sg.m}) drink', \textit{šrəpt} `you (\textsc{sg.m}) drank' or \textit{təmseg} `you (\textsc{sg.m}) catch', \textit{masakt} `you (\textsc{sg.m}) caught', cf. OA \textit{tamsiku} `you (\textsc{sg.m}) caught'.

\subsection{Morphology}
The influence of language contact is also observable in the domain of morphology.  For example, as discussed by Prochazka (\citeyear{Procházka2018Anatolian}: 182--183),  the numerals 11--19  in the Kozluk--Sason region show inversion of the unit and decimal positions, e.g. \textit{ʕašṛa sətte} (and not \textit{sətt ʕašra}) `sixteen'. See also Procházka (this volume) for discussion of the personal pronouns.

Some other cases of contact-induced changes such as reduplication, degree in adjectives and compounds are discussed below.



\subsubsection{Reduplication}

A type of reduplication due to contact with Turkish produces doublets with /m/. The consonant /m/ may be added initially to vowel-initial words, as in \REF{redup1}, or replaces the initial consonants in consonant-initial words, as in \REF{redup2} (see \citealt{Akkus2017,Lahdo2009}). The reduplication conveys vagueness, with a meaning paraphrasable with `et cetera' or `something like that'.

\ea\label{redup} Sason Arabic
		\ea \label{redup1}\gll a\v{z}\={i}n m-a\v{z}\={i}n\\
        dough m-dough\\
        \glt `dough or something like that'


	\ex \label{redup2} \gll h\={a}s m-\={a}s l\={a} təso \\
        sound m-sound \textsc{neg} make.\textsc{impf.2pl}\\
        \glt `Don't make any noise!' (Lit. `Don't make sound or something like that.') %\hfill (Sason Arabic)
\z
\z

\noindent Following the same restriction in Turkish, if a word starts with /m/, this type of reduplication is disallowed, e.g. \textit{m\={a}se} `table' cannot be reduplicated in a way that would result in \textit{m\={a}se} \textit{m\={a}se}. 


\subsubsection{Degree in adjectives} 
Adjectives in Anatolian Arabic follow the noun directly, agreeing with it in gender, number, and definiteness. In this respect, the situation is similar to most Arabic varieties. Degree, on the other hand, is not an inflectional category in Sason Arabic. Instead, this dialect has adopted the Turkish adverbs \textit{daha} `more' and \textit{en} `most' for comparative and superlative, respectively. Both these items precede the adjectival constituent, as shown in \REF{adj1} and \REF{adj2}.

\ea \label{adj} Sason Arabic 
		\ea \label{adj1}\gll mənn-i daha koys-e ye\\
   from-\textsc{obl.1sg} more beautiful-\textsc{f} \textsc{cop.3sg} \\
        \glt `She is more beautiful than me.'


	\ex \label{adj2} \gll en gbīr  \\
        most big\\
        \glt `the biggest'
\z
\z

\noindent The Tillo variety also uses the Turkish-derived \textit{an} `most' in superlative forms, with both Arabic-derived adjectives (in the elative form) and Turkish-derived adjectives (which lack an elative form), as in \REF{adjt1} and \REF{adjt1} respectively.

\newpage
\ea \label{adjt}\langinfo{Tillo Arabic}{}{\citealt[198]{Lahdo2009}}
		\ea \label{adjt1}\gll an\footnotemark	 {} aṭyap\\
    most	delicious.\textsc{ela} \\
        \glt `the most delicious'


	\ex \label{adjt2} \gll an 	yāqən  \\
        most	close\\
        \glt `the closest'
\z
\z
\footnotetext{\cite{Lahdo2009} describes this vowel as ``short front-to-back unrounded'' in Tillo. }

\noindent  On the other hand, the comparative in the Tillo variety is formed through the elative alone (which functions in other Arabic varieties as both comparative and superlative). The standard of comparison is introduced by the preposition \textit{mən} `from'. 

\ea\label{adjt3} \langinfo{Tillo Arabic}{}{\citealt[162]{Lahdo2009}}\\
\gll təllo 	iyy 		aṭyap mən 	ə\d{s}ṭanbūl\\  
     Tillo	\textsc{cop.3sg}	good.\textsc{ela}	than	Istanbul \\ 
\glt `Tillo is better than Istanbul.'
\z

%This superlative form in Sason and Tillo is most likely due to contact with Turkish. 

\subsubsection{Derivational affixes}

Through numerous loanwords, a few derivational suffixes have been introduced into Anatolian Arabic. These suffixes include the agentive morpheme \textit{-\v{g}i/-\v{c}i}, and the abessive suffix \textit{-səz}, which translates as `without'. Ingham (\citeyear[178]{Ingham2011afg}) points out that these suffixes, especially the former, are also found in the dialects of Iraq, Syria, and elsewhere (see also \citealt{Procházka-Eisl2018} for further details).

\ea
\ea Sason Arabic 
\\ \gll gahwa-\v{g}i \\
coffee-\textsc{agt}\\
\glt `coffee maker'

\ex Sason Arabic \\
\gll vi\v{g}dan-səz \\
conscience-\textsc{abess}\\
\glt `unconscientious'
\newpage
\ex Tillo Arabic (\citealt[199]{Lahdo2009})\\
\gll kəlla k\={a}nu m\d{t}ahhər-\v{c}i-yye \\
all be.\textsc{prf.3pl} circumcizer-\textsc{agt-pl}\\
\glt  `They all were circumcizers.'
\z
\z


\noindent The presence of these suffixes on lexemes of the local Arabic varieties, e.g. \textit{ḫ\={a}ser-\v{g}i} `yogurt maker, yogurt seller' (Sason Arabic) or \textit{m\d{t}ahhər-\v{c}i} `circumsiser' (Tillo Arabic), suggests that the forms above are not necessarily adopted as a whole. Rather, Arabic speakers may decompose the word and apply the suffix to other lexemes in some cases.

\subsubsection{Compounds}

Anatolian Arabic has borrowed the N+N compounding strategy from Turkish, where the right-hand member carries the compound linker morpheme \textit{-i}. This pattern is not generally found in other varieties of Arabic and it is most likely due to contact with Turkish. This type of compound is  often used with whole Turkish phrases. The examples are as follows (note that the buffer consonant \textit{-s} appears between the linker morpheme and the noun when the noun ends in a vowel):

\ea \label{coms} Sason Arabic (\citealt[41]{AkkusBenmamoun2018})
		\ea \label{com1}\gll lisa		mudur-i\\
    high\_school director-\textsc{link} \\
        \glt `high school director'


	\ex \label{com2} \gll qurs 	oratman-i  \\
        course	teacher-\textsc{link}\\
        \glt `course teacher'
\z
\z
        
      \ea \label{comt} Tillo Arabic (\citealt[199]{Lahdo2009})\\
      \gll fəstəq fabriqa-si \\
        pistachio factory-\textsc{link}\\
        \glt `pistachio factory' 
\z

\noindent This compounding strategy is found in other Arabic varieties spoken in Turkey as well, for instance, \textit{buz dolab-i} `refrigerator' (lit. `ice cupboard-\textsc{link}') in the Adana dialect. Whether compounding has been borrowed as a productive process as opposed to borrowing of the whole phrase requires further investigation.\footnote{Thanks to Stephan Proch\'{a}zka for the discussion and the example from the Adana dialect.}



\subsubsection{Vocative ending -\textit{o}}
Another morphological feature that Anatolian Arabic has acquired is the vocative particle -\textit{o}. When addressing a person directly, \textit{-o} is commonly affixed to kinship terms and given names. This appears to be available in the whole area. Unlike the situation in Syria and Iraq (see Proch\'{a}zka, this volume), this form of address is not usually used hypocoristically. Some examples are below:\ia{Procházka, Stephan@Procházka, Stephan}

\ea
\noindent \textit{amm-o} `(paternal) uncle!'\\
\textit{ǧem\={a}l-o} `Cemal!'\\
\textit{ḫāl-o} `(maternal) uncle!'\\
\z

\noindent The corresponding forms of feminine nouns end in -\textit{\={e}}, as in \textit{ḥabībt-ē} `darling!'. Grigore (\citeyear[203]{Grigore2007book}) suggests that this vocative -\textit{o} is borrowing of a morphological form from Kurdish (cf. \citealt{HaigÖpengin2018}), since the suffix, with masculine and feminine forms, is not historically available in Arabic. Note, however, the existence of cognates in other Semitic languages and \textit{-u} in the whole of North Africa, where Kurdish influence is not likely (see Prochazka, this volume).

In brief, contact with Turkish and other neighboring languages has led to various noticeable changes in the morphology of Anatolian Arabic, particularly the more easterly varieties. 

\subsection{Syntax}
Research on the syntax of Anatolian Arabic varieties, let alone work on contact-induced syntactic changes, lags significantly behind the research conducted on other aspects of these languages. Several factors might have contributed to this situation. Researchers' tendency to focus on phonological or lexical aspects and the lack of sufficient data from which to draw conclusions  are two possible factors. Another possibility that \cite{Ingham2005} raises for contact-induced syntactic change is that since the languages in contact are so typologically different, it is difficult for them to adopt syntactic features from each other without extensive language change taking place. 

This section introduces several syntactic phenomena that can be attributed to language contact, including copulas, marking of indefiniteness, light verb constructions and the periphrastic causative. Although the details are not elaborated on here, the conclusion we can arrive at is in line with \cite{Ingham2005}, in that the degree and intensity of contact with the neighboring languages leads to differences among Anatolian Arabic dialects. The more easterly varieties, e.g. the Kozluk--Sason--Mu\c{s} group, appear to be the most innovative, and the dialect group(s) most influenced by the language contact, whereas the Mardin group appears to be the most conservative (see \citealt{Akkus2017,Jastrow2011anatolian} for further discussion).

\subsubsection{Copula}
One of the most distinctive features of Anatolian Arabic is the existence of the copula in nominal sentences, based on the independent pronouns. This copula is realized as an enclitic suffix in most Anatolian dialects. Although researchers seem to differ with respect to the degree of the influence, they converge on the view that it is a matter of language contact, and that at least the development and the proliferation of the obligatory copula is under the influence of the neighboring languages -- Turkish, Kurdish, Zazaki and Aramaic -- which all have copulas in nonverbal clauses (see \citealt{Lahdo2009,Grigore2007article,Palva2011,Talay2007,Jastrow2011anatolian,Akkus2016,Akkus2017,AkkusBenmamoun2018}, for more discussion and illustrations). 

Although the copula forms themselves are not imported, the way they are used in Anatolian Arabic is exactly the same as it is in Kurdish, Turkish and Ṭuroyo (Aramaic), which have copula in the present tense. The copula is placed after the predicate (examples from \citealt{Grigore2007article}).  

\begin{exe}
\ex \label{cop} 
		\begin{xlist}
		\ex Kurdish \\\label{copkrm}\gll bav-ê min şivan-e		\\
    father-\textsc{ez} \textsc{poss.1sg}		shepherd-\textsc{cop.3sg} \\
       \glt `My father is a shepherd.'

	\ex Turkish \\
	\label{coptk} \gll baba-m 		çoban-dır\\
	father-\textsc{poss.1sg}		shepherd-\textsc{3sg}\\
        \glt  `My father is a shepherd.'
	
		\ex Ṭuroyo\\ \label{copara} \gll  bab-i		rəʕyo-yo\\
		father-\textsc{poss.1sg}		herder-\textsc{3sg}\\
      \glt  `My father is a herder.'
        \end{xlist}
      
\end{exe}

\noindent Some examples from Anatolian Arabic are illustrated below:\footnote{It should be noted that the copula is not necessarily realized as an enclitic in some dialects. For instance, in the dialect of 
Siirt (\citealt{Jastrow2011anatolian}) the copula precedes the
predicate. Moreover, the copula is identical to the personal pronoun in Siirt, whereas other Anatolian varieties use the shortened version of the pronoun in the 3\textsc{sg} and 3\textsc{pl}. See \cite{Akkus2016} for some discussion. }

\begin{exe}
\ex \label{copt} 
		\begin{xlist}
		\ex Kinderib Arabic (\citealt[131]{Jastrow1978}) \\\label{copkr}\gll malī\d{h}-we		\\
    beautiful-\textsc{3sg.m} \\
       \glt `He is beautiful.'

	\ex Sason Arabic \\
	\label{copts} \gll raḫw-īn nen\\
	sick-\textsc{pl} \textsc{3pl}\\		
        \glt  `They are sick.'
	
		\ex Daragözü Arabic (\citealt[40]{Jastrow1973})\\ \label{coparam} \gll  nā ḅāš-nā\\
		\textsc{1sg} good-\textsc{1sg}\\
      \glt  `I am good.'
        \end{xlist}
     
\end{exe}

%\noindent The paradigms from several dialects are given in \autoref{tab:1}. Some of the paradigms are not complete due to the original source (Siirt from \textcite{jastrow2006anatolian}, Cypriot Arabic from \textcite{borg1985}, Mardin data from \textcite{grigore2007} and Daragözü from \textcite{jastrow1973}) .
%The copula has the form of the independent personal pronouns or derived from them. Siirt, Mardin, Daragözü, Cyrpiot Maronite Arabic are such dialects.


%\begin{table}[H] 
%\begin{tabularx}{\textwidth}{| l || l | l | l | l |}
%\multicolumn{1}{c|}{}   & \textbf{Siirt} & \textbf{Cypriot}  & \textbf{Mardin}  &  \multicolumn{1}{c}{\textbf{Daragözü}}    \\    	\hline \hline
	%		\textbf{3m. sg.}  &    \specialcell{ūwe \textit{ūwe} awne \\ `he is here'}        &      \specialcell{áda \textit{o} xáyti \\ `this is my brother'}   &    \specialcell{hūwe gbīr \textit{we} \\ `he is big'}       &            hīyu  ... \textit{-ū}                     \\ \hline
			
	
%			\textbf{3f. sg.} &īye \textit{īye} awne          & \specialcell{l-ikníse \textit{e} maftúxa \\ `the church is open'}          &   hīya gbīre \textit{ye}          &            hīya lbayt-\textit{ī}                        \\ \hline
			
			
	%		\textbf{3pl} &    ənne \textit{ənne} awne      &    \specialcell{p-pkyára \textit{enne} maʕák \\ `the wells are deep'} 	      &    hənne gbār \textit{ənne}         &            hīyən … \textit{-ən}                       \\ \hline
			
			
	%		\textbf{2m. sg.}    & ənt \textit{ənt} awne   		%&    \textit{int}		 & 		ənt gbīr \textit{ənt} 	  &       	      	ənt məni \textit{ənt}				 \\ \hline
			
			
	%		\textbf{2f. sg.} &    ənti \textit{ənti} awne       % &    \textit{inti}      &        ənti gbīre \textit{ənti}     &                  ənte … \textit{ənte}                  \\ \hline
			
			
	%		\textbf{2pl} &        ənten \textit{ənten} awne    &    \textit{intu}      &       ənten gbār \textit{ənten}      &               ənto … ənto                      \\ \hline
			
			
	%		\textbf{1sg}    &    anā \textit{anā} awne		&    \textit{ana}		 & 	ana gbīr \textit{ana}	 	  &       	      		\specialcell{nā ḅāš \textit{nā} \\ `I am good'}			 \\ \hline
			
			
	%		\textbf{1pl}    &    nəḥne \textit{nəḥne} awne		&   \specialcell{náxni \textit{naxni} mpsallin \\ `we are educated'}  		 & 	nəḥne gbār \textit{nəḥne}	 	  &  naḥne ... \textit{naḥne}		 \\ \hline
%\end{tabularx}
%\caption{Copula in some Anatolian Arabic varieties}
%\label{tab:copula}
%\end{table}

\noindent In negative sentences as well, the same order of morphemes is attested. The negative morpheme (and the copula if there is one) follows the predicate in the neighboring languages, as the sentences in \REF{copneg} show.


\begin{exe}
\ex \label{copneg} 
		\begin{xlist}
		\ex Turkish \\\label{copnegtk}\gll hasta 	değil-ler		\\
    sick	\textsc{neg.cop-3pl} \\
       \glt `They are not sick.'
	\ex Kurdish \\\label{copnegkr}  \gll kemal	xwendekar  	nîn-e	 \\
    Kemal	student		\textsc{neg-cop.3sg}\\
    \glt    `Kemal is not a student.'
    
        
		\ex Zazaki \\ \label{copnegza} \gll  cinya 	niwaş   ni-yo\\
        child	sick	\textsc{neg-cop.3sg} \\
        \glt `The child is not sick.'
        \end{xlist}
      
\end{exe}

\noindent  The same order is found in Sason Arabic, in that the \textsc{neg+cop} follows the predicate.\footnote{This is not the most common order in Anatolian Arabic varieties, however. For more discussion, see \cite{Jastrow2011anatolian} and \cite{Akkus2016,Akkus2017}.} 
\newpage
\begin{exe}
\ex\label{copnegsa} Sason Arabic\\ 
\gll nihane me-nnen	\\
here 	\textsc{neg-cop.3pl} \\
        \glt `They are not here.'\footnote{In Sason Arabic, the 3\textsc{pl} personal pronoun can be \textit{innen} or \textit{iyen}. A shortened version of this pronoun is used both in affirmative, as in (\ref{copts}) and negative, as in (\ref{copnegsa}), non-verbal clauses. }
\end{exe}
	
\noindent Given that the copula is almost unknown in other Arabic speaking areas (but see \citealt{Blanc1964}; also Lucas \& Čéplö, this volume; Walter, this volume), it is safe to assume that the development of a full morphological paradigm for the copula along with its syntactic function is at least facilitated by contact with the neighboring languages.\ia{Walter, Mary Ann@Walter, Mary Ann}\ia{Lucas, Christopher@Lucas, Christopher}\ia{Čéplö, Slavomír@Čéplö, Slavomír}


\subsubsection{Light verb construction}

Light verb constructions are another domain where the influence of contact is clearly manifested. In surrounding languages, particularly Turkish, Kurdish and Zazaki, a light verb construction consists of a nominal part followed by the light verb, which is usually `to do' or `to be', e.g. Kurdish \textit{pacî kirin} (lit. `kiss do') `to kiss', Turkish \textit{motive etmek} (lit. `motivation do') `to motivate’. 

There are a relatively large number of compound verbs constructed with Arabic \textit{s\={a}wa} – \textit{ys\={a}wi} `to do' and a nominal borrowed from Turkish or Kurdish, as illustrated in \REF{light}. In the majority of the cases, the construction is a complete calque of its Turkish or Kurdish counterparts (see e.g. \citealt{Versteegh1997,Lahdo2009,Grigore2007article,Talay2007,Jastrow2011anatolian,Akkus2016,Akkus2017,AkkusBenmamoun2018} and \citealt{Bituna2016} for more examples).


\begin{exe}
\ex \label{light}  
		\begin{xlist}
		\ex Āzəḫ/Mardin Arabic \citep[184]{Talay2007}\\ \textit{s\={a}wa brīndār} `to injure',	cf. Kurdish \textit{brîndar kirin}\\
\textit{s\={a}wa \v{g}āmērtīye} `to act generously', cf. Kurdish \textit{camêrtî kirin}\\
\textit{s\={a}wa ɣōt} `to mow', cf. Kurdish  \textit{cot kirin}


	\ex \label{com3} Tillo Arabic \citep[202]{Lahdo2009}\\ \textit{s\={a}wa yārdəm} `to help', cf. Turkish \textit{yardım etmek}\\
\textit{ys\={a}waw dawām} `they continue …', cf.  Turkish \textit{devam etmek}\\
\textit{nsayy qa\d{h}waltə} `we have breakfast', cf. Turkish \textit{kahvalt{\i} etmek}
		\end{xlist}
		
\end{exe}


\noindent In Sason Arabic, the default order in this construction has reversed, in that in most cases the nominal is followed by the light verb. Thus, Sason manifests head-final order, undoubtedly due to contact with Turkish and Kurdish. Similarly, the nominal part of the construction can be borrowed from Turkish as in \REF{l}, including instances of reborrowing of an originally Arabic word, \REF{ltk1}, or Kurdish as in \REF{l3}. In fact, the nominal part might also be Arabic, as in \REF{lar}. 

\begin{exe}
\ex \label{l} Sason Arabic (Turkish borrowing) 
		\begin{xlist}
			\ex \label{ltk2} \gll qazan 	s\={a}wa  \\
        win	do.\textsc{prf.3sg.m}\\
        \glt `to win'
		
		\ex \label{ltk1}\gll išāret s\={a}wa \\
    sign  do.\textsc{prf.3sg.m}\\
        \glt `to sign'



		\end{xlist}
        
      \ex \label{l3} Sason Arabic (Kurdish borrowing) \\
      \gll ser asi \\
        watch do.\textsc{impf.1sg}-do\\
        \glt `I watch ...'
        
        
        \ex \label{lar} Sason Arabic \\
		\begin{xlist}
		\ex \gll gerre/h\={a}s s\={a}wa\\
    noise/sound do.\textsc{prf.3sg.m}\\
        \glt `to make noise/sound'


	\ex \label{lar2} \gll šəɣle l\={a} təsi, aməl si!  \\
      talk \textsc{neg} do.\textsc{impf.2sg.m} work do.\textsc{imp.2sg.m}\\
        \glt `Don't talk, do work!'
        
        \ex \label{lar3}\gll hu\v{g}ūm sinna \\
    attack  do.\textsc{prf.1pl}\\
        \glt `We attacked.'
		\end{xlist}
        
\end{exe}

\noindent Anatolian Arabic usually resorts to the same periphrastic construction when borrowing verbs from Turkish; it creates a complex predicate, rather than adapting a foreign verb directly to Arabic verbal morphology, a borrowing strategy seen also in the other languages in the region, such as Kurdish, Zazaki. In many cases, the complex predicate comprises of \textit{s\={a}wa} + the Turkish verbal form of the indefinite past (i.e. \textit{mi\c{s}}-verb), rather than the bare form of the verb, as illustrated in \REF{turkverb}.

\ea \label{turkverb} Anatolian Arabic \citep[184]{Talay2007}\\
\noindent \textit{sawa gačənməš} `to manage', cf. Turkish \textit{geçinmiş}\\
\textit{ba\v{s}lamə\v{s} sawa} `to begin',  cf. Turkish \textit{başlamış} \\
\z

Despite the widespread use of this process for loanwords, some borrowed verbal forms have been totally assimilated to the Arabic
verbal system; the majority of these verbs are formed according to verbal measures (stems) II or III, as can be seen in example \REF{azex}. 


\ea \label{azex} Āzəḫ (examples from \citealt{Talay2007})
\begin{tabular}{@{}lll@{}}
Stem II & \textit{qappat – īqappət} `to close' & cf. Tr. \textit{kapatmak} \\
Stem II & \textit{qayyad – īqayyəd} `to register' & cf. Tr. \textit{kayıt etmek}\\
Stem III & \textit{\d{d}āyan – ī\d{d}āyən} `to be patient, to bear up' & cf. Tr. \textit{dayanmak}\\
Stem III & \textit{tēlan} – \textit{ītēlən} `to rob' & cf. Kr. \textit{talan kirin} \\
\end{tabular}
\z


 \subsubsection{Marking of (in)definiteness}
 
 
In Classical Arabic and in modern varieties spoken in the Arab world, the indefinite noun phrase is unmarked or is preceded by an independent indefinite particle, whereas an NP becomes definite by prefixing the definite article \textit{al-/əl-/l-} etc. (\citealt{Brustad2000}). However, Kozluk--Sason--Mu\c{s} group dialects have adopted the reverse pattern (see also Uzbekistan Arabic; \citealt{Jastrow2005}), which is found in the neighboring languages Turkish and Kurdish. That is, the definite NP is left unmarked, and the enclitic \textit{-ma} is used to mark the indefiniteness of an NP (\citealt{Talay2007,Akkus2016,Akkus2017,AkinJastrowTalay2017,AkkusBenmamoun2018}), as illustrated in \REF{ma}. 

\ea \label{ma} Sason Arabic

\textit{mara} `the woman' > \textit{mara-ma} `a woman' \\
\textit{bayt} `the house' > \textit{bayt-ma} `a house'
\z

\noindent The parallel constructions in Kurdish and Turkish are illustrated in \REF{defkr} and \REF{deftk} respectively.

\begin{exe}
\ex \label{defkr} Kurdish\\	\textit{derɪ}̂ `the door' 	$>$ \textit{derɪ́-\textbf{yek}} `a door'
\ex \label{deftk} Turkish\\	\textit{kadın} `the woman' $>$ \textit{\textbf{bir} kadın} `a woman' (Turkish)
\end{exe}

 
\subsubsection{Periphrastic causative}

Sason Arabic resorts to periphrastic causative constructions rather than the root and pattern strategy found in other non-peripheral Arabic varieties. In this respect it is on a par with Kurdish, which uses the light verb \textit{bıdın} ‘to give’ to form the causative, as in \REF{perrk}. 

\begin{exe}
\ex \label{perrk}
\langinfo{Adıyaman Kurmanji Kurdish}{}{\citealt[62]{Atlamaz2012}}\\
\gll  	mı         piskilet      	   do      	çekır-ın-e	\\
        \textsc{obl.1sg} bicycle  give.\textsc{ptcp}   repair.\textsc{ptcp-ger-obl}	 \\
        \glt `I had the bicycle repaired.' (Lit: `I gave the bicycle to repairing.')
\end{exe}
										
Sason Arabic exhibits the same pattern for causative and applicative formation, as shown in \REF{persa}, which is most likely as a result of extensive contact with Kurdish.\footnote{Sason Arabic also has another periphrastic construction that is formed with the verb \textit{sa} `to do/make', which may embed a finite clause \REF{makefin} or a verbal-noun phrase \REF{makeinf}. 
\ea \label{make} \langinfo{Sason Arabic}{}{adapted from \citealt[221]{Taylan2017}}
\ea 
\label{makefin}  \gll doḫtor mə\v{s}a ali ku isi fiy-u (le yaddel) sipor \\
		doctor to Ali \textsc{cop.3sg.m} make.\textsc{impf.3sg.m} in-\textsc{3sg.m} (\textsc{comp} make.\textsc{impf.3sg.m}) sports\\\glt `The doctor is making Ali do sports.' 
	
		\ex \label{makeinf} \gll aɣa sa hazd ha\v{s}ī\v{s}\\
		headman make.\textsc{prf.3sg.m} cut.\textsc{inf} grass  \\
		\glt 	`The village headman had the grass cut.'
\z
\z

\noindent Although the origin of these constructions is not clear, they do not appear to be contact-induced.}

\begin{exe}
\ex \label{persa}
\langinfo{Sason Arabic}{}{\citealt[221]{Taylan2017}}\\
\gll  	əmm-a mə\v{s}a fatma \v{s}i adəd-u    	addil	\\
  mother-\textsc{obl.3sg.f} to Fatma food      give.\textsc{prf.3sg.f-3sg.m}	make	 \\
        \glt `Her mother made Fatma cook (Lit: Her mother gave food making to Fatma).' 
\end{exe}



\subsection{Lexicon}
Anatolian Arabic dialects have borrowed single words and whole phrases or expressions mainly from (Ottoman and modern) Turkish and Kurdish. The influence of these two languages on the Arabic lexicon is enormous. Aramaic words also survive in Anatolian Arabic to a lesser degree. A few illustrations are given in \REF{lex}.\footnote{See Vocke \& Waldner (\citeyear{VockeWaldner1982}: xxxix--li) for detailed statistics on Kurdish/Turkish/Aramaic loanwords. See also Lahdo (\citeyear{Lahdo2009}: 207--223) for a comprehensive glossary of Turkish and Kurdish loanwords in Tillo Arabic, most of which are found in other Anatolian Arabic varieties as well.} 

\ea \label{lex}
\noindent \textit{b\={o}\v{s}} `much', cf. Kurdish \textit{bo\c{s}}\\
\textit{b\={o}\v{s}qa} `different', cf. Turkish \textit{ba\c{s}ka}\\
\textit{\d{r}\={u}vi} `fox', cf. Kurdish \textit{rûvî}\\
\textit{hi\v{c}} `none, whatsoever', cf. Turkish \textit{hī\c{c}}\\
\textit{səpor} `sport', cf. Turkish \textit{spor}\\
\textit{magz\={u}n}, \textit{mazg\={u}n} (in Sason) `sickle', cf. Syriac \textit{magz\={u}n\={a}}; \d{T}uroyo \textit{magz\={u}no}\\
\z

\noindent As Jastrow (\citeyear[95]{Jastrow2011anatolian}) mentions, while more Turkish borrowings are found in bigger cities such as Mardin, Diyarbak{\i}r or Siirt, Kurdish borrowings constitute a bigger part of the lexicon of rural dialects. Anatolian Arabic dialects which have preserved the emphatics, pharyngeals or interdentals adapt borrowings into their phonology. For instance, Turkish \textit{halbuki} `however' is borrowed as \textit{\d{h}\={a}lb\={u}ki}. In most cases, the velar \textit{k} is turned into the uvular \textit{q}, e.g. \textit{\v{c}aqm\={a}q} `lighter', cf. Turkish \textit{\c{c}akmak}. Also, Kurdish feminine nouns (and even some Turkish nouns) are suffixed with the Arabic feminine morpheme \textit{-e/-a}, e.g. \textit{t\={u}re} `shoulder' (cf. Kurdish \textit{t\^{u}r}). 


There are several function words that are copied from Turkish into Arabic, e.g. Turkish \textit{ama} `but' is realized as \textit{hama} in Sason, and as \textit{a\d{m}a} in Tillo Arabic. The conjunction \textit{\c{c}\"{u}nk\"{u}} `because' from Turkish is attested in many Anatolian varieties, with the same function. Lahdo (\citeyear[179]{Lahdo2009}) notes that it expresses causal clauses in Tillo, as in (\ref{cunku}), and Biţună  (\citeyear[213]{Bituna2016}) reports the same role for Siirt. Jastrow (\citeyear[278]{Jastrow1981}) and Grigore (\citeyear[261]{Grigore2007book}) also confirm its existence in Ḥalanze and Mardin, respectively.

\begin{exe}
\ex Tillo Arabic \citep[179]{Lahdo2009}\\
\label{cunku} \gll m\={a} ʕa\d{t}aw-ni əzan \v{c}\"{u}nki \v{g}\={i}tu əl-anqara\\
\textsc{neg} give.\textsc{prf.3pl}-\textsc{1sg} permission because come.\textsc{prf.1sg} to-Ankara\\
\glt `They did not give me permission because I had come to Ankara.'
\end{exe}
 

\noindent \cite{Procházka2005} notes that particles such as \textit{b\={i}le} $<$ \textit{bile} `even', or \textit{z\={a}tan} $<$ \textit{zaten} `already' in the Adana region are also borrowed from Turkish (see also \citealt{Isaksson2005}). 




\section{Conclusion}

This chapter has dealt with contact-induced changes in the Anatolian Arabic dialects. We have seen that Anatolian Arabic has been primarily in contact with Turkish, Kurdish and Aramaic, and the influence of these neighboring languages on Anatolian Arabic is evident. We have surveyed some contact-induced changes at the phonological, morphological, syntactic and lexical level. %In particular, the influence of Kurdish (and to some extent Turkish) on Anatolian Arabic is clear in terms of changes at the syntax level.

Mardin and Siirt dialects have been covered much more comprehensively than other dialects in the literature. It is desirable to have more comprehensive investigations carried out for the dialects around the Bitlis, Mu\c{s} and Diyarbak{\i}r areas. This research has the potential to fill the gaps in our current state of knowledge about these dialects. 

Similarly, in terms of the linguistic features investigated, phonological and morphological properties (along with lexicon) have received more attention in the literature, whereas syntax, in particular, has been understudied. This situation, however, might change once we are at a point where we have enough recordings and transcriptions to investigate syntactic properties of the dialects. 

\section*{Further Reading}
 
\citet{Jastrow1978} is a seminal work, which provided the classification for Anatolian Arabic varieties. \\
\citet{Jastrow2011anatolian} is a concise, yet comprehensive encyclopedia entry on characteristic features of Anatolian Arabic.\\
\noindent \citet{Talay2011} is a good source for an overview of Arabic dialects in the Meso\-potamian region.

\section*{Acknowledgements}
I would like to thank Stephan Proch\'{a}zka and Mary Ann Walter for sharing their work, and to Stephan Proch\'{a}zka for reading another version of the paper, which also helped with this version. Thanks to Gabriel Bi\c{t}u\u{n}a for directing my attention to several important sources. I also thank the editors Chris Lucas and Stefano Manfredi for their help and patience. All remaining errors are of course my own. 

\section*{Abbreviations}
\begin{multicols}{2}
\begin{tabbing}
\textsc{ipfv} \hspace{1.5em} \= before common era\kill
\textsc{1, 2, 3} \> 1st, 2nd, 3rd person \\
I, II etc. \> 1st, 2nd etc. verbal derivation \\
\textsc{abess} \> abessive \\
\textsc{agt} \> agentive \\
\textsc{comp} \> complementizer \\
\textsc{cop} \> copula \\
CyA \> Cypriot Maronite Arabic \\
\textsc{def} \> definite article \\
\textsc{ela} \> elative degree \\
\textsc{ez} \> \textit{ezāfe} \\
\textsc{f} \> feminine \\
\textsc{ger} \> gerund \\
\textsc{impf} \> imperfect (prefix conjugation) \\
\textsc{inf} \> infinitive \\
Kr. \> Kurdish \\
\textsc{link} \> linker \\
\textsc{m} \> masculine \\
\textsc{neg} \> negation  \\
OA \> Old Arabic \\
\textsc{obl} \> oblique \\
\textsc{pl} \> plural \\
\textsc{poss} \> possessive  \\
\textsc{prf} \> perfect (suffix conjugation) \\
\textsc{ptcp} \> participle \\
\textsc{sg} \> singular \\
Tr. \> Turkish
\end{tabbing}
\end{multicols}

{\sloppy\printbibliography[heading=subbibliography,notkeyword=this]}

\end{document}
