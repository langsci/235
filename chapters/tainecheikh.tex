\documentclass[output=paper]{langsci/langscibook} 
\author{Catherine Taine-Cheikh\affiliation{CNRS, LACITO}}
\title{Ḥassāniyya Arabic}
\abstract{The area where Ḥassāniyya is spoken, located on the outskirts of the Arab world, is contiguous with those of several languages that do not belong to the \ili{Afro-Asiatic} phylum. However, the greatest influence on the evolution of Ḥassāniyya has been its contact with Berber and Classical Arabic. Loanwords from those languages are distinguished by specific features that have enriched and developed the phonological and morphological system of Ḥassāniyya. In other respects, Ḥassāniyya and Zenaga are currently in a state of either parallel evolution or reciprocal exchanges. }
\IfFileExists{../localcommands.tex}{
 % add all extra packages you need to load to this file 
\usepackage{graphicx}
\usepackage{tabularx}
\usepackage{amsmath} 
\usepackage{multicol}
\usepackage{lipsum}
\usepackage[stable]{footmisc}
\usepackage{adforn}
%%%%%%%%%%%%%%%%%%%%%%%%%%%%%%%%%%%%%%%%%%%%%%%%%%%%
%%%                                              %%%
%%%           Examples                           %%%
%%%                                              %%%
%%%%%%%%%%%%%%%%%%%%%%%%%%%%%%%%%%%%%%%%%%%%%%%%%%%%
% remove the percentage signs in the following lines
% if your book makes use of linguistic examples
\usepackage{./langsci/styles/langsci-optional} 
\usepackage{./langsci/styles/langsci-lgr}
\usepackage{morewrites} 
%% if you want the source line of examples to be in italics, uncomment the following line
% \def\exfont{\it}

\usepackage{enumitem}
\newlist{furtherreading}{description}{1}
\setlist[furtherreading]{font=\normalfont,labelsep=\widthof{~},noitemsep,align=left,leftmargin=\parindent,labelindent=0pt,labelwidth=-\parindent}
\usepackage{phonetic}
\usepackage{chronosys,tabularx}
\usepackage{csquotes}
\usepackage[stable]{footmisc} 

\usepackage{langsci-bidi}
\usepackage{./langsci/styles/langsci-gb4e} 

 \makeatletter
\let\thetitle\@title
\let\theauthor\@author 
\makeatother

\newcommand{\togglepaper}[1][0]{ 
  \bibliography{../localbibliography}
  \papernote{\scriptsize\normalfont
    \theauthor.
    \thetitle. 
    To appear in: 
    Christopher Lucas and Stefano Manfredi (eds.),  
    Arabic and contact-induced language change
    Berlin: Language Science Press. [preliminary page numbering]
  }
  \pagenumbering{roman}
  \setcounter{chapter}{#1}
  \addtocounter{chapter}{-1}
}

\newfontfamily\Parsifont[Script=Arabic]{ScheherazadeRegOT_Jazm.ttf} 
\newcommand{\arabscript}[1]{\RL{\Parsifont #1}}
\newcommand{\textarabic}[1]{{\arabicfont #1}}

\newcommand{\textstylest}[1]{{\color{red}#1}}

\patchcmd{\mkbibindexname}{\ifdefvoid{#3}{}{\MakeCapital{#3}
}}{\ifdefvoid{#3}{}{#3 }}{}{\AtEndDocument{\typeout{mkbibindexname could
not be patched.}}}

%command for italic r with dot below with horizontal correction to put the dot in the prolongation of the vertical stroke
%for some reason, the dot is larger than expected, so we explicitly reduce the font size (to \small)
%for the time being, the font is set to an absolute value. To be more robust, a relative reduction would be better, but this might not be required right now
\newcommand{\R}{r\kern-.05ex{\small{̣}}\kern.05ex}


\DeclareLabeldate{%
    \field{date}
    \field{year}
    \field{eventdate}
    \field{origdate}
    \field{urldate}
    \field{pubstate}
    \literal{nodate}
}

\renewbibmacro*{addendum+pubstate}{% Thanks to https://tex.stackexchange.com/a/154367 for the idea
  \printfield{addendum}%
  \iffieldequalstr{labeldatesource}{pubstate}{}
  {\newunit\newblock\printfield{pubstate}}
}
 
 %% hyphenation points for line breaks
%% Normally, automatic hyphenation in LaTeX is very good
%% If a word is mis-hyphenated, add it to this file
%%
%% add information to TeX file before \begin{document} with:
%% %% hyphenation points for line breaks
%% Normally, automatic hyphenation in LaTeX is very good
%% If a word is mis-hyphenated, add it to this file
%%
%% add information to TeX file before \begin{document} with:
%% %% hyphenation points for line breaks
%% Normally, automatic hyphenation in LaTeX is very good
%% If a word is mis-hyphenated, add it to this file
%%
%% add information to TeX file before \begin{document} with:
%% \include{localhyphenation}
\hyphenation{
affri-ca-te
affri-ca-tes
com-ple-ments
homo-phon-ous
start-ed
Meso-potam-ian
morpho-phono-logic-al-ly
morpho-phon-em-ic-s
Palestin-ian
re-present-ed
Ki-nubi
ḥawār-iyy-ūn
archa-ic-ity
fuel-ed
de-velop-ment
pros-od-ic
Arab-ic
in-duced
phono-logy
possess-um
possess-ive-s
templ-ate
spec-ial
espec-ial-ly
nat-ive
pass-ive
clause-s
potent-ial-ly
Lusignan
commun-ity
tobacco
posi-tion
Cushit-ic
Middle
with-in
re-finit-iz-ation
langu-age-s
langu-age
diction-ary
glossary
govern-ment
eight
counter-part
nomin-al
equi-valent
deont-ic
ana-ly-sis
Malt-ese
un-fortun-ate-ly
scient-if-ic
Catalan
Occitan
ḥammāl
cross-linguist-ic-al-ly
predic-ate
major-ity
ignor-ance
chrono-logy
south-western
mention-ed
borrow-ed
neg-ative
de-termin-er
European
under-mine
detail
Oxford
Socotra
numer-ous
spoken
villages
nomad-ic
Khuze-stan
Arama-ic
Persian
Ottoman
Ottomans
Azeri
rur-al
bi-lingual-ism
borrow-ing
prestig-ious
dia-lects
dia-lect
allo-phone
allo-phones
poss-ible
parallel
parallels
pattern
article
common-ly
respect-ive-ly
sem-antic
Moroccan
Martine
Harrassowitz
Grammatic-al-ization
grammatic-al-ization
Afro-asiatica
Afro-asiatic
continu-ation
Semit-istik
varieties
mono-phthong
mono-phthong-ized
col-loquial
pro-duct
document-ary
ex-ample-s
ex-ample
termin-ate
element-s
Aramaeo-grams
Centr-al
idioms
Arab-ic
Dadan-it-ic
sub-ordin-ator
Thamud-ic
difficult
common-ly
Revue
Bovingdon
under
century
attach
attached
bundle
graph-em-ic
graph-emes
cicada
contrast-ive
Corriente
Andalusi
Kossmann
morpho-logic-al
inter-action
dia-chroniques
islámica
occid-ent-al-ismo
dialecto-logie
Reichert
coloni-al
Milton
diphthong-al
linguist-ic
linguist-ics
affairs
differ-ent
phonetic-ally
kilo-metres
stabil-ization
develop-ments
in-vestig-ation
Jordan-ian
notice-able
level-ed
migrants
con-dition-al
certain-ly
general-ly
especial-ly
af-fric-ation
Jordan
counter-parts
com-plication
consider-ably
inter-dent-al
com-mun-ity
inter-locutors
com-pon-ent
region-al
socio-historical
society
simul-taneous
phon-em-ic
roman-ization
Classic-al
funeral
Kurmanji
pharyn-geal-ization
vocab-ulary
phon-et-ic
con-sonant
con-sonants
special-ized
latter
latters
in-itial
ident-ic-al
cor-relate
geo-graphic-al-ly
Öpengin
Kurd-ish
in-digen-ous
sunbul
Christ-ian
Christ-ians
sekin-în
fatala
in-tegration
dia-lect-al
Matras
morpho-logy
in-tens-ive
con-figur-ation
im-port-ant
com-plement
ḥaddād
e-merg-ence
Benjmamins
struct-ure
em-pir-ic-al
Orient-studien
Anatolia
American
vari-ation
Jastrow
Geoffrey
Yarshater
Ashtiany
Edmund
Mahnaz
En-cyclo-pædia
En-cyclo-paedia
En-cyclo-pedia
Leiden
dia-spora
soph-is-ic-ated
Sasan-ian
every-day
domin-ance
Con-stitu-tion-al
religi-ous
sever-al
Manfredi
re-lev-ance
re-cipi-ent
pro-duct-iv-ity
turtle
Morocco
ferman
Maghreb-ian
algérien
stand-ard
systems
Nicolaï
Mouton
mauritani-en
Gotho-burg-ensis
socio-linguist-ique
plur-al
archiv-al
Arab-ian
drop-ped
dihāt
de-velop-ed
ṣuḥbat
kitāba
kitābat
com-mercial
eight-eenth
region
Senegal
mechan-ics
Maur-itan-ia
Ḥassān-iyya
circum-cision
cor-relation
labio-velar-ization
vowel
vowels
cert-ain
īggīw
series
in-tegrates
dur-ative
inter-dent-als
gen-itive
Tuareg
tălămut
talawmāyət
part-icular
part-icular-ly
con-diment
vill-age
bord-er
polit-ical
Wiesbaden
Uni-vers-idad
Geuthner
typo-logie
Maur-itanie
nomades
Maur-itan-ian
dia-lecto-logy
Sahar-iennes
Uni-vers-ity
de-scend-ants
NENA-speak-ing
speak-ing
origin-al
re-captured
in-habit-ants
ethnic
minor-it-ies
drama-tic
local
long-stand-ing
regions
Nineveh
settle-ments
Ṣəndor
Mandate
sub-stitut-ing
ortho-graphy
re-fer-enced
origin-ate
twenti-eth
typ-ic-al-ly
Hobrack
never-the-less
character-ist-ics
character-ist-ic
masc-uline
coffee
ex-clus-ive-ly
verb-al
re-ana-ly-se-d
simil-ar-ities
de-riv-ation
im-pera-tive
part-iciple
dis-ambi-gu-ation
dis-ambi-gu-a-ing
phen-omen-on
phen-omen-a
traktar
com-mun-ity
com-mun-ities
dis-prefer-red
ex-plan-ation
con-struction
wide-spread
us-ual-ly
region-al
Bulut
con-sider-ation
afro-asia-tici
Franco-Angeli
Phono-logie
Volks-kundliche
dia-lectes
dia-lecte
select-ed
dis-appear-ance
media
under-stand-able
public-ation
second-ary
e-ject-ive
re-volu-tion
re-strict-ive
Gasparini
mount-ain
mount-ains
yellow
label-ing
trad-ition-al-ly
currently
dia-chronic
}
\hyphenation{
affri-ca-te
affri-ca-tes
com-ple-ments
homo-phon-ous
start-ed
Meso-potam-ian
morpho-phono-logic-al-ly
morpho-phon-em-ic-s
Palestin-ian
re-present-ed
Ki-nubi
ḥawār-iyy-ūn
archa-ic-ity
fuel-ed
de-velop-ment
pros-od-ic
Arab-ic
in-duced
phono-logy
possess-um
possess-ive-s
templ-ate
spec-ial
espec-ial-ly
nat-ive
pass-ive
clause-s
potent-ial-ly
Lusignan
commun-ity
tobacco
posi-tion
Cushit-ic
Middle
with-in
re-finit-iz-ation
langu-age-s
langu-age
diction-ary
glossary
govern-ment
eight
counter-part
nomin-al
equi-valent
deont-ic
ana-ly-sis
Malt-ese
un-fortun-ate-ly
scient-if-ic
Catalan
Occitan
ḥammāl
cross-linguist-ic-al-ly
predic-ate
major-ity
ignor-ance
chrono-logy
south-western
mention-ed
borrow-ed
neg-ative
de-termin-er
European
under-mine
detail
Oxford
Socotra
numer-ous
spoken
villages
nomad-ic
Khuze-stan
Arama-ic
Persian
Ottoman
Ottomans
Azeri
rur-al
bi-lingual-ism
borrow-ing
prestig-ious
dia-lects
dia-lect
allo-phone
allo-phones
poss-ible
parallel
parallels
pattern
article
common-ly
respect-ive-ly
sem-antic
Moroccan
Martine
Harrassowitz
Grammatic-al-ization
grammatic-al-ization
Afro-asiatica
Afro-asiatic
continu-ation
Semit-istik
varieties
mono-phthong
mono-phthong-ized
col-loquial
pro-duct
document-ary
ex-ample-s
ex-ample
termin-ate
element-s
Aramaeo-grams
Centr-al
idioms
Arab-ic
Dadan-it-ic
sub-ordin-ator
Thamud-ic
difficult
common-ly
Revue
Bovingdon
under
century
attach
attached
bundle
graph-em-ic
graph-emes
cicada
contrast-ive
Corriente
Andalusi
Kossmann
morpho-logic-al
inter-action
dia-chroniques
islámica
occid-ent-al-ismo
dialecto-logie
Reichert
coloni-al
Milton
diphthong-al
linguist-ic
linguist-ics
affairs
differ-ent
phonetic-ally
kilo-metres
stabil-ization
develop-ments
in-vestig-ation
Jordan-ian
notice-able
level-ed
migrants
con-dition-al
certain-ly
general-ly
especial-ly
af-fric-ation
Jordan
counter-parts
com-plication
consider-ably
inter-dent-al
com-mun-ity
inter-locutors
com-pon-ent
region-al
socio-historical
society
simul-taneous
phon-em-ic
roman-ization
Classic-al
funeral
Kurmanji
pharyn-geal-ization
vocab-ulary
phon-et-ic
con-sonant
con-sonants
special-ized
latter
latters
in-itial
ident-ic-al
cor-relate
geo-graphic-al-ly
Öpengin
Kurd-ish
in-digen-ous
sunbul
Christ-ian
Christ-ians
sekin-în
fatala
in-tegration
dia-lect-al
Matras
morpho-logy
in-tens-ive
con-figur-ation
im-port-ant
com-plement
ḥaddād
e-merg-ence
Benjmamins
struct-ure
em-pir-ic-al
Orient-studien
Anatolia
American
vari-ation
Jastrow
Geoffrey
Yarshater
Ashtiany
Edmund
Mahnaz
En-cyclo-pædia
En-cyclo-paedia
En-cyclo-pedia
Leiden
dia-spora
soph-is-ic-ated
Sasan-ian
every-day
domin-ance
Con-stitu-tion-al
religi-ous
sever-al
Manfredi
re-lev-ance
re-cipi-ent
pro-duct-iv-ity
turtle
Morocco
ferman
Maghreb-ian
algérien
stand-ard
systems
Nicolaï
Mouton
mauritani-en
Gotho-burg-ensis
socio-linguist-ique
plur-al
archiv-al
Arab-ian
drop-ped
dihāt
de-velop-ed
ṣuḥbat
kitāba
kitābat
com-mercial
eight-eenth
region
Senegal
mechan-ics
Maur-itan-ia
Ḥassān-iyya
circum-cision
cor-relation
labio-velar-ization
vowel
vowels
cert-ain
īggīw
series
in-tegrates
dur-ative
inter-dent-als
gen-itive
Tuareg
tălămut
talawmāyət
part-icular
part-icular-ly
con-diment
vill-age
bord-er
polit-ical
Wiesbaden
Uni-vers-idad
Geuthner
typo-logie
Maur-itanie
nomades
Maur-itan-ian
dia-lecto-logy
Sahar-iennes
Uni-vers-ity
de-scend-ants
NENA-speak-ing
speak-ing
origin-al
re-captured
in-habit-ants
ethnic
minor-it-ies
drama-tic
local
long-stand-ing
regions
Nineveh
settle-ments
Ṣəndor
Mandate
sub-stitut-ing
ortho-graphy
re-fer-enced
origin-ate
twenti-eth
typ-ic-al-ly
Hobrack
never-the-less
character-ist-ics
character-ist-ic
masc-uline
coffee
ex-clus-ive-ly
verb-al
re-ana-ly-se-d
simil-ar-ities
de-riv-ation
im-pera-tive
part-iciple
dis-ambi-gu-ation
dis-ambi-gu-a-ing
phen-omen-on
phen-omen-a
traktar
com-mun-ity
com-mun-ities
dis-prefer-red
ex-plan-ation
con-struction
wide-spread
us-ual-ly
region-al
Bulut
con-sider-ation
afro-asia-tici
Franco-Angeli
Phono-logie
Volks-kundliche
dia-lectes
dia-lecte
select-ed
dis-appear-ance
media
under-stand-able
public-ation
second-ary
e-ject-ive
re-volu-tion
re-strict-ive
Gasparini
mount-ain
mount-ains
yellow
label-ing
trad-ition-al-ly
currently
dia-chronic
}
\hyphenation{
affri-ca-te
affri-ca-tes
com-ple-ments
homo-phon-ous
start-ed
Meso-potam-ian
morpho-phono-logic-al-ly
morpho-phon-em-ic-s
Palestin-ian
re-present-ed
Ki-nubi
ḥawār-iyy-ūn
archa-ic-ity
fuel-ed
de-velop-ment
pros-od-ic
Arab-ic
in-duced
phono-logy
possess-um
possess-ive-s
templ-ate
spec-ial
espec-ial-ly
nat-ive
pass-ive
clause-s
potent-ial-ly
Lusignan
commun-ity
tobacco
posi-tion
Cushit-ic
Middle
with-in
re-finit-iz-ation
langu-age-s
langu-age
diction-ary
glossary
govern-ment
eight
counter-part
nomin-al
equi-valent
deont-ic
ana-ly-sis
Malt-ese
un-fortun-ate-ly
scient-if-ic
Catalan
Occitan
ḥammāl
cross-linguist-ic-al-ly
predic-ate
major-ity
ignor-ance
chrono-logy
south-western
mention-ed
borrow-ed
neg-ative
de-termin-er
European
under-mine
detail
Oxford
Socotra
numer-ous
spoken
villages
nomad-ic
Khuze-stan
Arama-ic
Persian
Ottoman
Ottomans
Azeri
rur-al
bi-lingual-ism
borrow-ing
prestig-ious
dia-lects
dia-lect
allo-phone
allo-phones
poss-ible
parallel
parallels
pattern
article
common-ly
respect-ive-ly
sem-antic
Moroccan
Martine
Harrassowitz
Grammatic-al-ization
grammatic-al-ization
Afro-asiatica
Afro-asiatic
continu-ation
Semit-istik
varieties
mono-phthong
mono-phthong-ized
col-loquial
pro-duct
document-ary
ex-ample-s
ex-ample
termin-ate
element-s
Aramaeo-grams
Centr-al
idioms
Arab-ic
Dadan-it-ic
sub-ordin-ator
Thamud-ic
difficult
common-ly
Revue
Bovingdon
under
century
attach
attached
bundle
graph-em-ic
graph-emes
cicada
contrast-ive
Corriente
Andalusi
Kossmann
morpho-logic-al
inter-action
dia-chroniques
islámica
occid-ent-al-ismo
dialecto-logie
Reichert
coloni-al
Milton
diphthong-al
linguist-ic
linguist-ics
affairs
differ-ent
phonetic-ally
kilo-metres
stabil-ization
develop-ments
in-vestig-ation
Jordan-ian
notice-able
level-ed
migrants
con-dition-al
certain-ly
general-ly
especial-ly
af-fric-ation
Jordan
counter-parts
com-plication
consider-ably
inter-dent-al
com-mun-ity
inter-locutors
com-pon-ent
region-al
socio-historical
society
simul-taneous
phon-em-ic
roman-ization
Classic-al
funeral
Kurmanji
pharyn-geal-ization
vocab-ulary
phon-et-ic
con-sonant
con-sonants
special-ized
latter
latters
in-itial
ident-ic-al
cor-relate
geo-graphic-al-ly
Öpengin
Kurd-ish
in-digen-ous
sunbul
Christ-ian
Christ-ians
sekin-în
fatala
in-tegration
dia-lect-al
Matras
morpho-logy
in-tens-ive
con-figur-ation
im-port-ant
com-plement
ḥaddād
e-merg-ence
Benjmamins
struct-ure
em-pir-ic-al
Orient-studien
Anatolia
American
vari-ation
Jastrow
Geoffrey
Yarshater
Ashtiany
Edmund
Mahnaz
En-cyclo-pædia
En-cyclo-paedia
En-cyclo-pedia
Leiden
dia-spora
soph-is-ic-ated
Sasan-ian
every-day
domin-ance
Con-stitu-tion-al
religi-ous
sever-al
Manfredi
re-lev-ance
re-cipi-ent
pro-duct-iv-ity
turtle
Morocco
ferman
Maghreb-ian
algérien
stand-ard
systems
Nicolaï
Mouton
mauritani-en
Gotho-burg-ensis
socio-linguist-ique
plur-al
archiv-al
Arab-ian
drop-ped
dihāt
de-velop-ed
ṣuḥbat
kitāba
kitābat
com-mercial
eight-eenth
region
Senegal
mechan-ics
Maur-itan-ia
Ḥassān-iyya
circum-cision
cor-relation
labio-velar-ization
vowel
vowels
cert-ain
īggīw
series
in-tegrates
dur-ative
inter-dent-als
gen-itive
Tuareg
tălămut
talawmāyət
part-icular
part-icular-ly
con-diment
vill-age
bord-er
polit-ical
Wiesbaden
Uni-vers-idad
Geuthner
typo-logie
Maur-itanie
nomades
Maur-itan-ian
dia-lecto-logy
Sahar-iennes
Uni-vers-ity
de-scend-ants
NENA-speak-ing
speak-ing
origin-al
re-captured
in-habit-ants
ethnic
minor-it-ies
drama-tic
local
long-stand-ing
regions
Nineveh
settle-ments
Ṣəndor
Mandate
sub-stitut-ing
ortho-graphy
re-fer-enced
origin-ate
twenti-eth
typ-ic-al-ly
Hobrack
never-the-less
character-ist-ics
character-ist-ic
masc-uline
coffee
ex-clus-ive-ly
verb-al
re-ana-ly-se-d
simil-ar-ities
de-riv-ation
im-pera-tive
part-iciple
dis-ambi-gu-ation
dis-ambi-gu-a-ing
phen-omen-on
phen-omen-a
traktar
com-mun-ity
com-mun-ities
dis-prefer-red
ex-plan-ation
con-struction
wide-spread
us-ual-ly
region-al
Bulut
con-sider-ation
afro-asia-tici
Franco-Angeli
Phono-logie
Volks-kundliche
dia-lectes
dia-lecte
select-ed
dis-appear-ance
media
under-stand-able
public-ation
second-ary
e-ject-ive
re-volu-tion
re-strict-ive
Gasparini
mount-ain
mount-ains
yellow
label-ing
trad-ition-al-ly
currently
dia-chronic
} 
 \togglepaper[1]%%chapternumber
}{}

\begin{document}
\maketitle 
 

\section{Current state and historical development}

\subsection{Historical development of Ḥassāniyya} %1.1. /

The arrival in Morocco of the Banī Maʕqil, travelling companions of the Banī Hilāl and Banī Sulaym, is dated to the thirteenth century. However, the gradual shift to the territories further south of one of their branches – that of the Banī Ḥassān, the origin of the name given to the dialect described here – began closer to the start of the subsequent century. 

At that time, the Sahel region of West Africa was inhabited by different communities: on the one hand there were the “white” nomadic \ili{Berber}-speaking tribes, on the other hand, the sedentary “black” communities.

Over the course of the following centuries, particularly during the seventeenth and eighteenth centuries, the sphere of \ili{Zenaga} \ili{Berber} gradually diminished, until it ceased to exist in the 1950s, other than in a few tribes in the southwest of Mauritania. At the same time, Ḥassāniyya \ili{Arabic} became the language of the nomads of the west Saharan group, maintaining a remarkable unity (\citealt{Taine-Cheikh2016,Taine-Cheikh2018historical}). There is virtually no direct documentation of the region’s linguistic history during these centuries. This absence of information itself suggests a very gradual transformation and an extended period of {bilingualism}.

Despite the lack of documentation of the {transfer} phenomenon, it seems highly likely that bilinguals played a very important role in the changes described in this chapter. 

\subsection{Current situation of Ḥassāniyya} %1.2. /

The presence of significant Ḥassāniyya-speaking communities is recognized in six countries. With the exception of Senegal and especially of Niger, the regions occupied by these communities, more or less adjacent, are situated primarily in Mauritania, in the north, northeast and east of the country. 

The greatest number of Ḥassāniyya speakers (approximately 2.8 out of a total of four million) are found in Mauritania, where they constitute the majority of the population (approximately 75\%). The Ḥassāniyya language tends to fulfil the role of the lingua franca without, however, having genuine official recognition beyond, or even equal to, that which it has acquired (often recently) in neighbouring countries.

\section{Contact languages} %2. /

\subsection{Contact with other Arabic varieties} %2.1. /

The Islamization of the Ḥassāniyya-speaking population took place at an early date, and Ḥassāniyya has therefore had lengthy exposure to \ili{Classical} \ili{Arabic}. For many centuries this contact remained superficial, however, except among the Marabout tribes, where proficiency in literary \ili{Arabic} was quite widespread and in some cases almost total. The teaching of Islamic sciences in other places reached quite exceptional levels in certain \textit{mḥāð̣ə{\R}} (a type of traditional desert university).\footnote{These may be referred to as universities both in terms of the standard of teaching and the length of students’ studies. They were, however, small-scale, local affairs, located either in nomadic encampments or in ancient caravan cities.} In the post-{colonial} era, the choice of \ili{Arabic} as official language, and the widespread {Arabization} of education, media and services, greatly increased the Ḥassāniyya-speaking population’s contact with literary \ili{Arabic} (including in its \ili{Modern Standard} form), though perfect fluency was not achieved, even among the young and educated populations. 

Excluding the limited influence of the Egyptian and {Lebanese}–{Syrian} dialects used by the media, the \ili{Arabic} dialects with which Ḥassāniyya comes into contact most often today are those of the neighbouring countries (southern \ili{Moroccan} and southern Algerian). Most recently \ili{Moroccan} koiné \ili{Arabic} has established a presence in the \ili{Western} Sahara, since the region came under the administration of Morocco.

\subsection{Contact with Berber languages} %2.2. /

Ḥassāniyya has always been in contact with \ili{Berber} languages. Currently, speakers of Ḥassāniyya are primarily in contact with \ili{Tašlḥiyt} (south Morocco), \ili{Tuareg} (Malian Sahara and the Timbuktu region) and \ili{Zenaga} (southwest Mauritania). In these areas, some speakers are bilingual in Ḥassāniyya and \ili{Berber}. 

In Mauritania, where \ili{Zenaga} previously occupied a much larger area, \ili{Berber} clearly appears as a {substrate}.

\subsection{Contact with languages of the Sahel} %2.3. /

Contacts between Ḥassāniyya speakers and the languages spoken in the Sahel have varied across regions and over time, but have left few clearly discernible traces on Ḥassāniyya.

The contact with \ili{Soninke} is ancient (cf. the toponym \textit{Chinguetti} < \ili{Soninke} \textit{sí-n-gèdé} ‘horse well’), but the effects are hardly noticeable outside of the old cities of Mauritania. The contact with \ili{Songhay} is both very old and still ongoing, but is limited to the eastern part of the region in which Ḥassāniyya is spoken (especially the region of Timbuktu).

The influence of \ili{Wolof}, albeit marginal, has always been more substantial in southwestern Mauritania, especially among the Awlād Banʸūg of the Rosso region. It peaked in the years 1950–70, in connection with the immigration to Senegal of many Moors (e.g. \textit{gordʸigen} ‘homosexual’, lit. ‘man-woman’). In Mauritania, the influence of \ili{Wolof} can still be heard in some areas of urban crafts (e.g. mechanics, electricity), but it is primarily a vehicle for borrowing from \ili{French}.

Although Pulaar speakers constitute the second-largest linguistic community of Mauritania, there is very limited contact between Ḥassāniyya and Pulaar, with the exception of a few bilingual groups (especially among the Harratins) in the Senegal River valley.

Certain communities (particularly among the Fulani) were traditionally known for their perfect mastery of Ḥassāniyya. As a result of migration into major cities and the aggressive {Arabization} policy led by the authorities, Ḥassāniyya has gained ground among all the non-\ili{Arabic} speakers of Mauritania (especially in the big cities and among younger people), but this has come at the cost of a sometimes very negative attitude towards the language. 

\subsection{Contact with Indo-European Languages} %2.4. /

Exposure to \ili{French} has prevailed in all the countries of the region, the only exception being the \ili{Western} Sahara, which, from the end of the nineteenth century until 1975, was under \ili{Spanish} occupation.

In Mauritania the French occupation came relatively late and was relatively insignificant. However, the influence of the colonizers’ language continued well after the country proclaimed its independence in 1960. That said, \ili{French} has tended to regress since the end of the twentieth century (especially with the rise of \ili{Standard} \ili{Arabic}, e.g. \textit{minəstr} has been replaced by \textit{wazīr} ‘minister’), whilst exposure to \ili{English} has become somewhat more significant, at least in the better educated sections of the population.

\section{Contact-induced changes in Ḥassāniyya} %3. /

\subsection{Phonology} %3.1. /

\subsubsection{Consonants} %3.1.1. /
\subsubsubsection{The consonant /ḍ/} %3.1.1.1. / 
As in other \ili{Bedouin} dialects, /ð̣/ is the normal equivalent of the 〈\kern 0.75pt{\arabscript{ض}}〉 of \ili{Classical} \ili{Arabic} (e.g. \textit{ð̣mər} ‘to have an empty stomach’ (\ili{CA} \textit{ḍamira}) and \textit{ð̣ḥak} ‘to laugh (\ili{CA} \textit{ḍaḥika}). Nonetheless, /ḍ/ is found in a number of lexemes in Ḥassāniyya. 

The form [ḍ] sometimes occurs as a phonetic realization of /\kern -0.5ptd/ simply due to contact with an {emphatic consonant} (compare \textit{ṣḍam} ‘to upset’ and \textit{ṣadma} ‘annoyance’, \ili{CA} \textit{√ṣdm}). However, /ḍ/ generally appears in the lexemes borrowed from \ili{Standard} \ili{Arabic}, either in all words of a {root}, or in a subset of them, for example: \textit{staḥḍa{\R}} ‘to be in agony’ and \textit{ḥaḍari} ‘urbanite’ but \textit{ḥð̣a{\R}} ‘to be present’ and \textit{maḥ\textsuperscript{ə}\kern -0.75ptð̣{\R}a} ‘Quranic school’. The opposition /ḍ/ vs. /ð̣/ can therefore distinguish a classical meaning from a dialectal meaning: compare \textit{staḥḍa{\R}} to \textit{staḥð̣a{\R}} ‘to remember’.

/ḍ/ is common in the vocabulary of the literate. The less educated speakers sometimes replace /ḍ/ with /ð̣/ (as in \textit{qāð̣i} for \textit{qāḍi} ‘judge’), but the stop realization is stable in many lexemes, including in {loanwords} not related to religion, such as \textit{ḍʕīv} ‘weak’.

The presence of the same {phoneme} /ḍ/ in \ili{Berber} might have facilitated the preservation of its counterpart in \ili{Standard} \ili{Arabic} loans, even though in \ili{Zenaga} /ḍ/ is often fricative (intervocalically). Moreover, the /ḍ/ of \ili{Berber} is normally devoiced in word-final position in Ḥassāniyya, just as in other \ili{Maghrebi} dialects, for example: \textit{ṣayvaṭ} ‘to say goodbye’, from \ili{Berber} \textit{√fḍ} ‘to send’.

\subsubsubsection{The consonant /ẓ/} %3.1.1.2. / 
/ẓ/ is one of the two {emphatic} phonemes of proto-\ili{Berber}. This {emphatic} {sibilant} sound regularly passes from the {source language} to the {recipient language} when \ili{Berber} words are used in Ḥassāniyya. For example: \textit{aẓẓ} \~{} \textit{āẓẓ} ‘wild pearl millet’ (\ili{Zenaga} \textit{īẓi}).

However, /ẓ/ is also present in lexemes of a different origin. Among Ḥassāniyya {roots} also attested in \ili{Classical} \ili{Arabic}, *z often becomes /ẓ/ in the environment of /{\R}/ (e.g. \textit{{\R}āẓ} ‘to try’, \ili{CA} \textit{rāza}; \textit{{\R}aẓẓa} ‘lightning’, \ili{CA} \textit{rizz}; \textit{ẓəb{\R}a} ‘anvil’, \ili{CA} \textit{zubra}). Sometimes /ẓ/ appears in lexemes with a pejorative connotation, e.g. \textit{ẓ{\R}aṭ} ‘fart; lie’ (\ili{CA} \textit{ḍaraṭa}), \textit{ẓagg} ‘make droppings (birds)’ (\ili{CA} \textit{zaqq}).

\subsubsubsection{The consonant /q/} %3.1.1.3. / 
The normal equivalent of the 〈{\arabscript{ق}}〉 of \ili{Classical} \ili{Arabic} is the velar stop /g/, as in other \ili{Bedouin} dialects (e.g. \textit{bag{\R}a} ‘cow’, \ili{CA} \textit{baqara}). However, /q/ is in no way rare.

First of all, /q/ appears, like /ḍ/, in a number of words borrowed from \ili{Classical} \ili{Arabic} by the literate: \textit{ʕaq\kern 0.5pt\textsuperscript{ə}\kern -1.5ptd} ‘religious marriage contract’; \textit{vassaq} ‘to pervert’. The opposition /g/ vs. /q/ can therefore produce two families of words, such as \textit{qibla} ‘Qibla, direction of {Mecca}’ and \textit{gəbla} ‘one of the cardinal directions (south, southwest or west, depending on the region)’. It can also create a distinction between the concrete meaning (with /g/) and the abstract meaning (with /q/): \textit{θgāl} ‘become heavy’, \textit{θqāl} ‘become painful’.

Next, /q/ is present in several lexemes of non-\ili{Arabic} origin, such as \textit{bsaq} ‘silo’, \textit{mzaw{\R}aq} ‘very diluted (of tea)’, (in southwest Mauritania) \textit{sə{\R}qəlla} ‘\ili{Soninke} people’, (in Néma) \textit{sasundaqa} ‘circumcision ceremony’, (in Walata) \textit{raqansak} ‘decorative pattern’, \textit{asanqās} ‘pipe plunger’, \textit{sayqad} ‘shouting in public’, and (in the southeast) \textit{šayqa} ‘to move sideways’. These lexemes, often rare and very local in use, seem to be borrowed mostly from the languages of the Sahel region.\footnote{I am currently unable to specify the origin of these terms except that \textit{bsaq} (attested in \ili{Zenaga}) could be of \ili{Wolof} origin.}

Finally, /q/ is the outcome of *ɣ in cases of gemination, (/ɣɣ/ > [qq]): compare \textit{raqqad} ‘to make porridge’ to \textit{rɣīda} ‘a variety of porridge’ (\ili{CA} \textit{raɣīda}). This correlation, attested in \ili{Zenaga} and more generally in \ili{Berber}, can be attributed to the {substrate}.

Insofar as the contrast between /ɣ/ and /q/ is poorly established in \ili{Berber}, the {substrate} could also explain the tendency, sometimes observed in the southwest, to velarize non-classical instances of /q/ (or at least instances not identified as classical): hence \textit{ɣandīr} ‘candle’ for \textit{qandīr} < \ili{CA} \textit{qandīl} – this is despite the fact that the shift /ɣ/ > /ʔ/ is very common in \ili{Zenaga}. However, the influence of \ili{Berber} does not explain the systematic shift of /ɣ/ to /q/ throughout the eastern part of the Ḥassāniyya region (including \ili{Mali}): thus eastern \textit{qlab} ‘defeat’ for southwestern \textit{ɣlab} (\ili{CA} \textit{ɣalaba}).\footnote{The regular passage from /ɣ/ to /q/ is a typical \ili{Bedouin} trait, related to the voiced realization (/g/) of *q. It occurs especially in southern Algeria, in various dialects of the Chad–\ili{Sudanese} area, and in some \ili{Eastern} dialects \citep[72]{Cantineau1960book}.}

\subsubsubsection{Glottal stop} %3.1.1.4. / 
The glottal stop is one of the phonemes of \ili{Zenaga} (its presence in the language is in fact a feature that is unique among \ili{Berber} varieties), however it is not found in Ḥassāniyya, with the exception of words borrowed from \ili{Standard} \ili{Arabic}, e.g. \textit{tʔabbad} ‘to live religiously’, \textit{danāʔa} ‘baseness’ and \textit{taʔḫīr} ‘postponement’. Very rarely the glottal stop is also maintained when it occurs at the end of a word as in \textit{ba{\R}{\R}aʔ} ‘to declare innocent’.

\subsubsubsection{Palatalized consonants} %3.1.1.5. / 
There are three palatalized consonants: two dental (/tʸ/ and /dʸ/) and a nasal /nʸ/. Unlike the phonemes discussed above, these are very rare in Ḥassāniyya, especially /nʸ/.

The palatalized consonants are also attested in certain neighbouring languages of the Sahel, as well as in \ili{Zenaga} (but these are not phonemes of Common \ili{Berber}). They are rather infrequent in the \ili{Zenaga} lexicon, occurring especially in syntagmatic contexts (\textit{{}-d}+\textit{y-}, \textit{{}-n}+\textit{y-}) and in morphological {derivation} ({formation} of the {passive} by affixation of a geminate /tʸ/).

In Ḥassāniyya, the palatalized consonants mostly appear in words borrowed from \ili{Zenaga} or languages of the Sahel. Interestingly, certain {loanwords} from \ili{Zenaga} are ultimately of \ili{Arabic} origin and constitute examples of phonological integration, as in \textit{tʸfāɣa}, a given name and, in the plural, the name of a tribe < \ili{Zenaga} \textit{atʸfāɣa} ‘marabout’ < \ili{CA} \textit{al-faqīh}, and \textit{ḫurūdʸ} ‘leave (from Quranic school)’ < \ili{Zenaga} \textit{ḫurūdʸ} < \ili{CA} \textit{ḫuruǧ} ‘exit’.

One should also note the palatization of /t/ in certain lexemes from particular semantic domains (such as the two verbs related to fighting \textit{tʸbəl} ‘to hit hard’ and \textit{kawtʸam} ‘boxer’). This may suggest the choice of a palatalized consonant for its expressive value (and would then be a marginal case of phonosymbolism).

\subsubsubsection{Labial and labiovelar consonants} %3.1.1.6. / 
The labiovelar consonants /mʷ, bʷ, f\kern 1ptʷ, vʷ/ or /ṃ, ḅ, f̣, ṿ/) are common in Ḥassāniyya, as they are in \ili{Zenaga}. In both cases, they often come in tandem with a realization [u] of the {phoneme} /ə/. 

This phenomenon may have originally arisen in \ili{Zenaga}, since the Ḥassāniyya of \ili{Mali} (where it was most likely in contact with other languages) exhibits greater preservation of a [u] vowel sound and, at the same time, less pronounced labiovelarization of consonants. 

The Ḥassāniyya of \ili{Mali} also has a voiceless use of the {phoneme} /f/, where the Ḥassāniyya of Mauritania is characterized by the use of /v/ in its place (\citealt{Heath2004}; an observation that my own studies have confirmed). This phonetic trait does not come directly from \ili{Zenaga} (in which /v/ exists but is very rare). However, it could be connected with the preference for voiced phonemes in \ili{Berber} generally and in \ili{Zenaga} in particular.

\subsubsection{Syllabic structures} %3.1.2. /

In Ḥassāniyya, \ili{Arabic}-derived syllabic structures do not contain short vowels in word-internal open syllables, with the exception of particular cases such as {passive} participles in \textit{mu-} (\textit{mudagdag} ‘broken’) and certain nouns of action (\textit{ḥašy} > \textit{ḥaši} ‘filling’). However, {loanwords} from literary \ili{Arabic} and other languages (notably \ili{Berber} and \ili{French}) display short vowels quite systematically in this context: \textit{abadan} ‘never’ and \textit{ḥazīn} ‘sad’ (from \ili{Standard} \ili{Arabic}); \textit{tamāt} ‘gum’ (from \ili{Zenaga} \textit{taʔmað}); \textit{taṃāta} ‘tomato’. In fact, it may be noted that, unlike the majority of \ili{Berber} varieties (particularly in the north), \ili{Zenaga} has a relatively substantial number of lexical items with short vowels (including \textit{ə}) in open syllables: \textit{ka{\R}að̣} ‘three’, \textit{tuðu}\textit{ṃaʔn} ‘a few drops of rain’ \textit{awayan} ‘languages’, \textit{əgəðih} ‘necklace made from plants'.\footnote{It is precisely for this reason that, regarding the loss of the short vowels in open syllables, I deem the hypothesis of a parallel evolution of syllabic structures in \ili{Maghrebi} \ili{Arabic} and \ili{Berber} to be more convincing than the frequently held alternative hypothesis of a one-way influence of the \ili{Berber} {substrate} on the \ili{Arabic} {adstrate}.} 

Furthermore, a {long vowel} \textit{ā} occurs word-finally in loaned nouns which in \ili{Standard} \ili{Arabic} end with \textit{{}-āʔ}: \textit{vidā/vidāy} ‘ransom’. In other cases, underlyingly long word-final vowels are only pronounced long when non-final in a genitive construct.

\subsection{Morphology} %3.2. /

\subsubsection{Nominal morphology} %3.2.1. /

\subsubsubsection{Standard forms} %3.2.1.1. /

Nouns and adjectives borrowed from \ili{Standard} \ili{Arabic} may often be identified by the presence of: a) open syllables with short vowels, e.g. \textit{vaḍa}\textit{lāt} ‘rest of a meal’, \textit{ɣaḍab} ‘anger’, \textit{vasād} ‘alteration’, \textit{ḥtimāl} ‘possibility’, b) short vowels /i/ (less frequently /u/) in a closed syllable: \textit{miḥ{\R}āb} ‘mihrab’, \textit{muḥarrir} ‘inspector; editor’.

Some syllables are only attested in {loanwords}, such as the nominal pattern CVCC, where the pronunciation of the double {coda} necessitates the insertion of a supporting vowel, in which case the dialect takes on the form CCVC: compare \textit{ʕaq\kern 0.5pt\textsuperscript{ə}\kern -1.5ptd} ‘religious marriage’ with \textit{ʕqal} ‘wisdom’.

The most characteristic {loanword} pattern, however, is that of \textit{taḥrīr} ‘liberation; verification (of an account)’. In Ḥassāniyya the equivalent of the pattern taCCīC is təCCāC. For the {root} \textit{√ḥrr}, this provides a verbal noun for other meanings of the verb \textit{ḥa{\R}{\R}a{\R}:} \textit{təḥ{\R}ā{\R}} ‘whipping of wool (to untangle it); adding flour to make dumplings’. As for the form taCaCCuC, the /u/ is sometimes lengthened: \textit{taḥammul} ‘obligation’, but \textit{tavakkūr} ‘contemplation’.

\subsubsubsection{Berber affixes} %3.2.1.2. /

Nouns borrowed from \ili{Berber} are characterized by the frequent presence of the vowels /a, ā, i, ī, u, ū/. These are of varying lengths, except that in a word-final closed syllable they are always long and stressed. Since these vowels appear in all types of syllables – open and closed – this results in much more varied syllabic patterns than in nouns of \ili{Arabic} origin. 

These loans are also characterized by the presence of affixes which, in the {source language}, are markers of {gender} and/or number: the prefix \textit{a/ā-} or \textit{i/ī-} for the masculine, to which the prefix \textit{t-} is also added for the feminine or, more frequently (especially in the singular), a circumfix \textit{t-...-t}. Compare \textit{iggīw} \~{} \textit{īggīw} ‘griot’ with the feminine form \textit{tiggiwīt} \~{} \textit{tīggīwīt}. A suffix in \textit{-(ə)n} characterizes the plurals of these {loanwords} which, moreover, differ from the singulars in terms of their vocalic form: \textit{iggāwən} \~{} \textit{īggāwən} ‘griots’, feminine \textit{tiggawātən} \~{} \textit{tīggawātən}. The presence of these affixes generally precludes the presence of the {definite} {article}.

Though these affixes pass from the {source language} to the target language along with the stems, the syllabic and vocalic patterns of such loans are often particular to Ḥassāniyya: compare Ḥassāniyya \textit{āršān}, plural \textit{īršyūn} \~{} \textit{īršīwən} ‘shallow pit’ with \ili{Zenaga} \textit{aʔraš}, plural \textit{aʔraššan} (see \citealt{Taine-Cheikh1997Zenaga}).

Ḥassāniyya speakers whose mother tongue is \ili{Zenaga} have most likely played a role in the {transfer} of these affixes and their affixation to nouns of all origins (including those of \ili{Arabic} origin: a possible example being \textit{tasūvra} ‘large decorated leather bag for travelling’, cf. \textit{sāvər} ‘to travel’). The forms that these speakers use can also be different from those used by other Ḥassāniyya speakers – especially if the latter have not been in contact with \ili{Berber} speakers for a long time. 

It is not proven that \ili{Berber} speakers are the only ones to have created and imposed these forms which are more Berberized than authentically \ili{Berber}. However, it may be noted that the {gender} of nouns borrowed from \ili{Berber} is generally well preserved in Ḥassāniyya, even for the feminine nouns losing their final \textit{{}-t}, other than in special cases such as the {collective} \textit{tayšəṭ} ‘thorny tree (\textit{Balanites} \textit{aegyptiaca})’ with a final \textit{{}-ṭ} (< \ili{Zenaga} \textit{tayšaḌ} for \textit{tayšaḍt}).\footnote{In \ili{Zenaga}, non-intervocalic geminates are distinguished not by length, but rather by tension, and it is this that is indicated by the use of uppercase for the final \textit{Ḍ.} } In fact, this indicates a deep penetration of the meaning of these affixes and of \ili{Berber} morphology in general (up to and including the incompatibility of these affixes with the {definite} {article}). 

The borrowing of the formants \textit{ən-} ‘he of’ and \textit{tən-} ‘she of’ (quasi-equivalents of the \ili{Arabic}-derived \textit{bū-} and \textit{ūṃ(ṃ)-}) is fairly widespread, in particular in the {formation} of proper nouns. It is also mostly in toponyms and anthroponyms that the {diminutive} form with prefix \textit{aɣ-} and suffix \textit{{}-t} is found, e.g. the toponym \textit{Agjoujt} (< \textit{aɣ-žoʔž-t} ‘small ditch’).

\subsubsection{Verbal Morphology} %3.2.2. /

\subsubsubsection{The derivation of \textit{sa-}}\label{sa}

The existence of verb forms with the prefix \textit{sa-} is one of the unique characteristics of Ḥassāniyya (\citealt{Cohen1963,Taine-Cheikh2003}). There is nothing, however, to indicate that the prefix is an ancient \ili{Semitic} feature that Ḥassāniyya has preserved since its earliest days. Instead, the regular correspondences between the three series of derived verb forms (causative--factitive vs. reflexive vs. {passive}) and the specialization of the morpheme \textit{t} as a specific marker of reflexivity underlie the creation of causative--factitives with \textit{sa-}. Neologisms with \textit{sa-} generally appear when forms with the prefix \textit{sta-} have a particular meaning: \textit{staslaʕ} ‘to get worse (an injury)’ – \textit{saslaʕ} ‘to worsen (injury)’; \textit{stab{\R}ak} ‘to seek blessings’ – \textit{sab{\R}ak} ‘to give a blessing’; \textit{stagwa} ‘to behave as a griot’ – \textit{sagwa} ‘to make someone a griot’; \textit{staqbal} ‘to head towards the Qibla’ – \textit{saqbal} ‘to turn an animal for slaughter in the direction of the Qibla’. 

Furthermore, the influence of \ili{Berber} has certainly played a role, since the prefix \textit{s(a)-} (or one of its variants) very regularly forms the causative--factitive structure in this branch of the \ili{Afro-Asiatic} language family. 

In \ili{Zenaga}, the most frequent realization of this prefix is with a palato-alveolar shibilant, but a {sibilant} realization also occurs, particularly with {roots} of \ili{Arabic} origin. For example: Hass. \textit{sādəb} (variant of \textit{ddəb}) – \ili{Zen.} \textit{yassiʔðab} ‘to train an animal (with a saddle)’ < \ili{CA} \textit{√ʔdb} (cf. \textit{ʔaddaba} ‘educate, carefully bring up’); Hass. \textit{sasla} \textit{–} \ili{Zen.} \textit{yassaslah} ‘to let a hide soak to give it a consistency similar to a placenta’ and Hass. \textit{stasla} – \ili{Zen.} \textit{staslah} ‘start to lose fur (of hides left to soak)’ < \ili{CA} \textit{√sly} (cf. \textit{salā} ‘placenta’).

Parallel to these examples where the \ili{Berber} forms (at least those with the prefix \textit{st(a)}{}-) are most likely themselves borrowed, we also find patterns with \textit{sa-/ša-} which are incontestably of \ili{Berber} origin: compare Ḥassāniyya \textit{niyyər} ‘to have a good sense of direction’, \textit{sanyar} ‘to show the way’, \textit{stanyar} ‘to know well how to orient oneself’ and \ili{Tuareg} \textit{ener} ‘to guide’, \textit{sener} ‘to make guide’. Typically, however, when Ḥassāniyya borrows {causative} forms from \ili{Berber}, it usually integrates the \ili{Berber} prefix as part of the Ḥassāniyya {root}, making it the first radical of a quadriliteral {root}, e.g. Hass. \textit{sadba} \textit{–} \ili{Tuareg} \textit{sidou} ‘to make s.o. leave in the afternoon’ and Hass. \textit{ssadba} \textit{(<} \textit{tsadba)} – \ili{Tuareg} \textit{adou} ‘to leave in the afternoon’. 

The parallelism between \ili{Arabic} and \ili{Berber} is not necessarily respected in all cases, but the forms with initial \textit{s-/š-} are usually {causative} or factitive in both cases. The only exception concerns certain \ili{Zenaga} verbal forms which have become irregular upon contact with Ḥassāniyya: thus \textit{yassəðbah} ‘to leave in the afternoon’ or \textit{yišnar} ‘to orient oneself’ (a variant of \textit{yinar}), of which the original {causative} value is now carried by a form with a double prefix (\textit{ž}+\textit{š}): \textit{yažəšnar} ‘to guide’.

\subsubsubsection{The Derivation of \textit{u-}} %3.2.2.2. /

The existence of a {passive} verbal prefix \textit{u-} for quadrilateral verbs and derived forms constitutes another unique feature of Ḥassāniyya. For example: \textit{udagdag}, {passive} of \textit{dagdag} ‘to break’; \textit{uṭabbab}, {passive} of \textit{ṭabbab} ‘to train (an animal)’; \textit{udāɣa}, {passive} of \textit{dāɣa} ‘to cheat (in a game)’.

The development of passives with \textit{u-} was most likely influenced by \ili{Classical} \ili{Arabic}, since here the passives of all verbal measures feature /u/ in the first syllable in both the perfect and the imperfect, e.g. \textit{fuʕila}, \textit{yu}\textit{fʕalu}; \textit{fuʕʕila}, \textit{yu}\textit{faʕʕalu}; and \textit{fūʕila}, \textit{yu}\textit{fāʕalu}, the respective passives of \textit{faʕala}, \textit{faʕʕala} and \textit{fāʕala}.

However, influence from \ili{Berber} cannot be excluded here since, in \ili{Zenaga}, the {formation} of passives with the prefix \textit{Tʸ} is directly parallel to those of the passives with \textit{u-} in Ḥassāniyya. Moreover, this prefix is \textit{t(t)u-} or \textit{t(t)w-} in other \ili{Berber} varieties (especially those of Morocco) and this could also have had an influence on the emergence of the prefix \textit{u-}. 

\subsection{Syntax} %3.3. /

\subsubsection{Ḥassāniyya–Zenaga parallelisms} %3.3.1. /

Ḥassāniyya and \ili{Zenaga} have numerous common features, and this is especially true in the realm of syntax. In general, the reason for these common traits is that they both belong to the \ili{Afro-Asiatic} family and remain conservative in various respects; for example, in their lack of a discontinuous negative construction.

There are, however, also features of several varieties of both languages documented in Mauritania that represent parallel innovations. Thus, corresponding to the {diminutive} forms particular to \ili{Zenaga}, we have in Ḥassāniyya \textit{mutatis} \textit{mutandis} a remarkably similar {extension} to verbs of the {diminutive} pattern with infix \textit{-ay-}, e.g. \textit{mayllas}, {diminutive} of \textit{mallas} ‘to smooth over’ (\citealt{Taine-Cheikh2008chapter}: 123–124).

In the case of aspectual--temporal forms, there are frequent parallels, such as Ḥassāniyya \textit{mā} \textit{tla} and \ili{Zenaga} \textit{war} \textit{yiššiy} ‘no longer’, Ḥassāniyya \textit{ma-zāl} and \ili{Zenaga} \textit{yaššiy} ‘still’, Ḥassāniyya \textit{tamm} and \ili{Zenaga} \textit{yuktay} ‘to continue to’, Ḥassāniyya \textit{ʕgab} and \ili{Zenaga} \textit{yaggara} ‘to end up doing’. One of the most notable parallel innovations, however, concerns the {future} morpheme: Ḥassāniyya \textit{lāhi} (invariable {participle} of an otherwise obsolete verb, but compare \textit{ltha} ‘to pass one’s time’) and \ili{Zenaga} \textit{yanhāya} (a conjugated verb also meaning ‘to busy oneself with something’, in addition to its {future} function). In both cases we have forms related to \ili{Classical} \ili{Arabic} \textit{lahā} ‘to amuse oneself’, with the \ili{Zenaga} form apparently being a borrowing. It seems, therefore, that this borrowing preceded the \textit{lāhi} of Ḥassāniyya and likely then influenced its adoption as a {future} {tense} marker. Note also that in the \ili{Arabic} dialect of the {Jews} of \ili{Algiers}, \textit{lāti} is a durative present {tense} marker (\citealt{Cohen1924}: 221; \citealt{Taine-Cheikh2004}: 224; \citealt{Taine-Cheikh2008chapter}: 126–127; \citealt{Taine-Cheikh2009}: 99).

Ḥassāniyya and \ili{Zenaga} also display common features with regard to complex phrases. For example, concerning completives, \ili{Zenaga} differs from other \ili{Berber} languages in its highly developed usage of \textit{ad} \~{} \textit{að}, and in particular in the grammaticalized usage of this demonstrative as a quotative particle after verbs of speaking and thinking \citep{Taine-Cheikh2010Zenaga}. This may have had an influence on the usage of the conjunctions \textit{an(n)-} and \textit{ʕan-} (the two forms tend to be confused) in the same function in Ḥassāniyya.

Finally, regarding the variable appearance of a resumptive pronoun in Ḥassāniyya object {relative} clauses, if influence from \ili{Berber} (where a resumptive pronoun is always absent) has played any role here, it has simply been to reinforce a construction already attested in the earliest \ili{Arabic}, whereby the resumptive pronoun is absent if the antecedent is {definite}, as in ‎(1).

\ea
\gll n{\R}ədd ʕlī-kum~~ ə{\R}-{\R}wāye lli {\R}add-Ø ʕlī-ya muḥammad\\
 tell.\textsc{impf.}1\textsc{sg}~~ on-2\textsc{pl} \textsc{def-}story~~ \textsc{rel}~~ tell.\textsc{prf.3sg.m-}Ø~~ on-\textsc{obl.1sg} Mohammed\\
\glt ‘I am going to tell you the story that Mohammed told me.’ 
\z

\subsubsection{Regional influence of Maghrebi Arabic} %3.3.2. /

The Ḥassāniyya spoken in the south of Morocco is rather heavily influenced by other \ili{Arabic} varieties spoken in the region. Even among those who conserve virtually all the characteristic features of Ḥassāniyya (preservation of interdentals, synthetic genitive construction, absence of the pre-verbal particle \textit{kā-} or \textit{tā-}, absence of discontinuous {negation}, absence of the {indefinite article}), particular features of the \ili{Moroccan} \ili{Arabic} koiné appear either occasionally or regularly among certain speakers. The most common such features are perhaps the genitive particle \textit{dyal} (\citealt{Taine-Cheikh1997socio}: 98) and the preverbal particle \textit{kā} (\citealt{Aguadé1998}: 211, §37; 213, §42).

In the Ḥassāniyya of \ili{Mali}, usage of a genitive particle remains marginal, although Heath (\citeyear[162]{Heath2004}) highlights a few uses of genitive \textit{(n)tāʕ} in his texts.

\subsection{Lexicon} %3.4. /

\subsubsection{Confirmed loanwords} %3.4.1. /

\subsubsubsection{Loanwords from Standard Arabic} %3.4.1.1. /

Verbs loaned from \ili{Standard} \ili{Arabic} are as common as nominal and adjectival loans. Whatever their category, loans are often distinctive in some way (whether because of their {syllabic structure}, the presence of particular phonemes or their morphological template), since the lexeme usually (though not always) has the same form in both the {recipient language} and the {source language}. Examples of loans without any distinctive features are \textit{ba{\R}{\R}a{\R}} ‘to justify’ and \textit{ðahbi} ‘golden’. 

A certain number of \ili{Standard} \ili{Arabic} verbs with the infix \textit{{}-}\textit{t-} or the prefix \textit{sta-} are borrowed, but these verbal patterns can be found elsewhere in Ḥassāniyya. 

Certain lexical fields exhibit a particularly high degree of loans from \ili{Standard} \ili{Arabic}: anything connected with Islamic studies or abstract concepts (religion, rights, morality, feelings, etc.) and, more recently, politics, media and modern material culture. These regularly retain the meaning (or one of the meanings) of the {source-language} item. 

\subsubsubsection{Loanwords from Berber} %3.4.1.2. /

There are many lexical items that are probable loans from \ili{Berber}, with a number of certain cases among them. 

Here we may point to several non-\ili{Arabic}-origin verbs with cognates across a wide range of \ili{Berber} languages, such as \textit{k{\R}aṭ} ‘to scrape off’ (\ili{Zenaga} \textit{yug{\R}að̣}); \textit{šayð̣að̣} ‘to make a lactating camel adopt an orphaned calf from another mother’ (\ili{Zenaga} \textit{yaṣṣuð̣að̣} ‘to breastfeed’, \textit{yuḍḍað̣} ‘to suckle’); \textit{santa} ‘to begin’ (\ili{Zenaga} \textit{yassanta} ‘to begin’, \ili{Tuareg} \textit{ent} ‘to be started, to begin’); \textit{gaymar} ‘to hunt from a distance’ (\ili{Berber} \textit{gmər} ‘to hunt'). 

Other verbs are derived from nouns loaned from \ili{Berber}. Hence, \textit{ɣawba} ‘to restrain a camel, put it in an \textit{aɣāba}’ (\ili{Tuareg} \textit{aɣaba} ‘jaws’). Sometimes there is both a verb and an adjective stemming from a loaned {root}, as in \textit{gaylal} ‘to have the tail cut’ and \textit{agīlāl} ‘having a cut tail’ (\ili{Tuareg} \textit{gilel} and \textit{agilal}). 

Some loaned Ḥassāniyya nouns are found with the same {root} (or an equivalent {root}) in \ili{Berber} languages other than \ili{Zenaga}. For example: \textit{agayš} ‘male bustard’ (\ili{Tuareg} \textit{gayəs}); \textit{āškə{\R}} ‘partridge’ (\ili{Kabyle} \textit{tasekkurt} in the feminine form); \textit{tayffārət} ‘fetlock (camel)’ (\ili{Zenaga} \textit{tiʔffart}, \ili{Tuareg} \textit{téffart}); \textit{azāɣər} ‘wooden mat ceiling between beams’ (\ili{Zenaga} \textit{azaɣri} ‘lintel, beam (of a well)’, \ili{Tuareg} \textit{ǝ\kern -0.5ptzgər} ‘to cross’, \textit{ăzəgər} ‘crossbeam’); \textit{talawmāyət} ‘dew’ (\ili{Zenaga} \textit{tayaṃut}, \ili{Tuareg} \textit{tălămut}); \textit{(n)tūrža} ‘\textit{Calotropis} \textit{procera}’ (\ili{Zenaga} \textit{turžah}, \ili{Tuareg} \textit{tərza}).

Most of the {loanwords} cited above are attested in \ili{Zenaga} (sometimes in a more innovative form than is found in other \ili{Berber} varieties, such as \textit{yaggīyyay} ‘to have a cut tail’ where /y/ < *l). However, there are numerous cases where a corresponding \ili{Berber} item is attested only in \ili{Zenaga}. In such cases it is difficult to precisely identify the {source language}, even if the phonology and/or morphology seems to indicate a non-\ili{Arabic} origin. 

Loanwords from \ili{Berber} seem to be particularly common in the lexicon of fauna, flora, and diseases, as well as in the field of traditional material culture (objects, culinary traditions, farming practices, etc.; \citealt{Taine-Cheikh2010lexiques,Taine-Cheikh2014}). Unlike the form of the loans, which is often quite divergent from that of the source items, their semantics tends to remain largely unchanged. However, there are some exceptions, notably when the verbs have a general meaning in \ili{Berber} (cf. above `to breastfeed' vs. `to make a lactating camel adopt an orphaned calf from another mother'). 

\subsubsubsection{Loanwords from Sahel languages} %3.4.1.3. /

Rather few Ḥassāniyya lexical items seem to be borrowed directly from African languages, and the origin of those that are is rarely known precisely. We may note, however, in addition to \textit{gadʸ} ‘dried fish’ (< \ili{Wolof}) and \textit{dʸəngra} ‘warehouse’ (< \ili{Soninke}), a few terms which appear to be borrowed from Pulaar: \textit{ḅamḅa} ‘to carry a child on one’s back’, \textit{tʸəhli} ‘roof on pillars’ and \textit{kīri} ‘boundary between two fields’.

In some regions we find a concentration of loans in particular domains in relation to specific contact languages. For example, in the ancient town of Tichitt, we find borrowings from \ili{Azer} and \ili{Soninke} (\citealt{Jacques-Meunié1961}; \citealt{Monteil1939}; \citealt{Diagana2013}): \textit{kā} ‘house’ (\ili{Azer} \textit{ka(ny)}, \ili{Soninke} \textit{ká}) in \textit{kā} \textit{n} \textit{laqqe} ‘entrance of the house’; \textit{killen} ‘path’ (\ili{Azer} \textit{kille}, \ili{Soninke} \textit{kìllé}); \textit{kunyu} \~{} \textit{kenyen} ‘cooking’ (\ili{Azer} \textit{knu} \~{} \textit{kenyu}, \ili{Soninke} \textit{kìnŋú}).

A significant list of {loanwords} from \ili{Songhay} has been compiled by \citet{Heath2004} in Mali, including e.g.: \textit{ṣawṣab} (< \textit{sosom} \~{} \textit{sosob}) ‘pound (millet) in mortar to remove bran from grains’; \textit{daydi} \~{} \textit{dayday} (< \textit{deydey}) ‘daily grocery purchase’; \textit{ākā{\R}āy} (< \textit{kaarey}) ‘crocodile’; \textit{sari} (< \textit{seri}) ‘millet porridge’. Only \textit{sari} has been recorded elsewhere in Mauritania (in the eastern town of Walata). On the other hand, all authors who have done field work on the Ḥassāniyya of \ili{Mali} (particularly in the region of Timbuktu and the Azawad), have noted {loanwords} from \ili{Songhay}. This is true also of \citet{Clauzel1960} who, as well as a number of \ili{Berber} {loanwords}, gives a small list of \ili{Songhay}-derived items used in the salt mine of Tāwdenni, such as \textit{titi} ‘cylinder of saliferous clay used as a seat by the miners’ (< \textit{tita}) and \textit{tʸar} ‘adze’ (< \textit{tʸara}).

\subsubsubsection{Loanwords from Indo-European languages} %3.4.1.4. /

The use of {loanwords} from European languages tends to vary over time. Thus, a large proportion of the \ili{French} {loanwords} borrowed during the {colonial} period have more recently gone out of use, such as \textit{bə{\R}ṭmāla} or \textit{qo{\R}ṭmāl} ‘wallet’ (< \textit{porte-monnaie}), \textit{dabbīš} ‘telegram’ (< \textit{dépêche} ‘dispatch’) or \textit{ṣa{\R}waṣ} ‘to be very close to the colonizers’ (\textit{<} \textit{service} ‘service’). This is true not only of items referring to obsolete concepts (such as the currency terms \textit{sūvāya} ‘sou’ or \textit{ftən/vəvtən} ‘cent’, likely < \textit{fifteen}), but also of those referring to still-current concepts which are, however, now referred to with a term drawn from \ili{Standard} \ili{Arabic} (e.g. \textit{minəstr} ‘minister’, replaced by \textit{wazīr}). This does not, however, eliminate the permanence of some old {loanwords} such as \textit{wata} ‘car’ (< \textit{voiture}) or \textit{ma{\R}ṣa} ‘market’ (< \textit{marché}).\footnote{\citet{OuldMohamedBaba2003} gives an extensive list of {loanwords} from \ili{French} and offers a classification by semantic field.}

Although not unique to Ḥassāniyya, the {frequency} of the {emphatic} phonemes (especially /ṣ/ and /ṭ/) in loans from European languages is notable. Consider, in addition to the treatment of \textit{service}, \textit{porte-monnaie} and \textit{marché} as noted above, that of \textit{baṭ{\R}ūn} ‘boss’ (< \textit{patron}), which gives rise to \textit{tbaṭ{\R}an} ‘to be(come) a boss’ \textit{ṭawn} ‘ton’ (< \textit{tonne}).

\subsubsection{More complex cases} %3.4.2. /

\subsubsubsection{\textit{Wanderwörter}} \label{wander}

Various \ili{Arabic} lexical items derive from \ili{Latin}, \ili{Armenian}, \ili{Turkish}, \ili{Persian}, and so on. In the case of, for example, the names of calendar months, or of items such as trousers (\textit{sərwāl)}, these terms are not borrowed directly from the {source language} by Ḥassāniyya and are found elsewhere (e.g. \textit{balbūẓa} ‘eyeball’ < \ili{Latin} \textit{bulbus}, attested throughout the Maghreb). The history of such items will not be dealt with here. We can, however, mention the case of some well-attested terms in Ḥassāniyya that appear to have been borrowed from sub-Saharan Africa.

One such is \textit{mā{\R}u} ‘rice’, which seems to come from \ili{Soninke} (\textit{máarò}), although it is also attested in \ili{Wolof} (\textit{maalo}) and \ili{Zenaga} (\textit{mārih}). Another term, which is just as emblematic, is \textit{mbū{\R}u} ‘bread’, whose origin has variously been attributed to \ili{Wolof}, \ili{Azer}, Mandigo, and even \ili{English} \textit{bread}.

To these very everyday terms, we may also add \textit{ṃutri} ‘pearl millet’ and \textit{makka} ‘maize’, which have the same form both in Ḥassāniyya and in \ili{Zenaga}. The first is a {loanword} from Pulaar (\textit{muutiri}). The second is attested in many languages and seems to have come from the placename {Mecca}. 

As for \textit{garta} ‘peanut’, \textit{ḷāḷo} \~{} \textit{ḷaḷu} ‘pounded baobab leaves that serve as a condiment’ (synonym of \textit{taqya} in the southwest of Mauritania) and \textit{kəddu} ‘spoon’, these appear to be used just as frequently in Pulaar as they are in \ili{Wolof}. 

\subsubsubsection{Berberized items} %3.4.2.2. /

Despite the absence of any \ili{Berber} affixes in the {loanwords} listed in §\ref{wander}, only \textit{kəddu} ‘spoon’ is regularly used with the {definite} {article}. In this regard, these {loanwords} act like words borrowed from \ili{Berber}, or more generally, those with \ili{Berber} affixes. 

It is, in fact, difficult to prove that a noun with this kind of affix is definitely of \ili{Berber} origin, since we find nouns of various origins with \ili{Berber} affixes. Some of them are {loanwords} from the languages of the sedentary people of the valley, such as \textit{adabāy} ‘village of former sedentary slaves (\textit{ḥ{\R}āṭīn})’ (< \ili{Soninke} \textit{dèbé} ‘village’); \textit{iggīw} \~{} \textit{īggīw} ‘griot’ (\ili{Zenaga} \textit{iggiwi}, borrowed from \ili{Wolof} \textit{geewel} or from Pulaar \textit{gawlo}). Others are borrowed from \ili{French}: \textit{agā{\R}āž} ‘garage’; \textit{təmbīskit} ‘biscuit’. Even terms of \ili{Arabic} origin are Berberized, as is likely the case with \textit{tasūvra} ‘large decorated leather bag for travelling’ (cf. \textit{sāvər} ‘to travel’) or \textit{tāẓəẓmīt} ‘asthma’ (cf. \ili{CA} \textit{zaǧma} ‘shortness of breath when giving birth’).

\subsubsubsection{Reborrowings} %3.4.2.3. /

Instances of back and forth between two languages – primarily Ḥassāniyya and \ili{Zenaga} – seem to be the reason for another type of mixed form, illustrated previously in §\ref{sa} by the \ili{Zenaga} verbs \textit{yassəðbah} ‘to leave in the afternoon’ and \textit{yišnar} ‘to orient oneself’.

Ḥassāniyya \textit{saɣnan} ‘to mix gum with water to make ink’ provides another example, where this time the points of departure and arrival seem to be from the \ili{Arabic} side. In fact, this {loanword} is a borrowing of \ili{Zenaga} \textit{yassuɣnan} ‘to thicken (ink) by adding gum’, a verb formed from \textit{əssaɣan} ‘gum’. This noun in turn appears to be an adaptation of the \ili{Arabic} \textit{samɣa} ‘ink’. 

In the case of \textit{sla} ‘placenta’, there is a double round-trip between the two languages, this time without metathesis: after a passage from \ili{Arabic} to \ili{Zenaga} (> \textit{əs(s)la}), there is return to Ḥassāniyya with the {causative} verb \textit{sasla} ‘to soak a hide’, and a second loan into \ili{Zenaga} with the reflexive form \textit{(yə)stasla} ‘to start to lose fur (of soaked hides)’.

\subsubsubsection{Calques} %3.4.2.4. /

Calques are undoubtedly common, but they are particularly frequent in locutions such as \textit{rəggət} \textit{əž-žəll} ‘susceptibility’ and \textit{bū-damʕa} ‘rinderpest’ (literally ‘thinness of skin’ and ‘the one with a tear’). These are exact calques of their \ili{Zenaga} equivalents \textit{taššəddi-n} \textit{əyim} and \textit{ən-anḍi} (\citealt{Taine-Cheikh2008chapter}).

\subsubsubsection{Individual variation} %3.4.2.5. /

Receptivity to {loanwords} differs from one individual to another. This is natural when we are dealing with bilingual speakers and this probably explains the special features of the Ḥassāniyya of the Awlād Banʸūg (often bilingual speakers of Ḥassāniyya and \ili{Wolof}) or the Ḥassāniyya of \ili{Mali} (where \ili{Arabic} speakers often speak \ili{Songhay} and sometimes Tamasheq). However, it also depends on the individuals in question in terms of what we might call their “loyalty” to the language, whether the language is under pressure from \ili{Moroccan} \ili{Arabic} koiné in Morocco (\citealt{Taine-Cheikh1997socio}; \citealt{Heath2002}; \citealt{Paciotti2017}), or whether it is imposed as a lingua franca in Mauritania \citep{Dia2007}.

\section{Conclusion} %4. /

The principal domain affected by contact in Ḥassāniyya is that of the lexicon (though an assessment in percentage terms is not at present possible). However, the integration of {loanwords} – in particular those from \ili{Standard} \ili{Arabic} and \ili{Berber} – has resulted in a significant enrichment of the phonological system and of the inventory of nominal patterns. The effects of contact on the verbal morphology and syntax of the dialect are more indirect. The major developments in Ḥassāniyya seem most likely to instead be a product of internal evolution. In certain cases, \ili{Zenaga} has probably had an influence; in others, we rather witness instances of parallel evolution. 

In {future}, by studying the vehicular Ḥassāniyya of Mauritania and of the border regions (southern Morocco, southern Algeria, Senegal, Niger, and so on) we will perhaps discover new developments as a result of contacts triggered by the political and societal changes of the twenty-first century. 

\section*{Further reading}

Links between Ḥassāniyya and other languages are particularly complex at the level of semantics and lexicon. On these topics, beyond the available Ḥassāniyya and \ili{Zenaga} dictionaries (\citealt{Heath2004}; \citealt{Taine-Cheikh1988dictionary,Taine-Cheikh2008dictionary}), readers may consult the available studies of specific fields (\citealt{Monteil1952}; \citealt{Taine-Cheikh2013}) or particular templates (\citealt{Taine-Cheikh2018quadri}).

\section*{Abbreviations}
\begin{multicols}{2}
\begin{tabbing}
\textsc{ipfv} \hspace{1em} \= before common era\kill
\textsc{1, 2, 3} \> 1st, 2nd, 3rd person \\
CA           \> Classical Arabic \\
\textsc{def}     \> {definite} \\
Hass.           \> Ḥassāniyya \\
\textsc{impf}   \> imperfect \\
\textsc{m}   \> masculine\\
\textsc{obl} \> oblique \\
\textsc{pl}  \> plural\\
\textsc{rel}     \> {relativizer}\\
\textsc{sg}  \> singular \\
Zen. \> Zenaga
\end{tabbing}
\end{multicols}



\sloppy
\printbibliography[heading=subbibliography,notkeyword=this]
\end{document}
