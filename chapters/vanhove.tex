\documentclass[output=paper]{langsci/langscibook} 
\author{Martine Vanhove\affiliation{LLACAN (CNRS, INALCO)}}
\title{Beja}
\abstract{This chapter argues for two types of outcomes of the long-standing and intense contact situation between Beja and Arabic in Sudan: borrowings at the phono\-logical, syntactic and lexical levels, and convergence at the morphological level.}
\IfFileExists{../localcommands.tex}{
  % add all extra packages you need to load to this file 
\usepackage{graphicx}
\usepackage{tabularx}
\usepackage{amsmath} 
\usepackage{multicol}
\usepackage{lipsum}
\usepackage[stable]{footmisc}
\usepackage{adforn}
%%%%%%%%%%%%%%%%%%%%%%%%%%%%%%%%%%%%%%%%%%%%%%%%%%%%
%%%                                              %%%
%%%           Examples                           %%%
%%%                                              %%%
%%%%%%%%%%%%%%%%%%%%%%%%%%%%%%%%%%%%%%%%%%%%%%%%%%%%
% remove the percentage signs in the following lines
% if your book makes use of linguistic examples
\usepackage{./langsci/styles/langsci-optional} 
\usepackage{./langsci/styles/langsci-lgr}
\usepackage{morewrites} 
%% if you want the source line of examples to be in italics, uncomment the following line
% \def\exfont{\it}

\usepackage{enumitem}
\newlist{furtherreading}{description}{1}
\setlist[furtherreading]{font=\normalfont,labelsep=\widthof{~},noitemsep,align=left,leftmargin=\parindent,labelindent=0pt,labelwidth=-\parindent}
\usepackage{phonetic}
\usepackage{chronosys,tabularx}
\usepackage{csquotes}
\usepackage[stable]{footmisc} 

\usepackage{langsci-bidi}
\usepackage{./langsci/styles/langsci-gb4e} 

  \makeatletter
\let\thetitle\@title
\let\theauthor\@author 
\makeatother

\newcommand{\togglepaper}[1][0]{ 
  \bibliography{../localbibliography}
  \papernote{\scriptsize\normalfont
    \theauthor.
    \thetitle. 
    To appear in: 
    Christopher Lucas and Stefano Manfredi (eds.),  
    Arabic and contact-induced language change
    Berlin: Language Science Press. [preliminary page numbering]
  }
  \pagenumbering{roman}
  \setcounter{chapter}{#1}
  \addtocounter{chapter}{-1}
}

\newfontfamily\Parsifont[Script=Arabic]{ScheherazadeRegOT_Jazm.ttf} 
\newcommand{\arabscript}[1]{\RL{\Parsifont #1}}
\newcommand{\textarabic}[1]{{\arabicfont #1}}

\newcommand{\textstylest}[1]{{\color{red}#1}}

\patchcmd{\mkbibindexname}{\ifdefvoid{#3}{}{\MakeCapital{#3}
}}{\ifdefvoid{#3}{}{#3 }}{}{\AtEndDocument{\typeout{mkbibindexname could
not be patched.}}}

%command for italic r with dot below with horizontal correction to put the dot in the prolongation of the vertical stroke
%for some reason, the dot is larger than expected, so we explicitly reduce the font size (to \small)
%for the time being, the font is set to an absolute value. To be more robust, a relative reduction would be better, but this might not be required right now
\newcommand{\R}{r\kern-.05ex{\small{̣}}\kern.05ex}


\DeclareLabeldate{%
    \field{date}
    \field{year}
    \field{eventdate}
    \field{origdate}
    \field{urldate}
    \field{pubstate}
    \literal{nodate}
}

\renewbibmacro*{addendum+pubstate}{% Thanks to https://tex.stackexchange.com/a/154367 for the idea
  \printfield{addendum}%
  \iffieldequalstr{labeldatesource}{pubstate}{}
  {\newunit\newblock\printfield{pubstate}}
}
 
  %% hyphenation points for line breaks
%% Normally, automatic hyphenation in LaTeX is very good
%% If a word is mis-hyphenated, add it to this file
%%
%% add information to TeX file before \begin{document} with:
%% %% hyphenation points for line breaks
%% Normally, automatic hyphenation in LaTeX is very good
%% If a word is mis-hyphenated, add it to this file
%%
%% add information to TeX file before \begin{document} with:
%% %% hyphenation points for line breaks
%% Normally, automatic hyphenation in LaTeX is very good
%% If a word is mis-hyphenated, add it to this file
%%
%% add information to TeX file before \begin{document} with:
%% \include{localhyphenation}
\hyphenation{
affri-ca-te
affri-ca-tes
com-ple-ments
homo-phon-ous
start-ed
Meso-potam-ian
morpho-phono-logic-al-ly
morpho-phon-em-ic-s
Palestin-ian
re-present-ed
Ki-nubi
ḥawār-iyy-ūn
archa-ic-ity
fuel-ed
de-velop-ment
pros-od-ic
Arab-ic
in-duced
phono-logy
possess-um
possess-ive-s
templ-ate
spec-ial
espec-ial-ly
nat-ive
pass-ive
clause-s
potent-ial-ly
Lusignan
commun-ity
tobacco
posi-tion
Cushit-ic
Middle
with-in
re-finit-iz-ation
langu-age-s
langu-age
diction-ary
glossary
govern-ment
eight
counter-part
nomin-al
equi-valent
deont-ic
ana-ly-sis
Malt-ese
un-fortun-ate-ly
scient-if-ic
Catalan
Occitan
ḥammāl
cross-linguist-ic-al-ly
predic-ate
major-ity
ignor-ance
chrono-logy
south-western
mention-ed
borrow-ed
neg-ative
de-termin-er
European
under-mine
detail
Oxford
Socotra
numer-ous
spoken
villages
nomad-ic
Khuze-stan
Arama-ic
Persian
Ottoman
Ottomans
Azeri
rur-al
bi-lingual-ism
borrow-ing
prestig-ious
dia-lects
dia-lect
allo-phone
allo-phones
poss-ible
parallel
parallels
pattern
article
common-ly
respect-ive-ly
sem-antic
Moroccan
Martine
Harrassowitz
Grammatic-al-ization
grammatic-al-ization
Afro-asiatica
Afro-asiatic
continu-ation
Semit-istik
varieties
mono-phthong
mono-phthong-ized
col-loquial
pro-duct
document-ary
ex-ample-s
ex-ample
termin-ate
element-s
Aramaeo-grams
Centr-al
idioms
Arab-ic
Dadan-it-ic
sub-ordin-ator
Thamud-ic
difficult
common-ly
Revue
Bovingdon
under
century
attach
attached
bundle
graph-em-ic
graph-emes
cicada
contrast-ive
Corriente
Andalusi
Kossmann
morpho-logic-al
inter-action
dia-chroniques
islámica
occid-ent-al-ismo
dialecto-logie
Reichert
coloni-al
Milton
diphthong-al
linguist-ic
linguist-ics
affairs
differ-ent
phonetic-ally
kilo-metres
stabil-ization
develop-ments
in-vestig-ation
Jordan-ian
notice-able
level-ed
migrants
con-dition-al
certain-ly
general-ly
especial-ly
af-fric-ation
Jordan
counter-parts
com-plication
consider-ably
inter-dent-al
com-mun-ity
inter-locutors
com-pon-ent
region-al
socio-historical
society
simul-taneous
phon-em-ic
roman-ization
Classic-al
funeral
Kurmanji
pharyn-geal-ization
vocab-ulary
phon-et-ic
con-sonant
con-sonants
special-ized
latter
latters
in-itial
ident-ic-al
cor-relate
geo-graphic-al-ly
Öpengin
Kurd-ish
in-digen-ous
sunbul
Christ-ian
Christ-ians
sekin-în
fatala
in-tegration
dia-lect-al
Matras
morpho-logy
in-tens-ive
con-figur-ation
im-port-ant
com-plement
ḥaddād
e-merg-ence
Benjmamins
struct-ure
em-pir-ic-al
Orient-studien
Anatolia
American
vari-ation
Jastrow
Geoffrey
Yarshater
Ashtiany
Edmund
Mahnaz
En-cyclo-pædia
En-cyclo-paedia
En-cyclo-pedia
Leiden
dia-spora
soph-is-ic-ated
Sasan-ian
every-day
domin-ance
Con-stitu-tion-al
religi-ous
sever-al
Manfredi
re-lev-ance
re-cipi-ent
pro-duct-iv-ity
turtle
Morocco
ferman
Maghreb-ian
algérien
stand-ard
systems
Nicolaï
Mouton
mauritani-en
Gotho-burg-ensis
socio-linguist-ique
plur-al
archiv-al
Arab-ian
drop-ped
dihāt
de-velop-ed
ṣuḥbat
kitāba
kitābat
com-mercial
eight-eenth
region
Senegal
mechan-ics
Maur-itan-ia
Ḥassān-iyya
circum-cision
cor-relation
labio-velar-ization
vowel
vowels
cert-ain
īggīw
series
in-tegrates
dur-ative
inter-dent-als
gen-itive
Tuareg
tălămut
talawmāyət
part-icular
part-icular-ly
con-diment
vill-age
bord-er
polit-ical
Wiesbaden
Uni-vers-idad
Geuthner
typo-logie
Maur-itanie
nomades
Maur-itan-ian
dia-lecto-logy
Sahar-iennes
Uni-vers-ity
de-scend-ants
NENA-speak-ing
speak-ing
origin-al
re-captured
in-habit-ants
ethnic
minor-it-ies
drama-tic
local
long-stand-ing
regions
Nineveh
settle-ments
Ṣəndor
Mandate
sub-stitut-ing
ortho-graphy
re-fer-enced
origin-ate
twenti-eth
typ-ic-al-ly
Hobrack
never-the-less
character-ist-ics
character-ist-ic
masc-uline
coffee
ex-clus-ive-ly
verb-al
re-ana-ly-se-d
simil-ar-ities
de-riv-ation
im-pera-tive
part-iciple
dis-ambi-gu-ation
dis-ambi-gu-a-ing
phen-omen-on
phen-omen-a
traktar
com-mun-ity
com-mun-ities
dis-prefer-red
ex-plan-ation
con-struction
wide-spread
us-ual-ly
region-al
Bulut
con-sider-ation
afro-asia-tici
Franco-Angeli
Phono-logie
Volks-kundliche
dia-lectes
dia-lecte
select-ed
dis-appear-ance
media
under-stand-able
public-ation
second-ary
e-ject-ive
re-volu-tion
re-strict-ive
Gasparini
mount-ain
mount-ains
yellow
label-ing
trad-ition-al-ly
currently
dia-chronic
}
\hyphenation{
affri-ca-te
affri-ca-tes
com-ple-ments
homo-phon-ous
start-ed
Meso-potam-ian
morpho-phono-logic-al-ly
morpho-phon-em-ic-s
Palestin-ian
re-present-ed
Ki-nubi
ḥawār-iyy-ūn
archa-ic-ity
fuel-ed
de-velop-ment
pros-od-ic
Arab-ic
in-duced
phono-logy
possess-um
possess-ive-s
templ-ate
spec-ial
espec-ial-ly
nat-ive
pass-ive
clause-s
potent-ial-ly
Lusignan
commun-ity
tobacco
posi-tion
Cushit-ic
Middle
with-in
re-finit-iz-ation
langu-age-s
langu-age
diction-ary
glossary
govern-ment
eight
counter-part
nomin-al
equi-valent
deont-ic
ana-ly-sis
Malt-ese
un-fortun-ate-ly
scient-if-ic
Catalan
Occitan
ḥammāl
cross-linguist-ic-al-ly
predic-ate
major-ity
ignor-ance
chrono-logy
south-western
mention-ed
borrow-ed
neg-ative
de-termin-er
European
under-mine
detail
Oxford
Socotra
numer-ous
spoken
villages
nomad-ic
Khuze-stan
Arama-ic
Persian
Ottoman
Ottomans
Azeri
rur-al
bi-lingual-ism
borrow-ing
prestig-ious
dia-lects
dia-lect
allo-phone
allo-phones
poss-ible
parallel
parallels
pattern
article
common-ly
respect-ive-ly
sem-antic
Moroccan
Martine
Harrassowitz
Grammatic-al-ization
grammatic-al-ization
Afro-asiatica
Afro-asiatic
continu-ation
Semit-istik
varieties
mono-phthong
mono-phthong-ized
col-loquial
pro-duct
document-ary
ex-ample-s
ex-ample
termin-ate
element-s
Aramaeo-grams
Centr-al
idioms
Arab-ic
Dadan-it-ic
sub-ordin-ator
Thamud-ic
difficult
common-ly
Revue
Bovingdon
under
century
attach
attached
bundle
graph-em-ic
graph-emes
cicada
contrast-ive
Corriente
Andalusi
Kossmann
morpho-logic-al
inter-action
dia-chroniques
islámica
occid-ent-al-ismo
dialecto-logie
Reichert
coloni-al
Milton
diphthong-al
linguist-ic
linguist-ics
affairs
differ-ent
phonetic-ally
kilo-metres
stabil-ization
develop-ments
in-vestig-ation
Jordan-ian
notice-able
level-ed
migrants
con-dition-al
certain-ly
general-ly
especial-ly
af-fric-ation
Jordan
counter-parts
com-plication
consider-ably
inter-dent-al
com-mun-ity
inter-locutors
com-pon-ent
region-al
socio-historical
society
simul-taneous
phon-em-ic
roman-ization
Classic-al
funeral
Kurmanji
pharyn-geal-ization
vocab-ulary
phon-et-ic
con-sonant
con-sonants
special-ized
latter
latters
in-itial
ident-ic-al
cor-relate
geo-graphic-al-ly
Öpengin
Kurd-ish
in-digen-ous
sunbul
Christ-ian
Christ-ians
sekin-în
fatala
in-tegration
dia-lect-al
Matras
morpho-logy
in-tens-ive
con-figur-ation
im-port-ant
com-plement
ḥaddād
e-merg-ence
Benjmamins
struct-ure
em-pir-ic-al
Orient-studien
Anatolia
American
vari-ation
Jastrow
Geoffrey
Yarshater
Ashtiany
Edmund
Mahnaz
En-cyclo-pædia
En-cyclo-paedia
En-cyclo-pedia
Leiden
dia-spora
soph-is-ic-ated
Sasan-ian
every-day
domin-ance
Con-stitu-tion-al
religi-ous
sever-al
Manfredi
re-lev-ance
re-cipi-ent
pro-duct-iv-ity
turtle
Morocco
ferman
Maghreb-ian
algérien
stand-ard
systems
Nicolaï
Mouton
mauritani-en
Gotho-burg-ensis
socio-linguist-ique
plur-al
archiv-al
Arab-ian
drop-ped
dihāt
de-velop-ed
ṣuḥbat
kitāba
kitābat
com-mercial
eight-eenth
region
Senegal
mechan-ics
Maur-itan-ia
Ḥassān-iyya
circum-cision
cor-relation
labio-velar-ization
vowel
vowels
cert-ain
īggīw
series
in-tegrates
dur-ative
inter-dent-als
gen-itive
Tuareg
tălămut
talawmāyət
part-icular
part-icular-ly
con-diment
vill-age
bord-er
polit-ical
Wiesbaden
Uni-vers-idad
Geuthner
typo-logie
Maur-itanie
nomades
Maur-itan-ian
dia-lecto-logy
Sahar-iennes
Uni-vers-ity
de-scend-ants
NENA-speak-ing
speak-ing
origin-al
re-captured
in-habit-ants
ethnic
minor-it-ies
drama-tic
local
long-stand-ing
regions
Nineveh
settle-ments
Ṣəndor
Mandate
sub-stitut-ing
ortho-graphy
re-fer-enced
origin-ate
twenti-eth
typ-ic-al-ly
Hobrack
never-the-less
character-ist-ics
character-ist-ic
masc-uline
coffee
ex-clus-ive-ly
verb-al
re-ana-ly-se-d
simil-ar-ities
de-riv-ation
im-pera-tive
part-iciple
dis-ambi-gu-ation
dis-ambi-gu-a-ing
phen-omen-on
phen-omen-a
traktar
com-mun-ity
com-mun-ities
dis-prefer-red
ex-plan-ation
con-struction
wide-spread
us-ual-ly
region-al
Bulut
con-sider-ation
afro-asia-tici
Franco-Angeli
Phono-logie
Volks-kundliche
dia-lectes
dia-lecte
select-ed
dis-appear-ance
media
under-stand-able
public-ation
second-ary
e-ject-ive
re-volu-tion
re-strict-ive
Gasparini
mount-ain
mount-ains
yellow
label-ing
trad-ition-al-ly
currently
dia-chronic
}
\hyphenation{
affri-ca-te
affri-ca-tes
com-ple-ments
homo-phon-ous
start-ed
Meso-potam-ian
morpho-phono-logic-al-ly
morpho-phon-em-ic-s
Palestin-ian
re-present-ed
Ki-nubi
ḥawār-iyy-ūn
archa-ic-ity
fuel-ed
de-velop-ment
pros-od-ic
Arab-ic
in-duced
phono-logy
possess-um
possess-ive-s
templ-ate
spec-ial
espec-ial-ly
nat-ive
pass-ive
clause-s
potent-ial-ly
Lusignan
commun-ity
tobacco
posi-tion
Cushit-ic
Middle
with-in
re-finit-iz-ation
langu-age-s
langu-age
diction-ary
glossary
govern-ment
eight
counter-part
nomin-al
equi-valent
deont-ic
ana-ly-sis
Malt-ese
un-fortun-ate-ly
scient-if-ic
Catalan
Occitan
ḥammāl
cross-linguist-ic-al-ly
predic-ate
major-ity
ignor-ance
chrono-logy
south-western
mention-ed
borrow-ed
neg-ative
de-termin-er
European
under-mine
detail
Oxford
Socotra
numer-ous
spoken
villages
nomad-ic
Khuze-stan
Arama-ic
Persian
Ottoman
Ottomans
Azeri
rur-al
bi-lingual-ism
borrow-ing
prestig-ious
dia-lects
dia-lect
allo-phone
allo-phones
poss-ible
parallel
parallels
pattern
article
common-ly
respect-ive-ly
sem-antic
Moroccan
Martine
Harrassowitz
Grammatic-al-ization
grammatic-al-ization
Afro-asiatica
Afro-asiatic
continu-ation
Semit-istik
varieties
mono-phthong
mono-phthong-ized
col-loquial
pro-duct
document-ary
ex-ample-s
ex-ample
termin-ate
element-s
Aramaeo-grams
Centr-al
idioms
Arab-ic
Dadan-it-ic
sub-ordin-ator
Thamud-ic
difficult
common-ly
Revue
Bovingdon
under
century
attach
attached
bundle
graph-em-ic
graph-emes
cicada
contrast-ive
Corriente
Andalusi
Kossmann
morpho-logic-al
inter-action
dia-chroniques
islámica
occid-ent-al-ismo
dialecto-logie
Reichert
coloni-al
Milton
diphthong-al
linguist-ic
linguist-ics
affairs
differ-ent
phonetic-ally
kilo-metres
stabil-ization
develop-ments
in-vestig-ation
Jordan-ian
notice-able
level-ed
migrants
con-dition-al
certain-ly
general-ly
especial-ly
af-fric-ation
Jordan
counter-parts
com-plication
consider-ably
inter-dent-al
com-mun-ity
inter-locutors
com-pon-ent
region-al
socio-historical
society
simul-taneous
phon-em-ic
roman-ization
Classic-al
funeral
Kurmanji
pharyn-geal-ization
vocab-ulary
phon-et-ic
con-sonant
con-sonants
special-ized
latter
latters
in-itial
ident-ic-al
cor-relate
geo-graphic-al-ly
Öpengin
Kurd-ish
in-digen-ous
sunbul
Christ-ian
Christ-ians
sekin-în
fatala
in-tegration
dia-lect-al
Matras
morpho-logy
in-tens-ive
con-figur-ation
im-port-ant
com-plement
ḥaddād
e-merg-ence
Benjmamins
struct-ure
em-pir-ic-al
Orient-studien
Anatolia
American
vari-ation
Jastrow
Geoffrey
Yarshater
Ashtiany
Edmund
Mahnaz
En-cyclo-pædia
En-cyclo-paedia
En-cyclo-pedia
Leiden
dia-spora
soph-is-ic-ated
Sasan-ian
every-day
domin-ance
Con-stitu-tion-al
religi-ous
sever-al
Manfredi
re-lev-ance
re-cipi-ent
pro-duct-iv-ity
turtle
Morocco
ferman
Maghreb-ian
algérien
stand-ard
systems
Nicolaï
Mouton
mauritani-en
Gotho-burg-ensis
socio-linguist-ique
plur-al
archiv-al
Arab-ian
drop-ped
dihāt
de-velop-ed
ṣuḥbat
kitāba
kitābat
com-mercial
eight-eenth
region
Senegal
mechan-ics
Maur-itan-ia
Ḥassān-iyya
circum-cision
cor-relation
labio-velar-ization
vowel
vowels
cert-ain
īggīw
series
in-tegrates
dur-ative
inter-dent-als
gen-itive
Tuareg
tălămut
talawmāyət
part-icular
part-icular-ly
con-diment
vill-age
bord-er
polit-ical
Wiesbaden
Uni-vers-idad
Geuthner
typo-logie
Maur-itanie
nomades
Maur-itan-ian
dia-lecto-logy
Sahar-iennes
Uni-vers-ity
de-scend-ants
NENA-speak-ing
speak-ing
origin-al
re-captured
in-habit-ants
ethnic
minor-it-ies
drama-tic
local
long-stand-ing
regions
Nineveh
settle-ments
Ṣəndor
Mandate
sub-stitut-ing
ortho-graphy
re-fer-enced
origin-ate
twenti-eth
typ-ic-al-ly
Hobrack
never-the-less
character-ist-ics
character-ist-ic
masc-uline
coffee
ex-clus-ive-ly
verb-al
re-ana-ly-se-d
simil-ar-ities
de-riv-ation
im-pera-tive
part-iciple
dis-ambi-gu-ation
dis-ambi-gu-a-ing
phen-omen-on
phen-omen-a
traktar
com-mun-ity
com-mun-ities
dis-prefer-red
ex-plan-ation
con-struction
wide-spread
us-ual-ly
region-al
Bulut
con-sider-ation
afro-asia-tici
Franco-Angeli
Phono-logie
Volks-kundliche
dia-lectes
dia-lecte
select-ed
dis-appear-ance
media
under-stand-able
public-ation
second-ary
e-ject-ive
re-volu-tion
re-strict-ive
Gasparini
mount-ain
mount-ains
yellow
label-ing
trad-ition-al-ly
currently
dia-chronic
} 
  \togglepaper[1]%%chapternumber
}{}

\begin{document}
\maketitle 
  


 \section{Current state and historical development}


 \subsection{Historical development of Beja}


Beja is the sole language of the {Northern} \ili{Cushitic} branch of the \ili{Afro-Asiatic} phylum. Recent archaeological discoveries show growing evidence that Beja is related to the extinct languages of the \isi{Medjay} (from which the ethnonym Beja is derived; \citealt[1175]{Rilly2014}), and \isi{Blemmye} tribes, first attested on \ili{Egyptian} inscriptions of the Twelfth Dynasty for the former, and on a Napatan stela of the late seventh century BCE for the latter. For recent discussions, see \citet{Browne2003}; \citet{El-Sayed2011}; \citet{Zibelius-Chen2014}; \citet{Rilly2014}; and \citet{Rilly2018}. The Medjays were nomads living in the eastern \ili{Nubian} Desert, between the first and second cataracts of the River Nile. The Blemmyes invaded and took part in defeating the \ili{Meroitic} kingdom, fought against the Romans up to the Sinai, and ruled Nubia from Talmis (modern Kalabsha, between Luxor and Aswan) for a few decades, before being defeated themselves by the \isi{Noubades} around 450 CE (\citealt{Rilly2018}). In late antiquity, the linguistic situation involved, in northern Lower Nubia, \ili{Cushitic} languages, {Northern} {Eastern} \ili{Sudanic} languages, to which \ili{Meroitic} and \ili{Nubian} belong, also \ili{Coptic} and \ili{Greek} to some extent, and in the south, Ethio-\ili{Semitic}. It is likely that there was mutual influence to an extent that is difficult to disentangle today. 


 
 \subsection{Current situation of Beja}


The Beja territory has shrunk a lot since late antiquity, and Beja (\textit{biɖawijeːt}) is mainly spoken today in the Red Sea and Kassala States in eastern Sudan, in the dry lands between the Red Sea and the Atbara River. The 1993 census, the last one to include a language question, recorded some 1,100,000 Beja speakers, and there is probably at least double that figure today. There are also some 60,000 speakers in northern Eritrea, and there may be still a few speakers left in Egypt, in the Nile valley at Aswan and Daraw, and along the coast towards Marsa Alam (\citealt{Morin1995}; \citealt{Wedekind2012}). In Sudan today, Beja speakers have also settled in Khartoum and cities in central and western Sudan (\citealt[67]{HamidAhmed2005book}).

All Bejas today are \isi{Muslims}. They consider themselves Bedouins, and call themselves \textit{arab} ‘Arab’;\footnote{In Sudan the term \textit{ʕarab} is widely used for referring to nomad groups in general, and not only to ethnically defined Arabs. Thanks to Stefano Manfredi for this information.\ia{Manfredi, Stefano@Manfredi, Stefano}.} they call the ethnic Arabs \textit{balawjeːt}. Before the introduction of modern means of transportation, they were traditionally the holders of the caravan trade in the desert towards the west, south and north of their territory, and they still move between summer and winter pastures with their cattle. They also produce sorghum and millet for daily consumption, and fruits and vegetables in the oases. The arrival of \isi{Rashaida} migrants from \isi{Saudi Arabia} in the nineteenth century created tensions in an area with meagre resources, but the first contemporary important social changes took place during the British mandate with the agricultural development of the Gash and Tokar areas, and the settlement of non-Beja farmers. The droughts of the mid-1980s brought about a massive exodus towards the cities, notably Port Sudan and Kassala, followed by job diversification, and increased access to education in \ili{Arabic}, although not generalized, especially for girls, who rarely go beyond primary level (\citealt{HamidAhmed2005book}).

Beja is mostly an oral language. In Eritrea, a \ili{Latin} script was introduced in schools after independence in 1993, but in Sudan no education in Beja exists. Attempts made by the Summer Institute of Linguistics and at the University of the Red Sea to implement an \ili{Arabic}-based script did not come to fruition. On the other hand, in the last few years school teachers in rural areas have begun to talk more and more in Beja in order to fight illiteracy (in \ili{Arabic}) and absenteeism (\citealt{Onour2015}).

\section{Arabic--Beja contact}

Contact between Bejas and Arabs started as early as the beginning of Islamization, and through trade relations with Muslim Egypt, as well as Arab incursions in search of gold and emerald. Evidence of these contacts lies in the early Arabicization of Beja anthroponyms (\citealt{Záhořík2007}). The date of the beginning of Islamization differs according to authors, but it seems it started as early as the tenth century, and slowly expanded until it became the sole religion between the eighteenth and nineteenth centuries (\citealt{Záhořík2007}).

We have no information concerning the onset and spread of Beja--\ili{Arabic} \isi{bilingualism}. It is thus often impossible to figure out if a \isi{transfer} occurred through Beja-dominant speakers or was imposed by fluent Beja--\ili{Arabic} bilingual speakers, and consequently to decide whether a contact-induced feature belongs to the borrowing or to the \isi{imposition} type of \isi{transfer} as advocated by Van Coetsem (\citeyear{VanCoetsem1988,VanCoetsem2000}) and his followers. What is certain though, is that socio-historical as well as linguistic evidence speaks in favour of Beja--\ili{Arabic} \isi{bilingualism} as an ancient phenomenon, but in unknown proportion among the population. With the spread of Islam since the Middle Ages, contact with \ili{Arabic} became more and more prevalent in Sudan. 

In this country, which will be the focus of this chapter, \isi{bilingualism} with Suda\-nese \ili{Arabic} is frequent, particularly for men, and expanding, including among women in cities and villages, but to a lesser extent. Bejas in Port Sudan are also in contact with varieties of \ili{Yemeni} \ili{Arabic}. Rural Bejas recently settled at the periphery of the big cities have the reputation of being more \isi{monolingual} than others, which was still the case fifteen years ago \citep{Vanhove2003}.

The Beja language is an integral part of the social and cultural \isi{identity} of the people, but it is not a necessary component. Tribes and clans that have switched to \ili{Arabic}, or \ili{Tigre}, such as the \isi{Beni Amer}, are considered Bejas. Beja is \isi{prestigious}, since it allows its speakers to uphold the ethical values of the society, and is considered to be aesthetically pleasing due to its allusive character. The attitude towards \ili{Arabic} is ambivalent. It is perceived as taboo-less, and thus contrary to the rules of honour, nevertheless it is possible to use it without transgressing them. \ili{Arabic} is also \isi{prestigious} because it is the language of \isi{social promotion} and modernity (\citealt{HamidAhmed2005article}). Language attitudes are rapidly changing, and there is some concern among the Beja \isi{diaspora} about the \isi{future} of the Beja language, even though it cannot be considered to be \isi{endangered}. Some parents avoid speaking Beja to their children, for fear that it would interfere with their learning of \ili{Arabic} at school, leaving to the grandparents the \isi{transmission} of Beja (\citealt{Wedekind2012}; \citealt{Vanhove2017}). But there is no reliable quantitative or qualitative sociolinguistic study of this phenomenon. Code-switching between Beja and \ili{Arabic} is spreading but understudied.

This sketch of the sociolinguistic situation of Beja speaks for at least two types of \isi{transfer}: (i) borrowing, where the agents of \isi{transfer} are dominant in the \isi{recipient language} (Beja); (ii) \isi{convergence} phenomena, since the difference in linguistic dominance between the languages of the bilingual speakers tends to be really small (at least among male speakers today, and probably earlier in the history of Beja; see Van Coetsem \citeyear{VanCoetsem1988}: 87). Imposition has probably also occurred of course, but it is not always easy to prove.


 \section{Contact-induced changes in Beja}


 \subsection{Phonology}


The few contact-induced changes in Beja phonology belong to the borrowing type.

The phonological system of Beja counts 21 consonantal phonemes, presented in \tabref{tab:vanhove:1}.

\begin{table}[H]
\fittable{\begin{tabularx}{\textwidth}{ l X X X X X X X X}
\lsptoprule
& \rotatebox{66}{Bilabial} & \rotatebox{66}{Labio-dental} & \rotatebox{66}{Alveolar} & \rotatebox{66}{Retroflex} & \rotatebox{66}{Palatal} & \rotatebox{66}{Velar} & \rotatebox{66}{Labio-velar} & \rotatebox{66}{Laryngeal}\\\midrule
Plosive
& & f & t & ʈ & & k & kʷ & ʔ \\
& b & & d & ɖ & & g & gʷ \\
Affricate
& & & & & ʤ \\
Fricative
& & & s & & ʃ & & & h \\
Nasal
& m & & n & & & \\
Trill
& & & r & & & & \\
Lateral 
& & & l & & & & \\
Approximant
& w & & & & j & \\\lspbottomrule
\end{tabularx}}
\caption{Beja consonants}
\label{tab:vanhove:1}
\end{table}

The \isi{voiced post-alveolar affricate} \textit{ʤ} (often realized as a voiced palatal plosive [ɟ] as in \ili{Sudanese} \ili{Arabic}) deserves attention as a possible outcome of contact with \ili{Arabic}. Since Reinisch (\citeyear{Reinisch1893}: 17), it is usually believed that this affricate is only present in \ili{Arabic} \isi{loanwords} and is not a \isi{phoneme} (\citealt{Roper1928}; \citealt{Hudson1976}; \citealt{Morin1995}). The existence of a number of minimal pairs in word-initial position invalidates the latter analysis: \textit{ʤiːk} ‘rooster’ {\textasciitilde} \textit{ʃiːk} ‘chewing tobacco’; \textit{ʤhar} ‘chance’ {\textasciitilde} \textit{dhar} ‘bless’; \textit{ʤaw} ‘quarrel’ {\textasciitilde} \textit{ɖaw} ‘jungle’ {\textasciitilde} \textit{ʃaw} ‘pregnancy’ {\textasciitilde} \textit{gaw} ‘house’ (\citealt{Vanhove2017}). As for the former claim, there are actually a few lexical items such as \textit{bʔaʤi} ‘bed’, \textit{gʷʔaʤi} ‘one-eyed’ (\textit{gʷʔad} ‘two eyes’), that cannot be traced back to \ili{Arabic} (the latter is pan-\ili{Cushitic}; \citealt{Blažek2000}). Nevertheless, it is the case that most items containing this \isi{phoneme} do come from (or through) (\ili{Sudanese}) \ili{Arabic}: \textit{aːlaʤ} ‘tease’, \textit{aʤiːn} ‘dough’, \textit{aʤib} ‘please’, \textit{ʔaʤala} ‘bicycle’, \textit{ʔiʤir} ‘divine reward’, \textit{ʤaːhil} ‘small child’, \textit{ʤabana} ‘coffee’, \textit{ʤallaːj} ‘because of’, \textit{ʤallab} ‘fish’, \textit{ʤanna} ‘paradise’, \textit{ʤantaːji} ‘djinn’, \textit{ʤarikaːn} ‘jerrycan’, \textit{ʤeːb} ‘pocket’, \textit{ʤhaliː} ‘coal’, \textit{ʤimʔa} ‘week’, \textit{ʤins} ‘sort’, \textit{ʤuwwa} ‘inside’, \textit{faʤil} ‘morning’, \textit{finʤaːn} ‘cup’, \textit{hanʤar} ‘dagger’, \textit{hiʤ} ‘pilgrimage’, \textit{maʤaʔa} ‘famine’, \textit{maʤlis} ‘reconciliation meeting’, \textit{siʤin} ‘prison’, \textit{tarʤimaːl} ‘translator’, \textit{waʤʤa} ‘appointment’, and \textit{xawaʤa} ‘foreigner’. It is clear that \textit{ʤ} is not marginal anymore. However \textit{ʤ} is unstable: it has several dialectal variants, \textit{ʧ,} \textit{g} and \textit{d}, and may alternate with the dental \textit{d} or retroflex \textit{ɖ}, in the original Beja lexicon (\textit{ʤiwʔoːr/ɖiwʔoːr} ‘honourable man’) as well as in \isi{loanwords} (\textit{aʤiːn/aɖiːn} ‘dough’) (\citealt{VanhoveHamidAhmed2011}; \citealt{Vanhove2017}). In my data, which counts some 50 male and female speakers of all age groups, this is rarely the case, meaning that there is a good chance that this originally marginal \isi{phoneme} will live on under the influence of (\ili{Sudanese}) \ili{Arabic}.

There are two other consonants in \ili{Arabic} \isi{loanwords} that are regularly used by the Beja speakers: \textit{z} and \textit{x}, neither of which can be considered phonemes since there are no minimal pairs.

Blažek (\citeyear[130]{Blažek2007}) established a regular correspondence between Beja \textit{d} and Proto-East-\ili{Cushitic} *z. In contemporary Beja \textit{z} only occurs in recent \isi{loanwords} from \ili{Sudanese} \ili{Arabic} such as \textit{ʤaza} ‘wage’, \textit{ʤoːz} ‘pair’, \textit{rizg} ‘job’, \textit{wazʔ} ‘offer’, \textit{xazna} ‘treasure’, \textit{zamaːn} ‘time’, \textit{zirʔa} ‘field’, \textit{zuːr} ‘visit’. It may alternate with \textit{d}, even within the speech of the same speaker as free variants, e.g. \textit{damaːn,} \textit{dirʔa}, \textit{duːr}. The \isi{fricative alveolar} pronunciation is more frequent among city dwellers, who are more often bilingual. It is difficult to ascertain whether Beja is in the process of re-acquiring the voiced fricative through contact with \ili{Sudanese} \ili{Arabic}, or whether it will undergo the same evolution to a dental stop as in the past.

A few recent \ili{Arabic} \isi{loanwords} may also retain the \isi{voiceless velar fricative} \textit{x} (see also \citealt{ManfrediSimeone-SenelleTosco2015}: 304--305): \textit{xazna} ‘treasure’, \textit{xawaʤa} ‘foreigner’, \textit{xaddaːm} ‘servant’, \textit{xaːtar} ‘be dangerous’, \textit{aːxar} ‘last’. In my data, this is usually the case in the speech of fluent bilingual speakers. We thus have here a probable \isi{imposition} type of \isi{transfer}. In older borrowings, even among these speakers, \ili{Arabic} \textit{x} shifted to \textit{h} (\textit{xajma} > \textit{heːma} ‘tent’). It may be because these older loans spread in a community which was at that time composed mainly of Beja-dominant speakers, but we have no means of proving this hypothesis.


 
 \subsection{Morphology} \label{morphologyv}
 \subsubsection{General remarks}

Most \ili{Cushitic} languages only have \isi{concatenative morphology}, the \isi{stem} and pattern schema being at best highly marginal (\citealt[256]{Cohen1988}). In addition to Beja, \ili{Afar} and \ili{Saho} (Lowland East-\ili{Cushitic} branch), Beja’s geographically closest sisters, are exceptions, and all three languages use also \isi{non-concatenative} morphology. In \ili{Afar} and \ili{Saho} it is far less pervasive than in Beja; in particular they do not use \isi{vocalic alternation} for verbal \isi{derivation}, this feature being restricted to the verb flexion of a minority of underived verbs.

Even though Beja and \ili{Arabic} share a similar type of morphology, the following overview shows that each language has developed its own system. Although they have been in contact for centuries, neither small-scale nor massive borrowing from \ili{Arabic} morphological patterns can be postulated for the Beja data. An interpretation in terms of a \isi{convergence} phenomenon is more relevant, both in terms of semantics and forms.

Non-\isi{concatenative morphology} concerns an important portion of the lexicon: a large part of the verb morphology (conjugations, verb derivations, verbal noun derivations), and part of the noun morphology (adjectives, nouns, “internal” plurals, and to a lesser extent, place and instrument nouns). In what follows, I build on \citet{Vanhove2012} and \citet{Vanhove2017}, correcting some inaccuracies.


 \subsubsection{Verb morphology}

Only one of the two Beja verb classes, the one conjugated with prefixes (or infixes), belongs to \isi{non-concatenative} morphology. This verb class (V1) is formed of a \isi{stem} which undergoes ablaut varying with tense--aspect--mood (TAM), person and number, to which prefixed personal indices for all TAMs are added (plural and \isi{gender} morphemes are also suffixes). V1 is diachronically the oldest pattern, which survives only in a few other \ili{Cushitic} languages. In Beja V1s are the majority (57\%), as against approximately 30\% in \ili{Afar} and \ili{Saho}, and only five verbs in \ili{Somali} and South \ili{Agaw} (\citealt[256]{Cohen1988}). \tabref{tab:vanhove:2} provides examples in the perfective and imperfective for bi-consonantal and tri-consonantal \isi{roots}. 

\begin{table}
\begin{tabular}{lll}
\lsptoprule
& Bi-consonantal \textit{dif} ‘go’ & Tri-consonantal \textit{kitim} ‘arrive’\\\midrule
\textsc{pfv} & \textit{i-dif} ‘he went’  & \textit{i-ktim} ‘he arrived’\\
{} & \textit{i-dif-na} ‘they went’ & \textit{i-ktim-na} ‘they arrived’\\
\textsc{ipfv} & \textit{i-n-diːf} ‘he goes’ & \textit{k<an>tiːm} ‘he arrives’\\
{} &\textit{eː-dif-na} ‘they go’ & \textit{eː-katim-na} ‘they arrive’\\
\lspbottomrule
\end{tabular} 
\caption{Perfective and imperfective patterns}
\label{tab:vanhove:2}
\end{table}

Prefix conjugations are used in \ili{Arabic} varieties and South \ili{Semitic} languages but their functions and origins are different. In South \ili{Semitic}, the prefix conjugation has an aspectual value of imperfective, while in \ili{Cushitic} it marks a particular morphological verb class. The \ili{Cushitic} prefix conjugation (in the singular) goes back to \isi{auxiliary verb(s)} meaning ‘say’ or ‘be’, while the prefix conjugation of South \ili{Semitic} has various origins, none of them including a verb ‘say’ or ‘be’ (\citealt{Cohen1984}). Although different \isi{grammaticalization} chains took place in the two branches of \ili{Afro-Asiatic}, this suggests that the \isi{root-and-pattern} system might have already been robust in Beja at an ancient stage of the language. It is noteworthy that there are at least traces of \isi{vocalic alternation} between the perfective and the imperfective in all \ili{Cushitic} branches (\citealt{Cohen1984}: 88--102), thus reinforcing the hypothesis of an ancient \isi{root-and-pattern} schema in Beja. In what proportion this schema was entrenched in the morphology of the proto-\ili{Cushitic} lexicon is impossible to decide.

Verb \isi{derivation} of V1s is also largely \isi{non-concatenative}. Beja is the only \ili{Cushitic} language which uses \isi{qualitative ablaut} in the \isi{stem} for the \isi{formation} of seman\-tic and voice \isi{derivation}. The ablaut can combine with prefixes. 

\tabref{tab:V1} presents the five verb \isi{derivation} patterns with ablaut, and \tabref{tab:comp} shows the absence of correspondence between the Beja and \ili{Arabic} (\ili{Classical} and \ili{Sudanese}) patterns. \ili{Sudanese} patterns are extracted from Bergman (\citeyear[32--34]{Bergman2002}), who does not provide semantic values.

\begin{table}
\begin{tabular}{lll}\lsptoprule 
& Monosyllabic V1 &  Plurisyllabic V1\\\midrule
\textsc{int} & \textit{boːs} (\textit{bis} ‘burry’) & \textit{kaːtim} (\textit{kitim} ‘arrive’)\\
\textsc{mid} & \textit{faf} (\textit{fif} ‘pour’) & \textit{rimad} (\textit{rimid} ‘avenge’)\\
\textsc{pass} & \textit{aːtoː-maːn} (\textit{min} ‘shave’) & \textit{at-dabaːl} (\textit{dibil} ‘gather’)\\
&  & \textit{am-heːjid} (\textit{haːjid} ‘sew’)\\
\textsc{recp} & \textit{amoː-gaːd} (\textit{gid} ‘throw’) & \textit{am-garaːm} (\textit{girim} ‘be inimical’)\\
\textsc{caus} &  & \textit{si-katim} (\textit{kitim} ‘arrive’)\\
\lspbottomrule
\end{tabular} 
\caption{V1 derivation patterns with ablaut}
\label{tab:V1}
\end{table}

\begin{table}
\begin{tabularx}{\textwidth}{lQQQ} 
\lsptoprule
& Beja Plurisyllabic V1 & \ili{Classical} \ili{Arabic} & \ili{Sudanese} \ili{Arabic}\\
\midrule
\textsc{int} & CaːCa \textit{kaːtim} < \textit{kitim} ‘arrive’ & CaCCaCa, CaːCaCa & CaCCaC, CaːCaC\\
\textsc{mid} & CiCaC \textit{rimad} < \textit{rimid} ‘avenge’ & CuCiCa, iC<t>aCaCa, ta-CaCCaCa, ta-CaːCaCa, in-CaCaCa, tas-CaCaCa, ista-CaCaCa & it-CaCCaC, it\nobreakdash-CaCaC\\
\textsc{pass} & at-CaCaːC \textit{at\nobreakdash-dabaːl} < \textit{dibil} ‘gather’ & CuCiCa, iC<t>aCaCa, & it-CaːCaC, it\nobreakdash-CaCaC, CiCiC, in\nobreakdash-CaCaC\\
& am-CeːCiC \textit{am\nobreakdash-heːjid} < \textit{haːjid} ‘sew’ & ta-CaCCaCa, ta-CaːCaCa, in-CaCaCa, ista-CaCaCa & \\
\textsc{recp} & am-CaCaːC \textit{am\nobreakdash-garaːm} < \textit{girim} ‘be inimical’ & CaːCaCa , ta\nobreakdash-CaːCaCa, iC<t>aCaCa & CaːCaC, it\nobreakdash-CaːCaC\\
\textsc{caus} & si-CaCiC \textit{si-katim} < \textit{kitim} ‘arrive’ & CaCCaCa, ʔa-CCaCa & CaCCaC, a\nobreakdash-CCaC\\
\lspbottomrule
\end{tabularx} 
\caption{Comparison between Beja and Arabic derivation patterns}
\label{tab:comp}
\end{table}

Among the \ili{Semitic} languages, an \isi{intensive} pattern similar to the Beja one is only known in some \ili{Modern South Arabian} languages spoken in eastern \isi{Yemen} (not in contact with Beja), where it is also used for causation and transitivization \citep[1091]{Simeone-Senelle2011}. The Modern \ili{South Arabian} languages are close relatives of Ethio-\ili{Semitic} languages and it is usually considered that the latter were brought to the Horn of Africa by South-Arabian speakers (\citealt{Ullendorf1955}). However, this ablaut pattern was not retained in Ethio-\ili{Semitic}. It is also unknown in \ili{Cushitic}. In \ili{Classical} \ili{Arabic}, the plurisyllabic pattern does not have an \isi{intensive} value, but a goal or sometimes \isi{reciprocal} meaning.

Beja is the sole \ili{Cushitic} language which differentiates between active and middle voices by means of \isi{vocalic alternation}. Remnants of this pattern exist in some \ili{Semitic} languages, among them \ili{Arabic}, in a fossilized form. 

In \ili{Cushitic}, \isi{qualitative ablaut} for the \isi{passive} voice only occurs in Beja. Passive \isi{formation} through ablaut exists in \ili{Classical} and \ili{Sudanese} \ili{Arabic}, but with different vowels. Bergman (\citeyear[34]{Bergman2002}) mentions that “a handful of verbs in S[udanese] A[rabic]” can be formed this way. For Stefano Manfredi\ia{Manfredi, Stefano@Manfredi, Stefano} (personal communication) it is a productive pattern in this \ili{Arabic} variety.

Like the \isi{passive} voice, the \isi{reciprocal} is characterized by a \isi{qualitative ablaut} in \textit{aː} in the \isi{stem}, but the prefix is different and consists of \textit{am(oː)-}. \textit{m} is not used for verbal \isi{derivation} in \ili{Arabic}, which uses the same ablaut, but for the first vowel of disyllabic stems, to express, marginally, the \isi{reciprocal} of the \isi{base form}. Most often the \isi{reciprocal} meaning is expressed by other forms with the \textit{t-} prefixed or infixed to the derived form or the \isi{base form}. In some other \ili{Cushitic} languages \textit{\nobreakdash-m} is used as a suffix for \isi{passive} or \isi{middle voice} (without ablaut). In Beja \textit{m-} can also marginally be used as a \isi{passive} marker, together with ablaut, for a few transitive \isi{intensive} verbs: \textit{ameː\nobreakdash-saj} ‘be flayed’, \textit{ameː\nobreakdash-biɖan} ‘be forgotten’.

Although a suffix \textit{{}-s} (not a prefix as in Beja) is common in \ili{Cushitic}, Beja is once more the only \ili{Cushitic} language which uses ablaut for the \isi{causative} derived form. Neither ablaut nor the \textit{s-} prefix exist in \ili{Arabic}. \ili{Arabic} uses different patterns for the \isi{causative}: the same as the \isi{intensive} one, i.e. with a geminated second \isi{root} consonant, and the (Ɂ)a-CcaC(a) pattern.

This brief overview shows that Beja has not borrowed patterns from (\ili{Sudanese}) \ili{Arabic}, but has at best similar, but not exact, \isi{cognate} patterns which are marginal in both \ili{Classical} and \ili{Sudanese} \ili{Arabic}.

Beja also has four non-finite verb forms. The simultaneity \isi{converb} of V1s is the only one with \isi{non-concatenative} morphology. The affirmative \isi{converb} is marked for both verb classes with a suffix \textit{{}-eː} added to the \isi{stem}: \textit{gid} ‘throw’, \textit{gid\nobreakdash-eː} ‘while throwing’; \textit{kitim} ‘arrive’, \textit{kitim-eː} ‘while arriving’. In the negative, the negative particle \textit{baː-} precedes the \isi{stem}, and V1s undergo ablaut in the \isi{stem} (CiːC and CaCiːC), and drop the suffix; it has a \isi{privative} meaning: \textit{baː-giːd} ‘without throwing’; \textit{baː-katiːm} ‘without arriving’. No similar patterns exist in \ili{Arabic} or other \ili{Cushitic} languages.


 \subsubsection{Verbal noun derivation}

In the verbal domain, \isi{non-concatenative} morphology concerns only V1s. With nouns, it applies to action nouns (\textit{maṣdars}) and agent nouns.

There are several \textit{maṣdar} patterns, with or without a prefix, with or without ablaut, depending mostly on the \isi{syllabic structure} of the verb. The most frequent ones with ablaut are presented below.

The pattern \textit{m(i(ː)/a)-}CV(ː)C applies to the majority of monosyllabic verbs. The \isi{stem} vowel varies and is not predictable: \textit{di} ‘say’, \textit{mi-jaːd} ‘saying’; \textit{dir} ‘kill’, \textit{ma-dar} ‘killing’; \textit{sʔa} ‘sit down’, \textit{ma\nobreakdash-sʔaː} ‘sitting’; \textit{ak} ‘become’, \textit{miː-kti} ‘becoming’; \textit{hiw} ‘give’, \textit{mi-jaw} ‘gift, act of giving’. A few disyllabic V1s comply to this pattern: \textit{rik\textsuperscript{w}}\textit{ij} ‘fear’, \textit{mi-rkʷa:j} ‘fearing’; \textit{jiwid} ‘curl’, \textit{miː-wad} ‘curling’. Some V1s of the CiC pattern have a CaːC pattern for \textit{maṣdars}, without a prefix: \textit{gid} ‘throw’, \textit{gaːd} ‘throwing’. In \ili{Classical} \ili{Arabic}, the marginal \textit{maṣdars} with a prefix concern trisyllabic verbs, none showing a \isi{long vowel} in the \isi{stem} or the prefix, nor a vowel \textit{i} in the prefix.

CiCiC and HaCiC\footnote{Where H stands for the laryngeals \textit{ʔ} and \textit{h}.} disyllabic verbs form their \textit{maṣdars} by vocalic ablaut to \textit{uː}: \textit{kitim} ‘arrive’, \textit{kituːm} ‘arriving’; \textit{ʔabik} ‘take’, \textit{ʔabuːk} ‘taking’; \textit{hamir} ‘be poor’, \textit{hamuːr} ‘being poor’. CiCaC V1s, and those ending in \textit{{}-j}, undergo vocalic ablaut to \textit{eː}: \textit{dig\textsuperscript{w}}\textit{ag\textsuperscript{w}} ‘catch up’, \textit{dig\textsuperscript{w}}\textit{eːg\textsuperscript{w}} ‘catching up’; \textit{biɖaːj} ‘yawn’, \textit{biɖeːj} ‘yawning’. In \ili{Classical} \ili{Arabic}, the \textit{maṣdar} pattern with \textit{uː} has a different vowel in the first syllable, \textit{a} (in Beja \textit{a} is conditioned by the initial laryngeal consonant), and it is limited almost exclusively to verbs expressing movements and body positions (\citealt[81]{BlachèreGaudefroy-Demombynes1975}). 

Bergman (\citeyear[35]{Bergman2002}) provides no information about verbal nouns of the \isi{base form} in \ili{Sudanese} \ili{Arabic} except that they are “not predictable”. 

As for agent nouns of V1s, they most often combine ablaut with the suffix \textit{\nobreakdash-aːna}, the same suffix as the one used to form agent nouns of V2 verbs, whose stems do not undergo ablaut. The ablaut pattern is the same as with the verbal \isi{intensive} \isi{derivation}: \textit{bir} ‘snatch’, \textit{boːr-aːna} ‘snatcher’; \textit{gid} ‘throw’,\textbf{ }\textit{geːd-aːna} ‘thrower, a good shot’; \textit{dibil} ‘pick up’, \textit{daːbl-ana} ‘one who picks up’. Some tri-consonantal stems have a suffix \textit{{}-i} instead of \textit{\nobreakdash-aːna}:  \textit{ʃibib} ‘look at’, \textit{ʃaːbb\nobreakdash-i} ‘guard, sentinel'. Some have both suffixes: \textit{kitim} ‘arrive’, \textit{kaːtm-aːna/kaːtim-i} ‘newcomer’.

These patterns are unknown in \ili{Arabic}.
 \subsubsection{Noun morphology}
 \subsubsubsection{General remarks}

The existence of verbal noun \isi{derivation} patterns and nominal plural patterns are well recognized in the literature about Beja morphology; for a recent overview, see \citet{Appleyard2007}. It is far from being the case for adjective and noun patterns. All noun and adjective patterns linked to V1s are listed below. \citet{Vanhove2012} provides an overview of these patterns which are summed up below.

 \subsubsubsection{Adjective patterns}
There are eight adjective patterns, two of which are shared with nouns. Most are derived from V1 verbs, but the reverse is also attested. A corresponding verb form is inexistent in a few cases. All patterns are based on ablaut, in two cases with an additional suffix \textit{\nobreakdash-a}, or gemination of the medial consonant. \ili{Arabic} has no dedicated adjective pattern (but the active \isi{participle} pattern of the verbal \isi{base form} CaːCiC may express properties). \tabref{tab:vanhove:5} provides the full list of patterns with examples. It is remarkable that none of them is similar to those of \ili{Classical} \ili{Arabic} or colloquial \ili{Sudanese} \ili{Arabic} \citep[17]{Bergman2002}.

\begin{table}
\begin{tabular}{lll}
\lsptoprule
Pattern & Adjective & Verb form\\\midrule
aCaːC & \textit{amaːg} ‘bad’ & \textit{mig} ‘do evil’\\
CaCCa & \textit{marɁa} ‘wide’ & \textit{mirɁ} ‘be wide’\\
CaːCi(C) & \textit{naːk\textsuperscript{w}}\textit{is} ‘short’ & \textit{nik\textsuperscript{w}}\textit{is} ‘be short’\\
& \textit{daːji} ‘good’ & Ø\\
CaCi(C) & \textit{dawil} ‘close’ & \textit{diwil} ‘be close’\\
CaCaːC & \textit{tak\textsuperscript{w}}\textit{aːk\textsuperscript{w}} ‘prepared’ & \textit{tik\textsuperscript{w}}\textit{ik\textsuperscript{w}} ‘prepare’\\
CaCaːC-a & \textit{ragaːg-a} ‘long’ & \textit{rigig} ‘stand up’\\
CiCaːC & \textit{ʃik\textsuperscript{w}}\textit{aːn} ‘aromatic’ & \textit{ʃik\textsuperscript{w}}\textit{an} ‘emit pleasant odour’\\
CaCCiC & \textit{ʃallik} ‘few’ & \textit{ʃilik} ‘be few’\\
\lspbottomrule
\end{tabular} 
\caption{Adjective patterns}
\label{tab:vanhove:5}
\end{table}

\subsubsubsection{Nouns}
There are eleven basic noun patterns related to V1 verbs. Most of the patterns for \isi{triconsonantal} \isi{roots} resemble those of \ili{Arabic} (but are not strictly identical), a coincidence which is not surprising since both languages have a limited number of vowels. \tabref{tab:vanhove:nounpat} provides the full list of these patterns. The CaCi pattern is shared with adjectives. The CiCi(C) pattern does not undergo ablaut.

\begin{table}
\begin{tabularx}{.95\textwidth}{llX}
\lsptoprule
Pattern & Noun & Verb form\\
\midrule
CaC & \textit{nak\textsuperscript{w}} ‘pregnancy’ & \textit{nik\textsuperscript{w}}\textit{i} ‘become pregnant’\\
CiCa & \textit{nisa} ‘advise’ & \textit{nisa} ‘advice’\\
CaCi & \textit{sari} ‘wakefulness’ & \textit{sir} ‘keep awake’\\
CaCa & \textit{nada} ‘dew’ & \textit{nidaj} ‘sweat, exude water’\\
CiCi(C) & \textit{mirɁi} ‘width’ & \textit{mirɁ} ‘be wide’\\
& \textit{riʃid} ‘wealth’ & \textit{riʃid} ‘raise, tend crops or cattle’\\
CaCiːC & \textit{ʃaɖiːɖ} ‘strip’ & \textit{ʃiɖiɖ} ‘strip off’\\
& \textit{ʃak\textsuperscript{w}}\textit{iːn} ‘fragrance’ & \textit{ʃik\textsuperscript{w}}\textit{an} ‘emit pleasant odour’\\
CaCiːC-a & \textit{raʃiːd-a} ‘cattle’ & \textit{riʃid} ‘raise, tend crops or cattle’\\
CiːCaːC & \textit{tiːlaːl} ‘stride’ & \textit{tilil} ‘stride far away from home’\\
CaCoːC & \textit{taboːk} ‘double handful’ & \textit{tiboːk} ‘fill scoop with cupped hands’\\
CiCuːC-a & \textit{tiluːl-a} ‘exile’ & \textit{tilil} ‘stride far away from home’\\
\lspbottomrule
\end{tabularx} 
\caption{Noun patterns}
\label{tab:vanhove:nounpat}
\end{table}


 
 
 
 \subsubsubsection{Nouns with prefix \textit{m(V)-}}




A few other semantic types of nouns, mostly instrument and place names, are formed through ablaut and a prefix \textit{m(V)-}, like in \ili{Arabic}. Contrary to \ili{Arabic} where these patterns are productive, they are frozen forms in Beja (some are not \isi{loanwords} from \ili{Arabic}, see the last three examples): \textit{Ɂafi} ‘prevent, secure’, \textit{m\nobreakdash-Ɂafaj} ‘nail, rivet, fastener’; \textit{himi} ‘cover’, \textit{m-himmeːj} ‘blanket’; \textit{ginif} ‘kneel’, \textit{mi-gnaf} ‘camp’; \textit{moːk} ‘take shelter’, \textit{ma\nobreakdash-k\textsuperscript{w}}\textit{a} ‘shelter’; \textit{rifif} ‘drag an object on the ground’, \textit{mi-rfaf} ‘reptile’.


 \subsubsection{Plural patterns}

The so-called “internal” plural patterns are common and frequent in \ili{Arabic} (and Ethio-\ili{Semitic}). Beja also has a limited set of \isi{internal plural} patterns, but it has developed its own system. \isi{Ablaut} patterns for plural \isi{formation} mainly concern non-derived nouns containing either a \isi{long vowel} or ending in a \isi{diphthong}. Both \textit{iː} and \textit{uː} turn to \textit{i} in the plural, and \textit{aː,} \textit{eː} and \textit{oː} turn to \textit{a}, sometimes with the addition of the plural suffix \textit{{}-}\textit{a}; nouns ending in \textit{{}-}\textit{aj} turn to a \isi{long vowel} \nobreakdash-\textit{eːj}:  \textit{ang\textsuperscript{w}}\textit{iːl}, pl. \textit{ang\textsuperscript{w}}\textit{il} ‘ear’; \textit{luːl}, pl. \textit{lil} ‘rope’; \textit{asuːl}, pl. \textit{asil} ‘blister’; \textit{hasaːl}, pl. \textit{hasal/hasal-a} ‘bridle’; \textit{meːk}, pl. \textit{mak} ‘donkey’; \textit{boːk}, pl. \textit{bak} ‘he-goat’; \textit{ganaj}, pl. \textit{ganeːj} ‘gazelle’ \citep{Vanhove2017}. 

Even though internal plurals can be considered as a genetic feature, the fact that they are very rare or absent in other \ili{Cushitic} languages \citep{Zaborski1986} speaks for a possible influence of \ili{Arabic} (in Sudan) upon Beja.


 
 \subsection{Syntax}
 \subsubsection{General remarks}

As far as we know, there are no syntactic calques from \ili{Arabic} in Beja. There are nevertheless a few borrowed lexical and grammatical items that gave rise to constructions concerning \isi{coordination} and subordination.


 \subsubsection{Coordination}

One of the three devices that mark \isi{coordination} is borrowed from \ili{Arabic} \textit{wa}. It is only used for noun phrases or nominalized clauses (deranked, temporal and \isi{relative} clauses), whereas the \ili{Arabic} source particle can be used with noun phrases and simple sentences. \textit{wa} is preposed to the coordinated element in \ili{Arabic}, but in Beja it is an enclitic particle \textit{=wa}, a position in line with the favoured SOV \isi{word order}. \textit{=wa} follows each of the coordinated elements. (\ref{sword}) illustrates the \isi{coordination} of two noun phrases.


\ea \label{sword}
{Beja (BEJ\_MV\_NARR\_01\_shelter\_057)\footnote{The sources of the examples are accessible online at \url{http://corporan.huma-num.fr/Archives/corpus.php}; the indications in parenthesis refer to the texts they are extracted from.}}\\
\gll bʔaɖaɖ=wa i=koːlej=wa sallam-ja=aj=heːb\\
     sword=\textsc{coord}~~~~ \textsc{def.m}=stick=\textsc{coord}~~~~ give-\textsc{pfv.3sg.m=csl=obj.1sg}\\
\glt `Since he had given me a sword and the stick…'
\z

Deranked clauses with non-finite verb forms, which partly have nominal properties (\citealt{Vanhove2016}), are also coordinated with \textit{=wa}. (\ref{plenty}) is an example with the manner \isi{converb}, and (\ref{happy}) with the simultaneity \isi{converb}. 

\ea\label{plenty}
{Beja (BEJ\_MV\_NARR\_14\_sijadok\_281-284)}\\
\gll winneːt~~~~ si-raːkʷ-oːm-a=b=wa gadab-aː=b=wa  ʔas-ti far-iːni\\
     plenty \textsc{caus}{}-\textsc{\textup{fear}}.\textsc{int-pass-cvb.mnr=indf.m.acc=coord} be\_sad-\textsc{cvb.mnr=indf.m.acc=coord} be\_up-\textsc{cvb.gnrl} jump-\textsc{ipfv.3sg.m}\\
\glt `Very frightened and sad, he jumps up.'
\z

\ea\label{happy}
{Beja (BEJ\_MV\_NARR\_13\_grave\_126-130)}\\
\gll afirh-a=b aka-jeː=wa i=dheːj=iːb hawaː-jeː=wa rh-ani\\
     be\_happy-\textsc{cvb.mnr=indf.m.acc}~~~~ become-\textsc{cvb.smlt=coord} \textsc{def.m}=people=\textsc{loc.sg}~~~~ play-\textsc{cvb.smlt=coord}~~~~ see-\textsc{pfv.1sg}\\
\glt `I saw him happy and playing among the people.'
\z

{R\isi{elative} and temporal subordinate clauses also have nominal properties: the \isi{relative} markers derive from the articles, and the temporal markers go back to nouns. (\ref{car}) illustrates the \isi{coordination} with a \isi{relative} clause which bears the} {\isi{coordination} marker, and (\ref{thing}) the \isi{coordination} of two temporal clauses.}

\ea\label{car}
{{Beja}{ (}{06\_foreigner\_22-24)}}\\
\gll uːn ani t=ʔarabijaːj=wa oː=maːl w=haːj jʔ-a=b a-kati=jeːb=wa kass=oː  a-niːw=hoːk\\
     \textsc{prox.sg.m.nom}~~	\textsc{1sg.nom}~~ \textsc{def.f=}car=\textsc{coord}~~ \textsc{def.sg.m.acc=}treasure \textsc{def.sg.m/rel=com}~~ come-\textsc{cvb.mnr=indf.m.acc} \textsc{1sg}-become\textsc{\textbackslash ipfv=rel.m=coord} all=\textsc{poss.3sg.acc} \textsc{1sg}-give.\textsc{ipfv=obj.2sg}\\
\glt `I’ll give you a car and all the fortune that I brought.'
\z

\ea\label{thing}
{{Beja (}BEJ\_MV\_CONV\_01\_rich\_SP2\_136-138)}\\
\gll naː=t bi=i-hiːw=oː=hoːb=wa i-niːw=oː=hoːb=wa\\
     thing=\textsc{indf.f}~~~~ \textsc{opt=3sg.m-}give\textsc{\textbackslash neg.opt=obj.1sg}=when=\textsc{coord} \textsc{3sg.m-}give.\textsc{ipfv=obj.1sg}=when=\textsc{coord}\\
\glt `Whether he gives it to me or not…' (lit. when he does not give me anything and when he gives me)
\z

Adversative \isi{coordination} between two simple clauses is also expressed with a borrowing from \ili{Arabic}: \textit{laːkin} ‘but’.


 \subsubsection{Subordination}

The \isi{reason conjunction} \textit{sabbiː} ‘because’ is a borrowing from the \ili{Arabic} noun \textit{sabab} ‘reason’. Like most balanced adverbial clauses, it is based on one of the \isi{relative} clause types, the one nominalized with the noun \textit{na} ‘thing’ in the genitive case. \textit{sabbiː} functions as the head of the \isi{relative} clause.

\ea
{Beja ({03\_camel\_192)}}\\
\gll ʔakir-a ɖab ɖaːb-iːn=eː=naː-ji sabbiː\\
     be\_strong-\textsc{cvb.mnr}~~ run.\textsc{ac}~~ run-\textsc{aor.3pl=rel}=thing-\textsc{gen}~~ because\\
\glt `Because it was running so fast…'
\z
\textit{sabbiː} can also be used after a noun or a pronoun in the genitive case: \textit{ombarijoːk} \textit{sabbiː} ‘because of you’.

Terminative adverbial clauses are expressed with a borrowing from \ili{Arabic}, \textit{hadiːd} ‘limit’. Again the borrowing is the head of the \isi{relative} clause.

\ea\label{ex:vanhove:}
{Beja (BEJ\_MV\_NARR\_51\_camel\_stallion\_026-030)}\\
\gll oːn i=kaːm=oːk heː=heːb i-ndi eːn baruːk oː=buːn gʷʔa-ti=eːb hadiːd\\
     \textsc{prox.sg.m.acc}~~ \textsc{def.m}=camel=\textsc{poss.2sg.acc}~~ give[.\textsc{imp.sg.m]=obj.1sg} \textsc{3sg.m}{}-say.\textsc{ipfv}~~ \textsc{dm} \textsc{2sg.m.nom}~~ \textsc{def.sg.m.acc}=coffee drink-\textsc{aor.2sg.m=rel.m}~~ until\\
\glt `Leave your camel with me, he says, until you have drunk your coffee!'
\z

\textit{hadiːd} can also be used as a postposition after a noun, in which case it can be abbreviated to \textit{had}: \textit{faʤil-had} ‘until morning’.


 
 \subsection{Lexicon}


The study of the Beja lexicon lacks research on the adaptation of \ili{Arabic} \isi{loanwords} and their chronological layers. There are no statistics on the proportions of lexical items borrowed from \ili{Arabic} or Ethio-\ili{Semitic} as compared to those inherited from \ili{Cushitic}, not to mention \ili{Afro-Asiatic} as a whole or borrowed from \ili{Nilo-Saharan}. Phonetic and morphological changes are bound to have blurred the etymological data, but what is certain is that massive lexical borrowings from \ili{Arabic} for all word categories took place at different periods of time, and that the process is still going on. Lexicostatistical studies (\citealt[267]{Cohen1988}; \citealt{Blažek1997}) have shown that Beja shares only 20\% of basic vocabulary with its closest relatives, \ili{Afar}, \ili{Saho} and \ili{Agaw}.

In this section I mainly concentrate on verbs, because they are often believed to be less  easily borrowed in language contact situations (see \citealt{Wohlgemuth2009} for an overview of the literature on this topic), which obviously is not the case for Beja.

\citet{Cohen1988} mentions that tri-consonantal V1s contain a majority of \ili{Semitic} borrowings. I conducted a search of Reinisch's (\citeyear{Reinisch1895}) dictionary, the only one to mention possible correspondences with \ili{Semitic} languages. It provided a total of 225 V1s, out of which only nine have no \ili{Semitic} cognates (four are cognates with \ili{Cushitic}, one is borrowed from \ili{Nubian}, and one \isi{cognate} with \ili{Egyptian}). Even if some of Reinisch’s comparisons are dubious, the overall picture is still in favour of massive borrowings from \ili{Semitic} (96\%). It is not easy to disentangle whether the source is an Ethio-\ili{Semitic} language or \ili{Arabic}, but until a more detailed study can be undertaken, the following can be said: 55 verbs (20\%) have cognates only in Ethio-\ili{Semitic} (\ili{Tigre}, \ili{Tigrinya}, \ili{Amharic}, and/or Ge’ez); out of the remaining 161 (72\%), 85 are attested only in \ili{Arabic}, 76 also in Ethio-\ili{Semitic}. Because of the long-standing contact with \ili{Arabic} for a large majority of Beja speakers in Sudan, and the marginality of contact with \ili{Tigre} limited to the south of the Beja domain, it is tempting to assume that almost 3/4 of the 76 verbs are of \ili{Arabic} origin. They may have been borrowed at an unknown time when the new suffix conjugation was still marginal. However, there are also tri-consonantal verbs (V2) which are conjugated with suffixes, albeit less numerous: 164. 141 have cognates with \ili{Semitic} languages (95 \ili{Arabic}, 31 Ethio-\ili{Semitic}, and 15 attested in both branches), six are pan-\ili{Cushitic}, one is pan-\ili{Afro-Asiatic}, one \ili{Nubian}, six are of dubious origin, and nine occur only in Beja. Does this mean that these borrowings occurred later than for V1s? In the current state of our knowledge of the historical development of Beja, it is not possible to answer this question.

On the other hand, Cohen (\citeyear[256]{Cohen1988}), in his count of consonants per \isi{stem} in eight \ili{Cushitic} and \ili{Omotic} languages, showed that biconsonantal stems are predominant in six of the languages. By contrast, they form 52.8\% of the 770 Beja stems in Roper's (\citeyear{Roper1928}) lexicon, and 42.7\% of the 611 \ili{Agaw} stems, almost on a par with bi-consonantal stems (42.2\%). What this shows is that massive borrowings from \ili{Arabic} (or from Ethio-\ili{Semitic} for \ili{Agaw}) helped to preserve tri-consonantal stems, which still form a majority of the stems in Beja, unlike in other \ili{Cushitic} languages.

\section{Conclusion}

This overview has shown that massive lexical borrowings from \ili{Arabic} in Beja have helped to significantly entrench \isi{non-concatenative} morphology in this language. Whether this is a preservation of an old \ili{Cushitic} system, or a more important development of this structure than in other \ili{Cushitic} languages under the influence of \ili{Arabic}, is open to debate, but what is certain is that it is not incidental that this system is so pervasive in Beja, the only \ili{Cushitic} language to have had a long history of intense language contact with \ili{Arabic}, the \ili{Semitic} language where \isi{non-concatenative} morphology is the most developed. What is important to recall is that Beja \isi{non-concatenative} morphology shows no borrowings of \ili{Arabic} patterns (unlike in Modern \ili{South Arabian} languages; see Bettega and Gasparini, this volume),\ia{Bettega, Simone@Bettega, Simone}\ia{Gasparini, Fabio@Gasparini, Fabio} leading to the conclusion that we are dealing with a \isi{convergence} phenomenon. Lexical borrowings and morphological \isi{convergence} are not paralleled in the phonological and syntactic domains where \ili{Arabic} influence seems marginal.

Much remains to be done concerning language contact between Beja and \ili{Arabic}, and we lack reliable sociolinguistic studies in this domain. We also lack a comprehensive historical investigation of the Beja lexicon, as well as a sufficiently elaborated theory of phonetic correspondences for \ili{Cushitic} (\citeyear[267]{Cohen1988}). Even though important progress has been made, in particular for Beja in the comparison of its consonant system with other \ili{Cushitic} languages and concerning the etymology of lexical items in some semantic fields, thanks to Blažek (\citeyear{Blažek2000,Blažek2003kinship,Blažek2003fauna,Blažek2006BB2,Blažek2006BB3}), the absence of a theory of lexical borrowings in Beja (and other \ili{Cushitic} languages) is still an impediment for a major breakthrough in the understanding of language contact between Beja and \ili{Arabic}.

\section*{Further reading}

Apart from \citet{Vanhove2012} on \isi{non-concatenative} morphology already summarized in §\ref{morphologyv}, we lack studies on contact-induced changes in Beja. \citet{Vanhove2003} is a brief \isi{article} on \isi{code-switching} in one tale and two jokes based on conversational analysis. \citet{Wedekind2012} is an appraisal of the changing sociolinguistic situation of Beja in Egypt, Sudan and Eritrea.

\section*{Acknowledgements}

My thanks are due to my Sudanese consultants and collaborators, in particular Ahmed Abdallah Mohamed-Tahir and his family in Sinkat, Mohamed-Tahir Hamid Ahmed in Khartoum, Yacine Ahmed Hamid and his family who hosted me in Khartoum. I am grateful to the two editors of this volume, and wish to acknowledge the financial support of LLACAN, the ANR projects CorpAfroAs and CorTypo (Principal Investigator Amina Mettouchi). This work was also partially supported by a public grant overseen by the French National Research Agency (ANR) as part of the program “Investissements d'Avenir” (reference: ANR-10-LABX-0083). It contributes to the IdEx Université de Paris - ANR-18-IDEX-0001.


\section*{Abbreviations}
\begin{multicols}{2}
\begin{tabbing}
\textsc{ipfv} \hspace{1em} \= before common era\kill
\textsc{ac} \> \isi{action noun}, \textit{maṣdar} \\
\textsc{acc} \>{{accusative} }\\
\textsc{aor} \> aorist \\
BCE \> before Common Era \\
\textsc{caus} \> {{causative} } \\
CE \> Common Era \\
\textsc{com} \> {{comitative} }\\
\textsc{coord} \> {{coordination} }\\
\textsc{csl} \> {{causal} }\\
\textsc{cvb} \> {{converb} }\\
\textsc{def} \> {{definite} }\\
\textsc{dm} \> {{discourse marker}} \\
\textsc{f} \> {{feminine} }\\
\textsc{gen} \> genitive \\
\textsc{gnrl} \> {{general} }\\
\textsc{imp} \> {{imperative} }\\
\textsc{indf} \> {{indefinite} }\\
\textsc{int} \> {{intensive} }\\
\textsc{ipfv} \> {{imperfective} }\\
\textsc{loc} \> {{locative} }\\
\textsc{m} \> {{masculine} }\\
\textsc{mid} \> {{middle}}\\
\textsc{mnr} \> {{manner} }\\
\textsc{neg} \> {{negation} }\\
\textsc{nom} \> {{nominative} }\\
\textsc{obj} \> {{object} }\\
\textsc{opt} \> {{optative} }\\
\textsc{pass} \> {{passive} }\\
\textsc{pfv} \> {{perfective} }\\
\textsc{pl} \> plural \\
\textsc{poss} \> {{possessive} }\\
\textsc{prox} \> {{proximal} }\\
\textsc{recp} \> {{reciprocal}}\\
\textsc{rel} \> {{relator} }\\
\textsc{sg} \> {{singular} } \\
\textsc{smlt} \> {{simultaneity}} \\
TAM \> {{tense--aspect--mood}}
\end{tabbing}
\end{multicols}




\sloppy
\printbibliography[heading=subbibliography,notkeyword=this]
\end{document}
